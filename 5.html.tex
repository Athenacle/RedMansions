\chap{五}{开生面梦演红楼梦 立新场情传幻境情}



\begin{parag}
    \begin{note}蒙:万种豪华原是幻,何尝造孽,何是风流?曲终人散有谁留,为甚营求?只爱蝇头!一番遭遇几多愁,点水根由,泉涌难酬!
    \end{note}
\end{parag}


\begin{parag}
    题曰:
\end{parag}


\begin{poem}
    \begin{pl}春困葳蕤拥绣衾,恍随仙子别红尘。\end{pl}

    \begin{pl}问谁幻入华胥境,千古风流造孽人。\end{pl}
\end{poem}


\begin{parag}
    却说薛家母子在荣府中寄居等事略已表明,此回则暂不能写矣。\begin{note}甲戌侧:此等处实又非别部小说之熟套起法。\end{note}
\end{parag}


\begin{parag}
    如今且说林黛玉\begin{note}甲戌眉:不叙宝钗,反仍叙黛玉。盖前回只不过欲出宝钗,非实写之文耳,此回若仍续写,则将二玉高搁矣,故急转笔仍旧至黛玉,使荣府正文方不至于冷落也。今写黛玉神妙之至,何也?因写黛玉实是写宝钗,非真有意去写黛玉,几乎又被作者瞒过。\end{note}自在荣府以来,贾母万般怜爱,寝食起居,一如宝玉,\begin{note}甲戌侧:妙极!所谓一击两鸣法,宝玉身分可知。\end{note}迎春、探春、惜春三个亲孙女倒且靠后。\begin{note}甲戌侧:此句写贾母。\end{note}便是宝玉和黛玉二人之亲密友爱处,亦自较别个不同,\begin{note}甲戌侧:此句妙,细思有多少文章。\end{note}日则同行同坐,夜则同息同止,真是言和意顺,略无参商。不想如今忽然来了一个薛宝钗,\begin{note}甲戌侧:总是奇峻之笔,写来健拔,似新出一人耳。甲戌眉:此处如此写宝钗,前回中略不一写,可知前回迥非十二钗之正文也。欲出宝钗便不肯从宝钗身上写来,却先款款叙出二玉,陡然转出宝钗,三人方可鼎立。行文之法又一变体。\end{note}年岁虽大不多,然品格端方,容貌丰美,人多谓黛玉所不及。\begin{note}甲戌侧:此句定评,想世人目中各有所取也。按黛玉宝钗二人,一如姣花,一如纤柳,各极其妙者,然世人性分甘苦不同之故耳。\end{note}而且宝钗行为豁达,随分从时,不比黛玉孤高自许,目无下尘,故比黛玉大得下人之心。\begin{note}甲戌侧:将两个行止摄总一写,实是难写,亦实系千部小说中未敢说写者。\end{note}便是那些小丫头子们,亦多喜与宝钗去顽。因此黛玉心中便有些悒郁不忿之意,\begin{note}甲戌侧:此一句是今古才人通病,如人人皆如我黛玉之为人,方许他妒。此是黛玉缺处。\end{note}宝钗却浑然不觉。\begin{note}甲戌侧:这还是天性,后文中则是又加学力了。\end{note}那宝玉亦在孩提之间,况自天性所禀来的一片愚拙偏僻,\begin{note}甲戌侧:四字是极不好,却是极妙。只不要被作者瞒过。\end{note}视姊妹弟兄皆出一意,并无亲疏远近之别。\begin{note}甲戌侧:如此反谓“愚痴”,正从世人意中写也。\end{note}其中因与黛玉同随贾母一处坐卧,故略比别个姊妹熟惯些。既熟惯,则更觉亲密,既亲密,则不免一时有求全之毁,不虞之隙。\begin{note}甲戌侧:八字定评,有趣。不独黛玉、宝玉二人,亦可为古今天下亲密人当头一喝。甲戌眉:八字为二玉一生文字之纲。\end{note}这日不知为何,他二人言语有些不合起来,黛玉又\begin{note}甲戌侧:“又”字妙极!补出近日无限垂泪之事矣,此仍淡淡写来,使后文来得不突然。\end{note}气的独在房中垂泪,宝玉又\begin{note}甲戌侧:“又”字妙极!凡用二“又”字,如双峰对峙,总补二玉正文。\end{note}自悔言语冒撞,前去俯就,那黛玉方渐渐的回转来。
\end{parag}


\begin{parag}
    因东边宁府中花园内梅花盛开,\begin{note}甲戌侧:元春消息动矣。\end{note}贾珍之妻尤氏乃治酒,请贾母、邢夫人、王夫人等赏花。是日先携了贾蓉之妻,二人来面请。贾母等于早饭后过来,就在会芳园\begin{note}甲戌侧:随笔带出,妙!字意可思。\end{note}游顽,先茶后酒,不过皆是宁荣二府女眷家宴小集,并无别样新文趣事可记。\begin{note}甲戌侧:这是第一家宴,偏如此草草写。此如晋人倒食甘蔗,渐入佳境一样。\end{note}
\end{parag}


\begin{parag}
    一时宝玉倦怠,欲睡中觉,贾母命人好生哄著,歇一回再来。贾蓉之妻秦氏便忙笑回道:“我们这里有给宝叔收拾下的屋子,老祖宗放心,只管交与我就是了。”又向宝玉的奶娘丫鬟等道:“嬷嬷姐姐们,请宝叔随我这里来。”贾母素知秦氏是个极妥当的人,\begin{note}甲戌侧:借贾母心中定评。\end{note}生的袅娜纤巧,行事又温柔和平,乃重孙媳中第一个得意之人,\begin{note}甲戌侧:又夹写出秦氏来。\end{note}见他去安置宝玉,自是安稳的。
\end{parag}


\begin{parag}
    当下秦氏引了一簇人来至上房内间。宝玉抬头看见一幅画贴在上面,画的人物固好,其故事乃是“燃藜图”,也不看系何人所画,心中便有些不快。\begin{note}甲戌眉:如此画联,焉能入梦?\end{note}又有一幅对联,写的是:
\end{parag}


\begin{poem}
    \begin{pl}世事洞明皆学问,人情练达即文章。\end{pl}
    \begin{note}甲戌双夹:看此联极俗,用于此则极妙。盖作者正因古今王孙公子,劈头先下金针。\end{note}
\end{poem}


\begin{parag}
    及看了这两句,纵然室宇精美,铺陈华丽,亦断断不肯在这里了,忙说:“出去,出去!”秦氏听了笑道:“这里还不好,可往那里去呢?不然往我屋里去吧。”宝玉点头微笑。有一个嬷嬷说道:“那里有个叔叔往侄儿房里睡觉的理?”秦氏笑道:“嗳哟哟!不怕他恼。他能多大呢,就忌讳这些个!上月你没看见我那个兄弟来了,\begin{note}甲戌眉:伏下秦钟,妙!\end{note}虽然与宝叔同年,两个人若站在一处,只怕那个还高些呢。”\begin{note}甲戌侧:又伏下一人,随笔便出,得隙便入,精细之极。\end{note}宝玉道:“我怎么没见过?你带他来我瞧瞧。”\begin{note}甲戌侧:侯门少年纨绔活跳下来。\end{note}众人笑道:“隔著二三十里,往那里带去,见的日子有呢。”说著大家来至秦氏房中。刚至房门,便有一股细细的甜香\begin{note}甲戌侧:此香名“引梦香”。\end{note}袭人而来。宝玉觉得眼饧骨软,连说: “好香!”\begin{note}甲戌侧:刻骨吸髓之情景,如何想得来,又如何写得来?[进房如梦境。]\end{note}入房向壁上看时,有唐伯虎画的《海棠春睡图》,\begin{note}甲戌侧:妙图。\end{note}两边有宋学士秦太虚写的一副对联,其联云:
\end{parag}


\begin{poem}
    \begin{pl}嫩寒锁梦因春冷,\end{pl}
    \begin{note}甲夹:艳极,淫极!\end{note}

    \begin{pl}芳气笼人是酒香。\end{pl}
    \begin{note}甲夹:已入梦境矣。\end{note}

\end{poem}


\begin{parag}
    案上设著武则天当日镜室中设的宝镜,\begin{note}甲戌侧:设譬调侃耳,若真以为然,则又被作者瞒过。\end{note}一边摆著飞燕立著舞过的金盘,盘内盛著安禄山掷过伤了太真乳的木瓜。上面设著寿昌公主于含章殿下卧的榻,悬的是同昌公主制的联珠帐。宝玉含笑连说:“这里好!”\begin{note}摆设就合著他的意。\end{note}秦氏笑道:“我这屋子大约神仙也可以住得了。”说著亲自展开了西子浣过的纱衾,移了红娘抱过的鸳枕,\begin{note}甲戌侧:一路设譬之文,迥非《石头记》大笔所屑,别有他属,余所不知。\end{note}于是众奶母伏侍宝玉卧好,款款散了,只留袭人、\begin{note}甲戌侧:一个再见。\end{note}媚人、\begin{note}甲戌侧:二新出。\end{note}晴雯、\begin{note}甲戌侧:三新出,名妙而文。\end{note}麝月\begin{note}甲戌侧:四新出,尤妙。看此四婢之名,则知历来小说难与并肩。\end{note}四个丫鬟为伴。\begin{note}甲戌眉:文至此不知从何处想来。\end{note}秦氏便分咐小丫鬟们,好生在廊檐下看著猫儿狗儿打架。\begin{note}甲戌侧:细极。\end{note}
\end{parag}


\begin{parag}
    那宝玉刚合上眼,便惚惚的睡去,犹似秦氏在前,遂悠悠荡荡,随了秦氏,至一所在。\begin{note}甲戌侧:此梦文情固佳,然必用秦氏引梦,又用秦氏出梦,竟不知立意何属?惟批书人知之。\end{note}但见朱栏白石,绿树清溪,真是人迹希逢,飞尘不到。\begin{note}甲戌侧:一篇《蓬莱赋》。\end{note}宝玉在梦中欢喜,想道:“这个去处有趣,我就在这里过一生,纵然失了家也愿意,强如天天被父母师傅打呢。”\begin{note}甲戌侧:一句忙里点出小儿心性。\end{note}正胡思之间,忽听山后有人作歌曰:
\end{parag}


\begin{poem}
    \begin{pl}春梦随云散,\end{pl}
    \begin{note}甲戌双夹:开口拿“春”字,最紧要!\end{note}

    \begin{pl}飞花逐水流。\end{pl}\begin{note}甲夹:二句比也。\end{note}

    \begin{pl}寄言众儿女,何必觅闲愁。\end{pl}
    \begin{note}甲夹:将通部人一喝。\end{note}
\end{poem}


\begin{parag}
    宝玉听了是女子的声音。\begin{note}甲戌侧:写出终日与女儿厮混最熟。\end{note}歌声未息,早见那边走出一个人来,蹁跹袅娜,端的与人不同。有赋为证:
\end{parag}
\begin{qute2sp}
    \textbf{
        方离柳坞,乍出花房。但行处,鸟惊庭树;将到时,影度回廊。仙袂乍飘兮,闻麝兰之馥郁;荷衣欲动兮,听环佩之铿锵。靥笑春桃兮,云堆翠髻;唇绽樱颗兮,榴齿含香。纤腰之楚楚兮,回风舞雪;珠翠之辉辉兮,满额鹅黄。出没花间兮,宜嗔宜喜;徘徊池上兮,若飞若扬。蛾眉颦笑兮,将言而未语;莲步乍移兮,待止而欲行。羡彼之良质兮,冰清玉润;羡彼之华服兮,闪灼文章;爱彼之貌容兮,香培玉琢;美彼之态度兮,凤翥龙翔。其素若何?春梅绽雪。其洁若何?秋菊被霜。其静若何?松生空谷。其艳若何?霞映澄塘。其文若何?龙游曲沼。其神若何?月射寒江。应惭西子,实愧王嫱。奇矣哉,生于孰地,来自何方?信矣乎,瑶池不二,紫府无双。果何人哉?如斯之美也!}
    \begin{note}甲戌眉:按此书凡例本无赞赋闲文,前有宝玉二词,今复见此一赋,何也?盖此二人乃通部大纲,不得不用此套。前词却是作者别有深意,故见其妙。此赋则不见长,然亦不可无者也。\end{note}

\end{qute2sp}


\begin{parag}
    宝玉见是一个仙姑,喜的忙来作揖问道:“神仙姐姐,\begin{note}甲戌侧:千古未闻之奇称,写来竟成千古未闻之奇语。故是千古未有之奇文。\end{note}不知从那里来,如今要往那里去?也不知这是何处,望乞携带携带。”那仙姑笑道:“吾居离恨天之上,灌愁海之中,乃放春山遣香洞太虚幻境警幻仙姑是也。\begin{note}甲戌侧:与首回中甄士隐梦景一照。\end{note}司人间之风情月债,掌尘世之女怨男痴。因近来风流冤孽,\begin{note}甲戌侧:四字可畏。\end{note}缠绵于此处,是以前来访察机会,布散相思。今忽与尔相逢,亦非偶然。此离吾境不远,别无他物,仅有自采仙茗一盏,亲酿美酒一瓮,素练魔舞歌姬数人,新填《红楼梦》\begin{note}甲戌侧:点题。盖作者自云所历不过红楼一梦耳。\end{note}仙曲十二支,试随吾一游否?”宝玉听说,便忘了秦氏在何处,\begin{note}甲戌侧:细极。\end{note}竟随了仙姑,至一所在,有石牌横建,上书“太虚幻境”四个大字,两边一副对联,\begin{note}甲戌侧:士隐曾见此匾对,而僧道不能领入,留此回警幻邀宝玉后文。\end{note}乃是:
\end{parag}


\begin{poem}
    \begin{pl}假作真时真亦假,无为有处有还无。\end{pl}\begin{note}甲双:正恐观者忘却首回,故特将甄士隐梦景重一滃染。\end{note}
\end{poem}


\begin{parag}
    转过牌坊,便是一座宫门,上面横书四个大字,道是“孽海情天”。又有一副对联,大书云:
\end{parag}


\begin{poem}
    \begin{pl}厚地高天,堪叹古今情不尽;痴男怨女,可怜风月债难偿\end{pl}
\end{poem}


\begin{parag}
    宝玉看了,\begin{note}甲戌眉:菩萨天尊皆因僧道而有,以点俗人,独不许幻造太虚幻境以警情者乎?观者恶其荒唐,余则喜其新鲜。有修庙造塔祈福者,余今意欲起太虚幻境以较修七十二司更有功德。\end{note}心下自思道:“原来如此。但不知何为古今之情,何为风月之债?从今倒要领略领略。”宝玉只顾如此一想,不料早把些邪魔招入膏肓了。\begin{note}甲戌侧:奇极妙文。\end{note}当下随了仙姑进入二层门内,至两边配殿,皆有匾额对联,一时看不尽许多,惟见有几处写的是:“痴情司”、“结怨司”、“朝啼司”、“夜怨司”、“春感司”、“秋悲司”。\begin{note}甲戌侧:虚陪六个。\end{note}看了,因向仙姑道:“敢烦仙姑引我到那各司中游玩游玩,不知可使得?”仙姑道:“此各司中皆贮的是普天之下所有的女子过去未来的簿册。尔凡眼尘躯,未便先知的。”宝玉听了,那里肯依,复央之再四。仙姑无奈,说: “也罢,就在此司内略随喜随喜罢了。”宝玉喜不自胜,抬头看这司的匾上,乃是“薄命司”\begin{note}甲戌侧:正文。\end{note}三字,两边对联写的是:
\end{parag}


\begin{poem}\begin{pl}春恨秋悲皆自惹,花容月貌为谁妍。\end{pl}\end{poem}


\begin{parag}
    宝玉看了,便知\begin{note}甲侧:便知二字是字法,最为紧要之至。\end{note}感叹。进入门来,只见有数十个大厨,皆用封条封着。看那封条上,皆是各省地名。宝玉一心只拣自己的家乡的封条看,遂无心看别省的了。只见那边厨上封条上大书七字云:金陵十二钗正册。\begin{note}甲侧:正文题。\end{note}宝玉因问:“何为金陵十二钗正册?”警幻道:“即贵省中十二冠首女子之册,故为正册。”宝玉道:“常听\begin{note}甲侧:常听二字,神理极妙。\end{note}人说,金陵极大,怎么只十二个女子?如今单我们家里,上上下下就有几百女孩子呢。”\begin{note}甲侧:贵公子口声。\end{note}警幻冷笑道:“贵省女子固多,不过择其紧要者录之。下边二厨则又次之。余者庸愚之辈,则无册可录矣。”宝玉听说,再看下首二厨上,果然一个写着金陵十二钗副册,又一个写着金陵十二钗又副册。宝玉便伸手先将又副册厨开了,拿出一本册来,揭开一看,只见这首页上画着一副画,又非人物,亦无山水,不过水墨滃染的满纸乌云浊雾而已。后有几行字,写的是:
\end{parag}


\begin{poem}
    \begin{pl}霁月难逢,彩云易散。\end{pl}

    \begin{pl}心比天高,身为下贱。\end{pl}

    \begin{pl}风流灵巧招人怨。\end{pl}

    \begin{pl}寿夭多因诽谤生,多情公子空牵念。\end{pl}
    \begin{note}甲双:恰极之至!「病补雀金裘」回中与此合看。\end{note}

\end{poem}


\begin{parag}
    宝玉看了,又见后面画着一簇鲜花,一床破席。也有几句言词,写道是:
\end{parag}


\begin{poem}
    \begin{pl}枉自温柔和顺,空云似桂如兰。\end{pl}

    \begin{pl}堪羡优伶有福,谁知公子无缘。\end{pl}
    \begin{note}甲双:骂死宝玉,却是自悔。\end{note}

\end{poem}


\begin{parag}
    宝玉看了不解。遂掷下这个,又去开了副册厨门,拿起一本册来,揭开看时,只见画着一株桂花,下面有一池沼,其中水涸泥干,莲枯藕败。画后书云:
\end{parag}


\begin{poem}
    \begin{pl}根并荷花一茎香,\end{pl}
    \begin{note}甲双:却是咏菱妙句。\end{note}

    \begin{pl}平生遭际实堪伤。\end{pl}

    \begin{pl}自从两地生孤木,\end{pl}
    \begin{note}甲夹:折(拆)字法。\end{note}

    \begin{pl}致使香魂返故乡。\end{pl}
\end{poem}


\begin{parag}
    宝玉看了仍不解他。又掷下,再去取正册看。只见头一页上便画着两株枯木,木上悬着一围玉带,又有一堆雪,雪下一股金簪。也有四句言词道:
\end{parag}


\begin{poem}
    \begin{pl}可叹停机德,\end{pl}\begin{note}甲夹:此句薛。\end{note}

    \begin{pl}堪怜咏絮才。\end{pl}\begin{note}甲夹:此句林。\end{note}

    \begin{pl}玉带林中挂,\end{pl}

    \begin{pl}金簪雪里埋。\end{pl}\begin{note}甲双:寓意深远,皆非生其地之意。\end{note}

\end{poem}


\begin{parag}
    宝玉看了仍不解。待要问时,情知他必不肯泄漏;待要丢下,又不舍。遂又往后看时,只见画著一张弓,弓上挂一香橼。也有一首歌词云:\begin{note}甲眉:世之好事者争传《推背图》之说,想前人断不肯煽惑愚迷,即有此说,亦非常人供谈之物。此回悉借其法,为众女子数运之机。无可以供茶酒之物,亦无干涉政事,真奇想奇笔。\end{note}
\end{parag}


\begin{poem}
    \begin{pl}二十年来辨是谁,\end{pl}

    \begin{pl}榴花开处照宫闱;\end{pl}

    \begin{pl}三春争及初春景,\end{pl}\begin{note}甲夹:显极。\end{note}

    \begin{pl}虎兕相逢大梦归。\end{pl}
\end{poem}


\begin{parag}
    后面又画著两人放风筝,一片大海,一只大船,船中有一女子掩面泣涕之状。也有四句写云:
\end{parag}


\begin{poem}
    \begin{pl}才自精明志自高,生于末世运偏消。\end{pl}
    \begin{note}甲双:感叹句,自寓。\end{note}

    \begin{pl}清明涕送江边舰,千里东风一望遥。\end{pl}
    \begin{note}甲夹:好句!\end{note}
\end{poem}


\begin{parag}
    后面又画几缕飞云,一湾逝水。其词曰:
\end{parag}


\begin{poem}
    \begin{pl}富贵又何为?襁褓之间父母违;\end{pl}

    \begin{pl}展眼吊斜晖,湘江水逝楚云飞。\end{pl}

\end{poem}


\begin{parag}
    后面又画著一块美玉,落在泥垢之中。其断语云:
\end{parag}


\begin{poem}
    \begin{pl}欲洁何曾洁,云空未必空!\end{pl}

    \begin{pl}可怜金玉质,落陷污泥中。\end{pl}
\end{poem}


\begin{parag}
    后面忽见画著个恶狼,追扑一美女,欲啖之意。其书云:
\end{parag}


\begin{poem}
    \begin{pl}子系中山狼,\end{pl}

    \begin{pl}得志便猖狂。\end{pl}\begin{note}甲夹:好句。\end{note}

    \begin{pl}金闺花柳质,\end{pl}

    \begin{pl}一载赴黄梁。\end{pl}

\end{poem}


\begin{parag}
    后面便是一所古庙,里面有一美人在内看经独坐。其判云:
\end{parag}


\begin{poem}
    \begin{pl}勘破三春景不长,缁衣顿改昔年妆。\end{pl}

    \begin{pl}可怜绣户侯门女,独卧青灯古佛傍。\end{pl}
    \begin{note}甲夹:好句。\end{note}
\end{poem}


\begin{parag}
    后面便是一片冰山,上面有一只雌凤。其判曰:
\end{parag}


\begin{poem}
    \begin{pl}凡鸟偏从末世来,\end{pl}

    \begin{pl}都知爱慕此身才。\end{pl}

    \begin{pl}一从二令三人木,\end{pl}\begin{note}甲夹:拆字法。\end{note}

    \begin{pl}哭向金陵事更哀。\end{pl}
\end{poem}


\begin{parag}
    后面又是一座荒村野店,有一美人在那里纺绩。其判云:
\end{parag}


\begin{poem}
    \begin{pl}事败休云贵,\end{pl}

    \begin{pl}家亡莫论亲。\end{pl}\begin{note}甲双:非经历过者,此二句则云纸上谈兵。过来人那得不哭!\end{note}

    \begin{pl}偶因济刘氏,\end{pl}

    \begin{pl}巧得遇恩人。\end{pl}

\end{poem}


\begin{parag}
    后面又画著一盆茂兰,旁有一位凤冠霞帔的美人。也有判云:
\end{parag}


\begin{poem}
    \begin{pl}桃李春风结子完,到头谁似一盆兰?\end{pl}

    \begin{pl}为冰为水空相妒,枉与他人作话谈。\end{pl}
    \begin{note}甲双:真心实语。\end{note}

\end{poem}


\begin{parag}
    后面又画著高楼大厦,有一美人悬梁自缢。其判云:
\end{parag}


\begin{poem}
    \begin{pl}情天情海幻情身,情既相逢必主淫。\end{pl}

    \begin{pl}谩言不肖皆荣出,造衅开端实在宁。\end{pl}
\end{poem}


\begin{parag}
    宝玉还欲看时,那仙姑知他天分高明,性情颖慧,\begin{note}甲戌眉:通部中笔笔贬宝玉,人人嘲宝玉,语语谤宝玉,今却于警幻意中忽写出此八字来,真是意外之意。此法亦别书中所无。\end{note}恐把仙机泄漏,遂掩了卷册,笑向宝玉道:“且随我去游玩奇景,\begin{note}甲戌侧:是哄小儿语,细甚。\end{note}何必在此打这闷葫芦!”\begin{note}甲戌侧:为前文“葫芦庙”一点。\end{note}
\end{parag}


\begin{parag}
    宝玉恍恍惚惚,不觉弃了卷册,\begin{note}甲戌侧:是梦中景况,细极。\end{note}又随了警幻来至后面。但见珠帘绣幕,画栋雕檐,说不尽那光摇朱户金铺地,雪照琼窗玉作宫。更见仙花馥郁,异草芬芳,真好个所在。\begin{note}甲戌侧:已为省亲别墅画下图式矣。\end{note}又听警幻笑道:“你们快出来迎接贵客!”一语未了,只见房中又走出几个仙子来,皆是荷袂蹁跹,羽衣飘舞,姣若春花,媚如秋月。一见了宝玉,都怨谤警幻道:“我们不知系何‘贵客’,忙的接了出来!姐姐曾说今日今时必有绛珠妹子\begin{note}甲戌侧:绛珠为谁氏?请观者细思首回。\end{note}的生魂前来游玩,故我等久待。何故反引这浊物来污染这清净女儿之境?”\begin{note}甲戌眉:奇笔摅奇文。作书者视女儿珍贵之至,不知今时女儿可知?余为作者痴心一哭,又为近之自弃自败之女儿一恨。\end{note}宝玉听如此说,便吓得欲退不能退,\begin{note}甲戌侧:贵公子不怒而反退,却是宝玉天分中一段情痴。\end{note}果觉自形污秽不堪。警幻忙携住宝玉的手,\begin{note}甲戌侧:妙!警幻自是个多情种子。\end{note}向众姊妹道:“你等不知原委:今日原欲往荣府去接绛珠,适从宁府所过,偶遇宁荣二公之灵,嘱吾云:‘吾家自国朝定鼎以来,功名奕世,富贵传流,虽历百年,奈运终数尽,不可挽回者。故遗之子孙虽多,竟无可以继业。\begin{note}甲戌侧:这是作者真正一把眼泪。\end{note}其中惟嫡孙宝玉一人,禀性乖张,生性怪谲,虽聪明灵慧,略可望成,无奈吾家运数合终,恐无人规引入正。幸仙姑偶来,万望先以情欲声色等事警其痴顽,\begin{note}甲戌侧:二公真无可奈何,开一觉世觉人之路也。\end{note}或能使彼跳出迷人圈子,然后入于正路,亦吾兄弟之幸矣。’如此嘱吾,故发慈心,引彼至此。先以彼家上中下三等女子之终身册籍,令彼熟玩,尚未觉悟。故引彼再至此处,令其再历饮馔声色之幻,或冀将来一悟,亦未可知也。”\begin{note}甲戌侧:一段叙出宁、荣二公,足见作者深意。\end{note}
\end{parag}


\begin{parag}
    说毕,携了宝玉入室。但闻一缕幽香,竟不知其所焚何物。宝玉遂不禁相问,警幻冷笑道:“此香尘世中既无,尔何能知!此香乃系诸名山胜境内初生异卉之精,合各种宝林珠树之油所制,名‘群芳髓’。”\begin{note}甲戌侧:好香!\end{note}宝玉听了,自是羡慕而已。大家入座,小丫鬟捧上茶来。宝玉自觉清香异味,纯美非常,因又问何名。警幻道:“此茶出在放春山遣香洞,又以仙花灵叶上所带之宿露而烹。此茶名曰‘千红一窟’。”\begin{note}甲戌侧:隐“哭”字。\end{note}宝玉听了,点头称赏。因看房内,瑶琴、宝鼎、古画、新诗,无所不有,更喜窗下亦有唾绒,奁间时渍粉污。壁上也见悬著一副对联,书云:
\end{parag}


\begin{poem}
    \begin{pl}幽微灵秀地,\end{pl}\begin{note}甲双:女儿之心,女儿之境。\end{note}

    \begin{pl}无可奈何天。\end{pl}\begin{note}甲双:两句尽矣。撰通部大书不难,最难是此等处,可知皆从无可奈何而有。\end{note}
\end{poem}


\begin{parag}
    宝玉看毕,无不羡慕。因又请问众仙姑姓名:一名痴梦仙姑,一名钟情大士,一名引愁金女,一名度恨菩提,各各道号不一。少刻,有小丫鬟来调桌安椅,设摆酒馔。真是:
\end{parag}


\begin{poem}
    \begin{pl} 琼浆满泛玻璃盏,玉液浓斟琥珀杯。\end{pl}
\end{poem}


\begin{parag}
    更不用再说那肴馔之盛。宝玉因闻得此酒清香甘冽,异乎寻常,又不禁相问。警幻道:“此酒乃以百花之蕊,万木之汁,加以麟髓之醅,凤乳之曲酿成,因名为‘万艳同杯’。”\begin{note}甲戌侧:与千红一窟一对,隐悲字。\end{note}宝玉称赏不迭。
\end{parag}


\begin{parag}
    饮酒间,又有十二个舞女上来,请问演何词曲。警幻道:“就将新制《红楼梦》十二支演上来。”舞女们答应了,便轻敲檀板,款按银筝。听他歌道是:
\end{parag}


\begin{qute2sp}

    开辟鸿蒙……\begin{note}甲夹:故作顿挫摇摆。\end{note}
\end{qute2sp}


\begin{parag}
    方歌了一句,警幻便说道:“此曲不比尘世中所填传奇之曲,必有生旦净末之则,又有南北九宫之限。此或咏叹一人,或感怀一事,偶成一曲,即可谱入管弦。若非个中人,\begin{note}甲戌侧:三字要紧。不知谁是个中人。宝玉即个中人乎?然则石头亦个中人乎?作者亦系个中人乎?观者亦个中人乎?\end{note}不知其中之妙。料尔亦未必深明此调,若不先阅其稿,后听其歌,翻成嚼蜡矣。”\begin{note}甲戌眉:警幻是个极会看戏人。近之大老观戏,必先翻阅角本。目睹其词,耳听彼歌,却从警幻处学来。\end{note}说毕,回头命小丫鬟取了《红楼梦》原稿来,递与宝玉。宝玉接来,一面目视其文,一面耳聆其歌曰:\begin{note}甲戌眉:作者能处,惯于自站地步,又惯于陡起波澜,又惯于故为曲折,最是行文秘诀。\end{note}
\end{parag}


\begin{qute2sp}
    \song{红楼梦 引子}
    开辟鸿蒙,谁为情种?\begin{note}甲侧:非作者为谁。余又曰:“亦非作者,乃石头耳。”\end{note}都只为风月情浓。趁着这奈何天、伤怀日、寂寞时,试遣愚衷\begin{note}甲侧:愚字自谦得妙。\end{note}。因此上,演出这怀金悼玉的《红楼梦》。\begin{note}甲双:读此几句,翻厌近之传奇中必用开场副末等套,累赘太甚。甲眉:怀金悼玉,大有深意。\end{note}
\end{qute2sp}


\begin{qute2sp}
    \song{终身误}
    都道是金玉良姻,俺只念木石前盟。空对著,山中高士晶莹雪;终不忘,世外仙姝寂寞林。叹人间,美中不足今方信。纵然是齐眉举案,到底意难平。\begin{note}甲眉:语句泼撒,不负自创北曲。\end{note}
\end{qute2sp}


\begin{qute2sp}
    \song{枉凝眉}
    一个是阆苑仙葩,一个是美玉无瑕。若说没奇缘,今生偏又遇着他;若说有奇缘,如何心事终须化!一个枉自嗟呀,一个空劳牵挂。一个是水中月,一个是镜中花。想眼中,能有多少泪珠儿,怎经得,秋流到冬尽春流到夏。
\end{qute2sp}


\begin{parag}
    宝玉听了此曲,散漫无稽,不见得好处,\begin{note}甲戌侧:自批驳,妙极!\end{note}但其声韵凄惋,竟能销魂醉魄。因此也不察其原委,问其来历,就暂以此释闷而已。\begin{note}甲戌眉:妙!设言世人亦应如此法看此《红楼梦》一书,更不必追究其隐寓。\end{note}因又看下道:
\end{parag}


\begin{qute2sp}
    \song{恨无常}
    喜荣华正好,恨无常又到。眼睁睁,把万事全抛;荡悠悠,把芳魂消耗。望家乡,路远山遥。故向爹娘梦里相寻告:儿命已入黄泉,天伦呵,须要退步抽身早。\begin{note}甲夹:悲险之至!\end{note}
\end{qute2sp}


\begin{qute2sp}
    \song{分骨肉}
    一帆风雨路三千,把骨肉家园齐来抛闪。恐哭损残年。告爹娘,莫把儿悬念。自古穷通皆有命,离合岂无缘。从今分两地,各自保平安。奴去也,莫牵连。
\end{qute2sp}


\begin{qute2sp}
    \song{乐中悲}
    襁褓中,父母叹双亡。\begin{note}甲侧:意真辞切,过来人见之不免失声。\end{note}纵居那绮罗丛,谁知娇养?幸生来,英雄阔大宽宏量,从未将儿女私情略萦心上。好一似,霁月光风耀玉堂。厮配得才貌仙郎,博得个地久天长,准折得幼年时坎坷形状。终久是云散高唐,水涸湘江。这是尘寰中消长数应当,何必枉悲伤!\begin{note}甲眉:悲壮之极,北曲中不能多得。\end{note}
\end{qute2sp}


\begin{qute2sp}
    \song{世难容}
    气质美如兰,才华阜比仙。\begin{note}甲侧:妙卿实当得起。\end{note}天生成孤癖人皆罕。你道是啖肉食腥膻,\begin{note}甲侧:绝妙!曲文填词中不能多见。\end{note}视绮罗俗厌;却不知太高人愈妒,过洁世同嫌。\begin{note}甲夹:至语。\end{note}可叹这,青灯古殿人将老;辜负了,红粉朱楼春色阑。到头来,依旧是风尘肮脏违心愿;好一似,无瑕美玉遭泥陷。又何须,王孙公子叹无缘。
\end{qute2sp}


\begin{qute2sp}
    \song{喜冤家}
    中山狼,无情兽,全不念当日根由。一味的,骄奢淫荡贪还构。觑著那,侯门艳质同蒲柳;作践的,公府千金似下流。叹芳魂艳魄,一载荡悠悠。\begin{note}甲双:题只十二钗,却无人不有,无事不备。\end{note}
\end{qute2sp}


\begin{qute2sp}
    \song{虚花悟}
    将那三春看破,桃红柳绿待如何?把这韶华打灭,觅那情淡天和。说什么,天上夭桃盛,云中杏蕊多!到头来,谁见把秋挨过?则看那,白杨村里人呜咽,青枫林下鬼吟哦。更兼着,连天衰草遮坟墓。这的是,昨贫今富人劳碌,春荣秋谢花折磨。似这般,生关死劫谁能躲?闻道说,西方宝树唤婆娑,上结著长生果。\begin{note}甲夹:末句、开句、收句。\end{note}
\end{qute2sp}


\begin{qute2sp}
    \song{聪明累}
    机关算尽太聪明,反算了卿卿性命。\begin{note}甲侧:警拔之句。\end{note}生前心已碎,死后性灵空。家富人宁,终有个,家亡人散各奔腾。枉费了,意慭慭半世心;好一似,荡悠悠三更梦。\begin{note}甲眉:过来人睹此,宁不放声一哭?\end{note}忽喇喇如大厦倾,昏惨惨似灯将尽。呀!一场欢喜忽悲辛。叹人世,终难定!\begin{note}甲夹:见得到。\end{note}
\end{qute2sp}


\begin{qute2sp}
    \song{留余庆}
    留余庆,留余庆,忽遇恩人;幸娘亲,幸娘亲,积得阴功。劝人生,济困扶穷,休似俺那银钱上,忘骨肉的狠舅奸兄!正是乘除加减,上有苍穹。
\end{qute2sp}


\begin{qute2sp}
    \song{晚韶华}
    镜里恩情,\begin{note}甲夹:起得妙!\end{note}更那堪梦里功名!那美韶华去之何迅!再休提绣帐鸳衾。只这戴珠冠,披凤袄,也抵不了无常性命。虽说是,人生莫受老来贫,也须要阴骘积儿孙。气昂昂头戴簪缨,光闪闪腰悬金印;威赫赫爵位高登,昏惨惨黄泉路近。问古来将相可还存?也只是虚名儿与后人钦敬。
\end{qute2sp}


\begin{qute2sp}
    \song{好事终}
    画梁春尽落香尘。\begin{note}甲侧:六朝妙句。\end{note}擅风情,秉月貌,便是败家的根本。箕裘颓堕皆从敬,\begin{note}甲侧:深意他人不解。\end{note}家事消亡首罪宁。宿孽总因情。\begin{note}甲双:是作者具菩萨之心,秉刀斧之笔,撰成此书,一字不可更,一语不可少。\end{note}
\end{qute2sp}


\begin{qute2sp}
    \song{收尾 飞鸟各投林}\begin{note}甲双:收尾愈觉悲惨可畏。\end{note}
    为官的,家业凋零;富贵的,金银散尽。\begin{note}甲侧:二句先总宁荣。\end{note}有恩的,死里逃生;无情的,分明照应。欠命的,命已还;欠泪的,泪已尽。冤冤相报实非轻,分离合聚皆前定。欲知命短问前生,老来富贵也真侥幸。看破的,遁入空门;痴迷的,枉送了性命。\begin{note}甲侧:将通部女子一总。\end{note}好一似食尽鸟投林,落了片白茫茫大地真干净!\begin{note}甲夹:又照看葫芦庙。与树倒猢狲散反照。\end{note}
\end{qute2sp}


\begin{parag}
    歌毕,还又歌别曲。\begin{note}甲侧:是极!香菱、晴雯辈岂可无,亦不必再。\end{note}警幻见宝玉甚无趣味,因叹:“痴儿竟尚未悟!”那宝玉忙止歌姬不必再曲,自觉朦胧恍惚,告醉求卧。警幻便命撤去残席,送宝玉至一香闺绣阁之中,其间铺陈之盛,乃素所未见之物。更可骇者,早有一位女子在内,其鲜艳妩媚,有似乎宝钗,风流袅娜,则又如黛玉。\begin{note}甲侧:难得双兼,妙极!\end{note}正不知何意。忽警幻道:“尘世中多少富贵之家,那些绿窗风月,绣阁烟霞,皆被淫污纨绔与那些流荡女子悉皆玷辱。\begin{note}甲侧:真极!\end{note}更可恨者,自古来多少轻薄浪子,皆以好色不淫为饰,又以情而不淫作案,此皆饰非掩丑之语也。好色即淫,知情更淫。是以巫山之会,云雨之欢,皆由既悦其色,复恋其情所致也。\begin{note}甲侧:“好色而不淫”,今翻案,奇甚!\end{note}吾所爱汝者,乃天下古今第一淫人也。”\begin{note}甲侧:多大胆量敢作如此之文!甲眉:绛芸轩中诸事情景由此而生。\end{note}宝玉听了,唬的忙答道:“仙姑差了。我因懒于读书,家父母尚每垂训饬,岂敢再冒淫字?况且年纪尚小。不知淫字为何物。”警幻道:“非也。淫虽一理。意则有别。如世之好淫者,不过悦容貌,喜歌舞,调笑无厌,云雨无时,恨不能尽天下之美女供我片时之趣兴,\begin{note}甲侧:说得恳切恰当之至!\end{note}此皆皮肤淫滥之蠢物耳。如尔则天分中生成一段痴情,吾辈推之为‘意淫’。\begin{note}甲侧:二字新雅。\end{note}‘意淫’二字,惟心会而不可口传,可神通而不能语达。\begin{note}甲侧:按宝玉一生心性,只不过是体贴二字,故曰意淫。\end{note}汝今独得此二字,在闺闼中,固可为良友,然于世道中未免迂阔怪诡,百口嘲谤,万目睚眦。今既遇令祖宁荣二公剖腹深嘱,吾不忍君独为我闺阁增光,见弃于世道,是特引前来,醉以灵酒,沁以仙茗,警以妙曲,再将吾妹一人,乳名兼美\begin{note}甲侧:妙!盖指薛林而言也。\end{note}字可卿者,许配于汝。今夕良时,即可成姻。不过领汝领略此仙闺幻境之风光,尚然如此,何况尘境之情哉?今而后万万解释,改悟前情,将谨勤有用的工夫,置身于经济之道。”说毕便秘授以云雨之事,推宝玉入帐。那宝玉恍恍惚惚,依警幻所嘱之言,未免有阳台巫峡之会。数日来,柔情绻缱,软语温存,与可卿难解难分。
\end{parag}


\begin{parag}
    那日,警幻携宝玉、可卿闲游至一个所在,但见荆榛遍地,狼虎同群,忽尔大河阻路,黑水淌洋,又无桥梁可通。\begin{note}甲侧:若有桥梁可通,则世路人情犹不算艰难。\end{note}宝玉正自徬徨,只听警幻道:“宝玉再休前进,作速回头要紧!”\begin{note}甲侧:机锋。点醒世人。\end{note}宝玉忙止步问道:“此系何处?”警幻道:“此即迷津也。深有万丈,遥亘千里,中无舟楫可通,只有一个木筏,乃木居士掌舵,灰侍者撑篙,不受金银之谢,但遇有缘者渡之。尔今偶游至此,如堕落其中,则深负我从前一番以情悟道、守理衷情之言。”宝玉方欲回言,只听迷津内水响如雷,竟有一夜叉般怪物撺出,直扑而来。吓得宝玉汗下如雨,一面失声喊叫:“可卿救我!可卿救我!”慌得袭人、媚人等上来扶起,拉手说:“宝玉别怕,我们在这里!”秦氏在外听见,连忙进来,一面说ㄚ鬟们好生看着猫儿狗儿打架,又闻宝玉口中连叫可卿救我,\begin{note}甲侧:云龙作雨,不知何为龙,何为云,何为雨。\end{note}因纳闷道:“我的小名,这里没人知道,他如何从梦里叫出来?”正是:
\end{parag}


\begin{poem}
    \begin{pl}一场幽梦同谁诉,千古情人独我知。\end{pl}
\end{poem}

