\chap{一十九}{情切切良宵花解语 意绵绵静日玉生香}


\begin{parag}
    \begin{note}蒙回前诗:彩笔辉光若转环,情心魔态几千般。写成浓淡兼深浅,活现痴人恋恋间。\end{note}
\end{parag}


\begin{parag}
    话说贾妃回宫,次日见驾谢恩,并回奏归省之事,龙颜甚悦,又发内帑彩缎金银等物,以赐贾政及各椒房等员,\begin{note}庚双夹:补还一句,细。方见省亲不独贾家一门是也。\end{note}不必细说。
\end{parag}


\begin{parag}
    且说荣宁二府中连日用尽心力,真是人人力倦,各各神疲,又将园中一应陈设动用之物收拾了两三天方完。第一个凤姐事多任重,别人或可偷安躲静,独他是不能脱得的;二则本性要强,不肯落人褒贬,只扎挣著与无事的人一样。\begin{note}庚双夹:伏下病源。\end{note}第一个宝玉是极无事最闲暇的。偏这日一早,袭人的母亲又亲来回过贾母,接袭人家去吃年茶,晚间才得回来。\begin{note}庚双夹:一回一回各生机轴,总在人意想之外。\end{note}因此,宝玉只和众丫头们掷骰子赶围棋作戏。\begin{note}庚双夹:写出正月光景。\end{note}正在房内顽的没兴头,忽见丫头们来回说:“东府珍大爷来请过去看戏、放花灯。”宝玉听了,便命换衣裳。才要去时,忽又有贾妃赐出糖蒸酥酪来;\begin{note}庚双夹:总是新正妙景。\end{note}宝玉想上次袭人喜吃此物,便命留与袭人了。自己回过贾母,过去看戏。
\end{parag}


\begin{parag}
    谁想贾珍这边唱的是《丁郎认父》、《黄伯央大摆阴魂阵》,更有《孙行者大闹天宫》、《姜子牙斩将封神》等类的戏文。\begin{note}庚双夹:真真热闹。\end{note}倏尔神鬼乱出,忽又妖魔毕露,甚至于扬幡过会,号佛行香,锣鼓喊叫之声闻于巷外。\begin{note}庚双夹:形容刻薄之至,弋阳腔能事毕矣。阅至此则有如耳内喧哗、目中离乱,后文至隔墙闻“袅晴丝”数曲,则有如魂随笛转、魄逐歌销。形容一事,一事毕肖,石头是第一能手矣。\end{note}满街之人个个都赞:“好热闹戏,别人家断不能有的。”\begin{note}庚双夹:必有之言。\end{note}宝玉见那繁华热闹到如此不堪的田地,只略坐了一坐,便走开各处闲耍。先是进内去和尤氏和丫鬟姬妾说笑了一回,便出二门来。尤氏等仍料他出来看戏,遂也不曾照管。贾珍、贾琏、薛蟠等只顾猜枚行令,百般作乐,也不理论,纵一时不见他在座,只道在里边去了,故也不问。至于跟宝玉的小厮们,那年纪大些的,知宝玉这一来了,必是晚上才散,因此偷空也有去会赌的,也有往亲友家去吃年茶的,更有或嫖或饮,都私散了,待晚间再来;那些小的,都钻进戏房里瞧热闹去了。
\end{parag}


\begin{parag}
    宝玉见一个人没有,因想“这里素日有个小书房,名……,\begin{subnote}按:此处有缺文\end{subnote}内曾挂著一轴美人,极画的得神。今日这般热闹,想那里自然……\begin{subnote}按:此处有缺文\end{subnote}那美人也自然是寂寞的,须得我去望慰他一回。”\begin{note}庚双夹:极不通极胡说中写出绝代情痴,宜乎众人谓之疯傻。\end{note}\begin{note}蒙侧:天生一段痴情,所谓“情不情”也。\end{note}想著,便往书房里来。刚到窗前,闻得房内有呻吟之韵。宝玉倒唬了一跳:敢是美人活了不成?\begin{note}庚双夹:又带出小儿心意,一丝不落。\end{note}乃乍著胆子,舔破窗纸,向内一看,那轴美人却不曾活,却是茗烟按著一个女孩子,也干那警幻所训之事。宝玉禁不住大叫:“了不得!”一脚踹进门去,将那两个唬开了,抖衣而颤。
\end{parag}


\begin{parag}
    茗烟见是宝玉,忙跪求不迭。宝玉道:“青天白日,这是怎么说。\begin{note}庚双夹:开口便好。\end{note}珍大爷知道,你是死是活?”一面看那丫头,虽不标致,倒还白净,些微亦动人处,羞的面红耳赤,低首无言。宝玉跺脚道:“还不快跑!”\begin{note}庚双夹:此等搜神夺魄至神至妙处只在囫囵不解处得。\end{note}一语提醒了那丫头,飞也似去了。宝玉又赶出去,叫道:“你别怕,我是不告诉人的。”\begin{note}庚双夹:活宝玉,移之他人不可。\end{note}急的茗烟在后叫:“祖宗,这是分明告诉人了!”宝玉因问:“那丫头十几岁了?”茗烟道:“大不过十六七岁了。”宝玉道:“连他的岁属也不问问,别的自然越发不知了。可见他白认得你了。可怜,可怜!”\begin{note}庚双夹:按此书中写一宝玉,其宝玉之为人是我辈于书中见而知有此人,实未目曾亲睹者。又写宝玉之发言每每令人不解,宝玉之生性件件令人可笑,不独不曾于世上亲见这样的人,即阅今古所有之小说奇传中亦未见这样的文字。于颦儿处更为甚。其囫囵不解之中实可解,可解之中又说不出理路,合目思之,却如真见一宝玉真闻此言者,移至第二人万不可,亦不成文字矣。余阅《石头记》中至奇至妙之文,全在宝玉颦儿至痴至呆囫囵不解之语中,其誓词雅迷酒令奇衣奇食奇玩等类固他书中未能,然在此书中评之,犹为二著。\end{note}又问:“名字叫什么?”茗烟大笑道:“若说出名字来话长,真真新鲜奇文,竟是写不出来的。\begin{note}庚双夹:若都写得出来,何以见此书中之妙?脂砚。\end{note}据他说,他母亲养他的时节做了一个梦,\begin{note}庚双夹:又一个梦,只是随手成趣耳。\end{note}梦见得了一匹锦,上面是五色富贵万不断头的花样,\begin{note}庚双夹:千奇百怪之想,所谓“牛溲马渤皆至乐也,鱼鸟昆虫皆妙文也”,天地间无一物不是妙物,无一物不可成文,但在人意舍取耳。此皆信手拈来随笔成趣,大游戏、大慧悟、大解脱之妙文也。\end{note}所以他的名字叫作万儿。”宝玉听了笑道:“真也新奇,想必他将来有些造化。”说著,沉思一会。
\end{parag}


\begin{parag}
    茗烟因问:“二爷为何不看这样的好戏?”宝玉道:“看了半日,怪烦的,出来逛逛,就遇见你们了。这会子作什么呢?”茗烟嘻嘻笑道:“这会子没人知道,我悄悄的引二爷往城外逛逛去,一会子再往这里来,他们就不知道了。”\begin{note}庚双夹:茗烟此时只要掩饰方才之过,故设此以悦宝玉之心。\end{note}宝玉道:“不好,仔细花子拐了去。便是他们知道了,又闹大了,不如往熟近些的地方去,还可就来。”茗烟道:“熟近地方,谁家可去?这却难了。”宝玉笑道:“依我的主意,咱们竟找你花大姐姐去,瞧他在家作什么呢。”\begin{note}庚双夹:妙!宝玉心中早安著这著,但恐茗烟不肯引去耳。恰遇茗烟私行淫媾,为宝玉所胁,故以城外引以悦其心,宝玉始悦,出往花家去。非茗烟适有罪所胁,万不敢如此私引出外。别家子弟尚不敢私出,况宝玉哉?况茗烟哉?文字著楔细甚。\end{note}茗烟笑道:“好,好!倒忘了他家。”又道:“若他们知道了,说我引著二爷胡走,要打我呢?”\begin{note}庚双夹:必不可少之语。\end{note}宝玉笑道:“有我呢。”茗烟听说,拉了马,二人从后门就走了。
\end{parag}


\begin{parag}
    幸而袭人家不远,不过一半里路程,展眼已到门前。茗烟先进去叫袭人之兄花自芳。\begin{note}庚双夹:随姓成名,随手成文。\end{note}此时袭人之母接了袭人与几个外甥女儿、\begin{note}庚双夹:一树千枝,一源万派,无意随手,伏脉千里。\end{note}几个侄女儿来家,正吃果茶。听见外面有人叫“花大哥”,花自芳忙出去看时,见是他主仆两个,唬的惊疑不止,连忙抱下宝玉来,至院内嚷道:“宝二爷来了!”别人听见还可,袭人听了,也不知为何,忙跑出来迎著宝玉,一把拉著问:“你怎么来了?”宝玉笑道:“我怪闷的,来瞧瞧你作什么呢。”袭人听了,才放下心来,\begin{note}庚双夹:精细周到。\end{note}嗐了一声,笑\begin{note}庚双夹:转至“笑”字,妙甚!\end{note}道:“你也忒胡闹了,\begin{note}庚双夹:该说,说得是。\end{note}可作什么来呢!”一面又问茗烟:“还有谁跟来?”\begin{note}庚双夹:细。\end{note}茗烟笑道:“别人都不知道,就只我们两个。”袭人听了,复又惊慌,\begin{note}庚双夹:是必有之神理,非特故作顿挫。\end{note}说道:“这还了得!倘或碰见了人,或是遇见了老爷,街上人挤车碰,马轿纷纷的,若有个闪失,也是顽得的!你们的胆子比斗还大。都是茗烟调唆的,回去我定告诉嬷嬷们打你。”\begin{note}庚双夹:该说,说得更是。\end{note}茗烟撅了嘴道:“二爷骂著打著,叫我引了来,这会子推到我身上。我说别来罢,不然我们还去罢。”\begin{note}庚双夹:茗烟贼。\end{note}花自芳忙劝:“罢了,已是来了,也不用多说了。只是茅檐草舍,又窄又脏,爷怎么坐呢?”
\end{parag}


\begin{parag}
    袭人之母也早迎了出来。袭人拉著宝玉进去。宝玉见房中三五个女孩儿,见他进来,都低了头,羞惭惭的。花自芳母子两个百般怕宝玉冷,又让他上炕,又忙另摆果桌,又忙倒好茶。\begin{note}庚双夹:连用三“又”字,上文一个,百般神理活现。\end{note}袭人笑道:“你们不用白忙,\begin{note}庚双夹:妙!不写袭卿,正是忙之至。若一写袭人忙,便是庸俗小派了。\end{note}我自然知道。果子也不用摆,也不敢乱给东西吃。”\begin{note}庚双夹:如此至微至小中便带出家常情,他书写不及此。\end{note}一面说,一面将自己的坐褥拿了铺在一个炕上,宝玉坐了;用自己的脚炉垫了脚,向荷包内取出两个梅花香饼儿来,又将自己的手炉掀开焚上,仍盖好,放与宝玉怀内;然后将自己的茶杯斟了茶,送与宝玉。\begin{note}庚双夹:用四“自己”字,写得宝袭二人素日如何亲洽如何尊荣,此时一盘托出。盖素日身居侯府绮罗锦绣之中,其安富尊荣之宝玉亲密浃洽勤慎委婉之袭人,是分所应当不必写者也。今于此一补,更见二人平素之情意,且暗透此回中所有母女兄长欲为赎身角口等未到之过文。\end{note}彼时他母兄已是忙另齐齐整整摆上一桌子果品来。袭人见总无可吃之物,\begin{note}庚双夹:补明宝玉自幼何等娇贵,以此一句留与下部后数十回“寒冬噎酸虀,雪夜围破毡”等处对看,可为后生过分之戒。叹叹!\end{note}因笑道:“既来了,没有空去之理,好歹尝一点儿,也是来我家一趟。”\begin{note}庚双夹:得意之态,是才与母兄较争以后之神理。最细。\end{note}说著,便拈了几个松子穰,\begin{note}庚双夹:唯此品稍可一拈,别品便大错了。\end{note}吹去细皮,用手帕托著送与宝玉。
\end{parag}


\begin{parag}
    宝玉看见袭人两眼微红,粉光融滑,\begin{note}庚双夹:八字画出一才收泪之女儿,是好形容,切实宝玉眼中意中。\end{note}因悄问袭人:“好好的哭什么?”袭人笑道:“何尝哭,才迷了眼揉的。”因此便遮掩过了。\begin{note}庚双夹:伏下后文所补未到多少文字。\end{note}当下宝玉穿著大红金蟒狐腋箭袖,外罩石青貂裘排穗褂。袭人道:“你特为往这里来又换新服,他们\begin{note}庚双夹:指晴雯麝月等。\end{note}就不问你往那去的?”\begin{note}庚双夹:必有是问。阅此则又笑尽小说中无故家常穿红挂绿绮绣绫罗等语,自谓是富贵语,究竟反是寒酸话。\end{note}宝玉笑道:“珍大哥那里去看戏换的。”袭人点头。又道:“坐一坐就回去罢,这个地方不是你来的。”宝玉笑道:“你就家去才好呢,我还替你留著好东西呢。”\begin{note}庚双夹:生受,切己之事。\end{note}袭人悄笑道:“悄悄的,叫他们听著什么意思。”\begin{note}庚双夹:想见二人来日情常。\end{note}一面又伸手从宝玉项上将通灵玉摘了下来,向他姊妹们笑道:“你们见识见识。时常说起来都当希罕,恨不能一见,今儿可尽力瞧了。再瞧什么希罕物儿,也不过是这么个东西。”\begin{note}庚双夹:行文至此,固好看之极,且勿论按此言固是袭人得意之话,盖言你等所稀罕不得一见之宝我却常守常见视为平物。然余今窥其用意之旨,则是作者借此正为贬玉原非大观者也。\end{note}说毕,递与他们传看了一遍,仍与宝玉挂好。\begin{note}庚眉:自“一把拉住”至此诸形景动作,袭卿有意微露锋芒,轩中隐事也。\end{note}又命他哥哥去或雇一乘小轿,或雇一辆小车,送宝玉回去。花自芳道:“有我送去,骑马也不妨了。”\begin{note}庚侧:只知保重耳。\end{note}袭人道:“不为不妨,为的是碰见人。”\begin{note}庚双夹:细极!\end{note}
\end{parag}


\begin{parag}
    花自芳忙去雇了一顶小轿来,众人也不敢相留,只得送宝玉出去。袭人又抓果子与茗烟,又把些钱与他买花炮放,教他:“不可告诉人,连你也有不是。”一直送宝玉至门前,看著上轿,放下轿帘。花、茗二人牵马跟随。来至宁府街,茗烟命住轿,向花自芳道:“须等我同二爷还到东府里混一混,才过去的,不然人家就疑惑了。”花自芳听说有理,忙将宝玉抱出轿来,送上马去。宝玉笑说:“倒难为你了。”\begin{note}庚侧:公子口气。\end{note}于是仍进后门来。俱不在话下。
\end{parag}


\begin{parag}
    却说宝玉自出了门,他房中这些丫鬟们都越发恣意的顽笑,也有赶围棋的,也有掷骰抹牌的,磕了一地瓜子皮。偏奶母李嬷嬷拄拐进来请安,瞧瞧宝玉,见宝玉不在家,丫鬟们只顾玩闹,十分看不过。\begin{note}庚双夹:人人都看不过,独宝玉看得过。\end{note}因叹道:“只从我出去了,不大进来,你们越发没了样儿了,\begin{note}庚双夹:说得是,原该说。\end{note}别的妈妈们越不敢说你们了。\begin{note}庚双夹:补得好!宝玉虽不吃乳,岂无伴从之媪妪哉?\end{note}那宝玉是个丈八的灯台——照见人家,照不见自家的。\begin{note}庚双夹:用俗语入妙。\end{note}只知嫌人家脏,这是他的屋子,由著你们糟蹋,越不成体统了。”\begin{note}庚双夹:所以为今古未有之一宝玉。\end{note}这些丫头们明知宝玉不讲究这些,二则李嬷嬷已是告老解事出去的了,\begin{note}庚双夹:调侃入微,妙妙!\end{note}如今管不著他们。因此只顾顽,并不理他。那李嬷嬷还只管问“宝玉如今一顿吃多少饭”、“什么时候睡觉”等语。丫头们总胡乱答应。有的说:“好一个讨厌的老货!”\begin{note}庚侧:实在有的。\end{note}
\end{parag}


\begin{parag}
    李嬷嬷又问道:“这盖碗里是酥酪,怎不送与我去?我就吃了罢”说毕,拿匙就吃。\begin{note}庚双夹:写龙钟奶母,便是龙钟奶母。\end{note}一个丫头道:“快别动!那是说了给袭人留著的,\begin{note}庚双夹:过下无痕。\end{note}回来又惹气了。\begin{note}庚双夹:照应茜雪枫露茶前案。\end{note}你老人家自己承认,别带累我们受气。”\begin{note}庚双夹:这等话语声口,必是晴雯无疑。\end{note}李嬷嬷听了,又气又愧,便说道:“我不信他这样坏了。别说我吃了一碗牛奶,就是再比这个值钱的,也是应该的。难道待袭人比我还重?难道他不想想怎么长大了?我的血变的奶,吃的长这么大,如今我吃他一碗牛奶,他就生气了?我偏吃了,看怎么样!你们看袭人不知怎样,那是我手里调理出来的毛丫头,什么阿物儿!”\begin{note}庚双夹:是暂委屈唐突袭卿,然亦怨不得李媪。\end{note}一面说,一面赌气将酥酪吃尽。又一丫头笑道:“他们不会说话,怨不得你老人家生气。宝玉还时常送东西孝敬你老去,岂有为这个不自在的。”\begin{note}庚双夹:听这声口,必是麝月无疑。\end{note}李嬷嬷道:“你们也不必妆狐媚子哄我,打量上次为茶撵茜雪的事我不知道呢。\begin{note}庚双夹:照应前文,又用一“撵”,屈杀宝玉,然在李媪心中口中毕肖。\end{note}明儿有了不是,我再来领!”说著,赌气去了。\begin{note}庚双夹:过至下回。\end{note}
\end{parag}


\begin{parag}
    少时,宝玉回来,命人去接袭人。只见晴雯躺在床上不动,\begin{note}庚双夹:娇态已惯。\end{note}宝玉因问:“敢是病了?再不然输了?”秋纹道:“他倒是赢的。谁知李老奶奶来了,混输了,他气的睡去了。”宝玉笑道:“你别和他一般见识,由他去就是了。” 说著,袭人已来,彼此相见。袭人又问宝玉何处吃饭,多早晚回来,又代母妹问诸同伴姊妹好。一时换衣卸妆。宝玉命取酥酪来,丫鬟们回说:“李奶奶吃了。”宝玉才要说话,袭人便忙笑说道:“原来是留的这个,多谢费心。前儿我吃的时候好吃,吃过了好肚子疼,足闹的吐了才好。他吃了倒好,搁在这里倒白糟蹋了。\begin{note}庚双夹:与前文应失手碎钟遥对,通部袭人皆是如此,一丝不错。\end{note}我只想风干栗子吃,你替我剥栗子,我去铺炕。”\begin{note}庚双夹:必如此方是。\end{note}
\end{parag}


\begin{parag}
    宝玉听了信以为真,方把酥酪丢开,取栗子来,自向灯前检剥。一面见众人不在房中,乃笑问袭人道:“今儿那个穿红的是你什么人?”\begin{note}庚双夹:若是见过女儿之后没有一段文字便不是宝玉,亦非《石头记》矣。\end{note}袭人道:“那是我两姨妹子。”宝玉听了,赞叹了两声。\begin{note}庚双夹:这一赞叹又是令人囫囵不解之语,只此便抵过一大篇文字。\end{note}袭人道:“叹什么?\begin{note}庚双夹:只一“叹”字便引出“花解语”一回来。\end{note}我知道你心里的缘故,想是说他那里配红的。”\begin{note}庚双夹:补出宝玉素喜红色,这是激语。\end{note}宝玉笑道:“不是,不是。那样的不配穿红的,谁还敢穿。\begin{note}庚双夹:活宝玉。\end{note}我因为见他实在好的很,怎么也得他在咱们家就好了。”\begin{note}庚双夹:妙谈妙意。\end{note}袭人冷笑道:“我一个人是奴才命罢了,难道连我的亲戚都是奴才命不成?定还要拣实在好的丫头才往你家来?”\begin{note}庚双夹:妙答。宝玉并未说“奴才”二字,袭人连补“奴才”二字最是劲节,怨不得作此语。\end{note}宝玉听了,忙笑道:“你又多心了。我说往咱们家来,必定是奴才不成?\begin{note}蒙双夹:勉强,如闻。\end{note}说亲戚就使不得?”\begin{note}庚双夹:更勉强。蒙侧:这样妙文,何处得来?非目见身行,岂能如此的确?\end{note}袭人道:“那也搬配不上。”\begin{note}庚双夹:说得是。\end{note}宝玉便不肯再说,只是剥粟子。袭人笑道:“怎么不言语了?想是我才冒撞冲犯了你?明儿赌气花几两银子买他们进来就是了。”\begin{note}庚双夹:总是故意激他。\end{note}宝玉笑道:“你说的话,怎么叫我答言呢。我不过是赞他好,正配生在这深堂大院里,没的我们这种浊物\begin{note}庚双夹:妙号!后文又曰“须眉浊物”之称,今古未有之一人始有此今古未有之妙称妙号。\end{note}倒生在这里。”\begin{note}庚双夹:此皆宝玉心中意中确实之念,非前勉强之词,所以谓今古未有之一人耳。听其囫囵不解之言,察其幽微感触之心,审其痴妄委婉之意,皆今古未见之人,亦是今古未见之文字。说不得贤,说不得愚,说不得不肖,说不得善,说不得恶,说不得光明正大,说不得混账恶赖,说不得聪明才俊,说不得庸俗平□,说不得好色好淫,说不得情痴情种,恰恰只有一颦儿可对,令他人徒加评论,总未摸著他二人是何等脱胎、何等心臆、何等骨肉。余阅此书,亦爱其文字耳,实亦不能评出此二人终是何等人物。后观《情榜》评曰“宝玉情不情”,“黛玉情情”,此二评自在评痴之上,亦属囫囵不解,妙甚!\end{note}袭人道:“他虽没这造化,倒也是娇生惯养的呢,我姨爹姨娘的宝贝。如今十七岁,各样的嫁妆都齐备了,明年就出嫁。”\begin{note}庚双夹:所谓不入耳之言也。\end{note}
\end{parag}


\begin{parag}
    宝玉听了“出嫁”二字,不禁又嗐了两声。\begin{note}庚双夹:心思另是一样,余前评可见。\end{note}正是不自在,又听袭人叹道:\begin{note}庚双夹:袭人亦叹,自有别论。\end{note}“只从我来这几年,姊妹们都不得在一处。如今我要回去了,他们又都去了。”宝玉听这话内有文章,\begin{note}庚双夹:余亦如此。\end{note}不觉一惊,\begin{note}庚双夹:余亦吃惊。\end{note}忙丢下粟子,问道:“怎么,你如今要回去了?”袭人道:“我今儿听见我妈和哥哥商议,教我再耐烦一年,明年他们上来,就赎我出去的呢。”\begin{note}庚双夹:即余今日犹难为情,况当日之宝玉哉?\end{note}宝玉听了这话,越发怔了,因问:“为什么要赎你?”袭人道:“这话奇了!我又比不得是这里的家生子儿,一家子都在别处,独我一个人在这里,怎么是个了局?”\begin{note}庚双夹:说得极是。\end{note}宝玉道:“我不叫你去也难。”\begin{note}庚双夹:是头一句驳,故用贵公子声口,无理。\end{note}袭人道:“从来没这道理。便是朝廷宫里,也有个定例,或几年一选,几年一入,也没有个长远留下人的理,别说你了!”\begin{note}庚双夹:一驳,更有理。\end{note}
\end{parag}


\begin{parag}
    宝玉想一想,果然有理。\begin{note}庚双夹:自然。\end{note}又道:“老太太不放你也难。”\begin{note}庚双夹:第二层伏祖母溺爱,更是无理。\end{note}袭人道:“为什么不放?我果然是个最难得的,或者感动了老太太、太太,\begin{note}庚双夹:宝玉并不提王夫人,袭人偏自补出,周密之至!\end{note}必不放我出去的,设或多给我们家几两银子,留下我,然或有之;其实我又不过是个平常的人,比我强的多而且多。自我从小儿来了,跟著老太太,先服侍了史大姑娘几年,\begin{note}庚双夹:百忙中又补出湘云来,真是七穿八达,得空便入。\end{note}如今又服侍了你几年。如今我们家来赎,正是该叫去的,只怕连身价也不要,就开恩叫我去呢。要说为服侍的你好,不叫我去,断然没有的事。那服侍的好,是分内应当的,\begin{note}庚侧:这却是真心话。\end{note}不是什么奇功。我去了,仍旧有好的来了,不是没了我就不成事。”\begin{note}庚双夹:再一驳,更精细更有理。\end{note}宝玉听了这些话,竟是有去的理,无留的理,\begin{note}庚双夹:自然。\end{note}心内越发急了,\begin{note}庚双夹:原当急。\end{note}因又道:“虽然如此说,我只一心留下你,不怕老太太不和你母亲说。多多给你母亲些银子,他也不好意思接你了。”\begin{note}庚双夹:急心肠,故入于霸道。无理。\end{note}袭人道:“我妈自然不敢强。且漫说和他好说,又多给银子;就便不和他好说,一个钱也不给,安心要强留下我,他也不敢不依。但只是咱们家从没有干过这倚势仗贵霸道的事。这比不得别的东西,因为你喜欢,加十倍利弄了来给你,那卖的人不得吃亏,可以行得。如今无故平空留下我,于你又无益,反叫我们骨肉分离,这件事,老太太、太太断不肯行的。”\begin{note}庚双夹:三驳,不独更有理,且又补出贾府自家慈善宽厚等事。\end{note}宝玉听了,思忖半晌,\begin{note}庚双夹:正是思忖只有去理实无留理。\end{note}乃说道:“依你说,你是去定了?”\begin{note}庚双夹:自然。\end{note}袭人道:“去定了。”\begin{note}庚侧:口气像极。\end{note}宝玉听了,自思道:“谁知这样一个人,这样薄情无义。”\begin{note}庚双夹:余亦如此见疑。\end{note}乃叹道:“早知道都是要去的,\begin{note}蒙双夹:“都是要去的”,妙!可谓触类旁通,活是宝玉。\end{note}\begin{note}蒙侧:上古至今及后世有情者同声一哭!\end{note}我就不该弄了来,临了剩了我一个孤鬼儿。”\begin{note}庚双夹:可谓见首知尾,活是宝玉。\end{note}说著,便赌气上床睡去了。\begin{note}庚双夹:又到无可奈何之时了。\end{note}
\end{parag}


\begin{parag}
    原来袭人在家,听见他母兄要赎他回去,\begin{note}庚双夹:补前文。\end{note}他就说至死也不回去的。又说:“当日原是你们没饭吃,就剩我还值几两银子,若不叫你们卖,没有个看著老子娘饿死的理。\begin{note}庚侧:孝女,义女。庚双夹:补出袭人幼时艰辛苦状,与前文之香菱、后文之晴雯大同小异,自是又副十二钗中之冠,故不得不补传之。\end{note}如今幸而卖到这个地方,\begin{note}庚双夹:可谓不幸中之幸。\end{note}吃穿和主子一样,又不朝打暮骂。况且如今爹虽没了,你们却又整理的家成业就,复了元气。若果然还艰难,把我赎出来,再多掏澄几个钱,也还罢了,\begin{note}庚侧:孝女,义女。\end{note}其实又不难了。这会子又赎我作什么?权当我死了,\begin{note}庚侧:可怜!\end{note}再不必起赎我的念头!”\begin{note}庚侧:我也要笑。\end{note}\begin{note}蒙侧:同心同志更觉幸福。\end{note}因此哭闹了一阵。\begin{note}庚双夹:以上补在家今日之事,与宝玉问哭一句针对。\end{note}
\end{parag}


\begin{parag}
    他母兄见他这般坚执,自然必不出来的了。况且原是卖倒的死契,明仗著贾宅是慈善宽厚之家,不过求一求,只怕身价银一并赏了这是有的事呢。\begin{note}庚双夹:又夹带出贾府平素施为来,与袭人口中针对。\end{note}二则,贾府中从不曾作践下人,只有恩多威少的。\begin{note}庚双夹:伏下多少后文。\end{note}且凡老少房中所有亲侍的女孩子们,更比待家下众人不同,平常寒薄人家的小姐,也不能那样尊重的。\begin{note}庚双夹:又伏下多少后文。现一句是传中陪客,此一句是传中本旨。\end{note}因此,他母子两个也就死心不赎了。\begin{note}庚双夹:既如此何得袭人又作前语以愚宝玉?不知何意,且看后文。\end{note}次后忽然宝玉去了,他二个又是那般景况,\begin{note}庚双夹:一件闲事一句闲文皆无,警甚。\end{note}他母子二人心下更明白了,越发石头落了地,而且是意外之想,彼此放心,再无赎念了。\begin{note}庚双夹:一段情结。\end{note}
\end{parag}


\begin{parag}
    如今且说袭人自幼见宝玉性格异常,\begin{note}庚双夹:四字好!所谓“说不得好,又说不得不好”也。\end{note}其淘气憨顽自是出于众小儿之外,更有几件千奇百怪口不能言的毛病儿。\begin{note}庚双夹:只如此说更好。所谓“说不得聪明贤良,说不得痴呆愚昧”也。\end{note}近来仗著祖母溺爱,父母亦不能十分严紧拘管,更觉放荡弛纵,\begin{note}庚双夹:四字妙评。\end{note}任性恣情,\begin{note}庚双夹:四字更好。亦不涉于恶,亦不涉于淫,亦不涉于娇,不过一味任性耳。\end{note}最不喜务正。\begin{note}庚双夹:这还是小儿同病。\end{note}每欲劝时,料不能听,今日可巧有赎身之论,故先用骗词,以探其情,以压其气,然后好下箴规。\begin{note}庚双夹:原来如此。\end{note}今见他默默睡去了,知其情有不忍,气已馁堕。\begin{note}庚双夹:不独解语,亦且有智。\end{note}自己原不想栗子吃的,只因怕为酥酪又生事故,亦如茜雪之茶等事,\begin{note}庚双夹:可谓贤而有智术之人。\end{note}是以假以栗子为由,混过宝玉不提就完了。于是命小丫头子们将栗子拿去吃了,自己来推宝玉。只见宝玉泪痕满面,\begin{note}庚双夹:正是无可奈何之时。\end{note}\begin{note}蒙侧:不知何故,我亦掩涕。\end{note}袭人便笑道:“这有什么伤心的,你果然留我,我自然不出去了。”宝玉见这话有文章,\begin{note}庚双夹:宝玉不愚。\end{note}便说道:“你倒说说,我还要怎么留你,我自己也难说了。”\begin{note}庚双夹:二人素常情意。\end{note}袭人笑道:“咱们素日好处,再不用说。但今日你安心留我,不在这上头。我另说出三件事来,你果然依了我,就是你真心留我了,刀搁在脖子上,我也是不出去的了。”
\end{parag}


\begin{parag}
    宝玉忙笑道:“你说,那几件?我都依你。好姐姐,好亲姐姐,\begin{note}庚双夹:叠二语活见从纸上走一宝玉下来,如闻其呼、见其笑。\end{note}别说两三件,就是两三百件,我也依。\begin{note}庚双夹:“两三百”不成话,却是宝玉口中。\end{note}只求你们同看著我,守著我,等我有一日化成了飞灰,\begin{note}庚双夹:脂砚斋所谓“不知是何心思,始得口出此等不成话之至奇至妙之话”,诸公请如何解得,如何评论?所劝者正为此,偏于劝时一犯,妙甚!\end{note}——飞灰还不好,灰还有形有迹,还有知识。\begin{note}庚双夹:厌“还有知识”,奇之不可甚言矣!余则谓人尚无知识者多多。\end{note}”“等我化成一股轻烟,风一吹便散了的时候,你们也管不得我,我也顾不得你们了。那时凭我去,我也凭你们爱那里去就去了。”\begin{note}庚双夹:是聪明,是愚昧,是小儿淘气?余皆不知,只觉悲感难言,奇瑰愈妙。\end{note}话未说完,急的袭人忙握他的嘴,说:“好好的,正为劝你这些,倒更说的狠了。”宝玉忙说道:“再不说这话了。”\begin{note}庚侧:只说今日一次。呵呵,玉兄,玉兄,你到底哄的那一个?\end{note}袭人道:“这是头一件要改的。”宝玉道:“改了。再要说,你就拧嘴。还有什么?”
\end{parag}


\begin{parag}
    袭人道:“第二件,你真喜读书也罢,假喜也罢,\begin{note}庚侧:新鲜,真新鲜!\end{note}只是在老爷跟前或在别人跟前,你别只管批驳诮谤,只作出个喜读书的样子来,\begin{note}庚双夹:所谓“开方便门”。\end{note}\begin{note}庚双夹:宝玉又诮谤读书人,恨此时不能一见如何诮谤。\end{note}也教老爷少生些气,\begin{note}庚侧:大家听听,可是个丫鬟说的话。\end{note}在人前也好说嘴。他心里想著,我家代代念书,只从有了你,不承望你不喜读书,已经他心里又气又恼了。而且背前背后乱说那些混话,凡读书上进的人,你就起个名字叫作‘禄蠹’;\begin{note}庚双夹:二字从古未见,新奇之至!难怨世人谓之可杀,余却最喜。\end{note}又说只除‘明明德’外无书,都是前人自己不能解圣人之书,便另出己意,混编纂出来的。\begin{note}庚双夹:宝玉目中犹有“明明德”三字,心中犹有“圣人”二字,又素日皆作如是等语,宜乎人人谓之疯傻不肖。\end{note}这些话,你怎么怨得老爷不气?不时时打你。叫别人怎么想你?”宝玉笑道:“再不说了。那原是那小时不知天高地厚,信口胡说,如今再不敢说了。\begin{note}庚双夹:又作是语,说不得不乖觉,然又是作者瞒人之处也。\end{note}还有什么?”
\end{parag}


\begin{parag}
    袭人道:“再不许毁僧谤道,\begin{note}庚双夹:一件,是妇女心意。\end{note}调脂弄粉。\begin{note}庚双夹:二件,若不如此,亦非宝玉。\end{note}还有更要紧的一件,\begin{note}庚双夹:忽又作此一语。\end{note}再不许吃人嘴上擦的胭脂了,\begin{note}庚双夹:此一句是闻所未闻之语,宜乎其父母严责也。\end{note}与那爱红的毛病儿。”宝玉道:“都改,都改。再有什么,快说。”袭人笑道:“再也没有了。只是百事检点些,不任意任情的就是了。\begin{note}庚双夹:总包括尽矣。其所谓“花解语”者,大矣!不独冗冗为儿女之分也。\end{note}你若果都依了,便拿八人轿也抬不出我去了。”宝玉笑道:“你在这里长远了,不怕没八人轿你坐。”袭人冷笑道:“这我可不希罕的。有那个福气,没有那个道理。纵坐了,也没甚趣。”\begin{note}庚双夹:调侃不浅,然在袭人能作是语,实可爱可敬可服之至,所谓“花解语”也。\end{note}\begin{note}庚眉:“花解语”一段乃袭卿满心满意将玉兄为终身得靠,千妥万当,故有是。余阅至此,余为袭卿一叹。丁亥春。畸笏叟。\end{note}
\end{parag}


\begin{parag}
    二人正说著,只见秋纹走进来,说:“快三更了,该睡了。方才老太太打发嬷嬷来问,我答应睡了。”宝玉命取表来\begin{note}庚双夹:照应前凤姐之前文。\end{note}看时,果然针已指到亥正,\begin{note}庚双夹:表则是表的写法,前形容自鸣钟则是自鸣钟,各尽其神妙。\end{note}方从新盥漱,宽衣安歇,不在话下。
\end{parag}


\begin{parag}
    至次日清晨,袭人起来,便觉身体发重,头疼目胀,四肢火热。先时还扎挣的住,次后挨不住,只要睡著,因而和衣躺在炕上。\begin{note}庚侧:过下引线。\end{note}宝玉忙回了贾母,传医诊视,说道:“不过偶感风寒,吃一两剂药疏散疏散就好了。”开方去后,令人取药来煎好,刚服下去,命他盖上被渥汗,宝玉自去黛玉房中来看视。\begin{note}庚双夹:为下文留地步。\end{note}
\end{parag}


\begin{parag}
    彼时黛玉自在床上歇午,丫鬟们皆出去自便,满屋内静悄悄的。宝玉揭起绣线软帘,进入里间,只见黛玉睡在那里,忙走上来推他道:“好妹妹,\begin{note}庚双夹:才住了“好姐姐”,又闻“好妹妹”,大约宝玉一日之中一时之内,此六个字未曾暂离口角。妙!\end{note}才吃了饭,又睡觉。”将黛玉唤醒。\begin{note}庚双夹:若是别部书中写,此时之宝玉一进来,便生不轨之心,突萌苟且之念,更有许多贼形鬼状等丑态邪言矣。此却反推唤醒他,毫不在意,所谓说不得淫荡是也。\end{note}黛玉见是宝玉,因说道:“你且出去逛逛,我前儿闹了一夜,今儿还没有歇过来,\begin{note}庚双夹:补出娇怯态度。\end{note}浑身酸疼。”宝玉道:“酸疼事小,睡出来的病大。我替你解闷儿,混过困去就好了。”\begin{note}庚双夹:宝玉又知养身。\end{note}黛玉只合著眼,说道:“我不困,只略歇歇儿,你且别处去闹会子再来。”宝玉推他道:“我往那里去呢,见了别人就怪腻的。”\begin{note}庚双夹:所谓只有一颦可对,亦属怪事。\end{note}
\end{parag}


\begin{parag}
    黛玉听了,嗤的一声笑道:“你既要在这里,那边去老老实实的坐著,咱们说话儿。”宝玉道:“我也歪著。”黛玉道:“你就歪著。”宝玉道:“没有枕头,\begin{note}庚双夹:缠绵秘密入微。\end{note}咱们在一个枕头上。”\begin{note}庚双夹:更妙!渐逼渐近,所谓“意绵绵”也。\end{note}黛玉道:“放屁!\begin{note}庚侧:如闻。\end{note}外面不是枕头?拿一个来枕著。”宝玉出至外间,看了一看,回来笑道:“那个我不要,也不知是那个脏婆子的。”黛玉听了,睁开眼,\begin{note}庚双夹:睁眼。\end{note}起身\begin{note}庚双夹:起身。\end{note}笑\begin{note}庚双夹:笑。\end{note}道:“真真你就是我命中的‘天魔星’!\begin{note}庚双夹:妙语,妙之至!想见其态度。\end{note}请枕这一个。”说著,将自己枕的推与宝玉,又起身将自己的再拿了一个来,自己枕了,二人对面躺下。
\end{parag}


\begin{parag}
    黛玉因看见宝玉左边腮上有钮扣大小的一块血渍,便欠身凑近前来,以手抚之细看,\begin{note}庚双夹:想见其缠绵态度。\end{note}又道:“这又是谁的指甲刮破了?”\begin{note}庚双夹:妙极!补出素日。\end{note}宝玉侧身,一面躲,\begin{note}庚侧:对“推醒”看。\end{note}一面笑道:“不是刮的,只怕是才刚替他们淘漉胭脂膏子,蹭上了一点儿。”\begin{note}庚双夹:遥与后文平儿于怡红院晚妆时对照。\end{note}说著,便找手帕子要揩拭。黛玉便用自己的帕子替他揩拭了,\begin{note}庚双夹:想见其情之脉脉,意之绵绵。\end{note}口内说道:“你又干这些事了。\begin{note}庚双夹:又是劝戒语。\end{note}干也罢了,\begin{note}庚双夹:一转,细极!这方是颦卿,不比别人一味固执死劝。\end{note}必定还要带出幌子来。便是舅舅看不见,别人看见了,又当奇事新鲜话儿去学舌讨好儿,\begin{note}庚双夹:补前文之未到,伏后文之线脉。\end{note}吹到舅舅耳朵里,又该大家不干净惹气。”\begin{note}庚双夹:“大家”二字何妙之至神之至细腻之至!乃父责其子,纵加以笞楚,何能使大家不干净哉?今偏大家不干净,则知贾母如何管孙责子怒于众,及自己心中多少抑郁。难堪难禁,代忧代痛,一齐托出。\end{note}
\end{parag}


\begin{parag}
    宝玉总未听见这些话,\begin{note}庚双夹:可知昨夜“情切切”之语亦属行云流水矣。\end{note}只闻得一股幽香,却是从黛玉袖中发出,闻之令人醉魂酥骨。\begin{note}庚双夹:却像似淫极,然究竟不犯一些淫意。\end{note}宝玉一把便将黛玉的袖子拉住,要瞧笼著何物。黛玉笑道:“冬寒十月,\begin{note}庚侧:口头语,指在春冷之时。\end{note}谁带什么香呢。”宝玉笑道:“既然如此,这香是从那里来的?”黛玉道:“连我也不知道。\begin{note}庚双夹:正是按谚云:“人在气中忘气,鱼在水中忘水。”余今续之曰:“美人忘容,花则忘香。”此则黛玉不知自骨肉中之香同。\end{note}想必是柜子里头的香气,衣服上熏染的也未可知。”\begin{note}庚双夹:有理。\end{note}宝玉摇头道:“未必。这香的气味奇怪,不是那些香饼子、香毬子、香袋子的香。”\begin{note}庚双夹:自然。\end{note}黛玉冷笑\begin{note}庚双夹:冷笑便是文章。\end{note}道:“难道我也有什么‘罗汉’‘真人’给我些香不成?便是得了奇香,也没有亲哥哥亲兄弟弄了花儿、朵儿、霜儿、雪儿替我炮制。\begin{note}庚双夹:活颦儿,一丝不错。\end{note}我有的是那些俗香罢了!”
\end{parag}


\begin{parag}
    宝玉笑道:“凡我说一句,你就拉上这么些,不给你个利害,也不知道,从今儿可不饶你了。”说著翻身起来,将两只手呵了两口,\begin{note}庚双夹:活画。\end{note}便伸手向黛玉膈肢窝内两胁下乱挠。黛玉素性触痒不禁,宝玉两手伸来乱挠,便笑的喘不过气来,口里说:“宝玉!你再闹,我就恼了。”\begin{note}庚双夹:如见如闻。\end{note}宝玉方住了手,笑问道:“你还说这些不说了?”黛玉笑道:“再不敢了。”一面理鬓笑道:“我有奇香,你有‘暖香’没有?”\begin{note}庚双夹:奇闻。\end{note}
\end{parag}


\begin{parag}
    宝玉见问,一时解不来,\begin{note}庚双夹:一时原难解,终逊黛卿一等,正在此等处。\end{note}因问:“什么‘暖香’?”黛玉点头叹笑道:“蠢才,蠢才!你有玉,人家就有金来配你;人家有‘冷香’,你就没有‘暖香’去配?”宝玉方听出来。\begin{note}庚双夹:是颦儿,活画。然这是阿颦一生心事,故每不禁自及之。\end{note}宝玉笑道:“方才求饶,如今更说狠了。”说著,又去伸手。黛玉忙笑道:“好哥哥,我可不敢了。”宝玉笑道:“饶便饶你,只把袖子我闻一闻。”说著,便拉了袖子笼在面上,闻个不住。黛玉夺了手道:“这可该去了。”宝玉笑道:“去,不能。咱们斯斯文文的躺著说话儿。”说著,复又倒下。黛玉也倒下,用手帕子盖上脸。\begin{note}庚双夹:画。\end{note}宝玉有一搭没一搭的说些鬼话,\begin{note}庚双夹:先一总。\end{note}黛玉总不理。宝玉问他几岁上京,路上见何景致古迹,扬州有何遗迹故事,土俗民风。黛玉只不答。
\end{parag}


\begin{parag}
    宝玉只怕他睡出病来,\begin{note}庚双夹:原来只为此故,不暇旁人嘲笑,所以放荡无忌处不特此一件耳。\end{note}便哄他道:“嗳哟!\begin{note}庚侧:像个说故事的。\end{note}你们扬州衙门里有一件大故事,你可知道?”黛玉见他说的郑重,且又正言厉色,只当是真事,因问:“什么事?”宝玉见问,便忍著笑顺口诌道:\begin{note}庚侧:又哄我看书人。\end{note}“扬州有一座黛山,山上有个林子洞。”黛玉笑道:“这就扯谎,自来也没听见这山。”\begin{note}庚侧:山名洞名,颦儿已知之矣。\end{note}宝玉道:“天下山水多著呢,你那里知道这些不成。等我说完了,\begin{note}庚侧:不先了此句,可知此谎再诌不完的。\end{note}你再批评。”黛玉道:“你且说。”宝玉又诌道:“林子洞里原来有群耗子精。那一年腊月初七日,老耗子升座议事,\begin{note}庚双夹:耗子亦能升座且议事,自是耗子有赏罚有制度矣。何今之耗子犹穿壁啮物,其升座者置而不问哉?\end{note}因说:‘明日是腊八,世上人都熬腊八粥。如今我们洞中果品短少,\begin{note}庚侧:难道耗子也要腊八粥吃?一笑。\end{note}须得趁此打劫些来方妙。’\begin{note}庚双夹:议得是,这事宜乎为鼠矣。\end{note}乃拔令箭一枝,遣一能干小耗\begin{note}庚双夹:原来能于此者便是小鼠。\end{note}前去打听。一时小耗回报:‘各处察访打听已毕,惟有山下庙里果米最多。’\begin{note}蒙双夹:庙里原来最多,妙妙!\end{note}老耗问:‘米有几样?果有几品?’小耗道:‘米豆成仓,不可胜记。果品有五种:一红枣,二栗子,三落花生,四菱角,五香芋。’老耗听了大喜,即时点耗前去。乃拔令箭问:‘谁去偷米?’一耗便接令去偷米。又拔令箭问:‘谁去偷豆?’又一耗接令去偷豆。然后一一的都各领令去了。\begin{note}庚侧:玉兄也知琐碎,以抄近为妙。\end{note}只剩了香芋一种,因又拔令箭问:‘谁去偷香芋?’只见一个极小极弱的小耗\begin{note}庚侧:玉兄,玉兄,唐突颦儿了!\end{note}应道:‘我愿去偷香芋。’老耗和众耗见他这样,恐不谙练,且怯懦无力,都不准他去。小耗道:‘我虽年小身弱,却是法术无边,口齿伶俐,机谋深远。\begin{note}庚双夹:凡三句暗为黛玉作评,讽得妙!\end{note}此去管比他们偷的还巧呢。”众耗忙问:‘如何比他们巧呢?’小耗道:‘我不学他们直偷。\begin{note}庚侧:不直偷,可畏可怕。\end{note}我只摇身一变,也变成个香芋,滚在香芋堆里,使人看不出,听不见,却暗暗的用分身法搬运,\begin{note}庚侧:可怕可畏。\end{note}渐渐的就搬运尽了。岂不比直偷硬取的巧些?’\begin{note}庚双夹:果然巧,而且最毒。直偷者可防,此法不能防矣。可惜这样才情这样学术却只一耗耳。\end{note}众耗听了,都道:‘妙却妙,只是不知怎么个变法?你先变个我们瞧瞧。’小耗听了,笑道:‘这个不难,等我变来。’说毕,摇身说‘变’,竟变了一个最标致美貌的一位小姐。\begin{note}庚侧:奇文怪文。\end{note}众耗忙笑说:‘变错了,变错了。原说变果子的,如何变出小姐来?’\begin{note}庚双夹:余亦说变错了。\end{note}小耗现形笑道:“我说你们没见世面,只认得这果子是香芋,却不知盐课林老爷的小姐才是真正的香玉呢。’”\begin{note}庚双夹:前有“试才题对额”,故紧接此一篇无稽乱话,前无则可,此无则不可,盖前系宝玉之懒为者,此系宝玉不得不为者。世人诽谤无碍,奖誉不必。\end{note}
\end{parag}


\begin{parag}
    黛玉听了,翻身爬起来,按著宝玉笑道:“我把你烂了嘴的!我就知道你是编我呢。”说著,便拧的宝玉连连央告,说:“好妹妹,饶我罢,再不敢了!我因为闻你香,忽然想起这个故典来。”黛玉笑道:“饶骂了人,还说是故典呢。”\begin{note}庚眉:“玉生香”是要与“小恙梨香院”对看,愈觉生动活泼,且前以黛玉后以宝钗,特犯不犯,好看煞!丁亥春。 笏叟。\end{note}
\end{parag}


\begin{parag}
    一语未了,只见宝钗走来,\begin{note}庚双夹:妙!\end{note}笑问:“谁说故典呢?我也听听。”黛玉忙让坐,笑道:“你瞧瞧,有谁!他饶骂了人,还说是故典。”宝钗笑道:“原来是宝兄弟,怨不得他,他肚子里的故典原多。\begin{note}庚双夹:妙讽。\end{note}只是可惜一件,\begin{note}庚双夹:妙转。\end{note}凡该用故典之时,他偏就忘了。\begin{note}庚双夹:更妙!\end{note}有今日记得的,前儿夜里的芭蕉诗就该记得。眼面前的倒想不起来,别人冷的那样,你急的只出汗。\begin{note}庚双夹:与前“拭汗”二字针对,不知此书何妙之如此,有许多妙谈妙语、机讽诙谐,各得其时,各尽其理,前梨香院黛玉之讽则偏见,越此则正而趣,二人真是对手,两不相犯。\end{note}这会子偏又有记性了。”黛玉听了笑道:“阿弥陀佛!到底是我的好姐姐,你一般也遇见对子了。可知一还一报,不爽不错的。”刚说到这里,只听宝玉房中一片声嚷,吵闹起来。正是——
\end{parag}


\begin{parag}
    \begin{note}蒙回末总评:若知宝玉真性情者,当留心此回。其于袭人何等留连,其于画美人何等古怪。其遇茗烟事何等怜惜,其于黛玉何等保护。再袭人之痴忠,画人之惹事,茗烟之屈奉,黛玉之痴情,千态万状,笔力劲尖,有水到渠成之象,无微不至。真画出一个上乘智慧之人,入于魔而不悟,甘心堕落。且影出诸魔之神通,亦非冷冷,有势不能登彼岸。凡我众生掩卷自思,或于身心少有补益。小子妄谈,诸公莫怪。\end{note}
\end{parag}


\begin{parag}
    \begin{note}梦:正是:戏谑主人调笑仆,相合姊妹合欢亲。\end{note}
\end{parag}
