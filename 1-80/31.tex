\chap{三十一}{撕扇子作千金一笑 因麒麟伏白首雙星}


\begin{parag}
    \begin{note}庚:“撕扇子”是以不情之物供嬌嗔不知情時之人一笑,所謂“情不情”。\end{note}
\end{parag}


\begin{parag}
    \begin{note}庚:“金玉姻緣”已定,又寫一金麒麟,是間色法也。何顰兒爲其所感?故顰兒謂“情情”。\end{note}
\end{parag}


\begin{parag}
    話說襲人見了自己吐的鮮血在地,也就冷了半截,想著往日常聽人說:“少年吐血,年月不保,縱然命長,終是廢人了。”想起此言,不覺將素日想著後來爭榮 誇耀之心盡皆灰了,眼中不覺滴下淚來。寶玉見他哭了,也不覺心酸起來,因問道:“你心裏覺的怎麼樣?”襲人勉強笑道:“好好的,覺怎麼呢。”寶玉的意思即 刻便要叫人燙黃酒,要山羊血黎洞丸來。襲人拉了他的手,笑道:“你這一鬧不打緊,鬧起多少人來,倒抱怨我輕狂。分明人不知道,倒鬧的人知道了,你也不好, 我也不好。正經明兒你打發小子問問王太醫去,弄點子藥喫喫就好了。人不知鬼不覺的可不好?”寶玉聽了有理,也只得罷了,向案上斟了茶來,給襲人漱了口。襲人知寶玉心內是不安穩的,待要不叫他伏侍,他又必不依;二則定要驚動別人,不如由他去罷:因此只在榻上由寶玉去伏侍。一交五更,寶玉也顧不的梳洗,忙穿衣 出來,將王濟仁叫來,親自確問。王濟仁問其原故,不過是傷損,便說了個丸藥的名字,怎麼服,怎麼敷。寶玉記了,回園依方調治。不在話下。
\end{parag}


\begin{parag}
    這日正是端陽佳節,蒲艾簪門,虎符係臂。午間,王夫人治了酒席,請薛家母女等賞午。寶玉見寶釵淡淡的,也不和他說話,自知是昨兒的原故。王夫人見寶玉 沒精打彩,也只當是金釧兒昨日之事,他沒好意思的,越發不理他。林黛玉見寶玉懶懶的,只當是他因爲得罪了寶釵的原故,心中不自在,形容也就懶懶的。鳳姐昨 日晚間王夫人就告訴了他寶玉金釧的事,知道王夫人不自在,自己如何敢說笑,也就隨著王夫人的氣色行事,更覺淡淡的。賈迎春姊妹見衆人無意思,也都無意思了。因此,大家坐了一坐就散了。
\end{parag}


\begin{parag}
    林黛玉天性喜散不喜聚。他想的也有個道理,他說,“人有聚就有散,聚時歡喜,到散時豈不清冷?既清冷則生傷感,所以不如倒是不聚的好。比如那花開時令人愛慕,謝時則增惆悵,所以倒是不開的好。”故此人以爲喜之時,他反以爲悲。那寶玉的情性只願常聚,生怕一時散了添悲;那花只願常開,生怕一時謝了沒趣; 只到筵散花謝,雖有萬種悲傷,也就無可如何了。因此,今日之筵,大家無興散了,林黛玉倒不覺得,倒是寶玉心中悶悶不樂,回至自己房中長吁短嘆。偏生晴雯上來換衣服,不防又把扇子失了手跌在地下,將股子跌折。寶玉因嘆道:“蠢才,蠢才!將來怎麼樣?明日你自己當家立事,難道也是這麼顧前不顧後的?”晴雯冷笑 道:“二爺近來氣大的很,行動就給臉子瞧。前兒連襲人都打了,今兒又來尋我們的不是。要踢要打憑爺去。就是跌了扇子,也是平常的事。先時連那麼樣的玻璃缸、瑪瑙碗不知弄壞了多少,也沒見個大氣兒,這會子一把扇子就這麼著了。何苦來!要嫌我們就打發我們,再挑好的使。好離好散的,倒不好?”寶玉聽了這些話,氣的渾身亂戰,因說道:“你不用忙,將來有散的日子!”
\end{parag}


\begin{parag}
    襲人在那邊早已聽見,忙趕過來向寶玉道:“好好的,又怎麼了?可是我說的:‘一時我不到,就有事故兒。’”晴雯聽了冷笑道:“姐姐既會說,就該早來, 也省了爺生氣。自古以來,就是你一個人伏侍爺的,我們原沒伏侍過。因爲你伏侍的好,昨日才挨窩心腳;我們不會伏侍的,到明兒還不知是個什麼罪呢!”襲人聽了這話,又是惱,又是愧,待要說幾句話,又見寶玉已經氣的黃了臉,少不得自己忍了性子,推晴雯道:“好妹妹,你出去逛逛,原是我們的不是。”晴雯聽他說 “我們”兩個字,自然是他和寶玉了,不覺又添了酸意,冷笑幾聲,道:“我倒不知道你們是誰,別教我替你們害臊了!便是你們鬼鬼祟祟乾的那事兒,也瞞不過我去,那裏就稱起‘我們’來了。明公正道,連個姑娘還沒掙上去呢,也不過和我似的,那裏就稱上‘我們’了!”襲人羞的臉紫脹起來,想一想,原來是自己把話說 錯了。寶玉一面說:“你們氣不忿,我明兒偏抬舉他。”襲人忙拉了寶玉的手道:“他一個糊塗人,你和他分證什麼?況且你素日又是有擔待的,比這大的過去了多 少,今兒是怎麼了?”晴雯冷笑道:“我原是糊塗人,那裏配和我說話呢!”襲人聽說道:“姑娘倒是和我拌嘴呢,是和二爺拌嘴呢?要是心裏惱我,你只和我說, 不犯著當著二爺吵;要是惱二爺,不該這們吵的萬人知道。我才也不過爲了事,進來勸開了,大家保重。姑娘倒尋上我的晦氣。又不象是惱我,又不象是惱二爺,夾槍帶棒,終久是個什麼主意?我就不多說,讓你說去。”說著便往外走。寶玉向晴雯道:“你也不用生氣,我也猜著你的心事了。我回太太去,你也大了,打發你出去好不好?”晴雯聽了這話,不覺又傷起心來,含恨說道:“爲什麼我出去?要嫌我,變著法兒打發我出去,也不能夠。”寶玉道:“我何曾經過這個吵鬧?一定是 你要出去了。不如回太太,打發你去吧。”說著,站起來就要走。襲人忙回身攔住,笑道:“往那裏去?”寶玉道:“回太太去。”襲人笑道:“好沒意思!真個的去回,你也不怕臊了?便是他認真的要去,也等把這氣下去了,等無事中說話兒回了太太也不遲。這會子急急的當作一件正經事去回,豈不叫太太犯疑?”寶玉道: “太太必不犯疑,我只明說是他鬧著要去的。”晴雯哭道:“我多早晚鬧著要去了?饒生了氣,還拿話壓派我。只管去回,我一頭碰死了也不出這門兒。”寶玉道: “這也奇了。你又不去,你又鬧些什麼?我經不起這吵,不如去了倒乾淨。”說著一定要去回。襲人見攔不住,只得跪下了。碧痕、秋紋、麝月等衆丫鬟見吵鬧,都鴉雀無聞的在外頭聽消息,這會子聽見襲人跪下央求,便一齊進來都跪下了。寶玉忙把襲人扶起來,嘆了一聲,在牀上坐下,叫衆人起去,向襲人道:“叫我怎麼樣 纔好!這個心使碎了也沒人知道。”說著不覺滴下淚來。襲人見寶玉流下淚來,自己也就哭了。
\end{parag}


\begin{parag}
    晴雯在旁哭著,方欲說話,只見林黛玉進來,便出去了。林黛玉笑道:“大節下怎麼好好的哭起來?難道是爲爭糉子喫爭惱了不成?”寶玉和襲人嗤的一笑。黛玉道:“二哥哥不告訴我,我問你就知道了。”一面說,一面拍著襲人的肩,笑道:“好嫂子,你告訴我。必定是你兩個拌了嘴了。告訴妹妹,替你們和勸和勸。” 襲人推他道:“林姑娘你鬧什麼?我們一個丫頭,姑娘只是混說。”黛玉笑道:“你說你是丫頭,我只拿你當嫂子待。”寶玉道:“你何苦來替他招罵名兒。饒這麼 著,還有人說閒話,還擱的住你來說他。”襲人笑道:“林姑娘,你不知道我的心事,除非一口氣不來死了倒也罷了。”林黛玉笑道:“你死了,別人不知怎麼樣, 我先就哭死了。”寶玉笑道:“你死了,我作和尚去。”襲人笑道:“你老實些罷,何苦還說這些話。”林黛玉將兩個指頭一伸,抿嘴笑道:“作了兩個和尚了。我從今以後都記著你作和尚的遭數兒。”寶玉聽得,知道是他點前兒的話,自己一笑也就罷了。
\end{parag}


\begin{parag}
    一時黛玉去後,就有人說“薛大爺請”,寶玉只得去了。原來是喫酒,不能推辭,只得盡席而散。晚間回來,已帶了幾分酒,踉蹌來至自己院內,只見院中早把乘涼枕榻設下,榻上有個人睡著。寶玉只當是襲人,一面在榻沿上坐下,一面推他,問道:“疼的好些了?”只見那人翻身起來說:“何苦來,又招我!”寶玉一 看,原來不是襲人,卻是晴雯。寶玉將他一拉,拉在身旁坐下,笑道:“你的性子越發慣嬌了。早起就是跌了扇子,我不過說了那兩句,你就說上那些話。說我也罷了,襲人好意來勸,你又括上他,你自己想想,該不該?”晴雯道:“怪熱的,拉拉扯扯作什麼!叫人來看見象什麼!我這身子也不配坐在這裏。”寶玉笑道:“你既知道不配,爲什麼睡著呢?”晴雯沒的話,嗤的又笑了,說:“你不來便使得,你來了就不配了。起來,讓我洗澡去。襲人麝月都洗了澡,我叫了他們來。”寶玉笑道:“我才又吃了好些酒,還得洗一洗。你既沒有洗,拿了水來咱們兩個洗。”晴雯搖手笑道:“罷,罷,我不敢惹爺。還記得碧痕打發你洗澡,足有兩三個時 辰,也不知道作什麼呢。我們也不好進去的。後來洗完了,進去瞧瞧,地下的水淹著牀腿,連席子上都汪著水,也不知是怎麼洗了,笑了幾天。我也沒那工夫收拾, 也不用同我洗去。今兒也涼快,那會子洗了,可以不用再洗。我倒舀一盆水來,你洗洗臉通通頭。纔剛鴛鴦送了好些果子來,都湃在那水晶缸裏呢,叫他們打發你喫。”寶玉笑道:“既這麼著,你也不許洗去,只洗洗手來拿果子來喫罷。”晴雯笑道:“我慌張的很,連扇子還跌折了,那裏還配打發喫果子。倘或再打破了盤 子,還更了不得呢。”寶玉笑道:“你愛打就打,這些東西原不過是借人所用,你愛這樣,我愛那樣,各自性情不同。比如那扇子原是扇的,你要撕著玩也可以使 得,只是不可生氣時拿他出氣。就如杯盤,原是盛東西的,你喜聽那一聲響,就故意的碎了也可以使得,只是別在生氣時拿他出氣。這就是愛物了。”晴雯聽了,笑道:“既這麼說,你就拿了扇子來我撕。我最喜歡撕的。”寶玉聽了,便笑著遞與他。晴雯果然接過來,嗤的一聲,撕了兩半,接著嗤嗤又聽幾聲。寶玉在旁笑著 說:“響的好,再撕響些!”正說著,只見麝月走過來,笑道:“少作些孽罷。”寶玉趕上來,一把將他手裏的扇子也奪了遞與晴雯。晴雯接了,也撕了幾半子,二人都大笑。麝月道:“這是怎麼說,拿我的東西開心兒?”寶玉笑道:“打開扇子匣子你揀去,什麼好東西!”麝月道:“既這麼說,就把匣子搬了出來,讓他盡力 的撕,豈不好?”寶玉笑道:“你就搬去。”麝月道:“我可不造這孽。他也沒折了手,叫他自己搬去。”晴雯笑著,倚在牀上說道:“我也乏了,明兒再撕罷。” 寶玉笑道:“古人云:‘千金難買一笑。’幾把扇子能值幾何!”一面說著,一面叫襲人。襲人才換了衣服走出來,小丫頭佳蕙過來拾去破扇,大家乘涼,不消細說。
\end{parag}


\begin{parag}
    至次日午間,王夫人、薛寶釵、林黛玉衆姊妹正在賈母房內坐著,就有人回:“史大姑娘來了。”一時果見史湘雲帶領衆多丫鬟媳婦走進院來。寶黛玉等忙迎至 階下相見。青年姊妹間經月不見,一旦相逢,其親密自不必細說。一時進入房中,請安問好,都見過了。賈母因說:“天熱,把外頭的衣服脫脫罷。” 史湘雲忙起身寬衣。王夫人因笑道:“也沒見穿上這些作什麼?”史湘雲笑道:“都是二嬸嬸叫穿的,誰願意穿這些。”寶釵一旁笑道:“姨娘不知道,他穿衣裳還 更愛穿別人的衣裳。可記得舊年三四月裏,他在這裏住著,把寶兄弟的袍子穿上,靴子也穿上,額子也勒上,猛一瞧倒象是寶兄弟,就是多兩個墜子。他站在那椅子 後邊,哄的老太太只是叫‘寶玉,你過來,仔細那上頭掛的燈穗子招下灰來迷了眼’。他只是笑,也不過去。後來大家撐不住笑了,老太太才笑了,說:‘倒扮上男人好看了。’”林黛玉道:“這算什麼。惟有前年正月裏接了他來,住了沒兩日就下起雪來,老太太和舅母那日想是才拜了影回來,老太太的一個新新的大紅猩猩氈鬥蓬放在那裏,誰知眼錯不見他就披了,又大又長,他就拿了個汗巾子攔腰繫上,和丫頭們在後院子撲雪人兒去,一跤栽到溝跟前,弄了一身泥水。”說著,大家想 著前情,都笑了。寶釵笑向那周奶媽道:“周媽,你們姑娘還是那麼淘氣不淘氣了?”周奶孃也笑了。迎春笑道:“淘氣也罷了,我就嫌他愛說話。也沒見睡在那裏還是咭咭呱呱,笑一陣,說一陣,也不知那裏來的那些話。”王夫人道:“只怕如今好了。前日有人家來相看,眼見有婆婆家了,還是那們著。”賈母因問:“今兒還是住著,還是家去呢?”周奶孃笑道:“老太太沒有看見衣服都帶了來,可不住兩天?”史湘雲問道:“寶玉哥哥不在家麼?”寶釵笑道:“他再不想著別人,只想寶兄弟,兩個人好憨的。這可見還沒改了淘氣。”賈母道:“如今你們大了,別提小名兒了。”
\end{parag}


\begin{parag}
    剛只說著,只見寶玉來了,笑道:“雲妹妹來了。怎麼前兒打發人接你去,怎麼不來?”王夫人道:“這裏老太太才說這一個,他又來提名道姓的了。”林黛玉 道:“你哥哥得了好東西,等著你呢。”史湘雲道:“什麼好東西?”寶玉笑道:“你信他呢!幾日不見,越發高了。”湘雲笑道:“襲人姐姐好?”寶玉道:“多謝你記掛。”湘雲道:“我給他帶了好東西來了。”說著,拿出手帕子來,挽著一個疙瘩。寶玉道:“什麼好的?你倒不如把前兒送來的那種絳紋石的戒指兒帶兩個 給他。”湘雲笑道:“這是什麼?”說著便打開。衆人看時,果然就是上次送來的那絳紋戒指,一包四個。林黛玉笑道:“你們瞧瞧他這主意。前兒一般的打發人給我們送了來,你就把他的帶來豈不省事?今兒巴巴的自己帶了來,我當又是什麼新奇東西,原來還是他。真真你是糊塗人。”史湘雲笑道:“你才糊塗呢!我把這理說出來,大家評一評誰糊塗。給你們送東西,就是使來的不用說話,拿進來一看,自然就知是送姑娘們的了;若帶他們的東西,這得我先告訴來人,這是那一個丫頭 的,那是那一個丫頭的,那使來的人明白還好,再糊塗些,丫頭的名字他也不記得,混鬧胡說的,反連你們的東西都攪糊塗了。若是打發個女人素日知道的還罷了, 偏生前兒又打發小子來,可怎麼說丫頭們的名字呢?橫豎我來給他們帶來,豈不清白。”說著,把四個戒指放下,說道:“襲人姐姐一個,鴛鴦姐姐一個,金釧兒姐 姐一個,平兒姐姐一個:這倒是四個人的,難道小子們也記得這們清白?”衆人聽了都笑道:“果然明白。”寶玉笑道:“還是這麼會說話,不讓人。”林黛玉聽 了,冷笑道:“他不會說話,他的金麒麟會說話。”一面說著,便起身走了。幸而諸人都不曾聽見,只有薛寶釵抿嘴一笑。寶玉聽見了,倒自己後悔又說錯了話,忽見寶釵一笑,由不得也笑了。寶釵見寶玉笑了,忙起身走開,找了林黛玉去說話。
\end{parag}


\begin{parag}
    賈母向湘雲道:“吃了茶歇一歇,瞧瞧你的嫂子們去。園裏也涼快,同你姐姐們去逛逛。”湘雲答應了,將三個戒指兒包上,歇了一歇,便起身要瞧鳳姐等人 去。衆奶孃丫頭跟著,到了鳳姐那裏,說笑了一回,出來便往大觀園來,見過了李宮裁,少坐片時,便往怡紅院來找襲人。因回頭說道:“你們不必跟著,只管瞧你 們的朋友親戚去,留下翠縷伏侍就是了。”衆人聽了,自去尋姑覓嫂,早剩下湘雲翠縷兩個人。翠縷道:“這荷花怎麼還不開?”史湘雲道:“時候沒到。”翠縷道: “這也和咱們家池子裏的一樣,也是樓子花?”湘雲道:“他們這個還不如咱們的。”翠縷道:“他們那邊有棵石榴,接連四五枝,真是樓子上起樓子,這也難爲他 長。”史湘雲道:“花草也是同人一樣,氣脈充足,長的就好。”翠縷把臉一扭,說道:“我不信這話。若說同人一樣,我怎麼不見頭上又長出一個頭來的人?”湘 雲聽了由不得一笑,說道:“我說你不用說話,你偏好說。這叫人怎麼好答言?天地間都賦陰陽二氣所生,或正或邪,或奇或怪,千變萬化,都是陰陽順逆。多少一 生出來,人罕見的就奇,究竟理還是一樣。”翠縷道:“這麼說起來,從古至今,開天闢地,都是陰陽了?”湘雲笑道:“糊塗東西,越說越放屁。什麼‘都是些陰 陽’,難道還有個陰陽不成!‘陰’‘陽’兩個字還只是一字,陽盡了就成陰,陰盡了就成陽,不是陰盡了又有個陽生出來,陽盡了又有個陰生出來。”翠縷道: “這糊塗死了我!什麼是個陰陽,沒影沒形的。我只問姑娘,這陰陽是怎麼個樣兒?”湘雲道:“陰陽可有什麼樣兒,不過是個氣,器物賦了成形。比如天是陽,地 就是陰;水是陰,火就是陽;日是陽,月就是陰。”翠縷聽了,笑道:“是了,是了,我今兒可明白了。怪道人都管著日頭叫‘太陽’呢,算命的管著月亮叫什麼 ‘太陰星’,就是這個理了。”湘雲笑道:“阿彌陀佛!剛剛的明白了。”翠縷道:“這些大東西有陰陽也罷了,難道那些蚊子、虼蚤、蠓蟲兒、花兒、草兒、瓦片兒、磚頭兒也有陰陽不成?”湘雲道:“怎麼有沒陰陽的呢?比如那一個樹葉兒還分陰陽呢,那邊向上朝陽的便是陽,這邊背陰覆下的便是陰。”翠縷聽 了,點頭笑道:“原來這樣,我可明白了。只是咱們這手裏的扇子,怎麼是陽,怎麼是陰呢?”湘雲道:“這邊正面就是陽,那邊反面就爲陰。”翠縷又點頭笑了, 還要拿幾件東西問,因想不起個什麼來,猛低頭就看見湘雲宮絛上系的金麒麟,便提起來問道:“姑娘,這個難道也有陰陽?”湘雲道:“走獸飛禽,雄爲陽,雌爲陰;牝爲陰,牡爲陽。怎麼沒有呢!”翠縷道:“這是公的,到底是母的呢?”湘雲道:“這連我也不知道。”翠縷道:“這也罷了,怎麼東西都有陰陽,咱們人倒 沒有陰陽呢?”湘雲照臉啐了一口道:“下流東西,好生走罷!越問越問出好的來了!” 翠縷笑道:“這有什麼不告訴我的呢?我也知道了,不用難我。”湘雲笑道:“你知道什麼?”翠縷道:“姑娘是陽,我就是陰。”說著,湘雲拿手帕子握著嘴,呵 呵的笑起來。翠縷道:“說是了,就笑的這樣了。”湘雲道:“很是,很是。”翠縷道:“人規矩主子爲陽,奴才爲陰。我連這個大道理也不懂得?”湘雲笑道: “你很懂得。”
\end{parag}


\begin{parag}
    一面說,一面走,剛到薔薇架下,湘雲道:“你瞧那是誰掉的首飾,金晃晃在那裏。”翠縷聽了,忙趕上拾在手裏攥著,笑道:“可分出陰陽來了。”說著,先 拿史湘雲的麒麟瞧。湘雲要他揀的瞧,翠縷只管不放手,笑道:“是件寶貝,姑娘瞧不得。這是從那裏來的?好奇怪!我從來在這裏沒見有人有這個。”湘雲笑道: “拿來我看。”翠縷將手一撒,笑道:“請看。”湘雲舉目一驗,卻是文彩輝煌的一個金麒麟,比自己佩的又大又有文彩。湘雲伸手擎在掌上,只是默默不語,正自 出神,忽見寶玉從那邊來了,笑問道:“你兩個在這日頭底下作什麼呢?怎麼不找襲人去?”湘雲連忙將那麒麟藏起道:“正要去呢。咱們一處走。”說著,大家進入怡紅院來。襲人正在階下倚檻追風,忽見湘雲來了,連忙迎下來,攜手笑說一向久別情況。一時進來歸坐,寶玉因笑道:“你該早來,我得了一件好東西,專等你呢。”說著,一面在身上摸掏,掏了半天,呵呀了一聲,便問襲人“那個東西你收起來了麼?”襲人道:“什麼東西?”寶玉道:“前兒得的麒麟。”襲人道:“你 天天帶在身上的,怎麼問我?”寶玉聽了,將手一拍說道:“這可丟了,往那裏找去!”就要起身自己尋去。湘雲聽了,方知是他遺落的,便笑問道:“你幾時又有了麒麟了?”寶玉道:“前兒好容易得的呢,不知多早晚丟了,我也糊塗了。”湘雲笑道:“幸而是頑的東西,還是這麼慌張。”說著,將手一撒,“你瞧瞧,是這個不是?”寶玉一見由不得歡喜非常,因說道……不知是如何,且聽下回分解。
\end{parag}


\begin{parag}
    \begin{note}庚:後數十回若蘭在射圃所佩之麒麟正此麒麟也。提綱伏於此回中,所謂“草蛇灰線,在千里之外”。\end{note}
\end{parag}

