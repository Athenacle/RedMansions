\chap{五十}{芦雪广争联即景诗 暖香坞创制春灯谜}


\begin{parag}
    \begin{note}蒙回前总评:此回著重在宝琴,却出色写湘云。写湘云联句极敏捷聪慧,而宝琴之联句不少于湘云,可知出色写湘云,正所以出色写宝琴。出色写宝琴者,全为与宝玉提亲作引也。金针暗渡,不可不知。\end{note}
\end{parag}


\begin{parag}
    话说薛宝钗道:“到底分个次序,让我写出来。”说著,便令众人拈阄为序。\begin{note}庚双夹:起首恰是李氏。一定要按次序,恰又不按次序,似脱落处而不脱落,文章歧路如此。然后按次各各开出。\end{note}(按:此段批语混入正文。)凤姐儿说道:“既是这样说,我也说一句在上头。”众人都笑说道: “更妙了!”宝钗便将稻香老农之上补了一个“凤”字,李纨又将题目讲与他听。凤姐儿想了半日,笑道:“你们别笑话我。我只有一句粗话,下剩的我就不知道了。”众人都笑道:“越是粗话越好,你说了只管干正事去罢。”凤姐儿笑道:“我想下雪必刮北风。昨夜听见了一夜的北风,我有了一句,就是‘一夜北风紧’,可使得?”众人听了,都相视笑道:“这句虽粗,不见底下的,这正是会作诗的起法。不但好,而且留了多少地步与后人。就是这句为首,稻香老农快写上续下去。”凤姐和李婶平儿又吃了两杯酒,自去了。这里李纨便写了:
\end{parag}


\begin{poem}
    \begin{pl} 一夜北风紧,\end{pl}
\end{poem}


\begin{parag}
    自己联道:
\end{parag}


\begin{poem}
    \begin{pl} 开门雪尚飘。入泥怜洁白,\end{pl}
\end{poem}


\begin{parag}
    香菱道:
\end{parag}


\begin{poem}
    \begin{pl} 匝地惜琼瑶。有意荣枯草,\end{pl}
\end{poem}


\begin{parag}
    探春道:
\end{parag}


\begin{poem}
    \begin{pl} 无心饰萎苕。价高村酿熟,\end{pl}
\end{poem}


\begin{parag}
    李绮道:
\end{parag}


\begin{poem}
    \begin{pl} 年稔府粱饶。葭动灰飞管,\end{pl}
\end{poem}


\begin{parag}
    李纹道:
\end{parag}


\begin{poem}
    \begin{pl} 阳回斗转杓。寒山已失翠,\end{pl}
\end{poem}


\begin{parag}
    岫烟道:
\end{parag}


\begin{poem}
    \begin{pl} 冻浦不闻潮。易挂疏枝柳,\end{pl}
\end{poem}


\begin{parag}
    湘云道:
\end{parag}


\begin{poem}
    \begin{pl} 难堆破叶蕉。麝煤融宝鼎,\end{pl}
\end{poem}


\begin{parag}
    宝琴道:
\end{parag}


\begin{poem}
    \begin{pl} 绮袖笼金貂。光夺窗前镜,\end{pl}
\end{poem}


\begin{parag}
    黛玉道:
\end{parag}


\begin{poem}
    \begin{pl} 香粘壁上椒。斜风仍故故,\end{pl}
\end{poem}


\begin{parag}
    宝玉道:
\end{parag}


\begin{poem}
    \begin{pl} 清梦转聊聊。何处梅花笛?\end{pl}
\end{poem}


\begin{parag}
    宝钗道:
\end{parag}


\begin{poem}
    \begin{pl} 谁家碧玉箫?鳌愁坤轴陷,\end{pl}
\end{poem}


\begin{parag}
    李纨笑道:“我替你们看热酒去罢。”
\end{parag}


\begin{parag}
    宝钗命宝琴续联,只见湘云站起来道:
\end{parag}


\begin{poem}
    \begin{pl} 龙斗阵云销。野岸回孤棹,\end{pl}
\end{poem}


\begin{parag}
    宝琴也站起道:
\end{parag}


\begin{poem}
    \begin{pl} 吟鞭指灞桥。赐裘怜抚戍,\end{pl}
\end{poem}


\begin{parag}
    湘云那里肯让人,且别人也不如他敏捷,都看他扬眉挺身的说道:
\end{parag}


\begin{poem}
    \begin{pl} 加絮念征徭。拗垤审夷险,\end{pl}
\end{poem}


\begin{parag}
    宝钗连声赞好,也便联道:
\end{parag}


\begin{poem}
    \begin{pl} 枝柯怕动摇。皑皑轻趁步,\end{pl}
\end{poem}


\begin{parag}
    黛玉忙联道:
\end{parag}


\begin{poem}
    \begin{pl} 剪剪舞随腰。煮芋成新赏,\end{pl}
\end{poem}


\begin{parag}
    一面说,一面推宝玉,命他联。宝玉正看宝钗、宝琴、黛玉三人共战湘云,十分有趣,那里还顾得联诗,今见黛玉推他,方联道:
\end{parag}


\begin{poem}
    \begin{pl} 撒盐是旧谣。苇蓑犹泊钓,\end{pl}
\end{poem}


\begin{parag}
    湘云笑道:“你快下去,你不中用,倒耽搁了我。”一面只听宝琴联道:
\end{parag}


\begin{poem}
    \begin{pl} 林斧不闻樵。伏象千峰凸,\end{pl}
\end{poem}


\begin{parag}
    湘云忙联道:
\end{parag}


\begin{poem}
    \begin{pl} 盘蛇一径遥。花缘经冷结,\end{pl}
\end{poem}


\begin{parag}
    宝钗与众人又忙赞好。探春又联道:
\end{parag}


\begin{poem}
    \begin{pl} 色岂畏霜凋。深院惊寒雀,\end{pl}
\end{poem}


\begin{parag}
    湘云正渴了,忙忙的吃茶,已被岫烟道:
\end{parag}


\begin{poem}
    \begin{pl} 空山泣老鸮。阶墀随上下,\end{pl}
\end{poem}


\begin{parag}
    湘云忙丢了茶杯,忙联道:
\end{parag}


\begin{poem}
    \begin{pl} 池水任浮漂。照耀临清晓,\end{pl}
\end{poem}


\begin{parag}
    黛玉联道:
\end{parag}


\begin{poem}
    \begin{pl} 缤纷入永宵。诚忘三尺冷,\end{pl}
\end{poem}


\begin{parag}
    湘云忙笑联道:
\end{parag}


\begin{poem}
    \begin{pl} 瑞释九重焦。僵卧谁相问,\end{pl}
\end{poem}


\begin{parag}
    宝琴也忙笑联道:
\end{parag}


\begin{poem}
    \begin{pl} 狂游客喜招。天机断缟带,\end{pl}
\end{poem}


\begin{parag}
    湘云又忙道:
\end{parag}


\begin{poem}
    \begin{pl} 海市失鲛绡。\end{pl}
\end{poem}


\begin{parag}
    林黛玉不容他出,接著便道:
\end{parag}


\begin{poem}
    \begin{pl} 寂寞对台榭,\end{pl}
\end{poem}


\begin{parag}
    湘云忙联道:
\end{parag}


\begin{poem}
    \begin{pl} 清贫怀箪瓢。\end{pl}
\end{poem}


\begin{parag}
    宝琴也不容情,也忙道:
\end{parag}


\begin{poem}
    \begin{pl} 烹茶冰渐沸,\end{pl}
\end{poem}


\begin{parag}
    湘云见这般,自为得趣,又是笑,又忙联道:
\end{parag}


\begin{poem}
    \begin{pl} 煮酒叶难烧。\end{pl}
\end{poem}


\begin{parag}
    黛玉也笑道:
\end{parag}


\begin{poem}
    \begin{pl} 没帚山僧扫,\end{pl}
\end{poem}


\begin{parag}
    宝琴也笑道:
\end{parag}


\begin{poem}
    \begin{pl} 埋琴稚子挑。\end{pl}
\end{poem}


\begin{parag}
    湘云笑的弯了腰,忙念了一句,众人问:“到底说的什么?”湘云喊道:
\end{parag}


\begin{poem}
    \begin{pl} 石楼闲睡鹤,\end{pl}
\end{poem}


\begin{parag}
    黛玉笑的握著胸口,高声嚷道:
\end{parag}


\begin{poem}
    \begin{pl} 锦罽暖亲猫。\end{pl}
\end{poem}


\begin{parag}
    宝琴也忙笑道:
\end{parag}


\begin{poem}
    \begin{pl} 月窟翻银浪,\end{pl}
\end{poem}


\begin{parag}
    湘云忙联道:
\end{parag}


\begin{poem}
    \begin{pl} 霞城隐赤标。\end{pl}
\end{poem}


\begin{parag}
    黛玉忙笑道:
\end{parag}


\begin{poem}
    \begin{pl} 沁梅香可嚼,\end{pl}
\end{poem}


\begin{parag}
    宝钗笑称好,也忙联道:
\end{parag}


\begin{poem}
    \begin{pl} 淋竹醉堪调。\end{pl}
\end{poem}


\begin{parag}
    宝琴也忙道:
\end{parag}


\begin{poem}
    \begin{pl} 或湿鸳鸯带,\end{pl}
\end{poem}


\begin{parag}
    湘云忙联道:
\end{parag}


\begin{poem}
    \begin{pl} 时凝翡翠翘。\end{pl}
\end{poem}


\begin{parag}
    黛玉又忙道:
\end{parag}


\begin{poem}
    \begin{pl} 无风仍脉脉,\end{pl}
\end{poem}


\begin{parag}
    宝琴又忙笑联道:
\end{parag}


\begin{poem}
    \begin{pl} 不雨亦潇潇。\end{pl}
\end{poem}


\begin{parag}
    湘云伏著已笑软了。众人看他三人对抢,也都不顾作诗,看著也只是笑。黛玉还推他往下联,又道:“你也有才尽之时。我听听还有什么舌根嚼了!”湘云只伏在宝钗怀里,笑个不住。宝钗推他起来道:“你有本事,把‘二萧’的韵全用完了,我才伏你。”湘云起身笑道:“我也不是作诗,竟是抢命呢。”\begin{note}庚:的是湘云。写海棠是一样笔墨,如今联句又是一样写法。\end{note}众人笑道:“倒是你说罢。”探春早已料定没有自己联的了,便早写出来,因说:“还没收住呢。”李纨听了,接过来便联了一句道:
\end{parag}


\begin{poem}
    \begin{pl} 欲志今朝乐,\end{pl}
\end{poem}


\begin{parag}
    李绮收了一句道:
\end{parag}


\begin{poem}
    \begin{pl} 凭诗祝舜尧。\end{pl}
\end{poem}


\begin{parag}
    李纨道:“够了,够了。虽没作完了韵,剩的字若生扭用了,倒不好了。”说著,大家来细细评论一回,独湘云的多,都笑道:“这都是那块鹿肉的功劳。”
\end{parag}


\begin{parag}
    李纨笑道:“逐句评去都还一气,只是宝玉又落了第了。”宝玉笑道:“我原不会联句,只好担待我罢。”李纨笑道:“也没有社社担待你的。又说韵险了,又整误了,又不会联句了,今日必罚你。我才看见栊翠庵的红梅有趣,我要折一枝来插瓶。可厌妙玉为人,我不理他。如今罚你去取一枝来。”众人都道这罚的又雅又有趣。宝玉也乐为,答应著就要走。湘云黛玉一齐说道:“外头冷得很,你且吃杯热酒再去。”湘云早执起壶来,黛玉递了一个大杯,满斟了一杯。湘云笑道:“你吃了我们的酒,你要取不来,加倍罚你。”宝玉忙吃一杯,冒雪而去。李纨命人好好跟著。黛玉忙拦说:“不必,有了人反不得了。”李纨点头说:“是。”一面命丫鬟将一个美女耸肩瓶拿来,贮了水准备插梅,因又笑道:“回来该咏红梅了。”湘云忙道:“我先作一首。”宝钗忙道:“今日断乎不容你再作了。你都抢了去,别人都闲著,也没趣。回来还罚宝玉,他说不会联句,如今就叫他自己作去。”\begin{note}庚双夹:想此刻宝玉已到庵中矣。\end{note}黛玉笑道:“这话很是。我还有个主意,方才联句不够,莫若拣著联的少的人作红梅。”宝钗笑道:“这话是极。方才邢李三位屈才,且又是客。琴儿和颦儿云儿三个人也抢了许多,我们一概都别作,只让他三个作才是。”李纨因说:“绮儿也不大会作,还是让琴妹妹作罢。”宝钗只得依允,\begin{note}庚双夹:想此刻二玉已会,不知肯见赐否。\end{note}又道:“就用 ‘红梅花’三个字作韵,每人一首七律。邢大妹妹作‘红’字,你们李大妹妹作‘梅’字,琴儿作‘花’字。”李纨道:“饶过宝玉去,我不服。”湘云忙道:“有个好题目命他作。”众人问何题目?湘云道:“命他就作‘访妙玉乞红梅’,岂不有趣?”众人听了,都说有趣。
\end{parag}


\begin{parag}
    一语未了,只见宝玉笑欣欣掮了一枝红梅进来。众丫鬟忙已接过,插入瓶内。众人都笑称谢。宝玉笑道:“你们如今赏罢,也不知费了我多少精神呢。”说著,探春早又递过一钟暖酒来,众丫鬟走上来接了蓑笠掸雪。各人房中丫鬟都添送衣服来,\begin{note}庚双夹:冬日午后景况。\end{note}袭人也遣人送了半旧的狐腋褂来。李纨命人将那蒸的大芋头盛了一盘,又将朱橘、黄橙、橄榄等物盛了两盘,命人带与袭人去。湘云且告诉宝玉方才的诗题,又催宝玉快作。宝玉道:“姐姐妹妹们,让我自己用韵罢,别限韵了。”众人都说:“随你作去罢。”
\end{parag}


\begin{parag}
    一面说一面大家看梅花。原来这枝梅花只有二尺来高,旁有一横枝纵横而出,约有五六尺长,其间小枝分歧,或如蟠螭,或如僵蚓,或孤削如笔,或密聚如林,花 码 脂,香欺兰蕙,\begin{note}庚双夹:一篇《红 犯场贰\end{note}各各称赏。谁知邢岫烟、李纹、薛宝琴三人都已吟成,各自写了出来。众人便依“红梅花”三字之序看去,写道是:
\end{parag}


\begin{poem}
    \begin{pl}咏红梅花\authorr{得“红”字 邢岫烟}\end{pl}

    \begin{pl}桃未芳菲杏未红,冲寒先已笑东风。\end{pl}

    \begin{pl}魂飞庾岭春难辨,霞隔罗浮梦未通。\end{pl}

    \begin{pl}绿萼添妆融宝炬,缟仙扶醉跨残虹。\end{pl}

    \begin{pl}看来岂是寻常色,浓淡由他冰雪中。\end{pl}
    \emptypl

    \begin{pl}咏红梅花\authorr{得“梅”字 李纹}\end{pl}

    \begin{pl}白梅懒赋赋红梅,逞艳先迎醉眼开。\end{pl}

    \begin{pl}冻脸有痕皆是血,酸心无恨亦成灰。\end{pl}

    \begin{pl}误吞丹药移真骨,偷下瑶池脱旧胎。\end{pl}

    \begin{pl}江北江南春灿烂,寄言蜂蝶漫疑猜。\end{pl}
    \emptypl

    \begin{pl}咏红梅花\authorr{得“花”字 薛宝琴}\end{pl}

    \begin{pl}疏是枝条艳是花,春妆儿女竞奢华。\end{pl}

    \begin{pl}闲庭曲槛无余雪,流水空山有落霞。\end{pl}

    \begin{pl}幽梦冷随红袖笛,游仙香泛绛河槎。\end{pl}

    \begin{pl}前身定是瑶台种,无复相疑色相差。\end{pl}
\end{poem}


\begin{parag}
    众人看了,都笑称赞了一番,又指末一首说更好。宝玉见宝琴年纪最小,才又敏捷,深为奇异。黛玉湘云二人斟了一小杯酒,齐贺宝琴。宝钗笑道:“三首各有各好。你们两个天天捉弄厌了我,如今捉弄他来了。”李纨又问宝玉:“你可有了?”宝玉忙道:“我倒有了,才一看见那三首,又吓忘了,等我再想。”湘云听了,便拿了一支铜火箸击著手炉,笑道:“我击鼓了,若鼓绝不成,又要罚的。”宝玉笑道:“我已有了。”黛玉提起笔来,说道:“你念,我写。”湘云便击了一下笑道:“一鼓绝。”宝玉笑道:“有了,你写吧。”众人听他念道:
\end{parag}

\begin{poem}
    \begin{pl}
        “酒未开樽句未裁”,
    \end{pl}
\end{poem}


\begin{parag}
    黛玉写了,摇头笑道:“起的平平。”湘云又道“快著!”宝玉笑道:
\end{parag}


\begin{poem}
    \begin{pl} 寻春问腊到蓬莱。\end{pl}
\end{poem}

\begin{parag}
    黛玉湘云都点头笑道:“有些意思了。”宝玉又道:
\end{parag}


\begin{poem}
    \begin{pl}
        不求大士瓶中露,为乞嫦娥槛外梅。
    \end{pl}
\end{poem}


\begin{parag}
    黛玉写了,又摇头道:“凑巧而已。”湘云忙催二鼓,宝玉又笑道:
\end{parag}


\begin{poem}
    \begin{pl}入世冷挑红雪去,离尘香割紫云来。槎枒谁惜诗肩瘦,衣上犹沾佛院苔。\end{pl}
\end{poem}


\begin{parag}
    黛玉写毕,湘云大家才评论时,又见几个丫鬟跑进来道:“老太太来了。”众人忙迎出来。大家又笑道:“怎么这等高兴!”说著,远远见贾母围了大斗篷,带著灰鼠暖兜,坐著小竹轿,打著青绸油伞,鸳鸯琥珀等五六个丫鬟,每人都是打著伞,拥轿而来。李纨等忙往上迎,贾母命人止住说:“只在那里就是了。”来至跟前,贾母笑道:“我瞒著你太太和凤丫头来了。大雪地下坐著这个无妨,没的叫他们来跴雪。”众人忙一面上前接斗篷,搀扶著,一面答应著。贾母来至室中,先笑道:“好俊梅花!你们也会乐,我来著了。”说著,李纨早命拿了一个大狼皮褥来铺在当中。贾母坐了,因笑道:“你们只管顽笑吃喝。我因为天短了,不敢睡中觉,抹了一回牌,想起你们来了,我也来凑个趣儿。”李纨早又捧过手炉来,探春另拿了一副杯箸来,亲自斟了暖酒,奉与贾母。贾母便饮了一口,问那个盘子里是什么东西。众人忙捧了过来,回说是糟鹌鹑。贾母道:“这倒罢了,撕一两点腿子来。”李纨忙答应了,要水洗手,亲自来撕。贾母又道:“你们仍旧坐下说笑我听。”又命李纨:“你也坐下,就如同我没来的一样才好,不然我就去了。”众人听了,方依次坐下,这李纨便挪到尽下边。贾母因问作何事了,众人便说作诗。贾母道:“有作诗的,不如作些灯谜,大家正月里好顽的。”众人答应了。说笑了一回,贾母便说:“这里潮湿,你们别久坐,仔细受了潮湿。”因说:“你四妹妹那里暖和,我们到那里瞧瞧他的画儿,赶年可有了。”众人笑道:“那里能年下就有了?只怕明年端阳有了。”贾母道:“这还了得!他竟比盖这园子还费工夫了。”
\end{parag}


\begin{parag}
    说著,仍坐了竹轿,大家围随,过了藕香榭,穿入一条夹道,东西两边皆有过街门,门楼上里外皆嵌著石头匾,如今进的是西门,向外的匾上凿著“穿云”二字,向里的凿著“度月”两字。来至当中,进了向南的正门,贾母下了轿,惜春已接了出来。从里边游廊过去,便是惜春卧房,门斗上有“暖香坞”三个字。\begin{note}庚双夹:看他又写出一处,从起至末一笔一部之文也有,千万笔成一部之文也有,一二笔成一部之文也有。如“试才”一回起若都说完,以后则索然无味,故留此几处以为后文之点染也。此方活泼不板,耳目屡新。\end{note}早有几个人打起猩红毡帘,已觉温香拂脸。\begin{note}庚双夹:各处皆如此,非独因“暖香”二字方有此景。戏注于此,以博一笑耳。\end{note}大家进入房中,贾母并不归坐,只问画在那里。惜春因笑回:“天气寒冷了,胶性皆凝涩不润,画了恐不好看,故此收起来。”贾母笑道: “我年下就要的。你别托懒儿,快拿出来给我快画。”一语未了,忽见凤姐儿披著紫羯褂,笑嘻嘻的来了,口内说道:“老祖宗今儿也不告诉人,私自就来了,要我好找。”贾母见他来了,心中自是喜悦,便道:“我怕你们冷著了,所以不许人告诉你们去。你真是个鬼灵精儿,到底找了我来。以理,孝敬也不在这上头。”凤姐儿笑道:“我那里是孝敬的心找了来?我因为到了老祖宗那里,鸦没雀静的,\begin{note}庚双夹:这四个字俗语中常闻,但不能落纸笔耳。便欲写时,究竟不知系何四字,今如此写来,真是不可移易。\end{note}问小丫头子们,他又不肯说,叫我找到园里来。我正疑惑,忽然来了两三个姑子,我心里才明白。我想姑子必是来送年疏,或要年例香例银子,老祖宗年下的事也多,一定是躲债来了。我赶忙问了那姑子,果然不错。我连忙把年例给了他们去了。如今来回老祖宗,债主已去,不用躲著了。已预备下希嫩的野鸡,请用晚饭去,再迟一回就老了。”他一行说,众人一行笑。
\end{parag}


\begin{parag}
    凤姐儿也不等贾母说话,便命人抬过轿子来。贾母笑著,搀了凤姐的手,仍旧上轿,带著众人,说笑出了夹道东门。一看四面粉妆银砌,忽见宝琴披著凫靥裘站在山坡上遥等,身后一个丫鬟抱著一瓶红梅。众人都笑道:“少了两个人,他却在这里等著,也弄梅花去了。”贾母喜的忙笑道:“你们瞧,这山坡上配上他的这个人品,又是这件衣裳,后头又是这梅花,象个什么?”众人都笑道:“就象老太太屋里挂的仇十洲画的《双艳图》。”贾母摇头笑道:“那画的那里有这件衣裳?人也不能这样好!”一语未了,只见宝琴背后转出一个披大红猩毡的人来。贾母道:“那又是那个女孩儿?”众人笑道:“我们都在这里,那是宝玉。”贾母笑道: “我的眼越发花了。”说话之间,来至跟前,可不是宝玉和宝琴。宝玉笑向宝钗黛玉等道:“我才又到了栊翠庵。妙玉每人送你们一枝梅花,我已经打发人送去了。”众人都笑说:“多谢你费心。”
\end{parag}


\begin{parag}
    说话之间,已出了园门,来至贾母房中。吃毕饭大家又说笑了一回。忽见薛姨妈也来了,说:“好大雪,一日也没过来望候老太太。今日老太太倒不高兴?正该赏雪才是。”贾母笑道:“何曾不高兴!我找了他们姊妹们去顽了一会子。”薛姨妈笑道:“昨日晚上,我原想著今日要和我们姨太太借一日园子,摆阶来志,请老太太赏雪的,又见老太太安息的早。我闻得女儿说,老太太心下不大爽,因此今日也没敢惊动。早知如此,我正该请。”贾母笑道:“这才是十月里头场雪,往后下雪的日子多呢,再破费不迟。”薛姨妈笑道:“果然如此,算我的孝心虔了。”凤姐儿笑道:“姨妈仔细忘了,如今先秤五十两银子来,交给我收著,一下雪,我就预备下酒,姨妈也不用操心,也不得忘了。”贾母笑道:“既这么说,姨太太给他五十两银子收著,我和他每人分二十五两,到下雪的日子,我装心里不快,混过去了,姨太太更不用操心,我和凤丫头倒得了实惠。”凤姐将手一拍,笑道:“妙极了,这和我的主意一样。”众人都笑了。贾母笑道:“呸!没脸的,就顺著竿子爬上来了!你不该说姨太太是客,在咱们家受屈,我们该请姨太太才是,那里有破费姨太太的理!不这样说呢,还有脸先要五十两银子,真不害臊!”凤姐儿笑道:“我们老祖宗最是有眼色的,试一试,姨妈若松呢,拿出五十两来,就和我分。这会子估量著不中用了,翻过来拿我做法子,说出这些大方话来。如今我也不和姨妈要银子,竟替姨妈出银子治了酒,请老祖宗吃了,我另外再封五十两银子孝敬老祖宗,算是罚我个包揽闲事。这可好不好?”话未说完,众人已笑倒在炕上。
\end{parag}


\begin{parag}
    贾母因又说及宝琴雪下折梅比画儿上还好,因又细问他的年庚八字并家内景况。薛姨妈度其意思,大约是要与宝玉求配。薛姨妈心中固也遂意,只是已许过梅家了,因贾母尚未明说,自己也不好拟定,遂半吐半露告诉贾母道:“可惜这孩子没福,前年他父亲就没了。他从小儿见的世面倒多,跟他父母四山五岳都走遍了。他父亲是好乐的,各处因有买卖,带著家眷,这一省逛一年,明年又往那一省逛半年,所以天下十停走了有五六停了。那年在这里,把他许了梅翰林的儿子,偏第二年他父亲就辞世了,他母亲又是痰症。”凤姐也不等说完,便嗐声跺脚的说:“偏不巧,我正要作个媒呢,又已经许了人家。”贾母笑道:“你要给谁说媒?”凤姐儿说道:“老祖宗别管,我心里看准了他们两个是一对。如今已许了人,说也无益,不如不说罢了。”贾母也知凤姐儿之意,听见已有了人家,也就不提了。大家又闲话了一会方散。一宿无话。
\end{parag}


\begin{parag}
    次日雪晴。饭后,贾母又亲嘱惜春:“不管冷暖,你只画去,赶到年下,十分不能便罢了。第一要紧把昨日琴儿和丫头梅花,照模照样,一笔别错,快快添上。”惜春听了虽是为难,只得应了。一时众人都来看他如何画,惜春只是出神。李纨因笑向众人道:“让他自己想去,咱们且说话儿。昨儿老太太只叫作灯谜,回家和绮儿纹儿睡不著,我就编了两个‘四书’的。他两个每人也编了两个。”众人听了,都笑道:“这倒该作的。先说了,我们猜猜。”李纨笑道:“‘观音未有世家传’,打《四书》一句。”湘云接著就说“在止于至善。”宝钗笑道:“你也想一想‘世家传’三个字的意思再猜。”李纨笑道:“再想。”黛玉笑道:“哦,是了。是‘虽善无征’。”众人都笑道:“这句是了。”李纨又道:“一池青草草何名。”湘云忙道:“这一定是‘蒲芦也’。再不是不成?”李纨笑道:“这难为你猜。纹儿的是‘水向石边流出冷’,打一古人名。”探春笑问道:“可是山涛?”李纹笑道:“是。”李纨又道:“绮儿的是个 ‘萤’字,打一个字。”众人猜了半日,宝琴笑道:“这个意思却深,不知可是花草的‘花’字?”李绮笑道:“恰是了。”众人道:“萤与花何干?”黛玉笑道: “妙得很!萤可不是草化的?”众人会意,都笑了说;“好!”宝钗道:“这些虽好,不合老太太的意思,不如作些浅近的物儿,大家雅俗共赏才好。”众人都道: “也要作些浅近的俗物才是。”湘云笑道:“我编了一支《点绛唇》,恰是俗物,你们猜猜。”说著便念道:
\end{parag}


\begin{poem}
    \begin{pl}溪壑分离,红尘游戏,真何趣?名利犹虚,后事终难继。\end{pl}
\end{poem}


\begin{parag}
    众人不解,想了半日,也有猜是和尚的,也有猜是道士的,也有猜是偶戏人的。宝玉笑了半日,道:“都不是,我猜著了,一定是耍的猴儿。”湘云笑道: “正是这个了。”众人道:“前头都好,末后一句怎么解?”湘云道:“那一个耍的猴子不是剁了尾巴去的?”众人听了,都笑起来,说:“他编个谜儿也是刁钻古怪的。”李纨道:“昨日姨妈说,琴妹妹见的世面多,走的道路也多,你正该编谜儿,正用著了。你的诗且又好,何不编几个我们猜一猜?”宝琴听了,点头含笑,自去寻思。宝钗也有了一个,念道:
\end{parag}


\begin{poem}
    \begin{pl}镂檀锲梓一层层,岂系良工堆砌成?\end{pl}

    \begin{pl}虽是半天风雨过,何曾闻得梵铃声!\end{pl}

\end{poem}


\begin{parag}
    打一物。
\end{parag}


\begin{parag}
    众人猜时,宝玉也有了一个,念道:
\end{parag}


\begin{poem}
    \begin{pl}天上人间两渺茫,琅玕节过谨隄防。\end{pl}

    \begin{pl}鸾音鹤信须凝睇,好把唏嘘答上苍。\end{pl}

\end{poem}


\begin{parag}
    黛玉也有了一个,念道是:
\end{parag}


\begin{poem}
    \begin{pl}騄駬何劳缚紫绳?驰城逐堑势狰狞。\end{pl}

    \begin{pl}主人指示风雷动,鳌背三山独立名。\end{pl}

\end{poem}


\begin{parag}
    探春也有了一个,方欲念时,宝琴走过来笑道:“我从小儿所走的地方的古迹不少,我今拣了十个地方的古迹,作了十首怀古的诗。诗虽粗鄙,却怀往事,又暗隐俗物十件,姐姐们请猜一猜。”众人听了,都说:“这倒巧,何不写出来大家一看?”要知端的
\end{parag}


\begin{parag}
    \begin{note}蒙回末总:诗词之峭丽、灯谜之隐秀不待言,须看他极整齐、极参差,愈忙迫愈安闲,一波一折路转峰回,一落一起山断云连,各人居度各人情性都现。至李纨主坛,而起句却在凤姐,李纨主坛,而结句却在最少之李绮,另是一样弄奇。\end{note}
\end{parag}


\begin{parag}
    \begin{note}蒙回末总:最爱他中幅惜春作画一段,似与本文无涉,而前后文之景色人物莫不筋动脉摇,而前后文之起伏照应莫不穿插映带。文字之奇难以言状。\end{note}
\end{parag}

