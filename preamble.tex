\chapter*{戚蓼生序}
\addcontentsline{toc}{chapter}{戚蓼生序}

\begin{qute2sp}

    \begin{parag}
        \large
        吾闻绛树两歌,一声在喉,一声在鼻;黄华二牍,左腕能楷,右腕能草。神乎技也,吾未之见也。今则两歌而不分乎喉鼻,二牍而无区乎左右,一声也而两歌,一手也而二牍,此万万不能有之事,不可得之奇,而竟得之《石头记》一书。嘻!异矣。夫敷华掞藻、立意遣词无一落前人窠臼,此固有目共赏,姑不具论;第观其蕴于心而抒于手也,注彼而写此,目送而手挥,似谲而正,似则而淫,如春秋之有微词、史家之多曲笔。试一一读而绎之:写闺房则极其雍肃也,而艶冶已满纸矣;状阀阅则极其丰整也,而式微已盈睫矣;写宝玉之淫而痴也,而多情善悟,不减历下琅琊;写黛玉之妒而尖也,而笃爱深怜,不啻桑娥石女。他如摹绘玉钗金屋,刻画芗泽罗襦,靡靡焉几令读者心荡神怡矣,而欲求其一字一句之粗鄙猥亵,不可得也。盖声止一声,手只一手,而淫佚贞静,悲戚欢愉,不啻双管之齐下也。噫!异矣。其殆稗官野史中之盲左、腐迁乎?然吾谓作者有两意,读者当具一心。譬之绘事,石有三面,佳处不过一峰;路看两蹊,幽处不逾一树。必得是意,以读是书,乃能得作者微旨。如捉水月,只挹清辉;如雨天花,但闻香气,庶得此书弦外音乎?乃或者以未窥全豹为恨,不知盛衰本是回环,万缘无非幻泡,作者慧眼婆心,正不必再作转语,而千万领悟,便具无数慈航矣。彼沾沾焉刻楮叶以求之者,其与开卷而寤者几希!
    \end{parag}

\end{qute2sp}

