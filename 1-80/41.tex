\chap{四十一}{賈寶玉品茶攏翠庵 劉姥姥醉臥怡紅院}

\begin{parag}
    \begin{note}蒙回前總:任乎牛馬從來樂,隨分清高方可安。自古世情難意擬,淡妝濃抹有千般。立松軒\end{note}
\end{parag}


\begin{parag}
    話說劉姥姥兩隻手比著說道:“花兒落了結個大倭瓜。”衆人聽了鬨堂大笑起來。於是喫過門杯,因又逗趣笑道:“實告訴說罷,我的手腳子粗笨,又喝了酒,仔細失手打了這瓷杯。有木頭的杯取個子來,我便失了手,掉了地下也無礙。”衆人聽了,又笑起來。鳳姐兒聽如此說,便忙笑道:“果真要木頭的,我就取了來。可有一句話先說下:這木頭的可比不得瓷的,他都是一套,定要喫遍一套方使得。”劉姥姥聽了心下敁敠道:“我方纔不過是趣話取笑兒,誰知他果真竟有。我時常在村莊鄉紳大家也赴過席,金盃銀盃倒都也見過,從來沒見有木頭杯之說。哦,是了,想必是小孩子們使的木碗兒,不過誆我多喝兩碗。別管他,橫豎這酒蜜水兒似的,多喝點子也無妨。”\begin{note}庚雙夾:爲登廁伏脈。\end{note}想畢,便說:“取來再商量。”鳳姐乃命豐兒:“到前面裏間屋,書架子上有十個竹根套杯取來。”豐兒聽了答應,纔要去,鴛鴦笑道:“我知道你這十個杯還小。況且你才說是木頭的,這會子又拿了竹根子的來,倒不好看。不如把我們那裏的黃楊根整摳的十個大套杯拿來,灌他十下子。”鳳姐兒笑道:“更好了。”鴛鴦果命人取來。劉姥姥一看,又驚又喜:驚的是一連十個挨次大小分下來,那大的足似個小盆子,第十個極小的還有手裏的杯子兩個大;喜的是雕鏤奇絕,一色山水樹木人物,並有草字以及圖印。因忙說道:“拿了那小的來就是了,怎麼這樣多?”鳳姐兒笑道:“這個杯沒有喝一個的理。我們家因沒有這大量的,所以沒人敢使他。姥姥既要,好容易尋了出來,必定要挨次喫一遍才使得。”劉姥姥唬的忙道:“這個不敢。好姑奶奶,饒了我罷。”\begin{note}蒙側:挨逗的苦惱\end{note}賈母、薛姨媽、王夫人知道他上了年紀的人,禁不起,忙笑道:“說是說,笑是笑,不可多吃了,只吃這頭一杯罷。”劉姥姥道: “阿彌陀佛!我還是小杯喫罷。把這大杯收著,我帶了家去慢慢的喫罷。”說的衆人又笑起來。鴛鴦無法,只得命人滿斟了一大杯,劉姥姥兩手捧著喝。賈母薛姨媽都道:“慢些,不要嗆了。”薛姨媽又命鳳姐兒布了菜。鳳姐笑道:“姥姥要喫什麼,說出名兒來,我搛了餵你。”劉姥姥道:“我知什麼名兒,樣樣都是好的。”賈母笑道:“你把茄鮺搛些喂他。” 鳳姐兒聽說,依言搛些茄鮺送入劉姥姥口中,因笑道:“你們天天喫茄子,也嚐嚐我們的茄子弄的可口不可口。”劉姥姥笑道:“別哄我了,茄子跑出這個味兒來了,我們也不用種糧食,只種茄子了。”衆人笑道:“真是茄子,我們再不哄你。”劉姥姥詫異道:“真是茄子?我白吃了半日。姑奶奶再餵我些,這一口細嚼嚼。”鳳姐果又搛了些放入口內。劉姥姥細嚼了半日,笑道:“雖有一點茄子香,只是還不象是茄子。告訴我是個什麼法子弄的,我也弄著喫去。”鳳姐兒笑道: “這也不難。你把纔下來的茄子把皮簽了,只要淨肉,切成碎釘子,用雞油炸了,再用雞脯子肉並香菌、新筍、蘑菇、五香腐乾、各色乾果子,都切成釘子,拿雞湯煨乾,將香油一收,外加糟油一拌,盛在瓷罐子裏封嚴,要喫時拿出來,用炒的雞瓜一拌就是。”劉姥姥聽了,搖頭吐舌說道:“我的佛祖!倒得十來只雞來配他,怪道這個味兒!”一面說笑,一面慢慢的喫完了酒,還只管細玩那杯。鳳姐笑道:“還是不足興,再喫一杯罷!”劉姥姥忙道:“了不得,那就醉死了。我因爲愛這樣範,虧他怎麼作了。”鴛鴦笑道:“酒喫完了,到底這杯子是什麼木的?”劉姥姥笑道:“怨不得姑娘不認得,你們在這金門繡戶的,如何認得木頭!我們成日家和樹林子作街坊,困了枕著他睡,乏了靠著他坐,荒年間餓了還喫他,眼睛裏天天見他,耳朵裏天天聽他,口兒裏天天講他,所以好歹真假,我是認得的。讓我認一認。”\begin{note}蒙側:好充懂的來看。\end{note}一面說,一面細細端詳了半日,道:“你們這樣人家斷沒有那賤東西,那容易得的木頭,你們也不收著了。我掂著這杯體重,斷乎不是楊木,這一定是黃松做的。”衆人聽了,鬨堂大笑起來。
\end{parag}


\begin{parag}
    只見一個婆子走來請問賈母,說:“姑娘們都到了藕香榭,請示下,就演罷還是再等一會子?”賈母忙笑道:“可是倒忘了他們,就叫他們演罷。”那個婆子答應去了。不一時,只聽得簫管悠揚,笙笛併發。正值風清氣爽之時,那樂聲穿林度水而來,自然使人神怡心曠。寶玉先禁不住,拿起壺來斟了一杯,一口飲盡。\begin{note}蒙側:寶玉似曾在座。\end{note}復又斟上,纔要飲,只見王夫人也要飲,命人換暖酒,寶玉連忙將自己的杯捧了過來,送到王夫人口邊,\begin{note}庚雙夾:妙極!忽寫寶玉如此便是天地間母子之至情至性。獻芹之民之意令人鼻酸。\end{note}王夫人便就他手內吃了兩口。一時暖酒來了,寶玉仍歸舊坐,王夫人提了暖壺下席來,衆人皆都出了席,薛姨媽也立起來,賈母忙命李、鳳二人接過壺來:“讓你姨媽坐了,大家才便。”王夫人見如此說,方將壺遞與鳳姐,自己歸坐。賈母笑道:“大家喫上兩杯,今日著實有趣。”說著擎杯讓薛姨媽,又向湘雲寶釵道:“你姐妹兩個也喫一杯。你妹妹雖不大會喫,也別饒他。”說著自己已幹了。湘雲、寶釵、黛玉也都幹了。當下劉姥姥聽見這般音樂,且又有了酒,越發喜的手舞足蹈起來。寶玉因下席過來向黛玉笑道:“你瞧劉姥姥的樣子。”黛玉笑道:“當日聖樂一奏,百獸率舞,如今才一牛耳。”\begin{note}蒙側:隨筆寫來,趣極。\end{note}衆姐妹都笑了。
\end{parag}


\begin{parag}
    須臾樂止,薛姨媽出席笑道:“大家的酒想也都有了,且出去散散再坐罷。”賈母也正要散散,於是大家出席,都隨著賈母遊玩。賈母因要帶著劉姥姥散悶,遂攜了劉姥姥至山前樹下盤桓了半晌,又說與他這是什麼樹,這是什麼石,這是什麼花。劉姥姥一一的領會,又向賈母道:“誰知城裏不但人尊貴,連雀兒也是尊貴的。偏這雀兒到了你們這裏,他也變俊了,也會說話了。”衆人不解,因問什麼雀兒變俊了,會講話。劉姥姥道:“那廊下金架子上站的綠毛紅嘴是鸚哥兒,我是認得的。那籠子裏的黑老鴰子怎麼又長出鳳頭來,也會說話呢。”衆人聽了都笑將起來。
\end{parag}


\begin{parag}
    一時只見丫鬟們來請用點心。賈母道:“吃了兩杯酒,倒也不餓。也罷,就拿了這裏來,大家隨便喫些罷。”丫鬟便去抬了兩張幾來,又端了兩個小捧盒。揭開看時,每個盒內兩樣:這盒內一樣是藕粉桂糖糕,一樣是松穰鵝油卷;那盒內一樣是一寸來大的小餃兒,賈母因問什麼餡兒,婆子們忙回是螃蟹的。賈母聽了,皺眉說:“這油膩膩的,誰喫這個!”那一樣是奶油炸的各色小面果,也不喜歡。因讓薛姨媽喫,薛姨媽只揀了一塊糕;賈母揀了一個卷子,只嚐了一嘗,剩的半個遞與丫鬟了。劉姥姥因見那小面果子都玲瓏剔透,便揀了一朵牡丹花樣的笑道:“我們那裏最巧的姐兒們,也不能鉸出這麼個紙的來。我又愛喫,又捨不得喫,包些家去給他們做花樣子去倒好。”\begin{note}蒙側:世上竟有這樣人。\end{note}衆人都笑了。賈母道:“家去我送你一罈子。你先趁熱喫這個罷。”別人不過揀各人愛喫的一兩點就罷了;劉姥姥原不曾喫過這些東西,且都作的小巧,不顯盤堆的,他和板兒每樣吃了些,就去了半盤子。剩的,鳳姐又命攢了兩盤並一個攢盒,與文官等喫去。忽見奶子抱了大姐兒來,大家哄他頑了一會。那大姐兒因抱著一個大柚子玩的,忽見板兒抱著一個佛手,便也要佛手。\begin{note}庚雙夾:小兒常情遂成千裏伏線。\end{note}丫鬟哄他取去,大姐兒等不得,便哭了。衆人忙把柚子與了板兒,\begin{note}蒙側:伏線千里。\end{note}將板兒的佛手哄過來與他才罷。那板兒因頑了半日佛手,此刻又兩手抓著些果子喫,又忽見這柚子又香又圓,更覺好頑,且當球踢著玩去,也就不要佛手了。\begin{note}庚雙夾:柚子即今香團之屬也,應與緣通。佛手者,正指迷津者也。以小兒之戲暗透前回通部脈絡,隱隱約約,毫無一絲漏泄,豈獨爲劉姥姥之俚言博笑而有此一大回文字哉?\end{note}\begin{note}蒙側:畫工。\end{note}
\end{parag}


\begin{parag}
    當下賈母等喫過茶,又帶了劉姥姥至櫳翠庵來。妙玉忙接了進去。至院中見花木繁盛,賈母笑道:“到底是他們修行的人,沒事常常修理,比別處越發好看。” 一面說,一面便往東禪堂來。妙玉笑往裏讓,賈母道:“我們才都吃了酒肉,你這裏頭有菩薩,衝了罪過。我們這裏坐坐,把你的好茶拿來,我們喫一杯就去了。” 妙玉聽了,忙去烹了茶來。寶玉留神看他是怎麼行事。只見妙玉親自捧了一個海棠花式雕漆填金雲龍獻壽的小茶盤,裏面放一個成窯五彩小蓋鍾,捧與賈母。\begin{note}靖眉:尚記丁巳春日謝園送茶乎?展眼二十年矣。丁丑仲春。畸笏。\end{note}賈母道:“我不喫六安茶。”妙玉笑說:“知道。這是老君眉。”賈母接了,又問是什麼水。妙玉笑回:“是舊年蠲的雨水。”賈母便吃了半盞,便笑著遞與劉姥姥說:“你嚐嚐這個茶。”劉姥姥便一口吃盡,笑道:“好是好,就是淡些,再熬濃些更好了。”賈母衆人都笑起來。然後衆人都是一色官窯脫胎填白蓋碗。
\end{parag}


\begin{parag}
    那妙玉便把寶釵和黛玉的衣襟一拉,二人隨他出去,寶玉悄悄的隨後跟了來。只見妙玉讓他二人在耳房內,寶釵坐在榻上,黛玉便坐在妙玉的蒲團上。妙玉自向風爐上扇滾了水,另泡一壺茶。寶玉便走了進來,笑道:“偏你們喫梯己茶呢。”二人都笑道:“你又趕了來飺茶喫。這裏並沒你的。”妙玉剛要去取杯,只見道婆收了上面的茶盞來。妙玉忙命:“將那成窯的茶杯別收了,擱在外頭去罷。”\begin{note}靖眉:妙玉偏闢處。此所謂過潔世同嫌也。他日瓜州渡口勸懲不哀哉屈從紅顏固能不枯骨各示□。\end{note}\begin{subnote}此批甚不可解。周汝昌校爲:妙玉偏闢處,此所謂「過潔世同嫌」也。他日現渡口,各示勸,紅顏固[不]能不屈從枯骨,[豈]不哀哉。或校爲:妙玉偏闢處,此所謂「過潔世同嫌」也。他日現渡口,紅顏固[?]屈從枯骨,不能各示勸懲,[豈]不哀哉。戴不凡校爲:[乃]妙玉偏僻處,此所謂「過潔世同嫌」也。他日瓜州渡口屈從,各示勸懲,[豈]不哀哉。紅顏固[不]能不[化爲]枯骨[也],[嘆嘆]!或校爲:此妙玉偏僻處,所謂「過潔世同嫌」 也。他日瓜州渡口屈從,[豈]不哀哉。固是勸懲,紅顏能不[爲]枯骨!(以「固是」「各示」爲各本因吳音相混錯抄之一例。)\end{subnote}寶玉會意,知爲劉姥姥吃了,他嫌髒不要了。又見妙玉另拿出兩隻杯來。一個旁邊有一耳,杯上鐫著“□(左分右瓜)瓟斝”三個隸字,後有一行小真字是“晉王愷珍玩”,又有“宋元豐五年四月眉山蘇軾見於祕府”一行小字。妙玉便斟了一斝,遞與寶釵。那一隻形似鉢而小,也有三個垂珠篆字,鐫著“杏犀䀉”。妙玉斟了一䀉與黛玉。仍將前番自己常日喫茶的那隻綠玉斗來斟與寶玉。寶玉笑道:“常言‘世法平等’,他兩個就用那樣古玩奇珍,我就是個俗器了。”妙玉道:“這是俗器?不是我說狂話,只怕你家裏未必找的出這麼一個俗器來呢。”寶玉笑道:“俗說‘隨鄉入鄉’,到了你這裏,自然把那金玉珠寶一概貶爲俗器了。”妙玉聽如此說,十分歡喜,遂又尋出一隻九曲十環一百二十節蟠虯整雕竹根的一個大海出來,笑道:“就剩了這一個,你可喫的了這一海?”寶玉喜的忙道:“喫的了。”妙玉笑道:“你雖喫的了,也沒這些茶糟蹋。\begin{note}庚雙夾:茶下“糟蹋”二字,成窯杯已不屑再要,妙玉真清潔高雅,然亦怪譎孤僻甚矣。實有此等人物,但罕耳。\end{note}豈不聞‘一杯爲品,二杯即是解渴的蠢物,三杯便是飲牛飲騾了’。你喫這一海便成什麼?”說的寶釵、黛玉、寶玉都笑了。妙玉執壺,只向海內斟了約有一杯。寶玉細細吃了,果覺輕浮無比,賞讚不絕。妙玉正色道:“你這遭喫的茶是託他兩個福,獨你來了,我是不給你喫的。”\begin{note}該批:玉兄獨至豈真無茶喫?作書人又弄狡猾,只瞞不過老朽。然不知落筆時作者如何想。丁亥夏。\end{note}寶玉笑道:“我深知道的,我也不領你的情,只謝他二人便是了。”妙玉聽了,方說:“這話明白。”黛玉因問:“這也是舊年的雨水?”妙玉冷笑道:“你這麼個人,竟是大俗人,連水也嘗不出來。這是五年前我在玄墓蟠香寺住著,收的梅花上的雪,共得了那一鬼臉青的花甕一甕,總捨不得喫,埋在地下,\begin{note}蒙側:妙手層層迭起,竟能以他人所畫之天王,作縱(?)神矣。\end{note}今年夏天才開了。我只喫過一回,這是第二回了。你怎麼嘗不出來?隔年蠲的雨水那有這樣輕浮,如何喫得。”黛玉知他天性怪僻,不好多話,亦不好多坐,喫過茶,便約著寶釵走了出來。\begin{note}該批:黛是解事人。\end{note}
\end{parag}


\begin{parag}
    寶玉和妙玉陪笑道:“那茶杯雖然髒了,白撂了豈不可惜?依我說,不如就給那貧婆子罷,他賣了也可以度日。你道可使得。”妙玉聽了,想了一想,點頭說道:“這也罷了。幸而那杯子是我沒喫過的,若我使過,我就砸碎了也不能給他。\begin{note}蒙側:更奇!世上我也見過此等人。\end{note}你要給他,我也不管你,只交給你,快拿了去罷。”寶玉道:“自然如此,你那裏和他說話授受去,越發連你也髒了。\begin{note}蒙側:人若達形,最喜此等言語。\end{note}只交與我就是了。”妙玉便命人拿來遞與寶玉。寶玉接了,又道:“等我們出去了,我叫幾個小幺兒來河裏打幾桶水來洗地如何?”妙玉笑道:“這更好了,只是你囑咐他們,抬了水只擱在山門外頭牆根下,別進門來。”\begin{note}蒙側:偏於無可寫處,深入一層。\end{note}寶玉道:“這是自然的。”說著,便袖著那杯,遞與賈母房中小丫頭拿著,說:“明日劉姥姥家去,給他帶去罷。”交代明白,賈母已經出來要回去。妙玉亦不甚留,送出山門,回身便將門閉了。不在話下。
\end{parag}


\begin{parag}
    且說賈母因覺身上乏倦,便命王夫人和迎春姊妹陪了薛姨媽去喫酒,自己便往稻香村來歇息。鳳姐忙命人將小竹椅抬來,賈母坐上,兩個婆子抬起,鳳姐李紈和衆丫鬟婆子圍隨去了,不在話下。這裏薛姨媽也就辭出。王夫人打發文官等出去,將攢盒散與衆丫鬟們喫去,自己便也乘空歇著,隨便歪在方纔賈母坐的榻上,命一個小丫頭放下簾子來,又命他捶著腿,吩咐他:“老太太那裏有信,你就叫我。”說著也歪著睡著了。
\end{parag}


\begin{parag}
    寶玉湘雲等看著丫鬟們將攢盒擱在山石上,也有坐在山石上的,也有坐在草地下的,也有靠著樹的,也有傍著水的,倒也十分熱鬧。一時又見鴛鴦來了,要帶著劉姥姥各處去逛,\begin{note}蒙側:又另是一番氣象。\end{note}衆人也都趕著取笑。一時來至“省親別墅”的牌坊底下,劉姥姥道:“噯呀!這裏還有個大廟呢。” 說著,便爬下磕頭。衆人笑彎了腰。劉姥姥道:“笑什麼?這牌樓上字我都認得。我們那裏這樣的廟宇最多,都是這樣的牌坊,那字就是廟的名字。”衆人笑道: “你認得這是什麼廟?”劉姥姥便抬頭指那字道:“這不是‘玉皇寶殿’四字?”衆人笑的拍手打腳,還要拿他取笑。劉姥姥覺得腹內一陣亂響,忙的拉著一個小丫頭,要了兩張紙就解衣。衆人又是笑,又忙喝他“這裏使不得!”忙命一個婆子帶了東北上去了。那婆子指與地方,便樂得走開去歇息。
\end{parag}


\begin{parag}
    那劉姥姥因喝了些酒,他脾氣不與黃酒相宜,且吃了許多油膩飲食,發渴多喝了幾碗茶,不免通瀉起來,蹲了半日方完。及出廁來,酒被風禁,且年邁之人,蹲了半天,忽一起身,只覺得眼花頭眩,辨不出路徑。四顧一望,皆是樹木山石樓臺房舍,卻不知那一處是往那裏去的了,只得認著一條石子路慢慢的走來。及至到了房舍跟前,又找不著門,再找了半日,忽見一帶竹籬,劉姥姥心中自忖道:“這裏也有扁豆架子。”一面想,一面順著花障走了來,得了一個月洞門進去。只見迎面忽有一帶水池,只有七八尺寬,石頭砌岸,裏邊碧瀏清水流往那邊去了,\begin{note}蒙側:借劉姥姥醉中,寫境中景。\end{note}上面有一塊白石橫架在上面。劉姥姥便度石過去,順著石子甬路走去,轉了兩個彎子,只見有一房門。於是進了房門,只見迎面一個女孩兒,滿面含笑迎了出來。劉姥姥忙笑道:“姑娘們把我丟下來了,要我碰頭碰到這裏來。”說了,只覺那女孩兒不答。劉姥姥便趕來拉他的手,“咕咚”一聲,便撞到板壁上,把頭碰的生疼。細瞧了一瞧,原來是一幅畫兒。劉姥姥自忖道: “原來畫兒有這樣活凸出來的。”一面想,一面看,一面又用手摸去,卻是一色平的,點頭嘆了兩聲。一轉身方得了一個小門,門上掛著蔥綠撒花軟簾。劉姥姥掀簾進去,抬頭一看,只見四面牆壁玲瓏剔透,琴劍瓶爐皆貼在牆上,錦籠紗罩,金彩珠光,連地下踩的磚,皆是碧綠鑿花,竟越發把眼花了,找門出去,那裏有門?左一架書,右一架屏。剛從屏後得了一門轉去,只見他親家母也從外面迎了進來。劉姥姥詫異,忙問道:“你想是見我這幾日沒家去,虧你找我來。那一位姑娘帶你進來的?”他親家只是笑,不還言。劉姥姥笑道:“你好沒見世面,見這園裏的花好,你就沒死活戴了一頭。”他親家也不答。便心下忽然想起:“常聽大富貴人家有一種穿衣鏡,這別是我在鏡子裏頭呢罷。”說畢伸手一摸,再細一看,可不是,四面雕空紫檀板壁將鏡子嵌在中間。因說:“這已經攔住,如何走出去呢?”一面說,一面只管用手摸。這鏡子原是西洋機括,可以開合。不意劉姥姥亂摸之間,其力巧合,便撞開消息,掩過鏡子,露出門來。劉姥姥又驚又喜,邁步出來,忽見有一副最精緻的牀帳。他此時又帶了七八分醉,又走乏了,便一屁股坐在牀上,只說歇歇,不承望身不由己,前仰後合的,朦朧著兩眼,一歪身就睡熟在牀上。
\end{parag}


\begin{parag}
    且說衆人等他不見,板兒見沒了他姥姥,急的哭了。衆人都笑道:“別是掉在茅廁裏了?快叫人去瞧瞧。”因命兩個婆子去找,回來說沒有。衆人各處搜尋不見。襲人敁敠其道路:“是他醉了迷了路,順著這一條路往我們後院子裏去了。若進了花障子到後房門進去,雖然碰頭,還有小丫頭們知道;若不進花障子再往西南上去,若繞出去還好,若繞不出去,可夠他繞回子好的。我且瞧瞧去。”一面想,一面回來,進了怡紅院便叫人,誰知那幾個房子裏小丫頭已偷空頑去了。
\end{parag}


\begin{parag}
    襲人一直進了房門,轉過集錦槅子,就聽的鼾齁如雷。忙進來,只聞見酒屁臭氣,滿屋一瞧,只見劉姥姥扎手舞腳的仰臥在牀上。襲人這一驚不小,慌忙趕上來將他沒死活的推醒。那劉姥姥驚醒,睜眼見了襲人,連忙爬起來道:“姑娘,我失錯了!並沒弄髒了牀帳。”一面說,一面用手去撣。襲人恐驚動了人,被寶玉知道了,只向他搖手,不叫他說話。忙將鼎內貯了三四把百合香,仍用罩子罩上。些須收拾收拾,所喜不曾嘔吐,忙悄悄的笑道:“不相干,有我呢。你隨我出來。”\begin{note}蒙側:這方是襲人的平素筆,至此不得不屈,再增支派則累矣。\end{note}\begin{note}蒙側:總是恰好便住。\end{note}劉姥姥跟了襲人,出至小丫頭們房中,命他坐了,向他說道: “你就說醉倒在山子石上打了個盹兒。”劉姥姥答應知道。又與他兩碗茶喫,方覺酒醒了,因問道:“這是那個小姐的繡房,這樣精緻?我就象到了天宮裏的一樣。”襲人微微笑道:“這個麼,是寶二爺的臥室。”那劉姥姥嚇的不敢作聲。襲人帶他從前面出去,見了衆人,只說他在草地下睡著了,帶了他來的。衆人都不理會,也就罷了。
\end{parag}


\begin{parag}
    一時賈母醒了,就在稻香村擺晚飯。賈母因覺懶懶的,也不喫飯,便坐了竹椅小敞轎,回至房中歇息,命鳳姐兒等去喫飯。他姊妹方復進園來。要知端的且聽下回分解。
\end{parag}


\begin{parag}
    \begin{note}蒙回末總:劉姥姥之憨從利,妙玉尼之怪圖名,寶玉之奇黛玉之妖亦自飲跡,何等畫工能將他人之天王,作我衛護之縱(?)神。文技至此,可爲之美。\end{note}
\end{parag}

