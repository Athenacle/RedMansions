\chap{二十}{王熙鳳正言彈妒意 林黛玉俏語謔嬌音}

\begin{parag}
    \begin{note}蒙回前詩:智慧生魔多象,魔生智慧方深。智魔寂滅萬緣根,不解智魔作甚。\end{note}
\end{parag}


\begin{parag}
    話說寶玉在林黛玉房中說“耗子精”,寶釵撞來,諷刺寶玉元宵不知“綠蠟”之典,三人正在房中互相譏刺取笑。那寶玉正恐黛玉飯後貪眠,一時存了食,或夜間走了困,皆非保養身體之法;\begin{note}庚雙夾:雲寶玉亦知醫理,卻只是在顰、釵等人前方露,亦如後回許多明理之語,只在閨前現露三分,越在雨村等經濟人前如癡如呆,實令人可恨。但雨村等視寶玉不是人物,豈知寶玉視彼等更不是人物,故不與接談也。寶玉之情癡,真乎?假乎?看官細評。\end{note}幸而寶釵走來,大家談笑,那林黛玉方不欲睡,自己才放了心。忽聽他房中嚷起來,大家側耳聽了一聽,林黛玉先笑道:“這是你媽媽和襲人叫嚷呢。那襲人也罷了,你媽媽再要認真排場他,可見老背晦了。”\begin{note}庚雙夾:襲卿能使顰卿一讚,愈見彼之爲人矣,觀者諸公以爲如何?\end{note}
\end{parag}


\begin{parag}
    寶玉忙要趕過來,寶釵忙一把拉住道:\begin{note}庚側:的是寶釵行事。\end{note}“你別和你媽媽吵纔是,他老糊塗了,倒要讓他一步爲是。”\begin{note}庚雙夾:寶釵如何?觀者思之。\end{note}寶玉道:“我知道了。”說畢走來,只見李嬤嬤拄著柺棍,在當地罵襲人:\begin{note}庚側:活像過時奶媽罵丫頭。\end{note}“忘了本的小娼婦!\begin{note}庚側:在襲卿身上去叫下撞天屈來。\end{note}我抬舉起你來,這會子我來了,你大模大樣的躺在炕上,見我來也不理一理。一心只想妝狐媚子哄寶玉,\begin{note}庚側:看這句幾把批書人嚇殺了。\end{note}哄的寶玉不理我,聽你們的話。\begin{note}庚側:幸有此二句,不然我石兄襲卿掃地矣。\end{note}你不過是幾兩臭銀子買來的毛丫頭,這屋裏你就作耗,如何使得!好不好拉出去配一個小子,\begin{note}庚側:雖寫得酷肖,然唐突我襲卿,實難爲情。\end{note}看你還妖精似的哄寶玉不哄!”\begin{note}庚側:若知“好事多魔”,方會作者這意。\end{note}襲人先只道李嬤嬤不過爲他躺著生氣,少不得分辨說“病了,纔出汗,蒙著頭,原沒看見你老人家”等語。後來只管聽他說“哄寶玉”、“妝狐媚”,又說“配小子”等,由不得又愧又委屈,禁不住哭起來。
\end{parag}


\begin{parag}
    寶玉雖聽了這些話,也不好怎樣,少不得替襲人分辨病了吃藥等話,又說:“你不信,只問別的丫頭們。”李嬤嬤聽了這話,益發氣起來了,說道:“你只護著那起狐狸,那裏認得我了,叫我問誰去?\begin{note}庚側:真有是語。\end{note}誰不幫著你呢,\begin{note}庚側:真有是事。\end{note}誰不是襲人拿下馬來的!\begin{note}庚側:冤枉冤哉!\end{note}我都知道那些事。\begin{note}庚側:囫圇語,難解。\end{note}我只和你在老太太,太太跟前去講了。把你奶了這麼大,\begin{note}庚側:奶媽拿手話。\end{note}到如今喫不著奶了,把我丟在一旁,逞著丫頭們要我的強。”\begin{note}庚眉:特爲乳母傳照,暗伏後文倚勢奶孃線脈。《石頭記》無閒文並虛字在此。壬午孟夏。畸笏老人。\end{note}一面說,一面也哭起來。彼時黛玉寶釵等也走過來勸說:“媽媽你老人家擔待他們一點子就完了。”李嬤嬤見他二人\begin{note}庚側:四字,嬤嬤是看重二人身分。\end{note}來了,便拉住訴委屈,將當日喫茶,茜雪出去,與昨日酥酪等事,嘮嘮叨叨說個不清。\begin{note}庚側:好極,妙極,畢肖極!\end{note}\begin{note}庚眉:茜雪至“獄神廟”方呈正文。襲人正文標目曰“花襲人有始有終”,餘隻見有一次謄清時,與“獄神廟慰寶玉”等五六稿,被借閱者迷失,嘆嘆!丁亥夏。畸笏叟。\end{note}
\end{parag}


\begin{parag}
    可巧鳳姐正在上房算完輸贏賬,聽得後面一片聲嚷,便知是李嬤嬤老病發了,排揎寶玉的人。--正值他今兒輸了錢,\begin{note}庚側:找上文。\end{note}遷怒於人。\begin{note}庚側:有是爭競事。\end{note}便連忙趕過來,拉了李嬤嬤,笑道:“好媽媽,別生氣。大節下老太太才喜歡了一日,你是個老人家,別人高聲,你還要管他們呢,難道你反不知道規矩,在這裏嚷起來,叫老太太生氣不成?\begin{note}庚側:阿鳳兩提“老太太”,是叫老嫗想襲卿是老太太的人,況又雙關大體,勿泛泛看去。\end{note}你只說誰不好,我替你打他。我家裏燒的滾熱的野雞,快來跟我喫酒去。”\begin{note}庚側:何等現成,何等自然,的是鳳卿筆法。\end{note}一面說,一面拉著走,又叫:“豐兒,替你李奶奶拿著柺棍子,擦眼淚的手帕子。”\begin{note}庚側:一絲不漏。\end{note}那李嬤嬤腳不沾地跟了鳳姐走了,一面還說:“我也不要這老命了,越性今兒沒了規矩,鬧一場子,討個沒臉,強如受那娼婦蹄子的氣!”後面寶釵黛玉隨著,見鳳姐兒這般,都拍手笑道:“虧這一陣風來,把個老婆子撮了去了。”\begin{note}庚側:批書人也是這樣說。看官將一部書中人一一想來,收拾文字非阿鳳俱有瑣細引跡事。《石頭記》得力處俱在此。\end{note}
\end{parag}


\begin{parag}
    寶玉點頭嘆道:“這又不知是那裏的帳,只揀軟的排揎。昨兒又不知是那個姑娘得罪了,上在他帳上。”一句未了,晴雯在旁笑道:“誰又不瘋了,得罪他作什麼。便得罪了他,就有本事承任,不犯帶累別人!”襲人一面哭,一面拉著寶玉道:“爲我得罪了一個老奶奶,你這會子又爲我得罪這些人,這還不夠我受的,還只是拉別人。”寶玉見他這般病勢,又添了這些煩惱,連忙忍氣吞聲,安慰他仍舊睡下出汗。又見他湯燒火熱,自己守著他,歪在旁邊,勸他只養著病,別想著些沒要緊的事生氣。襲人冷笑道:“要爲這些事生氣,這屋裏一刻還站不得了。\begin{note}庚側:實言,非謬語也。\end{note}但只是天長日久,只管這樣,可叫人怎麼樣纔好呢?時常我勸你,別爲我們得罪人,你只顧一時爲我們那樣,他們都記在心裏,遇著坎兒,說的好說不好聽,大傢什麼意思。”\begin{note}庚側:從“狐媚子”等語來,實實好語,的是襲卿。\end{note}一面說,一面禁不住流淚,又怕寶玉煩惱,只得又勉強忍著。\begin{note}庚眉:一段特爲怡紅襲人、晴雯、茜雪三環之性情見識身份而寫。己冬夜。\end{note}
\end{parag}


\begin{parag}
    一時雜使的老婆子煎了二和藥來。寶玉見他纔有汗意,不肯叫他起來,自己便端著就枕與他吃了,即命小丫頭子們鋪炕。襲人道:“你喫飯不喫飯,到底老太太,太太跟前坐一會子,\begin{note}庚側:心中時時刻刻正意語也。\end{note}和姑娘們頑一會子再回來。我就靜靜的躺一躺也好。”寶玉聽說,只得替他去了簪環,看他躺下,自往上房來。同賈母喫畢飯,賈母猶欲同那幾個老管家嬤嬤鬥牌解悶,寶玉記著襲人,便回至房中,見襲人朦朦睡去。自己要睡,天氣尚早。彼時晴雯、綺霰、秋紋、碧痕都尋熱鬧,找鴛鴦琥珀等耍戲去了,獨見麝月一個人在外間房裏燈下抹骨牌。寶玉笑問道:“你怎不同他們頑去?”麝月道:“沒有錢。”寶玉道:“牀底下堆著那麼些,還不夠你輸的?”麝月道:“都頑去了,這屋裏交給誰呢?\begin{note}庚側:正文。\end{note}那一個又病了。滿屋裏上頭是燈,地下是火。\begin{note}庚側:燈節。\end{note}那些老媽媽子們,老天拔地,伏侍一天,也該叫他們歇歇,小丫頭子們也是伏侍了一天,這會子還不叫他們頑頑去。所以讓他們都去罷,我在這裏看著。”\begin{note}庚眉:麝月閒閒無語,令餘酸鼻,正所謂對景傷情。丁亥夏。畸笏。\end{note}
\end{parag}


\begin{parag}
    寶玉聽了這話,公然又是一個襲人。\begin{note}庚側:豈敢。\end{note}因笑道:“我在這裏坐著,你放心去罷。”\begin{note}庚側:每於如此等處石兄何嘗輕輕放過不介意來?亦作者欲瞞看官,又被批書人看出,呵呵。\end{note}麝月道:“你既在這裏,越發不用去了,咱們兩個說話頑笑豈不好?”\begin{note}庚側:全是襲人口氣,所以後來代任。\end{note}寶玉笑道:“咱兩個作什麼呢?怪沒意思的,也罷了,早上你說頭癢,這會子沒什麼事,我替你篦頭罷。”麝月聽了便道:“就是這樣。”說著,將文具鏡匣搬來,卸去釵釧,打開頭髮,寶玉拿了篦子替他一一的梳篦。\begin{note}庚側:金閨細事如此寫。\end{note}只篦了三五下,只見晴雯忙忙走進來取錢。一見了他兩個,便冷笑道:“哦,交杯盞還沒喫,倒上頭了!”\begin{note}庚側:雖謔語,亦少露怡紅細事。\end{note}寶玉笑道:“你來,我也替你篦一篦。”晴雯道:“我沒那麼大福。”說著,拿了錢,便摔簾子出去了。
\end{parag}


\begin{parag}
    寶玉在麝月身後,麝月對鏡,二人在鏡內相視。\begin{note}庚側:此係石兄得意處。\end{note}寶玉便向鏡內笑道:“滿屋裏就只是他磨牙。”麝月聽說,忙向鏡中擺手,\begin{note}庚側:好看,趣。\end{note}寶玉會意。忽聽唿一聲簾子響,晴雯又跑進來問道:\begin{note}庚側:麝月搖手爲此,可兒可兒!\end{note}“我怎麼磨牙了?\begin{note}庚側:好看煞!\end{note}咱們倒得說說。”\begin{note}庚眉:嬌憨滿紙令人叫絕。壬午九月。\end{note}麝月笑道:“你去你的罷,又來問人了。”晴雯笑道:“你又護著。你們那瞞神弄鬼的,\begin{note}庚側:找上文。\end{note}我都知道。等我撈回本兒來再說話。”說著,一徑出去了。\begin{note}庚雙夾:閒閒一段兒女口舌,卻寫麝月一人。襲人出嫁之後,寶玉、寶釵身邊還有一人,雖不及襲人周到,亦可免微嫌小弊等患,方不負寶釵之爲人也。故襲人出嫁後雲“好歹留著麝月”一語,寶玉便依從此話。可見襲人雖去實未去也。寫晴雯之疑忌,亦爲下文跌扇角口等文伏脈,卻又輕輕抹去。正見此時都在幼時,雖微露其疑忌,見得人各稟天真之性,善惡不一,往後漸大漸生心矣。但觀者凡見晴雯諸人則惡之,何愚也哉!要知自古及今,愈是尤物,其猜忌愈甚。若一味渾厚大量涵養,則有何可令人憐愛護惜哉?然後知寶釵、襲人等行爲,並非一味蠢拙古板以女夫子自居,當繡幕燈前、綠窗月下,亦頗有或調或妒、輕俏豔麗等說,不過一時取樂買笑耳,非切切一味妒才嫉賢也,是以高諸人百倍。不然,寶玉何甘心受屈於二女夫子哉?看過後文則知矣。故觀書諸君子不必惡晴雯,正該感晴雯金閨繡閣中生色方是。\end{note}這裏寶玉通了頭,命麝月悄悄的伏侍他睡下,不肯驚動襲人。一宿無話。
\end{parag}


\begin{parag}
    至次日清晨起來,襲人已是夜間發了汗,覺得輕省了些,只吃些米湯靜養。寶玉放了心,因飯後走到薛姨媽這邊來閒逛。彼時正月內,學房中放年學,閨閣中忌針,卻都是閒時。賈環也過來頑,正遇見寶釵、香菱、鶯兒三個趕圍棋作耍,賈環見了也要頑。寶釵素習看他亦如寶玉,並沒他意,今兒聽他要頑,讓他上來坐了一處。一磊十個錢,頭一回自己贏了,心中十分歡喜。\begin{note}庚眉:寫環兄先贏,亦是天生地設現成文字。己冬夜。\end{note}後來接連輸了幾盤,便有些著急。趕著這盤正該自己擲骰子,若擲個七點便贏,若擲個六點,下該鶯兒擲三點就贏了。因拿起骰子來,狠命一擲,一個作定了五,那一個亂轉。鶯兒拍著手只叫“幺”,\begin{note}庚側:好看煞。\end{note}\begin{note}庚雙夾:嬌憨如此。\end{note}賈環便瞪著眼,“六——七——八”混叫。那骰子偏生轉出幺來。賈環急了,伸手便抓起骰子來,然後就拿錢,\begin{note}庚側:更也好看。\end{note}說是個六點。鶯兒便說:“分明是個幺!”寶釵見賈環急了,便瞅鶯兒說道:“越大越沒規矩,難道爺們還賴你?還不放下錢來呢!”鶯兒滿心委屈,見寶釵說,不敢則聲,只得放下錢來,口內嘟囔說:“一個作爺的,還賴我們這幾個錢,\begin{note}庚側:酷肖。\end{note}連我也不放在眼裏。前兒我和寶二爺頑,他輸了那些,也沒著急。\begin{note}庚側:倒捲簾法,實寫幼時往事。可傷。\end{note}下剩的錢,還是幾個小丫頭子們一搶,他一笑就罷了。”寶釵不等說完,連忙斷喝。賈環道:“我拿什麼比寶玉呢。你們怕他,都和他好,\begin{note}庚側:蠢驢!\end{note}都欺負我不是太太養的。”\begin{note}庚側:觀者至此,有不捲簾厭看者乎?餘替寶卿實難爲情。\end{note}說著,便哭了。寶釵忙勸他:“好兄弟,快別說這話,人家笑話你。”又罵鶯兒。
\end{parag}


\begin{parag}
    正值寶玉走來,見了這般形況,問是怎麼了。賈環不敢則聲。寶釵素知他家規矩,凡作兄弟的,都怕哥哥,\begin{note}庚雙夾:大族規矩原是如此,一絲兒不錯。\end{note}卻不知那寶玉是不要人怕他的。他想著:“兄弟們一併都有父母教訓,何必我多事,反生疏了。況且我是正出,他是庶出,饒這樣還有人背後談論,\begin{note}庚側:此意不呆。\end{note}還禁得轄治他了。”更有個呆意思存在心裏。\begin{note}庚眉:又用諱人語瞞著看官。己冬夜。\end{note}—— 你道是何呆意?因他自幼姊妹叢中長大,親姊妹有元春、探春,伯叔的有迎春、惜春,親戚中又有史湘雲、林黛玉、薛寶釵等諸人。他便料定,原來天生人爲萬物之靈,凡山川日月之精秀,只鍾於女兒,鬚眉男子不過是些渣滓濁沫而已。因有這個呆念在心,把一切男子都看成混沌濁物,可有可無。只是父親叔伯兄弟中。因孔子是亙古第一人說下的,不可忤慢,只得要聽他這句話。\begin{note}庚側:聽了這一個人之話,豈是呆子?由你自己說罷。我把你作極乖的人看。\end{note}所以,弟兄之間不過盡其大概的情理就罷了,並不想自己是丈夫,須要爲子弟之表率。是以賈環等都不怕他,卻怕賈母,才讓他三分。如今寶釵恐怕寶玉教訓他,倒沒意思,便連忙替賈環掩飾。寶玉道:“大正月裏哭什麼?這裏不好,你別處頑去。你天天唸書,倒唸糊塗了。比如這件東西不好,橫豎那一件好,就棄了這件取那個。難道你守著這個東西哭一會子就好了不成?你原是來取樂頑的,既不能取樂,就往別處去尋樂頑去。哭一會子,難道算取樂頑了不成?倒招自己煩惱,不如快去爲是。”\begin{note}庚側:呆子都會立這樣意,說這樣話?\end{note}賈環聽了,只得回來。
\end{parag}


\begin{parag}
    趙姨娘見他這般,因問:“又是那裏墊了踹窩來了?”\begin{note}庚側:多事人等口角談吐。\end{note}一問不答,\begin{note}庚側:畢肖。\end{note}再問時,賈環便說:“同寶姐姐頑的,鶯兒欺負我,賴我的錢,寶玉哥哥攆我來了。”趙姨娘啐道:“誰叫你上高臺盤去了?下流沒臉的東西!那裏頑不得?誰叫你跑了去討沒意思!”
\end{parag}


\begin{parag}
    正說著,可巧鳳姐在窗外過,都聽在耳內,便隔窗說道:“大正月又怎麼了?環兄弟小孩子家,一半點兒錯了,你只教導他,說這些淡話作什麼!憑他怎麼去,還有太太老爺管他呢,就大口啐他!\begin{note}庚側:反得了理了,所謂貶中褒,想趙姨即不畏阿鳳,亦無可回答。\end{note}他現是主子,不好了,橫豎有教導他的人,與你什麼相干!環兄弟,出來,跟我頑去。”\begin{note}庚側:嫡嫡是彼親生,句句竟成正中貶,趙姨實難答言。到此方知題標用“彈”字甚妥協。己冬夜。\end{note}賈環素日怕鳳姐比怕王夫人更甚,聽見叫他,忙唯唯的出來。趙姨娘也不敢則聲。\begin{note}庚側:“彈妒意”正文。\end{note}鳳姐向賈環道:“你也是個沒氣性的!時常說給你:要喫,要喝,要頑,要笑,只愛同那一個姐姐妹妹哥哥嫂子頑,就同那個頑。你不聽我的話,反叫這些人教的歪心邪意,\begin{note}蒙側:借人發脫,好阿鳳!好口齒!句句正言正禮,趙姨安得不抿翅低頭靜聽發揮批至不禁一大白又一大白矣!\end{note}狐媚子霸道的。自己不尊重,要往下流走,安著壞心,還只管怨人家偏心。輸了幾個錢?\begin{note}庚側:轉得好。\end{note}就這麼個樣兒!”賈環見問,只得諾諾的回說:“輸了一二百。”鳳姐道:“虧你還是爺,輸了一二百錢就這樣!”\begin{note}庚側:□者當記一大百乎。笑笑。\end{note}回頭叫豐兒:“去取一吊錢來,姑娘們都在後頭頑呢,把他送了頑去。\begin{note}庚側:收拾得好。\end{note}你明兒再這麼下流狐媚子,我先打了你,打發人告訴學裏,皮不揭了你的!爲你這個不尊重,\begin{note}庚側:又一折筆,更覺有味。\end{note}恨的你哥哥牙根癢癢,不是我攔著,窩心腳把你的腸子窩出來了。”喝命:“去罷!”\begin{note}庚側:本來面目,斷不可少。\end{note}賈環諾諾的跟了豐兒,得了錢,\begin{note}蒙夾:三字寫著環哥。\end{note}自己和迎春等頑去。不在話下。\begin{note}庚雙夾:一段大家子奴妾吆吻如見如聞,正爲下文五鬼作引也。餘爲寶玉肯效鳳姐一點餘風,亦可繼榮、寧之盛,諸公當爲如何?\end{note}
\end{parag}


\begin{parag}
    且說寶玉正和寶釵頑笑,忽見人說:“史大姑娘來了。”\begin{note}庚雙夾:妙極!凡寶玉、寶釵正閒相遇時,非黛玉來,即湘雲來,是恐泄漏文章之精華也。若不如此,則寶玉久坐忘情,必被寶卿見棄,杜絕後文成其夫婦時無可談舊之情,有何趣味哉?\end{note}寶玉聽了,抬身就走。寶釵笑道:“等著,\begin{note}庚眉:“等著”二字大有神情。看官閉目熟思,方知趣味。非批書人漫擬也。己冬夜。\end{note}咱們兩個一齊走,瞧瞧他去。”說著,下了炕,同寶玉一齊來至賈母這邊。只見史湘雲大笑大說的,見他兩個來,忙問好廝見。\begin{note}庚雙夾:寫湘雲又一筆法,特犯不犯。\end{note}正值林黛玉在旁,因問寶玉:“在那裏的?”寶玉便說:“在寶姐姐家的。” 黛玉冷笑道:“我說呢,虧在那裏絆住,不然早就飛了來了。”\begin{note}庚側:總是心中事語,故機括一動,隨機而出。\end{note}寶玉笑道:“只許同你頑,替你解悶兒。不過偶然去他那裏一趟,就說這話。”林黛玉道:“好沒意思的話!去不去管我什麼事,我又沒叫你替我解悶兒。可許你從此不理我呢!”說著,便賭氣回房去了。
\end{parag}


\begin{parag}
    寶玉忙跟了來,問道:“好好的又生氣了?就是我說錯了,你到底也還坐在那裏,和別人說笑一會子。又來自己納悶。”林黛玉道:“你管我呢!”寶玉笑道: “我自然不敢管你,只沒有個看著你自己作踐了身子呢。”林黛玉道:“我作踐壞了身子,我死,與你何干!”寶玉道:“何苦來,大正月裏,死了活了的。”林黛玉道:“偏說死!我這會子就死!你怕死,你長命百歲的,如何?”寶玉笑道:“要象只管這樣鬧,我還怕死呢?倒不如死了乾淨。”黛玉忙道:“正是了,要是這樣鬧,不如死了乾淨。”寶玉道:“我說我自己死了乾淨,別聽錯了話賴人。”正說著,寶釵走來道:“史大妹妹等你呢。”說著,便推寶玉走了。\begin{note}庚雙夾:此時寶釵尚未知他二人心性,故來勸,後文察其心性,故擲之不聞矣。\end{note}這裏黛玉越發氣悶,只向窗前流淚。沒兩盞茶的工夫,寶玉仍來了。\begin{note}庚雙夾:蓋寶玉亦是心中只有黛玉,見寶釵難卻其意,故暫隨彼去,以完寶釵之情,是以少坐仍來也。\end{note}林黛玉見了,越發抽抽噎噎的哭個不住。寶玉見了這樣,知難挽回,打疊起千百樣的款語溫言來勸慰。不料自己未張口,\begin{note}庚側:石頭慣用如此筆仗。\end{note}只見黛玉先說道:“你又來作什麼?橫豎如今有人和你頑,比我又會念,又會作,又會寫,又會說笑,又怕你生氣拉了你去,你又作什麼來?死活憑我去罷了!”寶玉聽了忙上來悄悄的說道:“你這麼個明白人,難道連‘親不間疏,先不僭後’\begin{note}庚側:八字足可消氣。\end{note}也不知道?我雖糊塗,卻明白這兩句話。頭一件,咱們是姑舅姊妹,寶姐姐是兩姨姊妹,論親戚,他比你疏。第二件,你先來,咱們兩個一桌喫,一牀睡,長的這麼大了,他是纔來的,豈有個爲他疏你的?”林黛玉啐道:“我難道爲叫你疏他?我成了個什麼人了呢!我爲的是我的心。”寶玉道:“我也爲的是我的心。難道你就知你的心,不知我的心不成?”\begin{note}庚雙夾:此二語不獨觀者不解,料作者亦未必解;不但作者未必解,想石頭亦不解;不過述寶、林二人之語耳。石頭既未必解,寶、林此刻更自己亦不解,皆隨口說出耳。若觀者必欲要解,須揣自身是寶、林之流,則洞然可解;若自料不是寶、林之流,則不必求解矣。萬不可記此二句不解,錯謗寶、林及石頭、作者等人。\end{note}林黛玉聽了,低頭一語不發,半日說道:“你只怨人行動嗔怪了你,你再不知道你自己慪人難受。就拿今日天氣比,分明今兒冷的這樣,你怎麼倒反把個青肷披風脫了呢?”\begin{note}庚雙夾:真正奇絕妙文,真如羚羊掛角,無跡可求。此等奇妙,非口中筆下可形容出者。\end{note}寶玉笑道:“何嘗不穿著,見你一惱,我一燥就脫了。” 黛玉嘆道:“回來傷了風,又該餓著吵喫的了。”\begin{note}庚側:一語仍歸兒女本傳,卻又輕輕抹去也。庚眉:明明寫湘雲來是正文,只用二三答言,反寫玉、林小角口,又用寶釵岔開,仍不了局。再用千句柔言百般溫態,正在情完未完之時,湘雲突至,“謔嬌音”之文終見。真是“賣弄有傢俬”之筆也。丁亥夏。笏叟。\end{note}
\end{parag}


\begin{parag}
    二人正說著,只見湘雲走來,笑道:“二哥哥,林姐姐,你們天天一處頑,我好容易來了,也不理我一理兒。”黛玉笑道:“偏是咬舌子愛說話,連個‘二’哥哥也叫不出來,只是‘愛’哥哥‘愛’哥哥的。回來趕圍棋兒,又該你鬧‘幺愛三四五’了。”寶玉笑道:“你學慣了他,明兒連你還咬起來呢。”\begin{note}庚雙夾:可笑近之野史中,滿紙羞花閉月、鶯啼燕語。殊不知真正美人方有一陋處,如太真之肥、飛燕之瘦、西子之病,若施於別個,不美矣。今見“咬舌”二字加之湘雲,是何大法手眼敢用此二字哉?不獨不見其陋,且更覺輕巧嬌媚,儼然一嬌憨湘雲立於紙上,掩卷合目思之,其“愛”“厄”嬌音如入耳內。然後將滿紙鶯啼燕語之字樣填糞窖可也。\end{note}史湘雲道:“他再不放人一點兒,專挑人的不好。你自己便比世人好,也不犯著見一個打趣一個。指出一個人來,你敢挑他,我就伏你。”黛玉忙問是誰。湘雲道:“你敢挑寶姐姐的短處,就算你是好的。我算不如你,他怎麼不及你呢。”黛玉聽了,冷笑道:“我當是誰,原來是他!我那裏敢挑他呢。”\begin{note}庚眉:此作者放筆寫,非褒釵貶顰也。\end{note}寶玉不等說完,忙用話岔開。湘雲笑道:“這一輩子我自然比不上你。我只保佑著明兒得一個咬舌的林姐夫,時時刻刻你可聽‘愛’‘厄’去。阿彌陀佛,那才現在我眼裏!”說的衆人一笑,湘雲忙回身跑了。要知端詳,下回分解。
\end{parag}


\begin{parag}
    \begin{note}蒙回末總評:此迴文字重作輕抹。得力處是鳳姐拉李媽媽去,借環哥彈壓趙姨娘。細緻處寶釵爲李媽媽勸寶玉,安慰環哥,斷喝鶯兒。至急處爲難處是寶、顰論心。無可奈何處是“就拿今日天氣比”,“黛玉冷笑道:‘我當誰,原來是他!’”。冷眼最好看處是寶釵、黛玉看鳳姐拉李嬤嬤“這一陣風”;玉、麝一節;湘雲到,寶玉就走,寶釵笑說“等著”;湘雲大笑大說;顰兒學咬舌;湘雲唸佛跑了數節可使看官於紙上耳聞目睹其音其形之文。\end{note}
\end{parag}