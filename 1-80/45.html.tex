\chap{四十五}{金兰契互剖金兰语 风雨夕闷制风雨词}


\begin{parag}
    \begin{note}蒙回前总:富贵荣华春暖,梦破皇粮愁晚,金玉坐楼基,也是戏场妆点。莫缓,莫缓,遗却灵光无限。\end{note}
\end{parag}


\begin{parag}
    话说凤姐儿正抚恤平儿,忽见众姊妹进来,忙让坐了,平儿斟上茶来。凤姐儿笑道:“今儿来的这么齐,倒象下帖子请了来的。”探春笑道:“我们有两件事:一件是我的,一件是四妹妹的,还夹著老太太的话。”凤姐儿笑道:“有什么事,这么要紧?”探春笑道:“我们起了个诗社,头一社就不齐全,众人脸软,所以就乱了。我想必得你去作个监社御史,铁面无私才好。再四妹妹为画园子,用的东西这般那般不全,回了老太太,老太太说:‘只怕后头楼底下还有当年剩下的,找一找,若有呢拿出来,若没有,叫人买去。’”凤姐笑道:“我又不会作什么湿的干的,要我吃东西去不成?”探春道:“你虽不会作,也不要你作。你只监察著我们里头有偷安怠惰的,该怎么样罚他就是了。”凤姐儿笑道:“你们别哄我,我猜著了,那里是请我作监社御史!分明是叫我作个进钱的铜商。你们弄什么社,必是要轮流作东道的。你们的月钱不够花了,想出这个法子来拗了我去,好和我要钱。可是这个主意?”一席话说的众人都笑起来了。李纨笑道:“真真你是个水晶心肝玻璃人。”凤姐儿笑道:“亏你是个大嫂子呢!把姑娘们原交给你带著念书学规矩针线的,他们不好,你要劝。这会子他们起诗社,能用几个钱,你就不管了?老太太、太太罢了,原是老封君。你一个月十两银子的月钱,比我们多两倍银子。老太太、太太还说你寡妇失业的,可怜,不够用,又有个小子,足的又添了十两,和老太太、太太平等。又给你园子地,各人取租子。年终分年例,你又是上上分儿。你娘儿们,主子奴才共总没十个人,吃的穿的仍旧是官中的。一年通共算起来,也有四五百银子。这会子你就每年拿出一二百两银子来陪他们顽顽,能几年的限?他们各人出了阁,难道还要你赔不成?这会子你怕花钱,调唆他们来闹我,我乐得去吃一个河涸海干,我还通不知道呢!”
\end{parag}


\begin{parag}
    李纨笑道:“你们听听,我说了一句,他就疯了,说了两车的无赖泥腿市俗专会打细算盘分斤拨两的话出来。\begin{note}庚双夹:心直口拙之人急了恨不得将万句话来并成一句说死那人,毕肖!\end{note}这东西亏他托生在诗书大宦名门之家做小姐,出了嫁又是这样,他还是这么著;若是生在贫寒小户人家,作个小子,还不知怎么下作贫嘴恶舌的呢!天下人都被你算计了去!昨儿还打平儿呢,亏你伸的出手来!那黄汤难道灌丧了狗肚子里去了?气的我只要给平儿打报不平儿。忖夺了半日,好容易‘狗长尾巴尖儿’的好日子,又怕老太太心里不受用,因此没来,究竟气还未平。你今儿又招我来了。给平儿拾鞋也不要,你们两个只该换一个过子才是。”说的众人都笑了。凤姐儿忙笑道:“竟不是为诗为画来找我,这脸子竟是为平儿来报仇的。竟不承望平儿有你这一位仗腰子的人。早知道,便有鬼拉著我的手打他,我也不打了。平姑娘,过来!我当著大奶奶姑娘们替你赔个不是,担待我酒后无德罢。”说著,众人又都笑起来了。李纨笑问平儿道:“如何?我说必定要给你争争气才罢。”平儿笑道:“虽如此,奶奶们取笑,我禁不起。”李纨道:“什么禁不起,有我呢。快拿了钥匙叫你主子开了楼房找东西去。”
\end{parag}


\begin{parag}
    凤姐儿笑道:“好嫂子,你且同他们回园子里去。才要把这米帐合算一算,那边大太太又打发人来叫,又不知有什么话说,须得过去走一趟。还有年下你们添补的衣服,还没打点给他们做去。”李纨笑道:“这些事情我都不管,你只把我的事完了我好歇著去,省得这些姑娘小姐闹我。”凤姐忙笑道:“好嫂子,赏我一点空儿。你是最疼我的,怎么今儿为平儿就不疼我了?往常你还劝我说,事情虽多,也该保养身子,捡点著偷空儿歇歇,你今儿反到逼我的命了。况且误了别人的年下衣裳无碍,他姊妹们的若误了,却是你的责任,老太太岂不怪你不管闲事,这一句现成的话也不说?我宁可自己落不是,岂敢带累你呢。”李纨笑道:“你们听听,说的好不好?把他会说话的!我且问你:这诗社你到底管不管?”凤姐儿笑道:“这是什么话,我不入社花几个钱,不成了大观园的反叛了,还想在这里吃饭不成?明儿一早就到任,下马拜了印,先放下五十两银子给你们慢慢作会社东道。过后几天,我又不作诗作文,只不过是个俗人罢了。‘监察’也罢,不‘监察’也罢,有了钱了,你们还撵出我来!”说的众人又都笑起来。凤姐儿道:“过会子我开了楼房,凡有这些东西都叫人搬出来你们看,若使得,留著使,若少什么,照你们单子,我叫人替你们买去就是了。画绢我就裁出来。那图样没有在太太跟前,还在那边珍大爷那里呢。说给你们,别碰钉子去。我打发人取了来,一并叫人连绢交给相公们矾去。如何?”李纨点首笑道:“这难为你,果然这样还罢了。既如此,咱们家去罢,等著他不送了去再来闹他。”说著,便带了他姊妹就走。凤姐儿道:“这些事再没两个人,都是宝玉生出来的。”李纨听了,忙回身笑道:“正是为宝玉来,反忘了他。头一社是他误了。我们脸软,你说该怎么罚他?”凤姐想了一想,说道: “没有别的法子,只叫他把你们各人屋子里的地罚他扫一遍才好。”众人都笑道:“这话不差。”
\end{parag}


\begin{parag}
    说著才要回去,只见一个小丫头扶了赖嬷嬷进来。凤姐儿等忙站起来,笑道:“大娘坐。”又都向他道喜。赖嬷嬷向炕沿上坐了,笑道:“我也喜,主子们也喜。若不是主子们的恩典,我们这喜从何来?昨儿奶奶又打发彩哥儿赏东西,我孙子在门上朝上磕了头了。”李纨笑道:“多早晚上任去?”赖嬷嬷叹道:“我那里管他们,由他们去罢!前儿在家里给我磕头,我没好话,我说:‘哥哥儿,你别说你是官儿了,横行霸道的!你今年活了三十岁,虽然是人家的奴才,一落娘胎胞,主子恩典,放你出来,上托著主子的洪福,下托著你老子娘,也是公子哥儿似的读书认字,也是丫头、老婆、奶子捧凤凰似的,长了这么大。你那里知道那“奴才” 两字是怎么写的!只知道享福,也不知道你爷爷和你老子受的那苦恼,熬了两三辈子,好容易挣出你这么个东西来。从小儿三灾八难,花的银子也照样打出你这么个银人儿来了。到二十岁上,又蒙主子的恩典,许你捐个前程在身上。你看那正根正苗的忍饥挨饿的要多少?你一个奴才秧子,仔细折了福!如今乐了十年,不知怎么弄神弄鬼的,求了主子,又选了出来。州县官儿虽小,事情却大,为那一州的州官,就是那一方的父母。你不安分守己,尽忠报国,孝敬主子,只怕天也不容你。 ’”李纨凤姐儿都笑道:“你也多虑。我们看他也就好了。先那几年还进来了两次,这有好几年没来了,年下生日,只见他的名字就罢了。前儿给老太太、太太磕头来,在老太太那院里,见他又穿著新官的服色,倒发的威武了,比先时也胖了。他这一得了官,正该你乐呢,反倒愁起这些来!他不好,还有他父亲呢,你只受用你的就完了。闲了坐个轿子进来,和老太太斗一日牌,说一天话儿,谁好意思的委屈了你。家去一般也是楼房厦厅,谁不敬你,自然也是老封君似的了。”
\end{parag}


\begin{parag}
    平儿斟上茶来,赖嬷嬷忙站起来接了,笑道:“姑娘不管叫那个孩子倒来罢了,又折受我。”说著,一面吃茶,一面又道:“奶奶不知道。这些小孩子们全要管的严。饶这么严,他们还偷空儿闹个乱子来叫大人操心。知道的说小孩子们淘气;不知道的,人家就说仗著财势欺人,连主子名声也不好。恨的我没法儿,常把他老子叫来骂一顿,才好些。”因又指宝玉道:“不怕你嫌我,如今老爷不过这么管你一管,老太太护在头里。当日老爷小时挨你爷爷的打,谁没看见的。老爷小时,何曾象你这么天不怕地不怕的了。还有那大老爷,虽然淘气,也没象你这扎窝子的样儿,也是天天打。还有东府里你珍哥儿的爷爷,那才是火上浇油的性子,说声恼了,什么儿子,竟是审贼!如今我眼里看著,耳朵里听著,那珍大爷管儿子倒也象当日老祖宗的规矩,只是管的到三不著两的。他自己也不管一管自己,这些兄弟侄儿怎么怨的不怕他?你心里明白,喜欢我说,不明白,嘴里不好意思,心里不知怎么骂我呢!”
\end{parag}


\begin{parag}
    正说著,只见赖大家的来了,接著周瑞家的张材家的都进来回事情。凤姐儿笑道:“媳妇来接婆婆来了。”赖大家的笑道:“不是接他老人家,倒是打听打听奶奶姑娘们赏脸不赏脸?”赖嬷嬷听了,笑道:“可是我糊涂了,正经说的话且不说,且说陈谷子烂芝麻的混捣熟。因为我们小子选了出来,众亲友要给他贺喜,少不得家里摆个酒。我想,摆一日酒,请这个也不是,请那个也不是。又想了一想,托主子洪福,想不到的这样荣耀,就倾了家,我也是愿意的。因此吩咐他老子连摆三日酒:头一日,在我们破花园子里摆几席酒,一台戏,请老太太、太太们、奶奶姑娘们去散一日闷;外头大厅上一台戏,摆几席酒,请老爷们、爷们去增增光;第二日再请亲友;第三日再把我们两府里的伴儿请一请。热闹三天,也是托著主子的洪福一场,光辉光辉。”李纨凤姐儿都笑道:“多早晚的日子?我们必去,只怕老太太高兴要去也定不得。”赖大家的忙道:“择了十四的日子,只看我们奶奶的老脸罢了。”凤姐笑道:“别人我不知道,我是一定去的。先说下,我是没有贺礼的,也不知道放赏,吃完了一走,可别笑话。”赖大家的笑道:“奶奶说那里话?奶奶要赏,赏我们三二万银子就有了。” 嬷嬷笑道:“我才去请老太太,老太太也说去,可算我这脸还好。”说毕又叮咛了一回,方起身要走,因看见周瑞家的,便想起一事来,因说道:“可是还有一句话问奶奶,这周嫂子的儿子犯了什么不是,撵了他不用?”凤姐儿听了,笑道:“正是我要告诉你媳妇,事情多也忘了。赖嫂子回去说给你老头子,两府里不许收留他小子,叫他各人去罢。”
\end{parag}


\begin{parag}
    赖大家的只得答应著。周瑞家的忙跪下央求。嬷嬷忙道:“什么事说给我评评。”凤姐儿道:“前日我生日,里头还没吃酒,他小子先醉了。老娘那边送了礼来,他不说在外头张罗,他倒坐著骂人,礼也不送进来。两个女人进来了,他才带著小幺们往里抬。小幺们倒好,他拿的一盒子倒失了手,撒了一院子馒头。人去了,打发彩明去说他,他倒骂了彩明一顿。这样无法无天的忘八羔子,不撵了作什么!”赖嬷嬷笑道:“我当什么事情,原来为这个。奶奶听我说:他有不是,打他骂他,使他改过,撵了去断乎使不得。他又比不得是咱们家的家生子儿,他现是太太的陪房。奶奶只顾撵了他,太太脸上不好看。依我说,奶奶教导他几板子,以戒下次,仍旧留著才是。不看他娘,也看太太。”凤姐儿听说,便向赖大家的说道:“既这样,打他四十棍,以后不许他吃酒。”赖大家的答应了。周瑞家的磕头起来,又要与赖嬷嬷磕头,赖大家的拉著方罢。然后他三人去了,李纨等也就回园中来。
\end{parag}


\begin{parag}
    至晚,果然凤姐命人找了许多旧收的画具出来,送至园中。宝钗等选了一回,各色东西可用的只有一半,将那一半又开了单子,与凤姐儿去照样置买,不必细说。
\end{parag}


\begin{parag}
    一日,外面矾了绢,起了稿子进来。宝玉每日便在惜春这里帮忙。\begin{note}庚双夹:自忙不暇,又加上一“帮”字,可笑可笑,所谓《春秋》笔法。\end{note}探春、李纨、迎春、宝钗等也多往那里闲坐,一则观画,二则便于会面。宝钗因见天气凉爽,夜复渐长,\begin{note}庚双夹:“复”字妙,补出宝钗每年夜长之事,皆《春秋》字法也。\end{note}遂至母亲房中商议打点些针线来。日间至贾母处王夫人处省候两次,不免又承色陪坐半时,园中姊妹处也要度时闲话一回,故日间不大得闲,每夜灯下女工必至三更方寝。\begin{note}庚双夹:代下收夕,写针线下“商议”二字,直将寡母训女多少温存活现在纸上。不写阿呆兄已见阿呆兄终日饱醉优游,怒则吼、喜则跃,家务一概无闻之形景毕露矣。《春秋》笔法。\end{note}黛玉每岁至春分秋分之后,必犯嗽疾;今秋又遇贾母高兴,多游玩了两次,未免过劳了神,近日又复嗽起来,觉得比往常又重,所以总不出门,只在自己房中将养。有时闷了,又盼个姊妹来说些闲话排遣;及至宝钗等来望候他,说不得三五句话又厌烦了。众人都体谅他病中,且素日形体娇弱,禁不得一些委屈,所以他接待不周,礼数粗忽,也都不苛责。
\end{parag}


\begin{parag}
    这日宝钗来望他,因说起这病症来。宝钗道:“这里走的几个太医虽都还好,只是你吃他们的药总不见效,不如再请一个高明的人来瞧一瞧,治好了岂不好?每年间闹一春一夏,又不老又不小,成什么?不是个常法。”黛玉道:“不中用。我知道我这样病是不能好的了。且别说病,只论好的日子我是怎么形景,就可知了。”宝钗点头道:“可正是这话。古人说:‘食谷者生。’你素日吃的竟不能添养精神气血,也不是好事。”黛玉叹道:“‘死生有命,富贵在天’,也不是人力可强的。今年比往年反觉又重了些似的。”说话之间,已咳嗽了两三次。宝钗道:“昨儿我看你那药方上,人参肉桂觉得太多了。虽说益气补神,也不宜太热。依我说,先以平肝健胃为要,肝火一平,不能克土,胃气无病,饮食就可以养人了。每日早起拿上等燕窝一两,冰糖五钱,用银铫子熬出粥来,若吃惯了,比药还强,最是滋阴补气的。”
\end{parag}


\begin{parag}
    黛玉叹道:“你素日待人,固然是极好的,然我最是个多心的人,只当你心里藏奸。从前日你说看杂书不好,又劝我那些好话,竟大感激你。往日竟是我错了,实在误到如今。细细算来,我母亲去世的早,又无姊妹兄弟,我长了今年十五岁,\begin{note}庚双夹:黛玉才十五岁,记清。\end{note}竟没一个人象你前日的话教导我。怨不得云丫头说你好,我往日见他赞你,我还不受用,昨儿我亲自经过,才知道了。比如若是你说了那个,我再不轻放过你的;你竟不介意,反劝我那些话,可知我竟自误了。若不是从前日看出来,今日这话,再不对你说。你方才说叫我吃燕窝粥的话,虽然燕窝易得,但只我因身上不好了,每年犯这个病,也没什么要紧的去处。请大夫,熬药,人参肉桂,已经闹了个天翻地覆,这会子我又兴出新文来熬什么燕窝粥,老太太、太太、凤姐姐这三个人便没话说,那些底下的婆子丫头们,未免不嫌我太多事了。你看这里这些人,因见老太太多疼了宝玉和凤丫头两个,他们尚虎视眈眈,背地里言三语四的,何况于我?况我又不是他们这里正经主子,原是无依无靠投奔了来的,他们已经多嫌著我了。如今我还不知进退,何苦叫他们咒我?”宝钗道:“这样说,我也是和你一样。”黛玉道:“你如何比我?你又有母亲,又有哥哥,这里又有买卖地土,家里又仍旧有房有地。你不过是亲戚的情分,白住了这里,一应大小事情,又不沾他们一文半个,要走就走了。我是一无所有,吃穿用度,一草一纸,皆是和他们家的姑娘一样,那起小人岂有不多嫌的。”宝钗笑道:“将来也不过多费得一副嫁妆罢了,如今也愁不到这里。”\begin{note}庚双夹:宝钗此一戏直抵通部黛玉之戏宝钗矣,又恳切、又真情、又平和、又雅致、又不穿凿、又不牵强,黛玉因识得宝钗后方吐真情,宝钗亦识得黛玉后方肯戏也,此是大关节大章法,非细心看不出。二人此时好看之极,真是儿女小窗中喁喁也。\end{note}黛玉听了,不觉红了脸,笑道:“人家才拿你当个正经人,把心里的烦难告诉你听,你反拿我取笑儿。”宝钗笑道:“虽是取笑儿,却也是真话。你放心,我在这里一日,我与你消遣一日。你有什么委屈烦难,只管告诉我,我能解的,自然替你解一日。我虽有个哥哥,你也是知道的,只有个母亲比你略强些。咱们也算同病相怜。你也是个明白人,何必作‘司马牛之叹’?\begin{note}庚双夹:通部众人必从宝钗之评方定,然宝钗亦必从颦儿之评始可,何妙之至!\end{note}你才说的也是,多一事不如省一事。我明日家去和妈妈说了,只怕我们家里还有,与你送几两,每日叫丫头们就熬了,又便宜,又不惊师动众的。”黛玉忙笑道:“东西事小,难得你多情如此。”宝钗道:“这有什么放在口里的!只愁我人人跟前失于应候罢了。只怕你烦了,我且去了。”黛玉道:“晚上再来和我说句话儿。”宝钗答应著便去了,不在话下。
\end{parag}


\begin{parag}
    这里黛玉喝了两口稀粥,仍歪在床上,不想日未落时天就变了,淅淅沥沥下起雨来。秋霖脉脉,阴晴不定,那天渐渐的黄昏,且阴的沉黑,兼著那雨滴竹梢,更觉凄凉。知宝钗不能来,便在灯下随便拿了一本书,却是《乐府杂稿》,有《秋闺怨》《别离怨》等词。黛玉不觉心有所感,亦不禁发于章句,遂成《代别离》一首,拟《春江花月夜》之格,乃名其词曰《秋窗风雨夕》。其词曰:
\end{parag}


\begin{poem}
    \begin{pl}秋花惨淡秋草黄,耿耿秋灯秋夜长。\end{pl}

    \begin{pl}已觉秋窗秋不尽,那堪风雨助凄凉!\end{pl}

    \begin{pl}助秋风雨来何速!惊破秋窗秋梦绿。\end{pl}

    \begin{pl}抱得秋情不忍眠,自向秋屏移泪烛。\end{pl}

    \begin{pl}泪烛摇摇蓺短檠,牵愁照恨动离情。\end{pl}

    \begin{pl}谁家秋院无风入?何处秋窗无雨声?\end{pl}

    \begin{pl}罗衾不奈秋风力,残漏声催秋雨急。\end{pl}

    \begin{pl}连宵脉脉复飕飕,灯前似伴离人泣。\end{pl}

    \begin{pl}寒烟小院转萧条,疏竹虚窗时滴沥。\end{pl}

    \begin{pl}不知风雨几时休,已教泪洒纱窗湿。\end{pl}

\end{poem}


\begin{parag}
    吟罢搁笔,方要安寝,丫鬟报说:“宝二爷来了。”一语未完,只见宝玉头上带著大箬笠,身上披著蓑衣。黛玉不觉笑了:“那里来的渔翁!”宝玉忙问:“今儿好些?\begin{note}庚双夹:一句。\end{note}吃了药没有?\begin{note}庚双夹:两句。\end{note}今儿一日吃了多少饭?”\begin{note}庚双夹:三句。\end{note}一面说,一面摘了笠,脱了蓑衣,忙一手举起灯来,一手遮住灯光,向黛玉脸上照了一照,觑著眼细瞧了一瞧,笑道:“今儿气色好了些。”
\end{parag}


\begin{parag}
    黛玉看脱了蓑衣,里面只穿半旧红绫短袄,系著绿汗巾子,膝下露出油绿绸撒花裤子,底下是掐金满绣的绵纱袜子,靸著蝴蝶落花鞋。黛玉问道:“上头怕雨,底下这鞋袜子是不怕雨的?也倒干净。”宝玉笑道:“我这一套是全的。有一双棠木屐,才穿了来,脱在廊檐上了。”黛玉又看那蓑衣斗笠不是寻常市卖的,十分细致轻巧,因说道:“是什么草编的?怪道穿上不象那刺猬似的。”宝玉道:“这三样都是北静王送的。他闲了下雨时在家里也是这样。你喜欢这个,我也弄一套来送你。别的都罢了,惟有这斗笠有趣,竟是活的。上头的这顶儿是活的,冬天下雪,带上帽子,就把竹信子抽了,去下顶子来,只剩了这圈子。下雪时男女都戴得,我送你一顶,冬天下雪戴。”黛玉笑道:“我不要他。戴上那个,成个画儿上画的和戏上扮的渔婆了。”及说了出来,方想起话未忖夺,与方才说宝玉的话相连,后悔不及,羞的脸飞红,便伏在桌上嗽个不住。\begin{note}庚双夹:妙极之文。使黛玉自己直说出夫妻来,却又云“画的”“扮的”,本是闲谈,却是暗隐不吉之兆。所谓 “画儿中爱宠”是也,谁曰不然?\end{note}
\end{parag}


\begin{parag}
    宝玉却不留心,\begin{note}庚双夹:必云“不留心”方好,方是宝玉,若著心则又有何文字?且直是一时时猎色一贼矣。\end{note}因见案上有诗,遂拿起来看了一遍,又不禁叫好。黛玉听了,忙起来夺在手内,向灯上烧了。宝玉笑道:“我已背熟了,烧也无碍。”黛玉道:“我也好了许多,谢你一天来几次瞧我,下雨还来。这会子夜深了,我也要歇著,你且请回去,明儿再来。”宝玉听说,回手向怀中掏出一个核桃大小的一个金表来,瞧了一瞧,那针已指到戌末亥初之间,忙又揣了,说道: “原该歇了,又扰的你劳了半日神。”说著,披蓑戴笠出去了,又翻身进来问道:“你想什么吃,告诉我,我明儿一早回老太太,岂不比老婆子们说的明白?”\begin{note}庚双夹:直与后部宝钗之文遥遥针对。想彼姊妹房中婆子丫鬟皆有,随便皆可遣使,今宝玉独云“婆子”而不云“丫鬟”者,心内已度定丫鬟之为人,一言一事无论大小,是方无错谬者也,一何可笑。\end{note}黛玉笑道:“等我夜里想著了,明儿早起告诉你。你听雨越发紧了,快去罢。可有人跟著没有?”有两个婆子答应:“有人,外面拿著伞点著灯笼呢。”黛玉笑道:“这个天点灯笼?”宝玉道:“不相干,是明瓦的,不怕雨。”黛玉听了,回手向书架上把个玻璃绣球灯拿了下来,命点一支小蜡来,递与宝玉,道:“这个又比那个亮,正是雨里点的。”宝玉道:“我也有这么一个,怕他们失脚滑倒了打破了,所以没点来。”黛玉道:“跌了灯值钱,跌了人值钱?你又穿不惯木屐子。那灯笼命他们前头点著。这个又轻巧又亮,原是雨里自己拿著的,你自己手里拿著这个,岂不好?明儿再送来。就失了手也有限的,怎么忽然又变出这‘剖腹藏珠’的脾气来!”宝玉听说,连忙接了过来,前头两个婆子打著伞提著明瓦灯,后头还有两个小丫鬟打著伞。宝玉便将这个灯递与一个小丫头捧著,宝玉扶著他的肩,一径去了。
\end{parag}


\begin{parag}
    就有蘅芜苑的一个婆子,也打著伞提著灯,送了一大包上等燕窝来,还有一包子洁粉梅片雪花洋糖。说:“这比买的强。姑娘说了:姑娘先吃著,完了再送来。”黛玉道:“回去说‘费心’。”命他外头坐了吃茶。婆子笑道:“不吃茶了,我还有事呢。”黛玉笑道:“我也知道你们忙。如今天又凉,夜又长,越发该会个夜局,痛赌两场了。”婆子笑道:“不瞒姑娘说,今年我大沾光儿了。横竖每夜各处有几个上夜的人,误了更也不好,不如会个夜局,又坐了更,又解闷儿。今儿又是我的头家,如今园门关了,就该上场了。”\begin{note}庚双夹:几句闲话将潭潭大宅夜间所有之事描写一尽。虽诺大一园,且值秋冬之夜,岂不寥落哉?今用老妪数语,更写得每夜深人定之后,各处灯光灿烂、人烟簇集,柳陌之上、花巷之中,或提灯同酒,或寒月烹茶者,竟仍有络绎人迹不绝,不但不见寥落,且觉更胜于日间繁华矣。此是大宅妙景,不可不写出,又伏下后文,且又衬出后文之冷落。此闲话中写出,正是不写之写也。脂砚斋评。\end{note}黛玉听说笑道:“难为你。误了你发财,冒雨送来。”命人给他几百钱,打些酒吃,避避雨气。那婆子笑道:“又破费姑娘赏酒吃。”说著,磕了一个头,外面接了钱,打伞去了。
\end{parag}


\begin{parag}
    紫鹃收起燕窝,然后移灯下帘,伏侍黛玉睡下。黛玉自在枕上感念宝钗,一时又羡他有母兄;一面又想宝玉虽素习和睦,终有嫌疑。又听见窗外竹梢焦叶之上,雨声淅沥,清寒透幕,不觉又滴下泪来。直到四更将阑,方渐渐的睡了。暂且无话。要知端的——
\end{parag}


\begin{parag}
    \begin{note}蒙回末总:请看赖大,则知贵家奴婢身份,而本主毫不以为过分,习惯自然故是有之。见者当自度是否可也。\end{note}
\end{parag}

