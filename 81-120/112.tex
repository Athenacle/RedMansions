\chap{一百一十二}{活冤孽妙尼遭大劫 死雠仇赵妾赴冥曹}



\begin{parag}
    话说凤姐命捆起上夜众女人送营审问,女人跪地哀求。林之孝同贾芸道:“你们求也无益。老爷派我们看家,没有事是造化,如今有了事,上下都担不是,谁救得你。若说是周瑞的干儿子,连太太起,里里外外的都不干净。”凤姐喘吁吁的说道:“这都是命里所招,和他们说什么,带了他们去就是了。这丢的东西你告诉营里去说,实在是老太太的东西,问老爷们才知道。等我们报了去,请了老爷们回来,自然开了失单送来。文官衙门里我们也是这样报。”贾芸林之孝答应出去。
\end{parag}


\begin{parag}
    惜春一句话也没有,只是哭道:“这些事我从来没有听见过,为什么偏偏碰在咱们两个人身上!明儿老爷太太回来叫我怎么见人!说把家里交给咱们,如今闹到这个分儿,还想活着么!”凤姐道:“咱们愿意吗!现在有上夜的人在那里。”惜春道:“你还能说,况且你又病着。我是没有说的。这都是我大嫂子害了我的,他撺掇着太太派我看家的。如今我的脸搁在那里呢!”说着,又痛哭起来。凤姐道:“姑娘,你快别这么想,若说没脸,大家一样的。你若这么糊涂想头,我更搁不住了。”
\end{parag}


\begin{parag}
    二人正说着,只听见外头院子里有人大嚷的说道:“我说那三姑六婆是再要不得的,我们甄府里从来是一概不许上门的,不想这府里倒不讲究这个呢。昨儿老太太的殡才出去,那个什么庵里的尼姑死要到咱们这里来,我吆喝着不准他们进来,腰门上的老婆子倒骂我,死央及叫放那姑子进去。那腰门子一会儿开着,一会儿关着,不知做什么,我不放心没敢睡,听到四更这里就嚷起来。我来叫门倒不开了,我听见声儿紧了,打开了门,见西边院子里有人站着,我便赶走打死了。我今儿才知道,这是四姑奶奶的屋子。那个姑子就在里头,今儿天没亮溜出去了,可不是那姑子引进来的贼么。”平儿等听着,都说:“这是谁这么没规矩?姑娘奶奶都在这里,敢在外头混嚷吗。”凤姐道:“你听见说‘他甄府里’,别就是甄家荐来的那个厌物罢。”惜春听得明白,更加心里过不的。凤姐接着问惜春道:“那个人混说什么姑子,你们那里弄了个姑子住下了?”惜春便将妙玉来瞧他留着下棋守夜的话说了。凤姐道:“是他么,他怎么肯这样,是再没有的话。但是叫这讨人嫌的东西嚷出来,老爷知道了也不好。”惜春愈想愈怕,站起来要走。凤姐虽说坐不住,又怕惜春害怕弄出事来,只得叫他先别走。“且看着人把偷剩下的东西收起来,再派了人看着才好走呢。”平儿道:“咱们不敢收,等衙门里来了踏看了才好收呢。咱们只好看着。但只不知老爷那里有人去了没有?”凤姐道:“你叫老婆子问去。”一回进来说:“林之孝是走不开,家下人要伺候查验的,再有的是说不清楚的,已经芸二爷去了。”凤姐点头,同惜春坐着发愁。
\end{parag}


\begin{parag}
    且说那伙贼原是何三等邀的,偷抢了好些金银财宝接运出去,见人追赶,知道都是那些不中用的人,要往西边屋内偷去,在窗外看见里面灯光底下两个美人:一个姑娘,一个姑子。那些贼那顾性命,顿起不良,就要踹进来,因见包勇来赶,才获赃而逃。只不见了何三。大家且躲入窝家。到第二天打听动静,知是何三被他们打死,已经报了文武衙门。这里是躲不住的,便商量趁早规入海洋大盗一处,去若迟了,通缉文书一行,关津上就过不去了。
\end{parag}


\begin{parag}
    内中一个人胆子极大,便说:“咱们走是走,我就只舍不得那个姑子,长的实在好看。不知是那个庵里的雏儿呢?”一个人道:“啊呀,我想起来了,必就是贾府园里的什么栊翠庵里的姑子。不是前年外头说他和他们家什么宝二爷有原故,后来不知怎么又害起相思病来了,请大夫吃药的就是他。”那一个人听了,说:“咱们今日躲一天,叫咱们大哥借钱置办些买卖行头,明儿亮钟时候陆续出关。你们在关外二十里坡等我。”众贼议定,分赃俵散。不题。
\end{parag}


\begin{parag}
    且说贾政等送殡,到了寺内安厝毕,亲友散去。贾政在外厢房伴灵,邢王二夫人等在内,一宿无非哭泣。到了第二日,重新上祭。正摆饭时,只见贾芸进来,在老太太灵前磕了个头,忙忙的跑到贾政跟前跪下请了安,喘吁吁的将昨夜被盗,将老太太上房的东西都偷去,包勇赶贼打死了一个,已经呈报文武衙门的话说了一遍。贾政听了发怔。邢王二夫人等在里头也听见了,都唬得魂不附体,并无一言,只有啼哭。贾政过了一会子问失单怎样开的,贾芸回道:“家里的人都不知道,还没有开单。”贾政道:“还好,咱们动过家的,若开出好的来反担罪名。快叫琏儿。”
\end{parag}


\begin{parag}
    贾琏领了宝玉等去别处上祭未回,贾政叫人赶了回来。贾琏听了,急得直跳,一见芸儿,也不顾贾政在那里,便把贾芸狠狠的骂了一顿说:“不配抬举的东西,我将这样重任托你,押着人上夜巡更,你是死人么!亏你还有脸来告诉!”说着,往贾芸脸上啐了几口。贾芸垂手站着,不敢回一言。贾政道:“你骂他也无益了。”贾琏然后跪下说:“这便怎么样?”贾政道:“也没法儿,只有报官缉贼。但只有一件:老太太遗下的东西咱们都没动,你说要银子,我想老太太死得几天,谁忍得动他那一项银子。原打谅完了事算了帐还人家,再有的在这里和南边置坟产的,再有东西也没见数儿。如今说文武衙门要失单,若将几件好的东西开上恐有碍,若说金银若干,衣饰若干,又没有实在数目,谎开使不得。倒可笑你如今竟换了一个人了,为什么这样料理不开!你跪在这里是怎么样呢!”贾琏也不敢答言,只得站起来就走。贾政又叫道:“你那里去?”贾琏又跪下道:“赶回去料理清楚再来回。”贾政哼的一声,贾琏把头低下。贾政道:“你进去回了你母亲,叫了老太太的一两个丫头去,叫他们细细的想了开单子。”贾琏心里明知老太太的东西都是鸳鸯经管,他死了问谁?就问珍珠,他们那里记得清楚。只不敢驳回,连连的答应了,起来走到里头。邢王夫人又埋怨了一顿,叫贾琏快回去,问他们这些看家的说“明儿怎么见我们!”贾琏也只得答应了出来,一面命人套车预备琥珀等进城,自己骑上骡子,跟了几个小厮,如飞的回去。贾芸也不敢再回贾政,斜签着身子慢慢的溜出来,骑上了马来赶贾琏。一路无话。
\end{parag}


\begin{parag}
    到回了家中,林之孝请了安,一直跟了进来。贾琏到了老太太上屋,见了凤姐惜春在那里,心里又恨又说不出来,便问林之孝道:“衙门里瞧了没有?”林之孝自知有罪,便跪下回道:“文武衙门都瞧了,来踪去迹也看了,尸也验了。”贾琏吃惊道:“又验什么尸?”林之孝又将包勇打死的伙贼似周瑞的干儿子的话回了贾琏。贾琏道:“叫芸儿。”贾芸进来也跪着听话。贾琏道:“你见老爷时怎么没有回周瑞的干儿子做了贼被包勇打死的话?”贾芸说道:“上夜的人说象他的,恐怕不真,所以没有回。”贾琏道:“好糊涂东西!你若告诉了我,就带了周瑞来一认可不就知道了。”林之孝回道:“如今衙门里把尸首放在市口儿招认去了。”贾琏道:“这又是个糊涂东西,谁家的人做了贼,被人打死,要偿命么!”林之孝回道:“这不用人家认,奴才就认得是他。”贾琏听了想道:“是啊,我记得珍大爷那一年要打的可不是周瑞家的么。”林之孝回说:“他和鲍二打架来着,还见过的呢。”贾琏听了更生气,便要打上夜的人。林之孝哀告道:“请二爷息怒,那些上夜的人,派了他们,还敢偷懒?只是爷府上的规矩,三门里一个男人不敢进去的,就是奴才们,里头不叫,也不敢进去。奴才在外同芸哥儿刻刻查点,见三门关的严严的,外头的门一重没有开。那贼是从后夹道子来的。”贾琏道:“里头上夜的女人呢。”林之孝将分更上夜奉奶奶的命捆着等爷审问的话回了。贾琏又问“包勇呢?”林之孝说:“又往园里去了。”贾琏便说:“去叫来。”小厮们便将包勇带来。说:“还亏你在这里,若没有你,只怕所有房屋里的东西都抢了去了呢。”包勇也不言语。惜春恐他说出那话,心下着急。凤姐也不敢言语。只见外头说:“琥珀姐姐等回来了。”大家见了,不免又哭一场。
\end{parag}


\begin{parag}
    贾琏叫人检点偷剩下的东西,只有些衣服尺头钱箱未动,余者都没有了。贾琏心里更加着急,想着“外头的棚杠银,厨房的钱都没有付给,明儿拿什么还呢!”便呆想了一会。只见琥珀等进去,哭了一会,见箱柜开着,所有的东西怎能记忆,便胡乱想猜,虚拟了一张失单,命人即送到文武衙门。贾琏复又派人上夜。凤姐惜春各自回房。贾琏不敢在家安歇,也不及埋怨凤姐,竟自骑马赶出城外。这里凤姐又恐惜春短见,又打发了丰儿过去安慰。
\end{parag}


\begin{parag}
    天已二更。不言这里贼去关门,众人更加小心,谁敢睡觉。且说伙贼一心想着妙玉,知是孤庵女众,不难欺负。到了三更夜静,便拿了短兵器,带了些闷香,跳上高墙。远远瞧见栊翠庵内灯光犹亮,便潜身溜下,藏在房头僻处。
\end{parag}


\begin{parag}
    等到四更,见里头只有一盏海灯,妙玉一人在蒲团上打坐。歇了一会,便嗳声叹气的说道:“我自元墓到京,原想传个名的,为这里请来,不能又栖他处。昨儿好心去瞧四姑娘,反受了这蠢人的气,夜里又受了大惊。今日回来,那蒲团再坐不稳,只觉肉跳心惊。”因素常一个打坐的,今日又不肯叫人相伴。岂知到了五更,寒颤起来。正要叫人,只听见窗外一响,想起昨晚的事,更加害怕,不免叫人。岂知那些婆子都不答应。自己坐着,觉得一股香气透入卤门,便手足麻木,不能动弹,口里也说不出话来,心中更自着急。只见一个人拿着明晃晃的刀进来。此时妙玉心中却是明白,只不能动,想是要杀自己,索性横了心,倒也不怕。那知那个人把刀插在背后,腾出手来将妙玉轻轻的抱起,轻薄了一会子,便拖起背在身上。此时妙玉心中只是如醉如痴。可怜一个极洁极净的女儿,被这强盗的闷香熏住,由着他掇弄了去了。
\end{parag}


\begin{parag}
    却说这贼背了妙玉来到园后墙边,搭了软梯,爬上墙跳出去了。外边早有伙计弄了车辆在园外等着,那人将妙玉放倒在车上,反打起官衔灯笼,叫开栅栏,急急行到城门,正是开门之时。门官只知是有公干出城的,也不及查诘。赶出城去,那伙贼加鞭赶到二十里坡和众强徒打了照面,各自分头奔南海而去。不知妙玉被劫或是甘受污辱,还是不屈而死,不知下落,也难妄拟。
\end{parag}


\begin{parag}
    只言栊翠庵一个跟妙玉的女尼,他本住在静室后面,睡到五更,听见前面有人声响,只道妙玉打坐不安。后来听见有男人脚步,门窗响动,欲要起来瞧看,只是身子发软懒怠开口,又不听见妙玉言语,只睁着两眼听着。到了天亮,终觉得心里清楚,披衣起来,叫了道婆预备妙玉茶水,他便往前面来看妙玉。岂知妙玉的踪迹全无,门窗大开。心里诧异,昨晚响动甚是疑心,说:“这样早,他到那里去了?”走出院门一看,有一个软梯靠墙立着,地下还有一把刀鞘,一条搭膊,便道:“不好了,昨晚是贼烧了闷香了!”急叫人起来查看,庵门仍是紧闭。那些婆子女侍们都说:“昨夜煤气熏着了,今早都起不起来,这么早叫我们做什么。”那女尼道:“师父不知那里去了。”众人道:“在观音堂打坐呢。”女尼道:“你们还做梦呢,你来瞧瞧。”众人不知,也都着忙,开了庵门,满园里都找到了,“想来或是到四姑娘那里去了。”
\end{parag}


\begin{parag}
    众人来叩腰门,又被包勇骂了一顿。众人说道:“我们妙师父昨晚不知去向,所以来找。求你老人家叫开腰门,问一问来了没来就是了。”包勇道:“你们师父引了贼来偷我们,已经偷到手了,他跟了贼受用去了。”众人道:“阿弥陀佛,说这些话的防着下割舌地狱!”包勇生气道:“胡说,你们再闹我就要打了。”众人陪笑央告道:“求爷叫开门我们瞧瞧,若没有,再不敢惊动你太爷了。”包勇道:“你不信你去找,若没有,回来问你们。”包勇说着叫开腰门,众人找到惜春那里。
\end{parag}


\begin{parag}
    惜春正是愁闷,惦着“妙玉清早去后不知听见我们姓包的话了没有,只怕又得罪了他,以后总不肯来。我的知己是没有了。况我现在实难见人。父母早死,嫂子嫌我,头里有老太太,到底还疼我些,如今也死了,留下我孤苦伶仃,如何了局!”想到:“迎春姐姐磨折死了,史姐姐守着病人,三姐姐远去,这都是命里所招,不能自由。独有妙玉如闲云野鹤,无拘无束。我能学他,就造化不小了。但我是世家之女,怎能遂意。这回看家已大担不是,还有何颜在这里。又恐太太们不知我的心事,将来的后事如何呢?”想到其间,便要把自己的青丝绞去,要想出家。彩屏等听见,急忙来劝,岂知已将一半头发绞去。彩屏愈加着忙,说道:“一事不了又出一事,这可怎么好呢!”
\end{parag}


\begin{parag}
    正在吵闹,只见妙玉的道婆来找妙玉。彩屏问起来由,先唬了一跳,说是昨日一早去了没来。里面惜春听见,急忙问道:“那里去了?”道婆们将昨夜听见的响动,被煤气熏着,今早不见有妙玉,庵内软梯刀鞘的话说了一遍。惜春惊疑不定,想起昨日包勇的话来,必是那些强盗看见了他,昨晚抢去了也未可知。但是他素来孤洁的很,岂肯惜命?”怎么你们都没听见么?”众人道:“怎么不听见!只是我们这些人都是睁着眼连一句话也说不出,必是那贼子烧了闷香。妙姑一人想也被贼闷住,不能言语,况且贼人必多,拿刀弄杖威逼着,他还敢声喊么?”正说着,包勇又在腰门那里嚷,说:“里头快把这些混账的婆子赶了出来罢,快关腰门!”彩屏听见恐担不是,只得叫婆子出去,叫人关了腰门。惜春于是更加苦楚,无奈彩屏等再三以礼相劝,仍旧将一半青丝笼起。大家商议不必声张,就是妙玉被抢也当作不知,且等老爷太太回来再说。惜春心里的死定下一个出家的念头,暂且不提。
\end{parag}


\begin{parag}
    且说贾琏回到铁槛寺,将到家中查点了上夜的人,开了失单报去的话回了。贾政道:“怎样开的?”贾琏便将琥珀所记得的数目单子呈出,并说:“这上头元妃赐的东西已经注明。还有那人家不大有的东西不便开上,等侄儿脱了孝出去托人细细的缉访,少不得弄出来的。”贾政听了合意,就点头不言。贾琏进内见了邢王二夫人,商量着“劝老爷早些回家才好呢,不然都是乱麻似的。”邢夫人道:“可不是,我们在这里也是惊心吊胆。”贾琏道:“这是我们不敢说的,还是太太的主意二老爷是依的。”邢夫人便与王夫人商议妥了。
\end{parag}


\begin{parag}
    过了一夜,贾政也不放心,打发宝玉进来说:“请太太们今日回家,过两三日再来。家人们已经派定了,里头请太太们派人罢。”邢夫人派了鹦哥等一干人伴灵,将周瑞家的等人派了总管,其余上下人等都回去。一时忙乱套车备马。贾政等在贾母灵前辞别,众人又哭了一场。
\end{parag}


\begin{parag}
    都起来正要走时,只见赵姨娘还爬在地下不起。周姨娘打谅他还哭,便去拉他。岂知赵姨娘满嘴白沫,眼睛直竖,把舌头吐出,反把家人唬了一大跳。贾环过来乱嚷。赵姨娘醒来说道:“我是不回去的,跟着老太太回南去。”众人道:“老太太那用你来!”赵姨娘道:“我跟了一辈子老太太,大老爷还不依,弄神弄鬼的来算计我。——我想仗着马道婆要出出我的气,银子白花了好些,也没有弄死了一个。如今我回去了,又不知谁来算计我。”众人听见,早知是鸳鸯附在他身上。邢王二夫人都不言语瞅着。只有彩云等代他央告道:“鸳鸯姐姐,你死是自己愿意的,与赵姨娘什么相干,放了他罢。”见邢夫人在这里,也不敢说别的。赵姨娘道:“我不是鸳鸯,他早到仙界去了。我是阎王差人拿我去的,要问我为什么和马婆子用魇魔法的案件。”说着便叫“好琏二奶奶,你在这里老爷面前少顶一句儿罢,我有一千日的不好还有一天的好呢。好二奶奶,亲二奶奶,并不是我要害你,我一时糊涂,听了那个老娼妇的话。”
\end{parag}


\begin{parag}
    正闹着,贾政打发人进来叫环儿。婆子们去回说:“赵姨娘中了邪了,三爷看着呢。”贾政道:“没有的事,我们先走了。”于是爷们等先回。这里赵姨娘还是混说,一时救不过来。邢夫人恐他又说出什么来,便说:“多派几个人在这里瞧着他,咱们先走,到了城里打发大夫出来瞧罢。”王夫人本嫌他,也打撒手儿。宝钗本是仁厚的人,虽想着他害宝玉的事,心里究竟过不去,背地里托了周姨娘在这里照应。周姨娘也是个好人,便应承了。李纨说道:“我也在这里罢。”王夫人道:“可以不必。”于是大家都要起身。贾环急忙道:“我也在这里吗?”王夫人啐道:“糊涂东西!你姨妈的死活都不知,你还要走吗!”贾环就不敢言语了。宝玉道:“好兄弟,你是走不得的。我进了城打发人来瞧你。”说毕,都上车回家。寺里只有赵姨娘,贾环,鹦鹉,等人。
\end{parag}


\begin{parag}
    贾政邢夫人等先后到家,到了上房哭了一场。林之孝带了家下众人请了安,跪着。贾政喝道:“去罢!明日问你!”凤姐那日发晕了几次,竟不能出接,只有惜春见了,觉得满面惭愧。邢夫人也不理他,王夫人仍是照常,李纨,宝钗拉着手说了几句话。独有尤氏说道:“姑娘,你操心了,倒照应了好几天!”惜春一言不答,只涨紫了脸。宝钗将尤氏一拉,使了个眼色,尤氏等各自归房去了。贾政略略地看了看,叹了口气,并不言语,到书房席地坐下,叫了贾琏,贾蓉,贾芸吩咐了几句话。宝玉要在书房来陪贾政,贾政道:“不必。”兰儿仍跟着他母亲,一宿无话。
\end{parag}


\begin{parag}
    次日,林之孝一早进书房跪着,贾政将前后被盗的事问了一遍,并将周瑞供了出来,又说:“衙门拿住了鲍二,身边搜出了失单上的东西,现在夹讯,要在他身上要这一伙贼呢。”贾政听了,大怒道:“家奴负恩,引贼偷窃家主,真是反了!”立刻叫人到城外将周瑞捆了,送到衙门审问。林之孝只管跪着,不敢起来。贾政道:“你还跪着干什么!”林之孝到:“奴才该死,求老爷开恩。”正说着,赖大等一干办事家人上来请安,呈上丧事帐薄。贾政道:“交给琏二爷算明了来回。”吆喝着林之孝起来出去了。
\end{parag}


\begin{parag}
    贾琏一腿跪着,在贾政身边说了一句话。贾政把眼一瞪道:“胡说!老太太的事,银两被贼偷去,难道就该罚奴才拿出来么?”贾琏红了脸,不敢言语,站起来也不敢动。贾政道:“你媳妇怎么样了?”贾琏又跪下说:“看来是不中用了。”贾政叹了口气道:“我不料家运衰败,一至如此!况且环哥他妈尚在庙中病着,也不知是什么症候。你们知道不知道?”贾琏也不敢言语。贾政道:“传出话去,让人带了大夫瞧瞧去。”贾琏急忙答应着出来,叫人带了大夫到铁槛寺去瞧赵姨娘。未知死活,下回分解。
\end{parag}