\chap{四十三}{閒取樂偶攢金慶壽 不了情暫撮土爲香}


\begin{parag}
    \begin{note}蒙回前總:了與不了在心頭,迷卻原來難自由。如有如無誰解得,相生相滅苐傳流。\end{note}
\end{parag}


\begin{parag}
    話說王夫人因見賈母那日在大觀園不過著了些風寒,不是什麼大病,請醫生吃了兩劑藥也就好了,便放了心,因命鳳姐來吩咐他預備給賈政帶送東西。正商議著,只見賈母打發人來請,王夫人忙引著鳳姐兒過來。王夫人又請問:“這會子可又覺大安些”?賈母道:“今日可大好了。方纔你們送來野雞崽子湯,我嚐了一嘗,倒有味兒,又吃了兩塊肉,心裏很受用。”王夫人笑道:“這是鳳丫頭孝敬老太太的。算他的孝心虔,不枉了素日老太太疼他。”賈母點頭笑道:“難爲他想著。若是還有生的,再炸上兩塊,鹹浸浸的,喫粥有味兒。那湯雖好,就只不對稀飯。”鳳姐聽了,連忙答應,命人去廚房傳話。
\end{parag}


\begin{parag}
    這裏賈母又向王夫人笑道:“我打發人請你來,不爲別的。初二是鳳丫頭的生日,上兩年我原早想替他做生日,偏到跟前有大事,就混過去了。今年人又齊全,料著又沒事,咱們大家好生樂一日。”\begin{note}庚雙夾:賈母猶雲“好生樂一日”,可見逐日雖樂,皆還不稱心也。所以世人不論貧富,各有愁腸,終不能時時遂心如意。此是至理,非不足語也。\end{note}王夫人笑道:“我也想著呢。既是老太太高興,何不就商議定了?”賈母笑道:“我想往年不拘誰作生日,都是各自送各自的禮,這個也俗了,也覺生分的似的。今兒我出個新法子,又不生分,又可取笑。”王夫人忙道:“老太太怎麼想著好,就是怎麼樣行。”賈母笑道:“我想著,咱們也學那小家子大家湊分子,\begin{note}庚雙夾:原來請分子是小家的事,近見多少人家紅白事一出且籌算分子之多寡,不知何說。\end{note}多少盡著這錢去辦,你道好頑不好頑?”\begin{note}庚雙夾:看他寫與寶釵作生日,後又偏寫與鳳姐作生日。阿鳳何人也,豈不爲彼之華誕大用一回筆墨哉?只是虧他如何想來。特寫於寶釵之後,較姊妹勝而有餘;於賈母之前,較諸父母相去不遠。一部書中若一個一個只管寫過生日,覆成何文哉?故起用寶釵,盛用阿鳳,終用賈母,各有妙文,各有妙景。餘者諸人或一筆不寫,或偶因一語帶過,或豐或簡,其情當理合,不表可知。豈必諄諄死筆按數而寫衆人之生日哉?迥不犯寶釵。\end{note}王夫人笑道:“這個很好,但不知怎麼湊法?”賈母聽說,益發高興起來,忙遣人去請薛姨媽邢夫人等,\begin{note}蒙側:世家之長上多犯此等“辦壽也要請人”毛病。\end{note}又叫請姑娘們並寶玉,那府裏珍兒媳婦並賴大家的等有頭臉管事的媳婦也都叫了來。
\end{parag}


\begin{parag}
    衆丫頭婆子見賈母十分高興也都高興,忙忙的各自分頭去請的請,傳的傳,沒頓飯的工夫,老的少的,上的下的,烏壓壓擠了一屋子。只薛姨媽和賈母對坐,邢夫人王夫人只坐在房門前兩張椅子上,寶釵姊妹等五六個人坐在炕上,寶玉坐在賈母懷前,地下滿滿的站了一地。賈母忙命拿幾個小杌子來,給賴大母親等幾個高年有體面的媽媽坐了。賈府風俗,年高服侍過父母的家人,比年輕的主子還有體面,所以尤氏鳳姐兒等只管地下站著,那賴大的母親等三四個媽媽告個罪,都坐在小杌子上了。
\end{parag}


\begin{parag}
    賈母笑著把方纔一席話說與衆人聽了。衆人誰不湊這趣兒?再也有和鳳姐兒好的,有情願這樣的;有畏懼鳳姐兒的,巴不得來奉承的:況且都是拿的出來的,所以一聞此言,都欣然應諾。賈母先道:“我出二十兩。”薛姨媽笑道:“我隨著老太太,也是二十兩了。”邢夫人王夫人笑道:“我們不敢和老太太並肩,自然矮一等,每人十六兩罷了。”尤氏李紈也笑道:“我們自然又矮一等,每人十二兩罷。”賈母忙和李紈道:“你寡婦失業的,那裏還拉你出這個錢,我替你出了罷。”\begin{note}庚雙夾:必如是方妙。\end{note}鳳姐忙笑道:“老太太別高興,且算一算賬再攬事。老太太身上已有兩分呢,這會子又替大嫂子出十二兩,說著高興,一會子回想又心疼了。過後兒又說:‘都是爲鳳丫頭花了錢。’使個巧法子,哄著我拿出三四分子來暗裏補上,我還做夢呢。”說的衆人都笑了。賈母笑道:“依你怎麼樣呢?”\begin{note}庚雙夾:又寫阿鳳一樣,更妙。若一筆直下,有何趣哉?\end{note}鳳姐笑道:“生日沒到,我這會子已經摺受的不受用了。我一個錢饒不出,驚動這些人實在不安,不如大嫂子這一分我替他出了罷了。我到了那一日多喫些東西,就享了福了。”邢夫人等聽了,都說:“很是。”賈母方允了。鳳姐兒又笑道:“我還有一句話呢。我想老祖宗自己二十兩,又有林妹妹寶兄弟的兩分子。姨媽自己二十兩,又有寶妹妹的一分子,這倒也公道。只是二位太太每位十六兩,自己又少,又不替人出,這有些不公道。老祖宗吃了虧了!”賈母聽了,忙笑道:“倒是我的鳳姐兒向著我,這說的很是。要不是你,我叫他們又哄了去了。”鳳姐笑道:“老祖宗只把他姐兒兩個交給兩位太太,一位佔一個,派多派少,每位替出一分就是了。”賈母忙說:“這很公道,就是這樣。”賴大的母親忙站起來笑說道:“這可反了!我替二位太太生氣。在那邊是兒子媳婦,在這邊是內侄女兒,倒不向著婆婆姑娘,倒向著別人。這兒媳婦成了陌路人,內侄女兒竟成了個外侄女兒了。”說的賈母與衆人都大笑起來了。\begin{note}庚雙夾:寫阿鳳全副精神,雖一戲,亦人想不到之文。\end{note}賴大之母因又問道:“少奶奶們十二兩,我們自然也該矮一等了。”賈母聽說,道:“這使不得。你們雖該矮一等,我知道你們這幾個都是財主,分位雖低,錢卻比他們多。\begin{note}庚雙夾:驚魂奪魄只此一句。所以一部書全是老婆舌頭,全是諷刺世事,反面春秋也。所謂“癡子弟正照風月鑑”,若單看了家常老婆舌頭,豈非癡子弟乎?\end{note}你們和他們一例才使得。”衆媽媽聽了,連忙答應。賈母又道: “姑娘們不過應個景兒,每人照一個月的月例就是了。”又回頭叫鴛鴦來,“你們也湊幾個人,商議湊了來。”鴛鴦答應著,去不多時帶了平兒、襲人、彩霞等還有幾個小丫鬟來,也有二兩的,也有一兩的。賈母因問平兒:“你難道不替你主子作生日,還入在這裏頭?”平兒笑道:“我那個私自另外有了,這是官中的,也該出一分。”賈母笑道:“這纔是好孩子。”鳳姐又笑道:“上下都全了。還有二位姨奶奶,他出不出,也問一聲兒。盡到他們是理,,不然,他們只當小看了他們了。”\begin{note}庚雙夾:純寫阿鳳以襯後文。\end{note}賈母聽了,忙說:“可是呢,怎麼倒忘了他們!只怕他們不得閒兒,叫一個丫頭問問去。”說著,早有丫頭去了,半日回來說道:“每位也出二兩。”賈母喜道:“拿筆硯來算明,共計多少。”尤氏因悄罵鳳姐道:“我把你這沒足厭的小蹄子!這麼些婆婆嬸子來湊銀子給你過生日,你還不足,又拉上兩個苦瓠子作什麼?”鳳姐也悄笑道:“你少胡說,一會子離了這裏,我才和你算賬。他們兩個爲什麼苦呢?有了錢也是白填送別人,不如拘來咱們樂。”\begin{note}庚雙夾:純寫阿鳳以襯後文,二人形景如見,語言如聞,真描畫得到。\end{note}
\end{parag}


\begin{parag}
    說著,早已合算了,共湊了一百五十兩有餘。賈母道:“一日戲酒用不了。”尤氏道:“既不請客,酒席又不多,兩三日的用度都夠了。頭等,戲不用錢,省在這上頭。”賈母道:“鳳丫頭說那一班好,就傳那一班。”鳳姐兒道:“咱們家的班子都聽熟了,倒是花幾個錢叫一班來聽聽罷。”賈母道:“這件事我交給珍哥媳婦了。越性叫鳳丫頭別操一點心,受用一日纔算。”\begin{note}庚雙夾:所以特受用了,纔有璉卿之變。樂極生悲,自然之理。\end{note}尤氏答應著。又說了一回話,都知賈母乏了,才漸漸的都散出來。
\end{parag}


\begin{parag}
    尤氏等送邢夫人王夫人二人散去,便往鳳姐房裏來商議怎麼辦生日的話。鳳姐兒道:“你不用問我,你只看老太太的眼色行事就完了。”尤氏笑道:“你這阿物兒,也忒行了大運了。我當有什麼事叫我們去,原來單爲這個。出了錢不算,還要我來操心,你怎麼謝我?”鳳姐笑道:“你別扯臊,我又沒叫你來,謝你什麼!你怕操心?你這會子就回老太太去,再派一個就是了。”尤氏笑道:“你瞧他興的這樣兒!我勸你收著些兒好。太滿了就潑出來了。”二人又說了一回方散。
\end{parag}


\begin{parag}
    次日將銀子送到寧國府來,尤氏方纔起來梳洗,因問是誰送過來的,丫鬟們回說:“是林大娘。”尤氏便命叫了他來。丫鬟走至下房,叫了林之孝家的過來。尤氏命他腳踏上坐了,一面忙著梳洗,一面問他:“這一包銀子共多少?”林之孝家的回說:“這是我們底下人的銀子,湊了先送過來。老太太和太太們的還沒有呢。” 正說著,丫鬟們回說:“那府裏太太和姨太太打發人送分子來了。”尤氏笑罵道:“小蹄子們,專會記得這些沒要緊的話。昨兒不過老太太一時高興,故意的要學那小家子湊分子,你們就記得,到了你們嘴裏當正經的說。\begin{note}蒙側:世家風調。\end{note}還不快接了進來好生待茶,再打發他們去。”丫鬟應著,忙接了進來,一共兩封,連寶釵黛玉的都有了。尤氏問還少誰的,林之孝家的道:“還少老太太、太太、姑娘們的和底下姑娘們的。”尤氏道:“還有你們大奶奶的呢?”林之孝家的道: “奶奶過去,這銀子都從二奶奶手裏發,\begin{note}蒙側:伏線。\end{note}一共都有了。”
\end{parag}


\begin{parag}
    說著,尤氏已梳洗了,命人伺候車輛。一時來至榮府,先來見鳳姐。只見鳳姐已將銀子封好,正要送去。尤氏問:“都齊了?”鳳姐兒笑道:\begin{note}庚雙夾: “笑”字就有神情。\end{note}“都有了,快拿了去罷,丟了我不管。”尤氏笑道:“我有些信不及,倒要當面點一點。”說著果然按數一點,只沒有李紈的一分。\begin{note}蒙側:點明題目。\end{note}尤氏笑道:“我說你肏鬼呢,怎麼你大嫂子的沒有?”鳳姐兒笑道:“那麼些還不夠使?短一分兒也罷了,等不夠了我再給你。”\begin{note}庚雙夾:可見阿鳳處處心機。\end{note}尤氏道:“昨兒你在人跟前作人,今兒又來和我賴,這個斷不依你。我只和老太太要去。”鳳姐兒笑道:“我看你利害。明兒有了事,我也‘丁是丁卯是卯’的,你也別抱怨。”尤氏笑道:“你一般的也怕。不看你素日孝敬我,我纔是不依你呢。”\begin{note}蒙側:處處是世情作趣,處處是隨筆埋伏。\end{note}說著,把平兒的一分拿了出來,說道:“平兒,來!把你的收起去,等不夠了,我替你添上。”平兒會意,因說道:“奶奶先使著,若剩下了再賞我一樣。”尤氏笑道:“只許你那主子作弊,就不許我作情兒。”\begin{note}蒙側:請看。\end{note}平兒只得收了。尤氏又道:“我看著你主子這麼細緻,弄這些錢那裏使去!使不了,明兒帶了棺材裏使去。”\begin{note}庚雙夾:此言不假,伏下後文短命。尤氏亦能幹事矣,惜不能勸夫治家,惜哉痛哉!\end{note}
\end{parag}


\begin{parag}
    一面說著,一面又往賈母處來。先請了安,大概說了兩句話,便走到鴛鴦房中和鴛鴦商議,只聽鴛鴦的主意行事,何以討賈母的喜歡。二人計議妥當。尤氏臨走時,也把鴛鴦二兩銀子還他,說:“這還使不了呢。”說著,一徑出來,又至王夫人跟前說了一回話。因王夫人進了佛堂,把彩雲一分也還了他。見鳳姐不在跟前,一時把周、趙二人的也還了。\begin{note}蒙側:另是一番作用。\end{note}他兩個還不敢收。尤氏道:“你們可憐見的,那裏有這些閒錢?鳳丫頭便知道了,有我應著呢。”二人聽說,千恩萬謝的方收了。\begin{note}庚雙夾:尤氏亦可謂有才矣。論有德比阿鳳高十倍,惜乎不能諫夫治家,所謂“人各有當”也。此方是至理至情,最恨近之野史中,惡則無往不惡,美則無一不美,何不近情理之如是耶?\end{note}於是尤氏一徑出來,坐車回家。不在話下。
\end{parag}


\begin{parag}
    展眼已是九月初二日,園中人都打聽得尤氏辦得十分熱鬧,不但有戲,連耍百戲並說書的男女先兒全有,\begin{note}蒙側:剩筆且影射能事者不獨阿鳳。\end{note}都打點取樂頑耍。李紈又向衆姊妹道:“今兒是正經社日,可別忘了。\begin{note}庚雙夾:看書者已忘,批書者亦已忘了,作者竟未忘,忽寫此事,真忙中愈忙、緊處愈緊也。\end{note}寶玉也不來,想必他只圖熱鬧,把清雅就丟開了。”\begin{note}庚雙夾:此獨寶玉乎?亦罵世人。餘亦爲寶玉忘了,不然何不來耶?\end{note}說著,便命丫鬟去瞧作什麼,快請了來。丫鬟去了半日,回說:“花大姐姐說,今兒一早就出門去了。”\begin{note}庚雙夾:奇文。\end{note}衆人聽了,都詫異說:“再沒有出門之理。這丫頭糊塗,不知說話。”因又命翠墨去。一時翠墨回來說:“可不真出了門了。說有個朋友死了,出去探喪去了。”\begin{note}庚雙夾:奇文。信有之乎?花團錦簇之日偏如此寫法。\end{note}探春道:“斷然沒有的事。憑他什麼,再沒今日出門之理。你叫襲人來,我問他。”剛說著,只見襲人走來。李紈等都說道:“今兒憑他有什麼事,也不該出門。頭一件,你二奶奶的生日,老太太都這等高興,兩府上下衆人來湊熱鬧,他倒走了;\begin{note}蒙側:因行文不肯平下一反筆,則文語並奇,好看煞人。\end{note}第二件,又是頭一社的正日子,他也不告假,就私自去了!”襲人嘆道:“昨兒晚上就說了,今兒一早起有要緊的事到北靜王府裏去,就趕回來的。勸他不要去,他必不依。今兒一早起來,又要素衣裳穿,想必是北靜王府裏的要緊姬妾沒了,也未可知。”李紈等道:“若果如此,也該去走走,只是也該回來了。”說著,大家又商議:“咱們只管作詩,等他回來罰他。”剛說著,只見賈母已打發人來請,便都往前頭來了。襲人回明寶玉的事,賈母不樂,便命人去接。
\end{parag}


\begin{parag}
    原來寶玉心裏有件私事,於頭一日就吩咐茗煙:“明日一早要出門,備下兩匹馬在後門口等著,不要別一個跟著。說給李貴,我往北府裏去了。倘或要有人找我,叫他攔住不用找,只說北府裏留下了,橫豎就來的。”茗煙也摸不著頭腦,只得依言說了。今兒一早,果然備了兩匹馬在園後門等著。天亮了,只見寶玉遍體純素,從角門出來,一語不發跨上馬,一彎腰,順著街就顛下去了。茗煙也只得跨馬加鞭趕上,在後面忙問:“往那裏去?”寶玉道:“這條路是往那裏去的?”茗煙道:“這是出北門的大道。出去了冷清清沒有可頑的。”寶玉聽說,點頭道:“正要冷清清的地方好。”說著,越性加了鞭,那馬早已轉了兩個彎子,出了城門。茗煙越發不得主意,只得緊緊跟著。
\end{parag}


\begin{parag}
    一氣跑了七八里路出來,人煙漸漸稀少,寶玉方勒住馬,回頭問茗煙道:“這裏可有賣香的?”焙茗道:“香倒有,不知是那一樣?” 寶玉想道:“別的香不好,須得檀、芸、降三樣。”茗煙笑道:“這三樣可難得。”寶玉爲難。茗煙見他爲難,因問道:“要香作什麼使?我見二爺時常小荷包有散香,何不找一找。”一句提醒了寶玉,便回手向衣襟上拉出一個荷包來,摸了一摸,竟有兩星沉速,心內歡喜:“只是不恭些。”再想自己親身帶的,倒比買的又好些。於是又問爐炭。茗煙道:“這可罷了。荒郊野外那裏有?用這些何不早說,帶了來豈不便宜。”寶玉道:“糊塗東西,若可帶了來,又不這樣沒命的跑了。”\begin{note}庚雙夾:奇奇怪怪不知爲何,看他下文怎樣。\end{note}茗煙想了半日,笑道:“我得了個主意,不知二爺心下如何?我想二爺不只用這個呢,只怕還要用別的。這也不是事。如今我們往前再走二里地,就是水仙庵了。”寶玉聽了忙問:“水仙庵就在這裏?更好了,我們就去。”說著,就加鞭前行,一面回頭向茗煙道:“這水仙庵的姑子長往咱們家去,咱們這一去到那裏,和他借香爐使使,他自然是肯的。”茗煙道:“別說他是咱們家的香火,就是平白不認識的廟裏,和他借,他也不敢駁回。只是一件,我常見二爺最厭這水仙庵的,如何今兒又這樣喜歡了?”寶玉道:“我素日因恨俗人不知原故,混供神混蓋廟,這都是當日有錢的老公們和那些有錢的愚婦們聽見有個神,就蓋起廟來供著,也不知那神是何人,因聽些野史小說,便信真了。\begin{note}庚雙夾:近聞剛丙廟又有三教庵,以如來爲尊,太上爲次,先師爲末,真殺有餘辜,所謂此書救世之溺不假。\end{note}比如這水仙庵裏面因供的是洛神,故名水仙庵,殊不知古來並沒有個洛神,那原是曹子建的謊話,誰知這起愚人就塑了像供著。今兒卻合我的心事,故借他一用。”
\end{parag}


\begin{parag}
    說著早已來至門前。那老姑子見寶玉來了,事出意外,竟象天上掉下個活龍來的一般,忙上來問好,命老道來接馬。寶玉進去,也不拜洛神之像,卻只管賞鑑。雖是泥塑的,卻真有“翩若驚鴻,婉若游龍”之態,“荷出綠波,日映朝霞”之姿。\begin{note}庚雙夾:妙計!用《洛神賦》譖洛神本地風光,愈覺新奇。\end{note}寶玉不覺滴下淚來。老姑子獻了茶。寶玉因和他借香爐。那姑子去了半日,連香供紙馬都預備了來。寶玉道:“一概不用。”便命茗煙捧著爐出至後園中,揀一塊乾淨地方兒,竟揀不出。茗煙道:“那井臺兒上如何?”寶玉點頭,一齊來至井臺上,將爐放下。\begin{note}庚雙夾:妙極之文。寶玉心中揀定是井臺上了,故意使茗煙說出,使彼不犯疑猜矣。寶玉亦有欺人之才,蓋不用耳。\end{note}
\end{parag}


\begin{parag}
    茗煙站過一旁。寶玉掏出香來焚上,含淚施了半禮,\begin{note}庚雙夾:奇文。只雲“施半禮”,終不知爲何事也。\end{note}回身命收了去。茗煙答應,且不收,忙爬下磕了幾個頭,口內祝道:“我茗煙跟二爺這幾年,二爺的心事,我沒有不知道的,只有今兒這一祭祀沒有告訴我,我也不敢問。只是這受祭的陰魂雖不知名姓,想來自然是那人間有一、天上無雙,極聰明極俊雅的一位姐姐妹妹了。二爺心事不能出口,讓我代祝:若芳魂有感,香魄多情,雖然陰陽間隔,既是知己之間,時常來望候二爺,未嘗不可。你在陰間保佑二爺來生也變個女孩兒,和你們一處相伴,再不可又託生這鬚眉濁物了。”說畢,又磕幾個頭,才爬起來。\begin{note}庚雙夾:忽插入茗煙一偏流言,粗看則小兒戲語,亦甚無味。細玩則大有深意,試思寶玉之爲人豈不應有一極伶俐乖巧之小童哉?此一祝亦如《西廂記》中雙文降香,第三柱則不語,紅娘則代祝數語,直將雙文心事道破。此處若寫寶玉一祝,則成何文字?若不祝則成一啞迷,如何散場?故寫茗煙一戲直戲入寶玉心中,又發出前文,又可收後文,又寫茗煙素日之乖覺可人,且襯出寶玉直似一個守禮代嫁的女兒一般,其素日脂香粉氣不待寫而全現出矣。今看此回,直欲將寶玉當作一個極清俊羞怯的女兒,看茗煙則極乖覺可人之丫鬟也。 該 批:這方是作者真意。\end{note}
\end{parag}


\begin{parag}
    寶玉聽他沒說完,便撐不住笑了,\begin{note}庚雙夾:方一笑,蓋原可發笑,且說得合心,愈見可笑也。\end{note}因踢他道:“休胡說,看人聽見笑話。”\begin{note}庚雙夾:也知人笑,更奇。\end{note}茗煙起來收過香爐,和寶玉走著,因道:“我已經和姑子說了,二爺還沒用飯,叫他隨便收拾了些東西,二爺勉強喫些。我知道今兒咱們裏頭大排筵宴,熱鬧非常,二爺爲此才躲了出來的。橫豎在這裏清淨一天,也就盡到禮了。若不喫東西,斷使不得。”寶玉道:“戲酒既不喫,這隨便素的喫些何妨。”茗煙道:“這便纔是。還有一說,咱們來了,還有人不放心。若沒有人不放心,便晚了進城何妨?若有人不放心,二爺須得進城回家去纔是。第一老太太、太太也放了心,第二禮也盡了,不過如此。就是家去了看戲喫酒,也並不是二爺有意,原不過陪著父母盡孝道。二爺若單爲了這個不顧老太太、太太懸心,就是方纔那受祭的陰魂也不安生。二爺想我這話如何?”寶玉笑道:“你的意思我猜著了,你想著只你一個跟了我出來,回來你怕擔不是,所以拿這大題目來勸我。\begin{note}庚雙夾:亦知這個大,妙極!\end{note}我纔來了,不過爲盡個禮,再去喫酒看戲,並沒說一日不進城。這已完了心願,趕著進城,大家放心,豈不兩盡其道。”\begin{note}庚雙夾:這是大通的意見,世人不及的去處。\end{note}茗煙道:“這更好了。”說著二人來至禪堂,果然那姑子收拾了一桌素菜,寶玉胡亂吃了些,茗煙也吃了。
\end{parag}


\begin{parag}
    二人便上馬仍回舊路。茗煙在後面只囑咐:“二爺好生騎著,這馬總沒大騎的,手裏提緊著。”\begin{note}庚雙夾:看他偏不寫鳳姐那樣熱鬧,卻寫這般清冷,真世人意料不到這一篇文字也。\end{note}一面說著,早已進了城,仍從後門進去,忙忙來至怡紅院中。襲人等都不在房裏,只有幾個老婆子看屋子,見他來了,都喜的眉開眼笑,說:“阿彌陀佛,可來了!把花姑娘急瘋了!上頭正坐席呢,二爺快去罷。”寶玉聽說忙將素服脫了,自去尋了華服換上,問在什麼地方坐席,老婆子回說在新蓋的大花廳上。
\end{parag}


\begin{parag}
    寶玉聽說,一徑往花廳來,耳內早已隱隱聞得歌管之聲。剛至穿堂那邊,只見玉釧兒獨坐在廊檐下垂淚,\begin{note}庚雙夾:總是千奇百怪的文字。\end{note}一見他來,便收淚說道:“鳳凰來了,快進去罷。再一會子不來,都反了。”\begin{note}庚雙夾:是平常言語,卻是無限文章,無限情理。看至後文在細思此言,則可知矣。\end{note}寶玉陪笑道:“你猜我往那裏去了?”玉釧兒不答,只管擦淚。\begin{note}庚雙夾:無限情理。\end{note}寶玉忙進廳裏,見了賈母王夫人等,衆人真如得了鳳凰一般。寶玉忙趕著與鳳姐兒行禮。賈母王夫人都說他不知道好歹,“怎麼也不說聲就私自跑了,這還了得!明兒再這樣,等老爺回家來,必告訴他打你。”說著又罵跟的小廝們都偏聽他的話,說那裏去就去,也不回一聲兒。一面又問他到底那去了,可吃了什麼,可唬著了。\begin{note}庚雙夾:奇文,畢肖。\end{note}寶玉只回說:“北靜王的一個愛妾昨日沒了,給他道惱去。他哭的那樣,不好撇下就回來,所以多等了一會子。”賈母道:“以後再私自出門,不先告訴我們,一定叫你老子打你。”寶玉答應著。因又要打跟的小子們,衆人又忙說情,又勸道:“老太太也不必過慮了,他已經回來,大家該放心樂一回了。”賈母先不放心,自然發狠,如今見他來了,喜且有餘,那裏還恨,也就不提了;還怕他不受用,或者別處沒喫飽,路上著了驚怕,反百般的哄他。襲人早過來伏侍。大家仍舊看戲。當日演的是《荊釵記》。賈母薛姨媽等都看的心酸落淚,也有嘆的,也有罵的。要知端的,下回分解。
\end{parag}


\begin{parag}
    \begin{note}蒙回末總:攢金辦壽家常樂,素服焚香無限情。\end{note}
\end{parag}


\begin{parag}
    \begin{note}蒙回末總:寫辦事不獨熙鳳,寫多情不漏亡人,情之所鍾必讓若輩。此所謂情情者也。\end{note}
\end{parag}

