\chap{一百}{破好事香菱结深恨 悲远嫁宝玉感离情}



\begin{parag}
    话说贾政去见了节度,进去了半日不见出来,外头议论不一。李十儿在外也打听不出什么事来,便想到报上的饥荒,实在也着急,好容易听见贾政出来,便迎上来跟着,等不得回去,在无人处便问:“老爷进去这半天,有什么要紧的事?”贾政笑道:“并没有事。只为镇海总制是这位大人的亲戚,有书来嘱托照应我,所以说了些好话。又说我们如今也是亲戚了。”李十儿听得,心内喜欢,不免又壮了些胆子,便竭力纵恿贾政许这亲事。贾政心想薛蟠的事到底有什么挂碍,在外头信息不早,难以打点,故回到本任来便打发家人进京打听,顺便将总制求亲之事回明贾母,如若愿意,即将三姑娘接到任所。家人奉命赶到京中,回明了王夫人,便在吏部打听得贾政并无处分,惟将署太平县的这位老爷革职,即写了禀帖安慰了贾政,然后住着等信。
\end{parag}


\begin{parag}
    且说薛姨妈为着薛蟠这件人命官司,各衙门内不知花了多少银钱,才定了误杀具题。原打量将当铺折变给人,备银赎罪。不想刑部驳审,又托人花了好些钱,总不中用,依旧定了个死罪,监着守候秋天大审。薛姨妈又气又疼,日夜啼哭。宝钗虽时常过来劝解,说是:“哥哥本来没造化。承受了祖父这些家业,就该安安顿顿的守着过日子。在南边已经闹的不象样,便是香菱那件事情就了不得,因为仗着亲戚们的势力,花了些银钱,这算白打死了一个公子。哥哥就该改过做起正经人来,也该奉养母亲才是,不想进了京仍是这样。妈妈为他不知受了多少气,哭掉了多少眼泪。给他娶了亲,原想大家安安逸逸的过日子,不想命该如此,偏偏娶的嫂子又是一个不安静的,所以哥哥躲出门的。真正俗语说的‘冤家路儿狭’,不多几天就闹出人命来了。妈妈和二哥哥也算不得不尽心的了,花了银钱不算,自己还求三拜四的谋干。无奈命里应该,也算自作自受。大凡养儿女是为着老来有靠,便是小户人家还要挣一碗饭养活母亲,那里有将现成的闹光了反害的老人家哭的死去活来的?不是我说,哥哥的这样行为,不是儿子,竟是个冤家对头。妈妈再不明白,明哭到夜,夜哭到明,又受嫂子的气。我呢,又不能常在这里劝解,我看见妈妈这样,那里放得下心。他虽说是傻,也不肯叫我回去。前儿老爷打发人回来说,看见京报唬的了不得,所以才叫人来打点的。我想哥哥闹了事,担心的人也不少。幸亏我还是在跟前的一样,若是离乡调远听见了这个信,只怕我想妈妈也就想杀了。我求妈妈暂且养养神,趁哥哥的活口现在,问问各处的账目。人家该咱们的,咱们该人家的,亦该请个旧伙计来算一算,看看还有几个钱没有。”薛姨妈哭着说道:“这几天为闹你哥哥的事,你来了,不是你劝我,便是我告诉你衙门的事。你还不知道,京里的官商名字已经退了,两个当铺已经给了人家,银子早拿来使完了。还有一个当铺,管事的逃了,亏空了好几千两银子,也夹在里头打官司。你二哥哥天天在外头要帐,料着京里的帐已经去了几万银子,只好拿南边公分里银子并住房折变才够。前两天还听见一个荒信,说是南边的公当铺也因为折了本儿收了。若是这么着,你娘的命可就活不成的了。”说着,又大哭起来。宝钗也哭着劝道:“银钱的事,妈妈操心也不中用,还有二哥哥给我们料理。单可恨这些伙计们,见咱们的势头儿败了,各自奔各自的去也罢了,我还听见说帮着人家来挤我们的讹头。可见我哥哥活了这么大,交的人总不过是些个酒肉弟兄,急难中是一个没有的。妈妈若是疼我,听我的话,有年纪的人,自己保重些。妈妈这一辈子。想来还不致挨冻受饿。家里这点子衣裳家伙,只好听凭嫂子去,那是没法儿的了。所有的家人婆子,瞧他们也没心在这里,该去的叫他们去。就可怜香菱苦了一辈子,只好跟着妈妈过去。实在短什么,我要是有的,还可以拿些个来,料我们那个也没有不依的。就是袭姑娘也是心术正道的,他听见我哥哥的事,他倒提起妈妈来就哭。我们那一个还道是没事的,所以不大着急,若听见了也是要唬个半死儿的。”薛姨妈不等说完,便说:“好姑娘,你可别告诉他。他为一个林姑娘几乎没要了命,如今才好了些。要是他急出个原故来,不但你添一层烦恼,我越发没了依靠了。”宝钗道:“我也是这么想,所以总没告诉他。”正说着,只听见金桂跑来外间屋里哭喊道:“我的命是不要的了!男人呢,已经是没有活的分儿了。咱们如今索性闹一闹,大伙儿到法场上去拼一拼。”说着。便将头往隔断板上乱撞,撞的披头散发。气得薛姨妈白瞪着两只眼,一句话也说不出来。还亏得宝钗嫂子长,嫂子短,好一句,歹一句的劝他。金桂道:“姑奶奶,如今你是比不得头里的了。你两口儿好好的过日子,我是个单身人儿,要脸做什么!”说着,便要跑到街上回娘家去,亏得人还多,扯住了,又劝了半天方住。把个宝琴唬的再不敢见他。若是薛蝌在家,他便抹粉施脂,描眉画鬓,奇情异致的打扮收拾起来,不时打从薛蝌住房前过,或故意咳嗽一声,或明知薛蝌在屋,特问房里何人。有时遇见薛蝌,他便妖妖乔乔,娇娇痴痴的问寒问热,忽喜忽嗔。丫头们看见,都赶忙躲开。他自己也不觉得,只是一意一心要弄得薛蝌感情时,好行宝蟾之计。那薛蝌却只躲着,有时遇见,也不敢不周旋一二,只怕他撒泼放刁的意思。更加金桂一则为色迷心,越瞧越爱,越想越幻,那里还看得出薛蝌的真假来。只有一宗,他见薛蝌有什么东西都是托香菱收着,衣服缝洗也是香菱,两个人偶然说话,他来了,急忙散开,一发动了一个醋字。欲待发作薛蝌,却是舍不得,只得将一腔隐恨都搁在香菱身上。却又恐怕闹了香菱得罪了薛蝌,倒弄得隐忍不发。
\end{parag}


\begin{parag}
    一日,宝蟾走来笑嘻嘻的向金桂道:“奶奶看见了二爷没有?”金桂道:“没有。”宝蟾笑道:“我说二爷的那种假正经是信不得的。咱们前日送了酒去,他说不会喝,刚才我见他到太太那屋里去,那脸上红扑扑儿的一脸酒气。奶奶不信,回来只在咱们院门口等他,他打那边过来时奶奶叫住他问问,看他说什么。”金桂听了,一心的怒气,便道:“他那里就出来了呢。他既无情义,问他作什么!”宝蟾道:“奶奶又迂了。他好说,咱们也好说,他不好说,咱们再另打主意。”金桂听着有理,因叫宝蟾瞧着他,看他出去了。宝蟾答应着出来。金桂却去打开镜奁,又照了一照,把嘴唇儿又抹了一抹,然后拿一条洒花绢子,才要出来,又似忘了什么的,心里倒不知怎么是好了。只听宝蟾外面说道:“二爷今日高兴呵,那里喝了酒来了?”金桂听了,明知是叫他出来的意思,连忙掀起帘子出来。只见薛蝌和宝蟾说道:“今日是张大爷的好日子,所以被他们强不过吃了半钟,到这时候脸还发烧呢。”一句话没说完,金桂早接口道:“自然人家外人的酒比咱们自己家里的酒是有趣儿的。”薛蝌被他拿话一激,脸越红了,连忙走过来陪笑道:“嫂子说那里的话。”宝蟾见他二人交谈,便躲到屋里去了。
\end{parag}


\begin{parag}
    这金桂初时原要假意发作薛蝌两句,无奈一见他两颊微红,双眸带涩,别有一种谨愿可怜之意,早把自己那骄悍之气感化到爪洼国去了,因笑说道:“这么说,你的酒是硬强着才肯喝的呢。”薛蝌道:“我那里喝得来。”金桂道:“不喝也好,强如象你哥哥喝出乱子来,明儿娶了你们奶奶儿,象我这样守活寡受孤单呢!”说到这里,两个眼已经乜斜了,两腮上也觉红晕了。薛蝌见这话越发邪僻了,打算着要走。金桂也看出来了,那里容得,早已走过来一把拉住。薛蝌急了道:“嫂子放尊重些。”说着浑身乱颤。金桂索性老着脸道:“你只管进来,我和你说一句要紧的话。”正闹着,忽听背后一个人叫道:“奶奶,香菱来了。”把金桂唬了一跳,回头瞧时,却是宝蟾掀着帘子看他二人的光景,一抬头见香菱从那边来了,赶忙知会金桂。金桂这一惊不小,手已松了。薛蝌得便脱身跑了。那香菱正走着,原不理会,忽听宝蟾一嚷,才瞧见金桂在那里拉住薛蝌往里死拽。香菱却唬的心头乱跳,自己连忙转身回去。这里金桂早已连吓带气,呆呆的瞅着薛蝌去了。怔了半天,恨了一声,自己扫兴归房,从此把香菱恨入骨髓。那香菱本是要到宝琴那里,刚走出腰门,看见这般,吓回去了。
\end{parag}


\begin{parag}
    是日,宝钗在贾母屋里听得王夫人告诉老太太要聘探春一事。贾母说道:“既是同乡的人,很好。只是听见那孩子到过我们家里,怎么你老爷没有提起?”王夫人道:“连我们也不知道。”贾母道:“好便好,但是道儿太远。虽然老爷在那里,倘或将来老爷调任,可不是我们孩子太单了吗。”王夫人道:“两家都是做官的,也是拿不定。或者那边还调进来,即不然,终有个叶落归根。况且老爷既在那里做官,上司已经说了,好意思不给么?想来老爷的主意定了,只是不做主,故遣人来回老太太的。”贾母道:“你们愿意更好。只是三丫头这一去了,不知三年两年那边可能回家?若再迟了,恐怕我赶不上再见他一面了。”说着,掉下泪来。王夫人道:“孩子们大了,少不得总要给人家的。就是本乡本土的人,除非不做官还使得,若是做官的,谁保得住总在一处。只要孩子们有造化就好。譬如迎姑娘倒配得近呢,偏是时常听见他被女婿打闹,甚至不给饭吃。就是我们送了东西去,他也摸不着。近来听见益发不好了,也不放他回来。两口子拌起来就说咱们使了他家的银钱。可怜这孩子总不得个出头的日子。前儿我惦记他,打发人去瞧他,迎丫头藏在耳房里不肯出来。老婆子们必要进去,看见我们姑娘这样冷天还穿着几件旧衣裳。他一包眼泪的告诉婆子们说:‘回去别说我这么苦,这也是命里所招,也不用送什么衣服东西来,不但摸不着,反要添一顿打。说是我告诉的。’老太太想想,这倒是近处眼见的,若不好更难受。倒亏了大太太也不理会他,大老爷也不出个头!如今迎姑娘实在比我们三等使唤的丫头还不如。我想探丫头虽不是我养的,老爷既看见过女婿,定然是好才许的。只请老太太示下,择个好日子,多派几个人送到他老爷任上。该怎么着,老爷也不肯将就。”贾母道:“有他老子作主,你就料理妥当,拣个长行的日子送去,也就定了一件事。”王夫人答应着“是”。宝钗听得明白,也不敢则声,只是心里叫苦:“我们家里姑娘们就算他是个尖儿,如今又要远嫁,眼看着这里的人一天少似一天了。”见王夫人起身告辞出去,他也送了出来,一径回到自己房中,并不与宝玉说话。见袭人独自一个做活,便将听见的话说了。袭人也很不受用。
\end{parag}


\begin{parag}
    却说赵姨娘听见探春这事,反欢喜起来,心里说道:“我这个丫头在家忒瞧不起我,我何从还是个娘,比他的丫头还不济。况且洑上水护着别人。他挡在头里,连环儿也不得出头。如今老爷接了去,我倒干净。想要他孝敬我,不能够了。只愿意他象迎丫头似的,我也称称愿。”一面想着,一面跑到探春那边与他道喜说:“姑娘,你是要高飞的人了,到了姑爷那边自然比家里还好。想来你也是愿意的。便是养了你一场,并没有借你的光儿。就是我有七分不好,也有三分的好,总不要一去了把我搁在脑杓子后头。”探春听着毫无道理,只低头作活,一句也不言语。赵姨娘见他不理,气忿忿的自己去了。
\end{parag}


\begin{parag}
    这里探春又气又笑,又伤心,也不过自己掉泪而已。坐了一回,闷闷的走到宝玉这边来。宝玉因问道:“三妹妹,我听见林妹妹死的时候你在那里来着。我还听见说,林妹妹死的时候远远的有音乐之声。或者他是有来历的也未可知。”探春笑道:“那是你心里想着罢了。只是那夜却怪,不似人家鼓乐之音。你的话或者也是。”宝玉听了,更以为实。又想前日自己神魂飘荡之时,曾见一人,说是黛玉生不同人,死不同鬼,必是那里的仙子临凡。忽又想起那年唱戏做的嫦娥,飘飘艳艳,何等风致。过了一回,探春去了。因必要紫鹃过来,立即回了贾母去叫他。无奈紫鹃心里不愿意,虽经贾母王夫人派了过来,也就没法,只是在宝玉跟前,不是嗳声,就是叹气的。宝玉背地里拉着他,低声下气要问黛玉的话,紫鹃从没好话回答。宝钗倒背底里夸他有忠心,并不嗔怪他。那雪雁虽是宝玉娶亲这夜出过力的,宝钗见他心地不甚明白,便回了贾母王夫人,将他配了一个小厮,各自过活去了。王奶妈养着他,将来好送黛玉的灵柩回南。鹦哥等小丫头仍伏侍了老太太。宝玉本想念黛玉,因此及彼,又想跟黛玉的人已经云散,更加纳闷。闷到无可如何,忽又想起黛玉死得这样清楚,必是离凡返仙去了,反又喜欢。忽然听见袭人和宝钗那里讲究探春出嫁之事,宝玉听了,啊呀的一声,哭倒在炕上。唬得宝钗袭人都来扶起说:“怎么了?”宝玉早哭的说不出来,定了一回子神,说道:“这日子过不得了!我姊妹们都一个一个的散了!林妹妹是成了仙去了。大姐姐呢已经死了,这也罢了,没天天在一块。二姐姐呢,碰着了一个混账不堪的东西。三妹妹又要远嫁,总不得见的了。史妹妹又不知要到那里去。薛妹妹是有了人家的。这些姐姐妹妹,难道一个都不留在家里,单留我做什么!”袭人忙又拿话解劝。宝钗摆着手说:“你不用劝他,让我来问他。”因问着宝玉道:“据你的心里,要这些姐妹都在家里陪到你老了,都不要为终身的事吗?若说别人,或者还有别的想头。你自己的姐姐妹妹,不用说没有远嫁的,就是有,老爷作主,你有什么法儿!打量天下独是你一个人爱姐姐妹妹呢,若是都象你,就连我也不能陪你了。大凡人念书,原为的是明理,怎么你益发糊涂了。这么说起来,我同袭姑娘各自一边儿去,让你把姐姐妹妹们都邀了来守着你。”宝玉听了,两只手拉住宝钗袭人道:“我也知道。为什么散的这么早呢?等我化了灰的时候再散也不迟。”袭人掩着他的嘴道:“又胡说。才这两天身上好些,二奶奶才吃些饭。若是你又闹翻了,我也不管了。”宝玉慢慢的听他两个人说话都有道理,只是心上不知道怎么才好,只得强说道:“我却明白,但只是心里闹的慌。”宝钗也不理他,暗叫袭人快把定心丸给他吃了,慢慢的开导他。袭人便欲告诉探春说临行不必来辞,宝钗道:“这怕什么。等消停几日,待他心里明白,还要叫他们多说句话儿呢。况且三姑娘是极明白的人,不象那些假惺惺的人,少不得有一番箴谏。他以后便不是这样了。”正说着,贾母那边打发过鸳鸯来说,知道宝玉旧病又发,叫袭人劝说安慰,叫他不要胡思乱想。袭人等应了。鸳鸯坐了一会子去了。那贾母又想起探春远行,虽不备妆奁,其一应动用之物俱该预备,便把凤姐叫来,将老爷的主意告诉了一遍,即叫他料理去。凤姐答应,不知怎么办理,下回分解。
\end{parag}