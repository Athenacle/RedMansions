\chap{一十五}{王凤姐弄权铁槛寺 秦鲸卿得趣馒头庵}


\begin{parag}
    \begin{note}甲:宝玉谒北静王辞对神色,方露出本来面目,迥非在闺阁中之形景。\end{note}
\end{parag}


\begin{parag}
    \begin{note}甲:北静王问玉上字果验否,政老对以未曾试过,是隐却多少捕风捉影闲文。\end{note}
\end{parag}


\begin{parag}
    \begin{note}甲:北静王论聪明伶俐,又年幼时为溺爱所累,亦大得病源之语。\end{note}
\end{parag}


\begin{parag}
    \begin{note}甲:凤姐中火,写纺线村姑,是宝玉闲花野景一得情趣。\end{note}
\end{parag}


\begin{parag}
    \begin{note}甲:凤姐另住,明明系秦、玉、智能幽事,却是为净虚钻营凤姐大大一件事作引。\end{note}
\end{parag}


\begin{parag}
    \begin{note}甲:秦、智幽情,忽写宝、秦事云:“不知算何账目,未见真切,不曾记得,此系疑案,不敢纂创。”是不落套中,且省却多少累赘笔墨。昔安南国使有题一丈红句云:“五尺墙头遮不得,留将一半与人看。”\end{note}
\end{parag}


\begin{parag}
    \begin{note}蒙:欲显铮铮不避嫌,英雄每入小人缘。鲸卿些子风流事,胆落魂销已可怜。\end{note}
\end{parag}


\begin{parag}
    话说宝玉举目见北静王水溶头上戴著洁白簪缨银翅王帽,穿著江牙海水五爪坐龙白蟒袍,系著碧玉红鞓带,面如美玉,目似明星,真好秀丽人物。宝玉忙抢上来参见,水溶连忙从轿内伸出手来挽住。见宝玉戴著束发银冠,勒著双龙出海抹额,穿著白蟒箭袖,围著攒珠银带,面若春花,目如点漆。\begin{note}甲侧:又换此一句,如见其形。\end{note}水溶笑道:“名不虚传,果然如‘宝’似‘玉’。”\begin{note}靖本眉:伤心笔。\end{note}因问:“衔的那宝贝在那里?”宝玉见问,连忙从衣内取了递与过去。水溶细细的看了,又念了那上头的字,因问:“果灵验否?”贾政忙道:“虽如此说,只是未曾试过。”水溶一面极口称奇道异,一面理好彩绦,亲自与宝玉带上,\begin{note}甲侧:钟爱之至。\end{note}又携手问宝玉几岁,读何书。宝玉一一的答应。
\end{parag}


\begin{parag}
    水溶见他语言清楚,谈吐有致,\begin{note}庚眉:八字道尽玉兄,如此等方是玉兄正文写照。壬午春。\end{note}一面又向贾政笑道:“令郎真乃龙驹凤雏,非小王在世翁前唐突,将来‘雏凤清于老凤声’,\begin{note}甲侧:妙极!开口便是西昆体,宝玉闻之,宁不刮目哉?\end{note}未可量也。”贾政忙陪笑道:“犬子岂敢谬承金奖。赖藩郡余祯 ,果如是言,亦荫生辈之幸矣。”\begin{note}庚侧:谦的得体。\end{note}水溶又道:“只是一件,令郎如是资质,想老太夫人、夫人辈自然钟爱极矣;但吾辈后生,甚不宜钟溺,钟溺则未免荒失学业。昔小王曾蹈此辙,想令郎亦未必不如是也。若令郎在家难以用功,不妨常到寒第。小王虽不才,却多蒙海上众名士凡至都者,未有不另垂青,是以寒第高人颇聚。令郎常去谈会谈会,则学问可以日进矣。”贾政忙躬身答应。
\end{parag}


\begin{parag}
    水溶又将腕上一串念珠卸了下来,递与宝玉道:“今日初会,伧促竟无敬贺之物,此系前日圣上亲赐鹡鸰香念珠一串,权为贺敬之礼。”宝玉连忙接了,回身奉与贾政。\begin{note}庚侧:转出没调教。\end{note}贾政与宝玉一齐谢过。于是贾赦、贾珍等一齐上来请回舆,水溶道:“逝者已登仙界,非碌碌你我尘寰中之人也。小王虽上叩天恩,虚邀郡袭,岂可越仙輀而进也?”贾赦等见执意不从,只得告辞谢恩回来,命手下掩乐停音,滔滔然将殡过完,\begin{note}庚侧:有层次,好看煞。\end{note}方让水溶回舆去了。不在话下。
\end{parag}


\begin{parag}
    且说宁府送殡,一路热闹非常。刚至城门前,又有贾赦、贾政、贾珍等诸同僚属下各家祭棚接祭,一一的谢过,然后出城,竟奔铁槛寺大路行来。彼时贾珍带贾蓉来到诸长辈前让坐轿上马,因而贾赦一辈的各自上了车轿,贾珍一辈的也将要上马。凤姐儿因记挂著宝玉,\begin{note}甲侧:千百件忙事内不漏一丝。\end{note}\begin{note}庚侧:细心人自应如是。\end{note}怕他在郊外纵性逞强,不服家人的话,贾政管不著这些小事,惟恐有个失闪,难见贾母,因此便命小厮来唤他。宝玉只得来到他车前。凤姐笑道:“好兄弟,你是个尊贵人,女孩儿一样的人品,\begin{note}甲侧:非此一句宝玉必不依,阿凤真好才情。\end{note}别学他们猴在马上。下来,咱们姐儿两个坐车,岂不好?”宝玉听说,忙下了马,爬入凤姐车上,二人说笑前进。
\end{parag}


\begin{parag}
    不一时,只见从那边两骑马压地飞来,\begin{note}庚侧:有气有声,有形有影。\end{note}离凤姐车不远,一齐蹿下来,扶车回说:“这里有下处,奶奶请歇更衣。”凤姐急命请邢夫人王夫人的示下,\begin{note}庚侧:有次序。\end{note}那人回来说:“太太们说不用歇了,叫奶奶自便罢。”凤姐听了,便命歇了再走。众小厮听了,一带辕马,岔出人群,往北飞走。宝玉在车内急命请秦相公。那时秦钟正骑马随著他父亲的轿,忽见宝玉的小厮跑来请他去打尖。秦钟看时,只见凤姐儿的车往北而去,后面拉著宝玉的马,搭著鞍笼,便知宝玉同凤姐坐车,自己也便带马赶上来,同入一庄门内。早有家人将众庄汉撵尽。那庄农人家无多房舍,婆娘们无处回避,只得由他们去了。那些村姑庄妇见了凤姐、宝玉、秦钟的人品衣服,礼数款段,岂有不爱看的?
\end{parag}


\begin{parag}
    一时凤姐进入茅堂,因命宝玉等先出去顽顽。宝玉等会意,因同秦钟出来,带著小厮们各处游顽。凡庄农动用之物,皆不曾见过。\begin{note}庚侧:真,毕真!\end{note}宝玉一见了锹、镢、锄、犁等物,皆以为奇,不知何项所使,其名为何。\begin{note}甲侧:凡膏粱子弟齐来著眼。\end{note}小厮在旁一一的告诉了名色,说明原委。\begin{note}甲侧:也盖因未见之故也。\end{note}宝玉听了,因点头叹道:“怪道古人诗上说:‘谁知盘中餐,粒粒皆辛苦。’正为此也。”\begin{note}甲侧:聪明人自是一喝即悟。\end{note}\begin{note}庚眉:写玉兄正文总于此等处,作者良苦。壬午季春。\end{note}一面说,一面又至一间房屋前,只见炕上有个纺车,宝玉又问小厮们:“这又是什么?”小厮们又告诉他原委。宝玉听说,便上来拧转作耍,自为有趣。只见一个约有十七八岁的村庄丫头跑了来乱嚷:“别动坏了!”\begin{note}庚侧:天生地设之文。\end{note}众小厮忙断喝拦阻,宝玉忙丢开手,陪笑说道:\begin{note}庚眉:一“忙”字,二“陪笑”字,写玉兄是在女儿分上。壬午季春。\end{note}“我因为没见过这个,所以试他一试。”那丫头道:“你们那里会弄这个,站开了,\begin{note}甲侧:如闻其声,见其形。\end{note}\begin{note}庚侧:三字如闻。\end{note}\begin{note}蒙侧:这丫头是技痒,是多情,是自己生活恐至损坏?宝玉此时一片心神,另有主张。\end{note}我纺与你瞧。”秦钟暗拉宝玉笑道:“此卿大有意趣。”\begin{note}庚侧:忙中闲笔;却伏下文。\end{note}宝玉一把推开,笑道:“该死的!\begin{note}甲侧:的是宝玉生性之言。\end{note}再胡说,我就打了!”\begin{note}庚侧:玉兄身分本心如此。\end{note}说著,只见那丫头纺起线来。宝玉正要说话时,\begin{note}庚眉:若说话,便不是《石头记》中文字也。\end{note}只听那边老婆子叫道:“二丫头,快过来!”那丫头听见,丢下纺车,一径去了。
\end{parag}


\begin{parag}
    宝玉怅然无趣。\begin{note}甲侧:处处点“情”,又伏下一段后文。\end{note}只见凤姐儿打发人来叫他两个进去。凤姐洗了手,换衣服抖灰,问他们换不换。宝玉不换,只得罢了。家下仆妇们将带著行路的茶壶茶杯、十锦屉盒、各样小食端来,凤姐等吃过茶,待他们收拾完备,便起身上车。外面旺儿预备下赏封,赏了那本村主人,庄妇等来叩赏。凤姐并不在意,宝玉却留心看时,内中并没有二丫头。\begin{note}庚侧:妙在不见。\end{note}一时上了车,出来走不多远,只见迎头二丫头怀里抱著他小兄弟,\begin{note}庚侧:妙在此时方见,错综之妙如此!\end{note}同著几个小女孩子说笑而来。宝玉恨不得下车跟了他去,料是众人不依的,少不得以目相送,争奈车轻马快,\begin{note}甲侧:四字有文章。人生离聚亦未尝不如此也。\end{note}一时展眼无踪。
\end{parag}


\begin{parag}
    走不多时,仍又跟上大殡了。早有前面法鼓金铙,幢幡宝盖:铁槛寺接灵众僧齐至。少时到入寺中,另演佛事,重设香坛。安灵于内殿偏室之中,宝珠安于里寝室相伴。外面贾珍款待一应亲友,也有扰饭的,也有不吃饭而辞的,一应谢过乏,从公侯伯子男一起一起的散去,至未末时分方才散尽了。里面的堂客皆是凤姐张罗接待,先从显官诰命散起,也到晌午大错时方散尽了。只有几个亲戚是至近的,等做过三日安灵道场方去。那时邢、王二夫人知凤姐必不能来家,也便就要进城。王夫人要带宝玉去,宝玉乍到郊外,那里肯回去,只要跟凤姐住著。王夫人无法,只得交与凤姐便回来了。
\end{parag}


\begin{parag}
    原来这铁槛寺原是宁荣二公当日修造,现今还是有香火地亩布施,以备京中老了人口,在此便宜寄放。其中阴阳两宅俱已预备妥贴,\begin{note}甲双夹:大凡创业之人,无有不为子孙深谋至细。奈后辈仗一时之荣显,犹为不足,另生枝叶,虽华丽过先,奈不常保,亦足可叹,争及先人之常保其朴哉!近世浮华子弟齐来著眼。\end{note}好为送灵人口寄居。\begin{note}甲侧:祖宗为子孙之心细到如此!\end{note}\begin{note}庚眉:《石头记》总于没要紧处闲三二笔,写正文筋骨。看官当用巨眼,不为被瞒过方好。壬午季春。\end{note}不想如今后辈人口繁盛,其中贫富不一,或性情参商,\begin{note}甲双夹:所谓“源远水则浊,枝繁果则稀”。余为天下痴心祖宗为子孙谋千年业者痛哭。\end{note}有那家业艰难安分的,\begin{note}甲侧:妙在艰难就安分,富贵则不安分矣。\end{note}便住在这里了;有那尚排场有钱势的,只说这里不方便,一定另外或村庄或尼庵寻个下处,为事毕宴退之所。\begin{note}甲侧:真真辜负祖宗体贴子孙之心。\end{note}即今秦氏之丧,族中诸人皆权在铁槛寺下榻,独有凤姐嫌不方便,\begin{note}甲侧:不用说,阿凤自然不肯将就一刻的。\end{note}因而早遣人来和馒头庵的姑子净虚说了,腾出两间房子来作下处。
\end{parag}


\begin{parag}
    原来这馒头庵就是水月庵,因他庙里做的馒头好,就起了这个浑号,离铁槛寺不远。\begin{note}甲双夹:前人诗云:“纵有千年铁门限,终须一个土馒头。”是此意。故“不远”二字有文章。\end{note}当下和尚工课已完,奠过晚茶,贾珍便命贾蓉请凤姐歇息。凤姐见还有几个妯娌们陪著女亲,自己便辞了众人,带著宝玉、秦钟往水月庵来。原来秦业年迈多病,\begin{note}甲侧:伏一笔。\end{note}不能在此,只命秦钟等待安灵罢了。那秦钟便只跟著凤姐、宝玉,一时到了水月庵,净虚带领智善、智能两个徒弟出来迎接,大家见过。凤姐等来至净室更衣净手毕,因见智能儿越发长高了,模样儿越发出息了,因说道:“你们师徒怎么这些日子也不往我们那里去?”净虚道:“可是这几天都没工夫,因胡老爷府里产了公子,太太送了十两银子来这里,叫请几位师父念三日《血盆经》,忙的没个空儿,就没来请奶奶的安。”\begin{note}甲侧:虚陪一个胡姓,妙!言是胡涂人之所为也。\end{note}
\end{parag}


\begin{parag}
    不言老尼陪著凤姐。且说秦钟、宝玉二人正在殿上顽耍,因见智能过来,宝玉笑道:“能儿来了。”秦钟道:“理那东西作什么?”宝玉笑道:“你别弄鬼,那一日在老太太屋里,一个人没有,你搂著他作什么呢?这会子还哄我。”\begin{note}甲侧:补出前文未到处,细思秦钟近日在荣府所为可知矣。\end{note}秦钟笑道:“这可是没有的话。”宝玉笑道:“有没有也不管你,你只叫他倒碗茶来我吃,就丢开手。”秦钟笑道:“这又奇了,你叫他倒去,还怕他不倒?何必要我说呢。”宝玉道: “我叫他倒的是无情意的,不及你叫他倒的是有情意的。”\begin{note}甲侧:总作如是等奇语。\end{note}秦钟只得说道:“能儿,倒碗茶来给我。”那智能儿自幼在荣府走动,无人不识,因常与宝玉秦钟顽笑。他如今大了,渐知风月,便看上了秦钟人物风流,那秦钟也极爱他妍媚,二人虽未上手,却已情投意合了。\begin{note}甲侧:不爱宝玉,却爱案钟,亦是各有情孽。\end{note}今智能见了秦钟,心眼俱开,走去倒了茶来。秦钟笑说:“给我。”\begin{note}甲侧:如闻其声。\end{note}宝玉叫:“给我!”智能儿抿著嘴笑道:“一碗茶也争,我难道手里有蜜!”\begin{note}甲侧:一语毕肖,如闻其语,观者已自酥倒,不知作者从何著想。\end{note}宝玉先抢得了,吃著,方要问话,只见智善来叫智能去摆茶碟子,一时来请他两个去吃茶果点心。他两个那里吃这些东西?坐一坐仍出来顽耍。
\end{parag}


\begin{parag}
    凤姐也略坐片时,便回至净室歇息,老尼相送。此时众婆娘媳妇见无事,都陆续散了,自去歇息,跟前不过几个心腹常服侍小婢,老尼便趁机说道:“我下有一事,要到府里求太太,先请奶奶一个示下。”凤姐因问何事。老尼道:“阿弥陀佛!\begin{note}甲侧:开口称佛,毕肖。可叹可笑!\end{note}只因当日我先在长安县内善才庵\begin{note}甲侧:“才”字妙。\end{note}内出家的时节,那时有个施主姓张,是大财主。他有个女儿小名金哥,\begin{note}甲侧:俱从“财”一字上发出。\end{note}那年都往我庙里来进香,不想遇见了长安府府太爷的小舅子李衙内。那李衙内一心看上,要娶金哥,打发人来求亲,不想金哥已受了原任守备的公子的聘定。张家若退亲,又怕守备不依,因此说已有了人家。谁知李公子执意不依,定要娶他女儿。张家正无计策,两处为难。不想守备家听了此信,也不管青红皂白,便来作践辱骂,说一个女孩儿许几家,偏不许退定礼,就打官司告状起来。\begin{note}甲双夹:守备一闻便问,断无此理。此必是张家惧府尹之势,必先退定礼,守备方不从,或有之。此时老尼,只欲与张家完事,故将此言遮饰,以便退亲,受张家之贿也。\end{note}那张家急了,\begin{note}甲双夹:如何便急了,话无头绪,可知张家理缺。此系作者巧摹老尼无头绪之语,莫认作者无头绪,正是神处奇处。摹一人,一人必到纸上活现。\end{note}只得著人上京来寻门路,赌气偏要退定礼。\begin{note}甲侧:如何?的是张家要与府尹攀亲!\end{note}我想如今长安节度云老爷与府上最契,可以求太太与老爷说声,打发一封书去,求云老爷和那守备说声,不怕那守备不依。若是肯行,张家连倾家孝顺,也都情愿。”\begin{note}甲双夹:坏极,妙极!若与府尹攀了亲,何惜张财不能再得?小人之心如此,良民遭害如此!\end{note}
\end{parag}


\begin{parag}
    凤姐听了笑道:“这事倒不大,\begin{note}甲侧:五字是阿凤心迹!\end{note}只是太太再不管这样的事。”老尼道:“太太不管,奶奶也可以主张了。”凤姐听说笑道:“我也不等银子使,也不做这样的事。”\begin{note}庚侧:口是心非,如闻已见。\end{note}净虚听了,打去妄想,半晌叹\begin{note}庚侧:一叹转出多少至恶不畏之文来。\end{note}道:“虽如此说,张家已知我来求府里,如今不管这事,张家不知道没工夫管这事,不希罕他的谢礼,倒像府里连这点子手段也没有的一般。”\begin{note}庚眉:闺阁营谋说事,往往被此等语惑了。\end{note}
\end{parag}


\begin{parag}
    凤姐听了这话,便发了兴头,说道:“你是素日知道我的,从来不信什么是阴司地狱报应的,\begin{note}庚侧:批书人深知卿有是心,叹叹!\end{note}凭是什么事,我说要行就行。你叫他拿三千银子来,我就替他出这口气。”老尼听说,喜不自禁,忙说:“有!有!这个不难。”凤姐又道:“我比不得他们扯篷拉纤的图银子。\begin{note}庚侧:欺人太甚。\end{note}这三千银子,不过是给打发说去的小厮作盘缠,使他赚几个辛苦钱,我一个钱也不要他的。\begin{note}庚眉:对如是之奸尼,阿凤不得不如是语。\end{note}便是三万两,我此刻也拿的出来。”\begin{note}甲侧:阿凤欺人如此。\end{note}老尼连忙答应,又说道:“既如此,奶奶明日就开恩也罢了。”凤姐道:“你瞧瞧我忙的,那一处少了我?既应了你,自然快快的了结。”老尼道:“这点子事,别人的跟前就忙的不知怎么样,若是奶奶的跟前,再添上些也不够奶奶一发挥的。\begin{note}蒙侧:“若是奶奶”等语,陷害杀无穷英明豪烈者。誉而不喜,毁而不怒,或可逃此等术法。\end{note}只是俗语说的‘能者多劳’,太太因大小事见奶奶妥贴,越发都推给奶奶了,奶奶也要保重金体才是。”一路话奉承的凤姐越发受用,也不顾劳乏,更攀谈起来。\begin{note}甲侧:总写阿凤聪明中的痴人。\end{note}
\end{parag}


\begin{parag}
    谁想秦钟趁黑无人,来寻智能。刚至后面房中,只见智能独在房中洗茶碗,秦钟跑来便搂著亲嘴。智能儿急的跺脚说:“这算什么!再这么我就叫唤。”秦钟求道:“好人,我已急死了。你今儿再不依,我就死在这里。”智能道:“你想怎样?除非我出了这牢坑,离了这些人,才依你。”秦钟道:“这也容易,只是远水救不得近渴。”说著,一口吹了灯,满屋漆黑,将智能抱到炕上,就云雨起来。\begin{note}庚侧:小风波事,亦在人意外。谁知为小秦伏线,大有根处。\end{note}\begin{note}庚眉:实表奸淫,尼庵之事如此。壬午季春。\end{note}\begin{note}庚批:又写秦钟智能事,尼庵之事如此。壬午季春。畸笏。\end{note}那智能百般的挣挫不起,又不好叫的,\begin{note}庚侧:还是不肯叫。\end{note}少不得依他了。正在得趣,只见一人进来,将他二人按住,也不则声。二人不知是谁,唬的不敢动一动。只听那人嗤的一声,掌不住笑了,\begin{note}庚侧:请掩卷细思此刻形景,真可喷饭。历来风月文字可有如此趣味者?\end{note}二人听声方知是宝玉。秦钟连忙起来,抱怨道:“这算什么?”宝玉笑道:“你倒不依,咱们就喊起来。”羞的智能趁黑地跑了。\begin{note}庚眉:若历写完,则不是《石头记》文字了,壬午季春。\end{note}宝玉拉了秦钟出来道:“你可还和我强?”\begin{note}蒙侧:请问此等光景,是强是顺?一片儿女之态,自与凡常不同。细极,妙极!\end{note}秦钟笑道:“好人,\begin{note}庚侧:前以二字称智能,今又称玉兄,看官细思。\end{note}你只别嚷的众人知道,你要怎样我都依你。”宝玉笑道:“这会子也不用说,等一会睡下,再细细的算帐。”一时宽衣要安歇的时节,凤姐在里间,秦钟宝玉在外间,满地下皆是家下婆子,打铺坐更。凤姐因怕通灵玉失落,便等宝玉睡下,命人拿来塞在自己枕边。宝玉不知与秦钟算何帐目,未见真切,未曾记得,此系疑案,不敢纂创。\begin{note}甲双夹:忽又作如此评断,似自相矛盾,却是最妙之文。若不如此隐去,则又有何妙文可写哉?这方是世人意料不到之大奇笔。若通部中万万件细微之事惧备,《石头记》真亦太觉死板矣。故特因此二三件隐事,指石之未见真切,淡淡隐去,越觉得云烟渺茫之中,无限丘壑在焉。\end{note}
\end{parag}


\begin{parag}
    一宿无话,至次日一早,便有贾母王夫人打发了人来看宝玉,又命多穿两件衣服,无事宁可回去。宝玉那里肯回去,又有秦钟恋著智能,调唆宝玉求凤姐再住一天。凤姐想了一想:\begin{note}甲侧:一想便有许多的好处。真好阿凤!\end{note}凡丧仪大事虽妥,还有一半点小事未曾安插,可以指此再住一日,岂不又在贾珍跟前送了满情;二则又可以完净虚那事;三则顺了宝玉的心,贾母听见,岂不欢喜?因有此三益,\begin{note}甲侧:世人只云一举两得,独阿凤一举更添一。\end{note}便向宝玉道:“我的事都完了,你要在这里逛,少不得索性辛苦一日罢了,明儿可是定要走的了。”宝玉听说,千姐姐万姐姐的央求:“只住一日,明儿回去的。”于是又住了一夜。
\end{parag}


\begin{parag}
    凤姐便命悄悄将昨日老尼之事,说与来旺儿。来旺儿心中俱已明白,急忙进城找著主文的相公,假托贾琏所嘱,修书一封,\begin{note}甲侧:不细。\end{note}连夜往长安县来,不过百里路程,两日工夫俱已妥协。那节度使名唤云光,久受贾府之情,这点小事,岂有不允之理,给了回书,旺儿回来。且不在话下。\begin{note}甲侧:一语过下。\end{note}
\end{parag}


\begin{parag}
    却说凤姐等又过了一日,次日方别了老尼,著他三日后往府里去讨信。\begin{note}甲侧:过至下回。\end{note}那秦钟与智能百般不忍分离,背地里多少幽期密约,俱不用细述,只得含恨而别。凤姐又到铁槛寺中照望一番。宝珠执意不肯回家,贾珍只得派妇女相伴。后回再见。
\end{parag}


\begin{parag}
    \begin{note}蒙:请看作者写势利之情,亦必因激动;写儿女之情,偏生含蓄不吐,可谓细针密缝。其述说一段,言语形迹无不逼真,圣手神文,敢不熏沐拜读?\end{note}
\end{parag}