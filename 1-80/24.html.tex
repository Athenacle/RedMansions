\chap{二十四}{醉金刚轻财尚义侠 痴女儿遗帕惹相思}


\begin{parag}
    \begin{note}庚:夹写“醉金刚”一回是书中之大净场,聊醒看官倦眼耳。然亦书中必不可少之文,必不可少之人。今写在市井俗人身上,又加一“侠”字,则大有深意存焉。\end{note}
\end{parag}


\begin{parag}
    \begin{note}蒙回前总:夹写醉金刚一回,是处中之大文字,聊醒看官倦眠而,然亦书中之必不可少之文字,必不可少之人,今写在市井俗人身上,加一“侠”字,则有大深意存焉。\end{note}
\end{parag}


\begin{parag}
    \begin{note}靖:“醉金刚”一回文字,伏芸哥仗义探庵。余三十年来得遇金刚之样人不少,不及金刚者亦不少。惜不便一一注明耳。壬午孟夏。\end{note}
\end{parag}


\begin{parag}
    话说林黛玉正自情思萦逗,缠绵固结之时,忽有人从背后击了一掌,说道:“你作什么一个人在这里?”林黛玉倒唬了一跳,回头看时,不是别人,却是香菱。林黛玉道:“你这个傻\begin{note}庚侧:此“傻”字加于香菱,则有多少丰神跳于纸上,其娇憨之态可想而知。\end{note}丫头,唬我这么一跳好的。你这会子打那里来?”香菱嘻嘻的笑道:“我来寻我们的姑娘的,找他总找不著。你们紫鹃也找你呢,\begin{note}庚侧:一丝不漏。\end{note}说琏二奶奶送了什么茶叶来给你的。走罢,回家去坐著。”\begin{note}庚侧:“回家去坐著”之言,是恐石上冷意。\end{note}一面说著,一面拉著黛玉的手回潇湘馆来了。果然凤姐儿送了两小瓶上用新茶来。林黛玉和香菱坐了。况他们有甚正事谈讲。\begin{note}庚侧:为学诗伏线。\end{note}不过说些这一个绣的好,那一个刺的精,又下一回棋,看两句书,\begin{note}庚双夹:棋不论盘,书不论章,皆是娇憨女儿神理,写得不即不离,似有似无,妙极!\end{note}香菱便走了。不在话下。\begin{note}庚眉:是书最好看如此等处,系画家山水树 非褊志惚 ,末用浓淡墨点苔法也。亥夏。畸笏叟。\end{note}
\end{parag}


\begin{parag}
    如今且说宝玉因被袭人找回房去,果见鸳鸯歪在床上看袭人的针线呢,见宝玉来了,便说道:“你往那里去了?老太太等著你呢,叫你过那边请大老爷的安去。还不快换了衣服走呢。”袭人便进房去取衣服。宝玉坐在床沿上,褪了鞋等靴子穿的工夫,回头见鸳鸯穿著水红绫子袄儿,青缎子背心,束著白绉绸汗巾儿,脸向那边低著头看针线,脖子上戴著花领子。宝玉便把脸凑在他脖项上,闻那香油气,不住用手摩挲,其白腻不在袭人之下,便猴上身去涎皮笑道:“好姐姐,把你嘴上的胭脂赏我吃了罢。”\begin{note}庚侧:胭脂是这样吃法。看官可经过否?\end{note}一面说著,一面扭股糖似的粘在身上。
\end{parag}


\begin{parag}
    鸳鸯便叫道:“袭人,你出来瞧瞧。\begin{note}庚侧:不向宝玉说话,又叫袭人,鸳鸯亦是幻情洞天也。\end{note}你跟他一辈子,也不劝劝,还是这么著。”袭人抱了衣服出来,向宝玉道:“左劝也不改,右劝也不改,你到底是怎么样?你再这么著,\begin{note}庚侧:此五字内有深意深心。\end{note}这个地方可就难住了。”一边说,一边催他穿了衣服,同鸳鸯往前面来见贾母。见过贾母,出至外面,人马俱已齐备。刚欲上马,只见贾琏请安回来了,\begin{note}庚侧:一丝不漏。\end{note}正下马,二人对面,彼此问了两句话。只见旁边转出一个人来,\begin{note}庚侧:芸哥此处一现,后文不见突然。\end{note}“请宝叔安”。宝玉看时,只见这人容长脸,长挑身材,年纪只好十八九岁,生得著实斯文清秀,倒也十分面善,只是想不起是那一房的,\begin{note}庚侧:大族人众,毕真,有是理。\end{note}叫什么名字。贾琏笑道:“你怎么发呆,连他也不认得?他是后廊上住的五嫂子的儿子芸儿。”宝玉笑道:“是了,是了,我怎么就忘了。”因问他母亲好,这会子什么勾当。贾芸指贾琏道:“找二叔说句话。”宝玉笑道:“你倒比先越发出挑了,\begin{note}庚侧:何尝是十二三岁小孩语。\end{note}倒象我的儿子。”贾琏笑道:“好不害臊!人家比你大四五岁呢,就替你作儿子了?”宝玉笑道:“你今年十几岁了?”贾芸道:“十八岁。”
\end{parag}


\begin{parag}
    原来这贾芸最伶俐乖觉,听宝玉这样说,便笑道:“俗语说的,‘摇车里的爷爷,拄拐的孙孙’。虽然岁数大,山高高不过太阳。只从我父亲没了,这几年也无人照管教导。\begin{note}庚侧:虽是随机而应,伶俐人之语,余却伤心。\end{note}如若宝叔不嫌侄儿蠢笨,认作儿子,就是我的造化了。”贾琏笑道:“你听见了?认儿子不是好开交的呢。”\begin{note}庚侧:是兄凑弟趣,可叹!\end{note}说著就进去了。宝玉笑道:“明儿你闲了,只管来找我,别和他们鬼鬼祟祟的。\begin{note}庚侧:何其堂皇正大之语。\end{note}这会子我不得闲儿。明儿你到书房里来,和你说天话儿,我带你园里顽耍去。”说著扳鞍上马,众小厮围随往贾赦这边来。
\end{parag}


\begin{parag}
    见了贾赦,不过是偶感些风寒,先述了贾母问的话,然后自己请了安。贾赦先站起来回了贾母话,\begin{note}庚侧:一丝不乱。\end{note}次后便唤人来:“带哥儿进去太太屋里坐著。”宝玉退出,来至后面,进入上房。邢夫人见了他来,先倒站了起来请过贾母安,\begin{note}庚侧:一丝不乱。\end{note}宝玉方请安。\begin{note}[好规矩。]\end{note}邢夫人拉他上炕坐了,方问别人好,又命人倒茶来。\begin{note}庚侧:好层次,好礼法,谁家故事?\end{note}一钟茶未吃完,只见那贾琮来问宝玉好。邢夫人道:“那里找活猴儿去!你那奶妈子死绝了,也不收拾收拾你,弄的黑眉乌嘴的,那里象大家子念书的孩子!”
\end{parag}


\begin{parag}
    正说著,只见贾环、贾兰小叔侄两个也来了,请过安,邢夫人便叫他两个椅子上坐了。贾环见宝玉同邢夫人坐在一个坐褥上,邢夫人又百般摩挲抚弄他,早已心中不自在了,\begin{note}庚侧:千里伏线。\end{note}坐不多时,便和贾兰使眼色儿要走。贾兰只得依他,一同起身告辞。宝玉见他们要走,自己也就起身,要一同回去。邢夫人笑道:“你且坐著,我还和你说话呢。”宝玉只得坐了。邢夫人向他两个道:“你们回去,各人替我问你们各人母亲好。你们姑娘、姐姐妹妹都在这里呢,闹的我头晕,今儿不留你们吃饭了。”\begin{note}庚侧:明显薄情之至。\end{note}贾环等答应著,便出来回家去了。
\end{parag}


\begin{parag}
    宝玉笑道:“可是姐姐们都过来了,怎么不见?”邢夫人道:“他们坐了一会子,都往后头不知那屋里去了。”宝玉道:“大娘方才说有话说,不知是什么话?” 邢夫人笑道:“那里有什么话,不过是叫你等著,同你姊妹们吃了饭去。还有一个好玩的东西给你带回去玩。”娘儿两个说话,不觉早又晚饭时节。调开桌椅,罗列杯盘,母女姊妹们吃毕了饭。宝玉去辞贾赦,同姊妹们一同回家,见过贾母,王夫人等,各自回房安息。不在话下。\begin{note}庚双夹:逐步一段为五鬼魇魔法作引。脂砚。\end{note}
\end{parag}


\begin{parag}
    且说贾芸进去见了贾琏,因打听可有什么事情。贾琏告诉他:“前儿倒有一件事情出来,偏生你婶子再三求了我,\begin{note}庚侧:反说体面话,惧内人累累如是。\end{note}给了贾芹了。他许了我,说明儿园里还有几处要栽花木的地方,等这个工程出来,一定给你就是了。”贾芸听了,半晌说道:“既是这样,我就等著罢。叔叔也不必先在婶子跟前提我今儿来打听的话,\begin{note}庚侧:已得了主意了。\end{note}到跟前再说也不迟。”贾琏道:“提他作什么,\begin{note}庚侧:已被芸哥瞒过了。\end{note}我那里有这些工夫说闲话儿呢。明儿一个五更,还要到兴邑去走一趟,须得当日赶回来才好。你先去等著,后日起更以后你来讨信儿,来早了我不得闲。”说著便回后面换衣服去了。
\end{parag}


\begin{parag}
    贾芸出了荣国府回家,一路思量,想出一个主意来,便一径往他母舅卜世仁家来。\begin{note}庚侧:既云“不是人”,如何肯共事?想芸哥此来空了。\end{note}原来卜世仁现开香料铺,方才从铺子里来,忽见贾芸进来,彼此见过了,因问他这早晚什么事跑了来。贾芸道:“有件事求舅舅帮衬帮衬。我有一件事,用些冰片麝香使用,好舅舅每样赊四两给我,八月里按数送了银子来 ”\begin{note}庚双夹:甥舅之谈如此,叹叹!\end{note}卜世仁冷笑道:“再休提赊欠一事。\begin{note}庚侧:何如,何如?余言不谬。\end{note}前儿也是我们铺子里一个伙计,替他的亲戚赊了几两银子的货,至今总未还上。因此我们大家赔上,立了合同,再不许替亲友赊欠。谁要赊欠,就要罚他二十两银子的东道。况且如今这个货也短,你就拿现银子到我们这不三不四的铺子里来买,\begin{note}庚侧:推脱之辞。\end{note}也还没有这些,只好倒扁儿去。这是一。二则你那里有正经事,不过赊了去又是胡闹。你只说舅舅见你一遭儿就派你一遭儿不是。你小人儿家很不知好歹,也到底立个主见,赚几个钱,弄得穿是穿吃是吃的,我看著也喜欢。”
\end{parag}


\begin{parag}
    贾芸笑道:“舅舅说的倒干净。我父亲没的时候,我年纪又小,不知事。后来听见我母亲说,都还亏舅舅们在我们家出主意,料理的丧事。难道舅舅就不知道的,还是有一亩地两间房子,如今在我手里花了不成?巧媳妇做不出没米的粥来,叫我怎么样呢?还亏是我呢,要是别个,死皮赖脸三日两头儿来缠著舅舅,\begin{note}庚侧:芸哥亦善谈,井井有理。\end{note}要三升米二升豆子的,\begin{note}庚侧:余二人亦不曾有是气?\end{note}舅舅也就没有法呢。”
\end{parag}


\begin{parag}
    卜世仁道:“我的儿,舅舅要有,还不是该的。我天天和你舅母说,只愁你没算计儿。你但凡立的起来,到你大房里,就是他们爷儿们见不著,便下个气,和他们的管家或者管事的人们嬉和嬉和,\begin{note}庚侧:可怜可叹,余竟为之一哭。\end{note}也弄个事儿管管。前日我出城去,撞见了你们三房里的老四,骑著大叫驴,带著五辆车,有四五十和尚道士,\begin{note}庚双夹:妙极!写小人口角,羡慕之言加一倍,毕肖。却又是背面傅粉法。\end{note}往家庙去了。他那不亏能干,这事就到他了!”贾芸听他韶刀的不堪,便起身告辞。\begin{note}庚侧:有志气,有果断。\end{note}卜世仁道:“怎么急的这样,吃了饭再去罢。”一句未完,只见他娘子说道:“你又糊涂了。\begin{note}庚侧:虽写小人家涩细,一吹一唱,酷肖之至,却是一气逼出,后文方不突然。《石头记》笔仗全在如此样者。\end{note}说著没有米,这里买了半斤面来下给你吃,这会子还装胖呢。留下外甥挨饿不成?”卜世仁说:“再买半斤来添上就是了。”他娘子便叫女孩儿:“银姐,往对门王奶奶家去问,有钱借二三十个,明儿就送过来。”夫妻两个说话,那贾芸早说了几个“不用费事”,去的无影无踪了。\begin{note}庚侧:有知识有果断人,自是不同。\end{note}
\end{parag}


\begin{parag}
    不言卜家夫妇,且说贾芸赌气离了母舅家门,一径回归旧路,心下正自烦恼,一边想,一边低头只管走,不想一头就碰在一个醉汉身上,把贾芸唬了一跳。\begin{note}庚批:自上看来,可是一口气否?\end{note}听醉汉骂道:“臊你娘的!瞎了眼睛,碰起我来了。”贾芸忙要躲身,早被那醉汉一把抓住,对面一看,不是别人,却是紧邻倪二。原来这倪二是个泼皮,专放重利债,在赌博场吃闲钱,专管打降吃酒。如今正从欠钱人家索了利钱,吃醉回来,不想被贾芸碰了一头,正没好气,抡拳就要打。\begin{note}庚眉:这一节对《水浒》杨志卖大刀遇没毛大虫一回看,觉好看多矣。己冬夜。脂砚。\end{note}只听那人叫道:“老二住手!是我冲撞了你。”倪二听见是熟人的语音,将醉眼睁开看时,见是贾芸,忙把手松了,趔趄著笑道:\begin{note}庚侧:写生之笔。\end{note}“原来是贾二爷,\begin{note}庚侧:如此称呼,可知芸哥素日行止,是“金盆虽破分量在”也。\end{note}我该死,我该死。这会子往那里去?”贾芸道:“告诉不得你,平白的又讨了个没趣儿。”\begin{note}庚侧:本无心之谈也。\end{note}倪二道: “不妨不妨,\begin{note}庚侧:如闻。\end{note}有什么不平的事,告诉我,替你出气。\begin{note}庚侧:写得酷肖,总是渐次逼出,不见一丝勉强。\end{note}这三街六巷,凭他是谁,有人得罪了我醉金刚倪二的街坊,管叫他人离家散!”贾芸道:“老二,你且别气,听我告诉你这原故。”\begin{note}庚侧:可是一顺而来?\end{note}说著,便把卜世仁一段事告诉了倪二。倪二听了大怒,“要不是令舅,我便骂不出好话来,\begin{note}庚侧:仗义人岂有不知礼者乎?何尝是破落户?冤杀金刚了。\end{note}真真气死我倪二。也罢,你也不用愁烦,我这里现有几两银子,你若用什么,只管拿去买办。但只一件,你我作了这些年的街坊,我在外头有名放帐,你却从没有和我张过口。也不知你厌恶我是个泼皮,\begin{note}庚侧:知己知彼之话。\end{note}怕低了你的身分,也不知是你怕我难缠,利钱重?若说怕利钱重,这银子我是不要利钱的,也不用写文约,若说怕低了你的身分,\begin{note}庚侧:知己知彼之话。\end{note}我就不敢借给你了,各自走开。”一面说,一面果然从搭包里掏出一卷银子来。
\end{parag}


\begin{parag}
    贾芸心下自思:“素日倪二虽然是泼皮无赖,却因人而使,\begin{note}庚侧:四字是评,难得难得,非豪杰不可当。\end{note}颇颇的有义侠之名。若今日不领他这情,怕他臊了,倒恐生事。不如借了他的,改日加倍还他也倒罢了。”想毕笑道:“老二,你果然是个好汉,我何曾不想著你,和你张口。但只是我见你所相与交结的,都是些有胆量的有作为的人,似我们这等无能无力的你倒不理。\begin{note}庚侧:芸哥亦善谈,好口齿。\end{note}我若和你张口,你岂肯借给我。今日既蒙高情,我怎敢不领,回家按例写了文约过来便是了。”倪二大笑道:“好会说话的人。我却听不上这话。\begin{note}庚侧:“光棍眼内揉不下沙子”是也。\end{note}既说‘相与交结’四个字,如何放帐给他,使他的利钱!\begin{note}庚侧:如今不单是亲友言利,不但亲友,即闺阁中亦然,不但生意新发户,即大户旧族颇颇有之。\end{note}既把银子借与他,图他的利钱,便不是相与交结了。闲话也不必讲。既肯青目,这是十五两三钱有零的银子,便拿去治买东西。你要写什么文契,趁早把银子还我,让我放给那些有指望的人使去。”\begin{note}庚侧:爽快人,爽快语。\end{note}贾芸听了,一面接了银子,一面笑道:“我便不写罢了,有何著急的。”倪二笑道:“这不是话。天气黑了,也不让茶让酒,我还到那边有点事情去,你竟请回去。我还求你带个信儿与舍下,叫他们早些关门睡罢,我不回家去了,倘或有要紧事儿,叫我们女儿明儿一早到马贩子王短腿家\begin{note}庚侧:常起坐处人,毕真。\end{note}来找我。”一面说,一面趔趄著脚儿去了,\begin{note}庚侧:仍应前。\end{note}不在话下。\begin{note}庚眉:读阅“醉金刚”一回,务吃刘铉丹家山楂丸一付,一笑。余卅年来得遇金刚之样人不少,不及金刚者亦不少,惜书上不便历历注上芳讳,是余不是心事也。壬午孟夏。\end{note}
\end{parag}


\begin{parag}
    且说贾芸偶然碰了这件事,心中也十分罕希,想那倪二倒果然有些意思,只是还怕他一时醉中慷慨,到明日加倍的要起来,便怎处,心内犹豫不决。\begin{note}庚侧:芸哥实怕倪二,并非以小人之心度君子也。\end{note}忽又想道:“不妨,等那件事成了,也可加倍还他。”想毕,一直走到个钱铺里,将那银子称一称,十五两三钱四分二厘。贾芸见倪二不撒谎,心下越发欢喜,收了银子,来至家门,先到隔壁将倪二的信捎了与他娘子知道,方回家来。见他母亲自在炕上拈线,见他进来,便问那去了一日。贾芸恐他母亲生气,便不说起卜世仁的事来,\begin{note}庚侧:孝子可敬。此人后来荣府事败,必有一番作为。\end{note}\begin{note}该批:果然。\end{note}只说在西府里等琏二叔的,问他母亲吃了饭不曾。他母亲已吃过了,说留的饭在那里。小丫头子拿过来与他吃。
\end{parag}


\begin{parag}
    那天已是掌灯时候,贾芸吃了饭收拾歇息,一宿无话。次日一早起来,洗了脸,便出南门,大香铺里买了冰麝,便往荣国府来。打听贾琏出了门,贾芸便往后面来。
\end{parag}


\begin{parag}
    到贾琏院门前,只见几个小厮拿著大高笤帚在那里扫院子呢。忽见周瑞家的从门里出来叫小厮们:“先别扫,奶奶出来了。”贾芸忙上前笑问:“二婶婶那去?”周瑞家的道:“老太太叫,想必是裁什么尺头。”正说著,只见一群人簇著凤姐出来了。\begin{note}庚侧:当家人有是派头。\end{note}贾芸深知凤姐是喜奉承尚排场的,\begin{note}庚侧:那一个不喜奉承。\end{note}忙把手逼著,恭恭敬敬抢上来请安。凤姐连正眼也不看,仍往前走著,只问他母亲好,“怎么不来我们这里逛逛?”贾芸道:“只是身上不大好,倒时常记挂著婶子,要来瞧瞧,又不能来。”凤姐笑道:“可是会撒谎,不是我提起他来,你就不说他想我了。”贾芸笑道:“侄儿不怕雷打了,就敢在长辈前撒谎。昨儿晚上还提起婶子来,说婶子身子生的单弱,事情又多,亏婶子好大精神,竟料理的周周全全,要是差一点儿的,早累的不知怎么样呢。”\begin{note}庚眉:自往卜世仁处去已安排下的。芸哥可用。己冬夜。\end{note}
\end{parag}


\begin{parag}
    凤姐听了满脸是笑,不由的便止了步,问道:“怎么好好的你娘儿们在背地里嚼起我来?”\begin{note}庚侧:过下无痕,天然而来文字。\end{note}贾芸道:“有个原故,\begin{note}庚侧:接得如何?\end{note}只因我有个朋友,家里有几个钱,现开香铺。只因他身上捐著个通判,前儿选了云南不知那一处,\begin{note}庚侧:随口语,极妙!\end{note}连家眷一齐去,把这香铺也不在这里开了。便把帐物攒了一攒,该给人的给人,该贱发的贱发了,\begin{note}蒙侧:世法人情,随便招来,皆是奇妙文章。\end{note}象这细贵的货,都分著送与亲朋。他就一共送了我些冰片,麝香。我就和我母亲商量,\begin{note}庚侧:像得紧,何尝撒谎?\end{note}若要转买,不但卖不出原价来,而且谁家拿这些银子买这个作什么,便是很有钱的大家子,也不过使个几分几钱就挺折腰了,若说送人,也没个人配使这些,\begin{note}蒙侧:作者是何神圣,具此等大光明眼,无微不照?\end{note}倒叫他一文不值半文转卖了。因此我就想起婶子来。往年间我还见婶子大包的银子买这些东西呢,别说今年贵妃宫中,就是这个端阳节下,不用说这些香料自然是比往常加上十倍去的。因此想来想去,只孝顺婶子一个人才合式,方不算遭塌这东西。”一边说,一边将一个锦匣举起来。
\end{parag}


\begin{parag}
    凤姐正是要办端阳的节礼,采买香料药饵的时节,忽见贾芸如此一来,听这一篇话,心下又是得意又是欢喜,便命丰儿:“接过芸哥儿的来,\begin{note}庚侧:像个婶子口气,好看杀!\end{note}送了家去,交给平儿。”因又说道:“看著你这样知好歹,怪道你叔叔常提你,说你说话儿也明白,心里有见识。”\begin{note}庚双夹:看官须记,凤姐所喜是奉承之言,打动了心,不是见物而欢喜,若说是见物而喜,便不是阿凤矣。\end{note}贾芸听这话入了港,便打进一步来,故意问道:“原来叔叔也曾提我的?”凤姐见问,才要告诉他与他管事情的那话,便忙又止住,心下想道:\begin{note}庚侧:的是阿凤行事心机笔意。\end{note}“我如今要告诉他那话,倒叫他看著我见不得东西似的,为得了这点子香,就混许他管事了。今儿先别提起这事。”想毕,便把派他监种花木工程的事都隐瞒的一字不提,随口说了两句淡话,便往贾母那里去了。贾芸也不好提的,只得回来。
\end{parag}


\begin{parag}
    因昨日见了宝玉,叫他到外书房等著,贾芸吃了饭便又进来,到贾母那边仪门外绮霰斋书房里来。只见焙茗,锄药两个小厮下象棋,为夺“车”正拌嘴,还有引泉、扫花、\begin{note}庚侧:好名色。\end{note}挑云、伴鹤四五个,又在房檐上掏小雀儿玩。
\end{parag}


\begin{parag}
    贾芸进入院内,把脚一跺,说道:“猴头们淘气,我来了。”众小厮看见贾芸进来,都才散了。贾芸进入房内,便坐在椅子上问:“宝二爷没下来?”焙茗道: “今儿总没下来。二爷说什么,我替你哨探哨探去。”\begin{note}庚侧:五遁之外,名曰“哨探遁”法。\end{note}说著,便出去了。这里贾芸便看字画古玩,有一顿饭工夫还不见来,再看看别的小厮,都顽去了。正是烦闷,只听门前娇声嫩语的叫了一声“哥哥”。
\end{parag}


\begin{parag}
    贾芸往外瞧时,看是一个十六七岁的丫头,生的倒也细巧干净。那丫头见了贾芸,便抽身躲了过去。恰当很走来,见那丫头在门前,便说道:“好,好,\begin{note}庚侧:二“好”字是遮饰半句来不到语。\end{note}正抓不著个信儿。”贾芸见了焙茗,也就赶了出来,问怎么样。焙茗道:“等了这一日,也没个人儿过来。这就是宝二爷房里的。好姑娘,\begin{note}庚侧:口气极像。\end{note}你进去带个信儿,就说廊上的二爷来了。”那丫头听说,方知是本家的爷们,便不似先前那等回避,\begin{note}庚侧:一句,礼当。\end{note}下死眼把贾芸钉了两眼。\begin{note}庚侧:这句是情孽上生。\end{note}听那贾芸说道:“什么是廊上廊下的,你只说是芸儿就是了。”半晌,那丫头冷笑了一笑:\begin{note}庚侧:神情是深知房中事的。\end{note}“依我说,二爷竟请回家去,有什么话明儿再来。今儿晚上得空儿我回了他。”焙茗道:“这是怎么说?”那丫头道:“他\begin{note}庚侧:一连两个“他”字,怡红院中使得,否则有假矣。\end{note}今儿也没睡中觉,自然吃的晚饭早。晚上他又不下来。难道只是耍的二爷在这里等著挨饿不成!不如家去,明儿来是正经。便是回来有人带信,那都是不中用的。他不过口里应著,他倒给带呢!”贾芸听这丫头说话简便俏丽,待要问他的名字,因是宝玉房里的,又不便问,只得说道:“这话倒是,我明儿再来。”说著便往外走。焙茗道:“我倒茶去,\begin{note}庚侧:滑贼。\end{note}二爷吃了茶再去。”贾芸一面走,一面回头说: “不吃茶,我还有事呢。”口里说话,眼睛瞧那丫头还站在那里呢。
\end{parag}


\begin{parag}
    那贾芸一径回家。至次日来至大门前,可巧遇见凤姐往那边去请安,才上了车,见贾芸来,便命人唤住,隔窗子笑道:“芸儿,你竟有胆子在我的跟前弄鬼。\begin{note}庚侧:也作得不像撒谎,用心机人可怕是此等处。\end{note}道你送东西给我,原来你有事求我。昨儿你叔叔才告诉我说你求他。”贾芸笑道:“求叔叔这事,婶子休提,我昨儿正后悔呢。早知这样,我竟一起头求婶子,这会子也早完了。谁承望叔叔竟不能的。”凤姐笑道:“怪道你那里没成儿,昨儿又来寻我。”贾芸道:“婶子辜负了我的孝心,我并没有这个意思。若有这个意思,昨儿还不求婶子。如今婶子既知道了,我倒要把叔叔丢下,少不得求婶子好歹疼我一点儿。”凤姐冷笑道:“你们要拣远路儿走,叫我也难说。\begin{note}庚侧:曹操语。\end{note}早告诉我一声儿,有什么不成的,多大点子事,耽误到这会子。那园子里还要种花,我只想不出一个人来,你早来不早完了。”贾芸笑道:“既这样,婶子明儿就派我罢。”凤姐半晌道:“这个我看著不大好。\begin{note}庚侧:又一折。\end{note}等明年正月里烟火灯烛那个大宗儿下来,再派你罢。”贾芸道:“好婶子,先把这个派了我罢。果然这个办的好,再派我那个。”凤姐笑道:“你倒会拉长线儿。罢了,要不是你叔叔说,我不管你的事。\begin{note}庚侧:总不认受冰麝贿。\end{note}我也不过吃了饭就过来,你到午错的时候来领银子,后儿就进去种树。”说毕,令人驾起香车,一径去了。
\end{parag}


\begin{parag}
    贾芸喜不自禁,来至绮霰斋打听宝玉,谁知宝玉一早便往北静王府里去了。贾芸便呆呆的坐到晌午,打听凤姐回来,便写个领票来领对牌。至院外,命人通报了,彩明走了出来,单要了领票进去,批了银数年月,一并连对牌交与了贾芸。贾芸接了,看那批上银数批了二百两,心中喜不自禁,翻身走到银库上,交与收牌票的,领了银子。回家告诉母亲,自是母子俱各欢喜。次日一个五鼓,贾芸先找了倪二,将前银按数还他。那倪二见贾芸有了银子,他便按数收回,不在话下。这里贾芸又拿了五十两,出西门找到花儿匠方椿家里去买树,不在话下。\begin{note}庚双夹:至此便完种树工程。一者见得趱赶工程原非正文,不过虚描盛时光景,借此以出情文。二者又为避难法。若不如此了,必曰其树其价怎么,买定必株,岂不烦絮矣?\end{note}
\end{parag}


\begin{parag}
    如今且说宝玉,自那日见了贾芸,曾说明日著他进来说话儿。如此说了之后,他原是富贵公子的口角,那里还把这个放在心上,因而便忘怀了。\begin{note}庚侧:若是一个女孩子,可保不忘的。\end{note}这日晚上,从北静王府里回来,见过贾母,王夫人等,回至园内,换了衣服,正要洗澡。袭人因被薛宝钗烦了去打结子,秋纹,碧痕两个去催水,檀云又因他母亲的生日接了出去,麝月又现在家中养病,虽还有几个作粗活听唤的丫头,估著叫不著他们,都出去寻伙觅伴的玩去了。不想这一刻的工夫,\begin{note}庚双夹:妙!必用“一刻”二字方是宝玉的房中,见得时时原有人的,又有今一刻无人,所谓凑巧其一也。\end{note}只剩了宝玉在房内。偏生的\begin{note}庚双夹:三字不可少。\end{note}宝玉要吃茶,一连叫了两三声,方见两三个老嬷嬷走进来。\begin{note}庚双夹:妙!文字细密,一丝不落,非批得出者。\end{note}宝玉见了他们,连忙摇手儿说:“罢,罢,不用你们了。”\begin{note}庚双夹:是宝玉口气。\end{note}老婆子们只得退出。
\end{parag}


\begin{parag}
    宝玉见没丫头们,只得自己下来,拿了碗向茶壶去倒茶。只听背后说道:“二爷仔细烫了手,让我们来倒。”\begin{note}庚侧:神龙变化之文,人岂能测?\end{note}一面说,一面走上来,早接了碗过去。宝玉倒唬了一跳,问:“你在那里的?忽然来了,唬我一跳。”那丫头一面递茶,一面回说:“我在后院子里,才从里间的后门进来,难道二爷就没听见脚步响?”宝玉一面吃茶,一面\begin{note}庚双夹:六个“一面”,是神情,并不觉厌。\end{note}仔细打量那丫头:穿著几件半新不旧的衣裳,倒是一头黑鬒鬒的头发,挽著个髻,容长脸面,细巧身材,却十分俏丽干净。\begin{note}庚双夹:与贾芸目中所见不差。\end{note}宝玉看了,便笑问道:\begin{note}庚双夹:神情写得出。\end{note}“你也是我这屋里的人么?”\begin{note}庚双夹:妙问。必如此问方是笼络前文。\end{note}那丫头道:“是的。”宝玉道:“既是这屋里的,我怎么不认得?”那丫头听说,便冷笑了一声道:\begin{note}庚双夹:神情如画。\end{note}“认不得的也多,岂只我一个。从来我又不递茶递水,拿东拿西,眼见的事一点儿不作,那里认得呢。”宝玉道:“你为什么不作那眼见的事?”\begin{note}庚侧:这是下情不能上达意语也。\end{note}那丫头道:
\end{parag}


\begin{parag}
    “这话我也难说。\begin{note}庚侧:不伏气语,况非尔可完,故云“难说”。\end{note}只是有一句话回二爷:昨儿有个什么芸儿来找二爷。我想二爷不得空儿,便叫焙茗回他,叫他今日早起来,不想二爷又往北府里去了。”刚说到这句话,只见秋纹,碧痕嘻嘻哈哈的说笑著进来,两个人共提著一桶水,一手撩著衣裳,趔趔趄趄,泼泼撒撒的。那丫头便忙迎去接。\begin{note}庚侧:好!有眼色。\end{note}那秋纹碧痕正对著抱怨,“你湿了我的裙子”,那个又说“你踹了我的鞋”。忽见走出一个人来接水,二人看时,不是别人,原来是小红。二人便都诧异,将水放下,忙进房来东瞧西望,\begin{note}庚侧:四字渐露大丫头素日怡红细事也。\end{note}\begin{note}庚眉:怡红细事俱用带笔白描,是大章法也。丁亥夏。畸笏叟。\end{note}并没个别人,只有宝玉,便心中大不自在。只得预备下洗澡之物,待宝玉脱了衣裳,二人便带上门出来,\begin{note}庚侧:清楚之至。\end{note}
\end{parag}


\begin{parag}
    走到那边房内便找小红,问他方才在屋里说什么。小红道:“我何曾在屋里的?只因我的手帕子不见了,往后头找手帕子去。不想二爷要茶吃,叫姐姐们一个没有,是我进去了,才倒了茶,姐姐们便来了。”秋纹听了,兜脸啐了一口,骂道:“没脸的下流东西!正经叫你去催水去,你说有事故,倒叫我们去,你可等著做这个巧宗儿。\begin{note}庚侧:难说小红无心,白描。\end{note}一里一里的,这不上来了。难道我们倒跟不上你了?你也拿镜子照照,配递茶递水不配!”\begin{note}庚侧:“难说” 二字全在此句来。\end{note}碧痕道:“明儿我说给他们,凡要茶要水送东送西的事,咱们都别动,只叫他去便是了。”秋纹道:“这么说,不如我们散了,单让他在这屋里呢。”二人你一句我一句,正闹著,只见有个老嬷嬷进来传凤姐的话说:“明日有人带花儿匠来种树,叫你们严禁些,衣服裙子别混晒混晾的。那土山上一溜都都拦著帏幙呢,可别混跑。”秋纹便问:\begin{note}庚侧:用秋纹问,是暗透之法。\end{note}“明儿不知是谁带进匠人来监工?”那婆子道:“说什么后廊上的芸哥儿。”秋纹,碧痕听了都不知道,只管混问别的话。那小红听见了,\begin{note}庚侧:可是暗透法。\end{note}心内却明白,就知是昨儿外书房所见那人了。
\end{parag}


\begin{parag}
    原来这小红本姓林,\begin{note}庚双夹:又是个林。\end{note}小名红玉,\begin{note}庚双夹:“红”字切“绛珠”,“玉”字则直通矣。\end{note}只因“玉”字犯了林黛玉、宝玉,\begin{note}庚双夹:妙文。\end{note}便都把这个字隐起来,便都叫他“小红”。原是荣国府中世代的旧仆,他父母现在收管各处房田事务。这红玉年方十六岁,因分人在大观园的时节,把他便分在怡红院中,倒也清幽雅静。不想后来命人进来居住,偏生这一所儿又被宝玉占了。这红玉虽然是个不谙事的丫头,却因他有三分容貌,\begin{note}庚双夹:有三分容貌尚且不肯受屈,况黛玉等一干才貌者乎?\end{note}心内著实妄想痴心的往上攀高,\begin{note}庚双夹:争夺者同来一看。\end{note}每每的要在宝玉面前现弄现弄。只是宝玉身边一干人,都是伶牙利爪的,\begin{note}庚侧:“难说”的原故在此。\end{note}那里插的下手去。不想今儿才有些消息,\begin{note}庚侧:余前批不谬。\end{note}又遭秋纹等一场恶意,心内早灰了一半。\begin{note}庚双夹:争名夺利者齐来一哭。\end{note}正闷闷的,忽然听见老嬷嬷说起贾芸来,不觉心中一动,便闷闷的回至房中,睡在床上暗暗盘算,翻来掉去,正没个抓寻。忽听窗外低低的叫道:“红玉,你的手帕子我拾在这里呢。”红玉听了忙走出来看,不是别人,正是贾芸。红玉不觉的粉面含羞,问道:“二爷在那里拾著的?”贾芸笑道:“你过来,我告诉你。”一面说,一面就上来拉他。那红玉急回身一跑,却被门槛绊倒。\begin{note}庚侧:睡梦中当然一跑,这方是怡红之鬟。\end{note}要知端的,下回分解。
\end{parag}


\begin{parag}
    \begin{note}庚:《红楼梦》写梦章法总不雷同。此梦更写的新奇,不见后文,不知是梦。\end{note}
\end{parag}


\begin{parag}
    \begin{note}红玉在怡红院为诸环所掩,亦可谓生不遇时,但看后四章供阿凤驱使可知。\end{note}
\end{parag}


\begin{parag}
    \begin{note}蒙回末总评:冷暖时,只自知,金刚卜氏浑闲事。眼中心,言中意,三生旧债原无底。任你贵比王侯,任你富似郭石,一时间,风流愿,不怕死!\end{note}
\end{parag}

