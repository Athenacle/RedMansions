\chap{六}{賈寶玉初試雲雨情 劉姥姥一進榮國府}


\begin{parag}
    \begin{note}甲:寶玉、襲人亦大家常事耳,寫得是已全領警幻意淫之訓。此回借劉嫗,卻是寫阿鳳正傳,並非泛文,且伏“二進”“三進”及巧姐之歸著。\end{note}
\end{parag}


\begin{parag}
    \begin{note}此回劉嫗一進榮國府,用周瑞家的,又過下回無痕,是無一筆寫一人文字之筆。\end{note}
\end{parag}


\begin{parag}
    \begin{note}蒙:風流真假一般看,借貸親疏觸眼痠。總是幻情無了處,銀燈挑盡淚漫漫。\end{note}
\end{parag}


\begin{parag}
    題曰:
\end{parag}


\begin{poem}
    \begin{pl}朝叩富兒門,富兒猶未足。\end{pl}

    \begin{pl}雖無千金酬,嗟彼勝骨肉。\end{pl}
\end{poem}


\begin{parag}
    卻說秦氏因聽見寶玉從夢中喚他的乳名,心中自是納悶,又不好細問。彼時寶玉迷迷惑惑,若有所失。衆人忙端上桂圓湯來,呷了兩口,遂起身整衣。襲人伸手與他系褲帶時,不覺伸手至大腿處,只覺冰涼一片沾溼。唬的忙退出手來,問是怎麼了。寶玉紅漲了臉,把他的手一捻。襲人本是個聰明女子,年紀本又比寶玉大兩歲,近來也漸通人事,今見寶玉如此光景,心中便覺察一半了,不覺也羞的紅漲了臉面,不敢再問。仍舊理好衣裳,遂至賈母處來,胡亂喫畢了晚飯,過這邊來。襲人忙趁衆奶孃丫鬟不在旁時,另取出一件中衣來與寶玉換上。寶玉含羞央告道:“好姐姐,千萬別告訴人。”襲人亦含羞笑問道:“你夢見什麼故事了?是那裏流出來的那些髒東西?”寶玉道:“一言難盡。”說著便把夢中之事細說與襲人聽了,然後說至警幻所授雲雨之情,羞的襲人掩面伏身而笑。寶玉亦素喜襲人柔媚嬌俏,遂強襲人同領警幻所訓雲雨之事。\begin{note}甲側:數句文完一回提綱文字。\end{note}襲人素知賈母已將自己與了寶玉的,今便如此,亦不爲越禮,\begin{note}甲雙夾:寫出襲人身份。\end{note}遂和寶玉偷試一番,幸得無人撞見。自此寶玉視襲人更比別個不同,\begin{note}甲雙夾:伏下晴雯。\end{note}襲人待寶玉更爲盡心。\begin{note}甲雙夾:一段小兒女之態,可謂追魂攝魄之筆。\end{note}暫且別無話說。\begin{note}甲雙夾:一句接住上回“紅樓夢”大篇文字,另起本回正文。\end{note}
\end{parag}


\begin{parag}
    按榮府中一宅人合算起來,人口雖不多,從上至下也有三四百丁,雖事不多,一天也有一二十件,竟如亂麻一般,並無個頭緒可作綱領。正尋思從那一件事自那一個人寫起方妙,恰好忽從千里之外,芥豆之微,小小一個人家,因與榮府略有些瓜葛,\begin{note}甲側:略有些瓜葛,是數十回後之正脈也。真千里伏線。 \end{note}這日正往榮府中來,因此便就此一家說來,倒還是頭緒。你道這一家姓甚名誰,又與榮府有甚瓜葛?諸公若嫌瑣碎粗鄙呢,則快擲下此書,另覓好書去醒目;若謂聊可破悶時,待蠢物\begin{note}甲雙夾:妙謙,是石頭口角。\end{note}逐細言來。
\end{parag}


\begin{parag}
    方纔所說的這小小之家,乃本地人氏,姓王,祖上曾作過小小的一個京官,昔年與鳳姐之祖王夫人之父認識。因貪王家的勢利,便連了宗認作侄兒。\begin{note}甲雙夾:與賈雨村遙遙相對。\end{note}那時只有王夫人之大兄鳳姐之父\begin{note}甲雙夾:兩呼兩起,不過欲觀者自醒。\end{note}與王夫人隨在京中的,知有此一門連宗之族,餘者皆不認識。目今其祖已故,只有一個兒子,名喚王成,因家業蕭條,仍搬出城外原鄉中住去了。王成新近亦因病故,只有其子,小名狗兒。狗兒亦生一子,小名板兒,嫡妻劉氏,又生一女,名喚青兒。\begin{note}甲雙夾:《石頭記》中公勳世宦之家以及草莽庸俗之族,無所不有,自能各得其妙。\end{note}一家四口,仍以務農爲業,因狗兒白日間又作些生計,劉氏又操井臼等事,青板姊妹兩個無人看管,狗兒遂將岳母劉姥姥\begin{note}甲雙夾:音老,出《諧聲字箋》。稱呼畢肖。\end{note}接來一處過活。這劉姥姥乃是個積年的老寡婦,膝下又無兒女,只靠兩畝薄田度日。今者女婿接來養活,豈不願意,遂一心一計,幫趁著女兒女婿過活起來。
\end{parag}


\begin{parag}
    因這年秋盡冬初,天氣冷將上來,家中冬事未辦,狗兒未免心中煩慮,吃了幾杯悶酒,在家閒尋氣惱,\begin{note}甲雙夾:病此病人不少,請來看狗兒。\end{note}劉氏也不敢頂撞。\begin{note}甲眉:自“紅樓夢”一回至此,則珍饈中之虀耳,好看煞!\end{note}因此劉姥姥看不過,乃勸道:“姑爺,你別嗔著我多嘴。咱們村莊人,那一個不是老老誠誠的,守多大碗兒喫多大的飯。\begin{note}甲側:能兩畝薄田度日,方說的出來。\end{note}你皆因年小的時候,託著你那老的福,\begin{note}甲雙夾:妙稱,何肖之至!\end{note}喫喝慣了,如今所以把持不住。有了錢就顧頭不顧尾,沒了錢就瞎生氣,成個什麼男子漢大丈夫呢!\begin{note}甲側:此口氣自何處得來?甲雙夾:爲紈絝下針,卻先從此等小處寫來。\end{note}如今咱們雖離城住著,終是天子腳下。這長安城中,遍地都是錢,只可惜沒人會去拿去罷了。在家跳蹋會子也不中用。”狗兒聽說,便急道:“你老只會炕頭兒上混說,難道叫我打劫偷去不成?”劉姥姥道:“誰叫你偷去呢。也到底想法兒大家裁度,不然那銀子錢自己跑到咱家來不成?”狗兒冷笑道:“有法兒還等到這會子呢。我又沒有收稅的親戚,\begin{note}甲雙夾:罵死。\end{note}作官的朋友,\begin{note}甲雙夾:罵死\end{note}\begin{note}脂批:罵死世人,可嘆可悲!\end{note}有什麼法子可想的?便有,也只怕他們未必來理我們呢!”
\end{parag}


\begin{parag}
    劉姥姥道:“這倒不然。謀事在人,成事在天。咱們謀到了,看菩薩的保佑,有些機會,也未可知。我倒替你們想出一個機會來。當日你們原是和金陵王家\begin{note}甲雙夾:四字便抵一篇世家傳。\end{note}連過宗的,二十年前,他們看承你們還好,如今自然是你們拉硬屎,不肯去親近他,故疏遠起來。想當初我和女兒還去過一遭。\begin{note}甲雙夾:補前文之未到處。\end{note}他們家的二小姐著實響快,會待人,倒不拿大。如今現是榮國府賈二老爺的夫人。聽得說,如今上了年紀,越發憐貧恤老,最愛齋僧敬道,舍米舍錢的。如今王府雖升了邊任,只怕這二姑太太還認得咱們。你何不去走動走動,或者他念舊,有些好處,也未可知。要是他發一點好心,拔一根寒毛比咱們的腰還粗呢。”劉氏一旁接口道:“你老雖說的是,但只你我這樣個嘴臉,怎樣好到他門上去的。先不先,他們那些門上的人也未必肯去通信。沒的去打嘴現世。”
\end{parag}


\begin{parag}
    誰知狗兒利名心最重,\begin{note}甲雙夾:調侃語。\end{note}聽如此一說,心下便有些活動起來。又聽他妻子這話,便笑接道:“姥姥既如此說,況且當年你又見過這姑太太一次,何不你老人家明日就走一趟,先試試風頭再說。”劉姥姥道:“噯呦呦!\begin{note}甲側:口聲如聞。\end{note}可是說的,‘侯門深似海’,我是個什麼東西,他家人又不認得我,我去了也是白去的。”狗兒笑道:“不妨,我教你老人家一個法子:你竟帶了外孫子板兒,先去找陪房周瑞,若見了他,就有些意思了。這周瑞先時曾和我父親交過一件事,我們極好的。”\begin{note}甲雙夾:欲赴豪門,必先交其僕。寫來一嘆。\end{note}劉姥姥道:“我也知道他的。只是許多時不走動,知道他如今是怎樣。這也說不得了,你又是個男人,又這樣個嘴臉,自然去不得,我們姑娘年輕媳婦子,也難賣頭賣腳的,倒還是舍著我這付老臉去碰一碰。果然有些好處,大家都有益,便是沒銀子來,我也到那公府侯門見一見世面,也不枉我一生。”說畢,大家笑了一回。當晚計議已定。
\end{parag}


\begin{parag}
    次日天未明,劉姥姥便起來梳洗了,又將板兒教訓了幾句。那板兒才五六歲的孩子,一無所知,聽見劉姥姥帶他進城逛去,\begin{note}甲雙夾:音光,去聲。遊也。出《諧聲字箋》。\end{note}便喜的無不應承。於是劉姥姥帶他進城,找至寧榮街。\begin{note}甲雙夾:街名。本地風光,妙!\end{note}來至榮府大門石獅子前,只見簇簇轎馬,劉姥姥便不敢過去,且撣了撣衣服,又教了板兒幾句話,然後蹭\begin{note}甲側:“蹭”字神理。\end{note}到角門前。只見幾個挺胸疊肚指手畫腳的人,坐在大板凳上,說東談西呢。\begin{note}甲雙夾:不知如何想來,又爲侯門三等豪奴寫照。\end{note}劉姥姥只得蹭上來問:“太爺們納福。”衆人打量了他一會,便問“那裏來的?”劉姥姥陪笑道:“我找太太的陪房周大爺的,煩那位太爺替我請他老出來。”那些人聽了,都不瞅睬,半日方說道:“你遠遠的在那牆角下等著,一會子他們家有人就出來的。”內中有一老年人說道:“不要誤他的事,何苦耍他。”因向劉姥姥道:“那周大爺已往南邊去了。他在後一帶住著,他娘子卻在家。你要找時,從這邊繞到后街上後門上去問就是了。”\begin{note}甲雙夾:有年紀人誠厚,亦是自然之理。\end{note}
\end{parag}


\begin{parag}
    劉姥姥聽了謝過,遂攜了板兒,繞到後門上。只見門前歇著些生意擔子,也有賣喫的,也有賣頑耍物件的,鬧吵吵三二十個小孩子在那裏廝鬧。\begin{note}甲雙夾:如何想來?閤眼如見。\end{note}劉姥姥便拉住一個道:“我問哥兒一聲,有個周大娘可在家麼?”孩子們道:“那個周大娘?我們這裏周大娘有三個呢,還有兩個周奶奶,不知是那一行當的?”劉姥姥道:“是太太的陪房周瑞。”孩子道:“這個容易,你跟我來。”說著,跳跳躥躥的引著劉姥姥進了後門,\begin{note}甲側:因女眷,又是後門,故容易引入。\end{note}至一院牆邊,指與劉姥姥道:“這就是他家。”又叫道:“周大娘,有個老奶奶來找你呢,我帶了來了。”
\end{parag}


\begin{parag}
    周瑞家的在內聽說,忙迎了出來,問:“是那位?”劉姥姥忙迎上來問道:“好呀,周嫂子!”周瑞家的認了半日,方笑道: “劉姥姥,你好呀!你說說,能幾年,我就忘了。\begin{note}甲側:如此口角,從何處出來?\end{note}請家裏來坐罷。”劉姥姥一壁裏走著,一壁笑說道:“你老是貴人多忘事,那裏還記得我們呢。”說著,來至房中。周瑞家的命僱的小丫頭倒上茶來喫著,周瑞家的又問板兒道:“你都長這們大了!”又問些別後閒話。又問劉姥姥:“今日還是路過,還是特來的?”\begin{note}甲側:問的有情理。\end{note}劉姥姥便說:“原是特來瞧瞧嫂子你,二則也請請姑太太的安。若可以領我見一見更好,若不能,便藉重嫂子轉致意罷了。”\begin{note}甲雙夾:劉婆亦善於權變應酬矣。\end{note}
\end{parag}


\begin{parag}
    周瑞家的聽了,便已猜著幾分來意。只因昔年他丈夫周瑞爭買田地一事,其中多得狗兒之力,今見劉姥姥如此而來,心中難卻其意,\begin{note}甲雙夾:在今世,周瑞夫婦算是個懷情不忘的正人。\end{note}二則也要顯弄自己的體面。\begin{note}甲眉:“也要顯弄”句爲後文作地步,也陪房本心本意實事。\end{note}聽如此說,便笑說道:“姥姥你放心,\begin{note}甲側:自是有寵人聲口。\end{note}大遠的誠心誠意來了,豈有個不教你見個真佛去的呢?\begin{note}甲雙夾:好口角。\end{note}論理,人來客至回話,卻不與我相干。我們這裏都是各佔一樣兒:\begin{note}甲側:略將榮府中帶一帶。\end{note}我們男的只管春秋兩季地租子,閒時只帶著小爺們出門子就完了,我只管跟太太奶奶們出門的事。皆因你原是太太的親戚,又拿我當個人,投奔了我來,我就破個例,給你通個信去。但只一件,姥姥有所不知,我們這裏又不比五年前了。如今太太竟不大管事,都是璉二奶奶管家了。你道這璉二奶奶是誰?就是太太的內侄女,當日大舅老爺的女兒,小名鳳哥的。”劉姥姥聽了,罕問道:“原來是他!怪道呢,我當日就說他不錯呢。\begin{note}甲雙夾:我亦說不錯。\end{note}這等說來,我今兒還得見他了。”周瑞家的道:“這自然的。如今太太事多心煩,有客來了,略可推得去的就推過去了,都是鳳姑娘周旋迎待。今兒寧可不會太太,倒要見他一面,纔不枉這裏來一遭。”劉姥姥道:“阿彌陀佛!全仗嫂子方便了。”周瑞家的道:“說那裏話。俗語說的:‘與人方便,自己方便。’不過用我說一句話罷了,害著我什麼。”說著,便叫小丫頭到倒廳上\begin{note}甲雙夾:一絲不亂。\end{note}悄悄的打聽打聽,老太太屋裏擺了飯了沒有。小丫頭去了。這裏二人又說些閒話。
\end{parag}


\begin{parag}
    劉姥姥因說:“這鳳姑娘今年大還不過二十歲罷了,就這等有本事,當這樣的家,可是難得的。”周瑞家的聽了道:“我的姥姥,告訴不得你呢。這位鳳姑娘年紀雖小,行事卻比世人都大呢。如今出挑的美人一樣的模樣兒,少說些有一萬個心眼子。再要賭口齒,十個會說話的男人也說他不過。回來你見了就信了。就只一件,待下人未免太嚴些個。”\begin{note}甲雙夾:略點一句,伏下後文。\end{note}說著,只見小丫頭回來說:“老太太屋裏已擺完了飯了,二奶奶在太太屋裏呢。”周瑞家的聽了,連忙起身,催著劉姥姥說:“快走,快走。這一下來他喫飯是個空子,咱們先趕著去。若遲一步,回事的人也多了,難說話。再歇了中覺,越發沒了時候了。”\begin{note}甲雙夾:寫出阿鳳勤勞冗雜,並驕矜珍貴等事來。甲眉:寫阿鳳勤勞等事,然卻是虛筆,故於後文不犯。蒙側:非身臨其境者不知。\end{note}說著一齊下了炕,打掃打掃衣服,又教了板兒幾句話,隨著周瑞家的,逶迤往賈璉的住處來。
\end{parag}


\begin{parag}
    先到了倒廳,周瑞家的將劉姥姥安插在那裏略等一等。自己先過了影壁,進了院門,知鳳姐未下來,先找著鳳姐的一個心腹通房大丫頭,\begin{note}甲雙夾:著眼。這也是書中一要緊人。《紅樓夢》曲內雖未見有名,想亦在副冊內者也。\end{note}\begin{note}脂批:觀警幻情榜方知言餘不謬。\end{note}名喚平兒的。\begin{note}甲雙夾:名字真極,文雅則假。\end{note}周瑞家的先將劉姥姥起初來歷說明,\begin{note}甲雙夾:細!蓋平兒原不知有此一人耳。\end{note}又說:“今日大遠的特來請安。當日太太是常會的,今日不可不見,所以我帶了他進來了。等奶奶下來,我細細回明,奶奶想也不責備我莽撞的。”平兒聽了,便作了主意:“叫他們進來,先在這裏坐著就是了。”\begin{note}甲雙夾:暗透平兒身份。\end{note}周瑞家的聽了,方出去引他兩個進入院來。上了正房臺磯,小丫頭打起猩紅氈簾,\begin{note}甲雙夾:是冬日。\end{note}才入堂屋,只聞一陣香撲了臉來,\begin{note}甲雙夾:是劉姥姥鼻中。 \end{note}竟不辨是何氣味,身子如在雲端裏一般。\begin{note}甲雙夾:是劉姥姥身子。 \end{note}滿屋中之物都耀眼爭光的,使人頭懸目眩。\begin{note}甲雙夾:是劉姥姥頭目。\end{note}劉姥姥此時惟點頭咂嘴唸佛而已。\begin{note}甲雙夾:六字盡矣,如何想來。\end{note}於是來至東邊這間屋內,乃是賈璉的女兒大姐兒睡覺之所。\begin{note}甲雙夾:記清。\end{note}平兒站在炕沿邊,打量了劉姥姥兩眼,\begin{note}甲雙夾:寫豪門侍兒。\end{note}只得\begin{note}甲雙夾:字法。\end{note}問個好讓坐。劉姥姥見平兒遍身綾羅,插金帶銀,花容玉貌的,\begin{note}甲雙夾:從劉姥姥心中目中略一寫,非平兒正傳。 \end{note}便當是鳳姐兒了。\begin{note}甲雙夾:畢肖。\end{note}纔要稱姑奶奶,忽見周瑞家的稱他是平姑娘,又見平兒趕著周瑞家的稱周大娘,方知不過是個有些體面的丫頭了。於是讓劉姥姥和板兒上了炕,平兒和周瑞家的對面坐在炕沿上,小丫頭子斟了茶來喫茶。
\end{parag}


\begin{parag}
    劉姥姥只聽見“咯噹”“咯噹”的響聲,大有似乎打籮櫃篩面的一般,\begin{note}甲雙夾:從劉姥姥心中意中幻擬出奇怪文字。\end{note}不免東瞧西望的。忽見堂屋中柱子上掛著一個匣子,底下又墜著一個秤砣般一物,卻不住的亂幌。\begin{note}甲雙夾:從劉姥姥心中目中設譬擬想,真是鏡花水月。\end{note}劉姥姥心中想著:“這是什麼愛物兒?有甚用呢?”正呆時,\begin{note}甲雙夾:三字有勁。\end{note}只聽得“當”的一聲,又若金鐘銅磬一般,不防倒唬的一展眼。接著又是一連八九下。\begin{note}甲側:寫得出。甲雙夾:細!是巳時。\end{note}方欲問時,只見小丫頭子們齊亂跑,說:“奶奶下來了。”周瑞家的與平兒忙起身,命劉姥姥:“只管等著,是時候我們來請你。”說著,都迎出去了。
\end{parag}


\begin{parag}
    劉姥姥屏聲側耳默候。只聽遠遠有人笑聲,\begin{note}甲側:寫得侍僕婦。\end{note}約有一二十婦人,衣裙窣窣,漸入堂屋,往那邊屋內去了。又見兩三個婦人,都捧著大漆捧盒,進這邊來等候。聽得那邊說了聲“擺飯”,漸漸的人才散出,只有伺候端菜的幾個人。半日鴉雀不聞之後,忽見二人抬了一張炕桌來,放在這邊炕上,桌上碗盤森列,仍是滿滿的魚肉在內,不過略動了幾樣。板兒一見了,便吵著要肉喫,劉姥姥一巴掌打了他去。忽見周瑞家的笑嘻嘻走過來,招手兒叫他。劉姥姥會意,於是帶了板兒下炕,至堂屋中,周瑞家的又和他唧咕了一會,方過這邊屋裏來。
\end{parag}


\begin{parag}
    只見門外鏨銅鉤上懸著大紅撒花軟簾,\begin{note}甲側:從門外寫來。\end{note}南窗下是炕,炕上大紅氈條,靠東邊板壁立著一個鎖子錦靠背與一個引枕,鋪著金心綠閃緞大坐褥,旁邊有雕漆痰盒。那鳳姐兒家常帶著秋板貂鼠昭君套,圍著攢珠勒子,穿著桃紅撒花襖,石青刻絲灰鼠披風,大紅洋縐銀鼠皮裙,粉光脂豔,端端正正坐在那裏,\begin{note}甲雙夾:一段阿鳳房室起居器皿家常正傳,奢侈珍貴好奇貨註腳,寫來真是好看。\end{note}手內拿著小銅火箸兒撥手爐內的灰。\begin{note}甲側:至平,實至奇,稗官中未見此筆。甲雙夾:這一句是天然地設,非別文杜撰妄擬者。\end{note}平兒站在炕沿邊,捧著小小的一個填漆茶盤,盤內一個小蓋鍾。鳳姐也不接茶,也不抬頭,\begin{note}甲側:神情宛肖。\end{note}只管撥手爐內的灰,慢慢的問道:“怎麼還不請進來?”\begin{note}甲側:此等筆墨,真可謂追魂攝魄。蒙側:“還不請進來”五字,寫盡天下富貴人待窮親戚的態度。\end{note}一面說,一面抬身要茶時,只見周瑞家的已帶了兩個人在地下站著呢。這才忙欲起身,猶未起身,滿面春風的問好,又嗔周瑞家的不早說。劉姥姥在地下已是拜了數拜,“問姑奶奶安。”鳳姐忙說:“周姐姐,快攙住不拜罷。請坐。我年輕,不大認得,可也不知是什麼輩數,不敢稱呼。”周瑞家的忙回道:“這就是我纔回的那姥姥了。”\begin{note}甲側:鳳姐雲“不敢稱呼”,周瑞家的雲“那個姥姥”。凡三四句一氣讀下,方是鳳姐聲口。\end{note}鳳姐點頭。劉姥姥已在炕沿上坐了,板兒便躲在背後,百般的哄他出來作揖,他死也不肯。
\end{parag}


\begin{parag}
    鳳姐兒笑\begin{note}甲側:二笑。\end{note}道:“親戚們不大走動,都疏遠了。知道的呢,說你們棄厭我們,不肯常來,\begin{note}甲側:阿鳳真真可畏可惡。\end{note}不知道的那起小人,還只當我們眼裏沒人似的。”劉姥姥忙唸佛\begin{note}甲側:如聞。\end{note}道:“我們家道艱難,走不起,來了這裏,沒的給姑奶奶打嘴,就是管家爺們看著也不像。”鳳姐兒笑\begin{note}甲側:三笑。\end{note}道:“這話沒的叫人噁心。不過借賴著祖父虛名,作個窮官兒,誰家有什麼,不過是個舊日的空架子。俗語說,‘朝廷還有三門子窮親戚’呢,何況你我。”說著,又問周瑞家的回了太太了沒有。\begin{note}甲側:一筆不肯落空,的是阿鳳。\end{note}周瑞家的道:“如今等奶奶的示下。”鳳姐道:“你去瞧瞧,要是有人有事就罷,得閒兒呢就回,看怎麼說。”周瑞家的答應著去了。
\end{parag}


\begin{parag}
    這裏鳳姐叫人抓些果子與板兒喫,剛問些閒話時,就有家下許多媳婦管事的來回話。\begin{note}甲側:不落空家務事,卻不實寫。妙極!妙極!\end{note}平兒回了,鳳姐道:“我這裏陪客呢,晚上再來回。若有很要緊的,你就帶進來現辦。”平兒出去了,一會進來說:“我都問了,沒什麼緊事,我就叫他們散了。”鳳姐點頭。只見周瑞家的回來,向鳳姐道:“太太說了,今日不得閒,二奶奶陪著便是一樣。多謝費心想著。白來逛逛呢便罷,若有甚說的,只管告訴二奶奶,都是一樣。”劉姥姥道:“也沒甚說的,不過是來瞧瞧姑太太,姑奶奶,也是親戚們的情分。”周瑞家的道:“沒甚說的便罷,若有話,只管回二奶奶,是和太太一樣的。”\begin{note}甲側:周婦系真心爲老嫗也,可謂得方便。\end{note}一面說,一面遞眼色與劉姥姥。\begin{note}甲側:何如?餘批不謬。\end{note}劉姥姥會意,未語先飛紅的臉,\begin{note}蒙側:開口告人難。\end{note}欲待不說,今日又所爲何來?只得忍恥\begin{note}甲眉:老嫗有忍恥之心,故後有招大姐之事。作者並非泛寫,且爲求親靠友下一棒喝。\end{note}說道:“論理今兒初次見姑奶奶,卻不該說,只是大遠的奔了你老這裏來,也少不的說了。”剛說到這裏,只聽二門上小廝們回說:“東府裏的小大爺進來了。”鳳姐忙止劉姥姥:“不必說了。”一面便問:“你蓉大爺在那裏呢?”\begin{note}甲側:慣用此等橫雲斷山法。\end{note}只聽一路靴子腳響,進來了一個十七八歲的少年,面目清秀,身材俊俏,輕裘寶帶,美服華冠。\begin{note}甲側:如紈絝寫照。\end{note}劉姥姥此時坐不是,立不是,藏沒處藏。鳳姐笑道:“你只管坐著,這是我侄兒。”劉姥姥方扭扭捏捏在炕沿上坐了。
\end{parag}


\begin{parag}
    賈蓉笑道:“我父親打發我來求嬸子,說上回老舅太太給嬸子的那架玻璃炕屏,明日請一個要緊的客,借了略擺一擺就送過來的。”\begin{note}甲側:夾寫鳳姐好獎譽。\end{note}鳳姐道:“說遲了一日,昨兒已經給了人了。”賈蓉聽著,嘻嘻的笑著,在炕沿上半跪道:“嬸子若不借,又說我不會說話了,又挨一頓好打呢。嬸子只當可憐侄兒罷。”鳳姐笑\begin{note}甲側:又一笑,凡五。\end{note}道:“也沒見我們王家的東西都是好的不成?一般你們那裏放著那些東西,只是看不見我的才罷。”賈蓉笑道:“那裏有這個好呢!只求開恩罷。”鳳姐道:“若碰一點兒,你可仔細你的皮!”因命平兒拿了樓房的鑰匙,傳幾個妥當人抬去。賈蓉喜的眉開眼笑,說:“我親自帶了人拿去,別由他們亂碰。”說著便起身出去了。
\end{parag}


\begin{parag}
    這裏鳳姐忽又想起一事來,便向窗外叫:“蓉哥回來。”外面幾個人接聲說:“蓉大爺快回來。”賈蓉忙復身轉來,垂手侍立,聽何指示。\begin{note}甲眉:傳神之筆,寫阿鳳躍躍紙上。\end{note}那鳳姐只管慢慢的喫茶,出了半日的神,又笑道:“罷了,你且去罷。晚飯後你來再說罷。這會子有人,我也沒精神了。”賈蓉應了一聲,方慢慢的退去。\begin{note}甲側:妙!卻是從劉姥姥身邊目中寫來。度至下回。\end{note}
\end{parag}


\begin{parag}
    這裏劉姥姥心神方定,才又說道:“今日我帶了你侄兒來,也不爲別的,只因他老子娘在家裏,連喫的都沒有。如今天又冷了,越想沒個派頭兒,只得帶了你侄兒奔了你老來。”說著又推板兒道:“你那爹在家怎麼教你來?打發咱們作煞事來?只顧喫果子咧。”鳳姐早已明白了,聽他不會說話,因笑止道:\begin{note}甲雙夾:又一笑,凡六。自劉姥姥來凡笑五次,寫得阿鳳乖滑伶俐,閤眼如立在前。若會說話之人便聽他說了,阿鳳厲害處正在此。問看官常有將挪移借貸已說明白了,彼仍推聾裝啞,這人爲阿鳳若何?呵呵,一嘆!\end{note}“不必說了,我知道了。”因問周瑞家的:“這姥姥不知可用了早飯沒有?”劉姥姥忙說道:“一早就往這裏趕咧,那裏還有喫飯的工夫咧。”鳳姐聽說,忙命快傳飯來。一時周瑞家的傳了一桌客飯來,擺在東邊屋內,過來帶了劉姥姥和板兒過去喫飯。鳳姐說道:“周姐姐,好生讓著些兒,我不能陪了。”於是過東邊房裏來。又叫過周瑞家的去,問他纔回了太太,說了些什麼?周瑞家的道:“太太說,他們家原不是一家子,不過因出一姓,當年又與太老爺在一處作官,偶然連了宗的。這幾年來也不大走動。當時他們來一遭,卻也沒空了他們。今兒既來了瞧瞧我們,是他的好意思,\begin{note}甲側:窮親戚來看是“好意思”,餘又自《石頭記》中見了,嘆嘆!\end{note}也不可簡慢了他。便是有什麼說的,叫奶奶裁度著就是了。”\begin{note}甲眉:王夫人數語令餘幾哭出。\end{note}鳳姐聽了說道:“我說呢,既是一家子,我如何連影兒也不知道。”
\end{parag}


\begin{parag}
    說話時,劉姥姥已喫畢了飯,拉了板兒過來,舚舌咂嘴的道謝。鳳姐笑道:“且請坐下,聽我告訴你老人家。方纔的意思,我已知道了。若論親戚之間,原該不等上門來就該有照應纔是。但如今家內雜事太煩,太太漸上了年紀,一時想不到也是有的。\begin{note}甲側:點“不待上門就該有照應”數語,此亦於《石頭記》再見話頭。\end{note}況是我近來接著管些事,都不知道這些親戚們。二則外頭看著雖是烈烈轟轟的,殊不知大有大的艱難去處,說與人也未必信罷。今兒你既老遠的來了,又是頭一次見我張口,怎好叫你空回去呢。\begin{note}甲側:也是《石頭記》再見了,嘆嘆!\end{note}可巧昨兒太太給我的丫頭們做衣裳的二十兩銀子,我還沒動呢,你若不嫌少,就暫且先拿了去罷。”那劉姥姥先聽見告艱難,只當是沒有,心裏便突突的,\begin{note}甲側:可憐可嘆!\end{note}後來聽見給他二十兩,喜的又渾身發癢起來,\begin{note}甲側:可憐可嘆!\end{note}說道:“噯,我也是知道艱難的。但俗語說的,‘瘦死的駱駝比馬大’,憑他怎樣,你老拔根寒毛比我們的腰還粗呢!”周瑞家的見他說的粗鄙,只管使眼色止他。鳳姐看見,笑而不睬,只命平兒把昨兒那包銀子拿來,再拿一吊錢來,\begin{note}甲側:這樣常例亦再見。\end{note}都送到劉姥姥的跟前。鳳姐乃道:“這是二十兩銀子,暫且給這孩子做件冬衣罷。若不拿著,就真是怪我了。這錢僱車坐罷。改日無事,只管來逛逛,方是親戚們的意思。天也晚了,也不虛留你們了,到家裏該問好的問個好兒罷。”一面說,一面就站了起來。
\end{parag}


\begin{parag}
    劉姥姥只管千恩萬謝,拿了銀錢,隨了周瑞家的來至外面。周瑞家的方道:“我的娘啊!你見了他怎麼倒不會說話了?開口就是‘你侄兒’。我說句不怕你惱的話,便是親侄兒,也要說和軟些。那蓉大爺纔是他的正經侄兒呢,他怎麼又跑出這麼個侄兒來了。”\begin{note}甲雙夾:與前“眼色”針對,可見文章中無一個閒字。爲財勢一哭。\end{note}劉姥姥笑道:“我的嫂子,\begin{note}甲側:赧顏如見。\end{note}我見了他,心眼兒裏愛還愛不過來,那裏還說的上話來呢。”二人說著,又到周瑞家坐了片時。劉姥姥便要留下一塊銀子與周瑞家孩子們買果子喫,周瑞家的如何放在眼裏,執意不肯。劉姥姥感謝不盡,仍從後門去了。正是:
\end{parag}


\begin{poem}
    \begin{pl}得意濃時易接濟,受恩深處勝親朋。\end{pl}
\end{poem}


\begin{parag}
    \begin{note}甲:一進榮府一回,曲折頓挫,筆如游龍,且將豪華舉止令觀者已得大概,想作者應是心花欲開之候。借劉嫗入阿鳳正文,“送宮花”寫“金玉初聚”爲引,作者真筆似游龍,變幻難測,非細究至再三再四不記數,那能領會也?嘆嘆!蒙:夢裏風流,醒後風流,試問何真何假?劉姆乞謀,蓉兒借求,多少顛倒相酬。英雄反正用計籌,不是死生看守。\end{note}
\end{parag}
