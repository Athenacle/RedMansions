\chap{三十二}{訴肺腑心迷活寶玉 含恥辱情烈死金釧}


\begin{parag}
    \begin{note}庚:前明顯祖湯先生有《懷人》詩一截,堪合此回,故錄之以待知音。曰:無情無盡卻情多,情到無多得盡麼?解道多情情盡處,月中無樹影無波。\end{note}
\end{parag}


\begin{parag}
    話說寶玉見那麒麟,心中甚是歡喜,便伸手來拿,笑道:“虧你揀著了。你是那裏揀的?”史湘雲笑道:“幸而是這個,明兒倘或把印也丟了,難道也就罷了不 成?”寶玉笑道:“倒是丟了印平常,若丟了這個,我就該死了。”襲人斟了茶來與史湘雲喫,一面笑道:“大姑娘,聽見前兒你大喜了。”史湘雲紅了臉,喫茶不答。襲人道:“這會子又害臊了。你還記得十年前,咱們在西邊暖閣住著,晚上你同我說的話兒?那會子不害臊,這會子怎麼又害臊了?”史湘雲笑道:“你還說呢。那會子咱們那麼好。後來我們太太沒了,我家去住了一程子,怎麼就把你派了跟二哥哥,我來了,你就不象先待我了。”襲人笑道:“你還說呢。先姐姐長姐姐短哄著我替你梳頭洗臉,作這個弄那個,\begin{note}蒙側:大家風範,情景逼真。\end{note}如今大了,就拿出小姐的款來。你既拿小姐的款,我怎敢親近呢?”史湘雲道:“阿彌陀佛,冤枉冤哉!我要這樣,就立刻死了。你瞧瞧,這麼大熱天,我來了,必定趕來先瞧瞧你。不信你問問縷兒,我在家時時刻刻那一回不念你幾聲。”話未了,忙的襲人和寶玉都勸道:“頑話你又認真了。還是這麼性急。”史湘雲道:“你不說你的話噎人,倒說人性急。”一面說,一面打開手帕子,將戒指遞與襲人。\begin{note}蒙側 批:心中意中多少情致。\end{note}襲人感謝不盡,因笑道:“你前兒送你姐姐們的,我已得了;今兒你親自又送來,可見是沒忘了我。只這個就試出你來了。戒指兒能值多 少,可見你的心真。”史湘雲道:“是誰給你的?”襲人道:“是寶姑娘給我的。”湘雲笑道:“我只當是林姐姐給你的,原來是寶釵姐姐給了你。我天天在家裏想 著,這些姐姐們再沒一個比寶姐姐好的。可惜我們不是一個娘養的。\begin{note}蒙側:感知己之一歡。\end{note}我但凡有這麼個親姐姐,就是沒了父母,也是沒妨礙的。”說著, 眼睛圈兒就紅了。寶玉道:“罷,罷,罷!不用提這個話。”史湘雲道:“提這個便怎麼?我知道你的心病,恐怕你的林妹妹聽見,又怪嗔我讚了寶姐姐。可是爲這個不是?”襲人在旁嗤的一笑,說道:“雲姑娘,你如今大了,越發心直口快了。”寶玉笑道:“我說你們這幾個人難說話,果然不錯。”史湘雲道:“好哥哥,你 不必說話教我噁心。只會在我們跟前說話,見了你林妹妹,又不知怎麼了。”\begin{note}蒙側:豪爽情性如畫。\end{note}
\end{parag}


\begin{parag}
    襲人道:“且別說頑話,正有一件事還要求你呢。”史湘雲便問:“什麼事?”襲人道:“有一雙鞋,摳了墊心子。我這兩日身上不好,不得做,你可有工夫替 我做做?”史湘雲笑道:“這又奇了,你家放著這些巧人不算,還有什麼針線上的,裁剪上的,怎麼教我做起來?你的活計叫誰做,誰好意思不做呢。”襲人笑道: “你又糊塗了。你難道不知道,我們這屋裏的針線,\begin{note}蒙側:“我們這屋裏”等字精神活跳。\end{note}是不要那些針線上的人做的。”史湘雲聽了,便知是寶玉的鞋了, 因笑道:“既這麼說,我就替你做了罷。只是一件,你的我才作,別人的我可不能。”襲人笑道:“又來了,我是個什麼,就煩你做鞋了。實告訴你,可不是我的。 你別管是誰的,橫豎我領情就是了。”史湘雲道:“論理,你的東西也不知煩我做了多少了,今兒我倒不做了的原故,你必定也知道。”襲人道:“倒也不知道。\begin{note}蒙側:反觀疊起,靈活之至。\end{note}” 史湘雲冷笑道:“前兒我聽見把我做的扇套子拿著和人家比,賭氣又鉸了。我早就聽見了,你還瞞我。這會子又叫我做,我成了你們的奴才了。”寶玉忙笑道:“前 兒的那事,本不知是你做的。”襲人也笑道:“他本不知是你做的。是我哄他的話,說是新近外頭有個會做活的女孩子,說扎的出奇的花,我叫他拿了一個扇套子試 試看好不好。他就信了,拿出去給這個瞧給那個看的。不知怎麼又惹惱了林姑娘,鉸了兩段。回來他還叫趕著做去,我才說了是你作的,他後悔的什麼似的。\begin{note}蒙側:描神!\end{note}”史湘雲道:“越發奇了。林姑娘他也犯不上生氣,他既會剪,就叫他做。”襲人道:“他可不作呢。饒這麼著,老太太還怕他勞碌著了。大夫又說好 生靜養纔好,誰還煩他做?舊年好一年的工夫,做了個香袋兒;今年半年,還沒見拿針線呢。”
\end{parag}


\begin{parag}
    正說著,有人來回說:“興隆街的大爺來了,老爺叫二爺出去會。”寶玉聽了,便知是賈雨村來了,心中好不自在。襲人忙去拿衣服。寶玉一面蹬著靴子,一面抱怨道:“有老爺和他坐著就罷了,\begin{note}蒙側:原本煩俗。\end{note}回回定要見我。”史湘雲一邊搖著扇子,笑道:“自然你能會賓接客,老爺才叫你出去呢。”寶玉道: “那裏是老爺,都是他自己要請我去見的。”湘雲笑道:“主雅客來勤,自然你有些警他的好處,他才只要會你。”寶玉道:“罷,罷,我也不敢稱雅,俗中又俗的 一個俗人,並不願同這些人往來。”\begin{note}蒙側:我也不知寶玉是俗是雅,請諸同類一擬。\end{note}湘雲笑道:“還是這個情性不改。如今大了,你就不願讀書去考舉人進士的,也該常常的會會這些爲官做宰的人們,談談講講些仕途經濟的學問,也好將來應酬世務,日後也有個朋友。沒見你成年家只在我們隊裏攪些什麼!”寶玉聽了 道:“姑娘請別的姊妹屋裏坐坐,我這裏仔細污了你知經濟學問的。”襲人道:“雲姑娘快別說這話。\begin{note}蒙側:此際不同湘雲一語,湘雲也定難出一語。\end{note}上回也是寶姑娘也說過一回,他也不管人臉上過的去過不去,他就咳了一聲,拿起腳來走了。這裏寶姑娘的話也沒說完,見他走了,登時羞的臉通紅,說又不是,不說又不是。幸而是寶姑娘,那要是林姑娘,不知又鬧到怎麼樣,哭的怎麼樣呢。提起這個話來,真真的寶姑娘叫人敬重,自己訕了一會子去了。我倒過不去,\begin{note}蒙側:襲人善解忿。\end{note}只當他惱了。誰知過後還是照舊一樣,真真有涵養,心地寬大。誰知這一個反倒同他生分了。那林姑娘見你賭氣不理他,你得賠多少不是呢。”寶玉道:“林姑娘從來說過這些混帳話不曾?若他也說過這些混帳話,我早和他生分了。”\begin{note}蒙側:花愛水清明,水憐花色新。浮落雖同流,空惹魚龍涎。\end{note}襲人和湘雲都點頭笑道:“這原是混帳話。”
\end{parag}


\begin{parag}
    原來林黛玉知道史湘雲在這裏,寶玉又趕來,一定說麒麟的原故。因此心下忖度著,近日寶玉弄來的外傳野史,多半才子佳人都因小巧玩物上撮合,或有鴛鴦, 或有鳳凰,或玉環金珮,或鮫帕鸞絛,皆由小物而遂終身。今忽見寶玉亦有麒麟,便恐藉此生隙,同史湘雲也做出那些風流佳事來。因而悄悄走來,見機行事,以察二人之意。不想剛走來,正聽見史湘雲說經濟一事,寶玉又說:“林妹妹不說這樣混帳話,若說這話,我也和他生分了。”林黛玉聽了這話,不覺又喜又驚,又悲又 嘆。所喜者,果然自己眼力不錯,素日認他是個知己,果然是個知己。所驚者,他在人前一片私心稱揚於我,其親熱厚密,竟不避嫌疑。所嘆者,你既爲我之知己, 自然我亦可爲你之知己矣;既你我爲知己,則又何必有金玉之論哉;既有金玉之論,亦該你我有之,則又何必來一寶釵哉!所悲者,父母早逝,雖有銘心刻骨之言, 無人爲我主張。況近日每覺神思恍惚,病已漸成,醫者更雲氣弱血虧,恐致勞怯之症。你我雖爲知己,但恐自不能久待;你縱爲我知己,奈我薄命何!想到此間,不 禁滾下淚來。\begin{note}蒙側:普天下才子佳人英雄俠士都同來一哭!我雖愚濁,也願同聲一哭。\end{note}待進去相見,自覺無味,便一面拭淚,一面抽身回去了。
\end{parag}


\begin{parag}
    這裏寶玉忙忙的穿了衣裳出來,忽見林黛玉在前面慢慢的走著,似有拭淚之狀,便忙趕上來,\begin{note}蒙側:關心情致。\end{note}笑道:“妹妹往那裏去?怎麼又哭了?又 是誰得罪了你?”林黛玉回頭見是寶玉,便勉強笑道:“好好的,我何曾哭了。”寶玉笑道:“你瞧瞧,眼睛上的淚珠兒未乾,還撒謊呢。”一面說,一面禁不住抬 起手來替他拭淚。林黛玉忙向後退了幾步,說道:“你又要死了!\begin{note}蒙側:嬌羞態!\end{note}作什麼這麼動手動腳的!”寶玉笑道:“說話忘了情,不覺的動了手,也就顧不的死活。”林黛玉道:“你死了倒不值什麼,只是丟下了什麼金,又是什麼麒麟,可怎麼樣呢?”一句話又把寶玉說急了,趕上來問道:“你還說這話,到底是咒我還是氣我呢?”林黛玉見問,方想起前日的事來,遂自悔自己又說造次了,忙笑道:“你別著急,我原說錯了。這有什麼的,筋都暴起來,急的一臉汗。”一面說,一面禁不住近前伸手替他拭面上的汗\begin{note}蒙側:癡情態。\end{note}。寶玉瞅了半天,方說道“你放心”三個字。\begin{note}蒙側:連我今日看之,也不懂是何等文章。\end{note}林黛玉聽了,怔了半天,方說道:“我有什麼不放心的?我不明白這話。你倒說說怎麼放心不放心?”寶玉嘆了一口氣,問道:“你果不明白這話?難道我素日在你身上的心都用錯了?連你的意思若體貼不著,就難怪你天天爲我生氣了。”林黛玉道:“果然我不明白放心不放心的話。”寶玉點頭嘆道:“好妹妹,你別哄我。果然不 明白這話,不但我素日之意白用了,且連你素日待我之意也都辜負了。\begin{note}蒙側:第二層。\end{note}你皆因總是不放心的原故,才弄了一身病。但凡寬慰些,\begin{note}蒙側:真 疼真愛真憐真惜中,每每生出此等心病來。\end{note}這病也不得一日重似一日。”林黛玉聽了這話,如轟雷掣電,細細思之,竟比自己肺腑中掏出來的還覺懇切,\begin{note}蒙側 批:何等神佛,開慧眼照見衆生孽障,爲現此錦繡文章,說此上乘功德法。\end{note}竟有萬句言語,滿心要說,只是半個字也不能吐,卻怔怔的望著他。此時寶玉心中也有萬句言語,不知從那一句上說起,卻也怔怔的望著黛玉。兩個人怔了半天,林黛玉只咳了一聲,兩眼不覺滾下淚來,回身便要走。\begin{note}蒙側:下筆時,用一“走”, 文之大力,孟憤(憤之右半)不苦也。\end{note}寶玉忙上前拉住,說道:“好妹妹,且略站住,我說一句話再走。”林黛玉一面拭淚,一面將手推開,說道:“有什麼可說 的。你的話我早知道了!”口裏說著,卻頭也不回竟去了。
\end{parag}


\begin{parag}
    寶玉站著,只管發起呆來。原來方纔出來慌忙,不曾帶得扇子,襲人怕他熱,忙拿了扇子趕來送與他,忽抬頭見了林黛玉和他站著。一時黛玉走了,他還站著不動,因而趕上來說道:“你也不帶了扇子去,虧我看見,趕了送來。”寶玉出了神,見襲人和他說話,並未看出是何人來,便一把拉住,說道:“好妹妹,我的這心 事,從來也不敢說,今兒我大膽說出來,死也甘心!我爲你也弄了一身的病在這裏,又不敢告訴人,只好掩著。只等你的病好了,只怕我的病才得好呢。睡裏夢裏也 忘不了你!”襲人聽了這話,嚇得魄消魂散,只叫“神天菩薩,坑死我了!”便推他道:“這是那裏的話!敢是中了邪?還不快去?”寶玉一時醒過來,方知是襲人 送扇子來,羞的滿面紫漲,奪了扇子,便忙忙的抽身跑了。
\end{parag}


\begin{parag}
    這裏襲人見他去了,自思方纔之言,一定是因黛玉而起,如此看來,將來難免不才之事,令人可驚可畏。想到此間,也不覺怔怔的滴下淚來,心下暗度如何處治 方免此醜禍。正裁疑間,忽有寶釵從那邊走來,笑道:“大毒日頭地下,出什麼神呢?”襲人見問,忙笑道:“那邊兩個雀兒打架,倒也好玩,我就看住了。”寶釵道:“寶兄弟這會子穿了衣服,忙忙的那去了?我纔看見走過去,倒要叫住問他呢。他如今說話越發沒了經緯,我故此沒叫他了,由他過去罷。”襲人道:“老爺叫 他出去。”寶釵聽了,忙道:“噯喲!這麼黃天暑熱的,叫他做什麼!別是想起什麼來生了氣,\begin{note}蒙側:偏是近。\end{note}叫出去教訓一場。”襲人笑道:“不是這個, 想是有客要會。”寶釵笑道:“這個客也沒意思,這麼熱天,不在家裏涼快,還跑些什麼!”襲人笑道:“倒是你說說罷。”
\end{parag}


\begin{parag}
    寶釵因而問道:“雲丫頭在你們家做什麼呢?”襲人笑道:“才說了一會子閒話。你瞧,我前兒粘的那雙鞋,明兒叫他做去。”寶釵聽見這話,便兩邊回頭,看 無人來往,便笑道:“你這麼個明白人,怎麼一時半刻的就不會體諒人情。我近來看著雲丫頭神情,再風裏言風裏語的聽起來,那雲丫頭在家裏竟一點兒作不得主。 他們家嫌費用大,竟不用那些針線上的人,差不多的東西多是他們娘兒們動手。爲什麼這幾次他來了,他和我說話兒,見沒人在跟前,他就說家裏累的很。我再問他 兩句家常過日子的話,他就連眼圈兒都紅了,口裏含含糊糊待說不說的。想其形景來,自然從小兒沒爹孃的苦。\begin{note}蒙側:真是知己,不枉湘雲前言。\end{note}我看著他, 也不覺的傷起心來。”襲人見說這話,將手一拍,說:“是了,是了。怪道上月我煩他打十根蝴蝶結子,過了那些日子纔打發人送來,還說‘打的粗,且在別處能著 使罷;要勻淨的,等明兒來住著再好生打罷’。如今聽寶姑娘這話,想來我們煩他他不好推辭,不知他在家裏怎麼三更半夜的做呢。可是我也糊塗了,早知是這樣, 我也不煩他了。”寶釵道:“上次他就告訴我,在家裏做活做到三更天,若是替別人做一點半點,他家的那些奶奶太太們還不受用呢。”襲人道:“偏生我們那個牛心左性的小爺,\begin{note}蒙側:多情的當有這樣牛心左性之癖。\end{note}憑著小的大的活計,一概不要家裏這些活計上的人作。我又弄不開這些。”寶釵笑道:“你理他呢!只 管叫人做去,只說是你做的就是了。”襲人笑道:“那裏哄的信他,他纔是認得出來呢。說不得我只好慢慢的累去罷了。\begin{note}蒙側:癡心的情願。\end{note}”寶釵笑道: “你不必忙,我替你作些如何?”襲人笑道:“當真的這樣,就是我的福了。晚上我親自送過來。”
\end{parag}


\begin{parag}
    一句話未了,忽見一個老婆子忙忙走來,說道:“這是那裏說起!金釧兒姑娘好好的投井死了!”襲人唬了一跳,忙問:“那個金釧兒?”那老婆子道:“那裏 還有兩個金釧兒呢?就是太太屋裏的。前兒不知爲什麼攆他出去,在家裏哭天哭地的,也都不理會他,誰知找他不見了。剛纔打水的人在那東南角上井裏打水,見一個屍首,趕著叫人打撈起來,誰知是他。他們家裏還只管亂著要救活,那裏中用了!”寶釵道:“這也奇了。”襲人聽說,點頭讚歎,想素日同氣之情,不覺流下淚 來。\begin{note}蒙側:又一哭法。\end{note}寶釵聽見這話,忙向王夫人處來道安慰。這裏襲人回去不提。
\end{parag}


\begin{parag}
    卻說寶釵來至王夫人處,只見鴉雀無聞,獨有王夫人在裏間房內坐著垂淚。\begin{note}蒙側:又一哭法。\end{note}寶釵便不好提這事,只得一旁坐了。王夫人便問:“你從那裏來?”寶釵道:“從園裏來。”王夫人道:“你從園裏來,可見你寶兄弟?”\begin{note}蒙側:世人多是凡事欲瞞人,偏不意中將要著開露,理之所無,事則多有,何 也?\end{note}寶釵道:“才倒看見了。他穿了衣服出去了,不知那裏去。”王夫人點頭哭道:“你可知道一樁奇事?金釧兒忽然投井死了!”寶釵見說,道:“怎麼好好的 投井?這也奇了。”王夫人道:“原是前兒他把我一件東西弄壞了,我一時生氣,打了他幾下,攆了他下去。我只說氣他兩天,還叫他上來,誰知他這麼氣性大,就 投井死了。豈不是我的罪過。”寶釵嘆道:“姨娘是慈善人,固然這麼想。據我看來,他並不是賭氣投井。多半他下去住著,或是在井跟前憨頑,失了腳掉下去的。 他在上頭拘束慣了,這一出去,自然要到各處去頑頑逛逛,豈有這樣大氣的理!縱然有這樣大氣,也不過是個糊塗人,也不爲可惜。\begin{note}蒙側:善勸人大見解!惜乎?不知其情,雖精美玉之言不中,奈何?\end{note}” 王夫人點頭嘆道:“這話雖然如此說,到底我心不安。”寶釵嘆道:“姨娘也不必念念於茲,十分過不去,不過多賞他幾兩銀子發送他,也就盡主僕之情了。”王夫 人道:“剛纔我賞了他娘五十兩銀子,原要還把你妹妹們的新衣服拿兩套給他妝裹。誰知鳳丫頭說可巧都沒什麼新做的衣服,只有你林妹妹作生日的兩套。我想你林 妹妹那個孩子素日是個有心的,況且他也三災八難的,既說了給他過生日,這會子又給人妝裹去,豈不忌諱。因爲這麼樣,我現叫裁縫趕兩套給他。要是別的丫頭, 賞他幾兩銀子也就完了,只是金釧兒雖然是個丫頭,素日在我跟前比我的女兒也差不多。”口裏說著,不覺淚下。寶釵忙道:“姨娘這會子又何用叫裁縫趕去,我前 兒倒做了兩套,拿來給他豈不省事。況且他活著的時候也穿過我的舊衣服,身量又相對。”王夫人道:“雖然這樣,難道你不忌諱?”寶釵笑道:“姨娘放心,我從 來不計較這些。”一面說,一面起身就走。王夫人忙叫了兩個人來跟寶姑娘去。
\end{parag}


\begin{parag}
    一時寶釵取了衣服回來,只見寶玉在王夫人旁邊坐著垂淚。王夫人正才說他,因寶釵來了,卻掩了口不說了。\begin{note}蒙側:雲龍現影法,可愛煞人。\end{note}寶釵見此光景,察言觀色,早知覺了八分,於是將衣服交割明白。王夫人將他母親叫來拿了去。再看下回便知。
\end{parag}


\begin{parag}
    \begin{note}蒙回末總:世上無情空大地,人間少愛景何窮。其中世界其中了,含笑同歸造化功。\end{note}
\end{parag}


\begin{parag}
    \begin{note}蒙回末總:襲人湘雲黛玉寶釵等之愛之哭,各具一心,各具一見。而寶玉黛玉之癡情癡性,行文如繪真,是現身說法,豈三家村老學究之可能實現者!不盡炷香再拜!\end{note}
\end{parag}

