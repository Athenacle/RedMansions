\chap{十}{金寡婦貪利權受辱 張太醫論病細窮源}


\begin{parag}
    \begin{note}蒙:新樣幻情慾收拾,可卿從此世無緣。和肝益氣渾閒事,誰知今日尋病源?\end{note}
\end{parag}


\begin{parag}
    話說金榮因人多勢衆,又兼賈瑞勒令,賠了不是,給秦鍾磕了頭,寶玉方纔不吵鬧了。大家散了學,金榮回到家中,越想越氣,說:“秦鐘不過是賈蓉的小舅子,又不是賈家的子孫,附學讀書,也不過和我一樣。他因仗著寶玉和他好,他就目中無人。他既是這樣,就該行些正經事,人也沒的說。他素日又和寶玉鬼鬼祟祟的,只當我們都是瞎子,看不見。今日他又去勾搭人,偏偏的撞在我眼裏。\begin{note}蒙側:偏是鬼鬼祟祟者,多以爲人不見其行,不知其心。\end{note}就是鬧出事來,我還怕什麼不成?”
\end{parag}


\begin{parag}
    他母親胡氏聽見他咕咕嘟嘟的說,因問道:“你又要爭什麼閒氣?好容易\begin{note}蒙側:“好容易”三字,寫盡天下迎逢要便宜苦惱。\end{note}我望你姑媽說了,你姑媽千方百計的才向他們西府裏的璉二奶奶跟前說了,你才得了這個唸書的地方。若不是仗著人家,咱們家裏還有力量請的起先生?況且人家學裏,茶也是現成的,飯也是現成的。你這二年在那裏唸書,家裏也省好大的嚼用呢。省出來的,你又愛穿件鮮明衣服。再者,不是因你在那裏唸書,你就認得什麼薛大爺了?那薛大爺一年不給不給,這二年也幫了咱們有七八十兩銀子。\begin{note}己側:因何無故給許多銀子?金母亦當細思之。\end{note}\begin{note}蒙側:可憐!婦人愛子,每每如此。自知所得者多,而不知所失者大,可勝嘆者!\end{note}你如今要鬧出了這個學房,再要找這麼個地方,我告訴你說罷,比登天還難呢!\begin{note}己側:如此弄銀,若有金榮在,亦可得。\end{note}你給我老老實實的頑一會子睡你的覺去,好多著呢。”於是金榮忍氣吞聲,不多一時他自去睡了。次日仍舊上學去了。不在話下。
\end{parag}


\begin{parag}
    且說他姑娘,原聘給的是賈家玉字輩的嫡派,名喚賈璜。但其族人那裏皆能象寧榮二府的富勢,原不用細說。這賈璜夫妻守著些小的產業,又時常到寧榮二府裏去請請安,又會奉承鳳姐兒並尤氏,所以鳳姐兒尤氏也時常資助資助他,\begin{note}蒙側:原來根由如此,大與秦鐘不同。\end{note}方能如此度日。今日正遇天氣晴明,又值家中無事,遂帶了一個婆子,坐上車,來家裏走走,瞧瞧寡嫂並侄兒。
\end{parag}


\begin{parag}
    閒話之間,金榮的母親偏提起昨日賈家學房裏的那事,從頭至尾,一五一十都向他小姑子說了。這璜大奶奶不聽則已,聽了,一時怒從心上起,說道:“這秦鍾小崽子是賈門的親戚,難道榮兒不是賈門的親戚?\begin{note}己側:這賈門的親戚比那賈門的親戚。\end{note}人都別忒勢利了,況且都作的是什麼有臉的好事!就是寶玉,也犯不上向著他到這個樣。等我去到東府瞧瞧我們珍大奶奶,再向秦鍾他姐姐說說,叫他評評這個理。\begin{note}己側:未必能如此說。\end{note}\begin{note}蒙側:狗仗人勢者,開口便有多少必勝之談,事要三思,免勞後悔。\end{note}\begin{note}靖側:這個理怕不能評。\end{note}這金榮的母親聽了這話,急的了不得,忙說道:“這都是我的嘴快,告訴了姑奶奶了,求姑奶奶別去,別管他們誰是誰非。\begin{note}己側:不論誰是誰非,有錢就可矣。蒙側:胡氏可謂善哉!\end{note}倘或鬧起來,怎麼在那裏站得住。若是站不住,家裏不但不能請先生,反倒在他身上添出許多嚼用來呢。”璜大奶奶聽了,說道:“那裏管得許多,你等我說了,看是怎麼樣!”也不容他嫂子勸,一面叫老婆子瞧了車,就坐上往寧府裏來。\begin{note}蒙側:何等氣派,何等聲勢,有射石飲羽之力,動天搖地,如項喑吒。\end{note}
\end{parag}


\begin{parag}
    到了寧府,進了車門,到了東邊小角門前下了車,進去見了賈珍之妻尤氏。也未敢氣高,殷殷勤勤敘過寒溫,說了些閒話,方問道:\begin{note}蒙側:何故興致索然?\end{note}“今日怎麼沒見蓉大奶奶?”\begin{note}己側:何不叫秦鐘的姐姐?\end{note}尤氏說道:“他這些日子不知怎麼著,經期有兩個多月沒來。叫大夫瞧了,又說並不是喜。那兩日,到了下半天就懶待動,話也懶待說,眼神也發眩。我說他:‘你且不必拘禮,早晚不必照例上來,你就好生養養罷。就是有親戚一家兒來,有我呢。就有長輩們怪你,等我替你告訴。’連蓉哥我都囑咐了,我說:‘你不許累他,不許招他生氣,叫他靜靜的養養就好了。\begin{note}蒙側:只一絲不露。\end{note}他要想什麼喫,只管到我這裏取來。倘或我這裏沒有,只管望你璉二嬸子那裏要去。倘或他有個好和歹,你再要娶這麼一個媳婦,這麼個模樣兒,這麼個性情的人兒,打著燈籠也沒地方找去。’\begin{note}己側:還有這麼個好小舅子。\end{note}他這爲人行事,那個親戚,那個一家的長輩不喜歡他?所以我這兩日好不煩心,焦的我了不得。偏偏今日早晨他兄弟來瞧他,誰知那小孩子家不知好歹,看見他姐姐身上不大爽快,就有事也不當告訴他,別說是這麼一點子小事,就是你受了一萬分的委曲,也不該向他說纔是。誰知他們昨兒學房裏打架,不知是那裏附學來的一個人欺侮了他了。\begin{note}己側:眼前竟像不知者。蒙側:文筆之妙,妙至於此。本是璜大奶奶不忿來告,又偏從尤氏口中先出,確是秦鍾之語,且是情理必然,形勢逼近。孫悟空七十二變,未有如此靈巧活跳。\end{note}裏頭還有些不乾不淨的話,都告訴了他姐姐。嬸子,你是知道那媳婦的:雖則見了人有說有笑,會行事兒,他可心細,心又重,不拘聽見個什麼話兒,都要度量個三日五夜才罷。這病就是打這個秉性上頭思慮出來的。今兒聽見有人欺負了他兄弟,又是惱,又是氣。惱的是那羣混帳狐朋狗友的扯是搬非、調三惑四那些人;氣的是他兄弟不學好,不上心念書,以致如此學裏吵鬧。他聽了這事,今日索性連早飯也沒喫。我聽見了,我方到他那邊安慰了他一會子,又勸解了他兄弟一會子。我叫他兄弟到那府裏去找寶玉去了,我纔看著他吃了半盞燕窩湯,我纔過來了。嬸子,你說我心焦不心焦?\begin{note}蒙側:這會子金氏聽了這話,心裏當如何料理,實在悔殺從前高興。天下事不得不豫爲三思,先爲防漸。\end{note}況且如今又沒個好大夫,我想到他這病上,我心裏倒象針扎似的。你們知道有什麼好大夫沒有?”\begin{note}蒙側:作無意相問語,是逼近一分,則金氏猶不免當爲分拆。一逼之下,實無可贅之詞。\end{note}
\end{parag}


\begin{parag}
    金氏聽了這半日話,把方纔在他嫂子家的那一團要向秦氏理論的盛氣,早嚇的都丟在爪窪國去了。\begin{note}己側:又何必爲金母著急。\end{note}\begin{note}該批:吾爲趨炎附勢,仰人鼻息者一嘆。\end{note}聽見尤氏問他有知道好大夫的話,連忙答道:“我們這麼聽著,實在也沒見人說有個好大夫。如今聽起大奶奶這個來,定不得還是喜呢。嫂子倒別教人混治。倘或認錯了,這可是了不得的。”尤氏道:“可不是呢。”正是說話間,賈珍從外進來,見了金氏,便向尤氏問道:“這不是璜大奶奶麼?”金氏向前給賈珍請了安。賈珍向尤氏說道:“讓這大妹妹吃了飯去。”賈珍說著話,就過那屋裏去了。\begin{note}靖眉:不知心中作何想。\end{note}金氏此來,原要向秦氏說說秦鍾欺負了他侄兒的事,聽見秦氏有病,不但不能說,亦且不敢提了。況且賈珍尤氏又待的很好,反轉怒爲喜,又說了一會子話兒,方家去了。\begin{note}蒙側:金氏何面目再見江東父老?然而如金氏者,世不乏其人。\end{note}
\end{parag}


\begin{parag}
    金氏去後,賈珍方過來坐下,問尤氏道:“今日他來,有什麼說的事情麼?”尤氏答道:“倒沒說什麼。一進來的時候,臉上倒象有些著了惱的氣色似的,及說了半天話,又提起媳婦這病,他倒漸漸的氣色平定了。你又叫讓他喫飯,他聽見媳婦這麼病,也不好意思只管坐著,又說了幾句閒話兒就去了,倒沒求什麼事。如今且說媳婦這病,你到那裏尋一個好大夫來與他瞧瞧要緊,可別耽誤了。現今咱們家走的這羣大夫,那裏要得?\begin{note}蒙側:醫毒。非止近世,從古有之。\end{note}一個個都是聽著人的口氣兒,人怎麼說,他也添幾句文話兒說一遍。可倒殷勤的很,三四個人一日輪流著倒有四五遍來看脈。他們大家商量著立個方子,吃了也不見效,倒弄得一日換四五遍衣裳,坐起來見大夫,其實於病人無益。”賈珍說道:“可是。這孩子也糊塗,何必脫脫換換的,倘再著了涼,更添一層病,那還了得。衣裳任憑是什麼好的,可又值什麼,孩子的身子要緊,就是一天穿一套新的,也不值什麼。我正進來要告訴你:方纔馮紫英來看我,他見我有些抑鬱之色,問我是怎麼了。我才告訴他說,媳婦忽然身子有好大的不爽快,因爲不得個好太醫,斷不透是喜是病,又不知有妨礙無妨礙,所以我這兩日心裏著實著急。馮紫英因說起他有一個幼時從學的先生,姓張名友士,學問最淵博的,更兼醫理極深,且能斷人的生死。\begin{note}己側:未必能如此。\end{note}\begin{note}蒙側:舉薦人的通套,多是如此說。\end{note}今年是上京給他兒子來捐官,現在他家住著呢。這麼看來,竟是合該媳婦的病在他手裏除災亦未可知。我即刻差人拿我的名帖請去了。\begin{note}蒙側:父母之心,昊天罔極。\end{note}今日倘或天晚了不能來,明日想必一定來。況且馮紫英又即刻回家親自去求他,務必叫他來瞧瞧。等這個張先生來瞧了再說罷。”
\end{parag}


\begin{parag}
    尤氏聽了,心中甚喜,因說道:“後日是太爺的壽日,到底怎麼辦?”賈珍說道:“我方纔到了太爺那裏去請安,兼請太爺來家來受一受一家子的禮。太爺因說道:‘我是清淨慣了的,我不願意往你們那是非場中去鬧去。你們必定說是我的生日,要叫我去受衆人些頭,莫過你把我從前注的《陰騭文》給我令人好好的寫出來刻了,比叫我無故受衆人的頭還強百倍呢。倘或後日這兩日一家子要來,你就在家裏好好的款待他們就是了。也不必給我送什麼東西來,連你後日也不必來,你要心中不安,你今日就給我磕了頭去。\begin{note}蒙側:將寫可卿之好事多慮。至於天生之文中,轉出好清靜之一番議論,清新醒目,立見不凡。\end{note}倘或後日你要來,又跟隨多少人來鬧我,我必和你不依。’如此說了又說,後日我是再不敢去的了。且叫來升來,吩咐他預備兩日的筵席。”尤氏因叫人叫了賈蓉來:“吩咐來升照舊例預備兩日的筵席,要豐豐富富的。你再親自到西府裏去請老太太、大太太、二太太和你璉二嬸子來逛逛。你父親今日又聽見一個好大夫,業已打發人請去了,想必明日必來。你可將他這些日子的病症細細的告訴他。”
\end{parag}


\begin{parag}
    賈蓉一一的答應著出去了。正遇著方纔去馮紫英家請那先生的小子回來了,因回道:“奴才方纔到了馮大爺家,拿了老爺的名帖請那先生去。那先生說道:‘方纔這裏大爺也向我說了。但是今日拜了一天的客,纔回到家,此時精神實在不能支持,就是去到府上也不能看脈。’他說等調息一夜,明日務必到府。\begin{note}蒙側:醫生多是推三阻四,拿腔做調。\end{note}他又說,他‘醫學淺薄,本不敢當此重薦,因我們馮大爺和府上的大人既已如此說了,又不得不去,你先替我回明大人就是了。大人的名帖實不敢當。’仍叫奴才拿回來了。哥兒替奴才回一聲兒罷。”賈蓉轉身復進去,回了賈珍尤氏的話,方出來叫了來升來,吩咐他預備兩日的筵席的話。來升聽畢,自去照例料理。不在話下。
\end{parag}


\begin{parag}
    且說次日午間,人回道:“請的那張先生來了。”賈珍遂延入大廳坐下。茶畢,方開言道:“昨承馮大爺示知老先生人品學問,又兼深通醫學,小弟不勝欽仰之至。”張先生道:“晚生粗鄙下士,本知見淺陋,昨因馮大爺示知,大人家第謙恭下士,又承呼喚,敢不奉命。但毫無實學,倍增顏汗。”賈珍道:“先生何必過謙。就請先生進去看看兒婦,仰仗高明,以釋下懷。”於是,賈蓉同了進去。到了賈蓉居室,見了秦氏,向賈蓉說道:“這就是尊夫人了?”賈蓉道:“正是。請先生坐下,讓我把賤內的病說一說再看脈如何?”那先生道:“依小弟的意思,竟先看過脈再說的爲是。我是初造尊府的,本也不曉得什麼,但是我們馮大爺務必叫小弟過來看看,小弟所以不得不來。如今看了脈息,看小弟說的是不是,再將這些日子的病勢講一講,大家斟酌一個方兒,可用不可用,那時大爺再定奪。”賈蓉道:“先生實在高明,如今恨相見之晚。就請先生看一看脈息,可治不可治,以便使家父母放心。”於是家下媳婦們捧過大迎枕來,一面給秦氏拉著袖口,露出脈來。先生方伸手按在右手脈上,調息了至數,寧神細診了有半刻的工夫,方換過左手,亦復如是。診畢脈息,說道:“我們外邊坐罷。”
\end{parag}


\begin{parag}
    賈蓉於是同先生到外間房裏牀上坐下,一個婆子端了茶來。賈蓉道:“先生請茶。”於是陪先生吃了茶,遂問道:“先生看這脈息,還治得治不得?”先生道:“看得尊夫人這脈息:左寸沉數,左關沉伏,右寸細而無力,右關需而無神。其左寸沉數者,乃心氣虛而生火;左關沉伏者,乃肝家氣滯血虧。右寸細而無力者,乃肺經氣分太虛;右關需而無神者,乃脾土被肝木剋制。心氣虛而生火者,應現經期不調,夜間不寐。肝家血虧氣滯者,必然肋下疼脹,月信過期,心中發熱。肺經氣分太虛者,頭目不時眩暈,寅卯間必然自汗,如坐舟中。脾土被肝木剋制者,必然不思飲食,精神倦怠,四肢痠軟。據我看這脈息,應當有這些症候纔對。或以這個脈爲喜脈,則小弟不敢從其教也。”旁邊一個貼身伏侍的婆子道:“何嘗不是這樣呢。真正先生說的如神,倒不用我們告訴了。如今我們家裏現有好幾位太醫老爺瞧著呢,都不能的當真切的這麼說。有一位說是喜,有一位說是病,這位說不相干,那位說怕冬至,總沒有個準話兒。求老爺明白指示指示。”
\end{parag}


\begin{parag}
    那先生笑\begin{note}蒙側:說是了,不覺笑,描出神情跳躍,如見其人。\end{note}道:“大奶奶這個症候,可是那衆位耽擱了。要在初次行經的日期就用藥治起來,不但斷無今日之患,而且此時已全愈了。如今既是把病耽誤到這個地位,也是應有此災。依我看來,這病尚有三分治得。吃了我的藥看,若是夜裏睡的著覺,那時又添了二分拿手了。據我看這脈息:大奶奶是個心性高強聰明不過的人,聰明忒過,則不如意事常有,不如意事常有,則思慮太過。此病是憂慮傷脾,肝木忒旺,經血所以不能按時而至。大奶奶從前的行經的日子問一問,斷不是常縮,必是常長的。\begin{note}蒙側:恐不合其方,又加一番議論,一方合爲藥,一爲夭亡症,無一字一句不前後照應者。\end{note}是不是?”這婆子答道:“可不是,從沒有縮過,或是長兩日三日,以至十日都長過。”先生聽了道:“妙啊!這就是病源了。從前若能夠以養心調經之藥服之,何至於此。這如今明顯出一個水虧木旺的症候來。待用藥看看。”於是寫了方子,遞與賈蓉,上寫的是:
\end{parag}


\begin{qute2sp}
    益氣養榮補脾和肝湯


    人蔘二錢 白朮二錢土炒 雲苓三錢 熟地四錢


    歸身二錢酒洗 白芍二錢 川芎錢半 黃芪三錢


    香附米二錢制 醋柴胡八分 懷山藥二錢炒 真阿膠二錢蛤粉炒


    延胡索錢半酒炒 炙甘草八分


    引用建蓮子七粒去心 紅棗二枚
\end{qute2sp}


\begin{parag}
    賈蓉看了,說:“高明的很。還要請教先生,這病與性命終久有妨無妨?”先生笑道:“大爺是最高明的人。人病到這個地位,非一朝一夕的症候,吃了這藥也要看醫緣了。依小弟看來,今年一冬是不相干的。總是過了春分,就可望全愈了。”賈蓉也是個聰明人,也不往下細問了。於是賈蓉送了先生去了,方將這藥方子並脈案都給賈珍看了,說的話也都回了賈珍並尤氏了。尤氏向賈珍說道:“從來大夫不象他說的這麼痛快,想必用的藥也不錯。”賈珍道:“人家原不是混飯喫久慣行醫的人。因爲馮紫英我們好,他好容易求了他來了。既有這個人,媳婦的病或者就能好了。他那方子上有人蔘,就用前日買的那一斤好的罷。”賈蓉聽畢話,方出來叫人打藥去煎給秦氏喫。不知秦氏服了此藥病勢如何,下回分解。
\end{parag}


\begin{parag}
    \begin{note}蒙:欲速可卿之死,故先有惡奴之兇頑,而後及以秦鍾來告,層層克入,點露其用心過當,種種文章逼之。雖貧女得居富室,諸凡遂心,終有不能不夭亡之道。我不知作者於著筆時何等妙心繡口,能道此無礙法語,令人不禁眼花撩亂。\end{note}
\end{parag}
