\chap{一百零一}{大觀園月夜感幽魂 散花寺神籤驚異兆}


\begin{parag}
    卻說鳳姐回至房中,見賈璉尚未回來,便分派那管辦探春行裝奩事的一干人。那天已有黃昏以後,因忽然想起探春來,要瞧瞧他去,便叫豐兒與兩個丫頭跟着,頭裏一個丫頭打着燈籠。走出門來,見月光已上,照耀如水。鳳姐便命打燈籠的“回去罷。”因而走至茶房窗下,聽見裏面有人嘁嘁喳喳的,又似哭,又似笑,又似議論什麼的。鳳姐知道不過是家下婆子們又不知搬什麼是非,心內大不受用,便命小紅進去,裝做無心的樣子細細打聽着,用話套出原委來。小紅答應着去了。鳳姐只帶着豐兒來至園門前,門尚未關,只虛虛的掩着。於是主僕二人方推門進去,只見園中月色比着外面更覺明朗,滿地下重重樹影,杳無人聲,甚是淒涼寂靜。剛欲往秋爽齋這條路來,只聽忽的一聲風過,吹的那樹枝上落葉滿園中唰喇喇的作響,枝梢上吱嘍嘍發哨,將那些寒鴉宿鳥都驚飛起來。鳳姐吃了酒,被風一吹,只覺身上發噤起來。那豐兒也把頭一縮說:“好冷!”鳳姐也撐不住,便叫豐兒:“快回去把那件銀鼠坎肩兒拿來,我在三姑娘那裏等着。”豐兒巴不得一聲,也要回去穿衣裳來,答應了一聲,回頭就跑了。
\end{parag}


\begin{parag}
    鳳姐剛舉步走了不遠,只覺身後咈咈哧哧,似有聞嗅之聲,不覺頭髮森然豎了起來。由不得回頭一看,只見黑油油一個東西在後面伸着鼻子聞他呢,那兩隻眼睛恰似燈光一般。鳳姐嚇的魂不附體,不覺失聲的咳了一聲。卻是一隻大狗。那狗抽頭回身,拖着一個掃帚尾巴,一氣跑上大土山上方站住了,回身猶向鳳姐拱爪兒。鳳姐兒此時心跳神移,急急的向秋爽齋來。已將來至門口,方轉過山子,只見迎面有一個人影兒一恍。鳳姐心中疑惑,心裏想着必是那一房裏的丫頭,便問:“是誰?”問了兩聲,並沒有人出來,已經嚇得神魂飄蕩。恍恍忽忽的似乎背後有人說道:“嬸孃連我也不認得了!”鳳姐忙回頭一看,只見這人形容俊俏,衣履風流,十分眼熟,只是想不起是那房那屋裏的媳婦來。只聽那人又說道:“嬸孃只管享榮華受富貴的心盛,把我那年說的立萬年永遠之基都付於東洋大海了。”鳳姐聽說,低頭尋思,總想不起。那人冷笑道:“嬸孃那時怎樣疼我了,如今就忘在九霄雲外了。”鳳姐聽了,此時方想起來是賈蓉的先妻秦氏,便說道:“噯呀,你是死了的人哪,怎麼跑到這裏來了呢!”啐了一口,方轉回身,腳下不防一塊石頭絆了一跤,猶如夢醒一般,渾身汗如雨下。雖然毛髮悚然,心中卻也明白,只見小紅豐兒影影綽綽的來了。鳳姐恐怕落人的褒貶,連忙爬起來說道:“你們做什麼呢,去了這半天?快拿來我穿上罷。”一面豐兒走至跟前伏侍穿上,小紅過來攙扶。鳳姐道:“我纔到那裏,他們都睡了。咱們回去罷。”一面說,一面帶了兩個丫頭急急忙忙回到家中。賈璉已回來了,只是見他臉上神色更變,不似往常,待要問他,又知他素日性格,不敢突然相問,只得睡了。至次日五更,賈璉就起來要往總理內庭都檢點太監裘世安家來打聽事務。因太早了,見桌上有昨日送來的抄報,便拿起來閒看。第一件是雲南節度使王忠一本,新獲了一起私帶神槍火藥出邊事,共有十八名人犯。頭一名鮑音,口稱系太師鎮國公賈化家人。第二件蘇州刺史李孝一本,參劾縱放家奴,倚勢凌辱軍民,以致因奸不遂殺死節婦一家人命三口事。兇犯姓時名福,自稱繫世襲三等職銜賈范家人。賈璉看見這兩件,心中早又不自在起來,待要看第三件,又恐遲了不能見裘世安的面,因此急急的穿了衣服,也等不得喫東西,恰好平兒端上茶來,喝了兩口,便出來騎馬走了。
\end{parag}


\begin{parag}
    平兒在房內收拾換下的衣服。此時鳳姐尚未起來,平兒因說道:“今兒夜裏我聽着奶奶沒睡什麼覺,我這會子替奶奶捶着,好生打個盹兒罷。”鳳姐半日不言語。平兒料着這意思是了,便爬上炕來坐在身邊輕輕的捶着。才捶了幾拳,那鳳姐剛有要睡之意,只聽那邊大姐兒哭了。鳳姐又將眼睜開,平兒連向那邊叫道:“李媽,你到底是怎麼着?姐兒哭了。你到底拍着他些。你也忒好睡了。”那邊李媽從夢中驚醒,聽得平兒如此說,心中沒好氣,只得狠命拍了幾下,口裏嘟嘟噥噥的罵道:“真真的小短命鬼兒,放着屍不挺,三更半夜嚎你孃的喪!”一面說,一面咬牙便向那孩子身上擰了一把。那孩子哇的一聲大哭起來了。鳳姐聽見,說“了不得!你聽聽,他該挫磨孩子了。你過去把那黑心的養漢老婆下死勁的打他幾下子,把妞妞抱過來。”平兒笑道:“奶奶別生氣,他那裏敢挫磨姐兒,只怕是不提防錯碰了一下子也是有的。這會子打他幾下子沒要緊,明兒叫他們背地裏嚼舌根,倒說三更半夜打人。”鳳姐聽了,半日不言語,長嘆一聲說道:“你瞧瞧,這會子不是我十旺八旺的呢!明兒我要是死了,剩下這小孽障,還不知怎麼樣呢!”平兒笑道:“奶奶這怎麼說!大五更的,何苦來呢!”鳳姐冷笑道:“你那裏知道,我是早已明白了。我也不久了。雖然活了二十五歲,人家沒見的也見了,沒喫的也吃了,也算全了。所有世上有的也都有了。氣也算賭盡了,強也算爭足了,就是壽字兒上頭缺一點兒,也罷了。”平兒聽說,由不的滾下淚來。鳳姐笑道:“你這會子不用假慈悲,我死了你們只有歡喜的。你們一心一計和和氣氣的,省得我是你們眼裏的刺似的。只有一件,你們知好歹只疼我那孩子就是了。”平兒聽說這話,越發哭的淚人似的。鳳姐笑道:“別扯你孃的臊了,那裏就死了呢。哭的那麼痛!我不死還叫你哭死了呢。”平兒聽說,連忙止住哭,道:“奶奶說得這麼傷心。”一面說,一面又捶,半日不言語,鳳姐又朦朧睡去。
\end{parag}


\begin{parag}
    平兒方下炕來要去,只聽外面腳步響。誰知賈璉去遲了,那裘世安已經上朝去了,不遇而回,心中正沒好氣,進來就問平兒道:“那些人還沒起來呢麼?”平兒回說:“沒有呢。”賈璉一路摔簾子進來,冷笑道:“好,好,這會子還都不起來,安心打擂臺打撒手兒!”一迭聲又要喫茶。平兒忙倒了一碗茶來。原來那些丫頭老婆見賈璉出了門又復睡了,不打諒這會子回來,原不曾預備。平兒便把溫過的拿了來。賈璉生氣,舉起碗來,譁啷一聲摔了個粉碎。
\end{parag}


\begin{parag}
    鳳姐驚醒,唬了一身冷汗,噯喲一聲,睜開眼,只見賈璉氣狠狠的坐在旁邊,平兒彎着腰拾碗片子呢。鳳姐道:“你怎麼就回來了?”問了一聲,半日不答應,只得又問一聲。賈璉嚷道:“你不要我回來,叫我死在外頭罷!”鳳姐笑道:“這又是何苦來呢!常時我見你不象今兒回來的快,問你一聲,也沒什麼生氣的。”賈璉又嚷道:“又沒遇見,怎麼不快回來呢!”鳳姐笑道:“沒有遇見,少不得奈煩些,明兒再去早些兒,自然遇見了。”賈璉嚷道:“我可不喫着自己的飯替人家趕獐子呢。我這裏一大堆的事沒個動秤兒的,沒來由爲人家的事,瞎鬧了這些日子,當什麼呢!正經那有事的人還在家裏受用,死活不知,還聽見說要鑼鼓喧天的擺酒唱戲做生日呢。我可瞎跑他孃的腿子!”一面說,一面往地下啐了一口,又罵平兒。鳳姐聽了,氣的乾嚥,要和他分證,想了一想,又忍住了,勉強陪笑道:“何苦來生這麼大氣,大清早起和我叫喊什麼。誰叫你應了人家的事?你既應了,就得耐煩些,少不得替人家辦辦。也沒見這個人自己有爲難的事還有心腸唱戲擺酒的鬧!”賈璉道:“你可說麼,你明兒倒也問問他!”鳳姐詫異道:“問誰?”賈璉道:“問誰!問你哥哥。”鳳姐道:“是他嗎?”賈璉道:“可不是他,還有誰呢!”鳳姐忙問道:“他又有什麼事叫你替他跑?”賈璉道:“你還在罈子裏呢。”鳳姐道:“真真這就奇了,我連一個字兒也不知道。”賈璉道:“你怎麼能知道呢,這個事連太太和姨太太還不知道呢。頭一件怕太太和姨太太不放心,二則你身上又常嚷不好,所以我在外頭壓住了,不叫裏頭知道的。說起來真真可人惱!你今兒不問我,我也不便告訴你。你打諒你哥哥行事象個人呢,你知道外頭人都叫他什麼?”鳳姐道:“叫他什麼?”賈璉道:“叫他什麼,叫他‘忘仁’!”鳳姐撲哧的一笑:“他可不叫王仁叫什麼呢。”賈璉道:“你打諒那個王仁嗎,是忘了仁義禮智信的那個‘忘仁’哪!”鳳姐道:“這是什麼人這麼刻薄嘴兒遭塌人。”賈璉道:“不是遭塌他嗎,今兒索性告訴你,你也不知道知道你那哥哥的好處,到底知道他給他二叔做生日啊!”鳳姐想了一想道:“噯喲,可是呵,我還忘了問你,二叔不是冬天的生日嗎?我記得年年都是寶玉去。前者老爺升了,二叔那邊送過戲來,我還偷偷兒的說,二叔爲人是最嗇刻的,比不得大舅太爺。他們各自家裏還烏眼雞似的。不麼,昨兒大舅太爺沒了,你瞧他是個兄弟,他還出了個頭兒攬了個事兒嗎!所以那一天說,趕他的生日咱們還他一班子戲,省了親戚跟前落虧欠。如今這麼早就做生日,也不知道是什麼意思。”賈璉道:“你還作夢呢。他一到京,接着舅太爺的首尾就開了一個吊,他怕咱們知道攔他,所以沒告訴咱們,弄了好幾千銀子。後來二舅嗔着他,說他不該一網打盡。他喫不住了,變了個法子就指着你們二叔的生日撒了個網,想着再弄幾個錢好打點二舅太爺不生氣,也不管親戚朋友冬天夏天的,人家知道不知道,這麼丟臉!你知道我起早爲什麼?這如今因海疆的事情御史參了一本,說是大舅太爺的虧空,本員已故,應着落其弟王子勝,侄王仁賠補。爺兒兩個急了,找了我給他們託人情。我見他們嚇的那麼個樣兒,再者又關係太太和你,我才應了。想着找找總理內庭都檢點老裘替辦辦,或者前任後任挪移挪移。偏又去晚了,他進裏頭去了,我白起來跑了一趟。他們家裏還那裏定戲擺酒呢。你說說,叫人生氣不生氣!”
\end{parag}


\begin{parag}
    鳳姐聽了,才知王仁所行如此。但他素性要強護短,聽賈璉如此說,便道:“憑他怎麼樣,到底是你的親大舅兒。再者,這件事死的大太爺活的二叔都感激你。罷了,沒什麼說的,我們家的事,少不得我低三下四的求你了,省的帶累別人受氣,背地裏罵我。”說着,眼淚早流下來,掀開被窩一面坐起來,一面挽頭髮,一面披衣裳。賈璉道:“你倒不用這麼着,是你哥哥不是人,我並沒說你呀。況且我出去了,你身上又不好,我都起來了,他們還睡覺。咱們老輩子有這個規矩麼!你如今作好好先生不管事了。我說了一句你就起來,明兒我要嫌這些人,難道你都替了他們麼。好沒意思啊!”鳳姐聽了這些話,才把淚止住了,說道:“天呢不早了,我也該起來了。你有這麼說的,你替他們家在心的辦辦,那就是你的情分了。再者也不光爲我,就是太太聽見也喜歡。”賈璉道:“是了,知道了。‘大蘿蔔還用屎澆’。”平兒道:“奶奶這麼早起來做什麼,那一天奶奶不是起來有一定的時候兒呢。爺也不知是那裏的邪火,拿着我們出氣。何苦來呢,奶奶也算替爺掙夠了,那一點兒不是奶奶擋頭陣。不是我說,爺把現成兒的也不知吃了多少,這會子替奶奶辦了一點子事,又關會着好幾層兒呢,就是這麼拿糖作醋的起來,也不怕人家寒心。況且這也不單是奶奶的事呀。我們起遲了,原該爺生氣,左右到底是奴才呀。奶奶跟前盡着身子累的成了個病包兒了,這是何苦來呢。”說着,自己的眼圈兒也紅了。那賈璉本是一肚子悶氣,那裏見得這一對嬌妻美妾又尖利又柔情的話呢,便笑道:“夠了,算了罷。他一個人就夠使的了,不用你幫着。左右我是外人,多早晚我死了,你們就清淨了。”鳳姐道:“你也別說那個話,誰知道誰怎麼樣呢。你不死我還死呢,早死一天早心淨。”說着,又哭起來。平兒只得又勸了一回。那時天已大亮,日影橫窗。賈璉也不便再說,站起來出去了。
\end{parag}


\begin{parag}
    這裏鳳姐自己起來,正在梳洗,忽見王夫人那邊小丫頭過來道:“太太說了,叫問二奶奶今日過舅太爺那邊去不去?要去,說叫二奶奶同着寶二奶奶一路去呢。”鳳姐因方纔一段話,已經灰心喪意,恨孃家不給爭氣,又兼昨夜園中受了那一驚,也實在沒精神,便說道:“你先回太太去,我還有一兩件事沒辦清,今日不能去。況且他們那又不是什麼正經事。寶二奶奶要去各自去罷。”小丫頭答應着,回去回覆了。不在話下。
\end{parag}


\begin{parag}
    且說鳳姐梳了頭,換了衣服,想了想,雖然自己不去,也該帶個信兒。再者,寶釵還是新媳婦,出門子自然要過去照應照應的。於是見過王夫人,支吾了一件事,便過來到寶玉房中。只見寶玉穿着衣服歪在炕上,兩個眼睛呆呆的看寶釵梳頭。鳳姐站在門口,還是寶釵一回頭看見了,連忙起身讓坐。寶玉也爬起來,鳳姐才笑嘻嘻的坐下。寶釵因說麝月道”你們瞧着二奶奶進來也不言語聲兒。”麝月笑着道:“二奶奶頭裏進來就擺手兒不叫言語麼。”鳳姐因向寶玉道:“你還不走,等什麼呢。沒見這麼大人了還是這麼小孩子氣的。人家各自梳頭,你爬在旁邊看什麼?成日家一塊子在屋裏還看不夠?也不怕丫頭們笑話。”說着,哧的一笑,又瞅着他咂嘴兒。寶玉雖也有些不好意思,還不理會,把個寶釵直臊的滿臉飛紅,又不好聽着,又不好說什麼,只見襲人端過茶來,只得搭訕着自己遞了一袋煙。鳳姐兒笑着站起來接了,道:“二妹妹,你別管我們的事,你快穿衣服罷。”寶玉一面也搭訕着找這個,弄那個。鳳姐道:“你先去罷,那裏有個爺們等着奶奶們一塊兒走的理呢。”寶玉道:“我只是嫌我這衣裳不大好,不如前年穿着老太太給的那件雀金呢好。”鳳姐因慪他道:“你爲什麼不穿?”寶玉道:“穿着太早些。”鳳姐忽然想起,自悔失言,幸虧寶釵也和王家是內親,只是那些丫頭們跟前已經不好意思了。襲人卻接着說道:“二奶奶還不知道呢,就是穿得,他也不穿了。”鳳姐兒道:“這是什麼原故?”襲人道:“告訴二奶奶,真真是我們這位爺的行事都是天外飛來的。那一年因二舅太爺的生日,老太太給了他這件衣裳,誰知那一天就燒了。我媽病重了,我沒在家。那時候還有晴雯妹妹呢,聽見說病着整給他補了一夜,第二天老太太纔沒瞧出來呢。去年那一天上學天冷,我叫焙茗拿了去給他披披。誰知這位爺見了這件衣裳想起晴雯來了,說了總不穿了,叫我給他收一輩子呢。”鳳姐不等說完,便道:“你提晴雯,可惜了兒的,那孩子模樣兒手兒都好,就只嘴頭子利害些。偏偏兒的太太不知聽了那裏的謠言,活活兒的把個小命兒要了。還有一件事,那一天我瞧見廚房裏柳家的女人他女孩兒,叫什麼五兒,那丫頭長的和晴雯脫了個影兒似的。我心裏要叫他進來,後來我問他媽,他媽說是很願意。我想着寶二爺屋裏的小紅跟了我去,我還沒還他呢,就把五兒補過來。平兒說太太那一天說了,凡象那個樣兒的都不叫派到寶二爺屋裏呢。我所以也就擱下了。這如今寶二爺也成了家了,還怕什麼呢,不如我就叫他進來。可不知寶二爺願意不願意?要想着晴雯,只瞧見這五兒就是了。”寶玉本要走,聽見這些話已呆了。襲人道:“爲什麼不願意,早就要弄了來的,只是因爲太太的話說的結實罷了。”鳳姐道:“那麼着明兒我就叫他進來。太太的跟前有我呢。”寶玉聽了,喜不自勝,才走到賈母那邊去了。這裏寶釵穿衣服。鳳姐兒看他兩口兒這般恩愛纏綿,想起賈璉方纔那種光景,好不傷心,坐不住,便起身向寶釵笑道:“我和你向老太太屋裏去罷。”笑着出了房門,一同來見賈母。
\end{parag}


\begin{parag}
    寶玉正在那裏回賈母往舅舅家去。賈母點頭說道:“去罷,只是少喫酒,早些回來。你身子纔好些。”寶玉答應着出來,剛走到院內,又轉身回來向寶釵耳邊說了幾句不知什麼。寶釵笑道:“是了,你快去罷。”將寶玉催着去了。這賈母和鳳姐寶釵說了沒三句話,只見秋紋進來傳說:“二爺打發焙茗轉來,說請二奶奶。”寶釵說道:“他又忘了什麼,又叫他回來?”秋紋道:“我叫小丫頭問了,焙茗說是‘二爺忘了一句話,二爺叫我回來告訴二奶奶:若是去呢,快些來罷,若不去呢,別在風地裏站着。’”說的賈母鳳姐並地下站着的衆老婆子丫頭都笑了。寶釵飛紅了臉,把秋紋啐了一口,說道:“好個糊塗東西!這也值得這樣慌慌張張跑了來說。”秋紋也笑着回去叫小丫頭去罵焙茗。那焙茗一面跑着,一面回頭說道:“二爺把我巴巴的叫下馬來,叫回來說的。我若不說,回來對出來又罵我了。這會子說了,他們又罵我。”那丫頭笑着跑回來說了。賈母向寶釵道:“你去罷,省得他這麼記掛。”說的寶釵站不住,又被鳳姐慪他頑笑,沒好意思,才走了。
\end{parag}


\begin{parag}
    只見散花寺的姑子大了來了,給賈母請安,見過了鳳姐,坐着喫茶。賈母因問他:“這一向怎麼不來?”大了道:“因這幾日廟中作好事,有幾位誥命夫人不時在廟裏起坐,所以不得空兒來。今日特來回老祖宗,明兒還有一家作好事,不知老祖宗高興不高興,若高興也去隨喜隨喜。”賈母便問:“做什麼好事?”大了道:“前月爲王大人府裏不乾淨,見神見鬼的,偏生那太太夜間又看見去世的老爺。因此昨日在我廟裏告訴我,要在散花菩薩跟前許願燒香,做四十九天的水陸道場,保佑家口安寧,亡者昇天,生者獲福。所以我不得空兒來請老太太的安。”卻說鳳姐素日最厭惡這些事的,自從昨夜見鬼,心中總是疑疑惑惑的,如今聽了大了這些話,不覺把素日的心性改了一半,已有三分信意,便問大了道:“這散花菩薩是誰?他怎麼就能避邪除鬼呢?”大了見問,便知他有些信意,便說道:“奶奶今日問我,讓我告訴奶奶知道。這個散花菩薩來歷根基不淺,道行非常。生在西天大樹國中,父母打柴爲生。養下菩薩來,頭長三角,眼橫四目,身長三尺,兩手拖地。父母說這是妖精,便棄在冰山之後了。誰知這山上有一個得道的老猢猻出來打食,看見菩薩頂上白氣沖天,虎狼遠避,知道來歷非常,便抱回洞中撫養。誰知菩薩帶了來的聰慧,禪也會談,與猢猻天天談道參禪,說的天花散漫繽紛。至一千年後飛昇了。至今山上猶見談經之處天花散漫,所求必靈,時常顯聖,救人苦厄。因此世人才蓋了廟,塑了像供奉。”鳳姐道:“這有什麼憑據呢?”大了道:“奶奶又來搬駁了。一個佛爺可有什麼憑據呢?就是撒謊,也不過哄一兩個人罷咧,難道古往今來多少明白人都被他哄了不成。奶奶只想,惟有佛家香火歷來不絕,他到底是祝國祝民,有些靈驗,人才信服。”鳳姐聽了大有道理,因道:“既這麼,我明兒去試試。你廟裏可有籤?我去求一簽,我心裏的事簽上批的出?批的出來我從此就信了。”大了道:“我們的籤最是靈的,明兒奶奶去求一簽就知道了。”賈母道:“既這麼着,索性等到後日初一你再去求。”說着,大了吃了茶,到王夫人各房裏去請了安,回去不提。
\end{parag}


\begin{parag}
    這裏鳳姐勉強扎掙着,到了初一清早,令人預備了車馬,帶着平兒並許多奴僕來至散花寺。大了帶了衆姑子接了進去。獻茶後,便洗手至大殿上焚香。那鳳姐兒也無心瞻仰聖像,一秉虔誠,磕了頭,舉起籤筒默默的將那見鬼之事並身體不安等故祝告了一回。才搖了三下,只聽唰的一聲,筒中攛出一支籤來。於是叩頭拾起一看,只見寫着“第三十三籤,上上大吉。”大了忙查籤薄看時,只見上面寫着“王熙鳳衣錦還鄉”。鳳姐一見這幾個字,喫一大驚,驚問大了道:“古人也有叫王熙鳳的麼?”大了笑道:“奶奶最是通今博古的,難道漢朝的王熙鳳求官的這一段事也不曉得?”周瑞家的在旁笑道:“前年李先兒還說這一回書的,我們還告訴他重着奶奶的名字不要叫呢。”鳳姐笑道:“可是呢,我倒忘了。”說着,又瞧底下的,寫的是:
\end{parag}


\begin{poem}
    \begin{pl}
        去國離鄉二十年,於今衣錦返家園。
    \end{pl}


    \begin{pl}
        蜂採百花成蜜後,爲誰辛苦爲誰甜!
    \end{pl}

\end{poem}


\begin{parag}
    行人至,音信遲,訟宜和,婚再議。看完也不甚明白。大了道:“奶奶大喜。這一簽巧得很,奶奶自幼在這裏長大,何曾回南京去了。如今老爺放了外任,或者接家眷來,順便還家,奶奶可不是‘衣錦還鄉’了?”一面說,一面抄了個籤經交與丫頭。鳳姐也半疑半信的。大了擺了齋來,鳳姐只動了一動,放下了要走,又給了香銀。大了苦留不住,只得讓他走了。鳳姐回至家中,見了賈母王夫人等,問起籤來,命人一解,都歡喜非常,”或者老爺果有此心,咱們走一趟也好。”鳳姐兒見人人這麼說,也就信了。不在話下。
\end{parag}


\begin{parag}
    卻說寶玉這一日正睡午覺,醒來不見寶釵,正要問時,只見寶釵進來。寶玉問道:“那裏去了?半日不見。”寶釵笑道:“我給鳳姐姐瞧一回籤。”寶玉聽說,便問是怎麼樣的。寶釵把籤帖唸了一回,又道:“家中人人都說好的。據我看,這‘衣錦還鄉’四字裏頭還有原故,後來再瞧罷了。”寶玉道:“你又多疑了,妄解聖意。‘衣錦還鄉’四字從古至今都知道是好的,今兒你又偏生看出緣故來了。依你說,這‘衣錦還鄉’還有什麼別的解說?”寶釵正要解說,只見王夫人那邊打發丫頭過來請二奶奶。寶釵立刻過去。未知何事,下回分解。
\end{parag}