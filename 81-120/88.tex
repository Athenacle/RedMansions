\chap{八十八}{博庭欢宝玉赞孤儿 正家法贾珍鞭悍仆}



\begin{parag}
    却说惜春正在那里揣摩棋谱,忽听院内有人叫彩屏,不是别人却是鸳鸯的声儿。彩屏出去,同着鸳鸯进来。那鸳鸯却带着一个小丫头,提了一个小黄绢包儿。惜春笑问道:“什么事?”鸳鸯道:“老太太因明年八十一岁,是个暗九。许下一场九昼夜的功德,发心要写三千六百五十零一部《金刚经》。这已发出外面人写了。但是俗说《金刚经》就象那道家的符壳,《心经》才算是符胆。故此《金刚经》内必要插着《心经》,更有功德。老太太因《心经》是更要紧的,观自在又是女菩萨,所以要几个亲丁奶奶姑娘们写上三百六十五部,如此又虔诚,又洁净。咱们家中除了二奶奶,头一宗他当家没有空儿,二宗他也写不上来,其余会写字的,不论写得多少,连东府珍大奶奶姨娘们都分了去,本家里头自不用说。”惜春听了,点头道:“别的我做不来,若要写经,我最信心的。你搁下喝茶罢。”鸳鸯才将那小包儿搁在桌上,同惜春坐下。彩屏倒了一钟茶来。惜春笑问道:“你写不写?”鸳鸯道:“姑娘又说笑话了。那几年还好,这三四年来姑娘见我还拿了拿笔儿么。”惜春道:“这却是有功德的。”鸳鸯道:“我也有一件事:向来服侍老太太安歇后,自己念上米佛,已经念了三年多了。我把这个米收好,等老太太做功德的时候,我将他衬在里头供佛施食,也是我一点诚心。”惜春道:“这样说来,老太太做了观音,你就是龙女了。”鸳鸯道:“那里跟得上这个分儿。却是除了老太太,别的也服侍不来,不晓得前世什么缘分儿。”说着要走,叫小丫头把小绢包打开,拿出来道:“这素纸一扎是写《心经》的。”又拿起一子儿藏香道:“这是叫写经时点着写的。”惜春都应了。
\end{parag}


\begin{parag}
    鸳鸯遂辞了出来,同小丫头来至贾母房中,回了一遍。看见贾母与李纨打双陆,鸳鸯旁边瞧着。李纨的骰子好,掷下去把老太太的锤打下了好几个去。鸳鸯抿着嘴儿笑。忽见宝玉进来,手中提了两个细蔑丝的小笼子,笼内有几个蝈蝈儿,说道:“我听说老太太夜里睡不着,我给老太太留下解解闷。”贾母笑道:“你别瞅着你老子不在家,你只管淘气。”宝玉笑道:“我没有淘气。”贾母道:“你没淘气,不在学房里念书,为什么又弄这个东西呢。”宝玉道:“不是我自己弄的。今儿因师父叫环儿和兰儿对对子,环儿对不来,我悄悄的告诉了他。他说了,师父喜欢,夸了他两句。他感激我的情,买了来孝敬我的。我才拿了来孝敬老太太的。”贾母道:“他没有天天念书么,为什么对不上来?对不上来就叫你儒大爷爷打他的嘴巴子,看他臊不臊。你也够受了,不记得你老子在家时,一叫做诗做词,唬的倒象个小鬼儿似的,这会子又说嘴了。那环儿小子更没出息,求人替做了,就变着方法儿打点人。这么点子孩子就闹鬼闹神的,也不害臊,赶大了还不知是个什么东西呢。”说的满屋子人都笑了。贾母又问道:“兰小子呢,做上来了没有?这该环儿替他了,他又比他小了。是不是?”宝玉笑道:“他倒没有,却是自己对的。”贾母道:“我不信,不然就也是你闹了鬼了。如今你还了得,‘羊群里跑出骆驼来了,就只你大。’你又会做文章了。”宝玉笑道:“实在是他作的。师父还夸他明儿一定有出息呢。老太太不信,就打发人叫了他来亲自试试,老太太就知道了。”贾母道:“果然这么着我才喜欢。我不过怕你撒谎。既是他做的,这孩子明儿大概还有一点儿出息。”因看着李纨,又想起贾珠来,”这也不枉你大哥哥死了,你大嫂子拉扯他一场,日后也替你大哥哥顶门壮户。”说到这里,不禁流下泪来。李纨听了这话,却也动心,只是贾母已经伤心,自己连忙忍住泪笑劝道:“这是老祖宗的余德,我们托着老祖宗的福罢咧。只要他应得了老祖宗的话,就是我们的造化了。老祖宗看着也喜欢,怎么倒伤起心来呢。”因又回头向宝玉道:“宝叔叔明儿别这么夸他,他多大孩子,知道什么。你不过是爱惜他的意思,他那里懂得,一来二去,眼大心肥,那里还能够有长进呢。”贾母道:“你嫂子这也说的是。就只他还太小呢,也别逼梏紧了他。小孩子胆儿小,一时逼急了,弄出点子毛病来,书倒念不成,把你的工夫都白糟踏了。”贾母说到这里,李纨却忍不住扑簌簌掉下泪来,连忙擦了。
\end{parag}


\begin{parag}
    只见贾环贾兰也都进来给贾母请了安。贾兰又见过他母亲,然后过来在贾母旁边侍立。贾母道:“我刚才听见你叔叔说你对的好对子,师父夸你来着。”贾兰也不言语,只管抿着嘴儿笑。鸳鸯过来说道:“请示老太太,晚饭伺候下了。”贾母道:“请你姨太太去罢。”琥珀接着便叫人去王夫人那边请薛姨妈。这里宝玉贾环退出。素云和小丫头们过来把双陆收起。李纨尚等着伺候贾母的晚饭,贾兰便跟着他母亲站着。贾母道:“你们娘儿两个跟着我吃罢。”李纨答应了。一时摆上饭来,丫鬟回来禀道:“太太叫回老太太,姨太太这几天浮来暂去,不能过来回老太太,今日饭后家去了。”于是贾母叫贾兰在身旁边坐下,大家吃饭,不必细述。
\end{parag}


\begin{parag}
    却说贾母刚吃完了饭,盥漱了,歪在床上说闲话儿。只见小丫头子告诉琥珀,琥珀过来回贾母道:“东府大爷请晚安来了。”贾母道:“你们告诉他,如今他办理家务乏乏的,叫他歇着去罢。我知道了。”小丫头告诉老婆子们,老婆子才告诉贾珍。贾珍然后退出。到了次日,贾珍过来料理诸事。门上小厮陆续回了几件事,又一个小厮回道:“庄头送果子来了。”贾珍道:“单子呢?”那小厮连忙呈上。贾珍看时,上面写着不过是时鲜果品,还夹带菜蔬野味若干在内。贾珍看完,问向来经管的是谁。门上的回道:“是周瑞。”便叫周瑞:“照帐点清,送往里头交代。等我把来帐抄下一个底子,留着好对。”又叫”告诉厨房,把下菜中添几宗给送果子的来人,照常赏饭给钱。”周瑞答应了。一面叫人搬至凤姐儿院子里去,又把庄上的帐同果子交代明白。出去了一回儿,又进来回贾珍道:“才刚来的果子,大爷曾点过数目没有?”贾珍道:“我那里有工夫点这个呢。给了你帐,你照帐点就是了。”周瑞道:“小的曾点过,也没有少,也不能多出来。大爷既留下底子,再叫送果子来的人问问,他这帐是真的假的。”贾珍道:“这是怎么说,不过是几个果子罢咧,有什么要紧。我又没有疑你。”说着,只见鲍二走来,磕了一个头,说道:“求大爷原旧放小的在外头伺候罢。”贾珍道:“你们这又是怎么着?”鲍二道:“奴才在这里又说不上话来。”贾珍道:“谁叫你说话。”鲍二道:“何苦来,在这里作眼睛珠儿。”周瑞接口道:“奴才在这里经管地租庄子,银钱出入每年也有三五十万来往,老爷太太奶奶们从没有说过话的,何况这些零星东西。若照鲍二说起来,爷们家里的田地房产都被奴才们弄完了。”贾珍想道:“必是鲍二在这里拌嘴,不如叫他出去。”因向鲍二说道:“快滚罢。”又告诉周瑞说:“你也不用说了,你干你的事罢。”二人各自散了。
\end{parag}


\begin{parag}
    贾珍正在厢房里歇着,听见门上闹的翻江搅海。叫人去查问,回来说道:“鲍二和周瑞的干儿子打架。”贾珍道:“周瑞的干儿子是谁?”门上的回道:“他叫何三,本来是个没味儿的,天天在家里喝酒闹事,常来门上坐着。听见鲍二与周瑞拌嘴,他就插在里头。”贾珍道:“这却可恶。把鲍二和那个什么何几给我一块儿捆起来!周瑞呢?”门上的回道:“打架时他先走了。”贾珍道:“给我拿了来!这还了得了!”众人答应了。正嚷着,贾琏也回来了,贾珍便告诉了一遍。贾琏道:“这还了得!”又添了人去拿周瑞。周瑞知道躲不过,也找到了。贾珍便叫都捆上。贾琏便向周瑞道:“你们前头的话也不要紧,大爷说开了,很是了。为什么外头又打架!你们打架已经使不得,又弄个野杂种什么何三来闹,你不压伏压伏他们,倒竟走了。”就把周瑞踢了几脚。贾珍道:“单打周瑞不中用。”喝命人把鲍二和何三各人打了五十鞭子,撵了出去,方和贾琏两个商量正事。下人背地里便生出许多议论来:也有说贾珍护短的,也有说不会调停的,也有说他本不是好人,前儿尤家姊妹弄出许多丑事来,那鲍二不是他调停着二爷叫了来的吗,这会子又嫌鲍二不济事,必是鲍二的女人伏侍不到了。人多嘴杂,纷纷不一。
\end{parag}


\begin{parag}
    却说贾政自从在工部掌印,家人中尽有发财的。那贾芸听见了,也要插手弄一点事儿,便在外头说了几个工头,讲了成数,便买了些时新绣货,要走凤姐儿门子。凤姐正在房中听见丫头们说:“大爷二爷都生了气,在外头打人呢。”凤姐听了,不知何故,正要叫人去问问,只见贾琏已进来了,把外面的事告诉了一遍。凤姐道:“事情虽不要紧,但这风俗儿断不可长。此刻还算咱们家里正旺的时候儿,他们就敢打架。以后小辈儿们当了家,他们越发难制伏了。前年我在东府里,亲眼见过焦大吃的烂醉,躺在台阶子底下骂人,不管上上下下一混汤子的混骂。他虽是有过功的人,到底主子奴才的名分,也要存点儿体统才好。珍大奶奶不是我说是个老实头,个个人都叫他养得无法无天的。如今又弄出一个什么鲍二,我还听见是你和珍大爷得用的人,为什么今儿又打他呢?”贾琏听了这话刺心,便觉讪讪的,拿话来支开,借有事,说着就走了。
\end{parag}


\begin{parag}
    小红进来回道:“芸二爷在外头要见奶奶。”凤姐一想,”他又来做什么?”便道:“叫他进来罢。”小红出来,瞅着贾芸微微一笑。贾芸赶忙凑近一步问道:“姑娘替我回了没有?”小红红了脸,说道:“我就是见二爷的事多。”贾芸道:“何曾有多少事能到里头来劳动姑娘呢。就是那一年姑娘在宝二叔房里,我才和姑娘——”小红怕人撞见,不等说完,赶忙问道:“那年我换给二爷的一块绢子,二爷见了没有?”那贾芸听了这句话,喜的心花俱开,才要说话,只见一个小丫头从里面出来,贾芸连忙同着小红往里走。两个人一左一右,相离不远,贾芸悄悄的道:“回来我出来还是你送出我来,我告诉你还有笑话儿呢。”小红听了,把脸飞红,瞅了贾芸一眼,也不答言。同他到了凤姐门口,自己先进去回了,然后出来,掀起帘子点手儿,口中却故意说道:“奶奶请芸二爷进来呢。”
\end{parag}


\begin{parag}
    贾芸笑了一笑,跟着他走进房来,见了凤姐儿,请了安,并说:“母亲叫问好。”凤姐也问了他母亲好。凤姐道:“你来有什么事?”贾芸道:“侄儿从前承婶娘疼爱,心上时刻想着,总过意不去。欲要孝敬婶娘,又怕婶娘多想。如今重阳时候,略备了一点儿东西。婶娘这里那一件没有,不过是侄儿一点孝心。只怕婶娘不肯赏脸。”凤姐儿笑道:“有话坐下说。”贾芸才侧身坐了,连忙将东西捧着搁在旁边桌上。凤姐又道:“你不是什么有余的人,何苦又去花钱。我又不等着使。你今日来意是怎么个想头儿,你倒是实说。”贾芸道:“并没有别的想头儿,不过感念婶娘的恩惠,过意不去罢咧。”说着微微的笑了。凤姐道:“不是这么说。你手里窄,我很知道,我何苦白白儿使你的。你要我收下这个东西,须先和我说明白了。要是这么含着骨头露着肉的,我倒不收。”贾芸没法儿,只得站起来陪着笑儿说道:“并不是有什么妄想。前几日听见老爷总办陵工,侄儿有几个朋友办过好些工程,极妥当的,要求婶娘在老爷跟前提一提。办得一两种,侄儿再忘不了婶娘的恩典。若是家里用得着,侄儿也能给婶娘出力。”凤姐道:“若是别的我却可以作主。至于衙门里的事,上头呢,都是堂官司员定的,底下呢,都是那些书办衙役们办的。别人只怕插不上手。连自己的家人,也不过跟着老爷伏侍伏侍。就是你二叔去,亦只是为的是各自家里的事,他也并不能搀越公事。论家事,这里是踩一头儿橇一头儿的,连珍大爷还弹压不住,你的年纪儿又轻,辈数儿又小,那里缠的清这些人呢。况且衙门里头的事差不多儿也要完了,不过吃饭瞎跑。你在家里什么事作不得,难道没了这碗饭吃不成。我这是实在话,你自己回去想想就知道了。你的情意我已经领了,把东西快拿回去,是那里弄来的,仍旧给人家送了去罢。”正说着,只见奶妈子一大起带了巧姐儿进来。那巧姐儿身上穿得锦团花簇,手里拿着好些顽意儿,笑嘻嘻走到凤姐身边学舌。贾芸一见,便站起来笑盈盈的赶着说道:“这就是大妹妹么?你要什么好东西不要?”那巧姐儿便哑的一声哭了。贾芸连忙退下。凤姐道:“乖乖不怕。”连忙将巧姐揽在怀里道:“这是你芸大哥哥,怎么认起生来了。”贾芸道:“妹妹生得好相貌,将来又是个有大造化的。”那巧姐儿回头把贾芸一瞧,又哭起来,迭连几次。贾芸看这光景坐不住,便起身告辞要走。凤姐道:“你把东西带了去罢。”贾芸道:“这一点子婶娘还不赏脸?”凤姐道:“你不带去,我便叫人送到你家去。芸哥儿,你不要这么样,你又不是外人,我这里有机会,少不得打发人去叫你,没有事也没法儿,不在乎这些东东西西上的。”贾芸看见凤姐执意不受,只得红着脸道:“既这么着,我再找得用的东西来孝敬婶娘罢。”凤姐儿便叫小红拿了东西,跟着贾芸送出来。
\end{parag}


\begin{parag}
    贾芸走着,一面心中想道:“人说二奶奶利害,果然利害。一点儿都不漏缝,真正斩钉截铁,怪不得没有后世。这巧姐儿更怪,见了我好象前世的冤家似的。真正晦气,白闹了这么一天。”小红见贾芸没得彩头,也不高兴,拿着东西跟出来。贾芸接过来,打开包儿拣了两件,悄悄的递给小红。小红不接,嘴里说道:“二爷别这么着,看奶奶知道了,大家倒不好看。”贾芸道:“你好生收着罢,怕什么,那里就知道了呢。你若不要,就是瞧不起我了。”小红微微一笑,才接过来,说道:“谁要你这些东西,算什么呢。”说了这句话,把脸又飞红了。贾芸也笑道:“我也不是为东西,况且那东西也算不了什么。”说着话儿,两个已走到二门口。贾芸把下剩的仍旧揣在怀内。小红催着贾芸道:“你先去罢,有什么事情,只管来找我。我今日在这院里了,又不隔手。”贾芸点点头儿,说道:“二奶奶太利害,我可惜不能长来。刚才我说的话,你横竖心里明白,得了空儿再告诉你罢。”小红满脸羞红,说道:“你去罢,明儿也长来走走。谁叫你和他生疏呢。”贾芸道:“知道了。”贾芸说着出了院门。这里小红站在门口,怔怔的看他去远了,才回来了。
\end{parag}


\begin{parag}
    却说凤姐在房中吩咐预备晚饭,因又问道:“你们熬了粥了没有?”丫鬟们连忙去问,回来回道:“预备了。”凤姐道:“你们把那南边来的糟东西弄一两碟来罢。”秋桐答应了,叫丫头们伺候。平儿走来笑道:“我倒忘了,今儿晌午奶奶在上头老太太那边的时候,水月庵的师父打发人来,要向奶奶讨两瓶南小菜,还要支用几个月的月银,说是身上不受用。我问那道婆来着:‘师父怎么不受用?’他说:‘四五天了,前儿夜里因那些小沙弥小道士里头有几个女孩子睡觉没有吹灯,他说了几次不听。那一夜看见他们三更以后灯还点着呢,他便叫他们吹灯,个个都睡着了,没有人答应,只得自己亲自起来给他们吹灭了。回到炕上,只见有两个人,一男一女,坐在炕上。他赶着问是谁,那里把一根绳子往他脖子上一套,他便叫起人来。众人听见,点上灯火一齐赶来,已经躺在地下,满口吐白沫子,幸亏救醒了。此时还不能吃东西,所以叫来寻些小菜儿的。’我因奶奶不在房中,不便给他。我说:‘奶奶此时没有空儿,在上头呢,回来告诉。’便打发他回去了。才刚听见说起南菜,方想起来了,不然就忘了。”凤姐听了,呆了一呆,说道:“南菜不是还有呢,叫人送些去就是了。那银子过一天叫芹哥来领就是了。”又见小红进来回道:“才刚二爷差人来,说是今晚城外有事,不能回来,先通知一声。”凤姐道:“是了。”
\end{parag}


\begin{parag}
    说着,只听见小丫头从后面喘吁吁的嚷着直跑到院子里来,外面平儿接着,还有几个丫头们,咕咕唧唧的说话。凤姐道:“你们说什么呢?”平儿道:“小丫头子有些胆怯,说鬼话。”凤姐叫那一个小丫头进来,问道:“什么鬼话?”那丫头道:“我才刚到后边去叫打杂儿的添煤,只听得三间空屋子里哗喇哗喇的响,我还道是猫儿耗子,又听得嗳的一声,象个人出气儿的似的。我害怕,就跑回来了。”凤姐骂道:“胡说!我这里断不兴说神说鬼,我从来不信这些个话。快滚出去罢。”那小丫头出去了。凤姐便叫彩明将一天零碎日用帐对过一遍,时已将近二更。大家又歇了一回,略说些闲话,遂叫各人安歇去罢。凤姐也睡下了。将近三更,凤姐似睡不睡,觉得身上寒毛一乍,自己惊醒了,越躺着越发起渗来,因叫平儿秋桐过来作伴。二人也不解何意。那秋桐本来不顺凤姐,后来贾琏因尤二姐之事不大爱惜他了,凤姐又笼络他,如今倒也安静,只是心里比平儿差多了,外面情儿。今见凤姐不受用,只得端上茶来。凤姐喝了一口,道:“难为你,睡去罢,只留平儿在这里就够了。”秋桐却要献勤儿,因说道:“奶奶睡不着,倒是我们两个轮流坐坐也使得。”凤姐一面说,一面睡着了。平儿秋桐看见凤姐已睡,只听得远远的鸡叫了,二人方都穿着衣服略躺了一躺,就天亮了,连忙起来伏侍凤姐梳洗。凤姐因夜中之事,心神恍惚不宁,只是一味要强,仍然扎挣起来。正坐着纳闷,忽听个小丫头子在院里问道:“平姑娘在屋里么?”平儿答应了一声,那小丫头掀起帘子进来,却是王夫人打发过来来找贾琏,说:“外头有人回要紧的官事。老爷才出了门,太太叫快请二爷过去呢。”凤姐听见唬了一跳。未知何事,下回分解。
\end{parag}