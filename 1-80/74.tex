\chap{七十四}{惑奸讒抄檢大觀園 矢孤介杜絕寧國府}


\begin{parag}
    \begin{note}蒙回前總:司棋一事在七十一回敘明,暗用山石伏線,七十三回用繡春囊在山石上一逗便住,至此回可直敘去,又用無數曲折漸漸逼來,及至司棋,忽然頓住,接到入畫,文氣如黃河出崑崙,橫流數萬裏,九曲至龍門,又有孟門、呂梁峽束,不得入海。是何等奇險怪特文字,令我拜服!\end{note}
\end{parag}


\begin{parag}
    話說平兒聽迎春說了正自好笑,忽見寶玉也來了。原來管廚房柳家媳婦之妹,也因放頭開賭得了不是。這園中有素與柳家不睦的,\begin{note}庚雙夾:前文已卯之伏線。\end{note}便又告出柳家來,說他和他妹子是夥計,雖然他妹子出名,其實賺了錢兩個人平分。因此鳳姐要治柳家之罪。那柳家的因得此信,便慌了手腳,因思素與怡紅院人最爲深厚,故走來悄悄地央求晴雯金星玻璃告訴了寶玉。寶玉因思內中迎春之乳母也現有此罪,不若來約同迎春討情,比自己獨去單爲柳家說情又更妥當,故此前來。忽見許多人在此,見他來時,都問:“你的病可好了?跑來作什麼?”寶玉不便說出討情一事,只說:“來看二姐姐。”當下衆人也不在意,且說些閒話。平兒便出去辦累絲金鳳一事。那王住兒媳婦緊跟在後,口內百般央求,只說:“姑娘好歹口內超生,我橫豎去贖了來。”平兒笑道:“你遲也贖,早也贖,既有今日,何必當初。你的意思得過去就過去了。既是這樣,我也不好意思告人,趁早去贖了來交與我送去,我一字不提。”王住兒媳婦聽說,方放下心來,就拜謝,又說:“姑娘自去貴幹,我趕晚拿了來,先回了姑娘,再送去,如何?”平兒道:“趕晚不來,可別怨我。”說畢,二人方分路各自散了。
\end{parag}


\begin{parag}
    平兒到房,鳳姐問他:“三姑娘叫你作什麼?”平兒笑道:“三姑娘怕奶奶生氣,叫我勸著奶奶些,問奶奶這兩天可喫些什麼。”鳳姐笑道:“倒是他還記掛著我。剛纔又出來了一件事:有人來告柳二媳婦和他妹子通同開局,凡妹子所爲,都是他作主。我想,你素日肯勸我‘多一事不如省一事’,就可閒一時心,自己保養保養也是好的。我因聽不進去,果然應了些,先把太太得罪了,而且自己反賺了一場病。如今我也看破了,隨他們鬧去罷,橫豎還有許多人呢。我白操一會子心,倒惹的萬人咒罵。我且養病要緊,便是好了,我也作個好好先生,得樂且樂,得笑且笑,一概是非都憑他們去罷。\begin{note}庚雙夾:歷來世人到此作此想,但悔不及矣。可傷可嘆。\end{note}所以我只答應著知道了,白不在我心上。”平兒笑道:“奶奶果然如此,便是我們的造化。”
\end{parag}


\begin{parag}
    一語未了,只見賈璉進來,拍手嘆氣道:“好好的又生事。前兒我和鴛鴦借當,那邊太太怎麼知道了。纔剛太太叫過我去,叫我不管那裏先遷挪二百銀子,做八月十五日節間使用。我回沒處遷挪。太太就說:‘你沒有錢就有地方遷挪,我白和你商量,你就搪塞我,你就說沒地方。前兒一千銀子的當是那裏的?連老太太的東西你都有神通弄出來,這會子二百銀子,你就這樣。幸虧我沒和別人說去。’我想太太分明不短,何苦來要尋事奈何人。”鳳姐兒道:“那日並沒一個外人,誰走了這個消息。”平兒聽了,也細想那日有誰在此,想了半日,笑道:“是了。那日說話時沒一個外人,但晚上送東西來的時節,老太太那邊傻大姐的娘也可巧來送漿洗衣服。他在下房裏坐了一會子,見一大箱子東西,自然要問,必是小丫頭們不知道,說了出來,也未可知。”\begin{note}庚雙夾:奇奇怪怪,從何處轉至素日上,真如常山之蛇。\end{note}因此便喚了幾個小丫頭來問,那日誰告訴呆大姐的娘。衆小丫頭慌了,都跪下賭咒發誓,說:“自來也不敢多說一句話。有人凡問什麼,都答應不知道。這事如何敢多說。”鳳姐詳情說:“他們必不敢,倒別委屈了他們。如今且把這事靠後,且把太太打發了去要緊。寧可咱們短些,又別討沒意思。”因叫平兒: “把我的金項圈拿來,且去暫押二百銀子來送去完事。”賈璉道:“越性多押二百,咱們也要使呢。”鳳姐道:“很不必,我沒處使錢。這一去還不知指那一項贖呢。”平兒拿去,吩咐一個人喚了旺兒媳婦來領去,不一時拿了銀子來。賈璉親自送去,不在話下。
\end{parag}


\begin{parag}
    這裏鳳姐和平兒猜疑,終是誰人走的風聲,竟擬不出人來。鳳姐兒又道:“知道這事還是小事,怕的是小人趁便又造非言,生出別的事來。當緊那邊正和鴛鴦結下仇了,如今聽得他私自借給璉二爺東西,那起小人眼饞肚飽,連沒縫兒的雞蛋還要下蛆呢,如今有了這個因由,恐怕又造出些沒天理的話來也定不得。在你璉二爺還無妨,只是鴛鴦正經女兒,帶累了他受屈,豈不是咱們的過失。”平兒笑道:“這也無妨。鴛鴦借東西看的是奶奶,並不爲的是二爺。一則鴛鴦雖應名是他私情,其實他是回過老太太的。老太太因怕孫男弟女多,這個也借,那個也要,到跟前撒個嬌兒,和誰要去,因此只裝不知道。\begin{note}庚雙夾:奇文神文!豈世人想得出者?前文雲“一想不若私自拿出”,賈母其睡夢中之人矣。蓋此等事作者曾經,批者曾經,實系一寫往事,非特造出,故弄新筆,究竟記不神也。鴛鴦借物一回於此便結了。\end{note}縱鬧了出來,究竟那也無礙。”鳳姐兒道:“理固如此。只是你我是知道的,那不知道的,焉得不生疑呢。”
\end{parag}


\begin{parag}
    一語未了,人報:“太太來了。”鳳姐聽了詫異,不知爲何事親來,與平兒等忙迎出來。只見王夫人氣色更變,\begin{note}庚夾:奇。\end{note}只帶一個貼己的小丫頭走來,一語不發,走至裏間坐下。鳳姐忙奉茶,因陪笑問道:“太太今日高興,到這裏逛逛。”王夫人喝命:“平兒出去!”平兒見了這般,著慌不知怎麼樣了,忙應了一聲,帶著衆小丫頭一齊出去,在房門外站住,越性將房門掩了,自己坐在臺磯上,所有的人,一個不許進去。鳳姐也著了慌,不知有何等事。只見王夫人含著淚,從袖內擲出一個香袋子來,說:“你瞧。”鳳姐忙拾起一看,見是十錦春意香袋,也嚇了一跳,忙問:“太太從那裏得來?”王夫人見問,越發淚如雨下,顫聲說道:“我從那裏得來!我天天坐在井裏,拿你當個細心人,所以我才偷個空兒。誰知你也和我一樣。這樣的東西大天白日明擺在園裏山石上,被老太太的丫頭拾著,不虧你婆婆遇見,早已送到老太太跟前去了。我且問你,這個東西如何遺在那裏來?”\begin{note}庚雙夾:奇問。\end{note}鳳姐聽得,也更了顏色,忙問:“太太怎知是我的?”\begin{note}庚雙夾:問甚的。\end{note}王夫人又哭又嘆說道:“你反問我!你想,一家子除了你們小夫小妻,餘者老婆子們,要這個何用?再女孩子們是從那裏得來?自然是那璉兒不長進下流種子那裏弄來。你們又和氣。當作一件頑意兒,年輕人兒女閨房私意是有的,你還和我賴!幸而園內上下人還不解事,尚未揀得。倘或丫頭們揀著,你姊妹看見,這還了得。不然有那小丫頭們揀著,出去說是園內揀著的,外人知道,這性命臉面要也不要?”
\end{parag}


\begin{parag}
    鳳姐聽說,又急又愧,登時紫漲了麪皮,便依炕沿雙膝跪下,也含淚訴道:“太太說的固然有理,我也不敢辯我並無這樣的東西。但其中還要求太太細詳其理:那香袋是外頭僱工仿著內工繡的,帶子穗子一概是市賣貨。我便年輕不尊重些,也不要這勞什子,自然都是好的,此其一。二者這東西也不是常帶著的,我縱有,也只好在家裏,焉肯帶在身上各處去?況且又在園裏去,個個姊妹我們都肯拉拉扯扯,倘或露出來,不但在姊妹前,就是奴才看見,我有什麼意思?我雖年輕不尊重,亦不能糊塗至此。三則論主子內我是年輕媳婦,算起奴才來,比我更年輕的又不止一個人了。況且他們也常進園,晚間各人家去,焉知不是他們身上的?四則除我常在園裏之外,還有那邊太太常帶過幾個小姨娘來,如嫣紅翠雲等人,皆系年輕侍妾,他們更該有這個了。還有那邊珍大嫂子,他不算甚老外,他也常帶過佩鳳等人來,焉知又不是他們的?五則園內丫頭太多,保的住個個都是正經的不成?也有年紀大些的知道了人事,或者一時半刻人查問不到偷著出去,或藉著因由同二門上小幺兒們打牙犯嘴,外頭得了來的,也未可知。如今不但我沒此事,就連平兒我也可以下保的。太太請細想。”王夫人聽了這一席話大近情理,因嘆道:“你起來。我也知道你是大家小姐出身,焉得輕薄至此,不過我氣急了,拿了話激你。但如今卻怎麼處?你婆婆纔打發人封了這個給我瞧,說是前日從傻大姐手裏得的,把我氣了個死。”鳳姐道:“太太快別生氣。若被衆人覺察了,保不定老太太不知道。且平心靜氣暗暗訪察,才得確實,縱然訪不著,外人也不能知道。這叫作‘胳膊折在袖內’。如今惟有趁著賭錢的因由革了許多的人這空兒,把周瑞媳婦旺兒媳婦等四五個貼近不能走話的人安插在園裏,以查賭爲由。再如今他們的丫頭也太多了,保不住人大心大,生事作耗,等鬧出事來,反悔之不及。如今若無故裁革,不但姑娘們委屈煩惱,就連太太和我也過不去。不如趁此機會,以後凡年紀大些的,或有些咬牙難纏的,拿個錯兒攆出去配了人。一則保得住沒有別的事,二則也可省些用度。太太想我這話如何?”王夫人嘆道:“你說的何嘗不是,但從公細想,你這幾個姊妹也甚可憐了。\begin{note}庚雙夾:猶雲“可憐”,妙!又在別人視之,今古無比矣;若在榮府論,實不能比先矣。\end{note}也不用遠比,只說如今你林妹妹的母親,未出閣時,是何等的嬌生慣養,是何等的金尊玉貴,那才象個千金小姐的體統。如今這幾個姊妹,不過比人家的丫頭略強些罷了。\begin{note}庚雙夾:所謂“觀於海者難爲水”,俗子謂王夫人不知足,是不可矣,又認作太過,真……\end{note}通共每人只有兩三個丫頭象個人樣,餘者縱有四五個小丫頭子,竟是廟裏的小鬼。如今還要裁革了去,不但於我心不忍,只怕老太太未必就依。雖然艱難,難不至此。我雖沒受過大榮華富貴,比你們是強的。如今我寧可省些,別委曲了他們。以後要省儉先從我來倒使的。如今且叫人傳了周瑞家的等人進來,就吩咐他們快快暗地訪拿這事要緊。”鳳姐聽了,即喚平兒進來吩咐出去。
\end{parag}


\begin{parag}
    一時,周瑞家的與吳興家的、鄭華家的、來旺家的、來喜家的現在五家陪房進來,餘者皆在南方各有執事。\begin{note}庚雙夾:又伏一筆。\end{note}王夫人正嫌人少不能勘察,忽見邢夫人的陪房王善保家的走來,方纔正是他送香囊來的。王夫人向來看視邢夫人之得力心腹人等原無二意,\begin{note}庚雙夾:大書看下人猶如此,可知待邢夫人矣。\end{note}今見他來打聽此事,十分關切,\begin{note}庚雙夾:小人外是內非,委皆如此。\end{note}便向他說:“你去回了太太,也進園內照管照管,不比別人又強些。” 這王善保家正因素日進園去那些丫鬟們不大趨奉他,他心裏大不自在,要尋他們的故事又尋不著,恰好生出這事來,以爲得了把柄。又聽王夫人委託,正撞在心坎上,說:“這個容易。不是奴才多話,論理這事該早嚴緊的。太太也不大往園裏去,這些女孩子們一個個倒象受了封誥似的。他們就成了千金小姐了。鬧下天來,誰敢哼一聲兒。不然,就調唆姑娘的丫頭們,說欺負了姑娘們了,誰還耽得起。”王夫人道:“這也有的常情,跟姑娘的丫頭原比別的嬌貴些。你們該勸他們。連主子們的姑娘不教導尚且不堪,何況他們。”王善保家的道:“別的都還罷了。太太不知道,一個寶玉屋裏的晴雯,那丫頭仗著他生的模樣兒比別人標緻些。又生了一張巧嘴,天天打扮的象個西施的樣子,在人跟前能說慣道,掐尖要強。一句話不投機,他就立起兩個騷眼睛來罵人,妖妖趫趫,大不成個體統。”\begin{note}庚雙夾:活畫出晴雯來。可知已前知晴雯必應遭妒者,可憐可傷,竟死矣。\end{note}王夫人聽了這話,猛然觸動往事,便問鳳姐道:“上次我們跟了老太太進園逛去,有一個水蛇腰,\begin{note}庚雙夾:妙妙,好腰!\end{note}削肩膀,\begin{note}庚雙夾:妙妙,好肩!俗雲:“水蛇腰則遊曲小也。”又云:“美人無肩。”又曰:“肩若削成。”皆是美之形也。凡寫美人皆用俗筆反筆,與他書不同也。\end{note}眉眼又有些像你林妹妹的,\begin{note}庚雙夾:更好,形容盡矣。\end{note}正在那裏罵小丫頭。我的心裏很看不上那狂樣子,因同老太太走,我不曾說得。後來要問是誰,又偏忘了。今日對了坎兒,這丫頭想必就是他了。”鳳姐道:“若論這些丫頭們,共總比起來,都沒晴雯生得好。論舉止言語,他原有些輕薄。方纔太太說的倒很像他,我也忘了那日的事,不敢亂說。”王善保家的便道:“不用這樣,此刻不難叫了他來太太瞧瞧。”王夫人道:“寶玉房裏常見我的只有襲人麝月,這兩個笨笨的倒好。若有這個,他自不敢來見我的。我一生最嫌這樣人,況且又出來這個事。好好的寶玉,倘或叫這蹄子勾引壞了,那還了得。”因叫自己的丫頭來,吩咐他到園裏去,“只說我說有話問他們,留下襲人麝月伏侍寶玉不必來,有一個晴雯最伶俐,叫他即刻快來。你不許和他說什麼。”
\end{parag}


\begin{parag}
    小丫頭子答應了,走入怡紅院,正值晴雯身上不自在,睡中覺纔起來,正發悶,聽如此說,只得隨了他來。素日這些丫鬟皆知王夫人最嫌趫妝豔飾語薄言輕者,故晴雯不敢出頭。今因連日不自在,\begin{note}庚雙夾:音神之至!所謂“魂早離會”矣,將死之兆也。若俗筆必雲十分妝飾,今雲不自在,想無掛礙之心,更不入王夫人之眼也。\end{note}並沒十分妝飾,自爲無礙。\begin{note}庚雙夾:好!\end{note}及到了鳳姐房中,王夫人一見他釵軃鬢松,衫垂帶褪,有春睡捧心之遺風,而且形容面貌恰是上月的那人,不覺勾起方纔的火來。王夫人原是天真爛漫之人,喜怒出於心臆,不比那些飾詞掩意之人,今既真怒攻心,又勾起往事,便冷笑道:“好個美人!真象個病西施了。你天天作這輕狂樣兒給誰看?你乾的事,打量我不知道呢!我且放著你,自然明兒揭你的皮!寶玉今日可好些?”晴雯一聽如此說,心內大異,便知有人暗算了他。雖然著惱,只不敢作聲。他本是個聰敏過頂的人,\begin{note}庚雙夾:深罪聰明,不錯一筆。\end{note}見問寶玉可好些,他便不肯以實話對,只說:“我不大到寶玉房裏去,又不常和寶玉在一處,好歹我不能知道,只問襲人麝月兩個。”王夫人道:“這就該打嘴!你難道是死人,要你們作什麼!”晴雯道:“我原是跟老太太的人。因老太太說園裏空大人少,寶玉害怕,所以撥了我去外間屋裏上夜,不過看屋子。我原回過我笨,不能伏侍。老太太罵了我,說:‘又不叫你管他的事,要伶俐的作什麼。’我聽了這話纔去的。不過十天半個月之內,寶玉悶了大家頑一會子就散了。至於寶玉飲食起坐,上一層有老奶奶老媽媽們,下一層又有襲人麝月秋紋幾個人。我閒著還要作老太太屋裏的針線,所以寶玉的事竟不曾留心。太太既怪,從此後我留心就是了。” 王夫人信以爲實了,忙說:“阿彌陀佛!你不近寶玉是我的造化,竟不勞你費心。既是老太太給寶玉的,我明兒回了老太太,再攆你。”因向王善保家的道:“你們進去,好生防他幾日,不許他在寶玉房裏睡覺。等我回過老太太,再處治他。”喝聲“去!站在這裏,我看不上這浪樣兒!誰許你這樣花紅柳綠的妝扮!”晴雯只得出來,這氣非同小可,一出門便拿手帕子握著臉,一頭走,一頭哭,直哭到園門內去。
\end{parag}


\begin{parag}
    這裏王夫人向鳳姐等自怨道:“這幾年我越發精神短了,照顧不到。這樣妖精似的東西竟沒看見。只怕這樣的還有,明日倒得查查。”鳳姐見王夫人盛怒之際,又因王善保家的是邢夫人的耳目,常調唆著邢夫人生事,縱有千百樣言詞,此刻也不敢說,只低頭答應著。王善保家的道:“太太請養息身體要緊,這些小事只交與奴才。如今要查這個主兒也極容易,等到晚上園門關了的時節,內外不通風,我們竟給他們個猛不防,帶著人到各處丫頭們房裏搜尋。想來誰有這個,斷不單隻有這個,自然還有別的東西。那時翻出別的來,自然這個也是他的。”王夫人道:“這話倒是。若不如此,斷不能清的清白的白。”因問鳳姐如何。鳳姐只得答應說: “太太說的是,就行罷了。”王夫人道:“這主意很是,不然一年也查不出來。”於是大家商議已定。
\end{parag}


\begin{parag}
    至晚飯後,待賈母安寢了,寶釵等入園時,王善保家的便請了鳳姐一併入園,喝命將角門皆上鎖,便從上夜的婆子處抄檢起,不過抄檢出些多餘攢下蠟燭燈油等物。\begin{note}庚雙夾:畢真。\end{note}王善保家的道:“這也是贓,不許動,等明兒回過太太再動。”於是先就到怡紅院中,喝命關門。當下寶玉正因晴雯不自在,忽見這一干人來,不知爲何直撲了丫頭們的房門去,因迎出鳳姐來,問是何故。鳳姐道:“丟了一件要緊的東西,因大家混賴,恐怕有丫頭們偷了,所以大家都查一查去疑。”一面說,一面坐下喫茶。王善保家的等搜了一回,又細問這幾個箱子是誰的,都叫本人來親自打開。襲人因見晴雯這樣,知道必有異事,又見這番抄檢,只得自己先出來打開了箱子並匣子,任其搜檢一番,不過是平常動用之物。隨放下又搜別人的,挨次都一一搜過。到了晴雯的箱子,因問:“是誰的,怎不開了讓搜?” 襲人等方欲代晴雯開時,只見晴雯挽著頭髮闖進來,豁一聲將箱子掀開,兩手捉著底子,朝天往地下盡情一倒,將所有之物盡都倒出。王善保家的也覺沒趣,看了一看,也無甚私弊之物。回了鳳姐,要往別處去。鳳姐兒道:“你們可細細的查,若這一番查不出來,難回話的。”衆人都道:“都細翻看了,沒什麼差錯東西。雖有幾樣男人物件,都是小孩子的東西,想是寶玉的舊物件,沒甚關係的。”鳳姐聽了,笑道:“既如此咱們就走,再瞧別處去。”
\end{parag}


\begin{parag}
    說著,一徑出來,因向王善保家的道:“我有一句話,不知是不是。要抄檢只抄檢咱們家的人,薛大姑娘屋裏,斷乎檢抄不得的。”王善保家的笑道:“這個自然。豈有抄起親戚家來。”鳳姐點頭道:“我也這樣說呢。”\begin{note}庚雙夾:寫阿鳳心灰意懶,且避禍,後時迥又是一個人矣。\end{note}一頭說,一頭到了瀟湘館內。黛玉已睡了,忽報這些人來,也不知爲甚事。纔要起來,只見鳳姐已走進來,忙按住他不許起來,只說:“睡罷,我們就走。”這邊且說些閒話。那個王善保家的帶了衆人到丫鬟房中,也一一開箱倒籠抄檢了一番。因從紫鵑房中抄出兩副寶玉常換下來的寄名符兒,一副束帶上的披帶,兩個荷包並扇套,套內有扇子。打開看時皆是寶玉往年往日手內曾拿過的。王善保家的自爲得了意,遂忙請鳳姐過來驗視,又說:“這些東西從那裏來的?”鳳姐笑道:“寶玉和他們從小兒在一處混了幾年,這自然是寶玉的舊東西。這也不算什麼罕事,撂下再往別處去是正經。”紫鵑笑道:“直到如今,我們兩下里的東西也算不清。要問這一個,連我也忘了是那年月日有的了。”王善保家的聽鳳姐如此說,也只得罷了。\begin{note}庚雙夾:一處一樣。\end{note}
\end{parag}


\begin{parag}
    又到探春院內,誰知早有人報與探春了。探春也就猜著必有原故,所以引出這等醜態來,\begin{note}庚雙夾:實注一筆。\end{note}遂命衆丫鬟秉燭開門而待。衆人來了。探春故問何事。鳳姐笑道:“因丟了一件東西,連日訪察不出人來,恐怕旁人賴這些女孩子們,所以越性大家搜一搜,使人去疑,倒是洗淨他們的好法子。”探春冷笑道:“我們的丫頭自然都是些賊,我就是頭一個窩主。既如此,先來搜我的箱櫃,他們所有偷了來的都交給我藏著呢。”說著便命丫頭們把箱櫃一齊打開,將鏡奩、妝盒、衾袱、衣包若大若小之物一齊打開,請鳳姐去抄閱。鳳姐陪笑道:“我不過是奉太太的命來,妹妹別錯怪我。何必生氣。”因命丫鬟們快快關上。平兒豐兒等忙著替待書等關的關,收的收。探春道:“我的東西倒許你們搜閱,要想搜我的丫頭,這卻不能。我原比衆人歹毒,凡丫頭所有的東西我都知道,都在我這裏間收著,一針一線他們也沒的收藏,要搜所以只來搜我。你們不依,只管去回太太,只說我違背了太太,該怎麼處治,我去自領。你們別忙,自然連你們抄的日子有呢!你們今日早起不曾議論甄家,自己家裏好好的抄家,果然今日真抄了。\begin{note}庚雙夾:奇極!此曰甄家事。\end{note}咱們也漸漸的來了。可知這樣大族人家,若從外頭殺來,一時是殺不死的,這是古人曾說的‘百足之蟲,死而不僵’,必須先從家裏自殺自滅起來,才能一敗塗地!” 說著,不覺流下淚來。鳳姐只看著衆媳婦們。周瑞家的便道:“既是女孩子的東西全在這裏,奶奶且請到別處去罷,也讓姑娘好安寢。”鳳姐便起身告辭。探春道: “可細細的搜明白了?若明日再來,我就不依了。”鳳姐笑道:“既然丫頭們的東西都在這裏,就不必搜了。”探春冷笑道:“你果然倒乖。連我的包袱都打開了,還說沒翻。明日敢說我護著丫頭們,不許你們翻了。你趁早說明,若還要翻,不妨再翻一遍。”鳳姐知道探春素日與衆不同的,只得陪笑道:“我已經連你的東西都搜查明白了。”探春又問衆人:“你們也都搜明白了不曾?”周瑞家的等都陪笑說:“都翻明白了。”那王善保家的本是個心內沒成算的人,素日雖聞探春的名,那是爲衆人沒眼力沒膽量罷了,那裏一個姑娘家就這樣起來,況且又是庶出,他敢怎麼。他自恃是邢夫人陪房,連王夫人尚另眼相看,何況別個。今見探春如此,他只當是探春認真單惱鳳姐,與他們無干。他便要趁勢作臉獻好,因越衆向前拉起探春的衣襟,故意一掀,嘻嘻笑道:“連姑娘身上我都翻了,果然沒有什麼。”鳳姐見他這樣,忙說:“媽媽走罷,別瘋瘋顛顛的。”一語未了,只聽“拍”的一聲,王家的臉上早著了探春一掌。探春登時大怒,指著王家的問道:“你是什麼東西,敢來拉扯我的衣裳!我不過看著太太的面上,你又有年紀,叫你一聲媽媽,你就狗仗人勢,天天作耗,專管生事。如今越性了不得了。你打諒我是同你們姑娘那樣好性兒,由著你們欺負他,就錯了主意!你搜檢東西我不惱,你不該拿我取笑。”說著,便親自解衣卸裙,拉著鳳姐兒細細的翻。又說:“省得叫奴才來翻我身上。”鳳姐平兒等忙與探春束裙整袂,口內喝著王善保家的說:“媽媽喫兩口酒就瘋瘋顛顛起來。前兒把太太也衝撞了。快出去,不要提起了。”又勸探春休得生氣。探春冷笑道:“我但凡有氣性,早一頭碰死了!不然豈許奴才來我身上翻賊贓了。明兒一早,我先回過老太太太太,然後過去給大娘陪禮,該怎麼,我就領。”那王善保家的討了個沒意思,在窗外只說:“罷了,罷了,這也是頭一遭捱打。我明兒回了太太,仍回老孃家去罷。這個老命還要他做什麼!”探春喝命丫鬟道:“你們聽他說的這話,還等我和他對嘴去不成。”待書等聽說,便出去說道:“你果然回老孃家去,倒是我們的造化了。只怕捨不得去。”鳳姐笑道:“好丫頭,真是有其主必有其僕。”探春冷笑道:“我們作賊的人,嘴裏都有三言兩語的。這還算笨的,背地裏就只不會調唆主子。”平兒忙也陪笑解勸,一面又拉了待書進來。周瑞家的等人勸了一番。鳳姐直待伏侍探春睡下,方帶著人往對過暖香塢來。
\end{parag}


\begin{parag}
    彼時李紈猶病在牀上他與惜春是緊鄰,又與探春相近,故順路先到這兩處。因李紈才吃了藥睡著,不好驚動,只到丫鬟們房中一一的搜了一遍,也沒有什麼東西,遂到惜春房中來。因惜春年少,尚未識事,嚇的不知當有什麼事,故鳳姐也少不得安慰他。誰知竟在入畫箱中尋出一大包金銀錁子來,約共三四十個,\begin{note}庚雙夾:奇。爲察姦情,反得賊贓。\end{note}又有一副玉帶板子並一包男人的靴襪等物。入畫也黃了臉。因問是那裏來的,入畫只得跪下哭訴真情,說:“這是珍大爺賞我哥哥的。\begin{note}庚雙夾:妙極是極。蓋入畫本系寧府之人也。\end{note}因我們老子娘都在南方,如今只跟著叔叔過日子。我叔叔嬸子只要喫酒賭錢,我哥哥怕交給他們又花了,所以每常得了,悄悄的煩了老媽媽帶進來叫我收著的。”惜春膽小,見了這個也害怕,說:“我竟不知道。這還了得!二嫂子,你要打他,好歹帶他出去打罷,我聽不慣的。”鳳姐笑道:“這話若果真呢,也倒可恕,只是不該私自傳送進來。這個可以傳遞,什麼不可以傳遞。這倒是傳遞人的不是了。若這話不真,倘是偷來的,你可就別想活了。”入畫跪著哭道:“我不敢扯謊。奶奶只管明日問我們奶奶和大爺去,若說不是賞的,就拿我和我哥哥一同打死無怨。”鳳姐道:“這個自然要問的,只是真賞的也有不是。誰許你私自傳送東西的!你且說是誰作接應,我便饒你。下次萬萬不可。”惜春道:“嫂子別饒他這次方可。這裏人多,若不拿一個人作法,那些大的聽見了,又不知怎樣呢。嫂子若饒他,我也不依。”\begin{note}庚雙夾:這是自己也不依的。各得自然之理,各有自然之妙。\end{note}鳳姐道:“素日我看他還好。誰沒一個錯,只這一次。二次犯下,二罪俱罰。但不知傳遞是誰。”惜春道:“若說傳遞,再無別個,必是後門上的張媽。他常肯和這些丫頭們鬼鬼祟祟的,這些丫頭們也都肯照顧他。”鳳姐聽說,便命人記下,將東西且交給周瑞家的暫拿著,等明日對明再議。於是別了惜春,方往迎春房內來。
\end{parag}


\begin{parag}
    迎春已經睡著了,丫鬟們也纔要睡,衆人叩門半日纔開。鳳姐吩咐:“不必驚動小姐。”遂往丫鬟們房裏來。因司棋是王善保的外孫女兒,\begin{note}庚雙夾:玄妙奇詭,出人意外。\end{note}鳳姐倒要看看王家的可藏私不藏,遂留神看他搜檢。先從別人箱子搜起,皆無別物。及到了司棋箱子中搜了一回,王善保家的說:“也沒有什麼東西。”纔要蓋箱時,周瑞家的道:“且住,這是什麼?”說著,便伸手掣出一雙男子的錦帶襪並一雙緞鞋來。\begin{note}庚雙夾:險極!\end{note}又有一個小包袱,打開看時,裏面有一個同心如意並一個字帖兒。一總遞與鳳姐。鳳姐因當家理事,每每看開帖並帳目,也頗識得幾個字了。便看那帖子是大紅雙喜箋帖,\begin{note}庚雙夾:紙就好。餘爲司棋心動。\end{note}上面寫道:“上月你來家後,父母已覺察你我之意。但姑娘未出閣,尚不能完你我之心願。若園內可以相見,你可託張媽給一信息。若得在園內一見,倒比來家得說話。千萬,千萬。再所賜香袋二個,今已查收外,特寄香珠一串,略表我心。千萬收好。表弟潘又安拜具。”\begin{note}庚雙夾:名字便妙。\end{note}鳳姐看罷,不怒而反樂。\begin{note}庚雙夾:惡毒之至。\end{note}別人並不識字。王家的素日並不知道他姑表姊弟有這一節風流故事,見了這鞋襪,心內已是有些毛病,又見有一紅帖,鳳姐又看著笑,他便說道:“必是他們胡寫的帳目,不成個字,所以奶奶見笑。”鳳姐笑道:“正是這個帳竟算不過來。你是司棋的老孃,他的表弟也該姓王,怎麼又姓潘呢?”王善保家的見問的奇怪,只得勉強告道:“司棋的姑媽給了潘家,所以他姑表兄弟姓潘。上次逃走了的潘又安就是他表弟。”鳳姐笑道:“這就是了。”因道:“我念給你聽聽。”說著從頭唸了一遍,大家都唬了一跳。這王家的一心只要拿人的錯兒,不想反拿住了他外孫女兒,又氣又臊。周瑞家的四人又都問著他:“你老可聽見了?明明白白,再沒的話說了。如今據你老人家,該怎麼樣?”這王家的只恨沒地縫兒鑽進去。鳳姐只瞅著他嘻嘻的笑,\begin{note}庚雙夾:惡毒之至。\end{note}向周瑞家的笑道:“這倒也好。不用你們作老孃的操一點兒心,他鴉雀不聞的給你們弄了一個好女婿來,大家倒省心。”\begin{note}庚雙夾:刻毒!按鳳姐雖系惡毒之至,然亦不應在下人前爲之,爲此等人前不得不如是也。\end{note}周瑞家的也笑著湊趣兒。王家的氣無處泄,便自己回手打著自己的臉,罵道: “老不死的娼婦,怎麼造下孽了!說嘴打嘴,現世現報在人眼裏。”衆人見這般,俱笑個不住,又半勸半諷的。鳳姐見司棋低頭不語,也並無畏懼慚愧之意,倒覺可異。料此時夜深,且不必盤問,只怕他夜間自愧去尋拙志,遂喚兩個婆子監守起他來。帶了人,拿了贓證回來,且自安歇,等待明日料理。誰知到夜裏又連起來幾次,下面淋血不止。
\end{parag}


\begin{parag}
    至次日,便覺身體十分軟弱,起來發暈,遂撐不住。請太醫來,診脈畢,遂立藥案雲:“看得少奶奶繫心氣不足,虛火乘脾,皆由憂勞所傷,以致嗜臥好眠,胃虛土弱,不思飲食。今聊用昇陽養榮之劑。”寫畢,遂開了幾樣藥名,不過是人蔘,當歸,黃芪等類之劑。一時退去,有老嬤嬤們拿了方子回過王夫人,不免又添一番愁悶,遂將司棋等事暫未理。
\end{parag}


\begin{parag}
    可巧這日尤氏來看鳳姐,坐了一回,到園中去又看過李紈。纔要望候衆姊妹們去,忽見惜春遣人來請,尤氏遂到了他房中來。惜春便將昨晚之事細細告訴與尤氏,又命將入畫的東西一概要來與尤氏過目。尤氏道:“實是你哥哥賞他哥哥的,只不該私自傳送,如今官鹽竟成了私鹽了。”因罵入畫,“糊塗脂油蒙了心的。” 惜春道:“你們管教不嚴,反罵丫頭。這些姊妹,獨我的丫頭這樣沒臉,我如何去見人。昨兒我立逼著鳳姐姐帶了他去,他只不肯。我想,他原是那邊的人,鳳姐姐不帶他去,也原有理。我今日正要送過去,嫂子來的恰好,快帶了他去。或打,或殺,或賣,我一概不管。”入畫聽說,又跪下哭求,說:“再不敢了。只求姑娘看從小兒的情常,好歹生死在一處罷。”尤氏和奶孃等人也都十分分解,說他“不過一時糊塗了,下次再不敢的。他從小兒伏侍你一場,到底留著他爲是。”誰知惜春雖然年幼,卻天生成一種百折不回的廉介孤獨僻性,任人怎說,他只以爲丟了他的體面,咬定牙斷乎不肯。更又說的好:“不但不要入畫,如今我也大了,連我也不便往你們那邊去了。況且近日我每每風聞得有人背地裏議論什麼多少不堪的閒話,我若再去,連我也編派上了。”尤氏道:“誰議論什麼?又有什麼可議論的!姑娘是誰,我們是誰。姑娘既聽見人議論我們,就該問著他纔是。”惜春冷笑道:“你這話問著我倒好。我一個姑娘家,只有躲是非的,我反去尋是非,成個什麼人了!還有一句話:我不怕你惱,好歹自有公論,又何必去問人。古人說得好,‘善惡生死,父子不能有所勖助’,何況你我二人之間。我只知道保得住我就夠了,不管你們。從此以後,你們有事別累我。”尤氏聽了,又氣又好笑,因向地下衆人道:“怪道人人都說這四丫頭年輕糊塗,我只不信。你們聽才一篇話,無原無故,又不知好歹,又沒個輕重。雖然是小孩子的話,卻又能寒人的心。”衆嬤嬤笑道:“姑娘年輕,奶奶自然要喫些虧的。”惜春冷笑道:“我雖年輕,這話卻不年輕。你們不看書不識幾個字,所以都是些呆子,看著明白人,倒說我年輕糊塗。”尤氏道:“你是狀元榜眼探花,古今第一個才子。我們是糊塗人,不如你明白,何如?”惜春道:“狀元榜眼難道就沒有糊塗的不成。可知他們也有不能了悟的。”尤氏笑道:“你倒好。纔是才子,這會子又作大和尚了,又講起了悟來了。”惜春道:“我不了悟,我也捨不得入畫了。”尤氏道:“可知你是個心冷口冷心狠意狠的人。”惜春道:“古人曾也說的,‘不作狠心人,難得自了漢’。我清清白白的一個人,爲什麼教你們帶累壞了我!”尤氏心內原有病,怕說這些話。聽說有人議論,已是心中羞惱激射,只是在惜春分上不好發作,忍耐了大半。今見惜春又說這句,因按捺不住,因問惜春道:“怎麼就帶累了你了?你的丫頭的不是,無故說我,我倒忍了這半日,你倒越發得了意,只管說這些話。你是千金萬金的小姐,我們以後就不親近,仔細帶累了小姐的美名。即刻就叫人將入畫帶了過去!”說著,便賭氣起身去了。惜春道:“若果然不來,倒也省了口舌是非,大家倒還清淨。”尤氏也不答話,一徑往前邊去了。不知後事如何
\end{parag}


\begin{parag}
    \begin{note}蒙回末總:諸院皆宴息,獨探春秉燭以待,大有提防,的是幹才,須另席款待。\end{note}
\end{parag}


\begin{parag}
    \begin{note}蒙回末總:鳳姐喜事,忽作打破虛空之語,惜春年幼,偏有老成練達之操,世態何常,知人何難!\end{note}
\end{parag}
