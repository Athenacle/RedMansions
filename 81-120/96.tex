\chap{九十六}{瞒消息凤姐设奇谋 泄机关颦儿迷本性}


\begin{parag}
    话说贾琏拿了那块假玉忿忿走出,到了书房。那个人看见贾琏的气色不好,心里先发了虚了,连忙站起来迎着。刚要说话,只见贾琏冷笑道:“好大胆,我把你这个混账东西!这里是什么地方儿,你敢来掉鬼!”回头便问:“小厮们呢?”外头轰雷一般几个小厮齐声答应。贾琏道:“取绳子去捆起他来。等老爷回来问明了,把他送到衙门里去。”众小厮又一齐答应“预备着呢。”嘴里虽如此,却不动身。那人先自唬的手足无措,见这般势派,知道难逃公道,只得跪下给贾琏碰头,口口声声只叫:“老太爷别生气。是我一时穷极无奈,才想出这个没脸的营生来。那玉是我借钱做的,我也不敢要了,只得孝敬府里的哥儿顽罢。”说毕,又连连磕头。贾琏啐道:“你这个不知死活的东西!这府里希罕你的那朽不了的浪东西!”正闹着,只见赖大进来,陪着笑向贾琏道:“二爷别生气了。靠他算个什么东西,饶了他,叫他滚出去罢。”贾琏道:“实在可恶。”赖大贾琏作好作歹,众人在外头都说道:“糊涂狗攮的,还不给爷和赖大爷磕头呢。快快的滚罢,还等窝心脚呢!”那人赶忙磕了两个头,抱头鼠窜而去。从此街上闹动了“贾宝玉弄出‘假宝玉’”来。
\end{parag}


\begin{parag}
    且说贾政那日拜客回来,众人因为灯节底下,恐怕贾政生气,已过去的事了,便也都不肯回。只因元妃的事忙碌了好些时,近日宝玉又病着,虽有旧例家宴,大家无兴,也无有可记之事。到了正月十七日,王夫人正盼王子腾来京,只见凤姐进来回说”今日二爷在外听得有人传说,我们家大老爷赶着进京,离城只二百多里地,在路上没了。太太听见了没有?”王夫人吃惊道:“我没有听见,老爷昨晚也没有说起,到底在那里听见的?”凤姐道:“说是在枢密张老爷家听见的。”王夫人怔了半天,那眼泪早流下来了,因拭泪说道:“回来再叫琏儿索性打听明白了来告诉我。”凤姐答应去了。王夫人不免暗里落泪,悲女哭弟,又为宝玉耽忧。如此连三接二,都是不随意的事,那里搁得住,便有些心口疼痛起来。又加贾琏打听明白了来说道:“舅太爷是赶路劳乏,偶然感冒风寒,到了十里屯地方,延医调治。无奈这个地方没有名医,误用了药,一剂就死了。但不知家眷可到了那里没有?”王夫人听了,一阵心酸,便心口疼得坐不住,叫彩云等扶了上炕,还扎挣着叫贾琏去回了贾政,“即速收拾行装迎到那里,帮着料理完毕,既刻回来告诉我们。好叫你媳妇儿放心。”贾琏不敢违拗,只得辞了贾政起身。贾政早已知道,心里很不受用,又知宝玉失玉以后神志惛愦,医药无效,又值王夫人心疼。那年正值京察,工部将贾政保列一等。二月,吏部带领引见。皇上念贾政勤俭谨慎,即放了江西粮道。即日谢恩,已奏明起程日期。虽有众亲朋贺喜,贾政也无心应酬,只念家中人口不宁,又不敢耽延在家。正在无计可施,只听见贾母那边叫“请老爷。”
\end{parag}


\begin{parag}
    贾政即忙进去,看见王夫人带着病也在那里。便向贾母请了安。贾母叫他坐下,便说:“你不日就要赴任,我有多少话与你说,不知你听不听?”说着,掉下泪来。贾政忙站起来说道:“老太太有话只管吩咐,儿子怎敢不遵命呢。”贾母咽哽着说道:“我今年八十一岁的人了,你又要做外任去,偏有你大哥在家,你又不能告亲老。你这一去了,我所疼的只有宝玉,偏偏的又病得糊涂,还不知道怎么样呢。我昨日叫赖升媳妇出去叫人给宝玉算算命,这先生算得好灵,说要娶了金命的人帮扶他,必要冲冲喜才好,不然只怕保不住。我知道你不信那些话,所以教你来商量。你的媳妇也在这里。你们两个也商量商量,还是要宝玉好呢,还是随他去呢?”贾政陪笑说道:“老太太当初疼儿子这么疼的,难道做儿子的就不疼自己的儿子不成么。只为宝玉不上进,所以时常恨他,也不过是恨铁不成钢的意思。老太太既要给他成家,这也是该当的,岂有逆着老太太不疼他的理。如今宝玉病着,儿子也是不放心。因老太太不叫他见我,所以儿子也不敢言语。我到底瞧瞧宝玉是个什么病。”王夫人见贾政说着也有些眼圈儿红,知道心里是疼的,便叫袭人扶了宝玉来。宝玉见了他父亲,袭人叫他请安,他便请了个安。贾政见他脸面很瘦,目光无神,大有疯傻之状,便叫人扶了进去,便想到:“自己也是望六的人了,如今又放外任,不知道几年回来。倘或这孩子果然不好,一则年老无嗣,虽说有孙子,到底隔了一层,二则老太太最疼的是宝玉,若有差错,可不是我的罪名更重了。”瞧瞧王夫人,一包眼泪,又想到他身上,复站起来说:“老太太这么大年纪,想法儿疼孙子,做儿子的还敢违拗?老太太主意该怎么便怎么就是了。但只姨太太那边不知说明白了没有?”王夫人便道:“姨太太是早应了的。只为蟠儿的事没有结案,所以这些时总没提起。”贾政又道:“这就是第一层的难处。他哥哥在监里,妹子怎么出嫁。况且贵妃的事虽不禁婚嫁,宝玉应照已出嫁的姐姐有九个月的功服,此时也难娶亲。再者我的起身日期已经奏明,不敢耽搁,这几天怎么办呢?”贾母想了一想:“说的果然不错。若是等这几件事过去,他父亲又走了。倘或这病一天重似一天,怎么好?只可越些礼办了才好。”想定主意,便说道:“你若给他办呢,我自然有个道理,包管都碍不着。姨太太那边我和你媳妇亲自过去求他。蟠儿那里我央蝌儿去告诉他,说是要救宝玉的命,诸事将就,自然应的。若说服里娶亲,当真使不得。况且宝玉病着,也不可教他成亲,不过是冲冲喜,我们两家愿意,孩子们又有金玉的道理,婚是不用合的了。即挑了好日子,按着咱们家分儿过了礼。赶着挑个娶亲日子,一概鼓乐不用,倒按宫里的样子,用十二对提灯,一乘八人轿子抬了来,照南边规矩拜了堂,一样坐床撒帐,可不是算娶了亲了么。宝丫头心地明白,是不用虑的。内中又有袭人,也还是个妥妥当当的孩子。再有个明白人常劝他更好。他又和宝丫头合的来。再者姨太太曾说,宝丫头的金锁也有个和尚说过,只等有玉的便是婚姻,焉知宝丫头过来,不因金锁倒招出他那块玉来,也定不得。从此一天好似一天,岂不是大家的造化。这会子只要立刻收拾屋子,铺排起来。这屋子是要你派的。一概亲友不请,也不排筵席,待宝玉好了,过了功服,然后再摆席请人。这么着都赶的上。你也看见了他们小两口的事,也好放心的去。”贾政听了,原不愿意,只是贾母做主,不敢违命,勉强陪笑说道:“老太太想的极是,也很妥当。只是要吩咐家下众人,不许吵嚷得里外皆知,这要耽不是的。姨太太那边,只怕不肯,若是果真应了,也只好按着老太太的主意办去。”贾母道:“姨太太那里有我呢。你去吧。”贾政答应出来,心中好不自在。因赴任事多,部里领凭,亲友们荐人,种种应酬不绝,竟把宝玉的事,听凭贾母交与王夫人凤姐儿了。惟将荣禧堂后身王夫人内屋旁边一大跨所二十余间房屋指与宝玉,余者一概不管。贾母定了主意叫人告诉他去,贾政只说很好,此是后话。
\end{parag}


\begin{parag}
    且说宝玉见过贾政,袭人扶回里间炕上。因贾政在外,无人敢与宝玉说话,宝玉便昏昏沉沉的睡去。贾母与贾政所说的话,宝玉一句也没有听见。袭人等却静静儿的听得明白。头里虽也听得些风声,到底影响,只不见宝钗过来,却也有些信真。今日听了这些话,心里方才水落归漕,倒也喜欢。心里想道:“果然上头的眼力不错,这才配得是。我也造化。若他来了,我可以卸了好些担子。但是这一位的心理只有一个林姑娘,幸亏他没有听见,若知道了,又不知要闹到什么分儿了。”袭人想到这里,转喜为悲,心想:“这件事怎么好?老太太,太太那里知道他们心里的事。一时高兴说给他知道,原想要他病好。若是他仍似前的心事:初见林姑娘便要摔玉砸玉,况且那年夏天在园里把我当作林姑娘,说了好些私心话,后来因为紫鹃说了句顽话儿,便哭得死去活来。若是如今和他说要娶宝姑娘,竟把林姑娘撂开,除非是他人事不知还可,若稍明白些,只怕不但不能冲喜,竟是催命了!我再不把话说明,那不是一害三个人了么。”袭人想定主意,待等贾政出去,叫秋纹照看着宝玉,便从里间出来,走到王夫人身旁,悄悄的请了王夫人到贾母后身屋里去说话。贾母只道是宝玉有话,也不理会,还在那里打算怎么过礼,怎么娶亲。
\end{parag}


\begin{parag}
    那袭人同了王夫人到了后间,便跪下哭了。王夫人不知何意,把手拉着他说:“好端端的,这是怎么说?有什么委屈起来说。”袭人道:“这话奴才是不该说的,这会子因为没有法儿了。”王夫人道:“你慢慢说。”袭人道:“宝玉的亲事老太太,太太已定了宝姑娘了,自然是极好的一件事。只是奴才想着,太太看去宝玉和宝姑娘好,还是和林姑娘好呢?”王夫人道:“他两个因从小儿在一处,所以宝玉和林姑娘又好些。”袭人道:“不是好些。”便将宝玉素与黛玉这些光景一一的说了,还说:“这些事都是太太亲眼见的。独是夏天的话我从没敢和别人说。”王夫人拉着袭人道:“我看外面儿已瞧出几分来了。你今儿一说,更加是了。但是刚才老爷说的话想必都听见了,你看他的神情儿怎么样?”袭人道:“如今宝玉若有人和他说话他就笑,没人和他说话他就睡。所以头里的话却倒都没听见。”王夫人道:“倒是这件事叫人怎么样呢?”袭人道:“奴才说是说了,还得太太告诉老太太,想个万全的主意才好。”王夫人便道:“既这么着,你去干你的,这时候满屋子的人,暂且不用提起,等我瞅空儿回明老太太,再作道理。”说着,仍到贾母跟前。
\end{parag}


\begin{parag}
    贾母正在那里和凤姐儿商议,见王夫人进来,便问道:“袭人丫头说什么?这么鬼鬼祟祟的。”王夫人趁问,便将宝玉的心事,细细回明贾母。贾母听了,半日没言语。王夫人和凤姐也都不再说了。只见贾母叹道:“别的事都好说。林丫头倒没有什么,若宝玉真是这样,这可叫人作了难了。”只见凤姐想了一想,因说道:“难倒不难,只是我想了个主意,不知姑妈肯不肯。”王夫人道:“你有主意只管说给老太太听,大家娘儿们商量着办罢了。”凤姐道:“依我想,这件事只有一个掉包儿的法子。”贾母道:“怎么掉包儿?”凤姐道:“如今不管宝兄弟明白不明白,大家吵嚷起来,说是老爷做主,将林姑娘配了他了。瞧他的神情儿怎么样。要是他全不管,这个包儿也就不用掉了。若是他有些喜欢的意思,这事却要大费周折呢。”王夫人道:“就算他喜欢,你怎么样办法呢?”凤姐走到王夫人耳边,如此这般的说了一遍。王夫人点了几点头儿,笑了一笑说道:“也罢了。”贾母便问道:“你娘儿两个捣鬼,到底告诉我是怎么着呀?”凤姐恐贾母不懂,露泄机关,便也向耳边轻轻的告诉了一遍。贾母果真一时不懂,凤姐笑着又说了几句。贾母笑道:“这么着也好,可就只忒苦了宝丫头了。倘或吵嚷出来,林丫头又怎么样呢?”凤姐道:“这个话原只说给宝玉听,外头一概不许提起,有谁知道呢。”正说间,丫头传进话来说:“琏二爷回来了。”王夫人恐贾母问及,使个眼色与凤姐。凤姐便迎着贾琏努了个嘴儿,同到王夫人屋里等着去了。一回儿王夫人进来,已见凤姐哭的两眼通红。贾琏请了安,将到十里屯料理王子腾的丧事的话说了一遍,便说:“有恩旨赏了内阁的职衔,谥了文勤公,命本宗扶柩回籍,着沿途地方官员照料。昨日起身,连家眷回南去了。舅太太叫我回来请安问好,说如今想不到不能进京,有多少话不能说。听见我大舅子要进京,若是路上遇见了,便叫他来到咱们这里细细的说。”王夫人听毕,其悲痛自不必言。凤姐劝慰了一番,“请太太略歇一歇,晚上来再商量宝玉的事罢。”说毕,同了贾琏回到自己房中,告诉了贾琏,叫他派人收拾新房。不题。
\end{parag}


\begin{parag}
    一日,黛玉早饭后带着紫鹃到贾母这边来,一则请安,二则也为自己散散闷。出了潇湘馆,走了几步,忽然想起忘了手绢子来,因叫紫鹃回去取来,自己却慢慢的走着等他。刚走到沁芳桥那边山石背后,当日同宝玉葬花之处,忽听一个人呜呜咽咽在那里哭。黛玉煞住脚听时,又听不出是谁的声音,也听不出哭着叨叨的是些什么话。心里甚是疑惑,便慢慢的走去。及到了跟前,却见一个浓眉大眼的丫头在那里哭呢。黛玉未见他时,还只疑府里这些大丫头有什么说不出的心事,所以来这里发泄发泄,及至见了这个丫头,却又好笑,因想到:这种蠢货有什么情种,自然是那屋里作粗活的丫头受了大女孩子的气了。细瞧了一瞧,却不认得。那丫头见黛玉来了,便也不敢再哭,站起来拭眼泪。黛玉问道:“你好好的为什么在这里伤心?”那丫头听了这话,又流泪道:“林姑娘你评评这个理。他们说话我又不知道,我就说错了一句话,我姐姐也不犯就打我呀。”黛玉听了,不懂他说的是什么,因笑问道:“你姐姐是那一个?”那丫头道:“就是珍珠姐姐。”黛玉听了,才知道他是贾母屋里的,因又问:“你叫什么?”那丫头道:“我叫傻大姐儿。”黛玉笑了一笑,又问:“你姐姐为什么打你?你说错了什么话了?”那丫头道:“为什么呢,就是为我们宝二爷娶宝姑娘的事情。”黛玉听了这一句,如同一个疾雷,心头乱跳。略定了定神,便叫了这丫头”你跟了我这里来。”那丫头跟着黛玉到那畸角儿上葬桃花的去处,那里背静。黛玉因问道:“宝二爷娶宝姑娘,他为什么打你呢?”傻大姐道:“我们老太太和太太二奶奶商量了,因为我们老爷要起身,说就赶着往姨太太商量把宝姑娘娶过来罢。头一宗,给宝二爷冲什么喜,第二宗——”说到这里,又瞅着黛玉笑了一笑,才说道:“赶着办了,还要给林姑娘说婆婆家呢。”黛玉已经听呆了。这丫头只管说道:“我又不知道他们怎么商量的,不叫人吵嚷,怕宝姑娘听见害臊。我白和宝二爷屋里的袭人姐姐说了一句:‘咱们明儿更热闹了,又是宝姑娘,又是宝二奶奶,这可怎么叫呢!’林姑娘你说我这话害着珍珠姐姐什么了吗,他走过来就打了我一个嘴巴,说我混说,不遵上头的话,要撵出我去。我知道上头为什么不叫言语呢,你们又没告诉我,就打我。”说着,又哭起来。
\end{parag}


\begin{parag}
    那黛玉此时心里竟是油儿酱儿糖儿醋儿倒在一处的一般,甜苦酸咸,竟说不上什么味儿来了。停了一会儿,颤巍巍的说道:“你别混说了。你再混说,叫人听见又要打你了。你去罢。”说着,自己移身要回潇湘馆去。那身子竟有千百斤重的,两只脚却象踩着棉花一般,早已软了。只得一步一步慢慢的走将来。走了半天,还没到沁芳桥畔,原来脚下软了。走的慢,且又迷迷痴痴,信着脚从那边绕过来,更添了两箭地的路。这时刚到沁芳桥畔,却又不知不觉的顺着堤往回里走起来。紫鹃取了绢子来,却不见黛玉。正在那里看时,只见黛玉颜色雪白,身子恍恍荡荡的,眼睛也直直的,在那里东转西转。又见一个丫头往前头走了,离的远,也看不出是那一个来。心中惊疑不定,只得赶过来轻轻的问道:“姑娘怎么又回去?是要往那里去?”黛玉也只模糊听见,随口应道:“我问问宝玉去!”紫鹃听了,摸不着头脑,只得搀着他到贾母这边来。
\end{parag}


\begin{parag}
    黛玉走到贾母门口,心里微觉明晰,回头看见紫鹃搀着自己,便站住了问道:“你作什么来的?”紫鹃陪笑道:“我找了绢子来了。头里见姑娘在桥那边呢,我赶着过来问姑娘,姑娘没理会。”黛玉笑道:“我打量你来瞧宝二爷来了呢,不然怎么往这里走呢。”紫鹃见他心里迷惑,便知黛玉必是听见那丫头什么话了,惟有点头微笑而已。只是心里怕他见了宝玉,那一个已经是疯疯傻傻,这一个又这样恍恍惚惚,一时说出些不大体统的话来,那时如何是好?心里虽如此想,却也不敢违拗,只得搀他进去。那黛玉却又奇怪了,这时不似先前那样软了,也不用紫鹃打帘子,自己掀起帘子进来,却是寂然无声。因贾母在屋里歇中觉,丫头们也有脱滑顽去的,也有打盹儿的,也有在那里伺候老太太的。倒是袭人听见帘子响,从屋里出来一看,见是黛玉,便让道:“姑娘屋里坐罢。”黛玉笑着道:“宝二爷在家么?”袭人不知底里,刚要答言,只见紫鹃在黛玉身后和他努嘴儿,指着黛玉,又摇摇手儿。袭人不解何意,也不敢言语。黛玉却也不理会,自己走进房来。看见宝玉在那里坐着,也不起来让坐,只瞅着嘻嘻的傻笑。黛玉自己坐下,却也瞅着宝玉笑。两个人也不问好,也不说话,也无推让,只管对着脸傻笑起来。袭人看见这番光景,心里大不得主意,只是没法儿。忽然听着黛玉说道:“宝玉,你为什么病了?”宝玉笑道:“我为林姑娘病了。”袭人紫鹃两个吓得面目改色,连忙用言语来岔。两个却又不答言,仍旧傻笑起来。袭人见了这样,知道黛玉此时心中迷惑不减于宝玉,因悄和紫鹃说道:“姑娘才好了,我叫秋纹妹妹同着你搀回姑娘歇歇去罢。”因回头向秋纹道:“你和紫鹃姐姐送林姑娘去罢,你可别混说话。”秋纹笑着,也不言语,便来同着紫鹃搀起黛玉。
\end{parag}


\begin{parag}
    那黛玉也就起来,瞅着宝玉只管笑,只管点头儿。紫鹃又催道:“姑娘回家去歇歇罢。”黛玉道:“可不是,我这就是回去的时候儿了。”说着,便回身笑着出来了,仍旧不用丫头们搀扶,自己却走得比往常飞快。紫鹃秋纹后面赶忙跟着走。黛玉出了贾母院门,只管一直走去。紫鹃连忙搀住叫道:“姑娘往这么来。”黛玉仍是笑着随了往潇湘馆来。离门口不远,紫鹃道:“阿弥陀佛,可到了家了!”只这一句话没说完,只见黛玉身子往前一栽,哇的一声,一口血直吐出来。未知性命如何,且听下回分解。
\end{parag}