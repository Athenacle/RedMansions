\chap{六十九}{弄小巧用借劍殺人 覺大限吞生金自逝}


\begin{parag}
    \begin{note}蒙回前總評:寫鳳姐寫不盡,卻從上下左右寫。寫秋桐極淫邪,正寫鳳姐極淫邪;寫平兒極義氣,正寫鳳姐極不義氣;寫使女欺壓二姐,正寫鳳姐欺壓二姐;寫下人感戴二姐,正寫下人不感戴鳳姐。史公用意非念死書子之所知。\end{note}
\end{parag}


\begin{parag}
    話說尤二姐聽了,又感謝不盡,只得跟了他來。尤氏那邊怎好不過來的,少不得也過來跟著鳳姐去回,方是大禮。鳳姐笑說:“你只別說話,等我去說。”尤氏道:“這個自然。但一有個不是,是往你身上推的。”說著,大家先來至賈母房中。
\end{parag}


\begin{parag}
    正值賈母和園中姊妹們說笑解悶,忽見鳳姐帶了一個標緻小媳婦進來,忙覷著眼看,說:“這是誰家的孩子!好可憐見的。”鳳姐上來笑道:“老祖宗倒細細的看看,好不好?”說著,忙拉二姐說:“這是太婆婆,快磕頭。”二姐忙行了大禮,展拜起來。又指著衆姊妹說:這是某人某人,你先認了,太太瞧過了再見禮。二姐聽了,一一又從新故意的問過,垂頭站在旁邊。賈母上下瞧了一遍,因又笑問:“你姓什麼?今年十幾了?”鳳姐忙又笑說:“老祖宗且別問,只說比我俊不俊。” 賈母又戴了眼鏡,命鴛鴦琥珀:“把那孩子拉過來,我瞧瞧肉皮兒。”衆人都抿嘴兒笑著,只得推他上去。賈母細瞧了一遍,又命琥珀:“拿出手來我瞧瞧。”鴛鴦又揭起裙子來。賈母瞧畢,摘下眼鏡來,笑說道:“更是個齊全孩子,我看比你俊些。”鳳姐聽說,笑著忙跪下,將尤氏那邊所編之話,一五一十細細的說了一遍, “少不得老祖宗發慈心,先許他進來,住一年後再圓房。”賈母聽了道:“這有什麼不是。既你這樣賢良,很好。只是一年後方可圓得房。”鳳姐聽了,叩頭起來,又求賈母著兩個女人一同帶去見太太們,說是老祖宗的主意。賈母依允,遂使二人帶去見了邢夫人等。王夫人正因他風聲不雅,深爲憂慮,見他今行此事,豈有不樂之理。於是尤二姐自此見了天日,挪到廂房住居。
\end{parag}


\begin{parag}
    鳳姐一面使人暗暗調唆張華,只叫他要原妻,這裏還有許多賠送外,還給他銀子安家過活。張華原無膽無心告賈家的,後來又見賈蓉打發人來對詞,那人原說的:“張華先退了親。我們皆是親戚。接到家裏住著是真,並無娶嫁之說。皆因張華拖欠了我們的債務,追索不與,方誣賴小的主人那些個。”察院都和賈王兩處有瓜葛,況又受了賄,只說張華無賴,以窮訛詐,狀子也不收,打了一頓趕出來。慶兒在外替他打點,也沒打重。又調唆張華:“親原是你家定的,你只要親事,官必還斷給你。”於是又告。王信那邊又透了消息與察院,察院便批:“張華所欠賈宅之銀,令其限內按數交還,其所定之親,仍令其有力時娶回。”又傳了他父親來當堂批准。他父親亦系慶兒說明,樂得人財兩進,便去賈家領人。鳳姐兒一面嚇的來回賈母,說如此這般,都是珍大嫂子幹事不明,並沒和那家退準,惹人告了,如此官斷。賈母聽了,忙喚了尤氏過來,說他作事不妥,“既是你妹子從小曾與人指腹爲婚,又沒退斷,使人混告了。”尤氏聽了,只得說:“他連銀子都收了,怎麼沒準。”鳳姐在旁又說:“張華的口供上現說不曾見銀子,也沒見人去。他老子說:‘原是親家母說過一次,並沒應準。親家母死了,你們就接進去作二房。’如此沒有對證,只好由他去混說。幸而璉二爺不在家,沒曾圓房,這還無妨。只是人已來了,怎好送回去,豈不傷臉。”賈母道:“又沒圓房,沒的強佔人家有夫之人,名聲也不好,不如送給他去。那裏尋不出好人來。”尤二姐聽了,又回賈母說:“我母親實於某年月日給了他十兩銀子退準的。他因窮急了告,又翻了口。我姐姐原沒錯辦。”賈母聽了,便說:“可見刁民難惹。既這樣,鳳丫頭去料理料理。”鳳姐聽了無法,只得應著。回來只命人去找賈蓉。賈蓉深知鳳姐之意,若要使張華領回,成何體統,便回了賈珍,暗暗遣人去說張華:“你如今既有許多銀子,何必定要原人。若只管執定主意,豈不怕爺們一怒,尋出個由頭,你死無葬身之地。你有了銀子,回家去什麼好人尋不出來。你若走時,還賞你些路費。”張華聽了,心中想了一想,這倒是好主意,和父親商議已定,約共也得了有百金,父子次日起個五更,回原籍去了。賈蓉打聽得真了,來回了賈母鳳姐,說:“張華父子妄告不實,懼罪逃走,官府亦知此情,也不追究,大事完畢。”鳳姐聽了,心中一想:若必定著張華帶回二姐去,未免賈璉回來再花幾個錢包占住,不怕張華不依。還是二姐不去,自己相伴著還妥當,且再作道理。只是張華此去不知何往,他倘或再將此事告訴了別人,或日後再尋出這由頭來翻案,豈不是自己害了自己。原先不該如此將刀靶付與外人去的。因此悔之不迭,復又想了一條主意出來,悄命旺兒遣人尋著了他,或說他作賊,和他打官司將他治死,或暗中使人算計,務將張華治死,方剪草除根,保住自己的名譽。旺兒領命出來,回家細想:人已走了完事,何必如此大作,人命關天,非同兒戲,我且哄過他去,再作道理。因此在外躲了幾日,回來告訴鳳姐,只說張華是有了幾兩銀子在身上,逃去第三日在京口地界五更天已被截路人打悶棍打死了。他老子唬死在店房,在那裏驗屍掩埋。鳳姐聽了不信,說:“你要扯謊,我再使人打聽出來敲你的牙!”自此方丟過不究。鳳姐和尤二姐和美非常,更比親姊親妹還勝十倍。
\end{parag}


\begin{parag}
    那賈璉一日事畢回來,先到了新房中,已竟悄悄的封鎖,只有一個看房子的老頭兒。賈璉問他原故,老頭子細說原委,賈璉只在鐙中跌足。少不得來見賈赦與邢夫人,將所完之事回明。賈赦十分歡喜,說他中用,賞了他一百兩銀子,又將房中一個十七歲的丫鬟名喚秋桐者,賞他爲妾。賈璉叩頭領去,喜之不盡。見了賈母和家中人,回來見鳳姐,未免臉上有些愧色。誰知鳳姐兒他反不似往日容顏,同尤二姐一同出迎,敘了寒溫。賈璉將秋桐之事說了,未免臉上有些得意之色,驕矜之容。鳳姐聽了,忙命兩個媳婦坐車在那邊接了來。心中一刺未除,又平空添了一刺,說不得且吞聲忍氣,將好顏面換出來遮掩。一面又命擺酒接風,一面帶了秋桐來見賈母與王夫人等。賈璉心中也暗暗的納罕。
\end{parag}


\begin{parag}
    那日已是臘月十二日,賈珍起身,先拜了宗祠,然後過來辭拜賈母等人。和族中人直送到灑淚亭方回,獨賈璉賈蓉二人送出三日三夜方回。一路上賈珍命他好生收心治家等語,二人口內答應,也說些大禮套話,不必煩敘。
\end{parag}


\begin{parag}
    且說鳳姐在家,外面待尤二姐自不必說得,只是心中又懷別意。無人處只和尤二姐說:“妹妹的聲名很不好聽,連老太太,太太們都知道了,說妹妹在家做女孩兒就不乾淨,又和姐夫有些首尾,‘沒人要的了你揀了來,還不休了再尋好的。’我聽見這話,氣得倒仰,查是誰說的,又查不出來。這日久天長,這些個奴才們跟前,怎麼說嘴。我反弄了個魚頭來拆。”說了兩遍,自己又氣病了,茶飯也不喫,除了平兒,衆丫頭媳婦無不言三語四,指桑說槐,暗相譏刺。秋桐自爲系賈赦之賜,無人僭他的,連鳳姐平兒皆不放在眼裏,豈肯容他。張口是“先奸後娶沒漢子要的娼婦,也來要我的強。”鳳姐聽了暗樂,尤二姐聽了暗愧暗怒暗氣。鳳姐既裝病,便不和尤二姐喫飯了。每日只命人端了菜飯到他房中去喫,那茶飯都系不堪之物。平兒看不過,自拿了錢出來弄菜與他喫,或是有時只說和他園中去頑,在園中廚內另做了湯水與他喫,也無人敢回鳳姐。只有秋桐一時撞見了,便去說舌告訴鳳姐說:“奶奶的名聲,生是平兒弄壞了的。這樣好菜好飯浪著不喫,卻往園裏去偷喫。”鳳姐聽了,罵平兒說:“人家養貓拿耗子,我的貓只倒咬雞。”平兒不敢多說,自此也要遠著了。又暗恨秋桐,難以出口。
\end{parag}


\begin{parag}
    園中姊妹和李紈迎春惜春等人,皆爲鳳姐是好意,然寶黛一干人暗爲二姐擔心。雖都不便多事,惟見二姐可憐,常來了,倒還都憫恤他。每日常無人處說起話來,尤二姐便淌眼抹淚,又不敢抱怨。鳳姐兒又並無露出一點壞形來。賈璉來家時,見了鳳姐賢良,也便不留心。況素習以來因賈赦姬妾丫鬟最多,賈璉每懷不軌之心,只未敢下手。如這秋桐輩等人,皆是恨老爺年邁昏憒,貪多嚼不爛,沒的留下這些人作什麼,因此除了幾個知禮有恥的,餘者或有與二門上小幺兒們嘲戲的。甚至於與賈璉眉來眼去相偷期的,只懼賈赦之威,未曾到手。這秋桐便和賈璉有舊,從未來過一次。今日天緣湊巧,竟賞了他,真是一對烈火乾柴,如膠投漆,燕爾新婚,連日那裏拆的開。那賈璉在二姐身上之心也漸漸淡了,只有秋桐一人是命。鳳姐雖恨秋桐,且喜借他先可發脫二姐,自己且抽頭,用“借劍殺人”之法,“坐山觀虎鬥”,等秋桐殺了尤二姐,自己再殺秋桐。主意已定,沒人處常又私勸秋桐說:“你年輕不知事。他現是二房奶奶,你爺心坎兒上的人,我還讓他三分,你去硬碰他,豈不是自尋其死?” 那秋桐聽了這話,越發惱了,天天大口亂罵說:“奶奶是軟弱人,那等賢惠,我卻做不來。奶奶把素日的威風怎都沒了。奶奶寬洪大量,我卻眼裏揉不下沙子去。讓我和他這淫婦做一回,他才知道。”鳳姐兒在屋裏,只裝不敢出聲兒。氣的尤二姐在房裏哭泣,飯也不喫,又不敢告訴賈璉。次日賈母見他眼紅紅的腫了,問他,又不敢說。秋桐正是抓乖賣俏之時,他便悄悄的告訴賈母王夫人等說:“專會作死,好好的成天家號喪,背地裏咒二奶奶和我早死了,他好和二爺一心一計的過。”賈母聽了便說:“人太生嬌俏了,可知心就嫉妒。鳳丫頭倒好意待他,他倒這樣爭鋒喫醋的。可是個賤骨頭。”因此漸次便不大喜歡。衆人見賈母不喜,不免又往下踏踐起來,弄得這尤二姐要死不能,要生不得。還是虧了平兒,時常背著鳳姐,看他這般,與他排解排解。
\end{parag}


\begin{parag}
    那尤二姐原是個花爲腸肚雪作肌膚的人,如何經得這般磨折,不過受了一個月的暗氣,便懨懨得了一病,四肢懶動,茶飯不進,漸次黃瘦下去。夜來合上眼,只見他小妹子手捧鴛鴦寶劍前來說:“姐姐,你一生爲人心癡意軟,終吃了這虧。休信那妒婦花言巧語,外作賢良,內藏奸狡,他發恨定要弄你一死方休。若妹子在世,斷不肯令你進來,即進來時,亦不容他這樣。此亦系理數應然,你我生前淫奔不才,使人家喪倫敗行,故有此報。你依我將此劍斬了那妒婦,一同歸至警幻案下,聽其發落。不然,你則白白的喪命,且無人憐惜。”尤二姐泣道:“妹妹,我一生品行既虧,今日之報既系當然,何必又生殺戮之冤。隨我去忍耐。若天見憐,使我好了,豈不兩全。”小妹笑道:“姐姐,你終是個癡人。自古‘天網恢恢,疏而不漏’,天道好還。你雖悔過自新,然已將人父子兄弟致於麀聚之亂,天怎容你安生。”尤二姐泣道:“既不得安生,亦是理之當然,奴亦無怨。”小妹聽了,長嘆而去。尤二姐驚醒,卻是一夢。等賈璉來看時,因無人在側,便泣說: “我這病便不能好了。我來了半年,腹中也有身孕,但不能預知男女。倘天見憐,生了下來還可,若不然,我這命就不保,何況於他。”賈璉亦泣說:“你只放心,我請明人來醫治。”於是出去即刻請醫生。
\end{parag}


\begin{parag}
    誰知王太醫亦謀幹了軍前效力,回來好討蔭封的。小廝們走去,便請了個姓胡的太醫,名叫君榮。進來診脈看了,說是經水不調,全要大補。賈璉便說:“已是三月庚信不行,又常作嘔酸,恐是胎氣。”胡君榮聽了,復又命老婆子們請出手來再看看。尤二姐少不得又從帳內伸出手來。胡君榮又診了半日,說:“若論胎氣,肝脈自應洪大。然木盛則生火,經水不調亦皆因由肝木所致。醫生要大膽,須得請奶奶將金面略露露,醫生觀觀氣色,方敢下藥。”賈璉無法,只得命將帳子掀起一縫,尤二姐露出臉來。胡君榮一見,魂魄如飛上九天,通身麻木,一無所知。一時掩了帳子,賈璉就陪他出來,問是如何。胡太醫道:“不是胎氣,只是迂血凝結。如今只以下迂血通經脈要緊。”於是寫了一方,作辭而去。賈璉命人送了藥禮,抓了藥來,調服下去。只半夜,尤二姐腹痛不止,誰知竟將一個已成形的男胎打了下來。於是血行不止,二姐就昏迷過去。賈璉聞知,大罵胡君榮。一面再遣人去請醫調治,一面命人去打告胡君榮。胡君榮聽了,早已捲包逃走。這裏太醫便說:“本來氣血生成虧弱,受胎以來,想是著了些氣惱,鬱結於中。這位先生擅用虎狼之劑,如今大人元氣十分傷其八九,一時難保就愈。煎丸二藥並行,還要一些閒言閒事不聞,庶可望好。”說畢而去。急的賈璉查是誰請了姓胡的來,一時查了出來,便打了半死。鳳姐比賈璉更急十倍,只說:“咱們命中無子,好容易有了一個,又遇見這樣沒本事的大夫。”於是天地前燒香禮拜,自己通陳禱告說:“我或有病,只求尤氏妹子身體大愈,再得懷胎生一男子,我願喫長齋唸佛。”賈璉衆人見了,無不稱讚。賈璉與秋桐在一處時,鳳姐又做湯做水的著人送與二姐。又罵平兒不是個有福的,“也和我一樣。我因多病了,你卻無病也不見懷胎。如今二奶奶這樣,都因咱們無福,或犯了什麼,衝的他這樣。”因又叫人出去算命打卦。偏算命的回來又說:“系屬兔的陰人衝犯。”大家算將起來,只有秋桐一人屬兔,說他衝的。秋桐近見賈璉請醫治藥,打人罵狗,爲尤二姐十分盡心,他心中早浸了一缸醋在內了。今又聽見如此說他衝了,鳳姐兒又勸他說:“你暫且別處去躲幾個月再來。”秋桐便氣的哭罵道:“理那起瞎肏的混咬舌根!我和他‘井水不犯河水’,怎麼就衝了他!好個愛八哥兒,在外頭什麼人不見,偏來了就有人衝了。白眉赤臉,那裏來的孩子?他不過指著哄我們那個棉花耳朵的爺罷了。縱有孩子,也不知姓張姓王。奶奶希罕那雜種羔子,我不喜歡!老了誰不成?誰不會養!一年半載養一個,倒還是一點攙雜沒有的呢!”罵的衆人又要笑,又不敢笑。可巧邢夫人過來請安,秋桐便哭告邢夫人說:“二爺奶奶要攆我回去,我沒了安身之處,太太好歹開恩。”邢夫人聽說,慌的數落鳳姐兒一陣,又罵賈璉:“不知好歹的種子,憑他怎不好,是你父親給的。爲個外頭來的攆他,連老子都沒了。你要攆他,你不如還你父親去倒好。”說著,賭氣去了。秋桐更又得意,越性走到他窗戶根底下大哭大罵起來。尤二姐聽了,不免更添煩惱。
\end{parag}


\begin{parag}
    晚間,賈璉在秋桐房中歇了,鳳姐已睡,平兒過來瞧他,又悄悄勸他:“好生養病,不要理那畜生。”尤二姐拉他哭道:“姐姐,我從到了這裏,多虧姐姐照應。爲我,姐姐也不知受了多少閒氣。我若逃的出命來,我必答報姐姐的恩德,只怕我逃不出命來,也只好等來生罷。”平兒也不禁滴淚說道:“想來都是我坑了你。我原是一片癡心,從沒瞞他的話。既聽見你在外頭,豈有不告訴他的。誰知生出這些個事來。”尤二姐忙道:“姐姐這話錯了。若姐姐便不告訴他,他豈有打聽不出來的,不過是姐姐說的在先。況且我也要一心進來,方成個體統,與姐姐何干。”二人哭了一回,平兒又囑咐了幾句,夜已深了,方去安息。
\end{parag}


\begin{parag}
    這裏尤二姐心下自思:“病已成勢,日無所養,反有所傷,料定必不能好。況胎已打下,無可懸心,何必受這些零氣,不如一死,倒還乾淨。常聽見人說,生金子可以墜死,豈不比上吊自刎又幹淨。”想畢,拃掙起來,打開箱子,找出一塊生金,也不知多重,恨命含淚便吞入口中,幾次狠命直脖,方嚥了下去。於是趕忙將衣服首飾穿戴齊整,上炕躺下了。當下人不知,鬼不覺。到第二日早晨,丫鬟媳婦們見他不叫人,樂得且自己去梳洗。鳳姐便和秋桐都上去了。平兒看不過,說丫頭們:“你們就只配沒人心的打著罵著使也罷了,一個病人,也不知可憐可憐。他雖好性兒,你們也該拿出個樣兒來,別太過逾了,牆倒衆人推。”丫鬟聽了,急推房門進來看時,卻穿戴的齊齊整整,死在炕上。於是方嚇慌了,喊叫起來。平兒進來看了,不禁大哭。衆人雖素習懼怕鳳姐,然想尤二姐實在溫和憐下,比鳳姐原強,如今死去,誰不傷心落淚,只不敢與鳳姐看見。
\end{parag}


\begin{parag}
    當下合宅皆知。賈璉進來,摟屍大哭不止。鳳姐也假意哭:“狠心的妹妹!你怎麼丟下我去了,辜負了我的心!”尤氏賈蓉等也來哭了一場,勸住賈璉。賈璉便回了王夫人,討了梨香院停放五日,挪到鐵檻寺去,王夫人依允。賈璉忙命人去開了梨香院的門,收拾出正房來停靈。賈璉嫌後門出靈不象,便對著梨香院的正牆上通街現開了一個大門。兩邊搭棚,安壇場做佛事。用軟榻鋪了錦緞衾褥,將二姐抬上榻去,用衾單蓋了。八個小廝和幾個媳婦圍隨,從內子牆一帶抬往梨香院來。那裏已請下天文生預備,揭起衾單一看,只見這尤二姐面色如生,比活著還美貌。賈璉又摟著大哭,只叫“奶奶,你死的不明,都是我坑了你!”賈蓉忙上來勸:“叔叔解著些兒,我這個姨娘自己沒福。”說著,又向南指大觀園的界牆,賈璉會意,只悄悄跌腳說:“我忽略了,終久對出來,我替你報仇。”天文生回說:“奶奶卒於今日正卯時,五日出不得,或是三日,或是七日方可。明日寅時入殮大吉。”賈璉道:“三日斷乎使不得,竟是七日。因家叔家兄皆在外,小喪不敢多停,等到外頭,還放五七,做大道場才掩靈。明年往南去下葬。”天文生應諾,寫了殃榜而去。寶玉已早過來陪哭一場。衆族中人也都來了。賈璉忙進去找鳳姐,要銀子治辦棺槨喪禮。鳳姐見抬了出去,推有病,回:“老太太、太太說我病著,忌三房,不許我去。”因此也不出來穿孝,且往大觀園中來。繞過羣山,至北界牆根下往外聽,隱隱綽綽聽了一言半語,回來又回賈母說如此這般。賈母道:“信他胡說,誰家癆病死的孩子不燒了一撒,也認真的開喪破土起來。既是二房一場,也是夫妻之分,停五七日擡出來,或一燒或亂葬地上埋了完事。”鳳姐笑道:“可是這話。我又不敢勸他。”正說著,丫鬟來請鳳姐,說:“二爺等著奶奶拿銀子呢。”鳳姐只得來了,便問他“什麼銀子?家裏近來艱難,你還不知道?咱們的月例,一月趕不上一月,雞兒吃了過年糧。昨兒我把兩個金項圈當了三百銀子,你還做夢呢。這裏還有二三十兩銀子,你要就拿去。”說著,命平兒拿了出來,遞與賈璉,指著賈母有話,又去了。恨的賈璉沒話可說,只得開了尤氏箱櫃,去拿自己的梯己。及開了箱櫃,一滴無存,只有些拆簪爛花並幾件半新不舊的綢絹衣裳,都是尤二姐素習所穿的,不禁又傷心哭了起來。自己用個包袱一齊包了,也不命小丫鬟來拿,便自己提著來燒。
\end{parag}


\begin{parag}
    平兒又是傷心,又是好笑,忙將二百兩一包的碎銀子偷了出來,到廂房拉住賈璉,悄遞與他說:“你只別作聲纔好,你要哭,外頭多少哭不得,又跑了這裏來點眼。”賈璉聽說,便說:“你說的是。”接了銀子,又將一條裙子遞與平兒,說:“這是他家常穿的,你好生替我收著,作個念心兒。”平兒只得掩了,自己收去。賈璉拿了銀子與衆人,走來命人先去買板。好的又貴,中的又不要。賈璉騎馬自去要瞧,至晚間果抬了一副好板進來,價銀五百兩賒著,連夜趕造。一面分派了人口穿孝守靈,晚來也不進去,只在這裏伴宿。正是──
\end{parag}


\begin{parag}
    \begin{note}蒙回後總評:鳳姐初念在張華領出二姐,轉念又恐仍爲外宅,轉念即欲殺張華,爲斬草除根計。時寫來覺滿腔都是荊棘,渾身都是爪牙,安得借鴛鴦劍手刃其首,以寒千古姦婦之膽。\end{note}
\end{parag}


\begin{parag}
    \begin{note}蒙回後總評:看三姐夢中相敘一段,真有孝子悌弟、義士忠臣之慨,我不禁淚流一斗,溼地三尺。\end{note}
\end{parag}
