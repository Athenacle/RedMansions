\chap{九十七}{林黛玉焚稿斷癡情 薛寶釵出閨成大禮}



\begin{parag}
    話說黛玉到瀟湘館門口,紫鵑說了一句話,更動了心,一時吐出血來,幾乎暈倒。虧了還同着秋紋,兩個人挽扶着黛玉到屋裏來。那時秋紋去後,紫鵑雪雁守着,見他漸漸甦醒過來,問紫鵑道:“你們守着哭什麼?”紫鵑見他說話明白,倒放了心了,因說:“姑娘剛纔打老太太那邊回來,身上覺着不大好,唬的我們沒了主意,所以哭了。”黛玉笑道:“我那裏就能夠死呢。”這一句話沒完,又喘成一處。原來黛玉因今日聽得寶玉寶釵的事情,這本是他數年的心病,一時急怒,所以迷惑了本性。及至回來吐了這一口血,心中卻漸漸的明白過來,把頭裏的事一字也不記得了。這會子見紫鵑哭,方模糊想起傻大姐的話來,此時反不傷心,惟求速死,以完此債。這裏紫鵑雪雁只得守着,想要告訴人去,怕又象上次招得鳳姐兒說他們失驚打怪的。
\end{parag}


\begin{parag}
    那知秋紋回去,神情慌遽。正值賈母睡起中覺來,看見這般光景,便問怎麼了。秋紋嚇的連忙把剛纔的事回了一遍。賈母大驚說:“這還了得!”連忙着人叫了王夫人鳳姐過來,告訴了他婆媳兩個。鳳姐道:“我都囑咐到了,這是什麼人走了風呢。這不更是一件難事了嗎。賈母道:“且別管那些,先瞧瞧去是怎麼樣了。”說着便起身帶着王夫人鳳姐等過來看視。見黛玉顏色如雪,並無一點血色,神氣昏沉,氣息微細。半日又咳嗽了一陣,丫頭遞了痰盒,吐出都是痰中帶血的。大家都慌了。只見黛玉微微睜眼,看見賈母在他旁邊,便喘吁吁的說道:“老太太,你白疼了我了!”賈母一聞此言,十分難受,便道:“好孩子,你養着罷,不怕的。”黛玉微微一笑,把眼又閉上了。外面丫頭進來回鳳姐道:“大夫來了。”於是大家略避。王大夫同着賈璉進來,診了脈,說道:“尚不妨事。這是鬱氣傷肝,肝不藏血,所以神氣不定。如今要用斂陰止血的藥,方可望好。”王大夫說完,同着賈璉出去開方取藥去了。
\end{parag}


\begin{parag}
    賈母看黛玉神氣不好,便出來告訴鳳姐等道:“我看這孩子的病,不是我咒他,只怕難好。你們也該替他預備預備,衝一衝。或者好了,豈不是大家省心。就是怎麼樣,也不至臨時忙亂。咱們家裏這兩天正有事呢。”鳳姐兒答應了。賈母又問了紫鵑一回,到底不知是那個說的。賈母心裏只是納悶,因說:“孩子們從小兒在一處兒頑,好些是有的。如今大了懂的人事,就該要分別些,纔是做女孩兒的本分,我才心裏疼他。若是他心裏有別的想頭,成了什麼人了呢!我可是白疼了他了。你們說了,我倒有些不放心。”因回到房中,又叫襲人來問。襲人仍將前日回王夫人的話並方纔黛玉的光景述了一遍。賈母道:“我方纔看他卻還不至糊塗,這個理我就不明白了。咱們這種人家,別的事自然沒有的,這心病也是斷斷有不得的。林丫頭若不是這個病呢,我憑着花多少錢都使得。若是這個病,不但治不好,我也沒心腸了。”鳳姐道:“林妹妹的事老太太倒不必張心,橫豎有他二哥哥天天同着大夫瞧看。倒是姑媽那邊的事要緊。今日早起聽見說,房子不差什麼就妥當了,竟是老太太,太太到姑媽那邊,我也跟了去,商量商量。就只一件,姑媽家裏有寶妹妹在那裏,難以說話,不如索性請姑媽晚上過來,咱們一夜都說結了,就好辦了。”賈母王夫人都道:“你說的是。今日晚了,明日飯後咱們娘兒們就過去。”說着,賈母用了晚飯。鳳姐同王夫人各自歸房。不提。
\end{parag}


\begin{parag}
    且說次日鳳姐吃了早飯過來,便要試試寶玉,走進裏間說道:“寶兄弟大喜,老爺已擇了吉日要給你娶親了。你喜歡不喜歡?”寶玉聽了,只管瞅着鳳姐笑,微微的點點頭兒。鳳姐笑道:“給你娶林妹妹過來好不好?”寶玉卻大笑起來。鳳姐看着,也斷不透他是明白是糊塗,因又問道:“老爺說你好了纔給你娶林妹妹呢,若還是這麼傻,便不給你娶了。”寶玉忽然正色道:“我不傻,你才傻呢。”說着,便站起來說:“我去瞧瞧林妹妹,叫他放心。”鳳姐忙扶住了,說:“林妹妹早知道了。他如今要做新媳婦了,自然害羞,不肯見你的。”寶玉道:“娶過來他到底是見我不見?”鳳姐又好笑,又着忙,心裏想:“襲人的話不差。提了林妹妹,雖說仍舊說些瘋話,卻覺得明白些。若真明白了,將來不是林妹妹,打破了這個燈虎兒,那饑荒才難打呢。”便忍笑說道:“你好好兒的便見你,若是瘋瘋顛顛的,他就不見你了。”寶玉說道:“我有一個心,前兒已交給林妹妹了。他要過來,橫豎給我帶來,還放在我肚子裏頭。”鳳姐聽着竟是瘋話,便出來看着賈母笑。賈母聽了,又是笑,又是疼,便說道:“我早聽見了。如今且不用理他,叫襲人好好的安慰他。咱們走罷。”
\end{parag}


\begin{parag}
    說着王夫人也來。大家到了薛姨媽那裏,只說惦記着這邊的事來瞧瞧。薛姨媽感激不盡,說些薛蟠的話。喝了茶,薛姨媽纔要人告訴寶釵,鳳姐連忙攔住說:“姑媽不必告訴寶妹妹。”又向薛姨媽陪笑說道:“老太太此來,一則爲瞧姑媽,二則也有句要緊的話特請姑媽到那邊商議。”薛姨媽聽了,點點頭兒說:“是了。”於是大家又說些閒話,便回來了。
\end{parag}


\begin{parag}
    當晚薛姨媽果然過來,見過了賈母,到王夫人屋裏來,不免說起王子騰來,大家落了一回淚。薛姨媽便問道:“剛纔我到老太太那裏,寶哥兒出來請安還好好兒的,不過略瘦些,怎麼你們說得很利害?”鳳姐便道:“其實也不怎麼樣,只是老太太懸心。目今老爺又要起身外任去,不知幾年纔來。老太太的意思,頭一件叫老爺看着寶兄弟成了家也放心,二則也給寶兄弟沖沖喜,借大妹妹的金瑣壓壓邪氣,只怕就好了。”薛姨媽心裏也願意,只慮着寶釵委屈,便道:“也使得,只是大家還要從長計較計較纔好。”王夫人便按着鳳姐的話和薛姨媽說,只說:“姨太太這會子家裏沒人,不如把裝奩一概蠲免。明日就打發蝌兒去告訴蟠兒,一面這裏過門,一面給他變法兒撕擄官事。”並不提寶玉的心事,又說:“姨太太,既作了親,娶過來早早好一天,大家早放一天心。”正說着,只見賈母差鴛鴦過來候信。薛姨媽雖恐寶釵委屈,然也沒法兒,又見這般光景,只得滿口應承。鴛鴦回去回了賈母。賈母也甚喜歡,又叫鴛鴦過來求薛姨媽和寶釵說明原故,不叫他受委屈。薛姨媽也答應了。便議定鳳姐夫婦作媒人。大家散了。王夫人姊妹不免又敘了半夜話兒。
\end{parag}


\begin{parag}
    次日,薛姨媽回家將這邊的話細細的告訴了寶釵,還說:“我已經應承了。”寶釵始則低頭不語,後來便自垂淚。薛姨媽用好言勸慰解釋了好些話。寶釵自回房內,寶琴隨去解悶。薛姨媽才告訴了薛蝌,叫他明日起身,”一則打聽審詳的事,二則告訴你哥哥一個信兒,你即便回來。”
\end{parag}


\begin{parag}
    薛蝌去了四日,便回來回覆薛姨媽道:“哥哥的事上司已經準了誤殺,一過堂就要題本了,叫咱們預備贖罪的銀子。妹妹的事,說‘媽媽做主很好的,趕着辦又省了好些銀子,叫媽媽不用等我,該怎麼着就怎麼辦罷。’”薛姨媽聽了,一則薛蟠可以回家,二則完了寶釵的事,心瑞安放了好些。便是看着寶釵心裏好象不願意似的,”雖是這樣,他是女兒家,素來也孝順守禮的人,知我應了,他也沒得說的。”便叫薛蝌:“辦泥金庚帖,填上八字,即叫人送到璉二爺那邊去。還問了過禮的日子來,你好預備。本來咱們不驚動親友,哥哥的朋友是你說的‘都是混賬人’,親戚呢,就是賈王兩家,如今賈家是男家,王家無人在京裏。史姑娘放定的事,他家沒有請咱們,咱們也不用通知。倒是把張德輝請了來,託他照料些,他上幾歲年紀的人,到底懂事。”薛蝌領命,叫人送帖過去。
\end{parag}


\begin{parag}
    次日賈璉過來,見了薛姨媽,請了安,便說:“明日就是上好的日子,今日過來回姨太太,就是明日過禮罷。只求姨太太不要挑飭就是了。”說着,捧過通書來。薛姨媽也謙遜了幾句,點頭應允。賈璉趕着回去回明賈政。賈政便道:“你回老太太說,既不叫親友們知道,諸事寧可簡便些。若是東西上,請老太太瞧了就是了,不必告訴我。”賈璉答應,進內將話回明賈母。
\end{parag}


\begin{parag}
    這裏王夫人叫了鳳姐命人將過禮的物件都送與賈母過目,並叫襲人告訴寶玉。那寶玉又嘻嘻的笑道:“這裏送到園裏,回來園裏又送到這裏。咱們的人送,咱們的人收,何苦來呢。”賈母王夫人聽了,都喜歡道:“說他糊塗,他今日怎麼這麼明白呢。”鴛鴦等忍不住好笑,只得上來一件一件的點明給賈母瞧,說:“這是金項圈,這是金珠首飾,共八十件。這是妝蟒四十匹。這是各色綢緞一百二十匹。這是四季的衣服共一百二十件。外面也沒有預備羊酒,這是折羊酒的銀子。”賈母看了都說“好”,輕輕的與鳳姐說道:“你去告訴姨太太,說:不是虛禮,求姨太太等蟠兒出來慢慢的叫人給他妹妹做來就是了。那好日子的被褥還是咱們這裏代辦了罷。”鳳姐答應了,出來叫賈璉先過去,又叫周瑞旺兒等,吩咐他們:“不必走大門,只從園裏從前開的便門內送去,我也就過去。這門離瀟湘館還遠,倘別處的人見了,囑咐他們不用在瀟湘館裏提起。”衆人答應着送禮而去。寶玉認以爲真,心裏大樂,精神便覺得好些,只是語言總有些瘋傻。那過禮的回來都不提名說姓,因此上下人等雖都知道,只因鳳姐吩咐,都不敢走漏風聲。
\end{parag}


\begin{parag}
    且說黛玉雖然服藥,這病日重一日。紫鵑等在旁苦勸,說道:“事情到了這個分兒,不得不說了。姑娘的心事,我們也都知道。至於意外之事是再沒有的。姑娘不信,只拿寶玉的身子說起,這樣大病,怎麼做得親呢。姑娘別聽瞎話,自己安心保重纔好。”黛玉微笑一笑,也不答言,又咳嗽數聲,吐出好些血來。紫鵑等看去,只有一息奄奄,明知勸不過來,惟有守着流淚,天天三四趟去告訴賈母。鴛鴦測度賈母近日比前疼黛玉的心差了些,所以不常去回。況賈母這幾日的心都在寶釵寶玉身上,不見黛玉的信兒也不大提起,只請太醫調治罷了。
\end{parag}


\begin{parag}
    黛玉向來病着,自賈母起,直到姊妹們的下人,常來問候。今見賈府中上下人等都不過來,連一個問的人都沒有,睜開眼,只有紫鵑一人。自料萬無生理,因扎掙着向紫鵑說道:“妹妹,你是我最知心的,雖是老太太派你伏侍我這幾年,我拿你就當我的親妹妹。”說到這裏,氣又接不上來。紫鵑聽了,一陣心酸,早哭得說不出話來。遲了半日,黛玉又一面喘一面說道:“紫鵑妹妹,我躺着不受用,你扶起我來靠着坐坐纔好。”紫鵑道:“姑娘的身上不大好,起來又要抖摟着了。”黛玉聽了,閉上眼不言語了。一時又要起來。紫鵑沒法,只得同雪雁把他扶起,兩邊用軟枕靠住,自己卻倚在旁邊。
\end{parag}


\begin{parag}
    黛玉那裏坐得住,下身自覺硌的疼,狠命的撐着,叫過雪雁來道:“我的詩本子。”說着又喘。雪雁料是要他前日所理的詩稿,因找來送到黛玉跟前。黛玉點點頭兒,又抬眼看那箱子。雪雁不解,只是發怔。黛玉氣的兩眼直瞪,又咳嗽起來,又吐了一口血。雪雁連忙回身取了水來,黛玉漱了,吐在盒內。紫鵑用絹子給他拭了嘴。黛玉便拿那絹子指着箱子,又喘成一處,說不上來,閉了眼。紫鵑道:“姑娘歪歪兒罷。”黛玉又搖搖頭兒。紫鵑料是要絹子,便叫雪雁開箱,拿出一塊白綾絹子來。黛玉瞧了,撂在一邊,使勁說道:“有字的。”紫鵑這才明白過來,要那塊題詩的舊帕,只得叫雪雁拿出來遞給黛玉。紫鵑勸道:“姑娘歇歇罷,何苦又勞神,等好了再瞧罷。”只見黛玉接到手裏,也不瞧詩,扎掙着伸出那隻手來狠命的撕那絹子,卻是隻有打顫的分兒,那裏撕得動。紫鵑早已知他是恨寶玉,卻也不敢說破,只說:“姑娘何苦自己又生氣!”黛玉點點頭兒,掖在袖裏,便叫雪雁點燈。雪雁答應,連忙點上燈來。
\end{parag}


\begin{parag}
    黛玉瞧瞧,又閉了眼坐着,喘了一會子,又道:“籠上火盆。”紫鵑打諒他冷。因說道:“姑娘躺下,多蓋一件罷。那炭氣只怕耽不住。”黛玉又搖頭兒。雪雁只得籠上,擱在地下火盆架上。黛玉點頭,意思叫挪到炕上來。雪雁只得端上來,出去拿那張火盆炕桌。那黛玉卻又把身子欠起,紫鵑只得兩隻手來扶着他。黛玉這纔將方纔的絹子拿在手中,瞅着那火點點頭兒,往上一撂。紫鵑唬了一跳,欲要搶時,兩隻手卻不敢動。雪雁又出去拿火盆桌子,此時那絹子已經燒着了。紫鵑勸道:“姑娘這是怎麼說呢。”黛玉只作不聞,回手又把那詩稿拿起來,瞧了瞧又撂下了。紫鵑怕他也要燒,連忙將身倚住黛玉,騰出手來拿時,黛玉又早拾起,撂在火上。此時紫鵑卻夠不着,乾急。雪雁正拿進桌子來,看見黛玉一撂,不知何物,趕忙搶時,那紙沾火就着,如何能夠少待,早已烘烘的着了。雪雁也顧不得燒手,從火裏抓起來撂在地下亂踩,卻已燒得所餘無幾了。那黛玉把眼一閉,往後一仰,幾乎不曾把紫鵑壓倒。紫鵑連忙叫雪雁上來將黛玉扶着放倒,心裏突突的亂跳。欲要叫人時,天又晚了,欲不叫人時,自己同着雪雁和鸚哥等幾個小丫頭,又怕一時有什麼原故。好容易熬了一夜。到了次日早起,覺黛玉又緩過一點兒來。飯後,忽然又嗽又吐,又緊起來。紫鵑看着不祥了,連忙將雪雁等都叫進來看守,自己卻來回賈母。那知到了賈母上房,靜悄悄的,只有兩三個老媽媽和幾個做粗活的丫頭在那裏看屋子呢。紫鵑因問道:“老太太呢?”那些人都說不知道。紫鵑聽這話詫異,遂到寶玉屋裏去看,竟也無人。遂問屋裏的丫頭,也說不知。紫鵑已知八九,”但這些人怎麼竟這樣狠毒冷淡!”又想到黛玉這幾天竟連一個人問的也沒有,越想越悲,索性激起一腔悶氣來,一扭身便出來了。自己想了一想,”今日倒要看看寶玉是何形狀!看他見了我怎麼樣過的去!那一年我說了一句謊話他就急病了,今日竟公然做出這件事來!可知天下男子之心真真是冰寒雪冷,令人切齒的!”一面走,一面想,早已來到怡紅院。只見院門虛掩,裏面卻又寂靜的很。紫鵑忽然想到:“他要娶親,自然是有新屋子的,但不知他這新屋子在何處?”正在那裏徘徊瞻顧,看見墨雨飛跑,紫鵑便叫住他。墨雨過來笑嘻嘻的道:“姐姐在這裏做什麼?”紫鵑道:“我聽見寶二爺娶親,我要來看看熱鬧兒。誰知不在這裏,也不知是幾兒。”墨雨悄悄的道:“我這話只告訴姐姐,你可別告訴雪雁他們。上頭吩咐了,連你們都不叫知道呢。就是今日夜裏娶,那裏是在這裏,老爺派璉二爺另收拾了房子了。”說着又問:“姐姐有什麼事麼?”紫鵑道:“沒什麼事,你去罷。”墨雨仍舊飛跑去了。紫鵑自己也發了一回呆,忽然想起黛玉來,這時候還不知是死是活。因兩淚汪汪,咬着牙發狠道:“寶玉,我看他明兒死了,你算是躲的過不見了!你過了你那如心如意的事兒,拿什麼臉來見我!”一面哭,一面走,嗚嗚咽咽的自回去了。還未到瀟湘館,只見兩個小丫頭在門裏往外探頭探腦的,一眼看見紫鵑,那一個便嚷道:“那不是紫鵑姐姐來了嗎。”紫鵑知道不好了,連忙擺手兒不叫嚷,趕忙進去看時,只見黛玉肝火上炎,兩顴紅赤。紫鵑覺得不妥,叫了黛玉的奶媽王奶奶來。一看,他便大哭起來。這紫鵑因王奶媽有些年紀,可以仗個膽兒,誰知竟是個沒主意的人,反倒把紫鵑弄得心裏七上八下。忽然想起一個人來,便命小丫頭急忙去請。你道是誰,原來紫鵑想起李宮裁是個孀居,今日寶玉結親,他自然迴避。況且園中諸事向系李紈料理,所以打發人去請他。
\end{parag}


\begin{parag}
    李紈正在那裏給賈蘭改詩,冒冒失失的見一個丫頭進來回說:“大奶奶,只怕林姑娘好不了,那裏都哭呢。”李紈聽了,嚇了一大跳,也來不及問了,連忙站起身來便走,素雲、碧月跟着,一頭走着,一頭落淚,想着:“姐妹在一處一場,更兼他那容貌才情真是寡二少雙,惟有青女素娥可以彷彿一二,竟這樣小小的年紀,就作了北邙鄉女!偏偏鳳姐想出一條偷樑換柱之計,自己也不好過瀟湘館來,竟未能少盡姊妹之情。真真可憐可嘆。”一頭想着,已走到瀟湘館的門口。裏面卻又寂然無聲,李紈倒着起忙來,想來必是已死,都哭過了,那衣衾未知裝裹妥當了沒有?連忙三步兩步走進屋子來。
\end{parag}


\begin{parag}
    裏間門口一個小丫頭已經看見,便說:“大奶奶來了。”紫鵑忙往外走,和李紈走了個對臉。李紈忙問:“怎麼樣?”紫鵑欲說話時,惟有喉中哽咽的分兒,卻一字說不出。那眼淚一似斷線珍珠一般,只將一隻手回過去指着黛玉。李紈看了紫鵑這般光景,更覺心酸,也不再問,連忙走過來。看時,那黛玉已不能言。李紈輕輕叫了兩聲,黛玉卻還微微的開眼,似有知識之狀,但隻眼皮嘴脣微有動意,口內尚有出入之息,卻要一句話一點淚也沒有了。李紈回身見紫鵑不在跟前,便問雪雁。雪雁道:“他在外頭屋裏呢。”李紈連忙出來,只見紫鵑在外間空牀上躺着,顏色青黃,閉了眼只管流淚,那鼻涕眼淚把一個砌花錦邊的褥子已溼了碗大的一片。李紈連忙喚他,那紫鵑才慢慢的睜開眼欠起身來。李紈道:“傻丫頭,這是什麼時候,且只顧哭你的!林姑娘的衣衾還不拿出來給他換上,還等多早晚呢。難道他個女孩兒家,你還叫他赤身露體精着來光着去嗎!”紫鵑聽了這句話,一發止不住痛哭起來。李紈一面也哭,一面着急,一面拭淚,一面拍着紫鵑的肩膀說:“好孩子,你把我的心都哭亂了,快着收拾他的東西罷,再遲一會子就了不得了。”正鬧着,外邊一個人慌慌張張跑進來,倒把李紈唬了一跳,看時卻是平兒。跑進來看見這樣,只是呆磕磕的發怔。李紈道:“你這會子不在那邊,做什麼來了?”說着,林之孝家的也進來了。平兒道:“奶奶不放心,叫來瞧瞧。既有大奶奶在這裏,我們奶奶就只顧那一頭兒了。”李紈點點頭兒。平兒道:“我也見見林姑娘。”說着,一面往裏走,一面早已流下淚來。這裏李紈因和林之孝家的道:“你來的正好,快出去瞧瞧去。告訴管事的預備林姑娘的後事。妥當了叫他來回我,不用到那邊去。”林之孝家的答應了,還站着。李紈道:“還有什麼話呢?”林之孝家的道:“剛纔二奶奶和老太太商量了,那邊用紫鵑姑娘使喚使喚呢。”李紈還未答言,只見紫鵑道:“林奶奶,你先請罷。等着人死了我們自然是出去的,那裏用這麼……”說到這裏卻又不好說了,因又改說道:“況且我們在這裏守着病人,身上也不潔淨。林姑娘還有氣兒呢,不時的叫我。”李紈在旁解說道:“當真這林姑娘和這丫頭也是前世的緣法兒。倒是雪雁是他南邊帶來的,他倒不理會。惟有紫鵑,我看他兩個一時也離不開。”林之孝家的頭裏聽了紫鵑的話,未免不受用,被李紈這番一說,卻也沒的說,又見紫鵑哭得淚人一般,只好瞅着他微微的笑,因又說道:“紫鵑姑娘這些閒話倒不要緊,只是他卻說得,我可怎麼回老太太呢。況且這話是告訴得二奶奶的嗎!”正說着,平兒擦着眼淚出來道:“告訴二奶奶什麼事?”林之孝家的將方纔的話說了一遍。平兒低了一回頭,說:“這麼着罷,就叫雪姑娘去罷。”李紈道:“他使得嗎?”平兒走到李紈耳邊說了幾句,李紈點點頭兒道:“既是這麼着,就叫雪雁過去也是一樣的。”林之孝家的因問平兒道:“雪姑娘使得嗎?”平兒道:“使得,都是一樣。”林家的道:“那麼姑娘就快叫雪姑娘跟了我去。我先去回了老太太和二奶奶去,這可是大奶奶和姑娘的主意。回來姑娘再各自回二奶奶去。”李紈道:“是了。你這麼大年紀,連這麼點子事還不耽呢。”林家的笑道:“不是不耽,頭一宗這件事老太太和二奶奶辦的,我們都不能很明白,再者又有大奶奶和平姑娘呢。”說着,平兒已叫了雪雁出來。原來雪雁因這幾日嫌他小孩子家懂得什麼,便也把心冷淡了。況且聽是老太太和二奶奶叫,也不敢不去。連忙收拾了頭,平兒叫他換了新鮮衣服。跟着林家的去了。隨後平兒又和李紈說了幾句話。李紈又囑咐平兒打那麼催着林之孝家的叫他男人快辦了來。平兒答應着出來,轉了個彎子,看見林家的帶着雪雁在前頭走呢,趕忙叫住道:“我帶了他去罷,你先告訴林大爺辦林姑娘的東西去罷。奶奶那裏我替回就是了。”那林家的答應着去了。這裏平兒帶了雪雁到了新房子裏,回明瞭自去辦事。
\end{parag}


\begin{parag}
    卻說雪雁看見這般光景,想起他家姑娘,也未免傷心,只是在賈母鳳姐跟前不敢露出。因又想道:“也不知用我作什麼,我且瞧瞧。寶玉一日家和我們姑娘好的蜜裏調油,這時候總不見面了,也不知是真病假病。怕我們姑娘不依,他假說丟了玉,裝出傻子樣兒來,叫我們姑娘寒了心。他好娶寶姑娘的意思。我看看他去,看他見了我傻不傻。莫不成今兒還裝傻麼!”一面想着,已溜到裏間屋子門口,偷偷兒的瞧。這時寶玉雖因失玉昏憒,但只聽見娶了黛玉爲妻,真乃是從古至今天上人間第一件暢心滿意的事了,那身子頓覺健旺起來,——只不過不似從前那般靈透,所以鳳姐的妙計百發百中——巴不得即見黛玉,盼到今日完姻,真樂得手舞足蹈,雖有幾句傻話,卻與病時光景大相懸絕了。雪雁看了,又是生氣又是傷心,他那裏曉得寶玉的心事,便各自走開。
\end{parag}


\begin{parag}
    這裏寶玉便叫襲人快快給他裝新,坐在王夫人屋裏。看見鳳姐尤氏忙忙碌碌,再盼不到吉時,只管問襲人道:“林妹妹打園裏來,爲什麼這麼費事,還不來?”襲人忍着笑道:“等好時辰。”回來又聽見鳳姐與王夫人道:“雖然有服,外頭不用鼓樂,咱們南邊規矩要拜堂的,冷清清使不得。我傳了家內學過音樂管過戲子的那些女人來吹打,熱鬧些。”王夫人點頭說:“使得。”
\end{parag}


\begin{parag}
    一時大轎從大門進來,家裏細樂迎出去,十二對宮燈,排着進來,倒也新鮮雅緻。儐相請了新人出轎。寶玉見新人蒙着蓋頭,喜娘披着紅扶着。下首扶新人的你道是誰,原來就是雪雁。寶玉看見雪雁,猶想:“因何紫鵑不來,倒是他呢?”又想道:“是了,雪雁原是他南邊家裏帶來的,紫鵑仍是我們家的,自然不必帶來。”因此見了雪雁竟如見了黛玉的一般歡喜。儐相贊禮拜了天地。請出賈母受了四拜,後請賈政夫婦登堂,行禮畢,送入洞房。還有坐牀撒帳等事,俱是按金陵舊例。賈政原爲賈母作主,不敢違拗,不信沖喜之說。那知今日寶玉居然象個好人一般,賈政見了,倒也喜歡,那新人坐了牀便要揭起蓋頭的,鳳姐早已防備,故請賈母王夫人等進去照應。
\end{parag}


\begin{parag}
    寶玉此時到底有些傻氣,便走到新人跟前說道:“妹妹身上好了?好些天不見了,蓋着這勞什子做什麼!”欲待要揭去,反把賈母急出一身冷汗來。寶玉又轉念一想道:“林妹妹是愛生氣的,不可造次。”又歇了一歇,仍是按捺不住,只得上前揭了。喜娘接去蓋頭,雪雁走開,鶯兒等上來伺候。寶玉睜眼一看,好象寶釵,心裏不信,自己一手持燈,一手擦眼,一看,可不是寶釵麼!只見他盛妝豔服,豐肩懦體,鬟低鬢嚲,眼瞬息微,真是荷粉露垂,杏花煙潤了。寶玉發了一回怔,又見鶯兒立在旁邊,不見了雪雁。寶玉此時心無主意,自己反以爲是夢中了,呆呆的只管站着。衆人接過燈去,扶了寶玉仍舊坐下,兩眼直視,半語全無。賈母恐他病發,親自扶他上牀。鳳姐尤氏請了寶釵進入裏間牀上坐下,寶釵此時自然是低頭不語。寶玉定了一回神,見賈母王夫人坐在那邊,便輕輕的叫襲人道:“我是在那裏呢?這不是做夢麼?”襲人道:“你今日好日子,什麼夢不夢的混說。老爺可在外頭呢。”寶玉悄悄兒的拿手指着道:“坐在那裏這一位美人兒是誰?”襲人握了自己的嘴,笑的說不出話來,歇了半日才說道:“是新娶的二奶奶。”衆人也都回過頭去,忍不住的笑。寶玉又道:“好糊塗,你說二奶奶到底是誰?”襲人道:“寶姑娘。”寶玉道:“林姑娘呢?”襲人道:“老爺作主娶的是寶姑娘,怎麼混說起林姑娘來。”寶玉道:“我纔剛看見林姑娘了麼,還有雪雁呢,怎麼說沒有。你們這都是做什麼頑呢?”鳳姐便走上來輕輕的說道:“寶姑娘在屋裏坐着呢。別混說,回來得罪了他,老太太不依的。”寶玉聽了,這會子糊塗更利害了。本來原有昏憒的病,加以今夜神出鬼沒,更叫他不得主意,便也不顧別的了,口口聲聲只要找林妹妹去。賈母等上前安慰,無奈他只是不懂。又有寶釵在內,又不好明說。知寶玉舊病復發,也不講明,只得滿屋裏點起安息香來,定住他的神魂,扶他睡下。衆人鴉雀無聞,停了片時,寶玉便昏沉睡去。賈母等才得略略放心,只好坐以待旦,叫鳳姐去請寶釵安歇。寶釵置若罔聞,也便和衣在內暫歇。賈政在外,未知內裏原由,只就方纔眼見的光景想來,心下倒寬了。恰是明日就是起程的吉日,略歇了一歇,衆人賀喜送行。賈母見寶玉睡着,也回房去暫歇。
\end{parag}


\begin{parag}
    次早,賈政辭了宗祠,過來拜別賈母,稟稱:“不孝遠離,惟願老太太順時頤養。兒子一到任所,即修稟請安,不必掛念。寶玉的事,已經依了老太太完結,只求老太太訓誨。”賈母恐賈政在路不放心,並不將寶玉復病的話說起,只說:“我有一句話,寶玉昨夜完姻,並不是同房。今日你起身,必該叫他遠送纔是。他因病沖喜,如今纔好些,又是昨日一天勞乏,出來恐怕着了風。故此問你,你叫他送呢,我即刻去叫他,你若疼他,我就叫人帶了他來,你見見,叫他給你磕頭就算了。”賈政道:“叫他送什麼,只要他從此以後認真唸書,比送我還喜歡呢。”賈母聽了,又放了一條心,便叫賈政坐着,叫鴛鴦去如此如此,帶了寶玉,叫襲人跟着來。鴛鴦去了不多一會,果然寶玉來了,仍是叫他行禮。寶玉見了父親,神志略斂些,片時清楚,也沒什麼大差。賈政吩咐了幾句,寶玉答應了。賈政叫人扶他回去了,自己回到王夫人房中,又切實的叫王夫人管教兒子,斷不可如前嬌縱。明年鄉試,務必叫他下場。王夫人一一的聽了,也沒提起別的。即忙命人扶了寶釵過來,行了新婦送行之禮,也不出房。其餘內眷俱送至二門而回。賈珍等也受了一番訓飭。大家舉酒送行,一班子弟及晚輩親友,直送至十里長亭而別。不言賈政起程赴任。且說寶玉回來,舊病陡發,更加昏憒,連飲食也不能進了。未知性命如何,下回分解。
\end{parag}
