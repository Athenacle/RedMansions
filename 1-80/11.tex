\chap{一十一}{慶壽辰寧府排家宴 見熙鳳賈瑞起淫心}

\begin{parag}
    詩曰:
\end{parag}


\begin{poem}
    \begin{pl}一步行來錯,回頭已百年。\end{pl}


    \begin{pl}古今風月鑑,多少泣黃泉!\end{pl}
\end{poem}


\begin{parag}
    \begin{note}蒙:幻境無端換境生,玉樓春暖述乖情。鬧中尋靜渾閒事,運得靈機屬鳳卿。\end{note}
\end{parag}


\begin{parag}
    話說是日賈敬的壽辰,賈珍先將上等可喫的東西,稀奇些的果品,裝了十六大捧盒,著賈蓉帶領家下人等與賈敬送去,向賈蓉說道:“你留神看太爺喜歡不喜歡,你就行了禮來。你說:‘我父親遵太爺的話未敢來,在家裏率領閤家都朝上行了禮了。’”賈蓉聽罷,即率領家人去了。
\end{parag}


\begin{parag}
    這裏漸漸的就有人來了。先是賈璉賈薔到來,先看了各處的座位,並問:“有什麼頑意兒沒有?”家人答道:“我們爺原算計請太爺今日來家來,所以未敢預備頑意兒。前日聽見太爺又不來了,現叫奴才們找了一班小戲兒並一檔子打十番的,都在園子裏戲臺上預備著呢。”
\end{parag}


\begin{parag}
    次後邢夫人、王夫人、鳳姐兒、寶玉都來了,賈珍並尤氏接了進去。尤氏的母親已先在這裏呢。大家見過了,彼此讓了坐。賈珍尤氏二人親自遞了茶,因說道:“老太太原是老祖宗,我父親又是侄兒,這樣日子,原不敢請他老人家,但是這個時候,天氣正涼爽,滿園的菊花又盛開,請老祖宗過來散散悶,看著衆兒孫熱鬧熱鬧,是這個意思。誰知老祖宗又不肯賞臉。”鳳姐兒未等王夫人開口,先說道:“老太太昨日還說要來著呢,因爲晚上看著寶兄弟他們喫桃兒,老人家又嘴饞,吃了有大半個,五更天的時候就一連起來了兩次,\begin{note}蒙側:此一問一答,即景生情,請教是真是假?非身經其事者,想不到,寫不出。\end{note}今日早晨略覺身子倦些。因叫我回大爺,今日斷不能來了,說有好喫的要幾樣,還要很爛的。”\begin{note}蒙側:是。\end{note}賈珍聽了笑道:“我說老祖宗是愛熱鬧的,今日不來,必定有個原故,若是這麼著就是了。”
\end{parag}


\begin{parag}
    王夫人道:“前日聽見你大妹妹說,蓉哥兒媳婦兒身上有些不大好,到底是怎麼樣?”尤氏道:“他這個病得的也奇。上月中秋還跟著老太太,太太們頑了半夜,回家來好好的。到了二十後,一日比一日覺懶,也懶待喫東西,這將近有半個多月了。經期又有兩個月沒來。”邢夫人接著說道:“別是喜罷?”\begin{note}蒙側:此書總是一幅《雲龍圖》。\end{note}
\end{parag}


\begin{parag}
    正說著,外頭人回道:“大老爺,二老爺並一家子的爺們都來了,在廳上呢。”賈珍連忙出去了。這裏尤氏方說道:“從前大夫也有說是喜的。昨日馮紫英薦了他從學過的一個先生,醫道很好,瞧了說不是喜,竟是很大的一個症候。昨日開了方子,吃了一劑藥,今日頭眩的略好些,別的仍不見怎麼樣大見效。”鳳姐兒道:“我說他不是十分支持不住,今日這樣的日子,再也不肯不扎掙著上來。”尤氏道:“你是初三日在這裏見他的,他強扎掙了半天,也是因你們孃兒兩個好的上頭,他才戀戀的捨不得去。”鳳姐兒聽了,眼圈兒紅了半天,半日方說道:“真是‘天有不測風雲,人有旦夕禍福’。\begin{note}蒙側:揣摩的極平常言語來寫無涯之幻景幻情,反作了悟之意,且又轉至別處,真是月下梨花,幾不能辨。\end{note}這個年紀,倘或就因這個病上怎麼樣了,人還活著有甚麼趣兒!”\begin{note}蒙側:大英雄多在此等處悟得,每能超凡入聖。\end{note}正說話間,賈蓉進來,給邢夫人、王夫人、鳳姐兒前都請了安,方回尤氏道:“方纔我去給太爺送喫食去,並回說我父親在家中伺候老爺們,款待一家子的爺們,遵太爺的話未敢來。太爺聽了甚喜歡,說:‘這纔是。’叫告訴父親母親好生伺候太爺太太們,叫我好生伺候叔叔嬸子們並哥哥們。還說那《陰騭文》,叫急急的刻出來,印一萬張散人。我將此話都回了我父親了。我這會子得快出去打發太爺們併合家爺們喫飯。”鳳姐兒說:“蓉哥兒,你且站住。你媳婦今日到底是怎麼著?”賈蓉皺皺眉說道:“不好麼!嬸子回來瞧瞧去就知道了。”\begin{note}蒙側:伏線自然。\end{note}於是賈蓉出去了。
\end{parag}


\begin{parag}
    這裏尤氏向邢夫人,王夫人道:“太太們在這裏喫飯阿,還是在園子裏喫去好?小戲兒現預備在園子裏呢。”王夫人向邢夫人道:“我們索性吃了飯再過去罷,也省好些事。”邢夫人道:“很好。” 於是尤氏就吩咐媳婦婆子們:“快送飯來。”門外一齊答應了一聲,都各人端各人的去了。不多一時,擺上了飯。尤氏讓邢夫人,王夫人並他母親都上了坐,他與鳳姐兒,寶玉側席坐了。邢夫人,王夫人道:“我們來原爲給大老爺拜壽,這不竟是我們來過生日來了麼?”鳳姐兒說道:“大老爺原是好養靜的,已經修煉成了,也算得是神仙了。太太們這麼一說,這就叫作‘心到神知’了。”\begin{note}蒙側:此等趣語,亦不肯無著落。\end{note}一句話說的滿屋裏的人都笑起來了。
\end{parag}


\begin{parag}
    於是,尤氏的母親並邢夫人、王夫人、鳳姐兒都喫畢飯,漱了口,淨了手,才說要往園子裏去,賈蓉進來向尤氏說道:“老爺們並衆位叔叔哥哥兄弟們也都吃了飯了。大老爺說家裏有事,二老爺是不愛聽戲又怕人鬧的慌,都纔去了。別的一家子爺們都被璉二叔並薔兄弟讓過去聽戲去了。方纔南安郡王、東平郡王、西寧郡王、北靜郡王四家王爺,並鎮國公牛府等六家,忠靖侯史府等八家,都差人持了名帖送壽禮來,俱回了我父親,先收在帳房裏了,禮單都上上檔子了。老爺的領謝的名帖都交給各來人了,各來人也都照舊例賞了,衆來人都讓吃了飯纔去了。母親該請二位太太,老孃,嬸子都過園子裏坐著去罷。”\begin{note}蒙側:人送壽禮,是爲園子;回人去的去了在的在,是爲可以過園子裏坐;園子裏坐可以轉入正文中之幻情;幻情裏有乖情,而乖情初寫,偏不乖。真是慧心神手!\end{note}尤氏道:“也是才喫完了飯,就要過去了。”
\end{parag}


\begin{parag}
    鳳姐兒說:“我回太太,我先瞧瞧蓉哥兒媳婦,我再過去。”王夫人道:“很是,我們都要去瞧瞧他,倒怕他嫌鬧的慌,\begin{note}蒙側:爲下文留地步。\end{note}說我們問他好罷。”尤氏道:“好妹妹,媳婦聽你的話,你去開導開導他,我也放心。你就快些過園子裏來。”寶玉也要跟了鳳姐兒去瞧秦氏去,王夫人道:“你看看就過去罷,那是侄兒媳婦。”於是尤氏請了邢夫人,王夫人並他母親都過會芳園去了。
\end{parag}


\begin{parag}
    鳳姐兒、寶玉方和賈蓉到秦氏這邊來。進了房門,悄悄的走到裏間房門口,秦氏見了,就要站起來,鳳姐兒說:“快別起來,看起猛了頭暈。”\begin{note}蒙側:知心每每如此。\end{note}於是鳳姐兒就緊走了兩步,拉住秦氏的手,說道:“我的奶奶!怎麼幾日不見,就瘦的這麼著了!”於是就坐在秦氏坐的褥子上。寶玉也問了好,坐在對面椅子上。賈蓉叫:“快倒茶來,嬸子和二叔在上房還未喝茶呢。”
\end{parag}


\begin{parag}
    秦氏拉著鳳姐兒的手,強笑道:“這都是我沒福。這樣人家,公公婆婆當自己的女孩兒似的待。\begin{note}蒙側:正寫幻情,偏作錐心刺骨語。呼渡河者三,是一意。\end{note}嬸孃的侄兒雖說年輕,卻也是他敬我,我敬他,從來沒有紅過臉兒。就是一家子的長輩同輩之中,除了嬸子倒不用說了,別人也從無不疼我的,也無不和我好的。這如今得了這個病,把我那要強的心一分也沒了。公婆跟前未得孝順一天,就是嬸孃這樣疼我,我就有十分孝順的心,如今也不能夠了。我自想著,未必熬的過年去呢。”
\end{parag}


\begin{parag}
    寶玉正眼瞅著那《海棠春睡圖》並那秦太虛寫的“嫩寒鎖夢因春冷,芳氣籠人是酒香”的對聯,不覺想起在這裏睡晌覺夢到“太虛幻境”的事來。正自出神,聽得秦氏說了這些話,如萬箭攢心,那眼淚不知不覺就流下來了。鳳姐兒心中雖十分難過,但恐怕病人見了衆人這個樣兒反添心酸,倒不是來開導勸解的意思了。見寶玉這個樣子,因說道:“寶兄弟,你忒婆婆媽媽的了。他病人不過是這麼說,那裏就到得這個田地了?況且能多大年紀的人,略病一病兒就這麼想那麼想的,這不是自己倒給自己添病了麼?”賈蓉道:“他這病也不用別的,只是喫得些飲食就不怕了。”\begin{note}蒙側:各人是各人伎倆,一絲不亂,一毫不遺。\end{note}鳳姐兒道:“寶兄弟,太太叫你快過去呢。你別在這裏只管這麼著,倒招的媳婦也心裏不好。太太那裏又惦著你。”因向賈蓉說道:“你先同你寶叔叔過去罷,\begin{note}蒙側:爲本。\end{note}我還略坐一坐兒。”賈蓉聽說,即同寶玉過會芳園來了。
\end{parag}


\begin{parag}
    這裏鳳姐兒又勸解了秦氏一番,又低低的說了許多衷腸話兒,尤氏打發人請了兩三遍,鳳姐兒才向秦氏說道:“你好生養著罷,我再來看你。合該你這病要好,所以前日就有人薦了這個好大夫來,再也是不怕的了。”秦氏笑道:“任憑神仙也罷,治得病治不得命。嬸子,我知道我這病不過是捱日子。”鳳姐兒說道:“你只管這麼想著,病那裏能好呢?總要想開了纔是。況且聽得大夫說,若是不治,怕的是春天不好呢。如今才九月半,還有四五個月的工夫,什麼病治不好呢?咱們若是不能喫人蔘的人家,這也難說了,你公公婆婆聽見治得好你,別說一日二錢人蔘,就是二斤也能夠喫的起。好生養著罷,我過園子裏去了。”秦氏又道:“嬸子,恕我不能跟過去了。閒了時候還求嬸子常過來瞧瞧我,咱們娘兒們坐坐,多說幾遭話兒。”鳳姐兒聽了,不覺得又眼圈兒一紅,遂說道:“我得了閒兒必常來看你。”於是鳳姐兒帶領跟來的婆子丫頭並寧府的媳婦婆子們,從裏頭繞進園子的便門來。\begin{note}蒙側:偏不獨行,用此等反克文字。\end{note}但只見:
\end{parag}


\begin{qute2sp}
    黃花滿地,白柳橫坡。小橋通若耶之溪,曲徑接天台之路。\begin{note}蒙側:點明題目。\end{note}石中清流激湍,籬落飄香,樹頭紅葉翩翻,疏林如畫。西風乍緊,初罷鶯啼,暖日當暄,又添蛩語。遙望東南,建幾處依山之榭,縱觀西北,結三間臨水之軒。笙簧盈耳。別有幽情,羅綺穿林,倍添韻致。
\end{qute2sp}


\begin{parag}
    鳳姐兒正自看園中景緻,一步步行來讚賞。猛然從假山石後走過一個人來,向前對鳳姐兒說道:“請嫂子安。”鳳姐兒猛然見了,將身子望後一退,說道:“這是瑞大爺不是?”賈瑞說道:“嫂子連我也不認得了?不是我是誰!”鳳姐兒道:“不是不認得,猛然一見,不想到是大爺到這裏來。”\begin{note}蒙側:作者何等心思,能在此等事想到如此出言。漸入之妙,無過於此。\end{note}賈瑞道:“也是合該我與嫂子有緣。我方纔偷出了席,在這個清淨地方略散一散,不想就遇見嫂子也從這裏來。這不是有緣麼?”\begin{note}蒙側:重點“有緣”二字,方是筆力。\end{note}一面說著,一面拿眼睛不住的覷著鳳姐兒。
\end{parag}


\begin{parag}
    鳳姐兒是個聰明人,見他這個光景,如何不猜透八九分呢,因向賈瑞假意含笑道:“怨不得你哥哥時常提你,說你很好。今日見了,聽你說這幾句話兒,就知道你是個聰明和氣的人了。這會子我要到太太們那裏去,不得和你說話兒,等閒了咱們再說話兒罷。”賈瑞道:“我要到嫂子家裏去請安,又恐怕嫂子年輕,不肯輕易見人。”鳳姐兒假意笑道:“一家子骨肉,說什麼年輕不年輕的話。”賈瑞聽了這話,再不想到今日得這個奇遇,那神情光景亦發不堪難看了。鳳姐兒說道:“你快入席去罷,仔細他們拿住罰你酒。”賈瑞聽了,身上已木了半邊,慢慢的一面走著,一面回過頭來看。鳳姐兒故意的把腳步放遲了些兒,見他去遠了,心裏暗忖道:“這纔是知人知面不知心呢,那裏有這樣禽獸的人呢!\begin{note}蒙側:大英雄氣概。作者以此命鳳,其有爲耶?\end{note}他如果如此,幾時叫他死在我的手裏,他才知道我的手段!”
\end{parag}


\begin{parag}
    於是鳳姐兒方移步前來。將轉過了一重山坡,見兩三個婆子慌慌張張的走來,見了鳳姐兒,笑說道:“我們奶奶見二奶奶只是不來,急的了不得,叫奴才們又來請奶奶來了。”\begin{note}蒙側:別者必將遇賈瑞的事聲張一番,以表情節。此文偏若無事,一則可以見熙鳳非凡,一則可以見熙鳳包含廣大。\end{note}鳳姐兒說道:“你們奶奶就是這麼急腳鬼似的。”鳳姐兒慢慢的走著,問:“戲唱了幾齣了?”那婆子回道:“有八九出了。”說話之間,已來到了天香樓的後門,見寶玉和一羣丫頭們在那裏玩呢。鳳姐兒說道:“寶兄弟,別忒淘氣了。”\begin{note}蒙側:照應前文。\end{note}有一個丫頭說道:“太太們都在樓上坐著呢,請奶奶就從這邊上去罷。”
\end{parag}


\begin{parag}
    鳳姐兒聽了,款步提衣上了樓,見尤氏已在樓梯口等著呢。尤氏笑說道:“你們孃兒兩個忒好了,見了面總捨不得來了。你明日搬來和他住著罷。你坐下,我先敬你一鍾。”於是鳳姐兒在邢、王二夫人前告了坐,又在尤氏的母親前周旋了一遍,仍同尤氏坐在一桌上喫酒聽戲。尤氏叫拿戲單來,讓鳳姐兒點戲,鳳姐兒說道:“親家太太和太太們在這裏,我如何敢點。”邢夫人王夫人說道:“我們和親家太太都點了好幾出了,你點兩出好的我們聽。”鳳姐兒立起身來答應了一聲,方接過戲單,從頭一看,點了一出《還魂》,一出《彈詞》,遞過戲單去說:“現在唱的這《雙官誥》,\begin{note}蒙側:點下文。\end{note}唱完了,再唱這兩出,也就是時候了。”王夫人道:“可不是呢,也該趁早叫你哥哥嫂子歇歇,他們又心裏不靜。”尤氏說道:“太太們又不常過來,娘兒們多坐一會子去,纔有趣兒,天還早呢。”鳳姐兒立起身來望樓下一看,說:“爺們都往那裏去了?”旁邊一個婆子道:“爺們纔到凝曦軒,帶了打十番的那裏喫酒去了。”鳳姐兒說道:“在這裏不便宜,背地裏又不知幹什麼去了!”\begin{note}蒙側:偏是愛喫酸醋。\end{note}尤氏笑道:“那裏都象你這麼正經人呢。”
\end{parag}


\begin{parag}
    於是說說笑笑,點的戲都唱完了,方纔撤下酒席,擺上飯來。喫畢,大家纔出園子來,到上房坐下,吃了茶,方纔叫預備車,向尤氏的母親告了辭。尤氏率同衆姬妾並家下婆子媳婦們方送出來,賈珍率領衆子侄都在車旁侍立,等候著呢,見了邢夫人王夫人道:“二位嬸子明日還過來逛逛。”王夫人道:“罷了,我們今日整坐了一日,也乏了,明日歇歇罷。”於是都上車去了。賈瑞猶不時拿眼睛覷著鳳姐兒。\begin{note}蒙側:無又不足不盡處。\end{note}賈珍等進去後,李貴才拉過馬來,寶玉騎上,隨了王夫人去了。這裏賈珍同一家子的弟兄子侄喫過了晚飯,方大家散了。
\end{parag}


\begin{parag}
    次日,仍是衆族人等鬧了一日,不必細說。此後鳳姐兒不時親自來看秦氏。秦氏也有幾日好些,也有幾日仍是那樣。賈珍、尤氏、賈蓉好不焦心。\begin{note}蒙側:陪襯補足。\end{note}
\end{parag}


\begin{parag}
    且說賈瑞到榮府來了幾次,偏都遇見鳳姐兒往寧府那邊去了。這年正是十一月三十日冬至。到交節的那幾日,賈母、王夫人、鳳姐兒日日差人去看秦氏,回來的人都說:“這幾日也沒見添病,也不見甚好。”王夫人向賈母說:“這個症候,遇著這樣大節不添病,就有好大的指望了。”賈母說:“可是呢,好個孩子,要是有些原故,可不叫人疼死。”說著,一陣心酸,叫鳳姐兒說道:“你們孃兒兩個也好了一場,明日大初一,過了明日,你後日再去看一看他去。你細細的瞧瞧他那光景,倘或好些兒,你回來告訴我,我也喜歡喜歡。那孩子素日愛喫的,你也常叫人做些給他送過去。”鳳姐兒一一的答應了。
\end{parag}


\begin{parag}
    到了初二日,吃了早飯,來到寧府,看見秦氏的光景,雖未甚添病,但是那臉上身上的肉全瘦幹了。於是和秦氏坐了半日,說了些閒話兒,又將這病無妨的話開導了一遍。秦氏說道:“好不好,春天就知道了。如今現過了冬至,又沒怎麼樣,或者好的了也未可知。嬸子回老太太、太太放心罷。\begin{note}蒙側:文字一變。人於將死時也應有一變。\end{note}昨日老太太賞的那棗泥餡的山藥糕,我倒吃了兩塊,倒象克化的動似的。”鳳姐兒說道:“明日再給你送來。我到你婆婆那裏瞧瞧,就要趕著回去回老太太的話去。”秦氏道:“嬸子替我請老太太、太太安罷。”
\end{parag}


\begin{parag}
    鳳姐兒答應著就出來了,到了尤氏上房坐下。尤氏道:“你冷眼瞧媳婦是怎麼樣?”鳳姐兒低了半日頭,說道:“這實在沒法兒了。你也該將一應的後事用的東西給他料理料理,衝一衝也好。”\begin{note}蒙側:伏下文代辦理喪事。\end{note}尤氏道:“我也叫人暗暗的預備了。就是那件東西不得好木頭,暫且慢慢的辦罷。”於是鳳姐兒吃了茶,說了一會子話兒,說道:“我要快回去回老太太的話去呢。”尤氏道:“你可緩緩的說,別嚇著老太太。”鳳姐兒道:“我知道。”於是鳳姐兒就回來了。到了家中,見了賈母,說:“蓉哥兒媳婦請老太太安,給老太太磕頭,說他好些了,求老祖宗放心罷。他再略好些,還要給老祖宗磕頭請安來呢。”賈母道:“你看他是怎麼樣?”鳳姐兒說:“暫且無妨,精神還好呢。”\begin{note}蒙側:“精神還好呢”五字,寫得出神入化。\end{note}賈母聽了,沉吟了半日,因向鳳姐兒說:“你換換衣服歇歇去罷。”
\end{parag}


\begin{parag}
    鳳姐兒答應著出來,見過了王夫人,到了家中,平兒將烘的家常的衣服給鳳姐兒換了。鳳姐兒方坐下,問道:“家裏沒有什麼事麼?”平兒方端了茶來,遞了過去,說道:“沒有什麼事。就是那三百銀子的利銀,旺兒媳婦送進來,我收了。\begin{note}蒙側:陪。\end{note}再有瑞大爺使人來打聽\begin{note}蒙側:正。\end{note}奶奶在家沒有,\begin{note}蒙側:沒他。\end{note}他要來請安說話。”鳳姐兒聽了,哼了一聲,說道:“這畜生合該作死,看他來了怎麼樣!”平兒因問道:“這瑞大爺是因什麼只管來?”鳳姐兒遂將九月裏寧府園子裏遇見他的光景,他說的話,都告訴了平兒。平兒說道:“癩蛤蟆想天鵝肉喫,沒人倫的混帳東西,起這個念頭,叫他不得好死!”鳳姐兒道:“等他來了,我自有道理。”不知賈瑞來時作何光景,且聽下回分解。
\end{parag}


\begin{parag}
    \begin{note}蒙:將可卿之病將死,作幻情一劫;又將賈瑞之遇唐突,作幻情一變。下回同歸幻境,真風馬牛不相及之談。同範並趨,毫無滯礙,靈活之至,飄飄欲仙。默思作者其人之心,其人之形,其人之神,其人之文,比宋玉、子建一般心性,一流人物。\end{note}
\end{parag}
