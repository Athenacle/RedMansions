\chap{二十八}{蔣玉菡情贈茜香羅 薛寶釵羞籠紅麝串}


\begin{parag}
    \begin{note}庚:茜香羅、紅麝串寫於一回,蓋琪官雖系優人,後回與襲人供奉玉兄寶卿得同終始者,非泛泛之文也。自“聞曲”回以後,回回寫藥方,是白描顰兒添病也。\end{note}
\end{parag}


\begin{parag}
    話說林黛玉只因昨夜晴雯不開門一事,錯疑在寶玉身上。至次日又可巧遇見餞花之期,正是一腔無明正未發泄,又勾起傷春愁思,因把些殘花落瓣去掩埋,由不得感花傷己,哭了幾聲,便隨口唸了幾句。不想寶玉在山坡上聽見,先不過點頭感嘆;次後聽到“儂今葬花人笑癡,他年葬儂知是誰”,“一朝春盡紅顏老,花落人亡兩不知”等句,不覺慟倒山坡之上,懷裏兜的落花撒了一地。試想林黛玉的花顏月貌,將來亦到無可尋覓之時,寧不心碎腸斷!既黛玉終歸無可尋覓之時,推之於他人,如寶釵、香菱、襲人等,亦可到無可尋覓之時矣。寶釵等終歸無可尋覓之時,則自己又安在哉?且自身尚不知何在何往,則斯處、斯園、斯花、斯柳,又不知當屬誰姓矣!因此一而二,二而三,反覆推求了去,\begin{note}庚側:百轉千回矣。\end{note}真不知此時此際欲爲何等蠢物,杳無所知,逃大造,出塵網,使可解釋這段悲傷。\begin{note}甲側:非大善知識,說不出這句話來。\end{note}\begin{note}甲眉:不言煉句煉字辭藻工拙,只想景想情想事想理,反覆推求悲傷感慨,乃玉兄一生之天性。真顰兒之知己,玉兄外實無一人。想昨阻批《葬花吟》之客,嫡是玉兄之化身無疑。餘幾作點金爲鐵之人,笨甚笨甚!\end{note}正是:花影不離身左右,鳥聲只在耳東西。\begin{note}甲側:二句作禪語參。\end{note}\begin{note}甲眉:一大篇《葬花吟》卻如此收拾,真好機思筆仗,令人焉的不叫絕稱奇!\end{note}
\end{parag}


\begin{parag}
    那林黛玉正自傷感,忽聽山坡上也有悲聲,心下想道:“人人都笑我有些癡病,難道還有一個癡子不成?”\begin{note}甲側:豈敢豈敢。\end{note}想著,抬頭一看,見是寶玉。林黛玉看見,便道:“啐!我道是誰,原來是這個狠心短命的……”剛說到“短命”二字,又把口掩住,\begin{note}甲側:“情情”,不忍道出“的”字來。\end{note}長嘆了一聲,\begin{note}庚側:不忍也。\end{note}自己抽身便走了。
\end{parag}


\begin{parag}
    這裏寶玉悲慟了一回,忽然抬頭不見了黛玉,便知黛玉看見他躲開了,自己也覺無味,抖抖土起來,下山尋歸舊路,\begin{note}甲側:折得好,誓不寫開門見山文字。\end{note}往怡紅院來。可巧\begin{note}庚側:哄人字眼。\end{note}看見林黛玉在前頭走,連忙趕上去,說道:“你且站住。我知你不理我,我只說一句話,從今後撂開手。”\begin{note}甲側:非此三字難留蓮步,玉兄之機變如此。\end{note}林黛玉回頭看見是寶玉,待要不理他,聽他說“只說一句話,從此撂開手”,這話裏有文章,少不得站住說道: “有一句話,請說來。”寶玉笑道:“兩句話,說了你聽不聽?”\begin{note}甲側:相離尚遠,用此句補空,好近阿顰。\end{note}黛玉聽說,回頭就走。\begin{note}庚側:走得是。\end{note}寶玉在身後面嘆道:“既有今日,何必當初!”\begin{note}甲側:自言自語,真是一句話。\end{note}林黛玉聽見這話,由不得站住,回頭道:“當初怎麼樣?今日怎麼樣?”寶玉嘆道:\begin{note}甲側:以下乃答言,非一句話也。\end{note}“當初姑娘來了,那不是我陪著頑笑?\begin{note}甲側:我阿顰之惱,玉兄實摸不著,不得不將自幼之苦心實事一訴,方可明心以白今日之故,勿作閒文看。\end{note}憑我心愛的,姑娘要,就拿去;我愛喫的,聽見姑娘也愛喫,連忙乾乾淨淨收著等姑娘喫。一桌子喫飯,一牀上睡覺。丫頭們想不到的,我怕姑娘生氣,我替丫頭們想到了。我心裏想著:姊妹們從小兒長大,親也罷,熱也罷,和氣到了兒,才見得比人好。\begin{note}庚側:要緊語。\end{note}如今誰承望姑娘人大心大,\begin{note}庚側:反派不是。\end{note}不把我放在眼睛裏,倒把外四路的什麼寶姐姐\begin{note}庚側:心事。\end{note}鳳姐姐\begin{note}甲側:用此人瞞看官也,瞞顰兒也。心動阿顰在此數句也。一節頗似說辭,玉兄口中卻是衷腸話。\end{note}的放在心坎兒上,倒把我三日不理四日不見的。我又沒個親兄弟親姊妹。──雖然有兩個,你難道不知道是和我隔母的?我也和你似的獨出,只怕同我的心一樣。誰知我是白操了這個心,弄的有\begin{note}寃\end{note}無處訴!”說著不覺滴下眼淚來。\begin{note}甲側:玉兄淚非容易有的。\end{note}
\end{parag}


\begin{parag}
    黛玉耳內聽了這話,眼內見了這形景,心內不覺灰了大半,也不覺滴下淚來,低頭不語。寶玉見他這般形景,遂又說道:“我也知道我如今不好了,但只憑著怎麼不好,萬不敢在妹妹跟前有錯處。\begin{note}庚側:有是語。\end{note}便有一二分錯處,你倒是或教導我,戒我下次,\begin{note}庚側:可憐語。\end{note}或罵我兩句,打我兩下,我都不灰心。誰知你總不理我,\begin{note}庚側:實難爲情。\end{note}叫我摸不著頭腦,少魂失魄,不知怎麼樣纔好。\begin{note}庚側:真有是事。\end{note}就便死了,也是個屈死鬼,任憑高僧高道懺悔也不能超生,\begin{note}庚側:又瞞看官及批書人。\end{note}還得你申明瞭緣故,我才得託生呢!”
\end{parag}


\begin{parag}
    黛玉聽了這個話,不覺將昨晚的事都忘在九霄雲外了,\begin{note}甲側:“情情”本來面目也。\end{note}\begin{note}庚側:“情情”衷腸。\end{note}便說道:“你既這麼說,昨兒爲什麼我去了,你不叫丫頭開門?”\begin{note}庚側:正文,該問。\end{note}寶玉詫異道:“這話從那裏說起?\begin{note}庚側:實實不知。\end{note}我要是這麼樣,立刻就死了!”\begin{note}甲側:急了。\end{note}林黛玉啐道:\begin{note}庚側:如聞。\end{note}“大清早起死呀活的,也不忌諱。你說有呢就有,沒有就沒有,起什麼誓呢。”寶玉道:“實在沒有見你去。就是寶姐姐坐了一坐,\begin{note}庚側:不要兄言,彼已親睹。\end{note}就出來了。”林黛玉想了一想,笑道:“是了。想必是你的丫頭們懶待動,喪聲歪氣的也是有的。”寶玉道:“想必是這個原故。等我回去問了是誰,教訓教訓他們就好了。”\begin{note}庚側:玉兄口氣畢真。\end{note}黛玉道:“你的那些姑娘們\begin{note}庚側:不快活之稱。\end{note}也該教訓教訓,\begin{note}庚側:照樣的妙!\end{note}只是我論理不該說。今兒得罪了我的事小,倘或明兒寶姑娘來,\begin{note}庚側:也還一句,的是心坎上人。\end{note}什麼貝姑娘來,也得罪了,事情豈不大了。”\begin{note}甲側:至此心事全無矣。\end{note}說著抿著嘴笑。寶玉聽了,又是咬牙,又是笑。
\end{parag}


\begin{parag}
    二人正說話,只見丫頭來請喫飯,\begin{note}甲側:收拾得乾淨。\end{note}遂都往前頭來了。王夫人見了林黛玉,因問道:“大姑娘,你喫那鮑太醫的藥可好些?”\begin{note}庚側:是新換了的口氣。\end{note}林黛玉道:“也不過這麼著。老太太還叫我喫王大夫的藥呢。”\begin{note}庚側:何如?\end{note}寶玉道:“太太不知道,林妹妹是內症,先天生的弱,所以禁不住一點風寒,不過喫兩劑煎藥就好了,散了風寒,還是喫丸藥\begin{note}甲側:引下文。\end{note}的好。”王夫人道:“前兒大夫說了個丸藥的名字,我也忘了。”寶玉道:“我知道那些丸藥,不過叫他喫什麼人蔘養榮丸。”王夫人道:“不是。”寶玉又道:“八珍益母丸?左歸?右歸?再不,就是麥味地黃丸。”王夫人道:“都不是。我只記得有個‘金剛’兩個字的。”\begin{note}甲側:奇文奇語。\end{note}寶玉扎手笑道:\begin{note}甲側:慈母前放肆了。\end{note}\begin{note}庚眉:此寫玉兄,亦是釋卻心中一夜半日要事,故大大一泄。己冬夜。\end{note}“從來沒聽見有個什麼‘金剛丸’。若有了‘金剛丸’,自然有‘菩薩散’了!”\begin{note}甲側:寶玉因黛玉事完,一心無掛礙,故不知不覺手之舞之,足之蹈之。\end{note}說的滿屋裏人都笑了。寶釵抿嘴笑道:“想是天王補心丹。”\begin{note}甲側:慧心人自應知之。\end{note}王夫人笑道:“是這個名兒。如今我也糊塗了。”寶玉道:“太太倒不糊塗,都是叫‘金剛’‘菩薩’支使糊塗了。”\begin{note}甲側:是語甚對,餘幼時所聞之語合符,哀哉傷哉!\end{note}王夫人道:“扯你孃的臊!又欠你老子捶你了。”\begin{note}庚側:伏線。\end{note}寶玉笑道:“我老子再不爲這個捶我的。”\begin{note}甲側:此語亦不假。\end{note}
\end{parag}


\begin{parag}
    王夫人又道:“既有這個名兒,明兒就叫人買些來喫。”\begin{note}庚眉:寫藥案是暗度顰卿病勢漸加之筆,非泛泛閒文也。丁亥夏。笏叟。\end{note}寶玉笑道:“這些都不中用的。太太給我三百六十兩銀子,我替妹妹配一料丸藥,包管一料不完就好了。”王夫人道:“放屁!什麼藥就這麼貴?”寶玉笑道:“當真的呢,我這個方子比別的不同。那個藥名兒也古怪,一時也說不清。只講那頭胎紫河車,\begin{note}庚側:只聞名。\end{note}人形帶葉參,三百六十兩還不夠。龜大何首烏,\begin{note}庚側:聽也不曾聽過。\end{note}千年松根茯苓膽,\begin{note}庚眉:寫得不犯冷香丸方子。前“玉生香”回中顰雲“他有金你有玉;他有冷香你豈不該有暖香?” 是寶玉無藥可配矣。今顰兒之劑若許材料皆系滋補熱性之藥,兼有許多奇物,而尚未擬名,何不竟以“暖香”名之?以代補寶玉之不足,豈不三人一體矣。己冬夜。\end{note}諸如此類的藥都不算爲奇,\begin{note}庚側:還有奇的。\end{note}只在羣藥裏算。那爲君的藥,說起來唬人一跳。前兒薛大哥哥求了我一二年,我纔給了他這方子。他拿了方子去又尋了二三年,花了有上千的銀子,才配成了。太太不信,只問寶姐姐。”寶釵聽說,笑著搖手兒說:“我不知道,也沒聽見。你別叫姨娘問我。”王夫人笑道:“到底是寶丫頭,好孩子,不撒謊。”寶玉站在當地,聽見如此說,一回身把手一拍,說道:“我說的倒是真話呢,倒說我撒謊。”口裏說著,忽一回身,只見林黛玉坐在寶釵身後抿著嘴笑,用手指頭在臉上畫著羞他。\begin{note}庚側:好看煞,在顰兒必有之。\end{note}
\end{parag}


\begin{parag}
    鳳姐因在裏間屋裏看著人放桌子,\begin{note}庚側:且不接寶玉文字,妙!\end{note}聽如此說,便走來笑道:“寶兄弟不是撒謊,這倒是有的。上日薛大哥親自和我來尋珍珠,我問他作什麼,他說配藥。他還抱怨說,不配也罷了,如今那裏知道這麼費事。我問他什麼藥,他說是寶兄弟的方子,說了多少藥,我也沒工夫聽。他說不然我也買幾顆珍珠了,只是定要頭上帶過的,所以來和我尋。他說:‘妹妹就沒散的,花兒上也得,掐下來,過後兒我揀好的再給妹妹穿了來。’我沒法兒,把兩枝珠花兒現拆了給他。還要了一塊三尺上用大紅紗去,乳鉢乳了隔面子呢。”鳳姐說一句,那寶玉念一句佛,說:“太陽在屋子裏呢!”鳳姐說完了,寶玉又道:“太太想,這不過是將就呢。正經按那方子,這珍珠寶石定要在古墳裏的,有那古時富貴人家裝裹的頭面,拿了來纔好。如今那裏爲這個去刨墳掘墓,所以只是活人帶過的,也可以使得。”王夫人道:“阿彌陀佛,不當家花花的!就是墳裏有這個,人家死了幾百年,這會子翻屍盜骨的,作了藥也不靈!”\begin{note}甲側:不止阿鳳圓謊,今作者亦爲圓謊了,看此數句則知矣。\end{note}
\end{parag}


\begin{parag}
    寶玉向林黛玉說道:“你聽見了沒有,難道二姐姐也跟著我撒謊不成?”臉望著黛玉說話,卻拿眼睛瞟著寶釵。黛玉便拉王夫人道:“舅母聽聽,寶姐姐不替他圓謊,他支吾著我。”王夫人也道:“寶玉很會欺負你妹妹。”寶玉笑道:“太太不知道這原故。寶姐姐先在家裏住著,那薛大哥哥的事,他也不知道,何況如今在裏頭住著呢,自然是越發不知道了。\begin{note}庚側:分析得是,不敢正犯。\end{note}林妹妹纔在背後羞我,打諒我撒謊呢。”
\end{parag}


\begin{parag}
    正說著,只見賈母房裏的丫頭找寶玉林黛玉去喫飯。林黛玉也不叫寶玉,便起身拉了那丫頭就走。那丫頭說等著寶玉一塊兒走。林黛玉道:“他不喫飯了,咱們走。我先走了。”說著便出去了。寶玉道:“我今兒還跟著太太喫罷。”王夫人道:“罷,罷,我今兒喫齋,你正經喫你的去罷。”寶玉道:“我也跟著喫齋。”說著便叫那丫頭“去罷”,自己先跑到桌子上坐了。王夫人向寶釵等笑道:“你們只管喫你們的,由他去罷。”寶釵因笑道:“你正經去罷。喫不喫,陪著林姑娘走一趟,他心裏打緊的不自在呢。”寶玉道:“理他呢,過一會子就好了。”\begin{note}庚側:後文方知。\end{note}
\end{parag}


\begin{parag}
    一時喫過飯,寶玉一則怕賈母記掛,二則也記掛著林黛玉,忙忙的要茶漱口。探春惜春都笑道:“二哥哥,你成日家忙些什麼?\begin{note}甲側:冷眼人自然了了。\end{note}喫飯喫茶也是這麼忙碌碌的。”寶釵笑道:“你叫他快吃了瞧林妹妹去罷,叫他在這裏胡羼些什麼。”寶玉吃了茶,便出來,一直往西院來。可巧走到鳳姐兒院門前,只見鳳姐蹬著門檻子拿耳挖子剔牙,\begin{note}庚側:也才吃了飯。\end{note}看著十來個小廝們挪花盆呢。\begin{note}庚側:是阿鳳身段。\end{note}見寶玉來了,笑道:“你來的好。進來,進來,替我寫幾個字兒。”寶玉只得跟了進來。到了屋裏,鳳姐命人取過筆硯紙來,向寶玉道:“大紅妝緞四十匹,蟒緞四十匹,上用紗各色一百匹,金項圈四個。”寶玉道:“這算什麼?又不是帳,又不是禮物,怎麼個寫法?”鳳姐兒道:“你只管寫上,橫豎我自己明白就罷了。”\begin{note}庚側:有是語,有是事。\end{note}寶玉聽說只得寫了。鳳姐一面收起,一面笑道:“還有句話告訴你,不知你依不依?你屋裏有個丫頭叫紅玉,我合你說說,要叫了來使喚,總也沒說,今兒見你纔想起來。”\begin{note}甲側:字眼。\end{note}寶玉道:“我屋裏的人也多的很,姐姐喜歡誰,只管叫了來,何必問我。”\begin{note}甲側:紅玉接杯倒茶,自紗屜內覓至迴廊下,再見此處如些寫來,可知玉兄除顰外,俱是行雲流水。\end{note}鳳姐笑道:“既這麼著,我就叫人帶他去了。”\begin{note}甲側:又了卻怡紅一冤孽,一嘆!\end{note}寶玉道:“只管帶去。”說著便要走。\begin{note}甲側:忙極!\end{note}鳳姐兒道:“你回來,我還有一句話呢。”寶玉道:“老太太叫我呢,\begin{note}甲側:非也,林妹妹叫我呢。一笑。\end{note}有話等我回來罷。”說著便來至賈母這邊,只見都已喫完飯了。賈母因問他:“跟著你娘吃了什麼好的?”寶玉笑道:“也沒什麼好的,我倒多吃了一碗飯。”\begin{note}甲側:安慰祖母之心也。\end{note}因問:“林妹妹在那裏?”\begin{note}甲側:何如?餘言不謬。\end{note}賈母道:“裏頭屋裏呢。”
\end{parag}


\begin{parag}
    寶玉進來,只見地下一個丫頭吹熨斗,炕上兩個丫頭打粉線,黛玉彎著腰拿著剪子裁什麼呢。寶玉走進來笑道:“哦,這是作什麼呢?才吃了飯,這麼空著頭,一會子又頭疼了。”黛玉並不理,只管裁他的。有一個丫頭說道:“那塊綢子角兒還不好呢,再熨他一熨。”黛玉便把剪子一撂,說道:“理他呢,過一會子就好了。”\begin{note}甲側:有意無意,暗合針對,無怪玉兄納悶。\end{note}寶玉聽了,只是納悶。只見寶釵探春等也來了,和賈母說了一回話。寶釵也進來問:“林妹妹作什麼呢?”因見林黛玉裁剪,因笑道:“妹妹越發能幹了,連裁剪都會了。”黛玉笑道:“這也不過是撒謊哄人罷了。”寶釵笑道:“我告訴你個笑話兒,纔剛爲那個藥,我說了個不知道,寶兄弟心裏不受用了。”林黛玉道:“理他呢,過會子就好了。”\begin{note}甲眉:連重二次前言,是顰、寶氣味暗合,勿認做有小人過言也。\end{note}寶玉向寶釵道:“老太太要抹骨牌,正沒人呢,你抹骨牌去罷。”寶釵聽說,便笑道:“我是爲抹骨牌纔來了?”說著便走了。林黛玉道:“你倒是去罷,這裏有老虎,看吃了你!”說著又裁。寶玉見他不理,只得還陪笑說道:“你也出去逛逛再裁不遲。”林黛玉總不理。寶玉便問丫頭們:“這是誰叫裁的?”林黛玉見問丫頭們,便說道:“憑他誰叫我裁,也不管二爺的事!”寶玉方欲說話,只見有人進來回說“外頭有人請”。寶玉聽了,忙撤身出來。黛玉向外頭說道:\begin{note}甲側:仍丟不下,嘆嘆!\end{note}“阿彌陀佛!趕你回來,我死了也罷了。”\begin{note}甲側:何苦來?餘不忍聽。\end{note}
\end{parag}


\begin{parag}
    寶玉出來,到外面,只見焙茗說道:“馮大爺家請。”寶玉聽了,知道是昨日的話,便說:“要衣裳去。”自己便往書房裏來。焙茗一直到了二門前等人,\begin{note}甲側:此門請出玉兄來,故信步又至書房,文人弄墨,虛點綴也。\end{note}只見一個老婆子出來了,焙茗上去說道:“寶二爺在書房裏等出門的衣裳,你老人家進去帶個信兒。”那婆子說:“你媽的屄!\begin{note}庚側:活現活跳。\end{note}倒好,寶二爺如今在園子裏住著,\begin{note}甲側:與夜間叫人對看。\end{note}跟他的人都在園子裏,你又跑了這裏來帶信兒!”焙茗聽了,笑道:“罵的是,我也糊塗了。”說著一徑往東邊二門前來。可巧門上小廝在甬路底下踢球,焙茗將原故說了。小廝跑了進去,半日抱了一個包袱出來,遞與焙茗。回到書房裏,寶玉換了,命人備馬,只帶著焙茗、鋤藥、雙瑞、雙壽四個小廝去了。
\end{parag}


\begin{parag}
    一徑到了馮紫英家門口,有人報與了馮紫英,出來迎接進去。只見薛蟠早已在那裏久候,還有許多唱曲兒的小廝並唱小旦的蔣玉菡、錦香院的妓女雲兒。大家都見過了,然後喫茶。寶玉擎茶笑道:“前兒所言幸與不幸之事,我晝懸夜想,今日一聞呼喚即至。”馮紫英笑道:“你們令表兄弟倒都心實。前日不過是我的設辭,誠心請你們一飲,恐又推託,故說下這句話。\begin{note}甲眉:若真有一事,則不成《石頭記》文字矣。作者的三昧在茲,批書人得書中三昧亦在茲。壬午孟夏。\end{note}今日一邀即至,誰知都信真了。”說畢大家一笑,然後擺上酒來,依次坐定。馮紫英先命唱曲兒的小廝過來讓酒,然後命雲兒也來敬。
\end{parag}


\begin{parag}
    那薛蟠三杯下肚,不覺忘了情,拉著雲兒的手笑道:“你把那梯己新樣兒的曲子唱個我聽,我喫一罈如何?”雲兒聽說,只得拿起琵琶來,唱道:
\end{parag}


\begin{parag}
    兩個寃家,都難丟下,想著你來又記掛著他。兩個人形容俊俏,都難描畫。想昨宵幽期私訂在荼蘼架,一個偷情,一個尋拿,拿住了三曹對案,我也無回話。\begin{note}甲側:此唱一曲爲直刺寶玉。\end{note}
\end{parag}


\begin{parag}
    唱畢笑道:“你喝一罈子罷了。”薛蟠聽說,笑道:“不值一罈,再唱好的來。”
\end{parag}


\begin{parag}
    寶玉笑道:“聽我說來:如此濫飲,易醉而無味。我先喝一大海,\begin{note}庚眉:大海飲酒,西堂產九臺靈芝日也,批書至此,寧不悲乎?壬午重陽日。\end{note}發一新令,有不遵者,連罰十大海,逐出席外與人斟酒。”\begin{note}甲側:誰曾經過?嘆嘆!西堂故事。\end{note}馮紫英蔣玉菡等都道:“有理,有理。”寶玉拿起海來一氣飲幹,說道:“如今要說悲、愁、喜、樂四字,卻要說出女兒來,還要註明這四字原故。說完了,飲門杯。酒面要唱一個新鮮時樣曲子;酒底要席上生風一樣東西,或古詩、舊對、《四書》、《五經》、成語。”薛蟠未等說完,先站起來攔道:“我不來,別算我。\begin{note}甲側:爽人爽語。\end{note}這竟是捉弄我呢!”\begin{note}庚側:豈敢?\end{note}雲兒也站起來,推他坐下,笑道:“怕什麼?這還虧你天天喫酒呢,難道你連我也不如!我回來還說呢。說是了,罷;不是了,不過罰上幾杯,那裏就醉死了。你如今一亂令,倒喝十大海,下去斟酒不成?”\begin{note}庚側:有理。\end{note}衆人都拍手道妙。薛蟠聽說無法,只得坐了。聽寶玉說道:
\end{parag}


\begin{poem}
    \begin{pl} 女兒悲,青春已大守空閨。\end{pl}

    \begin{pl} 女兒愁,悔教夫婿覓封侯。\end{pl}

    \begin{pl} 女兒喜,對鏡晨妝顏色美。\end{pl}

    \begin{pl} 女兒樂,鞦韆架上春衫薄。\end{pl}
\end{poem}


\begin{parag}
    衆人聽了,都道:“說得有理。”薛蟠獨揚著臉搖頭說:“不好,該罰!”衆人問:“如何該罰?”薛蟠道:“他說的我通不懂,怎麼不該罰?”雲兒便擰他一把,笑道:“你悄悄的想你的罷。回來說不出,又該罰了。”於是拿琵琶聽寶玉唱道:
\end{parag}


\begin{poem}
    \begin{pl}滴不盡相思血淚拋紅豆,\end{pl}

    \begin{pl}睡不穩紗窗風雨黃昏後,\end{pl}

    \begin{pl}忘不了新愁與舊愁,\end{pl}

    \begin{pl}咽不下玉粒金蓴噎滿喉,\end{pl}

    \begin{pl}照不見菱花鏡裏形容瘦。\end{pl}

    \begin{pl}展不開的眉頭,挨不明的更漏。\end{pl}

    \begin{pl}呀!恰便似遮不住的青山隱隱,流不斷的綠水悠悠。\end{pl}
\end{poem}


\begin{parag}
    唱完,大家齊聲喝彩,獨薛蟠說無板。寶玉飲了門杯,便拈起一片梨來,說道:“雨打梨花深閉門。”完了令。
\end{parag}


\begin{parag}
    下該馮紫英,說道:
\end{parag}


\begin{poem}
    \begin{pl}女兒悲,兒夫染病在垂危。\end{pl}

    \begin{pl}女兒愁,大風吹倒梳妝樓。\end{pl}

    \begin{pl}女兒喜,頭胎養了雙生子。\end{pl}

    \begin{pl}女兒樂,私向花園掏蟋蟀。\end{pl}\begin{note}甲側:紫英口中應當如是。\end{note}
\end{poem}


\begin{parag}
    說畢,端起酒來,唱道:
\end{parag}


\begin{qute2sp}
    \begin{poem}
        \begin{pl}你是個可人,你是個多情,你是個刁鑽古怪鬼靈精,你是個神仙也不靈。我說的話兒你全不信,只叫你去背地裏細打聽,才知道我疼你不疼!\end{pl}
    \end{poem}
\end{qute2sp}


\begin{parag}
    唱完,飲了門杯,說道:“雞聲茅店月。”令完,下該雲兒。
\end{parag}


\begin{parag}
    雲兒便說道:“女兒悲,將來終身指靠誰?”\begin{note}甲側:道著了。\end{note}薛蟠嘆道:“我的兒,有你薛大爺在,你怕什麼!”衆人都道:“別混他,別混他!”雲兒又道:“女兒愁,媽媽打罵何時休!”薛蟠道:“前兒我見了你媽,還吩咐他不叫他打你呢。”衆人都道:“再多言者罰酒十杯。”薛蟠連忙自己打了一個嘴巴子,說道:“沒耳性,再不許說了。”雲兒又道:“女兒喜,情郎不捨還家裏。女兒樂,住了簫管弄絃索。”說完,便唱道:
\end{parag}

\begin{qute2sp}
    \begin{poem}
        \begin{pl}豆蔻開花三月三,一個蟲兒往裏鑽。鑽了半日不得進去,爬到花兒上打鞦韆。肉兒小心肝,我不開了你怎麼鑽?\end{pl}
        \begin{note}甲側:雙關,妙!\end{note}
    \end{poem}
\end{qute2sp}


\begin{parag}
    唱畢,飲了門杯,說道:“桃之夭夭。”令完了,下該薛蟠。
\end{parag}


\begin{parag}
    薛蟠道:“我可要說了:女兒悲──”說了半日,不見說底下的。馮紫英笑道:“悲什麼?快說來。”薛蟠登時急的眼睛鈴鐺一般,瞪了半日,才說道:“女兒悲──”又咳嗽了兩聲,\begin{note}甲側:受過此急者,大都不止呆兄一人耳。\end{note}說道:“女兒悲,嫁了個男人是烏龜。”衆人聽了都大笑起來。\begin{note}甲眉:此段與《金瓶梅》內西門慶、應伯爵在李桂姐家飲酒一回對看,未知孰家生動活潑?\end{note}薛蟠道:“笑什麼,難道我說的不是?一個女兒嫁了漢子,要當忘八,他怎麼不傷心呢?”衆人笑的彎腰說道:“你說的很是,快說底下的。”薛蟠瞪了一瞪眼,又說道:“女兒愁──”說了這句,又不言語了。衆人道:“怎麼愁?”薛蟠道:“繡房攛出個大馬猴。”衆人呵呵笑道:“該罰,該罰!這句更不通,先還可恕。”\begin{note}甲側:不愁,一笑。\end{note}說著便要篩酒。寶玉笑道:“押韻就好。”薛蟠道: “令官都準了,你們鬧什麼?”衆人聽說,方纔罷了。雲兒笑道:“下兩句越發難說了,我替你說罷。”薛蟠道:“胡說!當真我就沒好的了!聽我說罷:女兒喜,洞房花燭朝慵起。”衆人聽了,都詫異道:“這句何其太韻?”薛蟠又道:“女兒樂,一根往裏戳。”\begin{note}甲側:有前韻句,故有是句。\end{note}衆人聽了,都扭著臉說道:“該死,該死該死,該死!快唱了罷。”薛蟠便唱道:“一個蚊子哼哼哼。”衆人都怔了,說“這是個什麼曲兒?”薛蟠還唱道:“兩個蒼蠅嗡嗡嗡。”衆人都道:“罷,罷,罷!”薛蟠道:“愛聽不聽!這是新鮮曲兒,叫作哼哼韻。你們要懶待聽,邊酒底都免了,我就不唱。\begin{note}甲側:何嘗呆?\end{note}”衆人都道:“免了罷,免了罷,倒別耽誤了別人家。”
\end{parag}


\begin{parag}
    於是蔣玉菡說道:
\end{parag}


\begin{poem}
    \begin{pl}女兒悲,丈夫一去不回歸。\end{pl}

    \begin{pl}女兒愁,無錢去打桂花油。\end{pl}

    \begin{pl}女兒喜,燈花\end{pl}\begin{note}甲側:佳讖也。\end{note}\begin{pl}並頭結雙蕊。\end{pl}

    \begin{pl}女兒樂,夫唱婦隨真和合。\end{pl}

\end{poem}


\begin{parag}
    說畢,唱道:
\end{parag}

\begin{qute2sp}
    \begin{poem}
        \begin{pl}可喜你天生成百媚嬌,恰便似活神仙離碧霄。度青春,年正小;配鸞鳳,真也著。呀!看天河正高,聽譙樓鼓敲,剔銀燈同入鴛幃悄。\end{pl}
    \end{poem}
\end{qute2sp}


\begin{parag}
    唱畢,飲了門杯,笑道:“這詩詞上我倒有限。幸而昨日見了一副對子,可巧\begin{note}甲側:真巧!\end{note}只記得這句,幸而席上還有這件東西。”\begin{note}甲側:瞞過衆人。\end{note}說畢,便幹了酒,拿起一朵木樨來,念道:“花氣襲人知晝暖。”
\end{parag}


\begin{parag}
    衆人倒都依了,完令。薛蟠又跳了起來,喧嚷道:“了不得,了不得!該罰,該罰!這席上又沒有寶貝,\begin{note}甲側:奇談。\end{note}你怎麼念起寶貝來?”蔣玉菡怔了,說道:“何曾有寶貝?”薛蟠道:“你還賴呢!你再念來。”蔣玉菡只得又唸了一遍。薛蟠道:“襲人可不是寶貝是什麼!你們不信,只問他。”說畢,指著寶玉。寶玉沒好意思起來,說:“薛大哥,你該罰多少?”薛蟠道:“該罰,該罰!”說著拿起酒來,一飲而盡。馮紫英與蔣玉菡等不知原故,雲兒便告訴了出來。\begin{note}甲側:用雲兒細說,的是章法。\end{note}\begin{note}庚眉:雲兒知怡紅細事,可想玉兄之風情月意也。壬午重陽。\end{note}蔣玉菡忙起身陪罪。衆人都道:“不知者不作罪。”
\end{parag}


\begin{parag}
    少刻,寶玉出席解手,蔣玉菡便隨了出來。二人站在廊檐下,蔣玉菡又陪不是。寶玉見他嫵媚溫柔,心中十分留戀,便緊緊的搭著他的手,叫他:“閒了往我們那裏去。還有一句話借問,也是你們貴班中,有一個叫琪官的,他在那裏?如今名馳天下,我獨無緣一見。”蔣玉菡笑道:“就是我的小名兒。”寶玉聽說,不覺欣然跌足笑道:“有幸,有幸!果然名不虛傳。今兒初會,便怎麼樣呢?”想了一想,向袖中取出扇子,將一個玉訣扇墜解下來,遞與琪官,道:“微物不堪,略表今日之誼。”琪官接了,笑道:“無功受祿,何以克當!也罷,我這裏得了一件奇物,今日早起方繫上,還是簇新的,聊可表我一點親熱之意。”說畢撩衣,將系小衣兒一條大紅汗巾子解了下來,遞與寶玉,道:“這汗巾子是茜香國女國王所貢之物,夏天系著,肌膚生香,不生汗漬。昨日北靜王給我的,今日才上身。若是別人,我斷不肯相贈。二爺請把自己系的解下來,給我係著。”寶玉聽說,喜不自禁,連忙接了,將自己一條松花汗巾解了下來,遞與琪官。\begin{note}甲側:紅綠牽巾是這樣用法。一笑。\end{note}二人方束好,只見一聲大叫:“我可拿住了!”只見薛蟠跳了出來,拉著二人道:“放著酒不喫,兩個人逃席出來幹什麼?快拿出來我瞧瞧。”二人都道:“沒有什麼。”薛蟠那裏肯依,還是馮紫英出來才解開了。於是復又歸坐飲酒,至晚方散。
\end{parag}


\begin{parag}
    寶玉回至園中,寬衣喫茶。襲人見扇子上的墜兒沒了,便問他:“往那裏去了?”寶玉道:“馬上丟了。”\begin{note}庚側:隨口謊言。\end{note}睡覺時只見腰裏一條血點似的大紅汗巾子,襲人便猜了八九分,因說道:“你有了好的繫褲子,把我那條還我罷。”寶玉聽說,方想起那條汗巾子原是襲人的,不該給人才是,心裏後悔,口裏說不出來,只得笑道:“我賠你一條罷。”襲人聽了,點頭嘆道:“我就知道又幹這些事!也不該拿著我的東西給那起混帳人去。也難爲你,心裏沒個算計兒。” 再要說幾句,又恐慪上他的酒來,少不得也睡了,一宿無話。
\end{parag}


\begin{parag}
    至次日天明,方纔醒了,只見寶玉笑道:“夜裏失了盜也不曉得,你瞧瞧褲子上。”襲人低頭一看,只見昨日寶玉系的那條汗巾子系在自己腰裏呢,便知是寶玉夜間換了,忙一頓把解下來,說道:“我不希罕這行子,趁早兒拿了去!”寶玉見他如此,只得委婉解勸了一回。襲人無法,只得系在腰裏。過後寶玉出去,終久解下來擲在個空箱子裏,自己又換了一條系著。
\end{parag}


\begin{parag}
    寶玉並未理論,因問起昨日可有什麼事情。襲人便回說:“二奶奶打發人叫了紅玉去了。他原要等你來的,我想什麼要緊,我就作了主,打發他去了。”寶玉道:“很是。我已知道了,不必等我罷了。”襲人又道:“昨兒貴妃打發夏太監出來,送了一百二十兩銀子,叫在清虛觀初一到初三打三天平安醮,唱戲獻供,叫珍大爺領著衆位爺們跪香拜佛呢。還有端午兒的節禮也賞了。”說著命小丫頭子來,將昨日所賜之物取了出來,只見上等宮扇兩柄,紅麝香珠二串,鳳尾羅二端,芙蓉簟一領。寶玉見了,喜不自勝,問“別人的也都是這個?”襲人道:“老太太的多著一個香如意,一個瑪瑙枕。太太、老爺、姨太太的只多著一個如意。你的同寶姑娘的一樣。\begin{note}甲側:金姑玉郎是這樣寫法。\end{note}林姑娘同二姑娘、三姑娘、四姑娘只單有扇子同數珠兒,別人都沒了。大奶奶、二奶奶他兩個是每人兩匹紗,兩匹羅,兩個香袋,兩個錠子藥。”寶玉聽了,笑道:“這是怎麼個原故?怎麼林姑娘的倒不同我的一樣,倒是寶姐姐的同我一樣!別是傳錯了罷?”襲人道:“昨兒拿出來,都是一份一份的寫著籤子,怎麼就錯了!你的是在老太太屋裏的,我去拿了來了。老太太說了,明兒叫你一個五更天進去謝恩呢。”寶玉道:“自然要走一趟。”說著便叫紫綃來:“拿了這個到林姑娘那裏去,就說是昨兒我得的,愛什麼留下什麼。”紫綃答應了,拿了去,不一時回來說:“林姑娘說了,昨兒也得了,二爺留著罷。”
\end{parag}


\begin{parag}
    寶玉聽說,便命人收了。剛洗了臉出來,要往賈母那裏請安去,只見林黛玉頂頭來了。寶玉趕上去笑道:“我的東西叫你揀,你怎麼不揀?”林黛玉昨日所惱寶玉的心事早又丟開,又顧今日的事了,因說道:“我沒這麼大福禁受,比不得寶姑娘,什麼金什麼玉的,我們不過是草木之人!”\begin{note}甲側:自道本是絳珠草也。\end{note}寶玉聽他提出“金玉”二字來,不覺心動疑猜,便說道:“除了別人說什麼金什麼玉,我心裏要有這個想頭,天誅地滅,萬世不得人身!”林黛玉聽他這話,便知他心裏動了疑,忙又笑道:“好沒意思,白白的說什麼誓?管你什麼金什麼玉的呢!”寶玉道:“我心裏的事也難對你說,日後自然明白。除了老太太、老爺、太太這三個人,第四個就是妹妹了。要有第五個人,我也說個誓。”林黛玉道:“你也不用說誓,我很知道你心裏有‘妹妹’,但只是見了‘姐姐’,就把‘妹妹’ 忘了。” 寶玉道:“那是你多心,我再不的。”林黛玉道:“昨兒寶丫頭不替你圓謊,爲什麼問著我呢?那要是我,你又不知怎麼樣了。”
\end{parag}


\begin{parag}
    正說著,只見寶釵從那邊來了,二人便走開了。寶釵分明看見,只裝看不見,低著頭過去了,到了王夫人那裏,坐了一回,然後到了賈母這邊,只見寶玉在這裏呢。\begin{note}甲側:寶釵往王夫人處去,故寶玉先在賈母處,一絲不亂。\end{note}薛寶釵因往日母親對王夫人等曾提過“金鎖是個和尚給的,等日後有玉的方可結爲婚姻”等語,\begin{note}甲側:此處表明以後二寶文章,宜換眼看。\end{note}所以總遠著寶玉。\begin{note}甲眉:峯巒全露,又用煙雲截斷,好文字。\end{note}昨兒見元春所賜的東西,獨他與寶玉一樣,心裏越發沒意思起來。幸虧寶玉被一個林黛玉纏綿住了,心心念念只記掛著林黛玉,並不理論這事。此刻忽見寶玉笑問道:“寶姐姐,我瞧瞧你的紅麝串子?”可巧寶釵左腕上籠著一串,見寶玉問他,少不得褪了下來。寶釵生的肌膚豐澤,容易褪不下來。寶玉在旁看著雪白一段酥臂,不覺動了羨慕之心,暗暗想道: “這個膀子要長在林妹妹身上,或者還得摸一摸,偏生長在他身上。”正是恨沒福得摸,忽然想起“金玉”一事來,再看看寶釵形容,只見臉若銀盆,眼似水杏,脣不點而紅,眉不畫而翠,\begin{note}甲側:太白所謂“清水出芙蓉”。\end{note}比林黛玉另具一種嫵媚風流,不覺就呆了,\begin{note}甲側:忘情,非呆也。\end{note}寶釵褪了串子來遞與他也忘了接。寶釵見他怔了,自己倒不好意思的,丟下串子,回身纔要走,只見林黛玉蹬著門檻子,嘴裏咬著手帕子笑呢。寶釵道:“你又禁不得風吹,怎麼又站在那風口裏?”林黛玉笑道:“何曾不是在屋裏的。只因聽見天上一聲叫喚,出來瞧了瞧,原來是個呆雁。”薛寶釵道:“呆雁在那裏呢?我也瞧一瞧。”林黛玉道: “我纔出來,他就‘忒兒’一聲飛了。”口裏說著,將手裏的帕子一甩,向寶玉臉上甩來。寶玉不防,正打在眼上,“噯喲”了一聲。要知端的,且聽下回分解。
\end{parag}


\begin{parag}
    \begin{note}甲:茜香羅、紅麝串寫於一回,蓋琪官雖系優人,後回與襲人供奉玉兄寶卿得同終始者,非泛泛之文也。自“聞曲”回以後,回回寫藥方,是白描顰兒添病也。前“玉生香”回中顰雲“他有金你有玉;他有冷香你豈不該有暖香?”是寶玉無藥可配矣。今顰兒之劑若許材料皆系滋補熱性之藥,兼有許多奇物,而尚未擬名,何不竟以“暖香”名之?以代補寶玉之不足,豈不三人一體矣。寶玉忘情,露於寶釵,是後回累累忘情之引。茜香羅暗繫於襲人腰中,系伏線之文。\end{note}
\end{parag}


\begin{parag}
    \begin{note}蒙回後總評:世間最苦是疑情,不遇知音休應聲。盟誓已成了,莫遲誤今生。\end{note}
\end{parag}
