\chap{九十八}{苦绛珠魂归离恨天 病神瑛泪洒相思地}



\begin{parag}
    话说宝玉见了贾政,回至房中,更觉头昏脑闷,懒待动弹,连饭也没吃,便昏沉睡去。仍旧延医诊治,服药不效,索性连人也认不明白了。大家扶着他坐起来,还是象个好人。一连闹了几天,那日恰是回九之期,若不过去,薛姨妈脸上过不去,若说去呢,宝玉这般光景。贾母明知是为黛玉而起,欲要告诉明白,又恐气急生变。宝钗是新媳妇,又难劝慰,必得姨妈过来才好。若不回九,姨妈嗔怪。便与王夫人凤姐商议道:“我看宝玉竟是魂不守舍,起动是不怕的。用两乘小轿叫人扶着从园里过去,应了回九的吉期,以后请姨妈过来安慰宝钗,咱们一心一意的调治宝玉,可不两全?”王夫人答应了,即刻预备。幸亏宝钗是新媳妇,宝玉是个疯傻的,由人掇弄过去了。宝钗也明知其事,心里只怨母亲办得糊涂,事已至此,不肯多言。独有薛姨妈看见宝玉这般光景,心里懊悔,只得草草完事。
\end{parag}


\begin{parag}
    到家,宝玉越加沉重,次日连起坐都不能了。日重一日,甚至汤水不进。薛姨妈等忙了手脚,各处遍请名医,皆不识病源。只有城外破寺中住着个穷医,姓毕,别号知庵的,诊得病源是悲喜激射,冷暖失调,饮食失时,忧忿滞中,正气壅闭:此内伤外感之症。于是度量用药,至晚服了,二更后果然省些人事,便要水喝。贾母王夫人等才放了心,请了薛姨妈带了宝钗都到贾母那里暂且歇息。
\end{parag}


\begin{parag}
    宝玉片时清楚,自料难保,见诸人散后,房中只有袭人,因唤袭人至跟前,拉着手哭道:“我问你,宝姐姐怎么来的?我记得老爷给我娶了林妹妹过来,怎么被宝姐姐赶了去了?他为什么霸占住在这里?我要说呢,又恐怕得罪了他。你们听见林妹妹哭得怎么样了?”袭人不敢明说,只得说道:“林姑娘病着呢。”宝玉又道:“我瞧瞧他去。”说着,要起来。岂知连日饮食不进,身子那能动转,便哭道:“我要死了!我有一句心里的话,只求你回明老太太:横竖林妹妹也是要死的,我如今也不能保。两处两个病人都要死的,死了越发难张罗。不如腾一处空房子,趁早将我同林妹妹两个抬在那里,活着也好一处医治伏侍,死了也好一处停放。你依我这话,不枉了几年的情分。”袭人听了这些话,便哭的哽嗓气噎。宝钗恰好同了莺儿过来,也听见了,便说道:“你放着病不保养,何苦说这些不吉利的话。老太太才安慰了些,你又生出事来。老太太一生疼你一个,如今八十多岁的人了,虽不图你的封诰,将来你成了人,老太太也看着乐一天,也不枉了老人家的苦心。太太更是不必说了,一生的心血精神,抚养了你这一个儿子,若是半途死了,太太将来怎么样呢。我虽是命薄,也不至于此。据此三件看来,你便要死,那天也不容你死的,所以你是不得死的。只管安稳着,养个四五天后,风邪散了,太和正气一足,自然这些邪病都没有了。”宝玉听了,竟是无言可答,半晌方才嘻嘻的笑道:“你是好些时不和我说话了,这会子说这些大道理的话给谁听?”宝钗听了这话,便又说道:“实告诉你说罢,那两日你不知人事的时候,林妹妹已经亡故了。”宝玉忽然坐起来,大声诧异道:“果真死了吗?”宝钗道:“果真死了。岂有红口白舌咒人死的呢。老太太,太太知道你姐妹和睦,你听见他死了自然你也要死,所以不肯告诉你。”宝玉听了,不禁放声大哭,倒在床上。
\end{parag}


\begin{parag}
    忽然眼前漆黑,辨不出方向,心中正自恍惚,只见眼前好象有人走来,宝玉茫然问道:“借问此是何处?”那人道:“此阴司泉路。你寿未终,何故至此?”宝玉道:“适闻有一故人已死,遂寻访至此,不觉迷途。”那人道:“故人是谁?”宝玉道:“姑苏林黛玉。”那人冷笑道:“林黛玉生不同人,死不同鬼,无魂无魄,何处寻访!凡人魂魄,聚而成形,散而为气,生前聚之,死则散焉。常人尚无可寻访,何况林黛玉呢。汝快回去罢。”宝玉听了,呆了半晌道:“既云死者散也,又如何有这个阴司呢?”那人冷笑道:“那阴司说有便有,说无就无。皆为世俗溺于生死之说,设言以警世,便道上天深怒愚人,或不守分安常,或生禄未终自行夭折,或嗜淫欲尚气逞凶无故自陨者,特设此地狱,囚其魂魄,受无边的苦,以偿生前之罪。汝寻黛玉,是无故自陷也。且黛玉已归太虚幻境,汝若有心寻访,潜心修养,自然有时相见。如不安生,即以自行夭折之罪囚禁阴司,除父母外,欲图一见黛玉,终不能矣。”那人说毕,袖中取出一石,向宝玉心口掷来。宝玉听了这话,又被这石子打着心窝,吓的即欲回家,只恨迷了道路。正在踌躇,忽听那边有人唤他。回首看时,不是别人,正是贾母,王夫人,宝钗,袭人等围绕哭泣叫着。自己仍旧躺在床上。见案上红灯,窗前皓月,依然锦锈丛中,繁华世界。定神一想,原来竟是一场大梦。浑身冷汗,觉得心内清爽。仔细一想,真正无可奈何,不过长叹数声而已。宝钗早知黛玉已死,因贾母等不许众人告诉宝玉知道,恐添病难治。自己却深知宝玉之病实因黛玉而起,失玉次之,故趁势说明,使其一痛决绝,神魂归一,庶可疗治。贾母王夫人等不知宝钗的用意,深怪他造次。后来见宝玉醒了过来,方才放心。立即到外书房请了毕大夫进来诊视。那大夫进来诊了脉,便道:“奇怪,这回脉气沉静,神安郁散,明日进调理的药,就可以望好了。”说着出去。众人各自安心散去。
\end{parag}


\begin{parag}
    袭人起初深怨宝钗不该告诉,惟是口中不好说出。莺儿背地也说宝钗道:“姑娘忒性急了。”宝钗道:“你知道什么!好歹横竖有我呢。”那宝钗任人诽谤,并不介意,只窥察宝玉心病,暗下针砭。一日,宝玉渐觉神志安定,虽一时想起黛玉,尚有糊涂。更有袭人缓缓的将“老爷选定的宝姑娘为人和厚,嫌林姑娘秉性古怪,原恐早夭,老太太恐你不知好歹,病中着急,所以叫雪雁过来哄你”的话时常劝解。宝玉终是心酸落泪。欲待寻死,又想着梦中之言,又恐老太太,太太生气,又不能撩开。又想黛玉已死,宝钗又是第一等人物,方信金石姻缘有定,自己也解了好些。宝钗看来不妨大事,于是自己心也安了,只在贾母王夫人等前尽行过家庭之礼后,便设法以释宝玉之忧。宝玉虽不能时常坐起,亦常见宝钗坐在床前,禁不住生来旧病。宝钗每以正言劝解,以”养身要紧,你我既为夫妇,岂在一时”之语安慰他。那宝玉心里虽不顺遂,无奈日里贾母王夫人及薛姨妈等轮流相伴,夜间宝钗独去安寝,贾母又派人服侍,只得安心静养。又见宝钗举动温柔,也就渐渐的将爱慕黛玉的心肠略移在宝钗身上,此是后话。
\end{parag}


\begin{parag}
    却说宝玉成家的那一日,黛玉白日已昏晕过去,却心头口中一丝微气不断,把个李纨和紫鹃哭的死去活来。到了晚间,黛玉去又缓过来了,微微睁开眼,似有要水要汤的光景。此时雪雁已去,只有紫鹃和李纨在旁。紫鹃便端了一盏桂圆汤和的梨汁,用小银匙灌了两三匙。黛玉闭着眼静养了一会子,觉得心里似明似暗的。此时李纨见黛玉略缓,明知是回光反照的光景,却料着还有一半天耐头,自己回到稻香村料理了一回事情。
\end{parag}


\begin{parag}
    这里黛玉睁开眼一看,只有紫鹃和奶妈并几个小丫头在那里,便一手攥了紫鹃的手,使着劲说道:“我是不中用的人了。你伏侍我几年,我原指望咱们两个总在一处。不想我。……”说着,又喘了一会子,闭了眼歇着。紫鹃见他攥着不肯松手,自己也不敢挪动,看他的光景比早半天好些,只当还可以回转,听了这话,又寒了半截。半天,黛玉又说道:“妹妹,我这里并没亲人。我的身子是干净的,你好歹叫他们送我回去。”说到这里又闭了眼不言语了。那手却渐渐紧了,喘成一处,只是出气大入气小,已经促疾的很了。
\end{parag}


\begin{parag}
    紫鹃忙了,连忙叫人请李纨,可巧探春来了。紫鹃见了,忙悄悄的说道:“三姑娘,瞧瞧林姑娘罢。”说着,泪如雨下。探春过来,摸了摸黛玉的手已经凉了,连目光也都散了。探春紫鹃正哭着叫人端水来给黛玉擦洗,李纨赶忙进来了。三个人才见了,不及说话。刚擦着,猛听黛玉直声叫道:“宝玉,宝玉,你好……”说到“好”字,便浑身冷汗,不作声了。紫鹃等急忙扶住,那汗愈出,身子便渐渐的冷了。探春李纨叫人乱着拢头穿衣,只见黛玉两眼一翻,呜呼,香魂一缕随风散,愁绪三更入梦遥!
\end{parag}


\begin{parag}
    当时黛玉气绝,正是宝玉娶宝钗的这个时辰。紫鹃等都大哭起来。李纨探春想他素日的可疼,今日更加可怜,也便伤心痛哭。因潇湘馆离新房子甚远,所以那边并没听见。一时大家痛哭了一阵,只听得远远一阵音乐之声,侧耳一听,却又没有了。探春李纨走出院外再听时,惟有竹梢风动,月影移墙,好不凄凉冷淡!一时叫了林之孝家的过来,将黛玉停放毕,派人看守,等明早去回凤姐。
\end{parag}


\begin{parag}
    凤姐因见贾母王夫人等忙乱,贾政起身,又为宝玉惛愦更甚,正在着急异常之时,若是又将黛玉的凶信一回,恐贾母王夫人愁苦交加,急出病来,只得亲自到园。到了潇湘馆内,也不免哭了一场。见了李纨探春,知道诸事齐备,便说:“很好。只是刚才你们为什么不言语,叫我着急?”探春道:“刚才送老爷,怎么说呢。”凤姐道:“还倒是你们两个可怜他些。这么着,我还得那边去招呼那个冤家呢。但是这件事好累坠,若是今日不回,使不得,若回了,恐怕老太太搁不住。”李纨道:“你去见机行事,得回再回方好。”凤姐点头,忙忙的去了。
\end{parag}


\begin{parag}
    凤姐到了宝玉那里,听见大夫说不妨事,贾母王夫人略觉放心,凤姐便背了宝玉,缓缓的将黛玉的事回明了。贾母王夫人听得都唬了一大跳。贾母眼泪交流说道:“是我弄坏了他了。但只是这个丫头也忒傻气!”说着,便要到园里去哭他一场,又惦记着宝玉,两头难顾。王夫人等含悲共劝贾母不必过去,“老太太身子要紧。”贾母无奈,只得叫王夫人自去。又说:“你替我告诉他的阴灵:‘并不是我忍心不来送你,只为有个亲疏。你是我的外孙女儿,是亲的了,若与宝玉比起来,可是宝玉比你更亲些。倘宝玉有些不好,我怎么见他父亲呢。’”说着,又哭起来。王夫人劝道:“林姑娘是老太太最疼的,但只寿夭有定。如今已经死了,无可尽心,只是葬礼上要上等的发送。一则可以少尽咱们的心,二则就是姑太太和外甥女儿的阴灵儿,也可以少安了。”贾母听到这里,越发痛哭起来。凤姐恐怕老人家伤感太过,明仗着宝玉心中不甚明白,便偷偷的使人来撒个谎儿哄老太太道:“宝玉那里找老太太呢。”贾母听见,才止住泪问道:“不是又有什么缘故?”凤姐陪笑道:“没什么缘故,他大约是想老太太的意思。”贾母连忙扶了珍珠儿,凤姐也跟着过来。
\end{parag}


\begin{parag}
    走至半路,正遇王夫人过来,一一回明了贾母。贾母自然又是哀痛的,只因要到宝玉那边,只得忍泪含悲的说道:“既这么着,我也不过去了。由你们办罢,我看着心里也难受,只别委屈了他就是了。”王夫人凤姐一一答应了。贾母才过宝玉这边来,见了宝玉,因问:“你做什么找我?”宝玉笑道:“我昨日晚上看见林妹妹来了,他说要回南去。我想没人留的住,还得老太太给我留一留他。”贾母听着,说:“使得,只管放心罢。”袭人因扶宝玉躺下。
\end{parag}


\begin{parag}
    贾母出来到宝钗这边来。那时宝钗尚未回九,所以每每见了人倒有些含羞之意。这一天见贾母满面泪痕,递了茶,贾母叫他坐下。宝钗侧身陪着坐了,才问道:“听得林妹妹病了,不知他可好些了?”贾母听了这话,那眼泪止不住流下来,因说道:“我的儿,我告诉你,你可别告诉宝玉。都是因你林妹妹,才叫你受了多少委屈。你如今作媳妇了,我才告诉你。这如今你林妹妹没了两三天了,就是娶你的那个时辰死的。如今宝玉这一番病还是为着这个,你们先都在园子里,自然也都是明白的。”宝钗把脸飞红了,想到黛玉之死,又不免落下泪来。贾母又说了一回话去了。自此宝钗千回万转,想了一个主意,只不肯造次,所以过了回九才想出这个法子来。如今果然好些,然后大家说话才不至似前留神。独是宝玉虽然病势一天好似一天,他的痴心总不能解,必要亲去哭他一场。贾母等知他病未除根,不许他胡思乱想,怎奈他郁闷难堪,病多反复。倒是大夫看出心病,索性叫他开散了,再用药调理,倒可好得快些。宝玉听说,立刻要往潇湘馆来。贾母等只得叫人抬了竹椅子过来,扶宝玉坐上。贾母王夫人即便先行。到了潇湘馆内,一见黛玉灵柩,贾母已哭得泪干气绝。凤姐等再三劝住。王夫人也哭了一场。李纨便请贾母王夫人在里间歇着,犹自落泪。
\end{parag}


\begin{parag}
    宝玉一到,想起未病之先来到这里,今日屋在人亡,不禁嚎啕大哭。想起从前何等亲密,今日死别,怎不更加伤感。众人原恐宝玉病后过哀,都来解劝,宝玉已经哭得死去活来,大家搀扶歇息。其余随来的,如宝钗,俱极痛哭。独是宝玉必要叫紫鹃来见,问明姑娘临死有何话说。紫鹃本来深恨宝玉,见如此,心里已回过来些,又见贾母王夫人都在这里,不敢洒落宝玉,便将林姑娘怎么复病,怎么烧毁帕子,焚化诗稿,并将临死说的话,一一的都告诉了。宝玉又哭得气噎喉干。探春趁便又将黛玉临终嘱咐带柩回南的话也说了一遍。贾母王夫人又哭起来。多亏凤姐能言劝慰,略略止些,便请贾母等回去。宝玉那里肯舍,无奈贾母逼着,只得勉强回房。
\end{parag}


\begin{parag}
    贾母有了年纪的人,打从宝玉病起,日夜不宁,今又大痛一阵,已觉头晕身热。虽是不放心惦着宝玉,却也挣扎不住,回到自己房中睡下。王夫人更加心痛难禁,也便回去,派了彩云帮着袭人照应,并说:“宝玉若再悲戚,速来告诉我们。”宝钗是知宝玉一时必不能舍,也不相劝,只用讽刺的话说他。宝玉倒恐宝钗多心,也便饮泣收心。歇了一夜,倒也安稳。明日一早,众人都来瞧他,但觉气虚身弱,心病倒觉去了几分。于是加意调养,渐渐的好起来。贾母幸不成病,惟是王夫人心痛未痊。那日薛姨妈过来探望,看见宝玉精神略好,也就放心,暂且住下。
\end{parag}


\begin{parag}
    一日,贾母特请薛姨妈过去商量说:“宝玉的命都亏姨太太救的,如今想来不妨了,独委屈了你的姑娘。如今宝玉调养百日,身体复旧,又过了姑娘的功服,正好圆房。要求姨太太作主,另择个上好的吉日。”薛姨妈便道:“老太太主意很好,何必问我。宝丫头虽生的粗笨,心里却还是极明白的。他的性情老太太素日是知道的。但愿他们两口儿言和意顺,从此老太太也省好些心,我姐姐也安慰些,我也放了心了。老太太便定个日子。还通知亲戚不用呢?”贾母道:“宝玉和你们姑娘生来第一件大事,况且费了多少周折,如今才得安逸,必要大家热闹几天。亲戚都要请的。一来酬愿,二则咱们吃杯喜酒,也不枉我老人家操了好些心。”薛姨妈听说,自然也是喜欢的,便将要办妆奁的话也说了一番。贾母道:“咱们亲上做亲,我想也不必这些。若说动用的,他屋里已经满了。必定宝丫头他心爱的要你几件,姨太太就拿了来。我看宝丫头也不是多心的人,不比的我那外孙女儿的脾气,所以他不得长寿。”说着,连薛姨妈也便落泪。恰好凤姐进来,笑道:“老太太姑妈又想着什么了?”薛姨妈道:“我和老太太说起你林妹妹来,所以伤心。”凤姐笑道:“老太太和姑妈且别伤心,我刚才听了个笑话儿来了,意思说给老太太和姑妈听。”贾母拭了拭眼泪,微笑道:“你又不知要编派谁呢,你说来我和姨太太听听。说不笑我们可不依。”只见那凤姐未从张口,先用两只手比着,笑弯了腰了。未知他说出些什么来,下回分解。
\end{parag}