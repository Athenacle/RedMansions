\chap{一百零六}{王熙凤致祸抱羞惭 贾太君祷天消祸患}



\begin{parag}
    话说贾政闻知贾母危急,即忙进去看视。见贾母惊吓气逆,王夫人鸳鸯等唤醒回来,即用疏气安神的丸药服了,渐渐的好些,只是伤心落泪。贾政在旁劝慰,总说是“儿子们不肖,招了祸来累老太太受惊。若老太太宽慰些,儿子们尚可在外料理;若是老太太有什么不自在,儿子们的罪孽更重了。”贾母道:“我活了八十多岁,自作女孩儿起到你父亲手里,都托着祖宗的福,从没有听见过那些事。如今到老了,见你们倘或受罪,叫我心里过得去么!倒不如合上眼随你们去罢了。”说着,又哭。
\end{parag}


\begin{parag}
    贾政此时着急异常,又听外面说:“请老爷,内廷有信。”贾政急忙出来,见是北静王府长史,一见面便说“大喜。”贾政谢了,请长史坐下,“请问王爷有何谕旨?”那长史道:“我们王爷同西平郡王进内复奏,将大人的惧怕的心,感激天恩之话都代奏了。主上甚是悯恤,并念及贵妃溘逝未久,不忍加罪,着加恩仍在工部员外上行走。所封家产,惟将贾赦的入官,余俱给还。并传旨令尽心供职。惟抄出借券令我们王爷查核,如有违禁重利的一概照例入官,其在定例生息的同房地文书尽行给还。贾琏着革去职衔,免罪释放。”贾政听毕即起身叩谢天恩,又拜谢王爷恩典。“先请长史大人代为禀谢,明晨到阙谢恩,并到府里磕头。”那长史去了。少停,传出旨来。承办官遵旨一一查清,入官者入官,给还者给还,将贾琏放出,所有贾赦名下男妇人等造册入官。
\end{parag}


\begin{parag}
    可怜贾琏屋内东西除将按例放出的文书发给外,其余虽未尽入官的,早被查抄的人尽行抢去,所存者只有家伙对象。贾琏始则惧罪,后蒙释放已是大幸,及想起历年积聚的东西并凤姐的体己不下七八万金,一朝而尽,怎得不痛。且他父亲现禁在锦衣府,凤姐病在垂危,一时悲痛。又见贾政含泪叫他,问道:“我因官事在身,不大理家,故叫你们夫妇总理家事。你父亲所为固难劝谏,那重利盘剥究竟是谁干的?况且非咱们这样人家所为。如今入了官,在银钱是不打紧的,这种声名出去还了得吗!”贾琏跪下说道:“侄儿办家事,并不敢存一点私心。所有出入的账目,自有赖大,吴新登,戴良等登记,老爷只管叫他们来查问。现在这几年,库内的银子出多入少,虽没贴补在内,已在各处做了好些空头,求老爷问太太就知道了。这些放出去的帐,连侄儿也不知道那里的银子,要问周瑞旺儿才知道。”贾政道:“据你说来,连你自己屋里的事还不知道,那些家中上下的事更不知道了。我这回也不来查问你,现今你无事的人,你父亲的事和你珍大哥的事还不快去打听打听。”贾琏一心委屈,含着眼泪答应了出去。贾政叹气连连的想道:“我祖父勤劳王事,立下功勋,得了两个世职,如今两房犯事都革去了。我瞧这些子侄没一个长进的。老天啊,老天啊!我贾家何至一败如此!我虽蒙圣恩格外垂慈,给还家产,那两处食用自应归并一处,叫我一人那里支撑的住。方才琏儿所说更加诧异,说不但库上无银,而且尚有亏空,这几年竟是虚名在外。只恨我自己为什么糊涂若此。倘或我珠儿在世,尚有膀臂,宝玉虽大,更是无用之物。”想到那里,不觉泪满衣襟。又想:“老太太偌大年纪,儿子们并没有自能奉养一日,反累他吓得死去活来。种种罪孽,叫我委之何人!”正在独自悲切,只见家人禀报各亲友进来看候。贾政一一道谢,说起:“家门不幸,是我不能管教子侄,所以至此。”有的说:“我久知令兄赦大老爷行事不妥,那边珍哥更加骄纵。若说因官事错误得个不是,于心无愧,如今自己闹出的,倒带累了二老爷。”有的说:“人家闹的也多,也没见御史参奏,不是珍老大得罪朋友,何至如此。”有的说:“也不怪御史,我们听见说是府上的家人同几个泥腿在外头哄嚷出来的。御史恐参奏不实,所以诓了这里的人去才说出来的。我想府上待下人最宽的,为什么还有这事。”有的说:“大凡奴才们是一个养活不得的。今儿在这里都是好亲友我才敢说,就是尊驾在外任,我保不得——你是不爱钱的,——那外头的风声也不好,都是奴才们闹的。你该堤防些。如今虽说没有动你的家,倘或再遇着主上疑心起来,好些不便呢。”贾政听说,心下着忙道:“众位听见我的风声怎样?”众人道:“我们虽没听见实据,只闻外面人说你在粮道任上怎么叫门上家人要钱。”贾政听了,便说道:“我是对得天的,从不敢起这要钱的念头。只是奴才在外招摇撞骗,闹出事来我就吃不住了。”众人道:“如今怕也无益,只好将现在的管家们都严严的查一查,若有抗主的奴才,查出来严严的办一办。”贾政听了点头。便见门上进来回禀说:“孙姑爷那边打发人来说,自己有事不能来,着人来瞧瞧。说大老爷该他一种银子,要在二老爷身上还的。”贾政心内忧闷,只说:“知道了。”众人都冷笑道:“人说令亲孙绍祖混账,真有些。如今丈人抄了家,不但不来瞧看帮补照应,倒赶忙的来要银子,真真不在理上。”贾政道:“如今且不必说他。那头亲事原是家兄配错的,我的侄女儿的罪已经受够了,如今又招我来。”正说着,只见薛蝌进来说道:“我打听锦衣府赵堂官必要照御史参的办去,只怕大老爷和珍大爷吃不住。”众人都道:“二老爷,还得是你出去求求王爷,怎么挽回挽回才好。不然这两家就完了。”贾政答应致谢,众人都散。
\end{parag}


\begin{parag}
    那时天已点灯时候,贾政进去请贾母的安,见贾母略略好些。回到自己房中,埋怨贾琏夫妇不知好歹,如今闹出放账取利的事情,大家不好。方见凤姐所为,心里很不受用。凤姐现在病重,知他所有什物尽被抄抢一光,心内郁结,一时未便埋怨,暂且隐忍不言。一夜无话。次早贾政进内谢恩,并到北静王府西平王府两处叩谢,求两位王爷照应他哥哥侄儿。两位应许。贾政又在同寅相好处托情。
\end{parag}


\begin{parag}
    且说贾琏打听得父兄之事不很妥,无法可施,只得回到家中。平儿守着凤姐哭泣,秋桐在耳房中抱怨凤姐。贾琏走近旁边,见凤姐奄奄一息,就有多少怨言,一时也说不出来。平儿哭道:“如今事已如此,东西已去不能复来。奶奶这样,还得再请个大夫调治调治才好。”贾琏啐道:“我的性命还不保,我还管他么!”凤姐听见,睁眼一瞧,虽不言语,那眼泪流个不尽,见贾琏出去,便与平儿道:“你别不达事务了,到了这样田地,你还顾我做什么。我巴不得今儿就死才好。只要你能够眼里有我,我死之后,你扶养大了巧姐儿,我在阴司里也感激你的。”平儿听了,放声大哭。凤姐道:“你也是聪明人。他们虽没有来说我,他必抱怨我。虽说事是外头闹的,我若不贪财,如今也没有我的事,不但是枉费心计,挣了一辈子的强,如今落在人后头。我只恨用人不当,恍惚听得那边珍大爷的事说是强占良民妻子为妾,不从逼死,有个姓张的在里头,你想想还有谁,若是这件事审出来,咱们二爷是脱不了的,我那时怎样见人。我要实时就死,又耽不起吞金服毒的。你到还要请大夫,可不是你为顾我反倒害了我了么。”平儿愈听愈惨,想来实在难处,恐凤姐自寻短见,只得紧紧守着。幸贾母不知底细,因近日身子好些,又见贾政无事,宝玉宝钗在旁天天不离左右,略觉放心。素来最疼凤姐,便叫鸳鸯“将我体己东西拿些给凤丫头,再拿些银钱交给平儿,好好的伏侍好了凤丫头,我再慢慢的分派。”又命王夫人照看了邢夫人。又加了宁国府第入官,所有财产房地等并家奴等俱造册收尽,这里贾母命人将车接了尤氏婆媳等过来。可怜赫赫宁府只剩得他们婆媳两个并佩凤偕鸾二人,连一个下人没有。贾母指出房子一所居住,就在惜春所住的间壁。又派了婆子四人丫头两个伏侍。一应饭食起居在大厨房内分送,衣裙什物又是贾母送去,零星需用亦在账房内开销,俱照荣府每人月例之数。那贾赦贾珍贾蓉在锦衣府使用,账房内实在无项可支。如今凤姐一无所有,贾琏况又多债务满身,贾政不知家务,只说已经托人,自有照应。贾琏无计可施,想到那亲戚里头薛姨妈家已败,王子腾已死,余者亲戚虽有,俱是不能照应,只得暗暗差人下屯将地亩暂卖了数千金作为监中使费。贾琏如此一行,那些家奴见主家势败,也便趁此弄鬼,并将东庄租税也就指名借用些。此是后话,暂且不提。
\end{parag}


\begin{parag}
    且说贾母见祖宗世职革去,现在子孙在监质审,邢夫人尤氏等日夜啼哭,凤姐病在垂危,虽有宝玉宝钗在侧,只可解劝,不能分忧,所以日夜不宁,思前想后,眼泪不干。一日傍晚,叫宝玉回去,自己扎挣坐起,叫鸳鸯等各处佛堂上香,又命自己院内焚起斗香,用拐拄着出到院中。琥珀知是老太太拜佛,铺下大红短毡拜垫。贾母上香跪下磕了好些头,念了一回佛,含泪祝告天地道:“皇天菩萨在上,我贾门史氏,虔诚祷告,求菩萨慈悲。我贾门数世以来,不敢行凶霸道。我帮夫助子,虽不能为善,亦不敢作恶。必是后辈儿孙骄侈暴佚,暴殄天物,以致合府抄检。现在儿孙监禁,自然凶多吉少,皆由我一人罪孽,不教儿孙,所以至此。我今即求皇天保佑:在监逢凶化吉,有病的早早安身。总有合家罪孽,情愿一人承当,只求饶恕儿孙。若皇天见怜,念我虔诚,早早赐我一死,宽免儿孙之罪。”默默说到此,不禁伤心,呜呜咽咽的哭泣起来。鸳鸯珍珠一面解劝,一面扶进房去。只见王夫人带了宝玉宝钗过来请晚安,见贾母悲伤,三人也大哭起来。宝钗更有一层苦楚:想哥哥也在外监,将来要处决,不知可减缓否,翁姑虽然无事,眼见家业萧条,宝玉依然疯傻,毫无志气。想到后来终身,更比贾母王夫人哭得更痛。宝玉见宝钗如此大恸,他亦有一番悲戚。想的是老太太年老不得安,老爷太太见此光景不免悲伤,众姐妹风流云散,一日少似一日。追想在园中吟诗起社,何等热闹,自从林妹妹一死,我郁闷到今,又有宝姐姐过来,未便时常悲切。见他忧兄思母,日夜难得笑容,今见他悲哀欲绝,心里更加不忍,竟嚎啕大哭。鸳鸯,彩云,莺儿,袭人见他们如此,也各有所思,便也呜咽起来。余者丫头们看得伤心,也便陪哭,竟无人解慰。满屋中哭声惊天动地,将外头上夜婆子吓慌,急报于贾政知道。那贾政正在书房纳闷,听见贾母的人来报,心中着忙,飞奔进内。远远听得哭声甚众,打谅老太太不好,急得魂魄俱丧,疾忙进来,只见坐着悲啼,神魂方定。说是“老太太伤心,你们该劝解,怎么的齐打伙儿哭起来了。”众人听得贾政声气,急忙止哭,大家对面发怔。贾政上前安慰了老太太,又说了众人几句。各自心想道:“我们原恐老太太悲伤,故来劝解,怎么忘情大家痛哭起来。”正自不解,只见老婆子带了史侯家的两个女人进来,请了贾母的安,又向众人请安毕,便说:“我们家老爷,太太,姑娘打发我来,说听见府里的事原没有什么大事,不过一时受惊。恐怕老爷太太烦恼,叫我们过来告诉一声,说这里二老爷是不怕的了。我们姑娘本要自己来的,因不多几日就要出阁,所以不能来了。”贾母听了,不便道谢,说:“你回去给我问好。这是我们的家运合该如此。承你老爷太太惦记,过一日再来奉谢。你家姑娘出阁,想来你们姑爷是不用说的了。他们的家计如何?”两个女人回道:“家计倒不怎么着,只是姑爷长的很好,为人又和平。我们见过好几次,看来与这里宝二爷差不多,还听得说才情学问都好的。”贾母听了,喜欢道:“咱们都是南边人,虽在这里住久了,那些大规矩还是从南方礼儿,所以新姑爷我们都没见过。我前儿还想起我娘家的人来,最疼的就是你们家姑娘,一年三百六十天,在我跟前的日子倒有二百多天,混得这么大了。我原想给他说个好女婿,又为他叔叔不在家,我又不便作主。他既造化配了个好姑爷,我也放心。月里出阁我原想过来吃杯喜酒的,不料我家闹出这样事来,我的心就象在热锅里熬的似的,那里能够再到你们家去。你回去说我问好,我们这里的人都说请安问好。你替另告诉你家姑娘,不要将我放在心里。我是八十多岁的人了,就死也算不得没福的了。只愿他过了门,两口子和顺,百年到老,我便安心了。”说着,不觉掉下泪来。那女人道:“老太太也不必伤心。姑娘过了门,等回了九,少不得同姑爷过来请老太太的安,那时老太太见了才喜欢呢。”贾母点头。那女人出去。别人都不理论,只有宝玉听了发了一回怔,心里想道:“如今一天一天的都过不得了。为什么人家养了女儿到大了必要出嫁,一出了嫁就改变。史妹妹这样一个人又被他叔叔硬压着配人了,他将来见了我必是又不理我了。我想一个人到了这个没人理的分儿,还活着做什么。”想到那里,又是伤心。见贾母此时才安,又不敢哭泣,只是闷闷的。
\end{parag}


\begin{parag}
    一时贾政不放心,又进来瞧瞧老太太,见是好些,便出来传了赖大,叫他将合府里管事家人的花名册子拿来,一齐点了一点,除去贾赦入官的人,尚有三十余家,共男女二百十二名。贾政叫现在府内当差的男人共二十一名进来,问起历年居家用度,共有若干进来,该用若干出去。那管总的家人将近来支用簿子呈上。贾政看时,所入不敷所出,又加连年宫里花用,帐上有在外浮借的也不少。再查东省地租,近年所交不及祖上一半,如今用度比祖上更加十倍。贾政不看则已,看了急得跺脚道:“这了不得!我打量虽是琏儿管事,在家自有把持,岂知好几年头里已就寅年用了卯年的,还是这样装好看,竟把世职俸禄当作不打紧的事情,为什么不败呢!我如今要就省俭起来,已是迟了。”想到那里,背着手踱来踱去,竟无方法。众人知贾政不知理家,也是白操心着急,便说道:“老爷也不用焦心,这是家家这样的。若是统总算起来,连王爷家还不够。不过是装着门面,过到那里就到那里。如今老爷到底得了主上的恩典,才有这点子家产,若是一并入了官,老爷就不用过了不成。”贾政嗔道:“放屁!你们这班奴才最没有良心的,仗着主子好的时候任意开销,到弄光了,走的走,跑的跑,还顾主子的死活吗!如今你们道是没有查封是好,那知道外头的名声。大本儿都保不住,还搁得住你们在外头支架子说大话诓人骗人,到闹出事来望主子身上一推就完了。如今大老爷与珍大爷的事,说是咱们家人鲍二在外传播的,我看这人口册上并没有鲍二,这是怎么说?”众人回道:“这鲍二是不在册档上的。先前在宁府册上,为二爷见他老实,把他们两口子叫过来了。及至他女人死了,他又回宁府去。后来老爷衙门有事,老太太们爷们往陵上去,珍大爷替理家事带过来的,以后也就去了。老爷数年不管家事,那里知道这些事来。老爷打量册上没有名字的就只有这个人,不知一个人手下亲戚们也有,奴才还有奴才呢。”贾政道:“这还了得!”想去一时不能清理,只得喝退众人,早打了主意在心里了,且听贾赦等事审得怎样再定。
\end{parag}


\begin{parag}
    一日正在书房筹算,只见一人飞奔进来说:“请老爷快进内廷问话。”贾政听了心下着忙,只得进去。未知凶吉,下回分解。
\end{parag}
