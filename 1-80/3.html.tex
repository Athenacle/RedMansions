\chap{三}{金陵城起复贾雨村 荣国府收养林黛玉}

\begin{parag}
    金陵城起复贾雨村 荣国府收养\begin{note}甲侧:二字触目凄凉之至!\end{note}林黛玉
\end{parag}

\begin{parag}
    \begin{note}蒙、戚回前:我为你持戒,我为你吃斋,我为你百行百计不舒怀,我为你泪眼愁眉难解。无人处,自疑猜,生怕那慧性灵心偷改。
        宝玉通灵可爱,天生有眼堪穿。万年幸一遇仙缘,从此春光美满。随时喜怒哀乐,远却离合悲欢。地久天长香影连,可意方舒心眼。
        宝玉衔来,是补天之余,落地已久,得地气收藏,因人而现。其性质内阳外阴,其形体光白温润,天生有眼可穿,故名曰宝玉,将欲得者尽宝爱此玉之意也。
        天地循环秋复春,生生死死旧重新。君家著笔描风月,宝玉颦颦解爱人。\end{note}
\end{parag}


\begin{parag}
    却说雨村忙回头看时,不是别人,乃是当日同僚一案参革的号张如圭\begin{note}甲侧:盖言如鬼如蜮也,亦非正人正言。\end{note}者。他本系此地人,革后家居,今打听得都中奏准起复旧员之信,他便四下里寻情找门路,忽遇见雨村,故忙道喜。二人见了礼,张如圭便将此信告诉雨村,雨村自是欢喜,忙忙的叙了两句,\begin{note}甲侧:画出心事。\end{note}遂作别各自回家。冷子兴听得此言,便忙献计,\begin{note}甲侧:毕肖赶热灶者。\end{note}令雨村央烦林如海,转向都中去央烦贾政。雨村领其意,作别回至馆中,忙寻邸报看真确了。\begin{note}甲侧:细。\end{note}次日,面谋之如海。如海道:“天缘凑巧,因贱荆去世,都中家岳母念及小女无人依傍教育,前已遣了男女船只来接,因小女未曾大痊,故未及行。此刻正思向蒙训教之恩未经酬报,遇此机会,岂有不尽心图报之理。但请放心,弟已预为筹划至此,已修下荐书一封,转托内兄务为周全协佐,方可稍尽弟之鄙诚,即有所费用之例,弟于内兄信中已注明白,亦不劳尊兄多虑矣。”雨村一面打恭,谢不释口,一面又问:“不知令亲大人现居何职?\begin{note}甲侧:奸险小人欺人语。\end{note}只怕晚生草率,不敢骤然入都干渎。”\begin{note}甲侧:全是假,全是诈。\end{note}如海笑道:“若论舍亲,与尊兄犹系同谱,乃荣公之孙。大内兄现袭一等将军,名赦,字恩侯,二内兄名政,字存周,\begin{note}甲侧:二名二字皆颂德而来,与子兴口中作证。\end{note}现任工部员外郎,其为人谦恭厚道,大有祖父遗风,非膏粱轻薄仕宦之流,\begin{note}复醒一笔。\end{note}故弟方致书烦托。否则不但有污尊兄之清操,即弟亦不屑为矣。”\begin{note}甲侧:写如海实写政老。所谓此书有不写之写是也。\end{note}雨村听了,心下方信了昨日子兴之言,于是又谢了林如海。如海乃说:“已择了出月初二日小女入都,尊兄即同路而往,岂不两便?”雨村唯唯听命,心中十分得意。
\end{parag}


\begin{parag}
    如海遂打点礼物幷饯行之事,雨村一一领了。
\end{parag}


\begin{parag}
    那女学生黛玉,身体方愈,原不忍弃父而往,无奈他外祖母致意务去,且兼如海说:“汝父年将半百,再无续室之意,且汝多病,年又极小,上无亲母教养,下无姊妹兄弟扶持,\begin{note}甲侧:可怜!一句一滴血,一句一滴血之文。\end{note}今依傍外祖母及舅氏姊妹去,正好减我顾盼之忧,何反云不往?”黛玉听了,方洒泪拜别,\begin{note}甲侧:实写黛玉。蒙侧:此一段是不肯使黛玉作弃父乐为远游者。以此可见作者之心宝爱黛玉如己。\end{note}随了奶娘及荣府几个老妇人登舟而去。雨村另有一只船,带两个小童,依附黛玉而行。\begin{note}甲侧:老师依附门生,怪道今时以收纳门生为幸。\end{note}
\end{parag}


\begin{parag}
    有日到了都中,\begin{note}甲侧:繁中简笔。\end{note}进入神京,雨村先整了衣冠,\begin{note}甲侧:且按下黛玉以待细写。今故先将雨村安置过一边,方起荣府中之正文也。\end{note}带了小童,\begin{note}甲侧:至此渐渐好看起来也。\end{note}拿著宗侄的名帖,\begin{note}甲侧:此帖妙极,可知雨村的品行矣。\end{note}至荣府的门前投了。彼时贾政已看了妹丈之书,即忙请入相会。见雨村相貌魁伟,言语不俗,且这贾政最喜读书人,礼贤下士,济弱扶危,大有祖风,况又系妹丈致意,因此优待雨村,\begin{note}甲侧:君子可欺其方也,况雨村正在王莽谦恭下士之时,虽政老亦为所惑,在作者系指东说西也。\end{note}更又不同,便竭力内中协助,题奏之日,轻轻谋\begin{note}甲侧:《春秋》字法。\end{note}了一个复职候缺,不上两个月,金陵应天府缺出,便谋补\begin{note}甲侧:《春秋》字法。\end{note}了此缺,拜辞了贾政,择日上任去了。\begin{note}甲侧:因宝钗故及之,一语过至下回。\end{note}不在话下。
\end{parag}


\begin{parag}
    且说黛玉自那日弃舟登岸时,\begin{note}甲侧:这方是正文起头处。此后笔墨,与前两回不同。\end{note}便有荣国府打发了轿子幷拉行李的车辆久候了。这林黛玉常听得\begin{note}甲侧:三字细。\end{note}母亲说过,他外祖母家与别家不同。他近日所见的这几个三等仆妇,吃穿用度,已是不凡了,何况今至其家。因此步步留心,时时在意,不肯轻易多说一句话,多行一步路,惟恐被人耻笑了他去。\begin{note}甲侧:写黛玉自幼之心机。[黛玉自忖之语。]\end{note}自上了轿,进入城中,从纱窗向外瞧了一瞧,其街市之繁华,人烟之阜盛,自与别处不同。\begin{note}甲侧:先从街市写来。\end{note}又行了半日,忽见街北蹲著两个大石狮子,三间兽头大门,门前列坐著十来个华冠丽服之人。正门却不开,只有东西两角门有人出入。正门之上有一匾,匾上大书“敕造宁国府”五个大字。\begin{note}甲侧:先写宁府,这是由东向西而来。\end{note}黛玉想道:“这必是外祖之长房了。”想著,又往西行,不多远,照样也是三间大门,方是荣国府了。却不进正门,只进了西边角门。那轿夫抬进去,走了一射之地,将转弯时,便歇下退出去了。后面的婆子们已都下了轿,赶上前来。另换了三四个衣帽周全十七八岁的小厮上来,复抬起轿子。众婆子步下围随至一垂花门前落下。众小厮退出,众婆子上来打起轿帘,扶黛玉下轿。林黛玉扶著婆子的手,进了垂花门,两边是抄手游廊,当中是穿堂,当地放著一个紫檀架子大理石的大插屏。转过插屏,小小的三间厅,厅后就是后面的正房大院。正面五间上房,皆雕梁画栋,两边穿山游廊厢房,挂著各色鹦鹉、画眉等鸟雀。台矶之上,坐著几个穿红著绿的丫头,一见他们来了,便忙都笑迎上来,说:“刚才老太太还念呢,可巧就来了。”\begin{note}甲侧:如见如闻,活现于纸上之笔。好看煞!\end{note}于是三四人争著打起帘笼,\begin{note}甲侧:真有是事,真有是事!\end{note}一面听得人回话:“林姑娘到了。”\begin{note}甲眉:此书得力处,全是此等地方,所谓“颊上三毫”也。\end{note}
\end{parag}


\begin{parag}
    黛玉方进入房时,只见两个人搀著一位鬓发如银的老母迎上来,黛玉便知是他外祖母。方欲拜见时,早被他外祖母一把搂入怀中,心肝儿肉叫著大哭起来。\begin{note}甲侧:几千斤力量写此一笔。\end{note}当下地下侍立之人,无不掩面涕泣,\begin{note}甲侧:旁写一笔,更妙!\end{note}黛玉也哭个不住。\begin{note}甲侧:自然顺写一笔。\end{note}一时众人慢慢解劝住了,黛玉方拜见了外祖母。\begin{note}甲眉:书中正文之人,却如此写出,却是天生地设章法,不见一丝勉强。\end{note}此即冷子兴所云之史氏太君,贾赦、贾政之母也。\begin{note}甲侧:书中人目太繁,故明注一笔,使观者省眼。\end{note}当下贾母一一指与黛玉:“这是你大舅母,\begin{note}邢氏。\end{note}这是你二舅母,\begin{note}王氏。\end{note}这是你先珠大哥的媳妇珠大嫂子。”黛玉一一拜见过。贾母又说:“请姑娘们来。今日远客才来,可以不必上学去了。”众人答应了一声,便去了两个。
\end{parag}


\begin{parag}
    不一时,只见三个奶嬷嬷幷五六个丫鬟,簇拥著三个姊妹来了。\begin{note}甲侧:声势如现纸上。甲眉:从黛玉眼中写三人。\end{note}第一个肌肤微丰,\begin{note}甲侧:不犯宝钗。\end{note}合中身材,腮凝新荔,鼻腻鹅脂,温柔沉默,观之可亲。\begin{note}甲侧:为迎春写照。\end{note}第二个削肩细腰,\begin{note}甲侧:《洛神赋》中云“肩若削成”是也。\end{note}长挑身材,鸭蛋脸面,俊眼修眉,顾盼神飞,文彩精华,见之忘俗。\begin{note}甲侧:为探春写照。\end{note}第三个身量未足,形容尚小。\begin{note}甲眉:浑写一笔更妙!必个个写去则板矣。可笑近之小说中有一百个女子,皆是如花似玉一副脸面。\end{note}其钗环裙袄,\begin{note}甲侧:是极。\end{note}三人皆是一样的妆饰。\begin{note}甲侧:毕肖。\end{note}黛玉忙起身迎上来见礼,\begin{note}甲侧:此笔亦不可少。\end{note}互相厮认过,大家归了坐。丫鬟们斟上茶来。不过说些黛玉之母如何得病,如何请医服药,如何送死发丧。不免贾母又伤感起来,\begin{note}甲侧:妙!\end{note}因说:“我这些儿女,所疼者独有你母,今日一旦先舍我而去,连面也不能一见,今见了你,我怎不伤心!”说著,搂了黛玉在怀,又呜咽起来。众人忙都宽慰解释,方略略止住。\begin{note}甲侧:总为黛玉自此不能别往。\end{note}
\end{parag}


\begin{parag}
    众人见黛玉年貌虽小,其举止言谈不俗,身体面庞虽怯弱不胜,\begin{note}甲侧:写美人是如此笔仗,看官怎得不叫绝称赏!\end{note}却有一段自然的风流态度,\begin{note}甲侧:为黛玉写照。众人目中,只此一句足矣。甲眉:从众人目中写黛玉。草胎卉质,岂能胜物耶?想其衣裙皆不得不勉强支持者也。\end{note}便知他有不足之症。因问:“常服何药,如何不急为疗治?”黛玉道:“我自来是如此,从会吃饮食时便吃药,到今日未断,请了多少名医修方配药,皆不见效。那一年我三岁时,听得说\begin{note}甲侧:文字细如牛毛。\end{note}来了一个癞头和尚,\begin{note}甲眉:奇奇怪怪一至于此。通部中假借癞僧、跛道二人点明迷情幻海中有数之人也。非袭《西游》中一味无稽、至不能处便用观世音可比。\end{note}说要化我去出家,我父母固是不从。他又说:‘既舍不得他,只怕他的病一生也不能好的了。若要好时,除非从此以后总不许见哭声,除父母之外,凡有外姓亲友之人,一概不见,方可平安了此一世。’疯疯癫癫,说了这些不经之谈,\begin{note}甲侧:是作书者自注。\end{note}也没人理他。如今还是吃人参养荣丸。”\begin{note}甲侧:人生自当自养荣卫。甲眉:甄英莲乃副十二钗之首,却明写癞僧一点。今黛玉为正十二钗之冠,反用暗笔。盖正十二钗人或洞悉可知,副十二钗或恐观者忽略,故写极力一提,使观者万勿稍加玩忽之意耳。\end{note}贾母道:“正好,我这里正配丸药呢。叫他们多配一料就是了。”\begin{note}甲侧:为后菖菱伏脉。\end{note}
\end{parag}


\begin{parag}
    一语未了,只听后院中有人笑声,\begin{note}甲侧:懦笔庸笔何能及此!\end{note}说:“我来迟了,不曾迎接远客!”\begin{note}甲侧:第一笔,阿凤三魂六魄已被作者拘定了,后文焉得不活跳纸上?此等文字非仙助即神助,从何而得此机括耶?甲眉:另磨新墨,搦锐笔,特独出熙凤一人。未见其人,先使闻声,所谓“绣幡开,遥见英雄俺”也。\end{note}黛玉纳罕道:“这些人个个皆敛声屏气,恭肃严整如此,这来者系谁,这样放诞无礼?”\begin{note}甲侧:原有此一想。\end{note}心下想时,只见一群媳妇丫鬟围拥著一个人从后房门进来。这个人打扮与众姑娘不同,彩绣辉煌,恍若神妃仙子:
\end{parag}

\begin{qute2sp}
    头上戴著金丝八宝攒珠髻,绾著朝阳五凤挂珠钗,\begin{note}甲侧:头。\end{note}项上戴著赤金盘螭璎珞圈,\begin{note}甲侧:颈。\end{note}裙边系著豆绿宫绦,双衡比目玫瑰佩,\begin{note}甲侧:腰。\end{note}身上穿著缕金百蝶穿花大红洋缎窄褃袄,外罩五彩刻丝石青银鼠褂,下著翡翠撒花洋绉裙。一双丹凤三角眼,两弯柳叶吊梢眉,身量苗条,体格风骚,粉面含春威不露,丹唇未启笑先开。\begin{note}甲侧:为阿凤写照。甲眉:试问诸公:从来小说中可有写形追像至此者?\end{note}
\end{qute2sp}

\begin{parag}
    黛玉连忙起身接见。贾母笑\begin{note}甲侧:阿凤一至,贾母方笑,与后文多少笑字作偶。\end{note}道:“你不认得他,他是我们这里有名的一个泼皮破落户儿,南省俗谓作‘辣子’,你只叫他‘凤辣子’就是了。”\begin{note}甲侧:阿凤笑声进来,老太君打诨,虽是空口传声,却是补出一向晨昏起居,阿凤于太君处承欢应候一刻不可少之人,看官勿以闲文淡文也。\end{note}黛玉正不知以何称呼,只见众姊妹都忙告诉他道:“这是琏嫂子。”黛玉虽不识,也曾听见母亲说过,大舅贾赦之子贾琏,娶的就是二舅母王氏之内侄女,自幼假充男儿教养的,学名王熙凤。\begin{note}甲侧:奇想奇文。以女子曰“学名”固奇,然此偏有学名的反倒不识字,不曰学名者反若假。\end{note}黛玉忙陪笑见礼,以“嫂”呼之。这熙凤携著黛玉的手,上下细细打谅了一回,\begin{note}甲侧:写阿凤全部传神第一笔也。\end{note}仍送至贾母身边坐下,因笑道:“天下真有这样标致的人物,我今儿才算见了!\begin{note}甲侧:这方是阿凤言语。若一味浮词套语,岂复为阿凤哉!甲眉:“真有这样标致人物”出自凤口,黛玉丰姿可知。宜作史笔看。\end{note}况且这通身的气派,竟不象老祖宗的外孙女儿,竟是个嫡亲的孙女,\begin{note}甲侧:仍归太君,方不失《石头记》文字,且是阿凤身心之至文。\end{note}怨不得老祖宗天天口头心头一时不忘。\begin{note}甲侧:却是极淡之语,偏能恰投贾母之意。\end{note}只可怜我这妹妹这样命苦,\begin{note}甲侧:这是阿凤见黛玉正文。\end{note}怎么姑妈偏就去世了!”\begin{note}甲侧:若无这几句,便不是贾府媳妇。\end{note}说著,便用帕拭泪。贾母笑道:“我才好了,你倒来招我。\begin{note}甲侧:文字好看之极。\end{note}你妹妹远路才来,身子又弱,也才劝住了,快再休提前话!”\begin{note}甲侧:反用贾母劝,看阿凤之术亦甚矣。\end{note}这熙凤听了,忙转悲为喜道:“正是呢!我一见了妹妹,一心都在他身上了,又是喜欢,又是伤心,竟忘记了老祖宗。该打,该打!”又忙携黛玉之手,问:“妹妹几岁了?可也上过学?现吃什么药?在这里不要想家,想要什么吃的,什么玩的,只管告诉我,丫头老婆们不好了,也只管告诉我。”一面又问婆子们:“林姑娘的行李东西可搬进来了?带了几个人来?\begin{note}甲侧:当家的人事如此,毕肖!\end{note}你们赶早打扫两间下房,让他们去歇歇。”
\end{parag}



\begin{parag}
    说话时,已摆了茶果上来,熙凤亲为捧茶捧果。\begin{note}甲侧:总为黛玉眼中写出。\end{note}又见二舅母问他:“月钱放过了不曾?”\begin{note}甲侧:不见后文,不见此笔之妙。\end{note}熙凤道:“月钱已放完了。才刚带著人到后楼上找缎子,\begin{note}甲侧:接闲文,是本意避繁也。\end{note}找了这半日,也幷没有见昨日太太说的那样的。\begin{note}甲侧:却是日用家常实事。\end{note}想是太太记错了?”王夫人道:“有没有,什么要紧。”因又说道:“该随手拿出两个来给你这妹妹去裁衣裳的,\begin{note}甲侧:仍归前文。妙妙!\end{note}等晚上想著叫人再去拿罢,可别忘了。”熙凤道:“这倒是我先料著了,知道妹妹不过这两日到的,我已预备下了,\begin{note}甲眉:余知此缎阿凤幷未拿出,此借王夫人之语机变欺人处耳。若信彼果拿出预备,不独被阿凤瞒过,亦且被石头瞒过了。\end{note}等太太回去过了目好送来。”\begin{note}甲侧:试看他心机。\end{note}王夫人一笑,点头不语。\begin{note}甲侧:深取之意。[凤姐是个当家人。]\end{note}
\end{parag}


\begin{parag}
    当下茶果已撤,贾母命两个老嬷嬷带了黛玉去见两个母舅。时贾赦之妻邢氏忙亦起身,笑回道:“我带了外甥女过去,倒也便宜。”贾母笑道:“正是呢,你也去罢,不必过来了。”邢夫人答应了一声“是”字,遂带了黛玉与王夫人作辞,大家送至穿堂前。出了垂花门,早有众小厮们拉过一辆翠幄青绸车。邢夫人携了黛玉,坐在上面,\begin{note}[未识黛卿能乘此否?]\end{note}众婆子们放下车帘,方命小厮们抬起,拉至宽处,方驾上驯骡,亦出了西角门,往东过荣府正门,便入一黑油大门中,至仪门前方下来。众小厮退出,方打起车帘,邢夫人搀著黛玉的手,进入院中。黛玉度其房屋院宇,必是荣府中花园隔断过来的。\begin{note}甲侧:黛玉之心机眼力。\end{note}进入三层仪门,果见正房厢庑游廊,悉皆小巧别致,不似方才那边轩峻壮丽,且院中随处之树木山石皆有。\begin{note}甲侧:为大观园伏脉。试思荣府园今在西,后之大观园偏写在东,何不畏难之若此?\end{note}一时进入正室,早有许多盛妆丽服之姬妾丫鬟迎著,邢夫人让黛玉坐了,一面命人到外面书房去请贾赦。\begin{note}甲侧:这一句都是写贾赦,妙在全是指东击西打草惊蛇之笔。若看其写一人即作此一人看,先生便呆了。\end{note}一时人来回话说:“老爷说了:‘连日身上不好,见了姑娘彼此倒伤心,\begin{note}甲侧:追魂摄魄。甲眉:余久不作此语矣,见此语未免一醒。\end{note}暂且不忍相见。\begin{note}甲侧:若一见时,不独死板,且亦大失情理,亦不能有此等妙文矣。\end{note}劝姑娘不要伤心想家,跟著老太太和舅母,即同家里一样。姊妹们虽拙,大家一处伴著,亦可以解些烦闷。\begin{note}甲侧:赦老亦能作此语,叹叹!\end{note}或有委屈之处,只管说得,不要外道才是。’”黛玉忙站起来,一一听了。再坐一刻,便告辞。邢夫人苦留吃过晚饭去,黛玉笑回道:“舅母爱惜赐饭,原不应辞,只是还要过去拜见二舅舅,恐领了赐去不恭,\begin{note}甲侧:得体。\end{note}异日再领,未为不可。望舅母容谅。”邢夫人听说,笑道:“这倒是了。”遂令两三个嬷嬷用方才的车好生送了姑娘过去,于是黛玉告辞。邢夫人送至仪门前,又嘱咐了众人几句,眼看著车去了方回来。
\end{parag}


\begin{parag}
    一时黛玉进了荣府,下了车。众嬷嬷引著,便往东转弯,穿过一个东西的穿堂,\begin{note}甲侧:这一个穿堂是贾母正房之南者,凤姐处所通者则是贾母正房之北。\end{note}向南大厅之后,仪门内大院落,上面五间大正房,两边厢房鹿顶耳房钻山,四通八达,轩昂壮丽,比贾母处不同。黛玉便知这方是正经正内室,一条大甬路,直接出大门的。进入堂屋中,抬头迎面先看见一个赤金九龙青地大匾,匾上写著斗大的三个大字,是“荣禧堂”,后有一行小字“某年月日,书赐荣国公贾源”,又有“万几宸翰之宝”。大紫檀雕螭案上,设著三尺来高青绿古铜鼎,悬著待漏随朝墨龙大画,一边是金蜼彝,\begin{note}甲侧:蜼,音垒。周器也。\end{note}一边是玻璃台。\begin{note}甲侧:(上台下皿),音海。盛酒之大器也。\end{note}地下两溜十六张楠木交椅。又有一副对联,乃乌木联牌,镶著錾银的字迹,\begin{note}甲侧:雅而丽,富而文。\end{note}道是:
\end{parag}


\begin{poem}
    \begin{pl} 座上珠玑昭日月,\end{pl}

    \begin{pl}堂前黼黻焕烟霞。\end{pl}\begin{note}甲夹:实贴。\end{note}
\end{poem}


\begin{parag}
    下面一行小字,道是:“同乡世教弟勋袭东安郡王穆莳拜手书。”\begin{note}甲侧:先虚陪一笔。\end{note}
\end{parag}


\begin{parag}
    原来王夫人时常居坐宴息,亦不在这正室,\begin{note}甲侧:黛玉由正室一段而来,是为拜见政老耳,故进东房。\end{note}只在这正室东边的三间耳房内。\begin{note}甲侧:若见王夫人,直写引至东廊小正室内矣。\end{note}于是老嬷嬷引黛玉进东房门来。临窗大炕上铺著猩红洋罽,正面设著大红金钱蟒靠背,石青金钱蟒引枕,秋香色金钱蟒大条褥。两边设一对梅花式洋漆小几。左边几上文王鼎匙箸香盒,右边几上汝窑美人觚“”觚内插著时鲜花卉,幷茗碗痰盒等物。地下面西一溜四张椅上,都搭著银红撒花椅搭,底下四副脚踏。椅之两边,也有一对高几,几上茗碗瓶花俱备。其余陈设,自不必细说。\begin{note}甲侧:此不过略叙荣府家常之礼数,特使黛玉一识阶级座次耳,余则繁。\end{note}老嬷嬷们让黛玉炕上坐,炕沿上却有两个锦褥对设,黛玉度其位次,便不上炕,只向东边椅子上坐了。\begin{note}甲侧:写黛玉心意。\end{note}本房内的丫鬟忙捧上茶来。黛玉一面吃茶,一面打谅这些丫鬟们,装饰衣裙,举止行动,果亦与别家不同。
\end{parag}


\begin{parag}
    茶未吃了,只见一个穿红绫袄青缎掐牙背心的丫鬟\begin{note}甲侧:金乎?玉乎?\end{note}走来笑说道:“太太说,请林姑娘到那边坐罢。”老嬷嬷听了,于是又引黛玉出来,到了东廊三间小正房内。正房炕上横设一张炕桌,桌上磊著书籍茶具,\begin{note}甲侧:伤心笔,堕泪笔。\end{note}靠东壁面西设著半旧的青缎靠背引枕。王夫人却坐在西边下首,亦是半旧的青缎靠背坐褥。见黛玉来了,便往东让。黛玉心中料定这是贾政之位。\begin{note}甲侧:写黛玉心到眼到,伧夫但云为贾府叙坐位,岂不可笑?\end{note}因见挨炕一溜三张椅子上,也搭著半旧的\begin{note}甲侧:三字有神。此处则一色旧的,可知前正室中亦非家常之用度也。可笑近之小说中,不论何处,则曰商彝周鼎、绣幕珠帘、孔雀屏、芙蓉褥等样字眼。甲眉:近闻一俗笑语云:一庄农人进京回家,众人问曰:“你进京去可见些个世面否?”庄人曰:“连皇帝老爷都见了。”众罕然问曰:“皇帝如何景况?”庄人曰:“皇帝左手拿一金元宝,右手拿一银元宝,马上稍著一口袋人参,行动人参不离口。一时要屙屎了,连擦屁股都用的是鹅黄缎子,所以京中掏茅厕的人都富贵无比。”试思凡稗官写富贵字眼者,悉皆庄农进京之一流也。盖此时彼实未身经目睹,所言皆在情理之外焉。又如人嘲作诗者亦往往爱说富丽语,故有“胫骨变成金玳瑁,,眼睛嵌作碧璃琉”之诮。余自是评《石头记》,非鄙弃前人也。\end{note}弹墨椅袱,黛玉便向椅上坐了。王夫人再四携他上炕,他方挨王夫人坐了。王夫人因说:“你舅舅今日斋戒去了,\begin{note}甲侧:点缀宦途。\end{note}再见罢。\begin{note}甲侧:赦老不见,又写政老。政老又不能见,是重不见重,犯不见犯。作者惯用此等章法。\end{note}只是有一句话嘱咐你:你三个姊妹倒都极好,以后一处念书认字学针线,或是偶一顽笑,都有尽让的。但我不放心的最是一件:我有一个孽根祸胎,\begin{note}甲侧:四字是血泪盈面,不得已无奈何而下。四字是作者痛哭。\end{note}是家里的‘混世魔王’,\begin{note}甲侧:与“绛洞花王”为对看。\end{note}今日因庙里还愿去了,\begin{note}甲侧:是富贵公子。\end{note}尚未回来,晚间你看见便知了。你只以后不要睬他,你这些姊妹都不敢沾惹他的。”
\end{parag}


\begin{parag}
    黛玉亦常听得母亲说过,二舅母生的有个表兄,乃衔玉而诞,顽劣异常,\begin{note}甲侧:与甄家子恰对。\end{note}极恶读书,\begin{note}甲侧:是极恶每日“诗云”“子曰”的读书。\end{note}最喜在内帏厮混,外祖母又极溺爱,无人敢管。今见王夫人如此说,便知说的是这表兄了。\begin{note}甲侧:这是一段反衬笔法。黛玉心用“猜度蠢物”等句对著去,方不失作者本旨。\end{note}因陪笑道:“舅母说的,可是衔玉所生的这位哥哥?在家时亦曾听见母亲常说,这位哥哥比我大一岁,小名就唤宝玉,\begin{note}甲侧:以黛玉道宝玉名,方不失正文。\end{note}虽\begin{note}甲侧:“虽”字是有情字,宿根而发,勿得泛泛看过。\end{note}极憨顽,说在姊妹情中极好的。况我来了,自然只和姊妹同处,兄弟们自是别院另室的,\begin{note}甲侧:又登开一笔,妙妙!\end{note}岂得去沾惹之理?”王夫人笑道:“你不知道原故。他与别人不同,自幼因老太太疼爱,原系同姊妹们一处娇养惯了的。\begin{note}甲侧:此一笔收回,是明通部同处原委也。\end{note}若姊妹们有日不理他,他倒还安静些,纵然他没趣,不过出了二门,背地里拿著他两个小么儿出气,咕唧一会子就完了。\begin{note}甲侧:这可是宝玉本性真情,前四十九字迥异之批今始方知。盖小人口碑累累如是。是是非非任尔口角,大都皆然。\end{note}若这一日姊妹们和他多说一句话,他心里一乐,便生出多少事来。所以嘱咐你别睬他。他嘴里一时甜言蜜语,一时有天无日,一时又疯疯傻傻,只休信他。”
\end{parag}


\begin{parag}
    黛玉一一的都答应著。\begin{note}甲眉:不写黛玉眼中之宝玉,却先写黛玉心中已早有一宝玉矣,幻妙之至!自冷子兴口中之后,余已极思欲一见,及今尚未得见,狡猾之至!\end{note}只见一个丫鬟来回:“老太太那里传晚饭了。”王夫人忙携黛玉从后房门\begin{note}甲侧:后房门。\end{note}由后廊\begin{note}甲侧:是正房后廊也。\end{note}往西,出了角门,\begin{note}甲侧:这是正房后西界墙角门。\end{note}是一条南北宽夹道。南边是倒座三间小小的抱厦厅,北边立著一个粉油大影壁,后有一半大门,小小一所房室。王夫人笑指向黛玉道:“这是你凤姐姐的屋子,回来你好往这里找他来,少什么东西,你只管和他说就是了。”这院门上也有\begin{note}甲侧:二字是他处不写之写也。\end{note}四五个才总角的小厮,都垂手侍立。王夫人遂携黛玉穿过一个东西穿堂,\begin{note}甲眉:这正是贾母正室后之穿堂也,与前穿堂是一带之屋,中一带乃贾母之下室也。记清。\end{note}便是贾母的后院了。\begin{note}甲侧:写得清,一丝不错。\end{note}于是,进入后房门,已有多人在此伺候,见王夫人来了,方安设桌椅。\begin{note}甲侧:不是待王夫人用膳,是恐使王夫人有失侍膳之礼耳。\end{note}贾珠之妻李氏捧饭,熙凤安箸,王夫人进羹。贾母正面榻上独坐,两边四张空椅,熙凤忙拉了黛玉在左边第一张椅上坐了,黛玉十分推让。贾母笑道:“你舅母你嫂子们不在这里吃饭。你是客,原应如此坐的。”黛玉方告了座,坐了。贾母命王夫人坐了。迎春姊妹三个告了座方上来。迎春便坐右手第一,探春左第二,惜春右第二。旁边丫鬟执著拂尘、漱盂、巾帕。李、凤二人立于案旁布让。外间伺候之媳妇丫鬟虽多,却连一声咳嗽不闻。寂然饭毕,各有丫鬟用小茶盘捧上茶来。当日林如海教女以惜福养身,云饭后务待饭粒咽尽,过一时再吃茶,方不伤脾胃。\begin{note}甲侧:夹写如海一派书气,最妙!\end{note}今黛玉见了这里许多事情不合家中之式,不得不随的,少不得一一改过来,因而接了茶。早见人又捧过漱盂来,黛玉也照样漱了口。盥手毕,又捧上茶来,这方是吃的茶。\begin{note}甲侧:总写黛玉以后之事,故只以此一件小事略为一表也。甲眉:余看至此,故想日前所阅“王敦初尚公主,登厕时不知塞鼻用枣,敦辄取而啖之,早为宫人鄙诮多矣”。今黛玉若不漱此茶,或饮一口,不为荣婢所诮乎?观此则知黛玉平生之心思过人。\end{note}贾母便说:“你们去罢,让我们自在说话儿。”王夫人听了,忙起身,又说了两句闲话,方引凤、李二人去了。贾母因问黛玉念何书。黛玉道:“只刚念了《四书》。”\begin{note}甲侧:好极!稗官专用“腹隐五车书”者来看。\end{note}黛玉又问姊妹们读何书。贾母道:“读的是什么书,不过是认得两个字,不是睁眼的瞎子罢了!”
\end{parag}


\begin{parag}
    一语未了,只听外面一阵脚步响,\begin{note}甲侧:与阿凤之来相映而不相犯。\end{note}丫鬟进来笑道:“宝玉来了!”\begin{note}甲侧:余为一乐。\end{note}黛玉心中正疑惑著:“这个宝玉,不知是怎生个惫懒人物,懵懂顽童?”\begin{note}甲侧:文字不反,不见正文之妙,似此应从《国策》得来。\end{note}倒不见那蠢物\begin{note}甲侧:这蠢物不是那蠢物,却有个极蠢之物相待。妙极!\end{note}也罢了。心中想著,忽见丫鬟话未报完,已进来了一位年轻的公子:
\end{parag}


\begin{qute2sp}
    头上戴著束发嵌宝紫金冠,齐眉勒著二龙抢珠金抹额,穿一件二色金百蝶穿花大红箭袖,束著五彩丝攒花结长穗宫绦,外罩石青起花八团倭锻排穗褂,登著青缎粉底小朝靴。面若中秋之月,\begin{note}甲眉:此非套“满月”,盖人生有面扁而青白色者,则皆可谓之秋月也。用“满月”者不知此意。\end{note}色如春晓之花。\begin{note}甲眉:“少年色嫩不坚牢”,以及“非夭即贫”之语,余犹在心。今阅至此,放声一哭。\end{note}鬓若刀裁,眉如墨画,面如桃瓣,目若秋波。虽怒时而若笑,即嗔视而有情。
\end{qute2sp}


\begin{parag}
    \begin{note}甲侧:真真写杀。\end{note}项上金螭璎珞,又有一根五色丝绦,系著一块美玉。黛玉一见,便吃一大惊,心下想道:“好生奇怪,倒象在那里见过一般,何等眼熟到如此!”\begin{note}甲侧:正是想必在灵河岸上三生石畔曾见过。\end{note}只见这宝玉向贾母请了安,贾母便命:“去见你娘来。”宝玉即转身去了。一时回来,再看,已换了冠带:头上周围一转的短发,都结成小辫,红丝结束,共攒至顶中胎发,总编一根大辫,黑亮如漆,从顶至梢,一串四颗大珠,用金八宝坠角,身上穿著银红撒花半旧大袄,仍旧带著项圈、宝玉、寄名锁、护身符等物,下面半露松花撒花绫裤腿,锦边弹墨袜,厚底大红鞋。越显得面如敷粉,唇若施脂,转盼多情,语言常笑。天然一段风骚,全在眉梢,平生万种情思,悉堆眼角。看其外貌最是极好,却难知其底细。后人有《西江月》二词,批宝玉极恰,\begin{note}甲眉:二词更妙。最可厌野史“貌如潘安”“才如子建”等语。\end{note}其词曰:
\end{parag}


\begin{poem}
    \begin{pl}无故寻愁觅恨,有时似傻如狂。\end{pl}
    \begin{pl}纵然生得好皮囊,腹内原来草莽。\end{pl}

    \begin{pl}潦倒不通世务,愚顽怕读文章。\end{pl}
    \begin{pl}行为偏僻性乖张,那管世人诽谤!\end{pl}

    \begin{pl}富贵不知乐业,贫穷难耐凄凉。\end{pl}
    \begin{pl}可怜辜负好韶光,于国于家无望。\end{pl}

    \begin{pl}天下无能第一,古今不肖无双。\end{pl}
    \begin{pl}寄言纨绔与膏粱,莫效此儿形状!\end{pl}
    \begin{note}甲眉:末二语最紧要。只是纨绔膏粱,亦未必不见笑我玉卿。可知能效一二者,亦必不是蠢然纨绔矣。\end{note}
\end{poem}


\begin{parag}
    贾母因笑道:“外客未见,就脱了衣裳,还不去见你妹妹!”宝玉早已看见多了一个姊妹,便料定是林姑妈之女,忙来作揖。厮见毕归坐,细看形容,\begin{note}甲眉:又从宝玉目中细写一黛玉,直画一美人图。\end{note}与众各别:两弯似蹙非蹙罥烟眉,\begin{note}甲侧:奇眉妙眉,奇想妙想。\end{note}一双似泣非泣含露目。\begin{note}甲侧:奇目妙目,奇想妙想。\end{note}态生两靥之愁,娇袭一身之病。泪光点点,娇喘微微。闲静时如姣花照水,行动处似弱柳扶风。\begin{note}甲侧:至此八句是宝玉眼中。\end{note}心较比干多一窍,\begin{note}甲侧:此一句是宝玉心中。甲眉:更奇妙之至!多一窍固是好事,然未免偏僻了,所谓“过犹不及”也。\end{note}病如西子胜三分。\begin{note}甲侧:此十句定评,直抵一赋。甲眉:不写衣裙妆饰,正是宝玉眼中不屑之物,故不曾看见。黛玉之举止容貌,亦是宝玉眼中看、心中评。若不是宝玉,断不能知黛玉是何等品貌。\end{note}宝玉看罢,因笑\begin{note}甲眉:黛玉见宝玉写一“惊”字,宝玉见黛玉写一“笑”字,一存于中,一发乎外,可见文于下笔必推敲的准稳,方才用字。\end{note}道:\begin{note}甲侧:看他第一句是何话。\end{note}“这个妹妹我曾见过的。”\begin{note}甲侧:疯话。与黛玉同心,却是两样笔墨。观此则知玉卿心中有则说出,一毫宿滞皆无。\end{note}贾母笑道:“可又是胡说,你又何曾见过他?”宝玉笑道:“虽然未曾见过他,然我看著面善,心里就算是旧相识,\begin{note}甲侧:一见便作如是语,宜乎王夫人谓之疯疯傻傻也。\end{note}今日只作远别重逢,亦未为不可。”\begin{note}甲侧:妙极奇语,全作如是等语。无怪人谓曰痴狂。\end{note}贾母笑道:“更好,更好。\begin{note}甲侧:作小儿语瞒过世人亦可。\end{note}若如此,更相和睦了。”\begin{note}甲侧:亦是真话。\end{note}宝玉便走近黛玉身边坐下,又细细打谅一番,\begin{note}甲侧:与黛玉两次打谅一对。\end{note}因问:“妹妹可曾读书?”\begin{note}甲侧:自己不读书,却问到人,妙!\end{note}黛玉道:“不曾读,只上了一年学,些须认得几个字。”宝玉又道:“妹妹尊名是那两个字?”黛玉便说了名。宝玉又问表字,黛玉道:“无字。”宝玉笑道:“我送妹妹一妙字,莫若‘颦颦’二字极妙。”探春\begin{note}甲侧:写探春。\end{note}便问何出。宝玉道:“《古今人物通考》上说:‘西方有石名黛,可代画眉之墨。’况这林妹妹眉尖若蹙,用取这两个字,岂不两妙!”探春笑道:“只恐又是你的杜撰。”宝玉笑道:“除《四书》外,杜撰的太多,偏只我是杜撰不成?”\begin{note}甲侧:如此等语,焉得怪彼世人谓之怪?只瞒不过批书者。\end{note}又问黛玉:“可也有玉没有?”\begin{note}甲侧:奇极怪极,痴极愚极,焉得怪人目为痴哉?\end{note}众人不解其语,黛玉便忖度著:“因他有玉,故问我有也无。”\begin{note}甲眉:奇之至,怪之至,又忽将黛玉亦写成一极痴女子,观此初会二人之心,则可知以后之事矣。\end{note}因答道:“我没有那个。想来那玉是一件罕物,岂能人人有的。”宝玉听了,登时发作起痴狂病来,摘下那玉,就狠命摔去,\begin{note}甲侧:试问石兄:此一摔,比在青埂峰下萧然坦卧何如?\end{note}骂道:“什么罕物,连人之高低不择,还说‘通灵’不‘通灵’呢!我也不要这劳什子了!”吓的众人一拥争去拾玉。贾母急的搂了宝玉道:“孽障!\begin{note}甲侧:如闻其声,恨极语却是疼极语。\end{note}你生气,要打骂人容易,何苦摔那命根子!”\begin{note}甲侧:一字一千斤重。\end{note}宝玉满面泪痕泣\begin{note}甲侧:千奇百怪,不写黛玉泣,却反先写宝玉泣。\end{note}道:“家里姐姐妹妹都没有,单我有,我说没趣,如今来了这们一个神仙似的妹妹也没有,可知这不是个好东西。”\begin{note}甲眉:“不是冤家不聚头”第一场也。\end{note}贾母忙哄他道:“你这妹妹原有这个来的,因你姑妈去世时,舍不得你妹妹,无法处,遂将他的玉带了去了。一则全殉葬之礼,尽你妹妹之孝心,二则你姑妈之灵,亦可权作见了女儿之意。因此他只说没有这个,不便自己夸张之意。你如今怎比得他?还不好生慎重带上,仔细你娘知道了。”说著,便向丫鬟手中接来,亲与他带上。宝玉听如此说,想一想大有情理,也就不生别论了。\begin{note}甲侧:所谓小儿易哄,余则谓“君子可欺以其方”云。\end{note}
\end{parag}


\begin{parag}
    当下,奶娘来请问黛玉之房舍。贾母说:“今将宝玉挪出来,同我在套间暖阁儿里,把你林姑娘暂安置  纱橱里。等过了残冬,春天再与他们收拾房屋,另作一番安置罢。”宝玉道:“好祖宗,\begin{note}甲侧:跳出一小儿。\end{note}我就在  纱橱外的床上很妥当,何必又出来闹的老祖宗不得安静。”贾母想了一想说:“也罢了。”每人一个奶娘幷一个丫头照管,余者在外间上夜听唤。一面早有熙凤命人送了一顶藕合色花帐,幷几件锦被缎褥之类。
\end{parag}


\begin{parag}
    黛玉只带了两个人来:一个是自幼奶娘王嬷嬷,一个是十岁的小丫头,亦是自幼随身的,名唤作雪雁。\begin{note}甲侧:新雅不落套,是黛玉之文章也。\end{note}贾母见雪雁甚小,一团孩气,王嬷嬷又极老,料黛玉皆不遂心省力的,便将自己身边的一个二等丫头,名唤鹦哥\begin{note}甲眉:妙极!此等名号方是贾母之文章。最厌近之小说中,不论何处,满纸皆是红娘、小玉、娇红、香翠等俗字。\end{note}者与了黛玉。外亦如迎春等例,每人除自幼乳母外,另有四个教引嬷嬷,除贴身掌管钗钏盥沐两个丫鬟外,另有五六个洒扫房屋来往使役的小丫鬟。当下,王嬷嬷与鹦哥陪侍黛玉在碧纱橱内。宝玉之乳母李嬷嬷,幷大丫鬟名唤袭人\begin{note}甲侧:奇名新名,必有所出。\end{note}者,陪侍在外面大床上。
\end{parag}


\begin{parag}
    原来这袭人亦是贾母之婢,本名珍珠。\begin{note}甲侧:亦是贾母之文章。前鹦哥已伏下一鸳鸯,今珍珠又伏下一  珀矣。以下乃宝玉之文章。\end{note}贾母因溺爱宝玉,生恐宝玉之婢无竭力尽忠之人,素喜袭人心地纯良,克尽职任,遂与了宝玉。宝玉因知他本姓花,又曾见旧人诗句上有“花气袭人”之句,遂回明贾母,更名袭人。这袭人亦有些痴处:\begin{note}甲侧:只如此写又好极!最厌近之小说中,满纸“千伶百俐”“这妮子亦通文墨”等语。\end{note}伏侍贾母时,心中眼中只有一个贾母,如今服侍宝玉,心中眼中又只有一个宝玉。只因宝玉性情乖僻,每每规谏宝玉,心中著实忧郁。\begin{note}蒙侧:我读至此,不觉放声大哭。\end{note}
\end{parag}


\begin{parag}
    是晚,宝玉李嬷嬷已睡了,他见里面黛玉和鹦哥犹未安息,他自卸了妆,悄悄进来,笑问:“姑娘怎么还不安息?”黛玉忙让:“姐姐请坐。”袭人在床沿上坐了。鹦哥笑道:“林姑娘正在这里伤心,\begin{note}甲侧:可知前批不谬。\end{note}自己淌眼抹泪\begin{note}甲侧:黛玉第一次哭却如此写来。甲眉:前文反明写宝玉之哭,今却反如此写黛玉,几被作者瞒过。这是第一次算还,不知下剩还该多少?\end{note}的说:‘今儿才来,就惹出你家哥儿的狂病,倘或摔坏了那玉,岂不是因我之过!’\begin{note}甲侧:所谓宝玉知己,全用体贴功夫。蒙:我也心疼,岂独颦颦!\end{note}因此便伤心,我好容易劝好了。”袭人道:“姑娘快休如此,将来只怕比这个更奇怪的笑话儿还有呢!若为他这种行止,你多心伤感,只怕你伤感不了呢。快别多心!”\begin{note}蒙侧:后百十回黛玉之泪,总不能出此二语。“月上窗纱人到阶,窗上影儿先进来”,笔未到而境先到矣。[应知此非伤感,来还甘露水也。]\end{note}黛玉道:“姐姐们说的,我记著就是了。究竟那玉不知是怎么个来历?上面还有字迹?”袭人道:“连一家子也不知来历,上头还有现成的眼儿,听得说,落草时是从他口里掏出来的。\begin{note}甲侧:癞僧幻术亦太奇矣。蒙侧:天生带来美玉,有现成可穿之眼,岂不可爱,岂不可惜!\end{note}等我拿来你看便知。”黛玉忙止道:“罢了,此刻夜深,明日再看也不迟。”\begin{note}甲侧:总是体贴,不肯多事。蒙侧:他天生带来的美玉,他自己不爱惜,遇知己替他爱惜,连我看书的人也著实心疼不了,不觉背人一哭,以谢作者。\end{note}大家又叙了一回,方才安歇。
\end{parag}


\begin{parag}
    次日起来,省过贾母,因往王夫人处来,正值王夫人与熙凤在一处拆金陵来的书信看,又有王夫人之兄嫂处遣了两个媳妇来说话的。黛玉虽不知原委,探春等却都晓得是议论金陵城中所居的薛家姨母之子姨表兄薛蟠,倚财仗势,打死人命,现在应天府案下审理。如今母舅王子腾得了信息,故遣他家内的人来告诉这边,意欲唤取进京之意。
\end{parag}


\begin{parag}
    \begin{note}蒙:补不完的是离恨天,所余之石岂非离恨石乎。而绛珠之泪偏不因离恨而落,为惜其石而落。可见惜其石必惜其人,其人不自惜,而知己能不千方百计为之惜乎?所以绛珠之泪至死不干,万苦不怨。所谓求仁得仁,又何怨。悲夫!\end{note}
\end{parag}