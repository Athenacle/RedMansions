\chap{六十八}{苦尤娘賺入大觀園 酸鳳姐大鬧寧國府}


\begin{parag}
    \begin{note}蒙回前總評:餘讀《左氏》見鄭莊,讀《後漢》見魏武,謂古之大奸巨滑惟此爲最。今讀《石頭記》,又見鳳姐,作威作福,用柔用剛,站步高,留步寬,殺得死,救得活,天生此等人琢喪元氣不少!\end{note}
\end{parag}


\begin{parag}
    話說賈璉起身去後,偏值平安節度巡邊在外,約一個月方回。賈璉未得確信,只得住在下處等候。及至回來相見,將事辦妥,回程已是將兩個月的限了。
\end{parag}


\begin{parag}
    誰知鳳姐心下早已算定,只待賈璉前腳走了,回來便傳各色匠役,收拾東廂房三間,照依自己正室一樣裝飾陳設。至十四日便回明賈母王夫人,說十五日一早要到姑子廟進香去。只帶了平兒,豐兒,周瑞媳婦,旺兒媳婦四人,未曾上車,便將原故告訴了衆人。又吩咐衆男人,素衣素蓋,一徑前來。
\end{parag}


\begin{parag}
    興兒引路,一直到了二姐門前扣門。鮑二家的開了。興兒笑說:“快回二奶奶去,大奶奶來了。”鮑二家的聽了這句,頂梁骨走了真魂,忙飛進報與尤二姐。尤二姐雖也一驚,但已來了,只得以禮相見,於是忙整衣迎了出來。至門前,鳳姐方下車進來。尤二姐一看,只見頭上皆是素白銀器,身上月白緞襖,青緞披風,白綾素裙。眉彎柳葉,高吊兩梢,目橫丹鳳,神凝三角。俏麗若三春之桃,清潔若九秋之菊。周瑞旺兒二女人攙入院來。尤二姐陪笑忙迎上來萬福,張口便叫:“姐姐下降,不曾遠接,望恕倉促之罪。”說著便福了下來。鳳姐忙陪笑還禮不迭。二人攜手同入室中。
\end{parag}


\begin{parag}
    鳳姐上座,尤二姐命丫鬟拿褥子來便行禮,說:“奴家年輕,一從到了這裏之事,皆系家母和家姐商議主張。今日有幸相會,若姐姐不棄奴家寒微,凡事求姐姐的指示教訓。奴亦傾心吐膽,只伏侍姐姐。”說著,便行下禮去。鳳姐兒忙下座以禮相還,口內忙說:“皆因奴家婦人之見,一味勸夫慎重,不可在外眠花臥柳,恐惹父母擔憂。此皆是你我之癡心,怎奈二爺錯會奴意。眠花宿柳之事瞞奴或可,今娶姐姐二房之大事亦人家大禮,亦不曾對奴說。奴亦曾勸二爺早行此禮,以備生育。不想二爺反以奴爲那等嫉妒之婦,私自行此大事,並不說知。使奴有冤難訴,惟天地可表。前於十日之先奴已風聞,恐二爺不樂,遂不敢先說。今可巧遠行在外,故奴家親自拜見過,還求姐姐下體奴心,起動大駕,挪至家中。你我姊妹同居同處,彼此合心諫勸二爺,慎重世務,保養身體,方是大禮。若姐姐在外,奴在內,雖愚賤不堪相伴,奴心又何安。再者,使外人聞知,亦甚不雅觀。二爺之名也要緊,倒是談論奴家,奴亦不怨。所以今生今世奴之名節全在姐姐身上。那起下人小人之言,未免見我素日持家太嚴,背後加減些言語,自是常情。姐姐乃何等樣人物,豈可信真。若我實有不好之處,上頭三層公婆,中有無數姊妹妯娌,況賈府代名家,豈容我到今日。今日二爺私娶姐姐在外,若別人則怒,我則以爲幸。正是天地神佛不忍我被小人們誹謗,故生此事。我今來求姐姐進去和我一樣同居同處,同分同例,同侍公婆,同諫丈夫。喜則同喜,悲則同悲,情似親妹,和比骨肉。不但那起小人見了,自悔從前錯認了我,就是二爺來家一見,他作丈夫之人,心中也未免暗悔。所以姐姐竟是我的大恩人,使我從前之名一洗無餘了。若姐姐不隨奴去,奴亦情願在此相陪。奴願作妹子,每日伏侍姐姐梳頭洗面。只求姐姐在二爺跟前替我好言方便方便,容我一席之地安身,奴死也願意。”說著,便嗚嗚咽咽哭將起來。尤二姐見了這般,也不免滴下淚來。
\end{parag}


\begin{parag}
    二人對見了禮,分序座下。平兒忙也上來要見禮。尤二姐見他打扮不凡,舉止品貌不俗,料定是平兒,連忙親身挽住,只叫:“妹子快休如此,你我是一樣的人。”鳳姐忙也起身笑說:“折死他了!妹子只管受禮,他原是咱們的丫頭。以後快別如此。”說著,又命周家的從包袱裏取出四匹上色尺頭,四對金珠簪環爲拜禮。尤二姐忙拜受了。二人喫茶,對訴已往之事。鳳姐口內全是自怨自錯,“怨不得別人,如今只求姐姐疼我”等語。尤二姐見了這般,便認他作是個極好的人,小人不遂心誹謗主子亦是常理,故傾心吐膽,敘了一回,竟把鳳姐認爲知己。又見周瑞等媳婦在旁邊稱揚鳳姐素日許多善政,只是喫虧心太癡了,惹人怨,又說“已經預備了房屋,奶奶進去一看便知。”尤氏心中早已要進去同住方好,今又見如此,豈有不允之理,便說:“原該跟了姐姐去,只是這裏怎樣?”鳳姐兒道:“這有何難,姐姐的箱籠細軟只管著小廝搬了進去。這些粗笨貨要他無用,還叫人看著。姐姐說誰妥當就叫誰在這裏。”尤二姐忙說:“今日既遇見姐姐,這一進去,凡事只憑姐姐料理。我也來的日子淺,也不曾當過家,世事不明白,如何敢作主。這幾件箱籠拿進去罷。我也沒有什麼東西,那也不過是二爺的。”鳳姐聽了,便命周瑞家的記清,好生看管著抬到東廂房去。於是催著尤二姐穿戴了,二人攜手上車,又同坐一處,又悄悄的告訴他:“我們家的規矩大。這事老太太一概不知,倘或知二爺孝中娶你,管把他打死了。如今且別見老太太,太太。我們有一個花園子極大,姊妹住著,容易沒人去的。你這一去且在園裏住兩天,等我設個法子回明白了,那時再見方妥。”尤二姐道:“任憑姐姐裁處。”那些跟車的小廝們皆是預先說明的,如今不去大門,只奔後門而來。
\end{parag}


\begin{parag}
    下了車,趕散衆人。鳳姐便帶尤氏進了大觀園的後門,來到李紈處相見了。彼時大觀園中十停人已有九停人知道了,今忽見鳳姐帶了進來,引動多人來看問。尤二姐一一見過。衆人見他標緻和悅,無不稱揚。鳳姐一一的吩咐了衆人:“都不許在外走了風聲,若老太太、太太知道,我先叫你們死。”園中婆子丫鬟都素懼鳳姐的,又系賈璉國孝家孝中所行之事,知道關係非常,都不管這事。鳳姐悄悄的求李紈收養幾日,“等回明瞭,我們自然過去的。”李紈見鳳姐那邊已收拾房屋,況在服中,不好倡揚,自是正理,只得收下權住。鳳姐又變法將他的丫頭一概退出,又將自己的一個丫頭送他使喚。暗暗吩咐園中媳婦們:“好生照看著他。若有走失逃亡,一概和你們算帳。”自己又去暗中行事。閤家之人都暗暗納罕的說:“看他如何這等賢惠起來了。”
\end{parag}


\begin{parag}
    那尤二姐得了這個所在,又見園中姊妹各各相好,倒也安心樂業的自爲得其所矣。誰知三日之後,丫頭善姐便有些不服使喚起來。尤二姐因說:“沒了頭油了,你去回聲大奶奶拿些來。”善姐便道:“二奶奶,你怎麼不知好歹沒眼色。我們奶奶天天承應了老太太,又要承應這邊太太那邊太太。這些妯娌姊妹,上下幾百男女,天天起來,都等他的話。一日少說,大事也有一二十件,小事還有三五十件。外頭的從娘娘算起,以及王公侯伯家多少人情客禮,家裏又有這些親友的調度。銀子上千錢上萬,一日都從他一個手一個心一個口裏調度,那裏爲這點子小事去煩瑣他。我勸你能著些兒罷。咱們又不是明媒正娶來的,這是他亙古少有一個賢良人才這樣待你,若差些兒的人,聽見了這話,吵嚷起來,把你丟在外,死不死,生不生,你又敢怎樣呢!”一席話,說的尤氏垂了頭,自爲有這一說,少不得將就些罷了。那善姐漸漸連飯也怕端來與他喫,或早一頓,或晚一頓,所拿來之物,皆是剩的。尤二姐說過兩次,他反先亂叫起來。尤二姐又怕人笑他不安分,少不得忍著。隔上五日八日見鳳姐一面,那鳳姐卻是和容悅色,滿嘴裏姐姐不離口。又說:“倘有下人不到之處,你降不住他們,只管告訴我,我打他們。”又罵丫頭媳婦說: “我深知你們,軟的欺,硬的怕,背開我的眼,還怕誰。倘或二奶奶告訴我一個不字,我要你們的命。”尤氏見他這般的好心,思想“既有他,何必我又多事。下人不知好歹,也是常情。我若告了,他們受了委屈,反叫人說我不賢良。”因此反替他們遮掩。
\end{parag}


\begin{parag}
    鳳姐一面使旺兒在外打聽細事,這尤二姐之事皆已深知。原來已有了婆家的,女婿現在才十九歲,成日在外嫖賭,不理生業,傢俬花盡,父親攆他出來,現在賭錢廠存身。父親得了尤婆十兩銀子退了親的,這女婿尚不知道。原來這小夥子名叫張華。鳳姐都一一盡知原委,便封了二十兩銀子與旺兒,悄悄命他將張華勾來養活,著他寫一張狀子,只管往有司衙門中告去,就告璉二爺“國孝家孝之中,背旨瞞親,仗財依勢,強逼退親,停妻再娶”等語。這張華也深知利害,先不敢造次。旺兒回了鳳姐,鳳姐氣的罵:“癩狗扶不上牆的種子。你細細的說給他,便告我們家謀反也沒事的。不過是借他一鬧,大家沒臉。若告大了,我這裏自然能夠平息的。”旺兒領命,只得細說與張華。鳳姐又吩咐旺兒:“他若告了你,你就和他對詞去。”如此如此,這般這般,“我自有道理。”旺兒聽了有他做主,便又命張華狀子上添上自己,說:“你只告我來往過付,一應調唆二爺做的。”張華便得了主意,和旺兒商議定了,寫了一紙狀子,次日便往都察院喊了冤。
\end{parag}


\begin{parag}
    察院坐堂看狀,見是告賈璉的事,上面有家人旺兒一人,只得遣人去賈府傳旺兒來對詞。青衣不敢擅入,只命人帶信。那旺兒正等著此事,不用人帶信,早在這條街上等候。見了青衣,反迎上去笑道:“起動衆位兄弟,必是兄弟的事犯了。說不得,快來套上。”衆青衣不敢,只說:“你老去罷,別鬧了。”於是來至堂前跪了。察院命將狀子與他看。旺兒故意看了一遍,碰頭說道:“這事小的盡知,小的主人實有此事。但這張華素與小的有仇,故意攀扯小的在內。其中還有別人,求老爺再問。”張華碰頭說:“雖還有人,小的不敢告他,所以只告他下人。”旺兒故意急的說:“糊塗東西,還不快說出來!這是朝廷公堂之上,憑是主子,也要說出來。”張華便說出賈蓉來。察院聽了無法,只得去傳賈蓉。鳳姐又差了慶兒暗中打聽,告了起來,便忙將王信喚來,告訴他此事,命他託察院只虛張聲勢警唬而已,又拿了三百銀子與他去打點。是夜王信到了察院私第,安了根子。那察院深知原委,收了贓銀。次日回堂,只說張華無賴,因拖欠了賈府銀兩,枉捏虛詞,誣賴良人。都察院又素與王子騰相好,王信也只到家說了一聲,況是賈府之人,巴不得了事,便也不提此事,且都收下,只傳賈蓉對詞。
\end{parag}


\begin{parag}
    且說賈蓉等正忙著賈珍之事,忽有人來報信,說有人告你們如此如此,這般這般,快作道理。賈蓉慌了,忙來回賈珍。賈珍說:“我防了這一著,只虧他大膽子。”即刻封了二百銀子著人去打點察院,又命家人去對詞。正商議之間,人報:“西府二奶奶來了。”賈珍聽了這個,倒吃了一驚,忙要同賈蓉藏躲。不想鳳姐進來了,說:“好大哥哥,帶著兄弟們乾的好事!”賈蓉忙請安,鳳姐拉了他就進來。賈珍還笑說:“好生伺候你姑娘,吩咐他們殺牲口備飯。”說了,忙命備馬,躲往別處去了。
\end{parag}


\begin{parag}
    這裏鳳姐兒帶著賈蓉走來上房,尤氏正迎了出來,見鳳姐氣色不善,忙笑說:“什麼事這等忙?”鳳姐照臉一口吐沫啐道:“你尤家的丫頭沒人要了,偷著只往賈家送!難道賈家的人都是好的,普天下死絕了男人了!你就願意給,也要三媒六證,大家說明,成個體統纔是。你痰迷了心,脂油蒙了竅,國孝家孝兩重在身,就把個人送來了。這會子被人家告我們,我又是個沒腳蟹,連官場中都知道我利害喫醋,如今指名提我,要休我。我來了你家,幹錯了什麼不是,你這等害我?或是老太太,太太有了話在你心裏,使你們做這圈套,要擠我出去。如今咱們兩個一同去見官,分證明白。回來咱們公同請了合族中人,大家覿面說個明白。給我休書,我就走路。”一面說,一面大哭,拉著尤氏,只要去見官。急的賈蓉跪在地下碰頭,只求“姑娘嬸子息怒。”鳳姐兒一面又罵賈蓉:“天雷劈腦子五鬼分屍的沒良心的種子!不知天有多高,地有多厚,成日家調三窩四,幹出這些沒臉面沒王法敗家破業的營生。你死了的娘陰靈也不容你,祖宗也不容,還敢來勸我!”哭罵著揚手就打。賈蓉忙磕頭有聲說:“嬸子別動氣,仔細手,讓我自己打。嬸子別動氣。”說著,自己舉手左右開弓自己打了一頓嘴巴子,又自己問著自己說:“以後可再顧三不顧四的混管閒事了?以後還單聽叔叔的話不聽嬸子的話了?”衆人又是勸,又要笑,又不敢笑。
\end{parag}


\begin{parag}
    鳳姐兒滾到尤氏懷裏,嚎天地,大放悲聲,只說:“給你兄弟娶親我不惱。爲什麼使他違旨背親,將混帳名兒給我背著?咱們只去見官,省得捕快皁隸來。再者咱們只過去見了老太太,太太和衆族人,大家公議了,我既不賢良,又不容丈夫娶親買妾,只給我一紙休書,我即刻就走。你妹妹我也親身接來家,生怕老太太,太太生氣,也不敢回,現在三茶六飯金奴銀婢的住在園裏。我這裏趕著收拾房子,一樣和我的道理,只等老太太知道了。原說接過來大家安分守己的,我也不提舊事了。誰知又有了人家的。不知你們乾的什麼事,我一概又不知道。如今告我,我昨日急了,縱然我出去見官,也丟的是你賈家的臉,少不得偷把太太的五百兩銀子去打點。如今把我的人還鎖在那裏。”說了又哭,哭了又罵,後來放聲大哭起祖宗爹媽來,又要尋死撞頭。把個尤氏揉搓成一個麪糰,衣服全是眼淚鼻涕,並無別語,只罵賈蓉:“孽障種子!和你老子作的好事!我就說不好的。”鳳姐兒聽說,哭著兩手搬尤氏的臉緊對相問道:“你發昏了?你的嘴裏難道有茄子塞著?不然他們給你嚼子銜上了?爲什麼你不告訴我去?你若告訴了我,這會子平安不了?怎得經官動府,鬧到這步田地,你這會子還怨他們。自古說:‘妻賢夫禍少,表壯不如裏壯。’你但凡是個好的,他們怎得鬧出這些事來!你又沒才幹,又沒口齒,鋸了嘴子的葫蘆,就只會一味瞎小心圖賢良的名兒。總是他們也不怕你,也不聽你。”說著啐了幾口。尤氏也哭道:“何曾不是這樣。你不信問問跟的人,我何曾不勸的,也得他們聽。叫我怎麼樣呢,怨不得妹妹生氣,我只好聽著罷了。”
\end{parag}


\begin{parag}
    衆姬妾丫鬟媳婦已是烏壓壓跪了一地,陪笑求說:“二奶奶最聖明的。雖是我們奶奶的不是,奶奶也作踐的夠了。當著奴才們,奶奶們素日何等的好來,如今還求奶奶給留臉。”說著,捧上茶來。鳳姐也摔了,一面止了哭挽頭髮,又哭罵賈蓉:“出去請大哥哥來。我對面問他,親大爺的孝才五七,侄兒娶親,這個禮我竟不知道。我問問,也好學著日後教導子侄的。”賈蓉只跪著磕頭,說:“這事原不與父母相干,都是兒子一時吃了屎,調唆叔叔作的。我父親也並不知道。如今我父親正要商量接太爺出殯,嬸子若鬧起來,兒子也是個死。只求嬸子責罰兒子,兒子謹領。這官司還求嬸子料理,兒子竟不能幹這大事。嬸子是何等樣人,豈不知俗語說的 ‘胳膊只折在袖子裏’。兒子糊塗死了,既作了不肖的事,就同那貓兒狗兒一般。嬸子既教訓,就不和兒子一般見識的,少不得還要嬸子費心費力將外頭的壓住了纔好。原是嬸子有這個不肖的兒子,既惹了禍,少不得委屈,還要疼兒子。”說著,又磕頭不絕。
\end{parag}


\begin{parag}
    鳳姐見他母子這般,也再難往前施展了,只得又轉過了一副形容言談來,與尤氏反陪禮說:“我是年輕不知事的人,一聽見有人告訴了,把我嚇昏了,不知方纔怎樣得罪了嫂子。可是蓉兒說的‘胳膊折了往袖子裏藏’,少不得嫂子要體諒我。還要嫂子轉替哥哥說了,先把這官司按下去纔好。”尤氏賈蓉一齊都說:“嬸子放心,橫豎一點兒連累不著叔叔。嬸子方纔說用過了五百兩銀子,少不得我娘兒們打點五百兩銀子與嬸子送過去,好補上的,不然豈有反教嬸子又添上虧空之名,越發我們該死了。但還有一件,老太太、太太們跟前嬸子還要周全方便,別提這些話方好。”鳳姐兒又冷笑道:“你們饒壓著我的頭幹了事,這會子反哄著我替你們周全。我雖然是個呆子,也呆不到如此。嫂子的兄弟是我的丈夫,嫂子既怕他絕後,我豈不更比嫂子更怕絕後。嫂子的令妹就是我的妹子一樣。我一聽見這話,連夜喜歡的連覺也睡不成,趕著傳人收拾了屋子,就要接進來同住。倒是奴才小人的見識,他們倒說:‘奶奶太好性了。若是我們的主意,先回了老太太、太太,看是怎樣,再收拾房子去接也不遲。’我聽了這話,教我要打要罵的,纔不言語。誰知偏不稱我的意,偏打我的嘴,半空裏又跑出一個張華來告了一狀。我聽見了,嚇的兩夜沒合眼兒,又不敢聲張,只得求人去打聽這張華是什麼人,這樣大膽。打聽了兩日,誰知是個無賴的花子。我年輕不知事,反笑了,說:‘他告什麼?’倒是小子們說:‘原是二奶奶許了他的。他如今正是急了,凍死餓死也是個死,現在有這個理他抓著,縱然死了,死的倒比凍死餓死還值些。怎麼怨的他告呢。這事原是爺做的太急了。國孝一層罪,家孝一層罪,背著父母私娶一層罪,停妻再娶一層罪。俗語說:”拼著一身剮,敢把皇帝拉下馬。“他窮瘋了的人,什麼事作不出來,況且他又拿著這滿理,不告等請不成。’嫂子說,我便是個韓信張良,聽了這話,也把智謀嚇回去了。你兄弟又不在家,又沒個商議,少不得拿錢去墊補,誰知越使錢越被人拿住了刀靶,越發來訛。我是耗子尾上長瘡──多少膿血兒。所以又急又氣,少不得來找嫂子。”賈氏賈蓉不等說完,都說:“不必操心,自然要料理的。”賈蓉又道:“那張華不過是窮急,故舍了命才告。咱們如今想了一個法兒,竟許他些銀子,只叫他應了妄告不實之罪,咱們替他打點完了官司。他出來時再給他些個銀子就完了。”鳳姐兒笑道:“好孩子,怨不得你顧一不顧二的作這些事出來。原來你竟糊塗。若你說得這話,他暫且依了,且打出官司來又得了銀子,眼前自然了事。這些人既是無賴之徒,銀子到手一旦光了,他又尋事故訛詐。倘又叨登起來這事,咱們雖不怕,也終擔心。擱不住他說既沒毛病爲什麼反給他銀子,終久是不了之局。”賈蓉原是個明白人,聽如此一說,便笑道:“我還有個主意,‘來是是非人,去是是非者’,這事還得我了纔好。如今我竟去問張華個主意,或是他定要人,或是他願意了事得錢再娶。他若說一定要人,少不得我去勸我二姨,叫他出來仍嫁他去,若說要錢,我們這裏少不得給他。”鳳姐兒忙道:“雖如此說,我斷捨不得你姨娘出去,我也斷不肯使他去。好侄兒,你若疼我,只能可多給他錢爲是。”賈蓉深知鳳姐口雖如此,心卻是巴不得只要本人出來,他卻做賢良人。如今怎說怎依。鳳姐兒歡喜了,又說:“外頭好處了,家裏終久怎麼樣?你也同我過去回明纔是。” 仁嫌只了,拉鳳姐討主意如何撒謊纔好。鳳姐冷笑道:“既沒這本事,誰叫你幹這事了。這會子又這個腔兒,我又看不上。待要不出個主意,我又是個心慈面軟的人,憑人撮弄我,我還是一片癡心。說不得讓我應起來。如今你們只別露面,我只領了你妹妹去與老太太,太太們磕頭,只說原系你妹妹,我看上了很好。正因我不大生長,原說買兩個人放在屋裏的,今既見你妹妹很好,而又是親上做親的,我願意娶來做二房。皆因家中父母姊妹新近一概死了,日子又艱難,不能度日,若等百日之後,無奈無家無業,實難等得。我的主意接了進來,已經廂房收拾了出來暫且住著,等滿了服再圓房。仗著我不怕臊的臉,死活賴去,有了不是,也尋不著你們了。你們母子想想,可使得?”尤氏賈蓉一齊笑說:“到底是嬸子寬洪大量,足智多謀。等事妥了,少不得我們娘兒們過去拜謝。”尤氏忙命丫鬟們伏侍鳳姐梳妝洗臉,又擺酒飯,親自遞酒揀菜。
\end{parag}


\begin{parag}
    鳳姐也不多坐,執意就走了。進園中將此事告訴與尤二姐,又說我怎麼操心打聽,又怎麼設法子,須得如此如此方救下衆人無罪,少不得我去拆開這魚頭,大家纔好。不知端詳,且聽下回分解。
\end{parag}


\begin{parag}
    \begin{note}蒙回後總評:人謂“鬧寧府”一節極兇猛,“賺二姐”一節極和藹,吾謂“鬧寧府”情有可恕,“賺二姐”法不容誅,“鬧寧府”聲聲是淚,“賺二姐”字字皆針(左金右夅)。\end{note}
\end{parag}
