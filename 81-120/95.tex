\chap{九十五}{因讹成实元妃薨逝 以假混真宝玉疯颠}



\begin{parag}
    话说焙茗在门口和小丫头子说宝玉的玉有了,那小丫头急忙回来告诉宝玉。众人听了,都推着宝玉出去问他,众人在廊下听着。宝玉也觉放心,便走到门口问道:“你那里得了?快拿来。”焙茗道:“拿是拿不来的,还得托人做保去呢。”宝玉道:“你快说是怎么得的,我好叫人取去。”焙茗道:“我在外头知道林爷爷去测字,我就跟了去。我听见说在当铺里找,我没等他说完,便跑到几个当铺里去。我比给他们瞧,有一家便说有。我说给我罢,那铺子里要票子。我说当多少钱,他说三百钱的也有,五百钱的也有。前儿有一个人拿这么一块玉当了三百钱去,今儿又有人也拿了一块玉当了五百钱去。”宝玉不等说完,便道:“你快拿三百五百钱去取了来,我们挑着看是不是。”里头袭人便啐道:“二爷不用理他。我小时候儿听见我哥哥常说,有些人卖那些小玉儿,没钱用便去当。想来是家家当铺里有的。”众人正在听得诧异,被袭人一说,想了一想,倒大家笑起来,说:“快叫二爷进来罢,不用理那糊涂东西了。他说的那些玉,想来不是正经东西。”
\end{parag}


\begin{parag}
    宝玉正笑着,只见岫烟来了。原来岫烟走到栊翠庵见了妙玉,不及闲话,便求妙玉扶乩。妙玉冷笑几声,说道:“我与姑娘来往,为的是姑娘不是势利场中的人。今日怎么听了那里的谣言,过来缠我。况且我并不晓得什么叫扶乩。”说着,将要不理。岫烟懊悔此来,知他脾气是这么着的,“一时我已说出,不好白回去,又不好与他质证他会扶乩的话。”只得陪着笑将袭人等性命关系的话说了一遍,见妙玉略有活动,便起身拜了几拜。妙玉叹道:“何必为人作嫁。但是我进京以来,素无人知,今日你来破例,恐将来缠绕不休。”岫烟道:“我也一时不忍,知你必是慈悲的。便是将来他人求你,愿不愿在你,谁敢相强。”妙玉笑了一笑,叫道婆焚香,在箱子里找出沙盘乩架,书了符,命岫烟行礼,祝告毕,起来同妙玉扶着乩。不多时,只见那仙乩疾书道:
\end{parag}


\begin{qute2sp}
    噫!来无迹,去无踪,青埂峰下倚古松。欲追寻,山万重,入我门来一笑逢。
\end{qute2sp}


\begin{parag}
    书毕,停了乩。岫烟便问请是何仙,妙玉道:“请的是拐仙。”岫烟录了出来,请教妙玉解识。妙玉道:“这个可不能,连我也不懂。你快拿去,他们的聪明人多着哩。”岫烟只得回来。进入院中,各人都问怎么样了。岫烟不及细说,便将所录乩语递与李纨。众姊妹及宝玉争看,都解的是:“一时要找是找不着的,然而丢是丢不了的,不知几时不找便出来了。但是青埂峰不知在那里?”李纨道:“这是仙机隐语。咱们家里那里跑出青埂峰来,必是谁怕查出,撂在有松树的山子石底下,也未可定。独是‘入我门来’这句,到底是入谁的门呢?”黛玉道:“不知请的是谁!”岫烟道:“拐仙。”探春道:“若是仙家的门,便难入了。”
\end{parag}


\begin{parag}
    袭人心里着忙,便捕风捉影的混找,没一块石底下不找到,只是没有。回到院中,宝玉也不问有无,只管傻笑。麝月着急道:“小祖宗!你到底是那里丢的,说明了,我们就是受罪也在明处啊。”宝玉笑道:“我说外头丢的,你们又不依。你如今问我,我知道么!”李纨探春道:“今儿从早起闹起,已到三更来的天了。你瞧林妹妹已经掌不住,各自去了。我们也该歇歇儿了,明儿再闹罢。”说着,大家散去。宝玉即便睡下。可怜袭人等哭一回,想一回,一夜无眠。暂且不提。
\end{parag}


\begin{parag}
    且说黛玉先自回去,想起金石的旧话来,反自喜欢,心里说道:“和尚道士的话真个信不得。果真金玉有缘,宝玉如何能把这玉丢了呢。或者因我之事,拆散他们的金玉,也未可知。”想了半天,更觉安心,把这一天的劳乏竟不理会,重新倒看起书来。紫鹃倒觉身倦,连催黛玉睡下。黛玉虽躺下,又想到海棠花上,说“这块玉原是胎里带来的,非比寻常之物,来去自有关系。若是这花主好事呢,不该失了这玉呀?看来此花开的不祥,莫非他有不吉之事?”不觉又伤起心来。又转想到喜事上头,此花又似应开,此玉又似应失,如此一悲一喜,直想到五更,方睡着。
\end{parag}


\begin{parag}
    次日,王夫人等早派人到当铺里去查问,凤姐暗中设法找寻。一连闹了几天,总无下落。还喜贾母贾政未知。袭人等每日提心吊胆,宝玉也好几天不上学,只是怔怔的,不言不语,没心没绪的。王夫人只知他因失玉而起,也不大着意。那日正在纳闷,忽见贾琏进来请安,嘻嘻的笑道:“今日听得军机贾雨村打发人来告诉二老爷说,舅太爷升了内阁大学士,奉旨来京,已定明年正月二十日宣麻。有三百里的文书去了,想舅太爷昼夜趱行,半个多月就要到了。侄儿特来回太太知道。”王夫人听说,便欢喜非常。正想娘家人少,薛姨妈家又衰败了,兄弟又在外任,照应不着。今日忽听兄弟拜相回京,王家荣耀,将来宝玉都有倚靠,便把失玉的心又略放开些了。天天专望兄弟来京。忽一天,贾政进来,满脸泪痕,喘吁吁的说道:“你快去禀知老太太,即刻进宫。不用多人的,是你伏侍进去。因娘娘忽得暴病,现在太监在外立等,他说太医院已经奏明痰厥,不能医治。”王夫人听说,便大哭起来。贾政道:“这不是哭的时候,快快去请老太太,说得宽缓些,不要吓坏了老人家。”贾政说着,出来吩咐家人伺候。王夫人收了泪,去请贾母,只说元妃有病,进去请安。贾母念佛道:“怎么又病了!前番吓的我了不得,后来又打听错了。这回情愿再错了也罢。”王夫人一面回答,一面催鸳鸯等开箱取衣饰穿戴起来。王夫人赶着回到自己房中,也穿戴好了,过来伺候。一时出厅上轿进宫。不题。
\end{parag}


\begin{parag}
    且说元春自选了凤藻宫后,圣眷隆重,身体发福,未免举动费力。每日起居劳乏,时发痰疾。因前日侍宴回宫,偶沾寒气,勾起旧病。不料此回甚属利害,竟至痰气壅塞,四肢厥冷。一面奏明,即召太医调治。岂知汤药不进,连用通关之剂,并不见效。内官忧虑,奏请预办后事。所以传旨命贾氏椒房进见。贾母王夫人遵旨进宫,见元妃痰塞口涎,不能言语,见了贾母,只有悲泣之状,却少眼泪。贾母进前请安,奏些宽慰的话。少时贾政等职名递进,宫嫔传奏,元妃目不能顾,渐渐脸色改变。内宫太监即要奏闻,恐派各妃看视,椒房姻戚未便久羁,请在外宫伺候。贾母王夫人怎忍便离,无奈国家制度,只得下来,又不敢啼哭,惟有心内悲感。朝门内官员有信。不多时,只见太监出来,立传钦天监。贾母便知不好,尚未敢动。稍刻,小太监传谕出来说:“贾娘娘薨逝。”是年甲寅年十二月十八日立春,元妃薨日是十二月十九日,已交卯年寅月,存年四十三岁。贾母含悲起身,只得出宫上轿回家。贾政等亦已得信,一路悲戚。到家中,邢夫人,李纨,凤姐,宝玉等出厅分东西迎着贾母请了安,并贾政王夫人请安,大家哭泣。不题。
\end{parag}


\begin{parag}
    次日早起,凡有品级的,按贵妃丧礼,进内请安哭临。贾政又是工部,虽按照仪注办理,未免堂上又要周旋他些,同事又要请教他,所以两头更忙,非比从前太后与周妃的丧事了。但元妃并无所出,惟谥曰”贤淑贵妃”。此是王家制度,不必多赘。只讲贾府中男女天天进宫,忙的了不得。幸喜凤姐儿近日身子好些,还得出来照应家事,又要预备王子腾进京接风贺喜。凤姐胞兄王仁知道叔叔入了内阁,仍带家眷来京。凤姐心里喜欢,便有些心病,有这些娘家的人,也便撂开,所以身子倒觉比前好了些。王夫人看见凤姐照旧办事,又把担子卸了一半,又眼见兄弟来京,诸事放心,倒觉安静些。独有宝玉原是无职之人,又不念书,代儒学里知他家里有事,也不来管他,贾政正忙,自然没有空儿查他。想来宝玉趁此机会,竟可与姊妹们天天畅乐,不料他自失了玉后,终日懒怠走动,说话也糊涂了。并贾母等出门回来,有人叫他去请安,便去,没人叫他,他也不动。袭人等怀着鬼胎,又不敢去招惹他,恐他生气。每天茶饭,端到面前便吃,不来也不要。袭人看这光景不象是有气,竟象是有病的。袭人偷着空儿到潇湘馆告诉紫鹃,说是“二爷这么着,求姑娘给他开导开导。”紫鹃虽即告诉黛玉,只因黛玉想着亲事上头一定是自己了,如今见了他,反觉不好意思:“若是他来呢,原是小时在一处的,也难不理他,若说我去找他,断断使不得。”所以黛玉不肯过来。袭人又背地里去告诉探春。那知探春心里明明知道海棠开得怪异,“宝玉”失的更奇,接连着元妃姐姐薨逝,谅家道不祥,日日愁闷,那有心肠去劝宝玉。况兄妹们男女有别,只好过来一两次。宝玉又终是懒懒的,所以也不大常来。
\end{parag}


\begin{parag}
    宝钗也知失玉。因薛姨妈那日应了宝玉的亲事,回去便告诉了宝钗。薛姨妈还说:“虽是你姨妈说了,我还没有应准,说等你哥哥回来再定。你愿意不愿意?”宝钗反正色的对母亲道:“妈妈这话说错了。女孩儿家的事情是父母做主的。如今我父亲没了,妈妈应该做主的,再不然问哥哥。怎么问起我来?”所以薛姨妈更爱惜他,说他虽是从小娇养惯的,却也生来的贞静,因此在他面前,反不提起宝玉了。宝钗自从听此一说,把”宝玉”两个字自然更不提起了。如今虽然听见失了玉,心里也甚惊疑,倒不好问,只得听旁人说去,竟象不与自己相干的。只有薛姨妈打发丫头过来了好几次问信。因他自己的儿子薛蟠的事焦心,只等哥哥进京便好为他出脱罪名,又知元妃已薨,虽然贾府忙乱,却得凤姐好了,出来理家,也把贾家的事撂开了。只苦了袭人,虽然在宝玉跟前低声下气的伏侍劝慰,宝玉竟是不懂,袭人只有暗暗的着急而已。
\end{parag}


\begin{parag}
    过了几日,元妃停灵寝庙,贾母等送殡去了几天。岂知宝玉一日呆似一日,也不发烧,也不疼痛,只是吃不象吃,睡不象睡,甚至说话都无头绪。那袭人麝月等一发慌了,回过凤姐几次。凤姐不时过来,起先道是找不着玉生气,如今看他失魂落魄的样子,只有日日请医调治。煎药吃了好几剂,只有添病的,没有减病的。及至问他那里不舒服,宝玉也不说出来。直至元妃事毕,贾母惦记宝玉,亲自到园看视。王夫人也随过来。袭人等忙叫宝玉接去请安。宝玉虽说是病,每日原起来行动,今日叫他接贾母去,他依然仍是请安,惟是袭人在旁扶着指教。贾母看了,便道:“我的儿,我打谅你怎么病着,故此过来瞧你。今你依旧的模样儿,我的心放了好些。”王夫人也自然是宽心的。但宝玉并不回答,只管嘻嘻的笑。贾母等进屋坐下,问他的话,袭人教一句,他说一句,大不似往常,直是一个傻子似的。贾母愈看愈疑,便说:“我才进来看时,不见有什么病,如今细细一瞧,这病果然不轻,竟是神魂失散的样子。到底因什么起的呢?”王夫人知事难瞒,又瞧瞧袭人怪可怜的样子,只得便依着宝玉先前的话,将那往南安王府里去听戏时丢了这块玉的话,悄悄的告诉了一遍。心里也彷徨的很,生恐贾母着急,并说:“现在着人在四下里找寻,求签问卦,都说在当铺里找,少不得找着的。”贾母听了,急得站起来,眼泪直流,说道:“这件玉如何是丢得的!你们忒不懂事了,难道老爷也是撂开手的不成!”王夫人知贾母生气,叫袭人等跪下,自己敛容低首回说:“媳妇恐老太太着急老爷生气,都没敢回。”贾母咳道:“这是宝玉的命根子。因丢了,所以他是这么失魂丧魄的。还了得!况是这玉满城里都知道,谁捡了去便叫你们找出来么!叫人快快请老爷,我与他说。”那时吓得王夫人袭人等俱哀告道:“老太太这一生气,回来老爷更了不得了。现在宝玉病着,交给我们尽命的找来就是了。”贾母道:“你们怕老爷生气,有我呢。”便叫麝月传人去请,不一时传进话来,说:“老爷谢客去了。”贾母道:“不用他也使得。你们便说我说的话,暂且也不用责罚下人,我便叫琏儿来写出赏格,悬在前日经过的地方,便说有人捡得送来者,情愿送银一万两,如有知人捡得送信找得者,送银五千两。如真有了,不可吝惜银子。这么一找,少不得就找出来了。若是靠着咱们家几个人找,就找一辈子,也不能得。”王夫人也不敢直言。贾母传话告诉贾琏,叫他速办去了。贾母便叫人:“将宝玉动用之物都搬到我那里去,只派袭人秋纹跟过来,余者仍留园内看屋子。”宝玉听了,终不言语,只是傻笑。
\end{parag}


\begin{parag}
    贾母便携了宝玉起身,袭人等搀扶出园。回到自己房中,叫王夫人坐下,看人收拾里间屋内安置,便对王夫人道:“你知道我的意思么?我为的园里人少,怡红院里的花树忽萎忽开,有些奇怪。头里仗着一块玉能除邪祟,如今此玉丢了,生恐邪气易侵,故我带他过来一块儿住着。这几天也不用叫他出去,大夫来就在这里瞧。”王夫人听说,便接口道:“老太太想的自然是。如今宝玉同着老太太住了,老太太福气大,不论什么都压住了。”贾母道:“什么福气,不过我屋里干净些,经卷也多,都可以念念定定心神。你问宝玉好不好?”那宝玉见问,只是笑。袭人叫他说“好”,宝玉也就说“好”。王夫人见了这般光景,未免落泪,在贾母这里,不敢出声。贾母知王夫人着急,便说道:“你回去罢,这里有我调停他。晚上老爷回来,告诉他不必见我,不许言语就是了。”王夫人去后,贾母叫鸳鸯找些安神定魄的药,按方吃了。不题。
\end{parag}


\begin{parag}
    且说贾政当晚回家,在车内听见道儿上人说道:“人要发财也容易的很。”那个问道:“怎么见得?”这个人又道:“今日听见荣府里丢了什么哥儿的玉了,贴着招帖儿,上头写着玉的大小式样颜色,说有人捡了送去,就给一万两银子,送信的还给五千呢。”贾政虽未听得如此真切,心里诧异,急忙赶回,便叫门上的人问起那事来。门上的人禀道:“奴才头里也不知道,今儿晌午琏二爷传出老太太的话,叫人去贴帖儿,才知道的。”贾政便叹气道:“家道该衰,偏生养这么一个孽障!才养他的时候满街的谣言,隔了十几年略好了些,这会子又大张晓谕的找玉,成何道理!”说着,忙走进里头去问王夫人。王夫人便一五一十的告诉。贾政知是老太太的主意,又不敢违拗,只抱怨王夫人几句。又走出来,叫瞒着老太太,背地里揭了这个帖儿下来。岂知早有那些游手好闲的人揭了去了。
\end{parag}


\begin{parag}
    过了些时,竟有人到荣府门上,口称送玉来。家内人们听见,喜欢的了不得,便说:“拿来,我给你回去。”那人便怀内掏出赏格来,指给门上人瞧,”这不是你府上的帖子么,写明送玉来的给银一万两。二太爷,你们这会子瞧我穷,回来我得了银子,就是个财主了。别这么待理不理的。”门上听他话头来得硬,说道:“你到底略给我瞧一瞧,我好给你回去。”那人初倒不肯,后来听人说得有理,便掏出那玉,托在掌中一扬说:“这是不是?”众家人原是在外服役,只知有玉,也不常见,今日才看见这玉的模样儿了。急忙跑到里头,抢头报似的。那日贾政贾赦出门,只有贾琏在家。众人回明,贾琏还细问真不真。门上人口称:“亲眼见过,只是不给奴才,要见主子,一手交银,一手交玉。”贾琏却也喜欢,忙去禀知王夫人,即便回明贾母。把个袭人乐得合掌念佛。贾母并不改口,一迭连声:“快叫琏儿请那人到书房内坐下,将玉取来一看,即便送银。”贾琏依言,请那人进来当客待他,用好言道谢:“要借这玉送到里头,本人见了,谢银分厘不短。”那人只得将一个红绸子包儿送过去。贾琏打开一看,可不是那一块晶莹美玉吗。贾琏素昔原不理论,今日倒要看看,看了半日,上面的字也仿佛认得出来,什么”除邪祟”等字。贾琏看了,喜之不胜,便叫家人伺候,忙忙的送与贾母王夫人认去。
\end{parag}


\begin{parag}
    这会子惊动了合家的人,都等着争看。凤姐见贾琏进来,便劈手夺去,不敢先看,送到贾母手里。贾琏笑道:“你这么一点儿事还不叫我献功呢。”贾母打开看时,只见那玉比先前昏暗了好些。一面擦摸,鸳鸯拿上眼镜儿来,戴着一瞧,说:“奇怪,这块玉倒是的,怎么把头里的宝色都没了呢?”王夫人看了一会子,也认不出,便叫凤姐过来看。凤姐看了道:“象倒象,只是颜色不大对。不如叫宝兄弟自己一看就知道了。”袭人在旁也看着未必是那一块,只是盼得的心盛,也不敢说出不象来。凤姐于是从贾母手中接过来,同着袭人拿来给宝玉瞧。这时宝玉正睡着才醒。凤姐告诉道:“你的玉有了。”宝玉睡眼朦胧,接在手里也没瞧,便往地上一撂道:“你们又来哄我了。”说着只是冷笑。凤姐连忙拾起来,道:“这也奇了,怎么你没瞧就知道呢。”宝玉也不答言,只管笑。王夫人也进屋里来了,见他这样,便道:“这不用说了。他那玉原是胎里带来的一种古怪东西,自然他有道理。想来这个必是人见了帖儿照样做的。”大家此时恍然大悟。贾琏在外间屋里听见这话,便说道:“既不是,快拿来给我问问他去,人家这样事,他敢来鬼混。”贾母喝住道:“琏儿,拿了去给他,叫他去罢。那也是穷极了的人没法儿了,所以见我们家有这样事,他便想着赚几个钱也是有的。如今白白的花了钱弄了这个东西,又叫咱们认出来了。依着我不要难为他,把这玉还他,说不是我们的,赏给他几两银子。外头的人知道了,才肯有信儿就送来呢。若是难为了这一个人,就有真的,人家也不敢拿来了。”贾琏答应出去。那人还等着呢,半日不见人来,正在那里心里发虚,只见贾琏气忿走出来了。未知何如,下回分解。
\end{parag}