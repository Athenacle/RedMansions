\chap{二十六}{蜂腰橋設言傳心事 瀟湘館春困發幽情}


\begin{parag}
    \begin{note}蒙回前詞:一個時才得傳消息,一個是舊喜化作新歌。真真假假二事堪疑,哭向花林月底。\end{note}
\end{parag}


\begin{parag}
    話說寶玉養過了三十三天之後,不但身體強壯,亦且連臉上瘡痕平服,仍回大觀園內去。這也不在話下。
\end{parag}


\begin{parag}
    且說近日寶玉病的時節,賈芸帶著家下小廝坐更看守,晝夜在這裏,那紅玉同衆丫鬟也在這裏守著寶玉,彼此相見多日,都漸漸混熟了。那紅玉見賈芸手裏拿的手帕子,倒象是自己從前掉的,待要問他,又不好問的。不料那和尚道士來過,用不著一切男人,賈芸仍種樹去了。這件事待要放下,心內又放不下,待要問去,又怕人猜疑,正是猶豫不決神魂不定之際,忽聽窗外問道:“姐姐在屋裏沒有?”\begin{note}甲側:岔開正文,卻是爲正文作引。\end{note}\begin{note}庚側:你看他偏不寫正文,偏有許多閒文,卻是補遺。\end{note}紅玉聞聽,在窗眼內望外一看,原來是本院的個小丫頭名叫佳蕙的,因答說:“在家裏,你進來罷。”佳蕙聽了跑進來,就坐在牀上,笑道:“我好造化!纔剛在院子裏洗東西,寶玉叫往林姑娘那裏送茶葉,\begin{note}甲側:交代井井有法。\end{note}\begin{note}庚側:前文有言。\end{note}花大姐姐交給我送去。可巧老太太那裏給林姑娘送錢來,\begin{note}庚側:是補寫否?\end{note}正分給他們的丫頭們呢。\begin{note}甲側:瀟湘常事出自別院婢口中,反覺新鮮。\end{note}見我去了,林姑娘就抓了兩把給我,也不知多少。你替我收著。”便把手帕子打開,把錢倒了出來,紅玉替他一五一十的數了收起。\begin{note}庚眉:此等細事是舊族大家閨中常情,今特爲暴發錢奴寫來作鑑。一笑。壬午夏,雨窗。\end{note}
\end{parag}


\begin{parag}
    佳蕙道:“你這一程子心裏到底覺怎麼樣?依我說,你竟家去住兩日,請一個大夫來瞧瞧,喫兩劑藥就好了。”紅玉道:“那裏的話,好好的,家去作什麼!” 佳蕙道:“我想起來了,林姑娘生的弱,時常他吃藥,\begin{note}庚側:是補寫否?\end{note}你就和他要些來喫,也是一樣。”\begin{note}甲側:閒言中敘出黛玉之弱。草蛇灰線。\end{note}紅玉道:“胡說!\begin{note}庚側:如聞。\end{note}藥也是混喫的。”佳蕙道:“你這也不是個長法兒,又懶喫懶喝的,終久怎麼樣?”\begin{note}庚側:從旁人眼中口中出,妙極!\end{note}紅玉道:“怕什麼,還不如早些兒死了倒乾淨!”\begin{note}甲側:此句令人氣噎,總在無可奈何上來。\end{note}佳蕙道:“好好的,怎麼說這些話?”紅玉道: “你那裏知道我心裏的事!”
\end{parag}


\begin{parag}
    佳蕙點頭想了一會,道:“可也怨不得,這個地方難站。就象昨兒老太太因寶玉病了這些日子,\begin{note}庚側:是補文否?\end{note}說跟著伏侍的這些人都辛苦了,如今身上好了,各處還完了願,\begin{note}庚側:是補寫否?\end{note}叫把跟著的人都按著等兒賞他們。\begin{note}庚側:是補寫否?\end{note}我們算年紀小,上不去,我也不抱怨;像你怎麼也不算在裏頭?\begin{note}庚側:道著心病。\end{note}我心裏就不服。襲人那怕他得十分兒,也不惱他,原該的。說良心話,誰還敢比他呢?\begin{note}庚側:確論公論,方見襲卿身份。\end{note}別說他素日殷勤小心,便是不殷勤小心,也拼不得。可氣晴雯、綺霰他們這幾個,都算在上等裏去,仗著老子孃的臉面,衆人倒捧著他去。你說可氣不可氣?”紅玉道:“也不犯著氣他們。俗語說的好,‘千里搭長棚,沒有個不散的筵席’,\begin{note}甲側:此時寫出此等言語,令人墮淚。\end{note}誰守誰一輩子呢?不過三年五載,各人幹各人的去了。那時誰還管誰呢?”這兩句話不覺感動了佳蕙的心腸,\begin{note}庚側:不但佳蕙,批書者亦淚下矣。\end{note}由不得眼睛紅了,又不好意思好端端的哭,只得勉強笑道:“你這話說的卻是。昨兒寶玉還說,\begin{note}庚側:還是補文。\end{note}明兒怎麼樣收拾房子,怎麼樣做衣裳,倒象有幾百年的熬煎。”\begin{note}甲側:卻是小女兒口中無味之談,實是寫寶玉不如一環婢。\end{note}\begin{note}甲眉:紅玉一腔委屈怨憤,系身在怡紅不能遂志,看官勿錯認爲芸兒害相思也。己冬。\end{note}\begin{note}甲眉:“獄神廟”紅玉、茜雪一大回文字惜迷失無稿。[庚眉多八字:嘆嘆!丁亥夏。畸笏叟。]\end{note}
\end{parag}


\begin{parag}
    紅玉聽了冷笑了兩聲,方要說話,\begin{note}甲側:文字又一頓。\end{note}只見一個未留頭的小丫頭子走進來,手裏拿著些花樣子並兩張紙,說道:“這是兩個樣子,叫你描出來呢。”說著向紅玉擲下,回身就跑了。紅玉向外問道:“倒是誰的?也等不得說完就跑,誰蒸下饅頭等著你,怕冷了不成!”那小丫頭在窗外只說得一聲: “是綺大姐姐的。”\begin{note}甲側:是不合式之言、擢心語。\end{note}抬起腳來咕咚咕咚又跑了。\begin{note}甲側:活現,活現之文。\end{note}紅玉便賭氣把那樣子擲在一邊,\begin{note}庚側:何如?\end{note}向抽屜內找筆,找了半天都是禿了的,因說道:“前兒一枝新筆,\begin{note}庚側:是補文否?\end{note}放在那裏了?怎麼一時想不起來。”\begin{note}庚側:既在矮檐下,怎敢不低頭?\end{note}一面說著,一面出神,\begin{note}甲側:總是畫境。\end{note}想了一會方笑道:“是了,前兒晚上鶯兒拿了去了。”\begin{note}庚側:還是補文。\end{note}便向佳蕙道:“你替我取了來。”佳蕙道:“花大姐姐還等著我替他抬箱子呢,你自己取去罷。”紅玉道:“他等著你,你還坐著閒打牙兒?\begin{note}庚側:襲人身份。\end{note}我不叫你取去,他也不等著你了。壞透了的小蹄子!”說著,自己便出房來,出了怡紅院,一徑往寶釵院內來。\begin{note}庚側:曲折再四,方逼出正文來。\end{note}
\end{parag}


\begin{parag}
    剛至沁芳亭畔,只見寶玉的奶孃李嬤嬤從那邊走來。\begin{note}甲側:奇文,真令人不得機關。\end{note}紅玉立住笑問道:“李奶奶,你老人家那去了?怎打這裏來?”李嬤嬤站住將手一拍道:“你說說,好好的又看上了\begin{note}甲側:囫圇不解語。\end{note}那個種樹的什麼雲哥兒雨哥兒的,\begin{note}甲側:奇文神文。\end{note}這會子逼著我叫了他來。明兒叫上房裏聽見,可又是不好。”\begin{note}甲側:更不解。\end{note}紅玉笑道:“你老人家當真的就依了他去叫了?”\begin{note}甲側:是遂心語。\end{note}李嬤嬤道:“可怎麼樣呢?”\begin{note}甲側:妙!的是老嫗口氣。\end{note}紅玉笑道:“那一個要是知道好歹,\begin{note}甲側:更不解。\end{note}就回不進來纔是。”\begin{note}甲雙夾:是私心語,神妙!\end{note}李嬤嬤道:“他又不癡,爲什麼不進來?”紅玉道:“既是進來,你老人家該同他一齊來,回來叫他一個人亂碰,可是不好呢。”\begin{note}甲雙夾:總是私心語,要直問又不敢,只用這等語慢慢的套出。有神理。\end{note}李嬤嬤道:“我有那樣工夫和他走?不過告訴了他,回來打發個小丫頭子或是老婆子,帶進他來就完了。” 說著,拄著柺杖一徑去了。紅玉聽說,便站著出神,且不去取筆。\begin{note}甲雙夾:總是不言神情,另出花樣。\end{note}
\end{parag}


\begin{parag}
    一時,只見一個小丫頭子跑來,見紅玉站在那裏,便問道:“林姐姐,你在這裏作什麼呢?”紅玉抬頭見是小丫頭子墜兒。\begin{note}甲雙夾:墜兒者,贅也。人生天地間已是贅疣,況又生許多冤情孽債。嘆嘆!\end{note}紅玉道:“那去?”墜兒道:“叫我帶進芸二爺來。”\begin{note}庚側:等的是這句話。\end{note}說著一徑跑了。這裏紅玉剛走至蜂腰橋門前,只見那邊墜兒引著賈芸來了。\begin{note}甲雙夾:妙!不說紅玉不走,亦不說走,只說“剛走到”三字,可知紅玉有私心矣。若說出必定不走必定走,則文字死板,且亦棱角過露,非寫女兒之筆也。\end{note}那賈芸一面走,一面拿眼把紅玉一溜;那紅玉只裝著和墜兒說話,也把眼去一溜賈芸:四目恰相對時,紅玉不覺臉紅了,\begin{note}甲雙夾:看官至此,須掩卷細想上三十回中篇篇句句點“紅”字處,可與此處想如何?\end{note}一扭身往蘅蕪苑去了。不在話下。
\end{parag}


\begin{parag}
    這裏賈芸隨著墜兒,逶迤來至怡紅院中。墜兒先進去回明瞭,然後方領賈芸進去。賈芸看時,只見院內略略有幾點山石,種著芭蕉,那邊有兩隻仙鶴在松樹下剔翎。一溜回廊上吊著各色籠子,各色仙禽異鳥。上面小小五間抱廈,一色雕鏤新鮮花樣隔扇,上面懸著一個匾額,四個大字,題道是“怡紅快綠”。賈芸想道:“怪道叫‘怡紅院’,原來匾上是恁樣四個字。”\begin{note}甲雙夾:傷哉,轉眼便紅稀綠瘦矣。嘆嘆!\end{note}正想著,只聽裏面隔著紗窗子笑說道:\begin{note}甲側:此文若張僧繇點睛之龍,破壁飛矣,焉得不拍案叫絕!\end{note}“快進來罷。我怎麼就忘了你兩三個月!”賈芸聽得是寶玉的聲音,連忙進入房內。抬頭一看,只見金碧輝煌,\begin{note}甲側:器皿疊疊。\end{note}\begin{note}庚側:不能細覽之文。\end{note}文章閃灼,\begin{note}甲側:陳設壘壘。\end{note}\begin{note}庚側:不得細玩之文。\end{note}卻看不見寶玉在那裏。\begin{note}甲側:武夷九曲之文。\end{note}一回頭,只見左邊立著一架大穿衣鏡,從鏡後轉出兩個一般大的十五六歲的丫頭來說:“請二爺裏頭屋裏坐。”賈芸連正眼也不敢看,連忙答應了。又進一道碧紗廚,只見小小一張填漆牀上,懸著大紅銷金撒花帳子。寶玉穿著家常衣服,靸著鞋,倚在牀上拿著本書,\begin{note}甲側:這是等芸哥看,故作款式。若果真看書,在隔紗窗子說話時已經放下了。玉兄若見此批,必雲:老貨,他處處不放鬆我,可恨可恨!回思將餘比作釵、顰等,乃一知己,餘何幸也!一笑。\end{note}看見他進來,將書擲下,早堆著笑立起身來。\begin{note}庚側:小叔身段。\end{note}賈芸忙上前請了安。寶玉讓坐,便在下面一張椅子上坐了。寶玉笑道:“只從那個月見了你,我叫你往書房裏來,誰知接接連連許多事情,就把你忘了。”賈芸笑道:“總是我沒福,偏偏又遇著叔叔身上欠安。叔叔如今可大安了?”寶玉道:“大好了。我倒聽見說你辛苦了好幾天。”賈芸道:“辛苦也是該當的。叔叔大安了,也是我們一家子的造化。”\begin{note}甲側:不倫不理,迎合字樣,口氣逼肖,可笑可嘆!\end{note}\begin{note}庚側:誰一家子?可發一大笑。\end{note}
\end{parag}


\begin{parag}
    說著,只見有個丫鬟端了茶來與他。那賈芸口裏和寶玉說著話,眼睛卻溜瞅那丫鬟:\begin{note}甲側:前寫不敢正眼,今又如此寫,是用茶來,有心人故留此神,於接茶時站起,方不突然。庚側:此句是認人,非前溜紅玉之文。\end{note}細挑身材,容長臉面,穿著銀紅襖兒,青緞背心,白綾細摺裙。──不是別個,卻是襲人。\begin{note}甲側:《水滸》文法用的恰,當是芸哥眼中也。\end{note}那賈芸自從寶玉病了幾天,他在裏頭混了兩日,他卻把那有名人口認記了一半。\begin{note}甲側:一路總是賈芸是個有心人,一絲不亂。\end{note}他也知道襲人在寶玉房中比別個不同,\begin{note}庚側:如何?可知餘前批不謬。\end{note}今見他端了茶來,寶玉又在旁邊坐著,便忙站起來笑道: “姐姐怎麼替我倒起茶來。我來到叔叔這裏,又不是客,讓我自己倒罷。”\begin{note}甲雙夾:總寫賈芸乖覺,一絲不亂。\end{note}寶玉道:“你只管坐著罷。丫頭們跟前也是這樣。”賈芸笑道:“雖如此說,叔叔房裏姐姐們,我怎麼敢放肆呢?”\begin{note}甲側:紅玉何以使得?\end{note}一面說,一面坐下喫茶。
\end{parag}


\begin{parag}
    那寶玉便和他說些沒要緊的散話。\begin{note}甲雙夾:妙極是極!況寶玉又有何正緊\begin{subnote}注:蒙本此處作“經”\end{subnote}可說的!\end{note}又說道誰家的戲子好,誰家的花園好,又告訴他誰家的丫頭標緻,誰家的酒席豐盛,又是誰家有奇貨,又是誰家有異物。\begin{note}甲雙夾:幾個“誰家”,自北靜王公侯駙馬諸大家包括盡矣,寫盡紈絝口角。\end{note}\begin{note}庚側:脂硯齋再筆:對芸兄原無可說之話。\end{note}那賈芸口裏只得順著他說,說了一會,見寶玉有些懶懶的了,便起身告辭。寶玉也不甚留,只說:“你明兒閒了,只管來。”仍命小丫頭子墜兒送他出去。
\end{parag}


\begin{parag}
    出了怡紅院,賈芸見四顧無人,便把腳慢慢停著些走,口裏一長一短和墜兒說話,先問他“幾歲了?名字叫什麼?你父母在那一行上?在寶叔房內幾年了?\begin{note}甲側:漸漸入港。\end{note}一個月多少錢?共總寶叔房內有幾個女孩子?”那墜兒見問,便一樁樁的都告訴他了。賈芸又道:“纔剛那個與你說話的,他可是叫小紅?” 墜兒笑道:“他倒叫小紅。你問他作什麼?”賈芸道:“方纔他問你什麼手帕子,我倒揀了一塊。”墜兒聽了笑道:“他問了我好幾遍,可有看見他的帕子。我有那麼大工夫管這些事!今兒他又問我,他說我替他找著了,他還謝我呢。\begin{note}庚側:“傳”字正文,此處方露。\end{note}纔在蘅蕪苑門口說的,二爺也聽見了,不是我撒謊。好二爺,你既揀了,給我罷。我看他拿什麼謝我。”
\end{parag}


\begin{parag}
    原來上月賈芸進來種樹之時,便揀了一塊羅帕,便知是所在園內的人失落的,但不知是那一個人的,故不敢造次。今聽見紅玉問墜兒,便知是紅玉的,心內不勝喜幸。又見墜兒追索,心中早得了主意,便向袖內將自己的一塊取了出來,向墜兒笑道:“我給是給你,你若得了他的謝禮,不許瞞著我。”墜兒滿口裏答應了,接了手帕子,送出賈芸,回來找紅玉,不在話下。\begin{note}甲雙夾:至此一頓,狡猾之甚!原非書中正文之人,寫來間色耳。\end{note}
\end{parag}


\begin{parag}
    如今且說寶玉打發了賈芸去後,意思懶懶的歪在牀上,似有朦朧之態。襲人便走上來,坐在牀沿上推他,說道:“怎麼又要睡覺?悶的很,你出去逛逛不是?” 寶玉見說,便拉他的手笑道:“我要去,只是捨不得你。”襲人笑道:“快起來罷!”\begin{note}甲側:不答得妙!\end{note}\begin{note}庚側:不答上文,妙極!\end{note}一面說,一面拉了寶玉起來。寶玉道:“可往那去呢?怪膩膩煩煩的。”\begin{note}庚側:玉兄最得意之文,起筆卻如此寫。\end{note}襲人道:“你出去了就好了。只管這麼葳蕤 ,越發心裏煩膩。”
\end{parag}


\begin{parag}
    寶玉無精打采的,只得依他。晃出了房門,在迴廊上調弄了一回雀兒;出至院外,順著沁芳溪看了一回金魚。只見那邊山坡上兩隻小鹿箭也似的跑來,寶玉不解其意,\begin{note}甲側:餘亦不解。\end{note}正自納悶,只見賈蘭在後面拿著一張小弓追了下來。\begin{note}甲側:前文。\end{note}\begin{note}庚側:此等文可是人能意料的?\end{note}一見寶玉在前面,便站住了,笑道:“二叔叔在家裏呢,我只當出門去了。”寶玉道:“你又淘氣了。好好的射他作什麼?”賈蘭笑道:“這會子不念書,閒著作什麼?所以演習演習騎射。”\begin{note}甲側:奇文奇語,默思之方意會。爲玉兄之毫無一正事,只知安富尊榮而寫。\end{note}\begin{note}庚側:答得何其堂皇正大,何其坦然之至!\end{note}寶玉道: “把牙栽了,那時纔不演呢。”
\end{parag}


\begin{parag}
    說著,順著腳一徑來至一個院門前,\begin{note}庚側:像無意。\end{note}只見鳳尾森森,龍吟細細。\begin{note}甲雙夾:與後文“落葉蕭蕭,寒煙漠漠”一對,可傷可嘆!\end{note}\begin{note}庚批:原無意。\end{note}舉目望門上一看,只見匾上寫著“瀟湘館”三字。\begin{note}甲側:無一絲心機,反似初至者,故接有忘形忘情話來。\end{note}\begin{note}庚側:三字如此出,足見真出無意。\end{note}寶玉信步走入,只見湘簾垂地,悄無人聲。走至窗前,覺得一縷幽香從碧紗窗中暗暗透出。\begin{note}甲側:寫得出,寫得出。\end{note}寶玉便將臉貼在紗窗上,往裏看時,耳內忽聽得\begin{note}甲雙夾:未曾看見先聽見,有神理。\end{note}細細的長嘆了一聲道:“‘每日家情思睡昏昏’。”\begin{note}甲側:用情忘情神化之文。\end{note}\begin{note}庚眉:先用“鳳尾森森,龍吟細細”八字,“一縷幽香自紗窗中暗暗透出”,“細細的長嘆一聲”等句,方引出“每日家情思昏睡睡”仙音妙音來,非純化功夫之筆不能,可見行文之難。\end{note}寶玉聽了,不覺心內癢將起來,再看時,只見黛玉在牀上伸懶腰。\begin{note}甲側:有神理,真真畫出。\end{note}寶玉在窗外笑道:“爲甚麼 ‘每日家情思睡昏昏’?”一面說,一面掀簾子進來了。\begin{note}庚眉:二玉這回文字,作者亦在無意上寫來,所謂“信手拈來無不是”也。\end{note}
\end{parag}


\begin{parag}
    林黛玉自覺忘情,不覺紅了臉,拿袖子遮了臉,翻身向裏裝睡著了。寶玉才走上來要搬他的身子,只見黛玉的奶孃並兩個婆子卻跟了進來\begin{note}甲側:一絲不漏,且避若干嚼蠟之文。\end{note}說:“妹妹睡覺呢,等醒了再請來。”剛說著,黛玉便翻身坐了起來,笑道:“誰睡覺呢。”\begin{note}甲側:妙極!可知黛玉是怕寶玉去也。\end{note}那兩三個婆子見黛玉起來,便笑道:“我們只當姑娘睡著了。”說著,便叫紫鵑說:“姑娘醒了,進來伺侯。”一面說,一面都去了。
\end{parag}


\begin{parag}
    黛玉坐在牀上,一面抬手整理鬢髮,一面笑向寶玉道:“人家睡覺,你進來作什麼?”寶玉見他星眼微餳,香腮帶赤,不覺神魂早蕩,一歪身坐在椅子上,笑道:“你才說什麼?”黛玉道:“我沒說什麼。”寶玉笑道:“給你個榧子喫!我都聽見了。”
\end{parag}


\begin{parag}
    二人正說話,只見紫鵑進來。寶玉笑道:“紫鵑,把你們的好茶倒碗我喫。”紫鵑道:“那裏是好的呢?要好的,只是等襲人來。”黛玉道:“別理他,你先給我舀水去罷。”紫鵑笑道:“他是客,自然先倒了茶來再舀水去。”說著倒茶去了。寶玉笑道:“好丫頭,‘若共你多情小姐同鴛帳,怎捨得疊被鋪牀?’”\begin{note}甲側:真正無意忘情。\end{note}\begin{note}庚側:真正無意忘情衝口而出之語。\end{note}\begin{note}庚眉:方纔見芸哥所拿之書一定是《西廂記》,不然如何忘情之此?\end{note}林黛玉登時撂下臉來,\begin{note}甲側:我也要惱。\end{note}說道:“二哥哥,你說什麼?”寶玉笑道:“我何嘗說什麼。”黛玉便哭道:“如今新興的,外頭聽了村話來,也說給我聽;看了混帳書,也來拿我取笑兒。我成了爺們解悶的。”一面哭著,一面下牀來往外就走。寶玉不知要怎樣,心下慌了,忙趕上來,“好妹妹,我一時該死,你別告訴去。我再要敢,嘴上就長個疔,爛了舌頭。”
\end{parag}


\begin{parag}
    正說著,只見襲人走來說道:“快回去穿衣服,老爺叫你呢。”\begin{note}庚眉:若無如此文字收拾二玉,寫顰無非至再哭慟哭,玉只以賠盡小心軟求漫懇,二人一笑而止。且書內若此亦多多矣,未免有犯雷同之病。故用險句結住,使二玉心中不得不將現事拋卻,各懷一驚心意,再作下文。壬午孟夏,雨窗。畸笏。\end{note}寶玉聽了,不覺打了個焦雷的一般,\begin{note}甲側:不止玉兄一驚,即阿顰亦不免一嚇,作者只顧寫來收拾二玉之文,忘卻顰兒也。想作者亦似寶玉道《西廂》之句,忘情而出也。\end{note}也顧不得別的,疾忙回來穿衣服。出園來,只見焙茗在二門前等著,寶玉便問道:“是作什麼?”焙茗道:“爺快出來罷,橫豎是見去的,到那裏就知道了。”一面說,一面催著寶玉。
\end{parag}


\begin{parag}
    轉過大廳,寶玉心裏還自狐疑,只聽牆角邊一陣呵呵大笑,回頭看時,見是薛蟠拍著手跳了出來,笑道:\begin{note}甲側:如此戲弄,非呆兄無人。欲釋二玉,非此戲弄不能立解,勿得泛泛看過。不知作者胸中有多少丘壑。\end{note}\begin{note}庚側:非呆兄行不出此等戲弄,但作者有多少丘壑在胸中,寫來酷肖。\end{note}“要不說姨夫叫你,你那裏出來的這麼快。”焙茗也笑著跪下了。寶玉怔了半天,方解過來了,是薛蟠哄他出來。薛蟠連忙打恭作揖陪不是,\begin{note}庚側:酷肖。\end{note}又求“不要難爲了小子,都是我逼他去的”。寶玉也無法了,只好笑問道:“你哄我也罷了,怎麼說我父親呢?我告訴姨娘去,評評這個理,可使得麼?”薛蟠忙道:“好兄弟,我原爲求你快些出來,就忘了忌諱這句話。改日你也哄我,說我的父親就完了。”\begin{note}甲側:寫粗豪無心人畢肖。\end{note}\begin{note}庚側:真真亂話。\end{note}寶玉道:“噯,噯,越發該死了。”又向焙茗道:“反叛肏的,還跪著作什麼!”焙茗連忙叩頭起來。薛蟠道:“要不是我也不敢驚動,只因明兒五月初三日是我的生日,誰知古董行的程日興,他不知那裏尋了來的這麼粗這麼長粉脆的鮮藕,\begin{note}庚側:如見如聞。\end{note}這麼大的大西瓜,這麼長一尾新鮮的鱘魚,這麼大的一個暹羅國進貢的靈柏香薰的暹豬。你說,他這四樣禮可難得不難得?那魚,豬不過貴而難得,這藕和瓜虧他怎麼種出來的。我連忙孝敬了母親,趕著給你們老太太、姨父、姨母送了些去。如今留了些,我要自己喫,恐怕折福,\begin{note}甲側:呆兄亦有此語,批書人至此誦《往生咒》至恆河沙數也。\end{note}左思右想,除我之外,惟有你還配喫,\begin{note}甲側:此語令人哭不得笑不得,亦真心語也。\end{note}所以特請你來。可巧唱曲兒的小麼兒又纔來了,我同你樂一天何如?”
\end{parag}


\begin{parag}
    一面說,一面來至他書房裏。只見詹光、程日興、胡斯來、單聘仁等並唱曲兒的都在這裏,見他進來,請安的,問好的,都彼此見過了。吃了茶,薛蟠即命人擺酒來。說猶未了,衆小廝七手八腳擺了半天,\begin{note}庚側:又一個寫法。\end{note}方纔停當歸坐。寶玉果見瓜藕新異,因笑道:“我的壽禮還未送來,倒先擾了。”薛蟠道:“可是呢,明兒你送我什麼?”\begin{note}庚側:逼真酷肖。\end{note}寶玉道:“我可有什麼可送的?若論銀錢喫穿等類的東西,\begin{note}甲側:誰說的出?經過者方說得出。嘆嘆!\end{note}究竟還不是我的,惟有我寫一張字,畫一張畫,纔算是我的。”
\end{parag}


\begin{parag}
    薛蟠笑道:“你提畫兒,我纔想起來。昨兒我看人家一張春宮,\begin{note}庚側:阿呆兄所見之畫也!\end{note}畫的著實好。上面還有許多的字,也沒細看,只看落的款,是‘庚黃’\begin{note}甲側:奇文,奇文!\end{note}畫的。真真的好的了不得!”寶玉聽說,心下猜疑道:“古今字畫也都見過些,那裏有個‘庚黃’?”想了半天,不覺笑將起來,命人取過筆來,在手心裏寫了兩個字,又問薛蟠道:“你看真了是‘庚黃’?”薛蟠道:“怎麼看不真!”\begin{note}甲眉:閒事順筆,罵死不學之紈絝。嘆嘆!\end{note}\begin{note}庚眉:閒事順筆將罵死不學之紈絝。壬午雨窗。畸笏。\end{note}寶玉將手一撒,與他看道:“別是這兩字罷?其實與‘庚黃’相去不遠。”衆人都看時,原來是“唐寅”兩個字,都笑道:“想必是這兩字,大爺一時眼花了也未可知。”薛蟠只覺沒意思,\begin{note}庚側:實心人。\end{note}笑道:“誰知他‘糖銀’‘果銀’的。”
\end{parag}


\begin{parag}
    正說著,小廝來回:“馮大爺來了。”寶玉便知是神武將軍馮唐之子馮紫英來了。薛蟠等一齊都叫:“快請。”說猶未了,只見馮紫英一路說笑,\begin{note}庚側:如見如聞。\end{note}已進來了。\begin{note}甲側:一派英氣如在紙上,特爲金閨潤色也。\end{note}衆人忙起席讓坐。馮紫英笑道:“好呀!也不出門了,在家裏高樂罷。”\begin{note}如見其人於紙上。\end{note}寶玉薛蟠都笑道:“一向少會,老世伯身上康健?”紫英答道:“家父倒也託庇康健。近來家母偶著了些風寒,不好了兩天。”\begin{note}庚眉:紫英豪俠小文三段,是爲金閨間色之文,壬午雨窗。\end{note}\begin{note}庚眉:寫倪二、紫英、湘蓮、玉菡俠文,皆各得傳真寫照之筆。丁亥夏。畸笏叟。\end{note}\begin{note}庚眉:惜“衛若蘭射圃”文字無稿。嘆嘆!丁亥夏。笏叟。\end{note}薛蟠見他面上有些青傷,便笑道:“這臉上又和誰揮拳的?掛了幌子了。”馮紫英笑道:“從那一遭把仇都尉的兒子打傷了,我就記了再不慪氣,如何又揮拳?這個臉上,是前日打圍,在鐵網山教兔鶻捎一翅膀。”\begin{note}庚側:如何著想?新奇字樣。\end{note}寶玉道:“幾時的話?”紫英道:“三月二十八日去的,前兒也就回來了。”寶玉道:“怪道前兒初三四兒,我在沈世兄家赴席不見你呢。我要問,不知怎麼就忘了。單你去了,還是老世伯也去了?”紫英道:“可不是家父去,我沒法兒,去罷了。難道我閒瘋了,咱們幾個人喫酒聽唱的不樂,尋那個苦惱去?這一次,大不幸之中又大幸。”\begin{note}甲側:似又伏一大事樣,英俠人累累如是,令人猜摹。\end{note}
\end{parag}


\begin{parag}
    薛蟠衆人見他喫完了茶,都說道:“且入席,有話慢慢的說。”\begin{note}庚側:□文再述。\end{note}馮紫英聽說,便立起身來說道:“論理,我該陪飲幾杯纔是,只是今兒有一件大大要緊的事,回去還要見家父面回,實不敢領。”薛蟠寶玉衆人那裏肯依,死拉著不放。馮紫英笑道:“這又奇了。\begin{note}庚側:如聞如見。\end{note}你我這些年,那回兒有這個道理的?果然不能遵命。若必定叫我領,拿大杯來,\begin{note}庚側:寫豪爽人如此。\end{note}我領兩杯就是了。”衆人聽說,只得罷了,薛蟠執壺,寶玉把盞,斟了兩大海。那馮紫英站著,一氣而盡。\begin{note}甲側:令人快活煞。\end{note}\begin{note}庚側:爽快人如此,令人羨煞。\end{note}寶玉道:“你到底把這個‘不幸之幸’說完了再走。”馮紫英笑道:“今兒說的也不盡興。我爲這個,還要特治一東,請你們去細談一談;二則還有所懇之處。”說著執手就走。薛蟠道:“越發說的人熱剌剌的丟不下。多早晚才請我們,告訴了。也免的人猶疑。”\begin{note}甲側:實心人如此,絲毫行跡俱無,令人痛快煞。\end{note}馮紫英道:“多則十日,少則八天。”一面說,一面出門上馬去了。衆人回來,依席又飲了一回方散。\begin{note}甲側:收拾得好。\end{note}
\end{parag}


\begin{parag}
    寶玉回至園中,襲人正記掛著他去見賈政,\begin{note}甲側:生員切己之事,時刻難忘。\end{note}不知是禍是福,\begin{note}庚側:下文伏線。\end{note}只見寶玉醉醺醺的回來,問其原故,寶玉一一向他說了。襲人道:“人家牽腸掛肚的等著,你且高樂去,也到底打發人來給個信兒。”寶玉道:“我何嘗不要送信兒,只因馮世兄來了,就混忘了。”
\end{parag}


\begin{parag}
    正說,只見寶釵走進來笑道:“偏了我們新鮮東西了。”寶玉笑道:“姐姐家的東西,自然先偏了我們了。”寶釵搖頭笑道:“昨兒哥哥倒特特的請我喫,我不喫他,叫他留著請人送人罷。我知道我命小福薄,不配喫那個。”\begin{note}甲側:暗對呆兄言寶玉配喫語。\end{note}說著,丫鬟倒了茶來,喫茶說閒話兒,不在話下。
\end{parag}


\begin{parag}
    卻說那林黛玉聽見賈政叫了寶玉去了,一日不回來,心中也替他憂慮。\begin{note}甲側:本是切己事。\end{note}至晚飯後,聞聽寶玉來了,心裏要找他問問是怎麼樣了。\begin{note}甲側:呆兄此席,的是合和筵也。一笑。\end{note}\begin{note}庚側:這席東道是和事酒不是?\end{note}一步步行來,見寶釵進寶玉的院內去了,\begin{note}甲側:《石頭記》最好看處是此等章法。\end{note}自己也便隨後走了來。剛到了沁芳橋,只見各色水禽都在池中浴水,也認不出名色來,但見一個個文彩炫耀,好看異常,因而站住看了一會。\begin{note}庚側:避難法。\end{note}再往怡紅院來,只見院門關著,黛玉便以手扣門。
\end{parag}


\begin{parag}
    誰知晴雯和碧痕正拌了嘴,沒好氣,忽見寶釵來了,那晴雯正把氣移在寶釵身上,\begin{note}庚眉:晴雯遷怒是常事耳,寫釵、顰二卿身上,與踢襲人之文,令人與何處設想著筆?丁亥夏。畸笏叟。\end{note}正在院內抱怨說:“有事沒事跑了來坐著,\begin{note}甲側:犯寶釵如此寫法。\end{note}叫我們三更半夜的不得睡覺!”\begin{note}甲側:指明人則暗寫。\end{note}忽聽又有人叫門,晴雯越發動了氣,也並不問是誰,\begin{note}甲側:犯黛玉如此寫明。\end{note}便說道:“都睡下了,明兒再來罷!”\begin{note}甲側:不知人則明寫。\end{note}林黛玉素知丫頭們的情性,他們彼此頑耍慣了,恐怕院內的丫頭沒聽真是他的聲音,只當是別的丫頭們來了,所以不開門,因而又高聲說道:“是我,還不開麼?”晴雯偏生還沒聽出來,\begin{note}甲側:想黛玉高聲亦不過你我平常說話一樣耳,況晴雯素昔浮躁多氣之人,如何辨得出?此刻須得批書人唱“大江東去”的喉嚨,嚷著“是我林黛玉叫門”方可。又想若開了門,如何有後面很多好字樣好文章,看官者意爲是否?\end{note}便使性子說道:“憑你是誰,二爺吩咐的,一概不許放人進來呢!”林黛玉聽了,不覺氣怔在門外,待要高聲問他,逗起氣來,自己又回思一番:“雖說是舅母家如同自己家一樣,到底是客邊。\begin{note}甲側:寄食者著眼,況顰兒何等人乎?\end{note}如今父母雙亡,無依無靠,現在他家依棲。如今認真淘氣,也覺沒趣。”一面想,一面又滾下淚珠來。正是回去不是,站著不是。正沒主意,只聽裏面一陣笑語之聲,細聽一聽,竟是寶玉、寶釵二人。林黛玉心中益發動了氣,左思右想,忽然想起了早起的事來:“必竟是寶玉惱我要告他的原故。但只我何嘗告你了,你也打聽打聽,就惱我到這步田地。你今兒不叫我進來,難道明兒就不見面了!”越想越傷感,也不顧蒼苔露冷,花徑風寒,獨立牆角邊花陰之下,悲悲慼慼嗚咽起來。\begin{note}甲側:可憐殺!可疼殺!餘亦淚下。\end{note}
\end{parag}


\begin{parag}
    原來這林黛玉秉絕代姿容,具希世俊美,不期這一哭,那附近柳枝花朵上的宿鳥棲鴉一聞此聲,俱忒楞楞飛起遠避,不忍再聽。真是:
\end{parag}


\begin{poem}
    \begin{pl}花魂默默無情緒,鳥夢癡癡何處驚。\end{pl}\begin{note}甲側:沉魚落雁,閉月羞花,原來是哭出來的。一笑。\end{note}
\end{poem}


\begin{parag}
    因有一首詩道:
\end{parag}


\begin{poem}
    \begin{pl}顰兒才貌世應希,獨抱幽芳出繡閨;\end{pl}

    \begin{pl}嗚咽一聲猶未了,落花滿地鳥驚飛。\end{pl}
\end{poem}


\begin{parag}
    那林黛玉正自啼哭,忽聽“吱嘍”一聲,院門開處,不知是那一個出來。要知端的,且聽下回分解。\begin{note}甲側:每閱此本,掩卷者十有八九,不忍下閱看完,想作者此時淚下如豆矣。\end{note}
\end{parag}


\begin{parag}
    \begin{note}甲:此回乃顰兒正文,故借小紅許多曲折瑣碎之筆作引。\end{note}
\end{parag}


\begin{parag}
    \begin{note}甲:怡紅院見賈芸,寶玉心內似有如無,賈芸眼中應接不暇。\end{note}
\end{parag}


\begin{parag}
    \begin{note}甲:“鳳尾森森,龍吟細細”八字,“一縷幽香從碧紗窗中暗暗透出”,又“細細的長嘆一聲”等句方引出“每日家情思睡昏昏”仙音妙音,俱純化工夫之筆。\end{note}
\end{parag}


\begin{parag}
    \begin{note}甲:二玉這回文字,作者亦在無意上寫來,所謂“信手拈來無不是”也。\end{note}
\end{parag}


\begin{parag}
    \begin{note}甲:收拾二玉文字,寫顰無非哭玉、再哭、慟哭,玉只以陪事小心軟求慢懇,二人一笑而止。且書內若此亦多多矣,未免有犯雷同之病。故險語結住,使二玉心中不得不將現事拋卻,各懷以驚心意,再作下文。\end{note}
\end{parag}


\begin{parag}
    \begin{note}甲:前回倪二、紫英、湘蓮、玉菡四樣俠文皆得傳真寫照之筆,惜“衛若蘭射圃”文字迷失無稿,嘆嘆!\end{note}
\end{parag}


\begin{parag}
    \begin{note}甲:晴雯遷怒系常事耳,寫於釵、顰二卿身上與踢襲人、打平兒之文,令人於何處設想著筆。\end{note}
\end{parag}


\begin{parag}
    \begin{note}甲:黛玉望怡紅之泣,是“每日家情思睡昏昏”上來。\end{note}
\end{parag}


\begin{parag}
    \begin{note}蒙回後總評:喜相逢,三生註定;遺手帕,月老紅絲。幸得人語說連理,又忽見他枝並蒂。難猜未解細追思,罔多疑,空向花枝哭月底。\end{note}
\end{parag}

