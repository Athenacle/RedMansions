\chap{三十八}{林瀟湘魁奪菊花詩 薛蘅蕪諷和螃蟹詠}

\begin{parag}
    \begin{note}蒙回前總:美人用別號,亦新奇花樣,且韻且雅,呼去覺滿口生香。起社出自探春意,作者已伏下興利除弊之文也。此回從放筆寫詩寫詞作札,看他詩復詩,詞複詞,札復札,總不相犯。\end{note}
\end{parag}


\begin{parag}
    \begin{note}蒙回前總:湘雲,詩客也,前回寫之其今才起社,後用不即不離閒人數語數折,仍歸社中。何巧活之筆如此?\end{note}
\end{parag}


\begin{parag}
    \begin{note}庚:題曰“菊花詩”、“螃蟹詠”,僞自太君前阿鳳若許詼諧中不失體、鴛鴦平兒寵婢中多少放肆之迎合取樂寫來,似難入題,卻輕輕用弄水戲魚之看花等遊玩事及王夫人云“這裏風大”一句收住入題,並無纖毫牽強,此重作輕抹法也。妙極!好看煞!\end{note}
\end{parag}


\begin{parag}
    話說寶釵湘雲二人計議已妥,一宿無話。湘雲次日便請賈母等賞桂花。賈母等都說道:“是他有興頭,須要擾他這雅興。”\begin{note}庚雙夾:若在世俗小家,則雲:“你是客,在我們舍下,怎麼反擾你的呢?”一何可笑。\end{note}至午,果然賈母帶了王夫人鳳姐兼請薛姨媽等進園來。賈母因問:“那一處好?”\begin{note}庚雙夾:必如此問方好。\end{note}王夫人道:“憑老太太愛在哪一處,就在哪一處。”\begin{note}庚雙夾:必是王夫人如此答方妙。\end{note}鳳姐道:“藕香榭已經擺下了,那山坡下兩顆桂花開的又好,河裏的水又碧清,坐在河當中亭子上豈不敞亮,看著水眼也清亮。”\begin{note}庚雙夾:智者樂水,豈具然乎?\end{note}賈母聽了,說:“這話很是。”說著,就引了衆人往藕香榭來。原來這藕香榭蓋在池中,四面有窗,左右有曲廊可通,亦是跨水接岸,後面又有曲折竹橋暗接。衆人上了竹橋,鳳姐忙上來攙著賈母,口裏說:“老祖宗只管邁大步走,不相干的,這竹子橋規矩 咯吱咯喳的。”\begin{note}庚雙夾:如見其勢,如臨其上,非走過者形容不到。\end{note}
\end{parag}


\begin{parag}
    一時進入榭中,只見欄杆外另放著兩張竹案,一個上面設著杯箸酒具,一個上頭設著茶筅茶盂各色茶具。那邊有兩三個丫頭煽風爐煮茶,這一邊另外幾個丫頭也煽風爐燙酒呢。賈母喜的忙問:“這茶想的到,且是地方,東西都乾淨。”湘雲笑道:“這是寶姐姐幫著我預備的。”賈母道:“我說這個孩子細緻,凡事想的妥當。”一面說,一面又看見柱上掛的黑漆嵌蚌的對子,命人念。湘雲念道:
\end{parag}


\begin{poem}
    \begin{pl}芙蓉影破歸蘭槳,菱藕香深寫竹橋。\end{pl}
    \begin{note}庚雙夾:妙極!此處忽又補出一處不入賈政“試才”一回,皆錯綜其事,不作一直筆也。\end{note}
\end{poem}


\begin{parag}
    賈母聽了,又抬頭看匾,因回頭向薛姨媽道:“我先小時,家裏也有這麼一個亭子,叫做什麼‘枕霞閣’。我那時也只象他們這麼大年紀,同姊妹們天天頑去。那日誰知我失了腳掉下去,幾乎沒淹死,好容易救了上來,到底被那木釘把頭碰破了。如今這鬢角上那指頭頂大一塊窩兒就是那殘破了。衆人都怕經了水,又怕冒了風,都說活不得了,誰知竟好了。”鳳姐不等人說,先笑道:“那時要活不得,如今這大福可叫誰享呢!可知老祖宗從小兒的福壽就不小,神差鬼使碰出那個窩兒來,好盛福壽的。壽星老兒頭上原是一個窩兒,因爲萬福萬壽盛滿了,所以倒凸高出些來了。”未及說完,賈母與衆人都笑軟了。\begin{note}庚雙夾:看他忽用賈母數語,閒閒又補出此書之前似已有一部《十二釵》的一般,令人遙憶不能一見,餘則將欲補出 枕霞閣中十二釵來,豈不又添一部新書?\end{note}賈母笑道:“這猴兒慣的了不得了,只管拿我取笑起來,恨的我撕你那油嘴。”鳳姐笑道:“回來喫螃蟹,恐積了冷在心裏,討老祖宗笑一笑開開心,一高興多喫兩個就無妨了。”賈母笑道:“明兒叫你日夜跟著我,我倒常笑笑覺的開心,不許回家去。”王夫人笑道:“老太太因爲喜歡他,才慣的他這樣,還這樣說,他明兒越發無禮了。”賈母笑道:“我喜歡他這樣,況且他又不是那不知高低的孩子。家常沒人,娘兒們原該這樣。橫豎禮體不錯就罷,沒的倒叫他從神兒似的作什麼。”\begin{note}庚雙夾:近之暴發專講理法竟不知禮法,此似無禮而禮法井井,所謂“整瓶不動半瓶搖”,又曰“習慣成自然”,真不謬也。\end{note}
\end{parag}


\begin{parag}
    說著,一齊進入亭子,獻過茶,鳳姐忙著搭桌子,要杯箸。上面一桌,賈母、薛姨媽、寶釵、黛玉、寶玉;東邊一桌,史湘雲、王夫人、迎、探、惜;西邊靠門一桌,李紈和鳳姐的,虛設坐位,二人皆不敢坐,只在賈母王夫人兩桌上伺候。鳳姐吩咐:“螃蟹不可多拿來,仍舊放在蒸籠裏,拿十個來,吃了再拿。”一面又要水洗了手,站在賈母跟前剝蟹肉,頭次讓薛姨媽。薛姨媽道:“我自己掰著喫香甜,不用人讓。”鳳姐便奉與賈母。二次的便與寶玉,又說:“把酒燙的滾熱的拿來。”又命小丫頭們去取了菊花葉兒桂花蕊燻的綠豆麪子來,預備著洗手。史湘雲陪著吃了一個,就下座來讓人,又出至外頭,令人盛兩盤子與趙姨娘周姨娘送去。又見鳳姐走來道:“你不慣張羅,你喫你的去。我先替你張羅,等散了我再喫。”湘雲不肯,又令人在那邊廊上擺了兩桌,讓鴛鴦、琥珀、彩霞、彩雲、平兒去坐。鴛鴦因向鳳姐笑道:“二奶奶在這裏伺候,我們可喫去了。”鳳姐兒道:“你們只管去,都交給我就是了。”說著,史湘雲仍入了席。鳳姐和李紈也胡亂應個景兒。鳳姐仍是下來張羅,一時出至廊上,鴛鴦等正喫的高興,見他來了,鴛鴦等站起來道:“奶奶又出來作什麼?讓我們也受用一會子。”鳳姐笑道:“鴛鴦小蹄子越發壞了,我替你當差,倒不領情,還抱怨我。還不快斟一鍾酒來我喝呢。”鴛鴦笑著忙斟了一杯酒,送至鳳姐脣邊,鳳姐一揚脖子吃了。平兒早剔了一殼黃子送來,鳳姐道:“多倒些姜醋。”一面也吃了,笑道:“你們坐著喫罷,我可去了。”鴛鴦笑道:“好沒臉,喫我們的東西。”鳳姐兒笑道:“你和我少作怪。你知道你璉二爺愛上了你,要和老太太討了你做小老婆呢。”鴛鴦道:“啐,這也是作奶奶說出來的話!我不拿腥手抹你一臉算不得。”說著趕來就要抹。鳳姐兒央道:“好姐姐,饒我這一遭兒罷。”琥珀笑道:“鴛丫頭要去了,平丫頭還饒他?你們看看他,沒有吃了兩個螃蟹,倒喝了一碟子醋,他也算不會攬酸了。”平兒手裏正掰了個滿黃的螃蟹,聽如此奚落他,便拿著螃蟹照著琥珀臉上抹來,口內笑罵“我把你這嚼舌根的小蹄子!”琥珀也笑著往旁邊一躲,平兒使空了,往前一撞,正恰恰的抹在鳳姐兒腮上。鳳姐兒正和鴛鴦嘲笑,不防唬了一跳,噯喲了一聲。衆人撐不住都哈哈的大笑起來。鳳姐也禁不住笑罵道:“死娼婦!喫離了眼了,混抹你孃的。” 平兒忙趕過來替他擦了,親自去端水。鴛鴦道:“阿彌陀佛!這是個報應。”賈母那邊聽見,一疊聲問:“見了什麼這樣樂,告訴我們也笑笑。”鴛鴦等忙高聲笑回道:“二奶奶來搶螃蟹喫,平兒惱了,抹了他主子一臉的螃蟹黃子。主子奴才打架呢。”賈母和王夫人等聽了也笑起來。賈母笑道:“你們看他可憐見的,把那小腿子臍子給他點子喫也就完了。”鴛鴦等笑著答應了,高聲又說道:“這滿桌子的腿子,二奶奶只管喫就是了。”鳳姐洗了臉走來,又伏侍賈母等吃了一回。黛玉獨不敢多喫,只吃了一點兒夾子肉就下來了。
\end{parag}


\begin{parag}
    賈母一時不吃了,大家方散,都洗了手,也有看花的,也有弄水看魚的,遊玩了一回。王夫人因回賈母說:“這裏風大,才又吃了螃蟹,老太太還是回房去歇歇罷了。若高興,明日再來逛逛。”賈母聽了,笑道:“正是呢。我怕你們高興,我走了又怕掃了你們的興。既這麼說,咱們就都去吧。”回頭又齦湘雲:“別讓你寶哥哥林姐姐多吃了。”湘雲答應著。又囑咐湘雲寶釵二人說:“你兩個也別多喫。那東西雖好喫,不是什麼好的,喫多了肚子疼。”二人忙應著送出園外,仍舊回來,令將殘席收拾了另擺。寶玉道:“也不用擺,咱們且作詩。把那大團圓桌就放在當中,酒菜都放著。也不必拘定坐位,有愛喫的大家去喫,散坐豈不便宜。”寶釵道:“這話極是。”湘雲道:“雖如此說,還有別人。”因又命另擺一桌,揀了熱螃蟹來,請襲人、紫鵑、司棋、侍書、入畫、鶯兒、翠墨等一處共坐。山坡桂樹底下鋪下兩條花氈,命答應的婆子並小丫頭等也都坐了,只管隨意喫喝,等使喚再來。
\end{parag}


\begin{parag}
    湘雲便取了詩題,用針綰在牆上。衆人看了,都說:“新奇固新奇,只怕作不出來。”湘雲又把不限韻的原故說了一番。寶玉道:“這纔是正理,我也最不喜限韻。”林黛玉因不大喫酒,又不喫螃蟹,自令人掇了一個繡墩倚欄坐著,拿著釣竿釣魚。寶釵手裏拿著一枝桂花玩了一回,俯在窗檻上掐了桂蕊擲向水面,引的游魚浮上來唼喋。湘雲出一回神,又讓一回襲人等,又招呼山坡下的衆人只管放量喫。探春和李紈惜春立在垂柳陰中看鷗鷺。迎春又獨在花陰下拿著花針穿茉莉花。\begin{note}庚雙夾:看他各人各式,亦如畫家有孤聳獨出則有攢三聚五,疏疏密密,直是一幅《百美圖》。\end{note}寶玉又看了一回黛玉釣魚,一回又俯在寶釵旁邊說笑兩句,一回又看襲人等喫螃蟹,自己也陪他飲兩口酒。襲人又剝一殼肉給他喫。黛玉放下釣竿,走至座間,拿起那烏銀梅花自斟壺來,\begin{note}庚雙夾:寫壺非寫壺,正寫黛玉。\end{note}揀了一個小小的海棠凍石蕉葉杯。\begin{note}庚雙夾:妙杯!非寫杯,正寫黛玉。“揀”字有神理,蓋黛玉不善飲,此任性也。\end{note}丫鬟看見,知他要飲酒,忙著走上來斟。黛玉道:“你們只管喫去,讓我自斟,這纔有趣兒。”說著便斟了半盞,看時卻是黃酒,因說道:“我吃了一點子螃蟹,覺得心口微微的疼,須得熱熱的喝口燒酒。”寶玉忙道:“有燒酒。”便令將那合歡花浸的酒燙一壺來。\begin{note}庚雙夾:傷哉!作者猶記矮□舫前以合歡花釀酒乎?屈指二十年矣。\end{note}黛玉也只吃了一口便放下了。寶釵也走過來,另拿了一隻杯來,也飲了一口,便蘸筆至牆上把頭一個《憶菊》勾了,底下又贅了一個“蘅”字。\begin{note}庚雙夾:妙極韻極!\end{note}寶玉忙道:“好姐姐,第二個我已經有了四句了,你讓我作罷。”寶釵笑道:“我好容易有了一首,你就忙的這樣。”黛玉也不說話,接過筆來把第八個《問菊》勾了,接著把第十一個《菊夢》也勾了,也贅一個“瀟”字。\begin{note}庚雙夾:這兩個妙題料定黛玉必喜,豈讓人作去哉?\end{note}寶玉也拿起筆來,將第二個《訪菊》也勾了,也贅上一個“絳”字。探春走來看看道:“竟沒有人作《簪菊》,讓我作這《簪菊》。”又指著寶玉笑道:“才宣過總不許帶出閨閣字樣來,你可要留神。”說著,只見史湘雲走來,將第四第五《對菊》《供菊》一連兩個都勾了,也贅上一個“湘”字。探春道:“你也該起個號。”湘雲笑道:“我們家裏如今雖有幾處軒館,我又不住著,借了來也沒趣。”\begin{note}庚雙夾:今之不讀書暴發戶偏愛起一別號。一笑。\end{note}寶釵笑道:“方纔老太太說,你們家也有這個水亭叫‘枕霞閣’,難道不是你的。如今雖沒了,你到底是舊主人。”衆人都道有理,寶玉不待湘雲動手,便代將“湘”字抹了,改了一個“霞”字。又有頓飯工夫,十二題已全,各自謄出來,都交與迎春,另拿了一張雪浪箋過來,一併謄錄出來,某人作的底下贅明某人的號。李紈等從頭看起:
\end{parag}


\begin{poem}
    \begin{pl}憶菊 \authorr{蘅蕪君}\end{pl}
    \begin{note}庚雙夾:真用此號,妙極!\end{note}

    \begin{pl}悵望西風抱悶思,蓼紅葦白斷腸時。\end{pl}

    \begin{pl}空籬舊圃秋無跡,瘦月清霜夢有知。\end{pl}

    \begin{pl}念念心隨歸雁遠,寥寥坐聽晚砧癡。\end{pl}

    \begin{pl}誰憐爲我黃花病,慰語重陽會有期。\end{pl}
    \emptypl

    \begin{pl}訪菊 \authorr{怡紅公子}\end{pl}

    \begin{pl}閒趁霜晴試一遊,酒杯藥盞莫淹留。\end{pl}

    \begin{pl}霜前月下誰家種,檻外籬邊何處秋。\end{pl}

    \begin{pl}蠟屐遠來情得得,冷吟不盡興悠悠。\end{pl}

    \begin{pl}黃花若解憐詩客,休負今朝掛杖頭。\end{pl}
    \emptypl

    \begin{pl}種菊 \authorr{怡紅公子}\end{pl}

    \begin{pl}攜鋤秋圃自移來,籬畔庭前故故栽。\end{pl}

    \begin{pl}昨夜不期經雨活,今朝猶喜帶霜開。\end{pl}

    \begin{pl}冷吟秋色詩千首,醉酹寒香酒一杯。\end{pl}

    \begin{pl}泉溉泥封勤護惜,好知井逕絕塵埃。\end{pl}
    \emptypl

    \begin{pl}對菊 \authorr{枕霞舊友}\end{pl}

    \begin{pl}別圃移來貴比金,一叢淺淡一叢深。\end{pl}

    \begin{pl}蕭疏籬畔科頭坐,清冷香中抱膝吟。\end{pl}

    \begin{pl}數去更無君傲世,看來惟有我知音。\end{pl}

    \begin{pl}秋光荏苒休辜負,相對原宜惜寸陰。\end{pl}
    \emptypl

    \begin{pl}供菊 \authorr{枕霞舊友}\end{pl}

    \begin{pl}彈琴酌酒喜堪儔,几案婷婷點綴幽。\end{pl}

    \begin{pl}隔座香分三徑露,拋書人對一枝秋。\end{pl}

    \begin{pl}霜清紙帳來新夢,圃冷斜陽憶舊遊。\end{pl}

    \begin{pl}傲世也因同氣味,春風桃李未淹留。\end{pl}
    \emptypl

    \begin{pl}詠菊 \authorr{瀟湘妃子}\end{pl}

    \begin{pl}無賴詩魔昏曉侵,繞籬欹石自沉音。\end{pl}

    \begin{pl}毫端蘊秀臨霜寫,口齒噙香對月吟。\end{pl}

    \begin{pl}滿紙自憐題素怨,片言誰解訴秋心。\end{pl}

    \begin{pl}一從陶令平章後,千古高風說到今。\end{pl}
    \emptypl

    \begin{pl}畫菊 \authorr{蘅蕪君}\end{pl}

    \begin{pl}詩餘戲筆不知狂,豈是丹青費較量。\end{pl}

    \begin{pl}聚葉潑成千點墨,攢花染出幾霜痕。\end{pl}

    \begin{pl}淡濃神會風前影,跳脫秋生腕底香。\end{pl}

    \begin{pl}莫認東籬閒採掇,粘屏聊以慰重陽。\end{pl}
    \emptypl

    \begin{pl}問菊 \authorr{瀟湘妃子}\end{pl}

    \begin{pl}欲訊秋情衆莫知,喃喃負手叩東籬。\end{pl}

    \begin{pl}孤標傲世偕誰隱,一樣花開爲底遲?\end{pl}

    \begin{pl}圃露庭霜何寂寞,雁歸蛩病可相思?\end{pl}

    \begin{pl}休言舉世無談者,解語何妨片語時。\end{pl}
    \emptypl

    \begin{pl}簪菊 \authorr{蕉下客}\end{pl}

    \begin{pl}瓶供籬栽日日忙,折來休認鏡中妝。\end{pl}

    \begin{pl}長安公子因花癖,彭澤先生是酒狂。\end{pl}

    \begin{pl}短鬢冷沾三徑露,葛巾香染九秋霜。\end{pl}

    \begin{pl}高情不入時人眼,拍手憑他笑路旁。\end{pl}
    \emptypl

    \begin{pl}菊影 \authorr{枕霞舊友}\end{pl}

    \begin{pl}秋光疊疊復重重,潛度偷移三徑中。\end{pl}

    \begin{pl}窗隔疏燈描遠近,籬篩破月鎖玲瓏。\end{pl}

    \begin{pl}寒芳留照魂應駐,霜印傳神夢也空。\end{pl}

    \begin{pl}珍重暗香休踏碎,憑誰醉眼認朦朧。\end{pl}
    \emptypl

    \begin{pl}菊夢 \authorr{瀟湘妃子}\end{pl}

    \begin{pl}籬畔秋酣一覺清,和雲伴月不分明。\end{pl}

    \begin{pl}登仙非慕莊生蝶,憶舊還尋陶令盟。\end{pl}

    \begin{pl}睡去依依隨雁斷,驚回故故惱蛩鳴。\end{pl}

    \begin{pl}醒時幽怨同誰訴,衰草寒煙無限情。\end{pl}
    \emptypl

    \begin{pl}殘菊 \authorr{蕉下客}\end{pl}

    \begin{pl}露凝霜重漸傾欹,宴賞才過小雪時。\end{pl}

    \begin{pl}蒂有餘香金淡泊,枝無全葉翠離披。\end{pl}

    \begin{pl}半牀落月蛩聲病,萬里寒雲雁陣遲。\end{pl}

    \begin{pl}明歲秋風知再會,暫時分手莫相思。\end{pl}
\end{poem}


\begin{parag}
    衆人看一首,贊一首,彼此稱揚不已。李紈笑道:“等我從公評來。通篇看來,各有各人的警句。今日公評:《詠菊》第一,《問菊》第二,《菊夢》第三,題目新,詩也新,立意更新,惱不得要推瀟湘妃子爲魁了;然後《簪菊》《對菊》《供菊》《畫菊》《憶菊》次之。”寶玉聽說,喜的拍手叫“極是,極公道。”黛玉道:“我那首也不好,到底傷於纖巧些。”李紈道:“巧的卻好,不露堆砌生硬。”黛玉道:“據我看來,頭一句好的是‘圃冷斜陽憶舊遊’,這句背面傅粉。‘拋書人對一枝秋’已經妙絕,將供菊說完,沒處再說,故翻回來想到未折未供之先,意思深透。”李紈笑道:“固如此說,你的‘口齒噙香’句也敵的過了。”探春又道:“到底要算蘅蕪君沉著,‘秋無跡’、‘夢有知’,把個憶字竟烘染出來了。”寶釵笑道:“你的‘短鬢冷沾’、‘葛巾香染’,也就把簪菊形容的一個縫兒也沒了。”湘雲道:“‘偕誰隱’、‘爲底遲’,真個把個菊花問的無言可對。”李紈笑道:“你的‘科頭坐’、‘抱膝吟’,竟一時也不能別開,菊花有知,也必膩煩了。”說的大家都笑了。寶玉笑道:“我又落第。難道‘誰家種’、‘何處秋’、‘蠟屐遠來’、‘冷吟不盡’,都不是訪,‘昨夜雨’、‘今朝霜’,都不是種不成?但恨敵不上‘口齒噙香對月吟’、‘清冷香中抱膝吟’、‘短鬢’、‘葛巾’、‘金淡泊’、‘翠離披’、‘秋無跡’、‘夢有知’這幾句罷了。”\begin{note}庚雙夾:總寫寶玉不及,妙極!\end{note}又道:“明兒閒了,我一個人作出十二首來。”李紈道:“你的也好,只是不及這幾句新巧就是了。”
\end{parag}


\begin{parag}
    大家又評了一回,復又要了熱蟹來,就在大圓桌子上吃了一回。寶玉笑道:“今日持螯賞桂,亦不可無詩。\begin{note}庚雙夾:全是他忙,全是他不及。妙極!\end{note}我已吟成,誰還敢作呢?”說著,便忙洗了手提筆寫出。\begin{note}庚雙夾:且莫看詩,只看他偏於如許一大回詩後又寫一回詩,豈世人想得到的?\end{note}衆人看道:
\end{parag}


\begin{poem}
    \begin{pl}持螯更喜桂陰涼,潑醋擂姜興欲狂。\end{pl}

    \begin{pl}饕餮王孫應有酒,橫行公子卻無腸。\end{pl}

    \begin{pl}臍間積冷饞忘忌,指上沾腥洗尚香。\end{pl}

    \begin{pl}原爲世人美口腹,坡仙曾笑一生忙。\end{pl}


\end{poem}


\begin{parag}
    黛玉笑道:“這樣的詩,要一百首也有。”\begin{note}庚雙夾:看他這一說。\end{note}寶玉笑道:“你這會子才力已盡,不說不能作了,還貶人家。”黛玉聽了,並不答言,也不思索,提起筆來一揮,已有了一首。衆人看道:
\end{parag}


\begin{poem}
    \begin{pl}鐵甲長戈死未忘,堆盤色相喜先嚐。\end{pl}

    \begin{pl}螯封嫩玉雙雙滿,殼凸紅脂塊塊香。\end{pl}

    \begin{pl}多肉更憐卿八足,助情誰勸我千觴。\end{pl}

    \begin{pl}對斯佳品酬佳節,桂拂清風菊帶霜。\end{pl}


\end{poem}


\begin{parag}
    寶玉看了正喝彩,黛玉便一把撕了,令人燒去,因笑道:“我的不及你的,我燒了他。你那個很好,比方纔的菊花詩還好,你留著他給人看。”寶釵接著笑道:“我也勉強了一首,未必好,寫出來取笑兒罷。”說著也寫了出來。大家看時,寫道是:
\end{parag}


\begin{poem}
    \begin{pl}桂靄桐陰坐舉觴,長安涎口盼重陽。\end{pl}

    \begin{pl}眼前道路無經緯,皮裏春秋空黑黃。\end{pl}

\end{poem}


\begin{parag}
    看到這裏,衆人不禁叫絕。寶玉道:“寫得痛快!我的詩也該燒了。”又看底下道:
\end{parag}


\begin{poem}
    \begin{pl}酒未敵腥還用菊,性防積冷定須姜。\end{pl}

    \begin{pl}於今落釜成何益,月浦空餘禾黍香。\end{pl}

\end{poem}


\begin{parag}
    衆人看畢,都說這是食螃蟹絕唱,這些小題目,原要寓大意纔算是大才,只是諷刺世人太毒了些。說著,只見平兒復進園來。不知作什麼,且聽下回分解。
\end{parag}


\begin{parag}
    \begin{note}蒙回末總:請看此回中,閨中女兒能作此等豪情韻事,且筆下各能自畫其性情,毫不乖舛。作者之錦繡口,無庸贅續。其用意之深,獎勵之勤,都此文者亦不得輕忽戒之也。\end{note}
\end{parag}
