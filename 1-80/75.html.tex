\chap{七十五}{开夜宴异兆发悲音 赏中秋新词得佳谶}


\begin{parag}
    \begin{note}蒙回前总:贾珍居长,不能承先启后丕震家风,兄弟问柳寻花,父子呼幺喝六,贾氏宗风,其坠地矣。安得不发先灵一叹!\end{note}
\end{parag}


\begin{parag}
    \begin{note}庚辰:乾隆二十一年五月初七日对清。\end{note}
\end{parag}


\begin{parag}
    \begin{note}庚辰:缺中秋诗俟雪芹。\end{note}
\end{parag}


\begin{parag}
    话说尤氏从惜春处赌气出来,正欲往王夫人处去。跟从的老嬷嬷们因悄悄的回道:“奶奶且别往上房去。才有甄家的几个人来,还有些东西,不知是作什么机密事。奶奶这一去恐不便。”尤氏听了道:“昨日听见你爷说,看邸报甄家犯了罪,现今抄没家私,调取进京治罪。怎么又有人来?”老嬷嬷道:“正是呢。才来了几个女人,气色不成气色,慌慌张张的,想必有什么瞒人的事情也是有的。”
\end{parag}


\begin{parag}
    尤氏听了,便不往前去,仍往李氏这边来了。\begin{note}庚辰双夹:前文有探春一语,过至此回又用尤氏陪点,且轻轻淡染出甄家事故,此画家未落墨之法也。\end{note}恰好太医才诊了脉去。李纨近日也略觉精爽了些,拥衾倚枕,坐在床上,正欲一二人来说些闲话。因见尤氏进来不似往日和蔼可亲,只呆呆的坐著。李纨因问道:“你过来了这半日,可在别屋里吃些东西没有?只怕饿了。”命素云瞧有什么新鲜点心拣了来。尤氏忙止道:“不必,不必。你这一向病著,那里有什么新鲜东西。况且我也不饿。”李纨道:“昨日他姨娘家送来的好茶面子,倒是对碗来你喝罢。”说毕,便吩咐人去对茶。尤氏出神无语。跟来的丫头媳妇们因问:“奶奶今日中晌尚未洗脸,这会子趁便可净一净好?”尤氏点头。李纨忙命素云来取自己的妆奁。素云一面取来,一面将自己的胭粉拿来,笑道:“我们奶奶就少这个。奶奶不嫌脏,这是我的,能著用些。”李纨道:“我虽没有,你就该往姑娘们那里取去。怎么公然拿出你的来。幸而是他,若是别人,岂不恼呢。”尤氏笑道:“这又何妨。自来我凡过来,谁的没使过,今日忽然又嫌脏了?”一面说,一面盘膝坐在炕沿上。银蝶上来忙代为卸去腕镯戒指,又将一大袱手巾盖在下截,将衣裳护严。小丫鬟炒豆儿捧了一大盆温水走至尤氏跟前,只弯腰捧著。李纨道:“怎么这样没规矩。”银蝶笑道:“说一个个没机变的,说一个葫芦就是一个瓢。奶奶不过待咱们宽些,在家里不管怎样罢了,你就得了意,不管在家出外,当著亲戚也只随著便了。”尤氏道:“你随他去罢,横竖洗了就完事了。”炒豆儿忙赶著跪下。尤氏笑道:“我们家下大小的人只会讲外面假礼假体面,究竟作出来的事都够使的了。”\begin{note}庚辰双夹:按尤氏犯七出之条不过只是“过于从夫”四字,此世间妇人之常情耳。其心术慈厚宽顺竟可出于阿凤之上,特用之明犯七出之人从公一论,可知贾宅中暗犯七出之人亦不少。似明犯者犹 慑端。其饰己非而扬人恶者,阴昧僻谲之流,实不能容于世者也。此为打草惊蛇法,实写邢夫人也。\end{note}李纨听如此说,便知他已知道昨夜的事,因笑道:“你这话有因,谁作事究竟够使了?”尤氏道:“你倒问我!你敢是病著死过去了!”
\end{parag}


\begin{parag}
    一语未了,只见人报:“宝姑娘来了。”忙说快请时,宝钗已走进来。尤氏忙擦脸起身让坐,因问:“怎么一个人忽然走来,别的姊妹都怎么不见?”宝钗道: “正是我也没有见他们。只因今日我们奶奶身上不自在,家里两个女人也都因时症未起炕,别的靠不得,我今儿要出去伴著老人家夜里作伴儿。要去回老太太,太太,我想又不是什么大事,且不用提,等好了我横竖进来的,所以来告诉大嫂子一声。”李纨听说,只看著尤氏笑。尤氏也只看著李纨笑。一时尤氏盥沐已毕,大家吃面茶。李纨因笑道:“既这样,且打发人去请姨娘的安,问是何病。我也病著,不能亲自来的。好妹妹,你去只管去,我自打发人去到你那里去看屋子。你好歹住一两天还进来,别叫我落不是。”宝钗笑道:“落什么不是呢,这也是通共常情,你又不曾卖放了贼。依我的主意,也不必添人过去,竟把云丫头请了来,你和他住一两日,岂不省事。”尤氏道:“可是史大妹妹往那里去了?”宝钗道:“我才打发他们找你们探丫头去了,叫他同到这里来,我也明白告诉他。”
\end{parag}


\begin{parag}
    正说著,果然报:“云姑娘和三姑娘来了。”大家让坐已毕,宝钗便说要出去一事,探春道:“很好。不但姨妈好了还来的,就便好了不来也使得。”尤氏笑道:“这话奇怪,怎么撵起亲戚来了?”探春冷笑道:“正是呢,有叫人撵的,不如我先撵。亲戚们好,也不在必要死住著才好。咱们倒是一家子亲骨肉呢,一个个不象乌眼鸡,恨不得你吃了我,我吃了你!”尤氏忙笑道:“我今儿是那里来的晦气,偏都碰著你姊妹们的气头儿上了。”探春道:“谁叫你赶热灶来了!”因问: “谁又得罪了你呢?”因又寻思道:“四丫头不犯罗唣你,却是谁呢?”尤氏只含糊答应。探春知他畏事不肯多言,因笑道:“你别装老实了。除了朝廷治罪,没有砍头的,你不必畏头畏尾。实告诉你罢,我昨日把王善保家那老婆子打了,我还顶著个罪呢。不过背地里说我些闲话,难道他还打我一顿不成!”宝钗忙问因何又打他,探春悉把昨夜怎的抄检,怎的打他,一一说了出来。尤氏见探春已经说了出来,便把惜春方才之事也说了出来。探春道:“这是他的僻性,孤介太过,我们再傲不过他的。”又告诉他们说:“今日一早不见动静,打听凤辣子又病了。我就打发我妈妈出去打听王善保家的是怎样。回来告诉我说,王善保家的挨了一顿打,大太太嗔著他多事。”尤氏李纨道:“这倒也是正理。”探春冷笑道:“这种掩饰谁不会作,且再瞧就是了。”尤氏李纨皆默无所答。一时估著前头用饭,湘云和宝钗回房打点衣衫,不在话下。
\end{parag}


\begin{parag}
    尤氏等遂辞了李纨,往贾母这边来。贾母歪在榻上,王夫人说甄家因何获罪,如今抄没了家产,回京治罪等语。贾母听了正不自在,恰好见他姊妹来了,因问: “从那里来的?可知凤姐妯娌两个的病今日怎样?”尤氏等忙回道:“今日都好些。”贾母点头叹道:“咱们别管人家的事,且商量咱们八月十五日赏月是正经。”\begin{note}庚辰双夹:贾母已看破狐悲兔死,故不改正,聊来自遣耳。\end{note}王夫人笑道:“都已预备下了。不知老太太拣那里好,只是园里空,夜晚风冷。”贾母笑道: “多穿两件衣服何妨,那里正是赏月的地方,岂可倒不去的。”说话之间,早有媳妇丫鬟们抬过饭桌来,王夫人尤氏等忙上来放箸捧饭。贾母见自己的几色菜已摆完,另有两大捧盒内捧了几色菜来,便知是各房另外孝敬的旧规矩。贾母因问:“都是些什么?上几次我就吩咐,如今可以把这些蠲了罢,你们还不听。如今比不得在先辐辏的时光了。”鸳鸯忙道:“我说过几次,都不听,也只罢了。” 王夫人笑道:“不过都是家常东西。今日我吃斋没有别的。那些面筋豆腐老太太又不大甚爱吃,只拣了一样椒油莼齑酱来。”贾母笑道:“这样正好,正想这个吃。” 鸳鸯听说,便将碟子挪在跟前。宝琴一一的让了,方归坐。贾母便命探春来同吃。探春也都让过了,便和宝琴对面坐下。待书忙去取了碗来。鸳鸯又指那几样菜道: “这两样看不出是什么东西来,大老爷送来的。这一碗是鸡髓笋,是外头老爷送上来的。”一面说,一面就只将这碗笋送至桌上。贾母略尝了两点,便命:“将那两样著人送回去,就说我吃了。以后不必天天送,我想吃自然来要。”媳妇们答应著,仍送过去,不在话下。
\end{parag}


\begin{parag}
    贾母因问:“有稀饭吃些罢了。”尤氏早捧过一碗来,说是红稻米粥。贾母接来吃了半碗,便吩咐:“将这粥送给凤哥儿吃去,”又指著“这一碗笋和这一盘风腌果子狸给颦儿宝玉两个吃去,那一碗肉给兰小子吃去。”又向尤氏道:“我吃了,你就来吃了罢。”尤氏答应,待贾母漱口洗手毕,贾母便下地和王夫人说闲话行食。尤氏告坐。探春宝琴二人也起来了,笑道:“失陪,失陪。”尤氏笑道:“剩我一个人,大排桌的吃不惯。”贾母笑道:“鸳鸯琥珀来趁势也吃些,又作了陪客。”尤氏笑道:“好,好,好,我正要说呢。”贾母笑道:“看著多多的人吃饭,最有趣的。”又指银蝶道:“这孩子也好,也来同你主子一块来吃,等你们离了我,再立规矩去。”尤氏道:“快过来,不必装假。”贾母负手看著取乐。因见伺候添饭的人手内捧著一碗下人的米饭,尤氏吃的仍是白粳米饭,贾母问道:“你怎么昏了,盛这个饭来给你奶奶。”那人道:“老太太的饭吃完了。今日添了一位姑娘,所以短了些。”鸳鸯道:“如今都是可著头做帽子了,要一点儿富余也不能的。”王夫人忙回道:“这一二年旱涝不定,田上的米都不能按数交的。这几样细米更艰难了,所以都可著吃的多少关去,生恐一时短了,买的不顺口。”贾母笑道:“这正是‘巧媳妇做不出没米的粥’来。”众人都笑起来。鸳鸯道:“既这然,就去把三姑娘的饭拿来添也是一样,就这样笨。”尤氏笑道:“我这个就够了,也不用取去。”鸳鸯道:“你够了,我不会吃的。”地下的媳妇们听说,方忙著取去了。\begin{note}庚辰双夹:总伏下文。\end{note}一时王夫人也去用饭,这里尤氏直陪贾母说话取笑。
\end{parag}


\begin{parag}
    到起更的时候,贾母说:“黑了,过去罢。”尤氏方告辞出来。走至大门前上了车,银蝶坐在车沿上。众媳妇放下帘子来,便带著小丫头们先直走过那边大门口等著去了。因二府之门相隔没有一箭之路,每日家常来往不必定要周备,况天黑夜晚之间回来的遭数更多,所以老嬷嬷带著小丫头,只几步便走了过来。两边大门上的人都到东西街口,早把行人断住。尤氏大车上也不用牲口,只用七八个小厮挽环拽轮,轻轻的便推拽过这边阶矶上来。于是众小厮退过狮子以外,众嬷嬷打起帘子,银蝶先下来,然后搀下尤氏来。大小七八个灯笼照的十分真切。尤氏因见两边狮子下放著四五辆大车,便知系来赴赌之人所乘,遂向银蝶众人道:“你看,坐车的是这样,骑马的还不知有几个呢。马自然在圈里拴著,咱们看不见。也不知道他娘老子挣下多少钱与他们,这么开心儿。”一面说,一面已到了厅上。贾蓉之妻带领家下媳妇丫头们,也都秉烛接了出来。尤氏笑道:“成日家我要偷著瞧瞧他们,也没得便。今儿倒巧,就顺便打他们窗户跟前走过去。”众媳妇答应著,提灯引路,又有一个先去悄悄的知会伏侍的小厮们不要失惊打怪。于是尤氏一行人悄悄的来至窗下,只听里面称三赞四,耍笑之音虽多,\begin{note}庚辰双夹:妙!先画赢家。\end{note}又兼有恨五骂六,忿怨之声亦不少。\begin{note}庚辰双夹:妙!又画输家。\end{note}
\end{parag}


\begin{parag}
    原来贾珍近因居丧,每不得游顽旷荡,又不得观优闻乐作遣。无聊之极,便生了个破闷之法。日间以习射为由,请了各世家弟兄及诸富贵亲友来较射。因说: “白白的只管乱射,终无裨益,不但不能长进,而且坏了式样,必须立个罚约,赌个利物,大家才有勉力之心。”因此在天香楼下箭道内立了鹄子,皆约定每日早饭后来射鹄子。贾珍不肯出名,便命贾蓉作局家。这些来的皆系世袭公子,人人家道丰富,且都在少年,正是斗鸡走狗,问柳评花的一干游荡纨裤。因此大家议定,每日轮流作晚饭之主,──每日来射,不便独扰贾蓉一人之意。于是天天宰猪割羊,屠鹅戮鸭,好似临潼斗宝一般,都要卖弄自己的好厨役好烹炮。不到半月工夫,贾赦贾政听见这般,不知就里,反说这才是正理,文既误矣,武事当亦该习,况在武荫之属。两处遂也命贾环、贾琮、宝玉、贾兰等四人于饭后过来,跟著贾珍习射一回,方许回去。
\end{parag}


\begin{parag}
    贾珍之志不在此,再过一二日便渐次以歇臂养力为由,晚间或抹抹骨牌,赌个酒东而已,至后渐次至钱。如今三四月的光景,竟一日一日赌胜于射了,公然斗叶掷骰,放头开局,夜赌起来。家下人借此各有些进益,巴不得的如此,所以竟成了势了。外人皆不知一字。近日邢夫人之胞弟邢德全也酷好如此,故也在其中。又有薛蟠,头一个惯喜送钱与人的,见此岂不快乐。邢德全虽系邢夫人之胞弟,却居心行事大不相同。这个邢德全只知吃酒赌钱,眠花宿柳为乐,手中滥漫使钱,待人无二心,好酒者喜之,不饮者则不去亲近,无论上下主仆皆出自一意,并无贵贱之分,因此都唤他“傻大舅”。薛蟠早已出名的呆大爷。今日二人皆凑在一处,都爱 “抢新快”爽利,便又会了两家,在外间炕上“抢新快”。别的又有几家在当地下大桌上打公番。里间又一起斯文些的,抹骨牌打天九。此间伏侍的小厮都是十五岁以下的孩子,若成丁的男子到不了这里,故仁戏角至窗外偷看。其中有两个十六七岁娈童以备奉酒的,都打扮的粉妆玉琢。今日薛蟠又输了一张,正没好气,幸而掷第二张完了,算来除翻过来倒反赢了,心中只是兴头起来。贾珍道:“且打住,吃了东西再来。”因问那两处怎样。里头打天九的,也作了帐等吃饭。打公番的未清,且不肯吃。于是各不能催,先摆下一大桌,贾珍陪著吃,命贾蓉落后陪那一起。薛蟠兴头了,便搂著一个娈童吃酒,又命将酒去敬邢傻舅。傻舅输家,没心绪,吃了两碗,便有些醉意,嗔著两个娈童只赶著赢家不理输家了,因骂道:“你们这起兔子,就是这样专洑上水。天天在一处,谁的恩你们不沾,只不过我这一会子输了几两银子,你们就三六九等了。难道从此以后再没有求著我们的事了!”众人见他带酒,忙说:“很是,很是。果然他们风俗不好。”因喝命:“快敬酒赔罪。”两个娈童都是演就的局套,忙都跪下奉酒,说:“我们这行人,师父教的不论远近厚薄,只看一时有钱有势就亲敬,便是活佛神仙,一时没了钱势了,也不许去理他。况且我们又年轻,又居这个行次,求舅太爷体恕些我们就过去了。”\begin{note}庚辰双夹:调侃,骂死世人。庚辰眉:此一段娈童语句太真,反不得其为钱为势之神,当改作委曲认罪语方妥。\end{note}说著,便举著酒俯膝跪下。邢大舅心内虽软了,只还故作怒意不理。众人又劝道:“这孩子是实情话。老舅是久惯怜香惜玉的,如何今日反这样起来?若不吃这酒,他两个怎样起来。”邢大舅已撑不住了,便说道:“若不是众位说,我再不理。”说著,方接过来一气喝干了。又斟一碗来。这邢大舅便酒勾往事,醉露真情起来,乃拍案对贾珍叹道:“怨不的他们视钱如命。多少世宦大家出身的,若提起‘钱势’二字,连骨肉都不认了。老贤甥,昨日我和你那边的令伯母赌气,你可知道否?” 贾珍道:“不曾听见。”邢大舅叹道:“就为钱这件混帐东西。利害,利害!”贾珍深知他与邢夫人不睦,每遭邢夫人弃恶,扳出怨言,因劝道:“老舅,你也太散漫些。若只管花去,有多少给老舅花的。”邢大舅道:“老贤甥,你不知我邢家底里。我母亲去世时我尚小,世事不知。他姊妹三个人,只有你令伯母年长出阁,一分家私都是他把持带来。如今二家姐虽也出阁,他家也甚艰窘,三家姐尚在家里,一应用度都是这里陪房王善保家的掌管。我便来要钱,也非要的是你贾府的,我邢家家私也就够我花了。无奈竟不得到手,所以有冤无处诉。”\begin{note}庚辰双夹:众恶之必察也。今邢夫人一人,贾母先恶之,恐贾母心偏,亦可解之。若贾琏阿凤之怨,恐儿女之私,亦可解之。若探春之怒,恐女子不识大而知小,亦可解之。今又忽用乃弟一怨,吾不知将又何如矣。\end{note}贾珍见他酒后叨叨,恐人听见不雅,连忙用话解劝。
\end{parag}


\begin{parag}
    外面尤氏听得十分真切,乃悄向银蝶笑道:“你听见了?这是北院里大太太的兄弟抱怨他呢。可怜他亲兄弟还是这样说,这就怨不得这些人了。”因还要听时,正值打公番者也歇住了,要吃酒。因有一个问道:“方才是谁得罪了老舅,我们竟不曾听明白,且告诉我们评评理。”邢德全见问,便把两个娈童不理输的只赶赢的话说了一遍。这一个年少的纨裤道:“这样说,原可恼的,怨不得舅太爷生气。我且问你两个:舅太爷虽然输了,输的不过是银子钱,并没有输丢了,怎就不理他了?”说著,众人大笑起来,连邢德全也喷了一地饭。尤氏在外面那牡啐了一口,骂道:“你听听,这一起子没廉耻的小挨刀的,才丢了脑袋骨子,就胡唚嚼毛了。再肏攮下黄汤去,还不知唚出些什么来呢。”一面说,一面便进去卸妆安歇。至四更时,贾珍方散,往佩凤房里去了。
\end{parag}


\begin{parag}
    次日起来,就有人回西瓜月饼都全了,只待分派送人。贾珍吩咐佩凤道:“你请你奶奶看著送罢,我还有别的事呢。”佩凤答应去了,回了尤氏,尤氏只得一一分派遣人送去。一时佩凤又来说:“爷问奶奶,今儿出门不出?说咱们是孝家,明儿十五过不得节,今儿晚上倒好,可以大家应个景儿,吃些瓜饼酒。”尤氏道: “我倒不愿出门呢。那边珠大奶奶又病了,凤丫头又睡倒了,我再不过去,越发没个人了。况且又不得闲,应什么景儿。”佩凤道:“爷说了,今儿已辞了众人,直等十六才来呢,好歹定要请奶奶吃酒的。”尤氏笑道:“请我,我没的还席。”佩凤笑著去了,一时又来笑道:“爷说,连晚饭也请奶奶吃,好歹早些回来,叫我跟了奶奶去呢。”尤氏道:“这样,早饭吃什么?快些吃了,我好走。”佩凤道:“爷说早饭在外头吃,请奶奶自己吃罢。”尤氏问道:“今日外头有谁?”佩凤道: “听见说外头有两个南京新来的,倒不知是谁。”说话之间,贾蓉之妻也梳妆了来见过。少时摆上饭来,尤氏在上,贾蓉之妻在下相陪,婆媳二人吃毕饭。尤氏便换了衣服,仍过荣府来,至晚方回去。
\end{parag}


\begin{parag}
    果然贾珍煮了一口猪,烧了一腔羊,余者桌菜及果品之类,不可胜记,就在会芳园丛绿堂中,屏开孔雀,褥设芙蓉,带领妻子姬妾。先饭后酒,开怀赏月作乐。将一更时分,真是风清月朗,上下如银。贾珍因要行令,尤氏便叫佩凤等四个人也都入席,下面一溜坐下,猜枚划拳,饮了一回。贾珍有了几分酒,益发高兴,便命取了一竿紫竹箫来,命佩凤吹箫,文花唱曲,喉清嗓嫩,真令人魄醉魂飞。唱罢复又行令。那天将有三更时分,贾珍酒已八分。大家正添衣饮茶,换盏更酌之际,忽听那边墙下有人长叹之声。大家明明听见,都悚然疑畏起来。\begin{note}庚辰双夹:余亦悚然疑畏。\end{note}贾珍忙厉声叱咤,问:“谁在那里?”连问几声,没有人答应。尤氏道:“必是墙外边家里人也未可知。”贾珍道:“胡说。这墙四面皆无下人的房子,况且那边又紧靠著祠堂,\begin{note}庚辰双夹:奇绝神想,余更为之惧矣。\end{note}焉得有人。”一语未了,只听得一阵风声,竟过墙去了。恍惚闻得祠堂内槅扇开阖之声。只觉得风气森森,比先更觉凉飒起来,月色惨淡,也不似先明朗。众人都觉毛发倒竖。贾珍酒已醒了一半,只比别人撑持得住些,心下也十分疑畏,便大没兴头起来。勉强又坐了一会子,就归房安歇去了。次日一早起来,乃是十五日,带领众子侄开祠堂行朔望之礼,细查祠内,都仍是照旧好好的,并无怪异之迹。贾珍自为醉后自怪,也不提此事。礼毕,仍闭上门,看著锁禁起来。\begin{note}庚辰双夹:未写荣府庆中秋,却先写宁府开夜宴,未写荣府数尽,先写宁府异道。盖宁乃家宅,凡有关于吉凶者,故必先示之。且列祖祠在此,岂无得而警乎?凡人先人虽远,然气运相关,必有之理也。非宁府之祖独有感应也。\end{note}
\end{parag}


\begin{parag}
    贾珍夫妻至晚饭后方过荣府来。只见贾赦贾政都在贾母房内坐著说闲话,与贾母取笑。贾琏,宝玉,贾环,贾兰皆在地下侍立。贾珍来了,都一一见过。说了两句话后,贾母命坐,贾珍方在近门小杌子上告了坐,警身侧坐。贾母笑问道:“这两日你宝兄弟的箭如何了?”贾珍忙起身笑道:“大长进了,不但样式好,而且弓也长了一个力气。”贾母道:“这也够了,且别贪力,仔细努伤。”贾珍忙答应几个“是”。贾母又道:“你昨日送来的月饼好,西瓜看著好,打开却也罢了。”贾珍笑道:“月饼是新来的一个专做点心的厨子,我试了试果然好,才敢做了孝敬。西瓜往年都还可以,不知今年怎么就不好了。”贾政道:“大约今年雨水太勤之故。”贾母笑道:“此时月已上了,咱们且去上香。”说著,便起身扶著宝玉的肩,带领众人齐往园中来。
\end{parag}


\begin{parag}
    当下园之正门俱已大开,吊著羊角大灯。嘉荫堂前月台上,焚著斗香,秉著风烛,陈献著瓜饼及各色果品。邢夫人等一干女客皆在里面久候。真是月明灯彩,人气香烟,晶艳氤氲,不可形状。地下铺著拜毯锦褥。贾母盥手上香拜毕,于是大家皆拜过。贾母便说:“赏月在山上最好。”因命在那山脊上的大厅上去。众人听说,就忙著在那里去铺设。贾母且在嘉荫堂中吃茶少歇,说些闲话。一时,人回:“都齐备了。”贾母方扶著人上山来。王夫人等因说:“恐石上苔滑,还是坐竹椅上去。”贾母道:“天天有人打扫,况且极平稳的宽路,何必不疏散疏散筋骨。”于是贾赦贾政等在前导引,又是两个老婆子秉著两把羊角手罩,鸳鸯、琥珀、尤氏等贴身搀扶,邢夫人等在后围随,从下逶迤而上,不过百余步,至山之峰脊上,便是这座敞厅。因在山之高脊,故名曰凸碧山庄。于厅前平台上列下桌椅,又用一架大围屏隔作两间。凡桌椅形式皆是圆的,特取团圆之意。上面居中贾母坐下,左垂首贾赦、贾珍、贾琏、贾蓉,右垂首贾政、宝玉、贾环、贾兰,团团围坐。只坐了半壁,下面还有半壁余空。贾母笑道:“常日倒还不觉人少,今日看来,还是咱们的人也甚少,算不得甚么。\begin{note}庚辰双夹:未饮先感人丁,总是将散之兆。\end{note}想当年过的日子,到今夜男女三四十个,何等热闹。今日就这样,太少了。待要再叫几个来,他们都是有父母的,家里去应景,不好来的。如今叫女孩们来坐那边罢。”于是令人向围屏后邢夫人等席上将迎春,探春,惜春三个请出来。贾琏宝玉等一齐出坐,先尽他姊妹坐了,然后在下方依次坐定。贾母便命折一枝桂花来,命一媳妇在屏后击鼓传花。若花到谁手中,饮酒一杯,罚说笑话一个。\begin{note}庚辰双夹:不犯前几次饮酒。\end{note}于是先从贾母起,次贾赦,一一接过。鼓声两转,恰恰在贾政手中住了,\begin{note}庚辰双夹:奇妙!偏在政老手中,竟能使政老一谑,真大文章矣。\end{note}只得饮了酒。众姊妹弟兄皆你悄悄的扯我一下,我暗暗的又捏你一把,都含笑倒要听是何笑话。\begin{note}庚辰双夹:余也要细听。\end{note}贾政见贾母喜悦,只得承欢。方欲说时,贾母又笑道:“若说的不笑了,还要罚。”贾政笑道:“只得一个,说来不笑,也只好受罚了。”因笑道:“一家子一个人最怕老婆的。”才说了一句,大家都笑了。因从不曾见贾政说过笑话,所以才笑。\begin{note}庚辰双夹:是极,摹神之至。\end{note}贾母笑道:“这必是好的。”贾政笑道:“若好,老太太多吃一杯。”贾母笑道:“自然。”贾政又说道:“这个怕老婆的人从不敢多走一步。偏是那日是八月十五,到街上买东西,便遇见了几个朋友,死活拉到家里去吃酒。不想吃醉了,便在朋友家睡著了,第二日才醒,后悔不及,只得来家赔罪。他老婆正洗脚,说:‘既是这样,你替我舔舔就饶你。’这男人只得给他舔,未免恶心要吐。他老婆便恼了,要打,说:‘你这样轻狂!’唬得他男人忙跪下求说:‘并不是奶奶的脚脏。只因昨晚吃多了黄酒,又吃了几块月饼馅子,所以今日有些作酸呢。’”说的贾母与众人都笑了。\begin{note}庚辰双夹:这方是贾政之谑\end{note}贾政忙斟了一杯,送与贾母。贾母笑道:“既这样,快叫人取烧酒来,别叫你们受累。”众人又都笑起来。
\end{parag}


\begin{parag}
    于是又击鼓,便从贾政传起,可巧传至宝玉鼓止。宝玉因贾政在坐,自是踧踖不安,花偏又在他手内,因想:“说笑话倘或不发笑,又说没口才,连一笑话不能说,何况是别的,这有不是。若说好了,又说正经的不会,只惯油嘴贫舌,更有不是。不如不说的好。”\begin{note}庚辰双夹:实写旧日往事。\end{note}乃起身辞道:“我不能说笑话,求再限别的罢了。”贾政道:“既这样,限一个‘秋’字,就即景作一首诗。若好,便赏你,若不好,明日仔细。”贾母忙道:“好好的行令,如何又要作诗?”贾政道:“他能的。”贾母听说,“既这样就作。”命人取了纸笔来,贾政道:“只不许用那些冰玉晶银彩光明素等样堆砌字眼,要另出己见,试试你这几年的情思。”宝玉听了,碰在心坎上,遂立想了四句,向纸上写了,呈与贾政看,道是……(按:此处有缺文。)贾政看了,点头不语。贾母见这般,知无甚大不好,便问:“怎么样?”贾政因欲贾母喜悦,便说:“难为他。只是不肯念书,到底词句不雅。”贾母道:“这就罢了。他能多大,定要他做才子不成!这就该奖励他,以后越发上心了。”贾政道:“正是。”因回头命个老嬷嬷出去吩咐书房内的小厮,“把我海南带来的扇子取两把给他。”宝玉忙拜谢,仍复归座行令。当下贾兰见奖励宝玉,他便出席也做一首递与贾政看时,写道是……(按:此处有缺文。)贾政看了喜不自胜,遂并讲与贾母听时,贾母也十分欢喜,也忙令贾政赏他。于是大家归坐,复行起令来。
\end{parag}


\begin{parag}
    这次在贾赦手内住了,只得吃了酒,说笑话。因说道:“一家子一个儿子最孝顺。偏生母亲病了,各处求医不得,便请了一个针灸的婆子来。婆子原不知道脉理,只说是心火,如今用针灸之法,针灸针灸就好了。这儿子慌了,便问:‘心见铁即死,如何针得?’婆子道:‘不用针心,只针肋条就是了。’儿子道,‘肋条离心甚远,怎么就好?’婆子道:‘不妨事。你不知天下父母心偏的多呢。’”众人听说,都笑起来。贾母也只得吃半杯酒,半日笑道:“我也得这个婆子针一针就好了。”贾赦听说,便知自己出言冒撞,贾母疑心,忙起身笑与贾母把盏,以别言解释。贾母亦不好再提,且行起令来。
\end{parag}


\begin{parag}
    不料这次花却在贾环手里。贾环近日读书稍进,其脾味中不好务正也与宝玉一样,故每常也好看些诗词,专好奇诡仙鬼一格。今见宝玉作诗受奖,他便技痒,只当著贾政不敢造次。如今可巧花在手中,便也索纸笔来立挥一绝与贾政。\begin{note}庚辰双夹:前文贾政戏谑已是异文,而贾环作诗更奇中又奇之奇文也,总在人意料之外。竟有人曰:“贾环如何又有好诗,似前文不搭后语矣。”盖不可向说问。贾环亦荣府公子正脉,虽少年顽劣,现今小儿之常情耳。读书岂无长进之理哉?况贾政之教是弟子目已大觉疏忽矣。若是贾环连一平仄也不知,岂荣府是寻常膏粱不知诗书之家哉?然后之宝玉之一种情思,正非有益子总明不得谓比诸人皆妙者也。\end{note}贾政看了,亦觉罕异,只是词句终带著不乐读书之意,遂不悦道:“可见是弟兄了。发言吐气总属邪派,将来都是不由规矩准绳,一起下流货。妙在古人中有‘二难 ’,你两个也可以称‘二难’了。只是你两个的‘难’字,却是作难以教训之‘难’字讲才好。哥哥是公然以温飞卿自居,如今兄弟又自为曹唐再世了。”说的贾赦等都笑了。贾赦乃要诗瞧了一遍,连声赞好,道:“这诗据我看甚是有骨气。想来咱们这样人家,原不比那起寒酸,定要‘雪窗荧火’,一日蟾宫折桂,方得扬眉吐气。咱们的子弟都原该读些书,不过比别人略明白些,可以做得官时就跑不了一个官的。何必多费了工夫,反弄出书呆子来。所以我爱他这诗,竟不失咱们侯门的气概。”因回头吩咐人去取了自己的许多玩物来赏赐与他。因又拍著贾环的头,笑道:“以后就这么做去,方是咱们的口气,将来这世袭的前程定跑不了你袭呢。”贾政听说,忙劝说:“不过他胡诌如此,那里就论到后事了。” 说著便斟上酒,又行了一回令。\begin{note}庚辰双夹:便又轻轻抹去也。\end{note}贾母便说:“你们去罢。自然外头还有相公们候著,也不可轻忽了他们。况且二更多了,你们散了,再让我和姑娘们多乐一回,好歇著了。”贾赦等听了,方止了令,又大家公进了一杯酒,方带著子侄们出去了。要知端详,再听下回。
\end{parag}


\begin{parag}
    \begin{note}蒙回末总:下回有一篇极清雅文字,下幅有半篇极整齐文字,故先叙抢快摸牌,沉湎酒色为反振,有骏马下坡势、鸟将翔势。\end{note}
\end{parag}


\begin{parag}
    \begin{note}蒙回末总:看聚赌一段,宛然“宵小群居众日图”,看赏月一段,又宛然“望族序齿燕毛录”,说火则热,而说水则寒,文心故,无所不可。\end{note}
\end{parag}
