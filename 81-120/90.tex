\chap{九十}{失綿衣貧女耐嗷嘈 送果品小郎驚叵測}


\begin{parag}
    卻說黛玉自立意自戕之後,漸漸不支,一日竟至絕粒。從前十幾天內,賈母等輪流看望,他有時還說幾句話,這兩日索性不大言語。心裏雖有時昏暈,卻也有時清楚。賈母等見他這病不似無因而起,也將紫鵑雪雁盤問過兩次,兩個那裏敢說。便是紫鵑欲向侍書打聽消息,又怕越鬧越真,黛玉更死得快了,所以見了侍書,毫不提起。那雪雁是他傳話弄出這樣緣故來,此時恨不得長出百十個嘴來說“我沒說”,自然更不敢提起。到了這一天黛玉絕粒之日,紫鵑料無指望了,守着哭了會子,因出來偷向雪雁道:“你進屋裏來好好兒的守着他。我去回老太太,太太和二奶奶去,今日這個光景大非往常可比了。”雪雁答應,紫鵑自去。
\end{parag}


\begin{parag}
    這裏雪雁正在屋裏伴着黛玉,見他昏昏沉沉,小孩子家那裏見過這個樣兒,只打諒如此便是死的光景了,心中又痛又怕,恨不得紫鵑一時回來纔好。正怕着,只聽窗外腳步走響,雪雁知是紫鵑回來,才放下心了,連忙站起來掀着裏間簾子等他。只見外面簾子響處,進來了一個人,卻是侍書。那侍書是探春打發來看黛玉的,見雪雁在那裏掀着簾子,便問道:“姑娘怎麼樣?”雪雁點點頭兒叫他進來。侍書跟進來,見紫鵑不在屋裏,瞧了瞧黛玉,只剩得殘喘微延,唬的驚疑不止,因問:“紫鵑姐姐呢?”雪雁道:“告訴上屋裏去了。”那雪雁此時只打諒黛玉心中一無所知了,又見紫鵑不在面前,因悄悄的拉了侍書的手問道:“你前日告訴我說的什麼王大爺給這裏寶二爺說了親,是真話麼?”侍書道:“怎麼不真。”雪雁道:“多早晚放定的?”侍書道:“那裏就放定了呢。那一天我告訴你時,是我聽見小紅說的。後來我到二奶奶那邊去,二奶奶正和平姐姐說呢,說那都是門客們藉着這個事討老爺的喜歡,往後好拉攏的意思。別說大太太說不好,就是大太太願意,說那姑娘好,那大太太眼裏看的出什麼人來!再者老太太心裏早有了人了,就在咱們園子裏的。大太太那裏摸的着底呢。老太太不過因老爺的話,不得不問問罷咧。又聽見二奶奶說,寶玉的事,老太太總是要親上作親的,憑誰來說親,橫豎不中用。”雪雁聽到這裏,也忘了神了,因說道:“這是怎麼說,白白的送了我們這一位的命了!”侍書道:“這是從那裏說起?”雪雁道:“你還不知道呢。前日都是我和紫鵑姐姐說來着,這一位聽見了,就弄到這步田地了。”侍書道:“你悄悄兒的說罷,看仔細他聽見了。”雪雁道:“人事都不省了,瞧瞧罷,左不過在這一兩天了。”正說着,只見紫鵑掀簾進來說:“這還了得!你們有什麼話,還不出去說,還在這裏說。索性逼死他就完了。”侍書道:“我不信有這樣奇事。”紫鵑道:“好姐姐,不是我說,你又該惱了。你懂得什麼呢!懂得也不傳這些舌了。”
\end{parag}


\begin{parag}
    這裏三個人正說着,只聽黛玉忽然又嗽了一聲。紫鵑連忙跑到炕沿前站着,侍書雪雁也都不言語了。紫鵑彎着腰,在黛玉身後輕輕問道:“姑娘喝口水罷。”黛玉微微答應了一聲。雪雁連忙倒了半鍾滾白水,紫鵑接了託着,侍書也走近前來。紫鵑和他搖頭兒,不叫他說話,侍書只得嚥住了。站了一回,黛玉又嗽了一聲。紫鵑趁勢問道:“姑娘喝水呀?”黛玉又微微應了一聲,那頭似有欲抬之意,那裏抬得起。紫鵑爬上炕去,爬在黛玉旁邊,端着水試了冷熱,送到脣邊,扶了黛玉的頭,就到碗邊,喝了一口。紫鵑纔要拿時,黛玉意思還要喝一口,紫鵑便託着那碗不動。黛玉又喝了一口,搖搖頭兒不喝了,喘了一口氣,仍舊躺下。半日,微微睜眼說道:“剛纔說話不是侍書麼?”紫鵑答應道:“是。”侍書尚未出去,因連忙過來問候。黛玉睜眼看了,點點頭兒,又歇了一歇,說道:“回去問你姑娘好罷。”侍書見這番光景,只當黛玉嫌煩,只得悄悄的退出去了。原來那黛玉雖則病勢沉重,心裏卻還明白。起先侍書雪雁說話時,他也模糊聽見了一半句,卻只作不知,也因實無精神答理。及聽了雪雁侍書的話,才明白過前頭的事情原是議而未成的,又兼侍書說是鳳姐說的,老太太的主意親上作親,又是園中住着的,非自己而誰?因此一想,陰極陽生,心神頓覺清爽許多,所以才喝了兩口水,又要想問侍書的話。恰好賈母,王夫人,李紈,鳳姐聽見紫鵑之言,都趕着來看。黛玉心中疑團已破,自然不似先前尋死之意了。雖身體軟弱,精神短少,卻也勉強答應一兩句了。鳳姐因叫過紫鵑問道:“姑娘也不至這樣,這是怎麼說,你這樣唬人。”紫鵑道:“實在頭裏看着不好,纔敢去告訴的,回來見姑娘竟好了許多,也就怪了。”賈母笑道:“你也別怪他,他懂得什麼。看見不好就言語,這倒是他明白的地方,小孩子家,不嘴懶腳懶就好。”說了一回,賈母等料着無妨,也就去了。正是:
\end{parag}


\begin{poem}
    \begin{pl}
        心病終須心藥治,解鈴還是繫鈴人。
    \end{pl}
\end{poem}


\begin{parag}
    不言黛玉病漸減退,且說雪雁紫鵑背地裏都念佛。雪雁向紫鵑說道:“虧他好了,只是病的奇怪,好的也奇怪。”紫鵑道:“病的倒不怪,就只好的奇怪。想來寶玉和姑娘必是姻緣,人家說的‘好事多磨’,又說道‘是姻緣棒打不回’。這樣看起來,人心天意,他們兩個竟是天配的了。再者,你想那一年我說了林姑娘要回南去,把寶玉沒急死了,鬧得家翻宅亂。如今一句話,又把這一個弄得死去活來。可不說的三生石上百年前結下的麼。”說着,兩個悄悄的抿着嘴笑了一回。雪雁又道:“幸虧好了。咱們明兒再別說了,就是寶玉娶了別的人家兒的姑娘,我親見他在那裏結親,我也再不露一句話了。”紫鵑笑道:“這就是了。”不但紫鵑和雪雁在私下裏講究,就是衆人也都知道黛玉的病也病得奇怪,好也好得奇怪,三三兩兩,唧唧噥噥議論著。不多幾時,連鳳姐兒也知道了,邢王二夫人也有些疑惑,倒是賈母略猜着了八九。
\end{parag}


\begin{parag}
    那時正值邢王二夫人鳳姐等在賈母房中說閒話,說起黛玉的病來。賈母道:“我正要告訴你們,寶玉和林丫頭是從小兒在一處的,我只說小孩子們,怕什麼?以後時常聽得林丫頭忽然病,忽然好,都爲有了些知覺了。所以我想他們若盡着擱在一塊兒,畢竟不成體統。你們怎麼說?”王夫人聽了,便呆了一呆,只得答應道:“林姑娘是個有心計兒的。至於寶玉,呆頭呆惱,不避嫌疑是有的,看起外面,卻還都是個小孩兒形象。此時若忽然或把那一個分出園外,不是倒露了什麼痕跡了麼。古來說的:‘男大須婚,女大須嫁。’老太太想,倒是趕着把他們的事辦辦也罷了。”賈母皺了一皺眉,說道:“林丫頭的乖僻,雖也是他的好處,我的心裏不把林丫頭配他,也是爲這點子。況且林丫頭這樣虛弱,恐不是有壽的。只有寶丫頭最妥。”王夫人道:“不但老太太這麼想,我們也是這樣。但林姑娘也得給他說了人家兒纔好,不然女孩兒家長大了,那個沒有心事?倘或真與寶玉有些私心,若知道寶玉定下寶丫頭,那倒不成事了。”賈母道:“自然先給寶玉娶了親,然後給林丫頭說人家,再沒有先是外人後是自己的。況且林丫頭年紀到底比寶玉小兩歲。依你們這樣說,倒是寶玉定親的話不許叫他知道倒罷了。”鳳姐便吩咐衆丫頭們道:“你們聽見了,寶二爺定親的話,不許混吵嚷。若有多嘴的,堤防着他的皮。”賈母又向鳳姐道:“鳳哥兒,你如今自從身上不大好,也不大管園裏的事了。我告訴你,須得經點兒心。不但這個,就象前年那些人喝酒耍錢,都不是事。你還精細些,少不得多分點心兒,嚴緊嚴緊他們纔好。況且我看他們也就只還服你。”鳳姐答應了。娘兒們又說了一回話,方各自散了。從此鳳姐常到園中照料。一日,剛走進大觀園,到了紫菱洲畔,只聽見一個老婆子在那裏嚷。鳳姐走到跟前,那婆子才瞧見了,早垂手侍立,口裏請了安。鳳姐道:“你在這裏鬧什麼?”婆子道:“蒙奶奶們派我在這裏看守花果,我也沒有差錯,不料邢姑娘的丫頭說我們是賊。”鳳姐道:“爲什麼呢?”婆子道:“昨兒我們家的黑兒跟着我到這裏頑了一回,他不知道,又往邢姑娘那邊去瞧了一瞧,我就叫他回去了。今兒早起聽見他們丫頭說丟了東西了。我問他丟了什麼,他就問起我來了。”鳳姐道:“問了你一聲,也犯不着生氣呀。”婆子道:“這裏園子到底是奶奶家裏的,並不是他們家裏的。我們都是奶奶派的,賊名兒怎麼敢認呢。”鳳姐照臉啐了一口,厲聲道:“你少在我跟前嘮嘮叨叨的!你在這裏照看,姑娘丟了東西,你們就該問哪,怎麼說出這些沒道理的話來。把老林叫了來,攆出他去。”丫頭們答應了。只見邢岫煙趕忙出來,迎着鳳姐陪笑道:“這使不得,沒有的事,事情早過去了。”鳳姐道:“姑娘,不是這個話。倒不講事情,這名分上太豈有此理了。”岫煙見婆子跪在地下告饒,便忙請鳳姐到裏邊去坐。鳳姐道:“他們這種人我知道,他除了我,其餘都沒上沒下的了。”岫煙再三替他討饒,只說自己的丫頭不好。鳳姐道:“我看着邢姑娘的分上,饒你這一次。”婆子纔起來,磕了頭,又給岫煙磕了頭,纔出去了。
\end{parag}


\begin{parag}
    這裏二人讓了坐。鳳姐笑問道:“你丟了什麼東西了?”岫煙笑道:“沒有什麼要緊的,是一件紅小襖兒,已經舊了的。我原叫他們找,找不着就罷了。這小丫頭不懂事,問了那婆子一聲,那婆子自然不依了。這都是小丫頭糊塗不懂事,我也罵了幾句,已經過去了,不必再提了。”鳳姐把岫煙內外一瞧,看見雖有些皮綿衣服,已是半新不舊的,未必能暖和。他的被窩多半是薄的。至於房中桌上擺設的東西,就是老太太拿來的,卻一些不動,收拾的乾乾淨淨。鳳姐心上便很愛敬他,說道:“一件衣服原不要緊,這時候冷,又是貼身的,怎麼就不問一聲兒呢。這撒野的奴才了不得了!”說了一回,鳳姐出來,各處去坐了一坐,就回去了。到了自己房中,叫平兒取了一件大紅洋縐的小襖兒,一件松花色綾子一斗珠兒的小皮襖,一條寶藍盤錦鑲花綿裙,一件佛青銀鼠褂子,包好叫人送去。
\end{parag}


\begin{parag}
    那時岫煙被那老婆子聒噪了一場,雖有鳳姐來壓住,心上終是不安。想起“許多姊妹們在這裏,沒有一個下人敢得罪他的,獨自我這裏,他們言三語四,剛剛鳳姐來碰見。”想來想去,終是沒意思,又說不出來。正在吞聲飲泣,看見鳳姐那邊的豐兒送衣服過來。岫煙一看,決不肯受。豐兒道:“奶奶吩咐我說,姑娘要嫌是舊衣裳,將來送新的來。”岫煙笑謝道:“承奶奶的好意,只是因我丟了衣服,他就拿來,我斷不敢受。你拿回去千萬謝你們奶奶,承你奶奶的情,我算領了。”倒拿個荷包給了豐兒。那豐兒只得拿了去了。不多時,又見平兒同着豐兒過來,岫煙忙迎着問了好,讓了坐。平兒笑說道:“我們奶奶說,姑娘特外道的了不得。”岫煙道:“不是外道,實在不過意。”平兒道:“奶奶說,姑娘要不收這衣裳,不是嫌太舊,就是瞧不起我們奶奶。剛纔說了,我要拿回去,奶奶不依我呢。”岫煙紅着臉笑謝道:“這樣說了,叫我不敢不收。”又讓了一回茶。
\end{parag}


\begin{parag}
    平兒同豐兒回去,將到鳳姐那邊,碰見薛家差來的一個老婆子,接着問好。平兒便問道:“你那裏來的?”婆子道:“那邊太太姑娘叫我來請各位太太,奶奶,姑娘們的安。我纔剛在奶奶前問起姑娘來,說姑娘到園中去了。可是從邢姑娘那裏來麼?”平兒道:“你怎麼知道?”婆子道:“方纔聽見說。真真的二奶奶和姑娘們的行事叫人感念。”平兒笑了一笑說:“你回來坐着罷。”婆子道:“我還有事,改日再過來瞧姑娘罷。”說着走了。平兒回來,回覆了鳳姐。不在話下。
\end{parag}


\begin{parag}
    且說薛姨媽家中被金桂攪得翻江倒海,看見婆子回來,述起岫煙的事,寶釵母女二人不免滴下淚來。寶釵道:“都爲哥哥不在家,所以叫邢姑娘多喫幾天苦。如今還虧鳳姐姐不錯。咱們底下也得留心,到底是咱們家裏人。”說着,只見薛蝌進來說道:“大哥哥這幾年在外頭相與的都是些什麼人,連一個正經的也沒有,來一起子,都是些狐羣狗黨。我看他們那裏是不放心,不過將來探探消息兒罷咧。這兩天都被我幹出去了。以後吩咐了門上,不許傳進這種人來。”薛姨媽道:“又是蔣玉菡那些人哪?”薛蝌道:“蔣玉菡卻倒沒來,倒是別人。”薛姨媽聽了薛蝌的話,不覺又傷心起來,說道:“我雖有兒,如今就象沒有的了,就是上司準了,也是個廢人。你雖是我侄兒,我看你還比你哥哥明白些,我這後輩子全靠你了。你自己從今更要學好。再者,你聘下的媳婦兒,家道不比往時了。人家的女孩兒出門子不是容易,再沒別的想頭,只盼着女婿能幹,他就有日子過了。若邢丫頭也象這個東西,”說着把手往裏頭一指,道:“我也不說了。邢丫頭實在是個有廉恥有心計兒的,又守得貧,耐得富。只是等咱們的事情過去了,早些把你們的正經事完結了,也了我一宗心事。”薛蝌道:“琴妹妹還沒有出門子,這倒是太太煩心的一件事。至於這個,可算什麼呢。”大家又說了一回閒話。
\end{parag}


\begin{parag}
    薛蝌回到自己房中,吃了晚飯,想起邢岫煙住在賈府園中,終是寄人籬下,況且又窮,日用起居,不想可知。況兼當初一路同來,模樣兒性格兒都知道的。可知天意不均:如夏金桂這種人,偏教他有錢,嬌養得這般潑辣,邢岫煙這種人,偏教他這樣受苦。閻王判命的時候,不知如何判法的。想到悶來也想吟詩一首,寫出來出出胸中的悶氣。又苦自己沒有工夫,只得混寫道:
\end{parag}


\begin{poem}
    \begin{pl}
        蛟龍失水似枯魚,兩地情懷感索居。
    \end{pl}


    \begin{pl}
        同在泥塗多受苦,不知何日向清虛。
    \end{pl}
\end{poem}


\begin{parag}
    寫畢看了一回,意欲拿來粘在壁上,又不好意思。自己沉吟道:“不要被人看見笑話。”又唸了一遍,道:“管他呢,左右粘上自己看着解悶兒罷。”又看了一回,到底不好,拿來夾在書裏。又想自己年紀可也不小了,家中又碰見這樣飛災橫禍,不知何日了局,致使幽閨弱質,弄得這般淒涼寂寞。正在那裏想時,只見寶蟾推門進來,拿着一個盒子,笑嘻嘻放在桌上。薛蝌站起來讓坐。寶蟾笑着向薛蝌道:“這是四碟果子,一小壺兒酒,大奶奶叫給二爺送來的。”薛蝌陪笑道:“大奶奶費心。但是叫小丫頭們送來就完了,怎麼又勞動姐姐呢。”寶蟾道:“好說。自家人,二爺何必說這些套話。再者我們大爺這件事,實在叫二爺操心,大奶奶久已要親自弄點什麼兒謝二爺,又怕別人多心。二爺是知道的,咱們家裏都是言合意不合,送點子東西沒要緊,倒沒的惹人七嘴八舌的講究。所以今日些微的弄了一兩樣果子,一壺酒,叫我親自悄悄兒的送來。”說着,又笑瞅了薛蝌一眼,道:“明兒二爺再別說這些話,叫人聽着怪不好意思的。我們不過也是底下的人,伏侍的着大爺就伏侍的着二爺,這有何妨呢。”薛蝌一則秉性忠厚,二則到底年輕,只是向來不見金桂和寶蟾如此相待,心中想到剛纔寶蟾說爲薛蟠之事也是情理,因說道:“果子留下罷,這個酒兒,姐姐只管拿回去。我向來的酒上實在很有限,擠住了偶然喝一鍾,平日無事是不能喝的。難道大奶奶和姐姐還不知道麼。”寶蟾道:“別的我作得主,獨這一件事,我可不敢應。大奶奶的脾氣兒,二爺是知道的,我拿回去,不說二爺不喝,倒要說我不盡心了。”薛蝌沒法,只得留下。寶蟾方纔要走,又到門口往外看看,回過頭來向着薛蝌一笑,又用手指着裏面說道:“他還只怕要來親自給你道乏呢。”薛蝌不知何意,反倒訕訕的起來,因說道:“姐姐替我謝大奶奶罷。天氣寒,看涼着。再者,自己叔嫂,也不必拘這些個禮。”寶蟾也不答言,笑着走了。
\end{parag}


\begin{parag}
    薛蝌始而以爲金桂爲薛蟠之事,或者真是不過意,備此酒果給自己道乏,也是有的。及見了寶蟾這種鬼鬼祟祟不尷不尬的光景,也覺了幾分。卻自己迴心一想:“他到底是嫂子的名分,那裏就有別的講究了呢。或者寶蟾不老成,自己不好意思怎麼樣,卻指着金桂的名兒,也未可知。然而到底是哥哥的屋裏人,也不好。”忽又一轉念:“那金桂素性爲人毫無閨閣理法,況且有時高興,打扮得妖調非常,自以爲美,又焉知不是懷着壞心呢?不然,就是他和琴妹妹也有了什麼不對的地方兒,所以設下這個毒法兒,要把我拉在渾水裏,弄一個不清不白的名兒,也未可知。”想到這裏,索性倒怕起來。正在不得主意的時候,忽聽窗外撲哧的笑了一聲,把薛蝌倒唬了一跳。未知是誰,下回分解。
\end{parag}