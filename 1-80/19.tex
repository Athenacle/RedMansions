\chap{一十九}{情切切良宵花解語 意綿綿靜日玉生香}


\begin{parag}
    \begin{note}蒙回前詩:彩筆輝光若轉環,情心魔態幾千般。寫成濃淡兼深淺,活現癡人戀戀間。\end{note}
\end{parag}


\begin{parag}
    話說賈妃回宮,次日見駕謝恩,並回奏歸省之事,龍顏甚悅,又發內帑綵緞金銀等物,以賜賈政及各椒房等員,\begin{note}庚雙夾:補還一句,細。方見省親不獨賈家一門是也。\end{note}不必細說。
\end{parag}


\begin{parag}
    且說榮寧二府中連日用盡心力,真是人人力倦,各各神疲,又將園中一應陳設動用之物收拾了兩三天方完。第一個鳳姐事多任重,別人或可偷安躲靜,獨他是不能脫得的;二則本性要強,不肯落人褒貶,只扎掙著與無事的人一樣。\begin{note}庚雙夾:伏下病源。\end{note}第一個寶玉是極無事最閒暇的。偏這日一早,襲人的母親又親來回過賈母,接襲人家去喫年茶,晚間才得回來。\begin{note}庚雙夾:一回一回各生機軸,總在人意想之外。\end{note}因此,寶玉只和衆丫頭們擲骰子趕圍棋作戲。\begin{note}庚雙夾:寫出正月光景。\end{note}正在房內頑的沒興頭,忽見丫頭們來回說:“東府珍大爺來請過去看戲、放花燈。”寶玉聽了,便命換衣裳。纔要去時,忽又有賈妃賜出糖蒸酥酪來;\begin{note}庚雙夾:總是新正妙景。\end{note}寶玉想上次襲人喜喫此物,便命留與襲人了。自己回過賈母,過去看戲。
\end{parag}


\begin{parag}
    誰想賈珍這邊唱的是《丁郎認父》、《黃伯央大擺陰魂陣》,更有《孫行者大鬧天宮》、《姜子牙斬將封神》等類的戲文。\begin{note}庚雙夾:真真熱鬧。\end{note}倏爾神鬼亂出,忽又妖魔畢露,甚至於揚幡過會,號佛行香,鑼鼓喊叫之聲聞於巷外。\begin{note}庚雙夾:形容刻薄之至,弋陽腔能事畢矣。閱至此則有如耳內喧譁、目中離亂,後文至隔牆聞“嫋晴絲”數曲,則有如魂隨笛轉、魄逐歌銷。形容一事,一事畢肖,石頭是第一能手矣。\end{note}滿街之人個個都贊:“好熱鬧戲,別人家斷不能有的。”\begin{note}庚雙夾:必有之言。\end{note}寶玉見那繁華熱鬧到如此不堪的田地,只略坐了一坐,便走開各處閒耍。先是進內去和尤氏和丫鬟姬妾說笑了一回,便出二門來。尤氏等仍料他出來看戲,遂也不曾照管。賈珍、賈璉、薛蟠等只顧猜枚行令,百般作樂,也不理論,縱一時不見他在座,只道在裏邊去了,故也不問。至於跟寶玉的小廝們,那年紀大些的,知寶玉這一來了,必是晚上才散,因此偷空也有去會賭的,也有往親友家去喫年茶的,更有或嫖或飲,都私散了,待晚間再來;那些小的,都鑽進戲房裏瞧熱鬧去了。
\end{parag}


\begin{parag}
    寶玉見一個人沒有,因想“這裏素日有個小書房,名……,\begin{subnote}按:此處有缺文\end{subnote}內曾掛著一軸美人,極畫的得神。今日這般熱鬧,想那裏自然……\begin{subnote}按:此處有缺文\end{subnote}那美人也自然是寂寞的,須得我去望慰他一回。”\begin{note}庚雙夾:極不通極胡說中寫出絕代情癡,宜乎衆人謂之瘋傻。\end{note}\begin{note}蒙側:天生一段癡情,所謂“情不情”也。\end{note}想著,便往書房裏來。剛到窗前,聞得房內有呻吟之韻。寶玉倒唬了一跳:敢是美人活了不成?\begin{note}庚雙夾:又帶出小兒心意,一絲不落。\end{note}乃乍著膽子,舔破窗紙,向內一看,那軸美人卻不曾活,卻是茗煙按著一個女孩子,也幹那警幻所訓之事。寶玉禁不住大叫:“了不得!”一腳踹進門去,將那兩個唬開了,抖衣而顫。
\end{parag}


\begin{parag}
    茗煙見是寶玉,忙跪求不迭。寶玉道:“青天白日,這是怎麼說。\begin{note}庚雙夾:開口便好。\end{note}珍大爺知道,你是死是活?”一面看那丫頭,雖不標緻,倒還白淨,些微亦動人處,羞的面紅耳赤,低首無言。寶玉跺腳道:“還不快跑!”\begin{note}庚雙夾:此等搜神奪魄至神至妙處只在囫圇不解處得。\end{note}一語提醒了那丫頭,飛也似去了。寶玉又趕出去,叫道:“你別怕,我是不告訴人的。”\begin{note}庚雙夾:活寶玉,移之他人不可。\end{note}急的茗煙在後叫:“祖宗,這是分明告訴人了!”寶玉因問:“那丫頭十幾歲了?”茗煙道:“大不過十六七歲了。”寶玉道:“連他的歲屬也不問問,別的自然越發不知了。可見他白認得你了。可憐,可憐!”\begin{note}庚雙夾:按此書中寫一寶玉,其寶玉之爲人是我輩於書中見而知有此人,實未目曾親睹者。又寫寶玉之發言每每令人不解,寶玉之生性件件令人可笑,不獨不曾於世上親見這樣的人,即閱今古所有之小說奇傳中亦未見這樣的文字。於顰兒處更爲甚。其囫圇不解之中實可解,可解之中又說不出理路,合目思之,卻如真見一寶玉真聞此言者,移至第二人萬不可,亦不成文字矣。餘閱《石頭記》中至奇至妙之文,全在寶玉顰兒至癡至呆囫圇不解之語中,其誓詞雅迷酒令奇衣奇食奇玩等類固他書中未能,然在此書中評之,猶爲二著。\end{note}又問:“名字叫什麼?”茗煙大笑道:“若說出名字來話長,真真新鮮奇文,竟是寫不出來的。\begin{note}庚雙夾:若都寫得出來,何以見此書中之妙?脂硯。\end{note}據他說,他母親養他的時節做了一個夢,\begin{note}庚雙夾:又一個夢,只是隨手成趣耳。\end{note}夢見得了一匹錦,上面是五色富貴萬不斷頭的花樣,\begin{note}庚雙夾:千奇百怪之想,所謂“牛溲馬渤皆至樂也,魚鳥昆蟲皆妙文也”,天地間無一物不是妙物,無一物不可成文,但在人意舍取耳。此皆信手拈來隨筆成趣,大遊戲、大慧悟、大解脫之妙文也。\end{note}所以他的名字叫作萬兒。”寶玉聽了笑道:“真也新奇,想必他將來有些造化。”說著,沉思一會。
\end{parag}


\begin{parag}
    茗煙因問:“二爺爲何不看這樣的好戲?”寶玉道:“看了半日,怪煩的,出來逛逛,就遇見你們了。這會子作什麼呢?”茗煙嘻嘻笑道:“這會子沒人知道,我悄悄的引二爺往城外逛逛去,一會子再往這裏來,他們就不知道了。”\begin{note}庚雙夾:茗煙此時只要掩飾方纔之過,故設此以悅寶玉之心。\end{note}寶玉道:“不好,仔細花子拐了去。便是他們知道了,又鬧大了,不如往熟近些的地方去,還可就來。”茗煙道:“熟近地方,誰家可去?這卻難了。”寶玉笑道:“依我的主意,咱們竟找你花大姐姐去,瞧他在家作什麼呢。”\begin{note}庚雙夾:妙!寶玉心中早安著這著,但恐茗煙不肯引去耳。恰遇茗煙私行淫媾,爲寶玉所脅,故以城外引以悅其心,寶玉始悅,出往花家去。非茗煙適有罪所脅,萬不敢如此私引出外。別家子弟尚不敢私出,況寶玉哉?況茗煙哉?文字著楔細甚。\end{note}茗煙笑道:“好,好!倒忘了他家。”又道:“若他們知道了,說我引著二爺胡走,要打我呢?”\begin{note}庚雙夾:必不可少之語。\end{note}寶玉笑道:“有我呢。”茗煙聽說,拉了馬,二人從後門就走了。
\end{parag}


\begin{parag}
    幸而襲人家不遠,不過一半里路程,展眼已到門前。茗煙先進去叫襲人之兄花自芳。\begin{note}庚雙夾:隨姓成名,隨手成文。\end{note}此時襲人之母接了襲人與幾個外甥女兒、\begin{note}庚雙夾:一樹千枝,一源萬派,無意隨手,伏脈千里。\end{note}幾個侄女兒來家,正喫果茶。聽見外面有人叫“花大哥”,花自芳忙出去看時,見是他主僕兩個,唬的驚疑不止,連忙抱下寶玉來,至院內嚷道:“寶二爺來了!”別人聽見還可,襲人聽了,也不知爲何,忙跑出來迎著寶玉,一把拉著問:“你怎麼來了?”寶玉笑道:“我怪悶的,來瞧瞧你作什麼呢。”襲人聽了,才放下心來,\begin{note}庚雙夾:精細周到。\end{note}嗐了一聲,笑\begin{note}庚雙夾:轉至“笑”字,妙甚!\end{note}道:“你也忒胡鬧了,\begin{note}庚雙夾:該說,說得是。\end{note}可作什麼來呢!”一面又問茗煙:“還有誰跟來?”\begin{note}庚雙夾:細。\end{note}茗煙笑道:“別人都不知道,就只我們兩個。”襲人聽了,復又驚慌,\begin{note}庚雙夾:是必有之神理,非特故作頓挫。\end{note}說道:“這還了得!倘或碰見了人,或是遇見了老爺,街上人擠車碰,馬轎紛紛的,若有個閃失,也是頑得的!你們的膽子比鬥還大。都是茗煙調唆的,回去我定告訴嬤嬤們打你。”\begin{note}庚雙夾:該說,說得更是。\end{note}茗煙撅了嘴道:“二爺罵著打著,叫我引了來,這會子推到我身上。我說別來罷,不然我們還去罷。”\begin{note}庚雙夾:茗煙賊。\end{note}花自芳忙勸:“罷了,已是來了,也不用多說了。只是茅檐草舍,又窄又髒,爺怎麼坐呢?”
\end{parag}


\begin{parag}
    襲人之母也早迎了出來。襲人拉著寶玉進去。寶玉見房中三五個女孩兒,見他進來,都低了頭,羞慚慚的。花自芳母子兩個百般怕寶玉冷,又讓他上炕,又忙另擺果桌,又忙倒好茶。\begin{note}庚雙夾:連用三“又”字,上文一個,百般神理活現。\end{note}襲人笑道:“你們不用白忙,\begin{note}庚雙夾:妙!不寫襲卿,正是忙之至。若一寫襲人忙,便是庸俗小派了。\end{note}我自然知道。果子也不用擺,也不敢亂給東西喫。”\begin{note}庚雙夾:如此至微至小中便帶出家常情,他書寫不及此。\end{note}一面說,一面將自己的坐褥拿了鋪在一個炕上,寶玉坐了;用自己的腳爐墊了腳,向荷包內取出兩個梅花香餅兒來,又將自己的手爐掀開焚上,仍蓋好,放與寶玉懷內;然後將自己的茶杯斟了茶,送與寶玉。\begin{note}庚雙夾:用四“自己”字,寫得寶襲二人素日如何親洽如何尊榮,此時一盤托出。蓋素日身居侯府綺羅錦繡之中,其安富尊榮之寶玉親密浹洽勤慎委婉之襲人,是分所應當不必寫者也。今於此一補,更見二人平素之情意,且暗透此回中所有母女兄長欲爲贖身角口等未到之過文。\end{note}彼時他母兄已是忙另齊齊整整擺上一桌子果品來。襲人見總無可喫之物,\begin{note}庚雙夾:補明寶玉自幼何等嬌貴,以此一句留與下部後數十回“寒冬噎酸虀,雪夜圍破氈”等處對看,可爲後生過分之戒。嘆嘆!\end{note}因笑道:“既來了,沒有空去之理,好歹嘗一點兒,也是來我家一趟。”\begin{note}庚雙夾:得意之態,是才與母兄較爭以後之神理。最細。\end{note}說著,便拈了幾個松子穰,\begin{note}庚雙夾:唯此品稍可一拈,別品便大錯了。\end{note}吹去細皮,用手帕託著送與寶玉。
\end{parag}


\begin{parag}
    寶玉看見襲人兩眼微紅,粉光融滑,\begin{note}庚雙夾:八字畫出一才收淚之女兒,是好形容,切實寶玉眼中意中。\end{note}因悄問襲人:“好好的哭什麼?”襲人笑道:“何嘗哭,才迷了眼揉的。”因此便遮掩過了。\begin{note}庚雙夾:伏下後文所補未到多少文字。\end{note}當下寶玉穿著大紅金蟒狐腋箭袖,外罩石青貂裘排穗褂。襲人道:“你特爲往這裏來又換新服,他們\begin{note}庚雙夾:指晴雯麝月等。\end{note}就不問你往那去的?”\begin{note}庚雙夾:必有是問。閱此則又笑盡小說中無故家常穿紅掛綠綺繡綾羅等語,自謂是富貴語,究竟反是寒酸話。\end{note}寶玉笑道:“珍大哥那裏去看戲換的。”襲人點頭。又道:“坐一坐就回去罷,這個地方不是你來的。”寶玉笑道:“你就家去纔好呢,我還替你留著好東西呢。”\begin{note}庚雙夾:生受,切己之事。\end{note}襲人悄笑道:“悄悄的,叫他們聽著什麼意思。”\begin{note}庚雙夾:想見二人來日情常。\end{note}一面又伸手從寶玉項上將通靈玉摘了下來,向他姊妹們笑道:“你們見識見識。時常說起來都當希罕,恨不能一見,今兒可盡力瞧了。再瞧什麼希罕物兒,也不過是這麼個東西。”\begin{note}庚雙夾:行文至此,固好看之極,且勿論按此言固是襲人得意之話,蓋言你等所稀罕不得一見之寶我卻常守常見視爲平物。然餘今窺其用意之旨,則是作者藉此正爲貶玉原非大觀者也。\end{note}說畢,遞與他們傳看了一遍,仍與寶玉掛好。\begin{note}庚眉:自“一把拉住”至此諸形景動作,襲卿有意微露鋒芒,軒中隱事也。\end{note}又命他哥哥去或僱一乘小轎,或僱一輛小車,送寶玉回去。花自芳道:“有我送去,騎馬也不妨了。”\begin{note}庚側:只知保重耳。\end{note}襲人道:“不爲不妨,爲的是碰見人。”\begin{note}庚雙夾:細極!\end{note}
\end{parag}


\begin{parag}
    花自芳忙去僱了一頂小轎來,衆人也不敢相留,只得送寶玉出去。襲人又抓果子與茗煙,又把些錢與他買花炮放,教他:“不可告訴人,連你也有不是。”一直送寶玉至門前,看著上轎,放下轎簾。花、茗二人牽馬跟隨。來至寧府街,茗煙命住轎,向花自芳道:“須等我同二爺還到東府裏混一混,纔過去的,不然人家就疑惑了。”花自芳聽說有理,忙將寶玉抱出轎來,送上馬去。寶玉笑說:“倒難爲你了。”\begin{note}庚側:公子口氣。\end{note}於是仍進後門來。俱不在話下。
\end{parag}


\begin{parag}
    卻說寶玉自出了門,他房中這些丫鬟們都越發恣意的頑笑,也有趕圍棋的,也有擲骰抹牌的,磕了一地瓜子皮。偏奶母李嬤嬤拄拐進來請安,瞧瞧寶玉,見寶玉不在家,丫鬟們只顧玩鬧,十分看不過。\begin{note}庚雙夾:人人都看不過,獨寶玉看得過。\end{note}因嘆道:“只從我出去了,不大進來,你們越發沒了樣兒了,\begin{note}庚雙夾:說得是,原該說。\end{note}別的媽媽們越不敢說你們了。\begin{note}庚雙夾:補得好!寶玉雖不喫乳,豈無伴從之媼嫗哉?\end{note}那寶玉是個丈八的燈臺——照見人家,照不見自家的。\begin{note}庚雙夾:用俗語入妙。\end{note}只知嫌人家髒,這是他的屋子,由著你們糟蹋,越不成體統了。”\begin{note}庚雙夾:所以爲今古未有之一寶玉。\end{note}這些丫頭們明知寶玉不講究這些,二則李嬤嬤已是告老解事出去的了,\begin{note}庚雙夾:調侃入微,妙妙!\end{note}如今管不著他們。因此只顧頑,並不理他。那李嬤嬤還只管問“寶玉如今一頓喫多少飯”、“什麼時候睡覺”等語。丫頭們總胡亂答應。有的說:“好一個討厭的老貨!”\begin{note}庚側:實在有的。\end{note}
\end{parag}


\begin{parag}
    李嬤嬤又問道:“這蓋碗裏是酥酪,怎不送與我去?我就吃了罷”說畢,拿匙就喫。\begin{note}庚雙夾:寫龍鍾奶母,便是龍鍾奶母。\end{note}一個丫頭道:“快別動!那是說了給襲人留著的,\begin{note}庚雙夾:過下無痕。\end{note}回來又惹氣了。\begin{note}庚雙夾:照應茜雪楓露茶前案。\end{note}你老人家自己承認,別帶累我們受氣。”\begin{note}庚雙夾:這等話語聲口,必是晴雯無疑。\end{note}李嬤嬤聽了,又氣又愧,便說道:“我不信他這樣壞了。別說我吃了一碗牛奶,就是再比這個值錢的,也是應該的。難道待襲人比我還重?難道他不想想怎麼長大了?我的血變的奶,喫的長這麼大,如今我喫他一碗牛奶,他就生氣了?我偏吃了,看怎麼樣!你們看襲人不知怎樣,那是我手裏調理出來的毛丫頭,什麼阿物兒!”\begin{note}庚雙夾:是暫委屈唐突襲卿,然亦怨不得李媼。\end{note}一面說,一面賭氣將酥酪吃盡。又一丫頭笑道:“他們不會說話,怨不得你老人家生氣。寶玉還時常送東西孝敬你老去,豈有爲這個不自在的。”\begin{note}庚雙夾:聽這聲口,必是麝月無疑。\end{note}李嬤嬤道:“你們也不必妝狐媚子哄我,打量上次爲茶攆茜雪的事我不知道呢。\begin{note}庚雙夾:照應前文,又用一“攆”,屈殺寶玉,然在李媼心中口中畢肖。\end{note}明兒有了不是,我再來領!”說著,賭氣去了。\begin{note}庚雙夾:過至下回。\end{note}
\end{parag}


\begin{parag}
    少時,寶玉回來,命人去接襲人。只見晴雯躺在牀上不動,\begin{note}庚雙夾:嬌態已慣。\end{note}寶玉因問:“敢是病了?再不然輸了?”秋紋道:“他倒是贏的。誰知李老奶奶來了,混輸了,他氣的睡去了。”寶玉笑道:“你別和他一般見識,由他去就是了。” 說著,襲人已來,彼此相見。襲人又問寶玉何處喫飯,多早晚回來,又代母妹問諸同伴姊妹好。一時換衣卸妝。寶玉命取酥酪來,丫鬟們回說:“李奶奶吃了。”寶玉纔要說話,襲人便忙笑說道:“原來是留的這個,多謝費心。前兒我喫的時候好喫,喫過了好肚子疼,足鬧的吐了纔好。他吃了倒好,擱在這裏倒白糟蹋了。\begin{note}庚雙夾:與前文應失手碎鍾遙對,通部襲人皆是如此,一絲不錯。\end{note}我只想風乾栗子喫,你替我剝栗子,我去鋪炕。”\begin{note}庚雙夾:必如此方是。\end{note}
\end{parag}


\begin{parag}
    寶玉聽了信以爲真,方把酥酪丟開,取栗子來,自向燈前檢剝。一面見衆人不在房中,乃笑問襲人道:“今兒那個穿紅的是你什麼人?”\begin{note}庚雙夾:若是見過女兒之後沒有一段文字便不是寶玉,亦非《石頭記》矣。\end{note}襲人道:“那是我兩姨妹子。”寶玉聽了,讚歎了兩聲。\begin{note}庚雙夾:這一讚嘆又是令人囫圇不解之語,只此便抵過一大篇文字。\end{note}襲人道:“嘆什麼?\begin{note}庚雙夾:只一“嘆”字便引出“花解語”一回來。\end{note}我知道你心裏的緣故,想是說他那裏配紅的。”\begin{note}庚雙夾:補出寶玉素喜紅色,這是激語。\end{note}寶玉笑道:“不是,不是。那樣的不配穿紅的,誰還敢穿。\begin{note}庚雙夾:活寶玉。\end{note}我因爲見他實在好的很,怎麼也得他在咱們家就好了。”\begin{note}庚雙夾:妙談妙意。\end{note}襲人冷笑道:“我一個人是奴才命罷了,難道連我的親戚都是奴才命不成?定還要揀實在好的丫頭才往你家來?”\begin{note}庚雙夾:妙答。寶玉並未說“奴才”二字,襲人連補“奴才”二字最是勁節,怨不得作此語。\end{note}寶玉聽了,忙笑道:“你又多心了。我說往咱們家來,必定是奴才不成?\begin{note}蒙雙夾:勉強,如聞。\end{note}說親戚就使不得?”\begin{note}庚雙夾:更勉強。蒙側:這樣妙文,何處得來?非目見身行,豈能如此的確?\end{note}襲人道:“那也搬配不上。”\begin{note}庚雙夾:說得是。\end{note}寶玉便不肯再說,只是剝粟子。襲人笑道:“怎麼不言語了?想是我才冒撞衝犯了你?明兒賭氣花幾兩銀子買他們進來就是了。”\begin{note}庚雙夾:總是故意激他。\end{note}寶玉笑道:“你說的話,怎麼叫我答言呢。我不過是贊他好,正配生在這深堂大院裏,沒的我們這種濁物\begin{note}庚雙夾:妙號!後文又曰“鬚眉濁物”之稱,今古未有之一人始有此今古未有之妙稱妙號。\end{note}倒生在這裏。”\begin{note}庚雙夾:此皆寶玉心中意中確實之念,非前勉強之詞,所以謂今古未有之一人耳。聽其囫圇不解之言,察其幽微感觸之心,審其癡妄委婉之意,皆今古未見之人,亦是今古未見之文字。說不得賢,說不得愚,說不得不肖,說不得善,說不得惡,說不得光明正大,說不得混賬惡賴,說不得聰明才俊,說不得庸俗平□,說不得好色好淫,說不得情癡情種,恰恰只有一顰兒可對,令他人徒加評論,總未摸著他二人是何等脫胎、何等心臆、何等骨肉。餘閱此書,亦愛其文字耳,實亦不能評出此二人終是何等人物。後觀《情榜》評曰“寶玉情不情”,“黛玉情情”,此二評自在評癡之上,亦屬囫圇不解,妙甚!\end{note}襲人道:“他雖沒這造化,倒也是嬌生慣養的呢,我姨爹姨娘的寶貝。如今十七歲,各樣的嫁妝都齊備了,明年就出嫁。”\begin{note}庚雙夾:所謂不入耳之言也。\end{note}
\end{parag}


\begin{parag}
    寶玉聽了“出嫁”二字,不禁又嗐了兩聲。\begin{note}庚雙夾:心思另是一樣,餘前評可見。\end{note}正是不自在,又聽襲人嘆道:\begin{note}庚雙夾:襲人亦嘆,自有別論。\end{note}“只從我來這幾年,姊妹們都不得在一處。如今我要回去了,他們又都去了。”寶玉聽這話內有文章,\begin{note}庚雙夾:餘亦如此。\end{note}不覺一驚,\begin{note}庚雙夾:餘亦喫驚。\end{note}忙丟下粟子,問道:“怎麼,你如今要回去了?”襲人道:“我今兒聽見我媽和哥哥商議,教我再耐煩一年,明年他們上來,就贖我出去的呢。”\begin{note}庚雙夾:即餘今日猶難爲情,況當日之寶玉哉?\end{note}寶玉聽了這話,越發怔了,因問:“爲什麼要贖你?”襲人道:“這話奇了!我又比不得是這裏的家生子兒,一家子都在別處,獨我一個人在這裏,怎麼是個了局?”\begin{note}庚雙夾:說得極是。\end{note}寶玉道:“我不叫你去也難。”\begin{note}庚雙夾:是頭一句駁,故用貴公子聲口,無理。\end{note}襲人道:“從來沒這道理。便是朝廷宮裏,也有個定例,或幾年一選,幾年一入,也沒有個長遠留下人的理,別說你了!”\begin{note}庚雙夾:一駁,更有理。\end{note}
\end{parag}


\begin{parag}
    寶玉想一想,果然有理。\begin{note}庚雙夾:自然。\end{note}又道:“老太太不放你也難。”\begin{note}庚雙夾:第二層伏祖母溺愛,更是無理。\end{note}襲人道:“爲什麼不放?我果然是個最難得的,或者感動了老太太、太太,\begin{note}庚雙夾:寶玉並不提王夫人,襲人偏自補出,周密之至!\end{note}必不放我出去的,設或多給我們家幾兩銀子,留下我,然或有之;其實我又不過是個平常的人,比我強的多而且多。自我從小兒來了,跟著老太太,先服侍了史大姑娘幾年,\begin{note}庚雙夾:百忙中又補出湘雲來,真是七穿八達,得空便入。\end{note}如今又服侍了你幾年。如今我們家來贖,正是該叫去的,只怕連身價也不要,就開恩叫我去呢。要說爲服侍的你好,不叫我去,斷然沒有的事。那服侍的好,是分內應當的,\begin{note}庚側:這卻是真心話。\end{note}不是什麼奇功。我去了,仍舊有好的來了,不是沒了我就不成事。”\begin{note}庚雙夾:再一駁,更精細更有理。\end{note}寶玉聽了這些話,竟是有去的理,無留的理,\begin{note}庚雙夾:自然。\end{note}心內越發急了,\begin{note}庚雙夾:原當急。\end{note}因又道:“雖然如此說,我只一心留下你,不怕老太太不和你母親說。多多給你母親些銀子,他也不好意思接你了。”\begin{note}庚雙夾:急心腸,故入於霸道。無理。\end{note}襲人道:“我媽自然不敢強。且漫說和他好說,又多給銀子;就便不和他好說,一個錢也不給,安心要強留下我,他也不敢不依。但只是咱們家從沒有幹過這倚勢仗貴霸道的事。這比不得別的東西,因爲你喜歡,加十倍利弄了來給你,那賣的人不得喫虧,可以行得。如今無故平空留下我,於你又無益,反叫我們骨肉分離,這件事,老太太、太太斷不肯行的。”\begin{note}庚雙夾:三駁,不獨更有理,且又補出賈府自家慈善寬厚等事。\end{note}寶玉聽了,思忖半晌,\begin{note}庚雙夾:正是思忖只有去理實無留理。\end{note}乃說道:“依你說,你是去定了?”\begin{note}庚雙夾:自然。\end{note}襲人道:“去定了。”\begin{note}庚側:口氣像極。\end{note}寶玉聽了,自思道:“誰知這樣一個人,這樣薄情無義。”\begin{note}庚雙夾:餘亦如此見疑。\end{note}乃嘆道:“早知道都是要去的,\begin{note}蒙雙夾:“都是要去的”,妙!可謂觸類旁通,活是寶玉。\end{note}\begin{note}蒙側:上古至今及後世有情者同聲一哭!\end{note}我就不該弄了來,臨了剩了我一個孤鬼兒。”\begin{note}庚雙夾:可謂見首知尾,活是寶玉。\end{note}說著,便賭氣上牀睡去了。\begin{note}庚雙夾:又到無可奈何之時了。\end{note}
\end{parag}


\begin{parag}
    原來襲人在家,聽見他母兄要贖他回去,\begin{note}庚雙夾:補前文。\end{note}他就說至死也不回去的。又說:“當日原是你們沒飯喫,就剩我還值幾兩銀子,若不叫你們賣,沒有個看著老子娘餓死的理。\begin{note}庚側:孝女,義女。庚雙夾:補出襲人幼時艱辛苦狀,與前文之香菱、後文之晴雯大同小異,自是又副十二釵中之冠,故不得不補傳之。\end{note}如今幸而賣到這個地方,\begin{note}庚雙夾:可謂不幸中之幸。\end{note}喫穿和主子一樣,又不朝打暮罵。況且如今爹雖沒了,你們卻又整理的家成業就,復了元氣。若果然還艱難,把我贖出來,再多掏澄幾個錢,也還罷了,\begin{note}庚側:孝女,義女。\end{note}其實又不難了。這會子又贖我作什麼?權當我死了,\begin{note}庚側:可憐!\end{note}再不必起贖我的念頭!”\begin{note}庚側:我也要笑。\end{note}\begin{note}蒙側:同心同志更覺幸福。\end{note}因此哭鬧了一陣。\begin{note}庚雙夾:以上補在家今日之事,與寶玉問哭一句針對。\end{note}
\end{parag}


\begin{parag}
    他母兄見他這般堅執,自然必不出來的了。況且原是賣倒的死契,明仗著賈宅是慈善寬厚之家,不過求一求,只怕身價銀一併賞了這是有的事呢。\begin{note}庚雙夾:又夾帶出賈府平素施爲來,與襲人口中針對。\end{note}二則,賈府中從不曾作踐下人,只有恩多威少的。\begin{note}庚雙夾:伏下多少後文。\end{note}且凡老少房中所有親侍的女孩子們,更比待家下衆人不同,平常寒薄人家的小姐,也不能那樣尊重的。\begin{note}庚雙夾:又伏下多少後文。現一句是傳中陪客,此一句是傳中本旨。\end{note}因此,他母子兩個也就死心不贖了。\begin{note}庚雙夾:既如此何得襲人又作前語以愚寶玉?不知何意,且看後文。\end{note}次後忽然寶玉去了,他二個又是那般景況,\begin{note}庚雙夾:一件閒事一句閒文皆無,警甚。\end{note}他母子二人心下更明白了,越發石頭落了地,而且是意外之想,彼此放心,再無贖唸了。\begin{note}庚雙夾:一段情結。\end{note}
\end{parag}


\begin{parag}
    如今且說襲人自幼見寶玉性格異常,\begin{note}庚雙夾:四字好!所謂“說不得好,又說不得不好”也。\end{note}其淘氣憨頑自是出於衆小兒之外,更有幾件千奇百怪口不能言的毛病兒。\begin{note}庚雙夾:只如此說更好。所謂“說不得聰明賢良,說不得癡呆愚昧”也。\end{note}近來仗著祖母溺愛,父母亦不能十分嚴緊拘管,更覺放蕩弛縱,\begin{note}庚雙夾:四字妙評。\end{note}任性恣情,\begin{note}庚雙夾:四字更好。亦不涉於惡,亦不涉於淫,亦不涉於嬌,不過一味任性耳。\end{note}最不喜務正。\begin{note}庚雙夾:這還是小兒同病。\end{note}每欲勸時,料不能聽,今日可巧有贖身之論,故先用騙詞,以探其情,以壓其氣,然後好下箴規。\begin{note}庚雙夾:原來如此。\end{note}今見他默默睡去了,知其情有不忍,氣已餒墮。\begin{note}庚雙夾:不獨解語,亦且有智。\end{note}自己原不想栗子喫的,只因怕爲酥酪又生事故,亦如茜雪之茶等事,\begin{note}庚雙夾:可謂賢而有智術之人。\end{note}是以假以栗子爲由,混過寶玉不提就完了。於是命小丫頭子們將栗子拿去吃了,自己來推寶玉。只見寶玉淚痕滿面,\begin{note}庚雙夾:正是無可奈何之時。\end{note}\begin{note}蒙側:不知何故,我亦掩涕。\end{note}襲人便笑道:“這有什麼傷心的,你果然留我,我自然不出去了。”寶玉見這話有文章,\begin{note}庚雙夾:寶玉不愚。\end{note}便說道:“你倒說說,我還要怎麼留你,我自己也難說了。”\begin{note}庚雙夾:二人素常情意。\end{note}襲人笑道:“咱們素日好處,再不用說。但今日你安心留我,不在這上頭。我另說出三件事來,你果然依了我,就是你真心留我了,刀擱在脖子上,我也是不出去的了。”
\end{parag}


\begin{parag}
    寶玉忙笑道:“你說,那幾件?我都依你。好姐姐,好親姐姐,\begin{note}庚雙夾:疊二語活見從紙上走一寶玉下來,如聞其呼、見其笑。\end{note}別說兩三件,就是兩三百件,我也依。\begin{note}庚雙夾:“兩三百”不成話,卻是寶玉口中。\end{note}只求你們同看著我,守著我,等我有一日化成了飛灰,\begin{note}庚雙夾:脂硯齋所謂“不知是何心思,始得口出此等不成話之至奇至妙之話”,諸公請如何解得,如何評論?所勸者正爲此,偏於勸時一犯,妙甚!\end{note}——飛灰還不好,灰還有形有跡,還有知識。\begin{note}庚雙夾:厭“還有知識”,奇之不可甚言矣!餘則謂人尚無知識者多多。\end{note}”“等我化成一股輕煙,風一吹便散了的時候,你們也管不得我,我也顧不得你們了。那時憑我去,我也憑你們愛那裏去就去了。”\begin{note}庚雙夾:是聰明,是愚昧,是小兒淘氣?餘皆不知,只覺悲感難言,奇瑰愈妙。\end{note}話未說完,急的襲人忙握他的嘴,說:“好好的,正爲勸你這些,倒更說的狠了。”寶玉忙說道:“再不說這話了。”\begin{note}庚側:只說今日一次。呵呵,玉兄,玉兄,你到底哄的那一個?\end{note}襲人道:“這是頭一件要改的。”寶玉道:“改了。再要說,你就擰嘴。還有什麼?”
\end{parag}


\begin{parag}
    襲人道:“第二件,你真喜讀書也罷,假喜也罷,\begin{note}庚側:新鮮,真新鮮!\end{note}只是在老爺跟前或在別人跟前,你別隻管批駁誚謗,只作出個喜讀書的樣子來,\begin{note}庚雙夾:所謂“開方便門”。\end{note}\begin{note}庚雙夾:寶玉又誚謗讀書人,恨此時不能一見如何誚謗。\end{note}也教老爺少生些氣,\begin{note}庚側:大家聽聽,可是個丫鬟說的話。\end{note}在人前也好說嘴。他心裏想著,我家代代唸書,只從有了你,不承望你不喜讀書,已經他心裏又氣又惱了。而且背前背後亂說那些混話,凡讀書上進的人,你就起個名字叫作‘祿蠹’;\begin{note}庚雙夾:二字從古未見,新奇之至!難怨世人謂之可殺,餘卻最喜。\end{note}又說只除‘明明德’外無書,都是前人自己不能解聖人之書,便另出己意,混編纂出來的。\begin{note}庚雙夾:寶玉目中猶有“明明德”三字,心中猶有“聖人”二字,又素日皆作如是等語,宜乎人人謂之瘋傻不肖。\end{note}這些話,你怎麼怨得老爺不氣?不時時打你。叫別人怎麼想你?”寶玉笑道:“再不說了。那原是那小時不知天高地厚,信口胡說,如今再不敢說了。\begin{note}庚雙夾:又作是語,說不得不乖覺,然又是作者瞞人之處也。\end{note}還有什麼?”
\end{parag}


\begin{parag}
    襲人道:“再不許毀僧謗道,\begin{note}庚雙夾:一件,是婦女心意。\end{note}調脂弄粉。\begin{note}庚雙夾:二件,若不如此,亦非寶玉。\end{note}還有更要緊的一件,\begin{note}庚雙夾:忽又作此一語。\end{note}再不許喫人嘴上擦的胭脂了,\begin{note}庚雙夾:此一句是聞所未聞之語,宜乎其父母嚴責也。\end{note}與那愛紅的毛病兒。”寶玉道:“都改,都改。再有什麼,快說。”襲人笑道:“再也沒有了。只是百事檢點些,不任意任情的就是了。\begin{note}庚雙夾:總包括盡矣。其所謂“花解語”者,大矣!不獨冗冗爲兒女之分也。\end{note}你若果都依了,便拿八人轎也抬不出我去了。”寶玉笑道:“你在這裏長遠了,不怕沒八人轎你坐。”襲人冷笑道:“這我可不希罕的。有那個福氣,沒有那個道理。縱坐了,也沒甚趣。”\begin{note}庚雙夾:調侃不淺,然在襲人能作是語,實可愛可敬可服之至,所謂“花解語”也。\end{note}\begin{note}庚眉:“花解語”一段乃襲卿滿心滿意將玉兄爲終身得靠,千妥萬當,故有是。餘閱至此,餘爲襲卿一嘆。丁亥春。畸笏叟。\end{note}
\end{parag}


\begin{parag}
    二人正說著,只見秋紋走進來,說:“快三更了,該睡了。方纔老太太打發嬤嬤來問,我答應睡了。”寶玉命取表來\begin{note}庚雙夾:照應前鳳姐之前文。\end{note}看時,果然針已指到亥正,\begin{note}庚雙夾:表則是表的寫法,前形容自鳴鐘則是自鳴鐘,各盡其神妙。\end{note}方從新盥漱,寬衣安歇,不在話下。
\end{parag}


\begin{parag}
    至次日清晨,襲人起來,便覺身體發重,頭疼目脹,四肢火熱。先時還扎掙的住,次後挨不住,只要睡著,因而和衣躺在炕上。\begin{note}庚側:過下引線。\end{note}寶玉忙回了賈母,傳醫診視,說道:“不過偶感風寒,喫一兩劑藥疏散疏散就好了。”開方去後,令人取藥來煎好,剛服下去,命他蓋上被渥汗,寶玉自去黛玉房中來看視。\begin{note}庚雙夾:爲下文留地步。\end{note}
\end{parag}


\begin{parag}
    彼時黛玉自在牀上歇午,丫鬟們皆出去自便,滿屋內靜悄悄的。寶玉揭起繡線軟簾,進入裏間,只見黛玉睡在那裏,忙走上來推他道:“好妹妹,\begin{note}庚雙夾:才住了“好姐姐”,又聞“好妹妹”,大約寶玉一日之中一時之內,此六個字未曾暫離口角。妙!\end{note}才吃了飯,又睡覺。”將黛玉喚醒。\begin{note}庚雙夾:若是別部書中寫,此時之寶玉一進來,便生不軌之心,突萌苟且之念,更有許多賊形鬼狀等醜態邪言矣。此卻反推喚醒他,毫不在意,所謂說不得淫蕩是也。\end{note}黛玉見是寶玉,因說道:“你且出去逛逛,我前兒鬧了一夜,今兒還沒有歇過來,\begin{note}庚雙夾:補出嬌怯態度。\end{note}渾身痠疼。”寶玉道:“痠疼事小,睡出來的病大。我替你解悶兒,混過困去就好了。”\begin{note}庚雙夾:寶玉又知養身。\end{note}黛玉只合著眼,說道:“我不困,只略歇歇兒,你且別處去鬧會子再來。”寶玉推他道:“我往那裏去呢,見了別人就怪膩的。”\begin{note}庚雙夾:所謂只有一顰可對,亦屬怪事。\end{note}
\end{parag}


\begin{parag}
    黛玉聽了,嗤的一聲笑道:“你既要在這裏,那邊去老老實實的坐著,咱們說話兒。”寶玉道:“我也歪著。”黛玉道:“你就歪著。”寶玉道:“沒有枕頭,\begin{note}庚雙夾:纏綿祕密入微。\end{note}咱們在一個枕頭上。”\begin{note}庚雙夾:更妙!漸逼漸近,所謂“意綿綿”也。\end{note}黛玉道:“放屁!\begin{note}庚側:如聞。\end{note}外面不是枕頭?拿一個來枕著。”寶玉出至外間,看了一看,回來笑道:“那個我不要,也不知是那個髒婆子的。”黛玉聽了,睜開眼,\begin{note}庚雙夾:睜眼。\end{note}起身\begin{note}庚雙夾:起身。\end{note}笑\begin{note}庚雙夾:笑。\end{note}道:“真真你就是我命中的‘天魔星’!\begin{note}庚雙夾:妙語,妙之至!想見其態度。\end{note}請枕這一個。”說著,將自己枕的推與寶玉,又起身將自己的再拿了一個來,自己枕了,二人對面躺下。
\end{parag}


\begin{parag}
    黛玉因看見寶玉左邊腮上有鈕釦大小的一塊血漬,便欠身湊近前來,以手撫之細看,\begin{note}庚雙夾:想見其纏綿態度。\end{note}又道:“這又是誰的指甲刮破了?”\begin{note}庚雙夾:妙極!補出素日。\end{note}寶玉側身,一面躲,\begin{note}庚側:對“推醒”看。\end{note}一面笑道:“不是刮的,只怕是纔剛替他們淘漉胭脂膏子,蹭上了一點兒。”\begin{note}庚雙夾:遙與後文平兒於怡紅院晚妝時對照。\end{note}說著,便找手帕子要揩拭。黛玉便用自己的帕子替他揩拭了,\begin{note}庚雙夾:想見其情之脈脈,意之綿綿。\end{note}口內說道:“你又幹這些事了。\begin{note}庚雙夾:又是勸戒語。\end{note}幹也罷了,\begin{note}庚雙夾:一轉,細極!這方是顰卿,不比別人一味固執死勸。\end{note}必定還要帶出幌子來。便是舅舅看不見,別人看見了,又當奇事新鮮話兒去學舌討好兒,\begin{note}庚雙夾:補前文之未到,伏後文之線脈。\end{note}吹到舅舅耳朵裏,又該大家不乾淨惹氣。”\begin{note}庚雙夾:“大家”二字何妙之至神之至細膩之至!乃父責其子,縱加以笞楚,何能使大家不乾淨哉?今偏大家不乾淨,則知賈母如何管孫責子怒於衆,及自己心中多少抑鬱。難堪難禁,代憂代痛,一齊托出。\end{note}
\end{parag}


\begin{parag}
    寶玉總未聽見這些話,\begin{note}庚雙夾:可知昨夜“情切切”之語亦屬行雲流水矣。\end{note}只聞得一股幽香,卻是從黛玉袖中發出,聞之令人醉魂酥骨。\begin{note}庚雙夾:卻像似淫極,然究竟不犯一些淫意。\end{note}寶玉一把便將黛玉的袖子拉住,要瞧籠著何物。黛玉笑道:“冬寒十月,\begin{note}庚側:口頭語,指在春冷之時。\end{note}誰帶什麼香呢。”寶玉笑道:“既然如此,這香是從那裏來的?”黛玉道:“連我也不知道。\begin{note}庚雙夾:正是按諺雲:“人在氣中忘氣,魚在水中忘水。”餘今續之曰:“美人忘容,花則忘香。”此則黛玉不知自骨肉中之香同。\end{note}想必是櫃子裏頭的香氣,衣服上薰染的也未可知。”\begin{note}庚雙夾:有理。\end{note}寶玉搖頭道:“未必。這香的氣味奇怪,不是那些香餅子、香毬子、香袋子的香。”\begin{note}庚雙夾:自然。\end{note}黛玉冷笑\begin{note}庚雙夾:冷笑便是文章。\end{note}道:“難道我也有什麼‘羅漢’‘真人’給我些香不成?便是得了奇香,也沒有親哥哥親兄弟弄了花兒、朵兒、霜兒、雪兒替我炮製。\begin{note}庚雙夾:活顰兒,一絲不錯。\end{note}我有的是那些俗香罷了!”
\end{parag}


\begin{parag}
    寶玉笑道:“凡我說一句,你就拉上這麼些,不給你個利害,也不知道,從今兒可不饒你了。”說著翻身起來,將兩隻手呵了兩口,\begin{note}庚雙夾:活畫。\end{note}便伸手向黛玉膈肢窩內兩脅下亂撓。黛玉素性觸癢不禁,寶玉兩手伸來亂撓,便笑的喘不過氣來,口裏說:“寶玉!你再鬧,我就惱了。”\begin{note}庚雙夾:如見如聞。\end{note}寶玉方住了手,笑問道:“你還說這些不說了?”黛玉笑道:“再不敢了。”一面理鬢笑道:“我有奇香,你有‘暖香’沒有?”\begin{note}庚雙夾:奇聞。\end{note}
\end{parag}


\begin{parag}
    寶玉見問,一時解不來,\begin{note}庚雙夾:一時原難解,終遜黛卿一等,正在此等處。\end{note}因問:“什麼‘暖香’?”黛玉點頭嘆笑道:“蠢才,蠢才!你有玉,人家就有金來配你;人家有‘冷香’,你就沒有‘暖香’去配?”寶玉方聽出來。\begin{note}庚雙夾:是顰兒,活畫。然這是阿顰一生心事,故每不禁自及之。\end{note}寶玉笑道:“方纔求饒,如今更說狠了。”說著,又去伸手。黛玉忙笑道:“好哥哥,我可不敢了。”寶玉笑道:“饒便饒你,只把袖子我聞一聞。”說著,便拉了袖子籠在面上,聞個不住。黛玉奪了手道:“這可該去了。”寶玉笑道:“去,不能。咱們斯斯文文的躺著說話兒。”說著,復又倒下。黛玉也倒下,用手帕子蓋上臉。\begin{note}庚雙夾:畫。\end{note}寶玉有一搭沒一搭的說些鬼話,\begin{note}庚雙夾:先一總。\end{note}黛玉總不理。寶玉問他幾歲上京,路上見何景緻古蹟,揚州有何遺蹟故事,土俗民風。黛玉只不答。
\end{parag}


\begin{parag}
    寶玉只怕他睡出病來,\begin{note}庚雙夾:原來只爲此故,不暇旁人嘲笑,所以放蕩無忌處不特此一件耳。\end{note}便哄他道:“噯喲!\begin{note}庚側:像個說故事的。\end{note}你們揚州衙門裏有一件大故事,你可知道?”黛玉見他說的鄭重,且又正言厲色,只當是真事,因問:“什麼事?”寶玉見問,便忍著笑順口謅道:\begin{note}庚側:又哄我看書人。\end{note}“揚州有一座黛山,山上有個林子洞。”黛玉笑道:“這就扯謊,自來也沒聽見這山。”\begin{note}庚側:山名洞名,顰兒已知之矣。\end{note}寶玉道:“天下山水多著呢,你那裏知道這些不成。等我說完了,\begin{note}庚側:不先了此句,可知此謊再謅不完的。\end{note}你再批評。”黛玉道:“你且說。”寶玉又謅道:“林子洞裏原來有羣耗子精。那一年臘月初七日,老耗子升座議事,\begin{note}庚雙夾:耗子亦能升座且議事,自是耗子有賞罰有制度矣。何今之耗子猶穿壁齧物,其升座者置而不問哉?\end{note}因說:‘明日是臘八,世上人都熬臘八粥。如今我們洞中果品短少,\begin{note}庚側:難道耗子也要臘八粥喫?一笑。\end{note}須得趁此打劫些來方妙。’\begin{note}庚雙夾:議得是,這事宜乎爲鼠矣。\end{note}乃拔令箭一枝,遣一能幹小耗\begin{note}庚雙夾:原來能於此者便是小鼠。\end{note}前去打聽。一時小耗回報:‘各處察訪打聽已畢,惟有山下廟裏果米最多。’\begin{note}蒙雙夾:廟裏原來最多,妙妙!\end{note}老耗問:‘米有幾樣?果有幾品?’小耗道:‘米豆成倉,不可勝記。果品有五種:一紅棗,二栗子,三落花生,四菱角,五香芋。’老耗聽了大喜,即時點耗前去。乃拔令箭問:‘誰去偷米?’一耗便接令去偷米。又拔令箭問:‘誰去偷豆?’又一耗接令去偷豆。然後一一的都各領令去了。\begin{note}庚側:玉兄也知瑣碎,以抄近爲妙。\end{note}只剩了香芋一種,因又拔令箭問:‘誰去偷香芋?’只見一個極小極弱的小耗\begin{note}庚側:玉兄,玉兄,唐突顰兒了!\end{note}應道:‘我願去偷香芋。’老耗和衆耗見他這樣,恐不諳練,且怯懦無力,都不准他去。小耗道:‘我雖年小身弱,卻是法術無邊,口齒伶俐,機謀深遠。\begin{note}庚雙夾:凡三句暗爲黛玉作評,諷得妙!\end{note}此去管比他們偷的還巧呢。”衆耗忙問:‘如何比他們巧呢?’小耗道:‘我不學他們直偷。\begin{note}庚側:不直偷,可畏可怕。\end{note}我只搖身一變,也變成個香芋,滾在香芋堆裏,使人看不出,聽不見,卻暗暗的用分身法搬運,\begin{note}庚側:可怕可畏。\end{note}漸漸的就搬運盡了。豈不比直偷硬取的巧些?’\begin{note}庚雙夾:果然巧,而且最毒。直偷者可防,此法不能防矣。可惜這樣才情這樣學術卻只一耗耳。\end{note}衆耗聽了,都道:‘妙卻妙,只是不知怎麼個變法?你先變個我們瞧瞧。’小耗聽了,笑道:‘這個不難,等我變來。’說畢,搖身說‘變’,竟變了一個最標緻美貌的一位小姐。\begin{note}庚側:奇文怪文。\end{note}衆耗忙笑說:‘變錯了,變錯了。原說變果子的,如何變出小姐來?’\begin{note}庚雙夾:餘亦說變錯了。\end{note}小耗現形笑道:“我說你們沒見世面,只認得這果子是香芋,卻不知鹽課林老爺的小姐纔是真正的香玉呢。’”\begin{note}庚雙夾:前有“試才題對額”,故緊接此一篇無稽亂話,前無則可,此無則不可,蓋前系寶玉之懶爲者,此係寶玉不得不爲者。世人誹謗無礙,獎譽不必。\end{note}
\end{parag}


\begin{parag}
    黛玉聽了,翻身爬起來,按著寶玉笑道:“我把你爛了嘴的!我就知道你是編我呢。”說著,便擰的寶玉連連央告,說:“好妹妹,饒我罷,再不敢了!我因爲聞你香,忽然想起這個故典來。”黛玉笑道:“饒罵了人,還說是故典呢。”\begin{note}庚眉:“玉生香”是要與“小恙梨香院”對看,愈覺生動活潑,且前以黛玉後以寶釵,特犯不犯,好看煞!丁亥春。 笏叟。\end{note}
\end{parag}


\begin{parag}
    一語未了,只見寶釵走來,\begin{note}庚雙夾:妙!\end{note}笑問:“誰說故典呢?我也聽聽。”黛玉忙讓坐,笑道:“你瞧瞧,有誰!他饒罵了人,還說是故典。”寶釵笑道:“原來是寶兄弟,怨不得他,他肚子裏的故典原多。\begin{note}庚雙夾:妙諷。\end{note}只是可惜一件,\begin{note}庚雙夾:妙轉。\end{note}凡該用故典之時,他偏就忘了。\begin{note}庚雙夾:更妙!\end{note}有今日記得的,前兒夜裏的芭蕉詩就該記得。眼面前的倒想不起來,別人冷的那樣,你急的只出汗。\begin{note}庚雙夾:與前“拭汗”二字針對,不知此書何妙之如此,有許多妙談妙語、機諷詼諧,各得其時,各盡其理,前梨香院黛玉之諷則偏見,越此則正而趣,二人真是對手,兩不相犯。\end{note}這會子偏又有記性了。”黛玉聽了笑道:“阿彌陀佛!到底是我的好姐姐,你一般也遇見對子了。可知一還一報,不爽不錯的。”剛說到這裏,只聽寶玉房中一片聲嚷,吵鬧起來。正是——
\end{parag}


\begin{parag}
    \begin{note}蒙回末總評:若知寶玉真性情者,當留心此回。其於襲人何等留連,其於畫美人何等古怪。其遇茗煙事何等憐惜,其於黛玉何等保護。再襲人之癡忠,畫人之惹事,茗煙之屈奉,黛玉之癡情,千態萬狀,筆力勁尖,有水到渠成之象,無微不至。真畫出一個上乘智慧之人,入於魔而不悟,甘心墮落。且影出諸魔之神通,亦非冷冷,有勢不能登彼岸。凡我衆生掩卷自思,或於身心少有補益。小子妄談,諸公莫怪。\end{note}
\end{parag}


\begin{parag}
    \begin{note}夢:正是:戲謔主人調笑僕,相合姊妹合歡親。\end{note}
\end{parag}
