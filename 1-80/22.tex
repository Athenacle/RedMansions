\chap{二十二}{聽曲文寶玉悟禪機 制燈迷賈政悲讖語}

\begin{parag}
    \begin{note}蒙回前詩:禪理偏成曲調,燈謎巧引讖言。其中冷暖自尋看,盡夜因循暗轉。\end{note}
\end{parag}


\begin{parag}
    話說賈璉聽鳳姐兒說有話商量,因止步問是何話。鳳姐道:“二十一是薛妹妹的生日,\begin{note}庚雙夾:好!\end{note}你到底怎麼樣呢?”賈璉道:“我知道怎麼樣!你連多少大生日都料理過了,這會子倒沒了主意?”鳳姐道:“大生日料理,不過是有一定的則例在那裏。如今他這生日,大又不是,小又不是,所以和你商量。”\begin{note}庚雙夾:有心機人在此。\end{note}賈璉聽了,低頭想了半日道:“你今兒糊塗了。現有比例,那林妹妹就是例。往年怎麼給林妹妹過的,如今也照依給薛妹妹過就是了。”\begin{note}庚雙夾:比例引的極是。無怪賈政委以家務也。\end{note}鳳姐聽了,冷笑道:“我難道連這個也不知道?我原也這麼想定了。但昨兒聽見老太太說,問起大家的年紀生日來,聽見薛大妹妹今年十五歲,雖不是整生日,也算得將笄之年。老太太說要替他作生日。想來若果真替他作,自然比往年與林妹妹的不同了。”賈璉道:“既如此,比林妹妹的多增些。”鳳姐道:“我也這們想著,所以討你的口氣。我若私自添了東西,你又怪我不告訴明白你了。”賈璉笑道:“罷,罷,這空頭情我不領。你不盤察我就夠了,我還怪你!”說著,一徑去了,不在話下。\begin{note}庚雙夾:一段題綱寫得如見如聞,且不失前篇懼內之旨。最奇者黛玉乃賈母溺愛之人也,不聞爲作生辰,卻去特意與寶釵,實非人想得著之文也。此書通部皆用此法,瞞過多少見者,餘故云不寫而寫是也。\end{note}\begin{note}庚眉:將薛、林作甄玉、賈玉看書,則不失執筆人本家。丁亥夏。笏叟。\end{note}
\end{parag}


\begin{parag}
    且說史湘雲住了兩日,因要回去。賈母因說:“等過了你寶姐姐的生日,看了戲再回去。”史湘雲聽了,只得住下。又一面遣人回去,將自己舊日作的兩色針線活計取來,爲寶釵生辰之儀。
\end{parag}


\begin{parag}
    誰想賈母自見寶釵來了,喜他穩重和平,\begin{note}庚雙夾:四字評倒黛玉,是以特從賈母眼中寫出。\end{note}正值他才過第一個生辰,便自己蠲資二十兩,\begin{note}庚雙夾:寫出太君高興,世家之常事耳。\end{note}\begin{note}庚眉:前看鳳姐問作生日數語甚泛泛,至此見賈母蠲資,方知作者寫阿鳳心機無絲毫漏筆。己冬夜。\end{note}喚了鳳姐來,交與他置酒戲。鳳姐湊趣笑道:“一個老祖宗給孩子們作生日,\begin{note}庚側:家常話,卻是空中樓閣,陡然架起。\end{note}不拘怎樣,誰還敢爭,又辦什麼酒戲。既高興要熱鬧,就說不得自己花上幾兩。巴巴的找出這黴爛的二十兩銀子來作東道,這意思還叫我賠上。果然拿不出來也罷了,金的、銀的、圓的、扁的,壓塌了箱子底,\begin{note}庚眉:小科諢解頤,卻爲借當伏線。壬午九月。\end{note}只是勒掯我們。舉眼看看,誰不是兒女?難道將來只有寶兄弟頂了你老人家上五臺山不成?那些梯己只留於他,我們如今雖不配使,也別苦了我們。這個夠酒的?夠戲的?” 說的滿屋裏都笑起來。賈母亦笑道:“你們聽聽這嘴!我也算會說的,怎麼說不過這猴兒。你婆婆也不敢強嘴,你和我嗙嗙的。”鳳姐笑道:“我婆婆也是一樣的疼寶玉,我也沒處去訴冤,倒說我強嘴。”說著,又引著賈母笑了一回,\begin{note}庚側:正文在此一句。\end{note}賈母十分喜悅。
\end{parag}


\begin{parag}
    到晚間,衆人都在賈母前,定昏之餘,大家孃兒姊妹等說笑時,賈母因問寶釵愛聽何戲,愛喫何物等語。寶釵深知賈母年老人,喜熱鬧戲文,愛喫甜爛之食,便總依賈母往日素喜者說了出來。\begin{note}庚側:看他寫寶釵,比顰兒如何?\end{note}賈母更加歡悅。次日便先送過衣服玩物禮去,王夫人、鳳姐、黛玉等諸人皆有隨分不一,不須多記。
\end{parag}


\begin{parag}
    至二十一日,就賈母內院中搭了家常小巧戲臺,\begin{note}庚雙夾:另有大禮所用之戲臺也,侯門風俗斷不可少。\end{note}定了一班新出小戲,昆弋兩腔皆有!\begin{note}蒙雙夾:是賈母好熱鬧之故。\end{note}就在賈母上房排了几席家宴酒席,\begin{note}庚雙夾:是家宴,非東閣盛設也。非世代公子再想不及此。\end{note}並無一個外客,只有薛姨媽、史湘雲、寶釵是客,餘者皆是自己人。\begin{note}庚雙夾:將黛玉亦算爲自己人,奇甚!\end{note}這日早起,寶玉因不見林黛玉,\begin{note}庚雙夾:又轉至黛玉文字,人不可少也。\end{note}便到他房中來尋,只見林黛玉歪在炕上。寶玉笑道:“起來喫飯去,就開戲了。你愛看那一出?我好點。”林黛玉冷笑道:“你既這樣說,你特叫一班戲來,揀我愛的唱給我看。這會子犯不上跐著人借光兒問我。”\begin{note}庚雙夾:好聽之極,令人絕倒。\end{note}寶玉笑道:“這有什麼難的。明兒就這樣行,也叫他們借咱們的光兒。”一面說,一面拉起他來,攜手出去。
\end{parag}


\begin{parag}
    吃了飯點戲時,賈母一定先叫寶釵點。寶釵推讓一遍,無法,只得點了一折《西遊記》。\begin{note}庚雙夾:是順賈母之心也。\end{note}賈母自是歡喜,然後便命鳳姐點。鳳姐亦知賈母喜熱鬧,更喜謔笑科諢,\begin{note}庚雙夾:寫得周到,想得奇趣,實是必真有之。\end{note}便點了一出《劉二當衣》。\begin{note}庚眉:鳳姐點戲,脂硯執筆事,今知者寥寥矣,不怨夫?\end{note}\begin{note}庚眉:前批“知者寥寥”,今丁亥夏只剩朽物一枚,寧不悲乎!\end{note}\begin{note}靖眉:前批“知者寥寥”,芹溪、脂硯、杏齋諸子皆相繼別去,今丁亥夏只剩朽物一枚,寧不痛殺!\end{note}賈母果真更又喜歡,然後便命黛玉點。\begin{note}庚雙夾:先讓鳳姐點者,是非待鳳先而後玉也。蓋亦素喜鳳嘲笑得趣之故,今故命彼點,彼亦自知,並不推讓,承命一點,便合其意。此篇是賈母取樂,非禮筵大典,故如此寫。\end{note}黛玉因讓薛姨媽王夫人等。賈母道:“今日原是我特帶著你們取笑,咱們只管咱們的,別理他們。我巴巴的唱戲擺酒,爲他們不成?他們在這裏白聽白喫,已經便宜了,還讓他們點呢!”說著,大家都笑了。黛玉方點了一出。\begin{note}蒙雙夾:不題何戲,妙!蓋黛玉不喜看戲也。正是與後文“妙曲警芳心”留地步,正見此時不過草草隨衆而已,非心之所願也。\end{note}然後寶玉、史湘雲、迎、探、惜、李紈等俱各點了,接出扮演。
\end{parag}


\begin{parag}
    至上酒席時,賈母又命寶釵點。寶釵點了一出《魯智深醉鬧五臺山》。寶玉道:“只好點這些戲。”寶釵道:“你白聽了這幾年的戲,那裏知道這齣戲的好處,排場又好,詞藻更妙。”寶玉道:“我從來怕這些熱鬧。”寶釵笑道:“要說這一出熱鬧,你還算不知戲呢。\begin{note}庚雙夾:是極!寶釵可謂博學矣,不似黛玉只一《牡丹亭》便心身不自主矣。真有學問如此,寶釵是也。\end{note}你過來,我告訴你,這一齣戲熱鬧不熱鬧。”“是一套北《點絳脣》,鏗鏘頓挫,韻律不用說是好的了,只那詞藻中有一支《寄生草》,填的極妙,你何曾知道。”寶玉見說的這般好,便湊近來央告:“好姐姐,念與我聽聽。”寶釵便念道:“漫搵英雄淚,相離處士家。謝慈悲剃度在蓮臺下。沒緣法轉眼分離乍。赤條條來去無牽掛。那裏討煙蓑雨笠卷單行?一任俺芒鞋破鉢隨緣化!”\begin{note}庚雙夾:此闋出自《山門》傳奇。近之唱者將“一任俺”改爲“早辭卻”,無理不通之甚。必從“一任俺”三字,則“隨緣”二字方不脫落。\end{note}
\end{parag}


\begin{parag}
    寶玉聽了,喜的拍膝畫圈,稱賞不已,又贊寶釵無書不知,林黛玉道:“安靜看戲罷,還沒唱《山門》,你倒《妝瘋》了。”\begin{note}庚雙夾:趣極!今古利口莫過於優伶。此一詼諧,優伶亦不得如此急速得趣,可謂才人百技也。一段醋意可知。\end{note}說的湘雲也笑了。於是大家看戲。
\end{parag}


\begin{parag}
    至晚散時,賈母深愛那作小旦的與一個作小丑的,因命人帶進來,細看時益發可憐見。\begin{note}庚雙夾:是賈母眼中之見、心內之想。\end{note}因問年紀,那小旦才十一歲,小丑才九歲,大家嘆息一回。賈母令人另拿些肉果與他兩個,又另外賞錢兩串。鳳姐笑道:“這個孩子扮上活象一個人,\begin{note}庚側:明明不叫人說出。\end{note}你們再看不出來。”寶釵心裏也知道,便只一笑,不肯說。\begin{note}庚雙夾:寶釵如此。\end{note}寶玉也猜著了,亦不敢說。\begin{note}庚雙夾:不敢少。\end{note}史湘雲接著笑道:“倒象林妹妹的模樣兒。”\begin{note}庚側:事無不可對人言。\end{note}\begin{note}庚雙夾:口直心快,無有不可說之事。\end{note}\begin{note}庚眉:湘雲探春二卿,正“事無不可對人言”芳性。丁亥夏。笏叟。\end{note}寶玉聽了,忙把湘雲瞅了一眼,使個眼色。衆人卻都聽了這話,留神細看,都笑起來了,說果然不錯。一時散了。
\end{parag}


\begin{parag}
    晚間,湘雲更衣時,便命翠縷把衣包打開收拾,都包了起來。翠縷道:“忙什麼,等去的日子再包不遲。”湘雲道:“明兒一早就走。在這裏作什麼?――看人家的鼻子眼睛,什麼意思!”\begin{note}庚雙夾:此是真惱,非顰兒之惱可比,然錯怪寶玉矣。亦不可不惱。\end{note}寶玉聽了這話,忙趕近前拉他說道:“好妹妹,你錯怪了我。林妹妹是個多心的人。別人分明知道,不肯說出來,也皆因怕他惱。誰知你不防頭就說了出來,他豈不惱你。我是怕你得罪了他,所以才使眼色。你這會子惱我,不但辜負了我,而且反倒委曲了我。若是別人,那怕他得罪了十個人,與我何干呢。”湘雲摔手道:“你那花言巧語別哄我。我也原不如你林妹妹,別人說他,拿他取笑都使得,只我說了就有不是。我原不配說他。他是小姐主子,我是奴才丫頭,得罪了他,使不得!”寶玉急的說道:“我倒是爲你,反爲出不是來了。我要有外心,\begin{note}庚側:玉兄急了。\end{note}立刻就化成灰,叫萬人踐踹!”\begin{note}庚雙夾:千古未聞之誓,懇切盡情。寶玉此刻之心爲如何?\end{note}湘雲道:“大正月裏,少信嘴胡說。\begin{note}庚側:迴護石兄。\end{note}這些沒要緊的惡誓,散話,歪話,說給那些小性兒,行動愛惱的人,會轄治你的人\begin{note}庚側:此人爲誰?\end{note}聽去!別叫我啐你。”說著,一徑至賈母裏間,忿忿的躺著去了。
\end{parag}


\begin{parag}
    寶玉沒趣,只得又來尋黛玉。剛到門檻前,黛玉便推出來,將門關上。寶玉又不解其意,在窗外只是吞聲叫“好妹妹”。黛玉總不理他。寶玉悶悶的垂頭自審。襲人早知端的,當此時斷不能勸。\begin{note}庚雙夾:寶玉在此時一勸必崩了,襲人見機甚妙。\end{note}那寶玉只是呆呆的站在那裏。
\end{parag}


\begin{parag}
    黛玉只當他回房去了,便起來開門,只見寶玉還站在那裏。黛玉反不好意思,不好再關,只得抽身上牀躺著。寶玉隨進來問道:“凡事都有個原故,說出來,人也不委曲。好好的就惱了,終是什麼原故起的?”林黛玉冷笑道:“問的我倒好,我也不知爲什麼原故。我原是給你們取笑的,”“拿我比戲子取笑。”寶玉道:“我並沒有比你,我並沒笑,爲什麼惱我呢?”黛玉道:“你還要比?你還要笑?\begin{note}庚側:可謂“官斷十條路”是也。\end{note}你不比不笑,比人比了笑了的還利害呢!”寶玉聽說,無可分辯,不則一聲。\begin{note}庚雙夾:何便無言可辯?真令人不解。前文湘雲方來,“正言彈妒意”一篇中,顰、玉角口後收至褂子一篇,餘已註明不解矣。回思自心自身是玉、顰之心,則洞然可解,否則無可解也。身非寶玉,則有辯有答;若寶玉,則再不能辯不能答。何也?總在二人心上想來。\end{note}\begin{note}庚眉:此書如此等文章多多不勝枚舉,機括神思自從天分而有。其毛錐寫人口氣傳神攝魄處,怎不令人拍案稱奇叫絕!丁亥夏。笏叟。\end{note}
\end{parag}


\begin{parag}
    黛玉又道:“這一節還恕得。再你爲什麼又和雲兒使眼色?這安的是什麼心?莫不是他和我頑,他就自輕自賤了?他原是公侯的小姐,我原是貧民的丫頭,他和我頑,設若我回了口,豈不他自惹人輕賤呢。是這主意不是?這卻也是你的好心,只是那一個偏又不領你這好情,一般也惱了。\begin{note}庚雙夾:顰兒自知雲兒惱,用心甚矣!\end{note}你又拿我作情,倒說我小性兒,\begin{note}庚雙夾:顰兒卻又聽見,用心甚矣!\end{note}行動肯惱。你又怕他得罪了我,我惱他。我惱他,與你何干?他得罪了我,又與你何干?”\begin{note}庚雙夾:問的卻極是,但未必心應。若能如此,將來淚盡夭亡已化烏有,世間亦無此一部《紅樓夢》矣。\end{note}\begin{note}庚眉:神工乎,鬼工乎?文思至此盡矣。丁亥夏。畸笏。\end{note}
\end{parag}


\begin{parag}
    寶玉見說,方纔與湘雲私談,他也聽見了。細想自己原爲他二人,怕生隙惱,方在中調和,不想並未調和成功,反已落了兩處的貶謗。正合著前日所看《南華經》上,有“巧者勞而智者憂,無能者無所求,飽食而遨遊,泛若不繫之舟”,又曰“山木自寇,\begin{note}庚雙夾:按原注:“山木,漆樹也。精脈自出,豈人所使之?故云‘自寇’,言自相戕賊也。”\end{note}源泉自盜”等語。\begin{note}庚雙夾:源泉味甘,然後人爭取之,自尋乾涸也,亦如山木意,皆寓人智能聰明多知之害也。前文無心雲看《南華經》,不過襲人等惱時,無聊之甚,偶以釋悶耳。殊不知用於今日,大解悟大覺迷之功甚矣。市徒見此必雲:前日看的是外篇《胠篋》,如何今日又知若許篇?然則彼時只曾看外篇數語乎?想其理,自然默默看過幾篇,適至外篇,故偶觸其機,方續之也。若雲只看了那幾句便續,則寶玉彼時之心是有意續《莊子》,並非釋悶時偶續之也。且更有見前所續,則曰續的不通,更可笑矣。試思寶玉雖愚,豈有安心立意與莊叟爭衡哉?且寶玉有生以來,此身此心爲諸女兒應酬不暇,眼前多少現成有益之事尚無暇去做,豈忽然要分心於腐言糟粕之中哉?可知除閨閣之外,並無一事是寶玉立意作出來的。大則天地陰陽,小則功名榮枯,以及吟篇琢句,皆是隨分觸情。偶得之,不喜;失之,不悲。若當作有心,謬矣。只看大觀園題詠之文,已算平生得意之句得意之事矣,然亦總不見再吟一句,再題一事,據此可見矣。然後可知前夜是無心順手拈了一本《莊子》在手,且酒興醮醮,芳愁默默,順手不計工拙,草草一續也。若使順手拈一本近時鼓詞,或如“鍾無豔赴會,齊太子走國”等草野風邪之傳,必亦續之矣。觀者試看此批,然後謂餘不謬。所以可恨者,彼夜卻不曾拈了《山門》一出傳奇。若使《山門》在案,彼時拈著,又不知於《寄生草》後續出何等超凡入聖大覺大悟諸語錄來。黛玉一生是聰明所誤,寶玉是多事所誤。多事者,情之事也,非世事也。多情曰多事,亦宗《莊》筆而來,蓋餘亦偏矣,可笑。阿鳳是機心所誤,寶釵是博識所誤,湘雲是自愛所誤,襲人是好勝所誤,皆不能跳出莊叟言外,悲亦甚矣。再筆。\end{note}因此越想越無趣。再細想來,目下不過這兩個人,尚未應酬妥協,將來猶欲爲何?\begin{note}庚雙夾:看他只這一筆,寫得寶玉又如何用心於世道。言閨中紅粉尚不能周全,何碌碌偕欲治世待人接物哉?視閨中自然如兒戲,視世道如虎狼矣,誰雲不然?\end{note}想到其間也無庸分辯回答自己轉身回房來。\begin{note}庚雙夾:顰兒雲“與你何干”,寶玉如此一回則曰“與我何干”可也。口雖未出,心已悟矣,但恐不常耳。若常存此念,無此一部書矣。看他下文如何轉折。\end{note}林黛玉見他去了,便知回思無趣,賭氣去了,一言也不曾發,不禁自己越發添了氣,\begin{note}庚雙夾:只此一句又勾起波浪。去則去,來則來,又何氣哉?總是斷不了這根孽腸,忘不了這個禍害,既無而又有也。\end{note}便說道:“這一去,一輩子也別來,也別說話。”
\end{parag}


\begin{parag}
    寶玉不理,\begin{note}庚雙夾:此是極心死處,將來如何?\end{note}回房躺在牀上,只是瞪瞪的。襲人深知原委,不敢就說,\begin{note}庚雙夾:一說必崩。\end{note}\begin{note}蒙雙夾:一說就惱。\end{note}只得以他事來解釋,因說道:“今兒看了戲,又勾出幾天戲來。寶姑娘一定要還席的。”寶玉冷笑道:“他還不還,管誰什麼相干。”\begin{note}庚雙夾:大奇大神之文。此“相干”之語仍是近文與顰兒之語之“相干”也。上文未說,終存於心,卻於寶釵身上發泄。素厚者唯顰、雲,今爲彼等尚存此心,況於素不契者有不直言者乎?情理筆墨,無不盡矣。\end{note}襲人見這話不是往日的口吻,因又笑道:“這是怎麼說?好好的大正月裏,娘兒們姊妹們都喜喜歡歡的,你又怎麼這個形景了?”寶玉冷笑道:“他們娘兒們姊妹們歡喜不歡喜,也與我無干。”\begin{note}庚雙夾:先及寶釵,後及衆人,皆一顰之禍流毒於衆人。寶玉之心僅有一顰乎。\end{note}襲人笑道:“他們既隨和,你也隨和,豈不大家彼此有趣。”寶玉道:“什麼是‘大家彼此’!他們有‘大家彼此’,我是‘赤條條來去無牽掛’。”\begin{note}庚雙夾:拍案叫好!當此一發,西方諸佛亦來聽此棒喝,參此語錄。\end{note}談及此句,不覺淚下。\begin{note}庚雙夾:還是心中不靜、不了、斬不斷之故。\end{note}襲人見此光景,不肯再說。寶玉細想這句趣味,不禁大哭起來,\begin{note}庚雙夾:此是忘機大悟,世人所謂瘋癲是也。\end{note}翻身起來至案,遂提筆立佔一偈雲:
\end{parag}


\begin{poem}
    \begin{pl}你證我證,心證意證。\end{pl}

    \begin{pl}是無有證,斯可雲證。\end{pl}

    \begin{pl}無可雲證,是立足境。\end{pl}
    \begin{note}蒙雙夾:已悟已覺。是好偈矣。寶玉悟禪亦由情,讀書亦由情,讀《莊》亦由情。可笑。\end{note}
\end{poem}


\begin{parag}
    寫畢,自雖解悟,又恐人看此不解,\begin{note}庚雙夾:自悟則自了,又何用人亦解哉?此正是猶未正覺大悟也。\end{note}因此亦填一支《寄生草》,也寫在偈後。\begin{note}庚雙夾:此處亦續《寄生草》。餘前批雲不曾見續,今卻見之,是意外之幸也。蓋前夜《莊子》是道悟,此日是禪悟,天花散漫之文也。\end{note}自己又念一遍,自覺無掛礙,中心自得,便上牀睡了。\begin{note}庚雙夾:前夜已悟,今夜又悟,二次翻身不出,故一世墮落無成也。不寫出曲文何辭,卻留於寶釵眼中寫出,是交代過節也。\end{note}
\end{parag}


\begin{parag}
    誰想黛玉見寶玉此番果斷而去,故以尋襲人爲由,來視動靜。\begin{note}庚雙夾:這又何必?總因慧刀不利,未斬毒龍之故也。大都如此,嘆嘆!\end{note}襲人笑回:“已經睡了。”黛玉聽說,便要回去。襲人笑道:“姑娘請站住,有一個字帖兒,瞧瞧是什麼話。”說著,便將方纔那曲子與偈語悄悄拿來,遞與黛玉看。黛玉看了,知是寶玉一時感忿而作,不覺可笑可嘆,\begin{note}庚雙夾:是個善知覺。何不趁此大家一解,齊證上乘,甘心墮落迷津哉?\end{note}便向襲人道:“作的是玩意兒,無甚關係。”\begin{note}庚雙夾:黛玉說“無關係”,將來必無關係。餘正恐顰、玉從此一悟則無妙文可看矣。不想顰兒視之爲漠然,更曰“無關係”,可知寶玉不能悟也。餘心稍慰。蓋寶玉一生行爲,顰知最確,故餘聞語則信而又信,不必寶玉而後證之方信也,餘雲恐他二人一悟則無妙文可看,然欲爲開我懷,爲醒我目,卻願他二人永墮迷津,生出孽障,餘心甚不公矣。世雲損人利己者,餘此願是矣。試思之,可發一笑。今自呈於此,亦可爲後人一笑,以助茶前酒後之興耳。而今後天地間豈不又添一趣談乎?凡書皆以趣談讀去,其理自明,其趣自得矣。\end{note}說畢,便攜了回房去,與湘雲同看。\begin{note}庚雙夾:卻不同湘雲分崩,有趣!\end{note}次日又與寶釵看。寶釵看其詞\begin{note}庚雙夾:出自寶釵目中,正是大關鍵處。\end{note}曰:
\end{parag}


\begin{poem}
    \begin{pl}無我原非你,從他不解伊。\end{pl}

    \begin{pl}肆行無礙憑來去。\end{pl}

    \begin{pl}茫茫著甚悲愁喜,紛紛說甚親疏密。\end{pl}

    \begin{pl}從前碌碌卻因何,到如今回頭試想真無趣!\end{pl}
    \begin{note}庚雙夾:看此一曲,試思作者當日發願不作此書,卻立意要作傳奇,則又不知有如何詞曲矣。\end{note}
\end{poem}


\begin{parag}
    看畢,又看那偈語,又笑道:“這個人悟了。都是我的不是,都是我昨兒一支曲子惹出來的。這些道書禪機最能移性。\begin{note}庚雙夾:拍案叫絕!此方是大悟徹語錄,非寶卿不能談此也。\end{note}明兒認真說起這些瘋話來,存了這個意思,都是從我這一隻曲子上來,我成了個罪魁了。”說著,便撕了個粉碎,遞與丫頭們說:“快燒了罷。”黛玉笑道:“不該撕,等我問他。你們跟我來,包管叫他收了這個癡心邪話。”
\end{parag}


\begin{parag}
    三人果然都往寶玉屋裏來。一進來,黛玉便笑道:“寶玉,我問你:至貴者是‘寶’,至堅者是‘玉’。爾有何貴?爾有何堅?”\begin{note}庚雙夾:拍案叫絕!大和尚來答此機鋒,想亦不能答也。非顰兒,第二人無此靈心慧性也。\end{note}寶玉竟不能答。三人拍手笑道:“這樣鈍愚,還參禪呢。”黛玉又道:“你那偈末雲:‘無可雲證,是立足境。’固然好了,只是據我看,還未盡善。我再續兩句在後。”因念雲:“無立足境,是方乾淨。”\begin{note}庚雙夾:拍案叫絕!此又深一層也。亦如諺雲:“去年貧,隻立錐;今年貧,錐也無。”其理一也。\end{note}寶釵道:“實在這方悟徹。當日南宗六祖惠能,\begin{note}庚眉:用得妥當之極!\end{note}初尋師至韶州,聞五祖弘忍在黃梅,他便充役火頭僧。五祖欲求法嗣,令徒弟諸僧各出一偈。上座神秀說道:‘身是菩提樹,心如明鏡臺,時時勤拂拭,莫使有塵埃。’彼時惠能在廚房碓米,聽了這偈,說道:‘美則美矣,了則未了。’因自念一偈曰:‘菩提本非樹,明鏡亦非臺,本來無一物,何處惹塵埃?‘五祖便將衣鉢傳他。\begin{note}庚雙夾:出語錄。總寫寶卿博學宏覽,勝諸才人;顰兒卻聰慧靈智,非學力所致——皆絕世絕倫之人也。寶玉寧不愧殺!\end{note}今兒這偈語,亦同此意了。只是方纔這句機鋒,尚未完全了結,這便丟開手不成?”黛玉笑道:“彼時不能答,就算輸了,這會子答上了也不爲出奇。只是以後再不許談禪了。連我們兩個所知所能的,你還不知不能呢,還去參禪呢。”寶玉自己以爲覺悟,不想忽被黛玉一問,便不能答,寶釵又比出“語錄”來,此皆素不見他們能者。自己想了一想:“原來他們比我的知覺在先,尚未解悟,我如今何必自尋苦惱。”\begin{note}庚眉:前以《莊子》爲引,故偶繼之。又借顰兒詩一鄙駁,兼不寫著落,以爲瞞過看官矣。此回用若許曲折,仍用老莊引出一偈來,再續一《寄生草》,可爲大覺大悟矣。以之上承果位,以後無書可作矣。卻又作黛玉一問機鋒,又續偈言二句,並用寶釵講五祖六祖問答二實偈子,使寶玉無言可答,仍將一大善知識,始終跌不出警幻幻榜中,作下回若干書。真有機心遊龍不測之勢,安得不叫絕?且歷來不說中萬寫不到者。己冬夜。\end{note}想畢,便笑道:“誰又參禪,不過一時頑話罷了。”說著,四人仍復如舊。\begin{note}庚雙夾:輕輕抹去也。“心靜難”三字不謬。\end{note}
\end{parag}


\begin{parag}
    忽然人報,娘娘差人送出一個燈謎兒,命你們大家去猜,猜著了每人也作一個進去。四人聽說忙出去,至賈母上房。只見一個小太監,拿了一盞四角平頭白紗燈,專爲燈謎而制,上面已有一個,衆人都爭看亂猜。小太監又下諭道:“衆小姐猜著了,不要說出來,每人只暗暗的寫在紙上,一齊封進宮去,娘娘自驗是否。”寶釵等聽了,近前一看,是一首七言絕句,並無甚新奇,口中少不得稱讚,只說難猜,故意尋思,其實一見就猜著了。寶玉、黛玉、湘雲、探春\begin{note}庚雙夾:此處透出探春,正是草蛇灰線,後文方不突然。\end{note}四個人也都解了,各自暗暗的寫了半日。一併將賈環,賈蘭等傳來,一齊各揣機心都猜了,\begin{note}庚雙夾:寫出猜謎人形景,看他偏於兩次戒機後,寫此機心機事,足見作意至深至遠。\end{note}寫在紙上。然後各人拈一物作成一謎,恭楷寫了,掛在燈上。
\end{parag}


\begin{parag}
    太監去了,至晚出來傳諭:“前娘娘所制,俱已猜著,惟二小姐與三爺猜的不是。\begin{note}庚雙夾:迎春、賈環也。交錯有法。\end{note}小姐們作的也都猜了,不知是否。”說著,也將寫的拿出來。也有猜著的,也有猜不著的,都胡亂說猜著了。太監又將頒賜之物送與猜著之人,每人一個宮制詩筒,\begin{note}庚雙夾:詩筒,身邊所佩之物,以待偶成之句草錄暫收之,其歸至窗前不致有忘也。或茜牙成,或琢香屑,或以綾素爲之不一,想來奇特事,從不知也。\end{note}一柄茶筅,\begin{note}庚雙夾:破竹如帚,以淨茶具之積也。二物極微極雅。\end{note}獨迎春、賈環二人未得。迎春自爲玩笑小事,並不介意,\begin{note}庚雙夾:大家小姐。\end{note}賈環便覺得沒趣。且又聽太監說:“三爺說的這個不通,娘娘也沒猜,叫我帶回問三爺是個什麼。”衆人聽了,都來看他作的什麼,寫道是:
\end{parag}


\begin{parag}
    大哥有角只八個,二哥有角只兩根。大哥只在牀上坐,二哥愛在房上蹲。\begin{note}庚雙夾:可發一笑,真環哥之謎。諸卿勿笑,難爲了作者摹擬。\end{note}
\end{parag}


\begin{parag}
    衆人看了,大發一笑。賈環只得告訴太監說:“一個枕頭,一個獸頭。”\begin{note}庚雙夾:虧他好才情,怎麼想來?\end{note}太監記了,領茶而去。
\end{parag}


\begin{parag}
    賈母見元春這般有興,自己越發喜樂,便命速作一架小巧精緻圍屏燈來,設於當屋,命他姊妹各自暗暗的作了,寫出來粘於屏上,然後預備下香茶細果以及各色玩物,爲猜著之賀。賈政朝罷,見賈母高興,況在節間,晚上也來承歡取樂。設了酒果,備了玩物,上房懸了彩燈,請賈母賞燈取樂。上面賈母、賈政、寶玉一席,下面王夫人、寶釵、黛玉、湘雲又一席,迎、探、惜三個又一席。地下婆娘丫鬟站滿。李宮裁、王熙鳳二人在裏間又一席。\begin{note}庚側:細緻。\end{note}賈政因不見賈蘭,便問:“怎麼不見蘭哥?”\begin{note}庚雙夾:看他透出賈政極愛賈蘭。\end{note}地下婆娘忙進裏間問李氏,李氏起身笑著回道:“他說方纔老爺並沒去叫他,他不肯來。”婆娘回覆了賈政。衆人都笑說:“天生的牛心古怪。”賈政忙遣賈環與兩個婆娘將賈蘭喚來。賈母命他在身旁坐了,抓果品與他喫。大家說笑取樂。
\end{parag}


\begin{parag}
    往常間只有寶玉長談闊論,今日賈政在這裏,便惟有唯唯而已。\begin{note}庚雙夾:寫寶玉如此。非世家曾經嚴父之訓者,斷寫不出此一句。\end{note}餘者湘雲雖系閨閣弱女,卻素喜談論,今日賈政在席,也自緘口禁言。\begin{note}庚雙夾:非世家經明訓者,斷不知此一句。寫湘雲如此。\end{note}黛玉本性懶與人共,原不肯多語。\begin{note}庚雙夾:黛玉如此。與人多話則不肯,何得與寶玉話更多哉?\end{note}寶釵原不妄言輕動,便此時亦是坦然自若。\begin{note}庚雙夾:瞧他寫寶釵,真是又曾經嚴父慈母之明訓,又是世府千金,自己又天性從禮合節,前三人之長並歸一身。前三人向有捏作之態,故唯寶釵一人作坦然自若,亦不見逾規越矩也。\end{note}故此一席雖是家常取樂,反見拘束不樂。\begin{note}庚雙夾:非世家公子斷寫不及此。想近時之家,縱其兒女哭笑索飲,長者反以爲樂,其理不法,何如是耶!\end{note}賈母亦知因賈政一人在此所致之故,\begin{note}庚雙夾:這一句又明補出賈母亦是世家明訓之千金也,不然斷想不及此。\end{note}酒過三巡,便攆賈政去歇息。賈政亦知賈母之意,攆了自己去後,好讓他們姊妹兄弟取樂的。賈政忙陪笑道:“今日原聽見老太太這裏大設春燈雅謎,故也備了彩禮酒席,特來入會。何疼孫子孫女之心,便不略賜以兒子半點?”\begin{note}庚雙夾:賈政如此,餘亦淚下。\end{note}賈母笑道:“你在這裏,他們都不敢說笑,沒的倒叫我悶。你要猜謎時,我便說一個你猜,猜不著是要罰的。”賈政忙笑道:“自然要罰。若猜著了,也是要領賞的。”賈母道:“這個自然。”說著便念道:
\end{parag}


\begin{parag}
    猴子身輕站樹梢。\begin{note}庚雙夾:所謂“樹倒猢猻散”是也。\end{note}打一果名。
\end{parag}


\begin{parag}
    賈政已知是荔枝,\begin{note}庚雙夾:的是賈母之謎。\end{note}便故意亂猜別的,罰了許多東西,然後方猜著,也得了賈母的東西。然後也念一個與賈母猜,念道:
\end{parag}


\begin{parag}
    身自端方,體自堅硬。雖不能言,有言必應。\begin{note}庚雙夾:好極!的是賈老之謎,包藏賈府祖宗自身,“必”字隱“筆”字。妙極,妙極!\end{note}打一用物。
\end{parag}


\begin{parag}
    說畢,便悄悄的說與寶玉。寶玉意會,又悄悄的告訴了賈母。賈母想了想,\begin{note}庚側:太君身份。\end{note}果然不差,便說:“是硯臺。”賈政笑道:“到底是老太太,一猜就是。”回頭說:“快把賀彩送上來。”地下婦女答應一聲,大盤小盤一齊捧上。賈母逐件看去,都是燈節下所用所頑新巧之物,甚喜,遂命:“給你老爺斟酒。”寶玉執壺,迎春送酒。賈母因說:“你瞧瞧那屏上,都是他姊妹們做的,再猜一猜我聽。”賈政答應,起身走至屏前,只見頭一個寫道是:
\end{parag}


\begin{poem}
    \begin{pl}能使妖魔膽盡摧,身如束帛氣如雷。\end{pl}

    \begin{pl}一聲震得人方恐,回首相看已化灰。\end{pl}
    \begin{note}庚雙夾:此元春之謎。才得僥倖,奈壽不長,可悲哉!\end{note}
\end{poem}


\begin{parag}
    賈政道:“這是炮竹嗄。”寶玉答道:“是。”賈政又看道:
\end{parag}


\begin{poem}
    \begin{pl}天運人功理不窮,有功無運也難逢。\end{pl}

    \begin{pl}因何鎮日紛紛亂,只爲陰陽數不同。\end{pl}
    \begin{note}庚雙夾:此迎春一生遭際,惜不得其夫何!\end{note}
\end{poem}


\begin{parag}
    賈政道:“是算盤。”迎春笑道:“是。”又往下看是:
\end{parag}


\begin{poem}
    \begin{pl}階下兒童仰面時,清明妝點最堪宜。\end{pl}

    \begin{pl}遊絲一斷渾無力,莫向東風怨別離。\end{pl}
    \begin{note}庚雙夾:此探春遠適之讖也。使此人不遠去,將來事敗,諸子孫不致流散也,悲哉傷哉!\end{note}
\end{poem}


\begin{parag}
    賈政道:“這是風箏。”探春笑道:“是。”又看道是:
\end{parag}


\begin{poem}
    \begin{pl}前身色相總無成,不聽菱歌聽佛經。\end{pl}
    \begin{note}庚眉:此後破失,系再補。\end{note}

    \begin{pl}莫道此生沉黑海,性中自有大光明。\end{pl}
    \begin{note}庚雙夾:此惜春爲尼之讖也。公府千金至緇衣乞食,寧不悲夫! \end{note}
\end{poem}


\begin{parag}
    賈政道:“這是佛前海燈嗄。”惜春笑答道:“是海燈。”
\end{parag}


\begin{note}
    庚本、俄藏本二十二回正文到此爲止,明顯有缺文;以下文字系以戚序本配入並以諸本匯校。
\end{note}


\begin{parag}
    賈政心內沉思道:“娘娘所作爆竹,此乃一響而散之物。迎春所作算盤,是打動亂如麻。探春所作風箏,乃飄飄浮蕩之物。惜春所作海燈,一發清淨孤獨。今乃上元佳節,如何皆作此不祥之物爲戲耶?”心內愈思愈悶,因在賈母之前,不敢形於色,只得仍勉強往下看去。只見後面寫著七言律詩一首,卻是寶釵所作,隨念道:
\end{parag}


\begin{poem}
    \begin{note} 庚暫記寶釵制謎雲:\end{note}

    \begin{pl} 朝罷誰攜兩袖煙,琴邊衾裏總無緣。\end{pl}

    \begin{pl} 曉籌不用雞人報,五夜無煩侍女添。\end{pl}

    \begin{pl} 焦首朝朝還暮暮,煎心日日復年年。\end{pl}

    \begin{pl} 光陰荏苒須當惜,風雨陰晴任變遷。\end{pl}
\end{poem}


\begin{parag}
    \begin{note}庚:此回未成而芹逝矣,嘆嘆!丁亥夏。笏叟。\end{note}
\end{parag}


\begin{parag}
    賈政看完,心內自忖道:“此物還倒有限。只是小小之人作此詞句,更覺不祥,皆非永遠福壽之輩。”想到此處,愈覺煩悶,大有悲慼之狀,因而將適才的精神減去十分之八九,只垂頭沉思。
\end{parag}


\begin{parag}
    賈母見賈政如此光景,想到或是他身體勞乏亦未可定,又兼之恐拘束了衆姊妹不得高興頑耍,即對賈政雲:“你竟不必猜了,去安歇罷。讓我們再坐一會,也好散了。”賈政一聞此言,連忙答應幾個“是”字,又勉強勸了賈母一回酒,方纔退出去了。回至房中只是思索,翻來覆去竟難成寐,不由傷悲感慨,不在話下。
\end{parag}


\begin{parag}
    且說賈母見賈政去了,便道:“你們可自在樂一樂罷。”一言未了,早見寶玉跑至圍屏燈前,指手畫腳,滿口批評,這個這一句不好,那一個破的不恰當,如同開了鎖的猴子一般。寶釵便道:“還象適才坐著,大家說說笑笑,豈不斯文些兒。”鳳姐自裏間忙出來插口道:“你這個人,就該老爺每日令你寸步不離方好。適才我忘了,爲什麼不當著老爺,攛掇叫你也作詩謎兒。若果如此,怕不得這會子正出汗呢。”說的寶玉急了,扯著鳳姐兒,扭股兒糖似的只是廝纏。賈母又與李宮裁併衆姊妹說笑了一會,也覺有些困倦起來。聽了聽已是漏下四鼓,命將食物撤去,賞散與衆人,隨起身道:“我們安歇罷。明日還是節下,該當早起。明日晚間再玩罷。”且聽下回分解。
\end{parag}


\begin{parag}
    \begin{note}蒙回末總評:作者具菩提心,捉筆現身說法,每於言外警人再三再四。而讀者但以小說古詞目之,則大罪過。其先以莊子爲引,及偈曲句作醒悟之語,以警覺世人。猶恐不入,再以燈謎試伸致意,自解自嘆,以不成寐,爲言其用心之切之誠。讀者忍不留心而慢忽之耶?\end{note}
\end{parag}

