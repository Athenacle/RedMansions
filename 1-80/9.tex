\chap{九}{戀風流情友入家塾 起嫌疑頑童鬧學堂}

\begin{parag}
    \begin{note}蒙:君子愛人以道,不能減牽戀之情;小人圖謀以霸,何可逃推頹之辱?幻境幻情,又造出一番小妝新樣。\end{note}
\end{parag}


\begin{parag}
    話說秦業父子專候賈家的人來送上學擇日之信。原來寶玉急於要和秦鐘相遇,\begin{note}蒙雙夾:妙!不知是怎樣相遇。\end{note}卻顧不得別的,遂擇了後日一定上學。“後日一早,請秦相公到我這裏,會齊了,一同前去。”——打發了人送了信。
\end{parag}


\begin{parag}
    至是日一早,寶玉起來時,襲人早已把書筆文物包好,收拾得停停妥妥,坐在牀沿上發悶。\begin{note}蒙側:此等神理,方是此書的正文。蒙雙夾:神理可思,忽又寫小兒學堂中一篇文字,亦別書中未有。\end{note}見寶玉醒來,只得伏待他梳洗。寶玉見他悶悶的,因笑問道:“好姐姐,\begin{note}蒙雙夾:開口斷不可少之三字。\end{note}你怎麼又不自在了?難道怪我上學去丟的你們冷清了不成?”襲人笑道:“這是那裏話。讀書是極好的事,不然就潦倒一輩子,終久怎麼樣呢。但只一件,只是唸書的時節想著書,\begin{note}蒙側:襲人方纔的悶悶,此時的正論,請教諸公,設身處地,亦必是如此方是,真是曲盡情理,一字也不可少者。\end{note}不念的時節想著家些。別和他們一處玩鬧,\begin{note}蒙側:長亭之囑,不過如此。\end{note}碰見老爺不是頑的。雖說是奮志要強,那工課寧可少些,一則貪多嚼不爛,二則身子也要保重。這就是我的意思,你可要體諒。”\begin{note}蒙雙夾:書正語細囑一番。蓋襲卿心中,明知寶玉他並非真心奮志之意,襲人自別有說不出來之語。\end{note}襲人說一句,寶玉答應一句。襲人又道:“大毛衣服我也包好了,交出給小子們去了。學裏冷,好歹想著添換,比不得家裏有人照顧。腳爐手爐的炭也交出去了,你可逼著他們添。那一起懶賊,你不說,他們樂得不動,白凍壞了你。”寶玉道:“你放心,出外頭我自己都會調停的。\begin{note}蒙側:無人體貼,自己扶持。\end{note}你們也別悶死在這屋裏,長和林妹妹一處去頑笑纔好。”說著,俱已穿戴齊備,襲人催他去見賈母、賈政、王夫人等。寶玉且又囑咐了晴雯麝月等幾句,\begin{note}蒙側:這纔是寶玉的本來面目。\end{note}方出來見賈母。賈母也未免有幾句囑咐的話。然後去見王夫人,又出來書房中見賈政。
\end{parag}


\begin{parag}
    偏生這日賈政回家早些,\begin{note}蒙雙夾:若俗筆則又方不在家矣。試想若再不見,則成何文字哉?所謂不敢作安苟且塞責文字。\end{note}正在書房中與相公清客們閒談。忽見寶玉進來請安,回說上學裏去,賈政冷笑道:“你如果再提‘上學’兩個字,連我也羞死了。\begin{note}蒙雙夾:這一句才補出已往許多文字。是嚴父之聲。\end{note}依我的話,你竟頑你的去是正理。仔細站髒了我這地,靠髒了我的門!”\begin{note}蒙雙夾:畫出寶玉的俯首挨壁形象來。\end{note}衆清客相公們都早起身笑道:“老世翁何必又如此。今日世兄一去,三二年就可顯身成名的了,斷不似往年仍作小兒之態了。天也將飯時,世兄竟快請罷。”說著便有兩個年老的攜了寶玉出去。
\end{parag}


\begin{parag}
    賈政因問:“跟寶玉的是誰?”只聽外面答應了兩聲,早進來三四個大漢,打千兒請安。賈政看時,認得是寶玉的奶母之子,名喚李貴。因向他道:“你們成日家跟他上學,他到底唸了些什麼書!倒唸了些流言混話在肚子裏,學了些精緻的淘氣。等我閒一閒,先揭了你的皮,再和那不長進的算賬!”\begin{note}蒙側:此等話似覺無味無理,然而作父母的,到無可如何處,每多用此種法術,所謂百計經營、心力俱瘁者。\end{note}嚇的李貴忙雙膝跪下,摘了帽子,碰頭有聲,連連答應“是”,又回說:“哥兒已經唸到第三本《詩經》,什麼‘呦呦鹿鳴,荷葉浮萍’,小的不敢撒謊。”說的滿座鬨然大笑起來。賈政也掌不住笑了。因說道:“那怕再念三十本《詩經》,也都是掩耳偷鈴,哄人而已。你去請學裏太爺的安,就說我說了:什麼《詩經》古文,一概不用虛應故事,只是先把《四書》一氣講明背熟,是最要緊的。”李貴忙答應“是”,見賈政無話,方退出去。
\end{parag}


\begin{parag}
    此時寶玉獨站在院外屏聲靜候,待他們出來,便忙忙的走了。李貴等一面彈衣服,一面說道:“哥兒可聽見了不曾?可先要揭我們的皮呢!人家的奴才跟主子賺些好體面,我們這等奴才白陪挨打受罵的。從此後也可憐見些纔好。”寶玉笑道:“好哥哥,你別委曲,我明兒請你。”李貴道:“小祖宗,誰敢望你請?只求聽一句半句話就有了。”說著,又至賈母這邊,秦鍾已早來候著了,賈母正和他說話兒呢。\begin{note}蒙雙夾:此處便寫賈母愛秦鍾一如其孫,至後文方不突然。\end{note}於是二人見過,辭了賈母。寶玉忽想起未辭黛玉,\begin{note}蒙雙夾:妙極!何頓挫之至!餘已忘卻,至此心神一暢,一絲不漏。\end{note}因又忙至黛玉房中來作辭。彼時黛玉纔在窗下對鏡理妝,聽寶玉說上學去,因笑道:“好!這一去,可定是要‘蟾宮折桂’去了。\begin{note}蒙側:此寫黛玉,差強人意。《西廂》雙文,能不抱愧!\end{note}我不能送你了。”寶玉道:“好妹妹,等我下學再喫晚飯。和胭脂膏子也等我來再製。”勞叨了半日,方撤身去了。\begin{note}蒙雙夾:如此總一句,更妙!\end{note}黛玉忙又叫住問道:“你怎麼不去辭辭你寶姐姐呢?”寶玉笑而不答。\begin{note}蒙側:黛玉之問,寶玉之笑,兩心一照,何等神工鬼斧之筆。蒙雙夾:必有是語,方是黛玉,此又系黛玉平生之病。\end{note}一徑同秦鐘上學去了。\begin{note}該批:此豈寶玉所樂爲者?然不入家塾則何能有後回“試才”、“結社”文字?作者從不作安逸苟且文字,於此可見。\end{note}\begin{note}該批:此以俗眼讀《石頭記》者,作者之意又豈是俗人所能知。餘謂《石頭記》不得與俗人讀。\end{note}
\end{parag}


\begin{parag}
    原來這賈家義學離此也不甚遠,不過一里之遙,原系始祖所立,恐族中子弟有貧窮不能請師者,即入此中肄業。凡族中有官爵之人,皆供給銀兩,按俸之多寡幫助,爲學中之費。特共舉年高有德之人爲塾掌,專爲訓課子弟。\begin{note}蒙側:創立者之用心,可謂至矣。\end{note}如今寶秦二人來了,一一的都互相拜見過,讀起書來。自此以後,他二人同來同往,同起同坐,愈加親密。又兼賈母愛惜,也時常的留下秦鍾,住上三天五日,與自己的重孫一般疼愛。因見秦鐘不甚寬裕,更又助他些衣履等物。不上一月之工,秦鍾在榮府便熟了。\begin{note}蒙雙夾:交待得清。\end{note}寶玉終是不安分之人,\begin{note}蒙雙夾:寫寶玉總作如此筆。\end{note}\begin{note}靖眉:安分守己,也不是寶玉了。\end{note}竟一味的隨心所欲,因此又發了癖性,又特向秦鍾悄說道:“咱們倆個人一樣的年紀,況又是同窗,以後不必論叔侄,只論弟兄朋友就是了。”\begin{note}蒙側:悄說之時何時?舍尊就卑何心?隨心所欲何癖?相親愛密何情?\end{note}先是秦鐘不肯,當不得寶玉不依,只叫他“兄弟”,或叫他的表字“鯨卿”,秦鍾也只得混著亂叫起來。
\end{parag}


\begin{parag}
    原來這學中雖都是本族人丁與些親戚家的子弟,俗語說的好,“一龍生九種,種種各別。”未免人多了,就有龍蛇混雜,下流人物在內。\begin{note}蒙雙夾:伏一筆。\end{note}自寶、秦二人來了,都生的花朵兒一般的模樣,又見秦鍾靦腆溫柔,未語面先紅,怯怯羞羞,有女兒之風;寶玉又是天生成慣能做小服低,賠身下氣,性情體貼,話語綿纏,\begin{note}蒙雙夾:凡四語十六字,上用“天生成”三字,真正寫盡古今情種人也。\end{note}因此二人更加親厚,也怨不得那起同窗人起了疑,背地裏你言我語,詬誶謠諑,佈滿書房內外。\begin{note}蒙雙夾:伏下文“阿呆爭風”一回。\end{note}
\end{parag}


\begin{parag}
    原來薛蟠自來王夫人處住後,便知有一家學,學中廣有青年子弟,不免偶動了龍陽之興,因此也假來上學讀書,不過是三日打魚,兩日曬網,白送些束脩禮物與賈代儒,卻不曾有一些兒進益,只圖結交些契弟。誰想這學內就有好幾個小學生,圖了薛蟠的銀錢喫穿,被他哄上手的,也不消多記。\begin{note}蒙雙夾:先虛寫幾個淫浪蠢物,以陪下文,方不孤不板。[伏下金榮。]\end{note}更有兩個多情的小學生,\begin{note}蒙雙夾:此處用“多情”二字方妙。\end{note}亦不知是那一房的親眷,亦未考真名姓,\begin{note}蒙雙夾:一併隱其姓名,所謂“具菩提之心,秉刀斧之筆”。\end{note}只因生得嫵媚風流,滿學中都送了他兩個外號,一號“香憐”,一號“玉愛”。誰都有竊慕之意,將不利於孺子之心,\begin{note}蒙雙夾:詼諧得妙,又似李笠翁書中之趣語。\end{note}只是都懼薛蟠的威勢,不敢來沾惹。如今寶、秦二人一來了,見了他兩個,也不免繾綣羨慕,亦因知系薛蟠相知,故未敢輕舉妄動。香、玉二人心中,也一般的留情與寶、秦。因此四人心中雖有情意,只未發跡。每日一入學中,四處各坐,卻八目勾留,或設言托意,或詠桑寓柳,遙以心照,卻外面自爲避人眼目。\begin{note}蒙雙夾:小兒之態活現,掩耳盜鈴者亦然,世人亦復不少。\end{note}不意偏又有幾個滑賊看出形景來,都背後擠眉弄眼,或咳嗽揚聲,\begin{note}蒙側:才子輩偏無不解之事。\end{note}\begin{note}蒙雙夾:又畫出歷來學中一羣頑皮來。\end{note}這也非此一日。
\end{parag}


\begin{parag}
    可巧這日代儒有事,早已回家去了,又留下一句七言對聯,命學生對了,明日再來上書;將學中之事,又命賈瑞\begin{note}蒙雙夾:又出一賈瑞。\end{note}暫且管理。妙在薛蟠如今不大來學中應卯了,因此秦鍾趁此和香憐擠眉弄眼,遞暗號兒,二人假裝出小恭,走至後院說體己話。秦鍾先問他:“家裏的大人可管你交朋友不管?”\begin{note}蒙雙夾:妙問,真真活跳出兩個小兒來。\end{note}一語未了,只聽背後咳嗽了一聲。\begin{note}蒙雙夾:太急了些,該再聽他二人如何結局,正所謂小兒之態也,酷肖之至。\end{note}二人唬的忙回頭看時,原來是窗友名金榮\begin{note}蒙雙夾:妙名,蓋雲有金自榮,廉恥何益哉?\end{note}者。香憐本有些性急,羞怒相激,問他道:“你咳嗽什麼?難道不許我兩個說話不成?”金榮笑道:“許你們說話,難道不許我咳嗽不成?我只問你們:有話不明說,許你們這樣鬼鬼祟祟的幹什麼故事?我可也拿住了,還賴什麼!先得讓我抽個頭兒,咱們一聲兒不言語,不然大家就奮起來。”秦、香二人急得飛紅的臉,便問道:“你拿住什麼了?”金榮笑道:“我現拿住了是真的。”說著,又拍著手笑嚷道:“貼的好燒餅!你們都不買一個喫去?”秦鍾香憐二人又氣又急,忙進來向賈瑞前告金榮,說金榮無故欺負他兩個。
\end{parag}


\begin{parag}
    原來這賈瑞最是個圖便宜沒行止的人,每在學中以公報私,勒索子弟們請他;\begin{note}蒙側:學中亦自有此輩,可爲痛哭。\end{note}後又附助著薛蟠,圖些銀錢酒肉,一任薛蟠橫行霸道,他不但不去管約,反助紂爲虐討好兒。偏那薛蟠本是浮萍心性,今日愛東,明日愛西,近來又有了新朋友,把香、玉二人丟開一邊。就連金榮亦是當日的好朋友,自有了香、玉二人,便棄了金榮。近日連香、玉亦已見棄。故賈瑞也無了提攜幫襯之人,不說薛蟠得新棄舊,只怨香、玉二人不在薛蟠前提攜幫補他,\begin{note}蒙雙夾:無恥小人,真有此心。\end{note}\begin{note}該批:前有幻境遇可卿,今又出學中小兒淫浪之態,後文更放筆寫賈瑞正照。看書人細心體貼,方許你看。\end{note}因此賈瑞金榮等一干人,也正在醋妒他兩個。今兒見秦、香二人來告金榮,賈瑞心中便不自在起來,不好呵叱秦鍾,卻拿著香憐作法,反說他多事,著實搶白了幾句。香憐反討了沒趣,連秦鍾也訕訕的各歸坐位去了。金榮越發得了意,搖頭咂嘴的,口內還說許多閒話,玉愛偏又聽了不忿,兩個人隔座咕咕唧唧的角起口來。金榮只一口咬定說:“方纔明明的撞見他兩個在後院子裏親嘴摸屁股,一對一肏,撅草棍兒抽長短,\begin{note}蒙側:“怎麼長短”四字,何等韻雅,何等渾含!俚語得文人提來,便覺有金玉爲聲之象。\end{note}誰長誰先幹。”金榮只顧得意亂說,卻不防還有別人。誰知早又觸怒了一個。你道這個是誰?
\end{parag}


\begin{parag}
    原來這一個名喚賈薔,\begin{note}蒙雙夾:新而絕,得空便入。\end{note}亦系寧府中之正派玄孫,父母早亡,從小兒跟賈珍過活,如今長了十六歲,比賈蓉生的還風流俊俏。他兄弟二人最相親厚,常相共處。寧府人多口雜,那些不得志的奴僕們,專能造言誹謗主人,因此不知又有了什麼小人詬誶謠諑之辭。賈珍想亦風聞得些口聲不大好,自己也要避些嫌疑,如今竟分與房舍,命賈薔搬出寧府,自去立門戶過活去了。\begin{note}蒙側:此等嫌疑不敢認真搜查,悄爲分計,皆以含而不漏爲文,真實靈活至極之筆。\end{note}這賈薔外相既美,\begin{note}蒙雙夾:亦不免招謗,難怪小人之口。\end{note}內性又聰明,雖然應名來上學,亦不過虛掩眼目而已。仍是鬥雞走狗,賞花玩柳。總恃上有賈珍溺愛,\begin{note}蒙雙夾:貶賈珍最重。\end{note}下有賈蓉匡助,\begin{note}蒙雙夾:貶賈蓉次之。\end{note}因此族中人誰敢來觸逆於他。他既和賈蓉最好,今見有人欺負秦鍾,如何肯依?如今自己要挺身出來報不平,心中卻忖度一番,\begin{note}蒙雙夾:這一忖度,方是聰明人之心機,寫的最好看,最細緻。\end{note}想道:“金榮賈瑞一干人,都是薛大叔的相知,向日我又與薛大叔相好,倘或我一出頭,他們告訴了老薛,\begin{note}蒙雙夾:先曰“薛大叔”,此曰“老薛”,寫盡嬌侈紈絝。\end{note}我們豈不傷和氣?待要不管,如此謠言,說的大家沒趣。如今何不用計制服,又止息了口聲,又不傷了臉面。”想畢,也裝出小恭,走至外面,悄悄的把跟寶玉的書童名喚茗煙\begin{note}蒙雙夾:又出一茗煙。\end{note}者喚到身邊,如此這般調撥他幾句。\begin{note}蒙雙夾:如此便好,不必細述。\end{note}
\end{parag}


\begin{parag}
    這茗煙乃是寶玉第一個得用的,且又年輕不暗世事,如今聽賈薔說金榮如此欺負秦鍾,連他爺寶玉都干連在內,不給他個利害,下次越發狂縱難制了。這茗煙無故就要欺壓人的,如今得了這個信,又有賈薔助著,便一頭進來找金榮,也不叫金相公了,只說:“姓金的,你是什麼東西!”賈薔遂跺一跺靴子,故意整整衣服,看看日影兒說:“是時候了。”遂先向賈瑞說有事要早一步。賈瑞不敢強他,只得隨他去了。這裏茗煙先一把揪住金榮,\begin{note}蒙側:豪奴輩,雖系主人親故亦隨便欺慢,即有一二不服氣者,而豪家多是偏護家人。理之所無,而事之盡有,不知是何心思,是非凡常可能測略。\end{note}問道:“我們肏屁股不肏屁股,管你相干?橫豎沒肏你爹去罷了!你是好小子,出來動一動你茗大爺!”嚇的滿屋中子弟都怔怔的癡望。賈瑞忙吆喝:“茗煙不得撒野!”金榮氣黃了臉,說:“反了!奴才小子都敢如此,我和你主子說。”便奪手要去抓打寶玉秦鍾。\begin{note}蒙雙夾:好看之極!\end{note}尚未去時,從得腦後“颼”的一聲,早見一方硯瓦飛來,\begin{note}蒙雙夾:好看好笑之極!\end{note}並不知系何人打來的,幸未打著,卻又打了旁人的座上,這座上乃是賈藍賈菌。
\end{parag}


\begin{parag}
    賈菌亦系榮府近派的重孫,\begin{note}蒙雙夾:先寫一寧派,又寫一榮派,互相錯綜得妙。\end{note}其母亦少寡,獨守著賈菌,這賈菌與賈藍最好,所以二人同桌而坐。誰知賈菌年紀雖小,志氣最大,極是淘氣不怕人的。\begin{note}蒙雙夾:要知沒志氣小兒,必不會淘氣。\end{note}他在座上冷眼看見金榮的朋友暗助金榮,飛硯來打茗煙,偏沒打著茗煙,便落在他座上,正打在面前,將一個磁硯水壺打了個粉碎,濺了一書黑水。\begin{note}蒙雙夾:這等忙,有此閒處用筆。\end{note}賈菌如何依得,便罵:“好囚攮的們,這不都動了手了麼!”\begin{note}蒙雙夾:好聽煞。靖眉:聲口如聞。\end{note}罵著,也抓起硯磚來要打回去。\begin{note}蒙雙夾:先瓦硯,次磚硯,轉換得妙極。\end{note}賈藍是個省事的,忙按住硯,極口勸道:“好兄弟,不與咱們相干。”\begin{note}蒙雙夾:是賈蘭口氣。\end{note}賈菌如何忍得住,便兩手抱起書匣子來,照那邊掄了去。\begin{note}蒙雙夾:先“飛”後“掄”,用字得神,好看之極!\end{note}終是身小力薄,卻掄不到那裏,剛到寶玉秦鍾桌案上就落了下來,只聽“譁啷啷”一聲,砸在桌上,書本紙片等至於筆硯之物撒了一桌,又把寶玉的一碗茶也砸得碗碎茶流。\begin{note}蒙雙夾:好看之極!不打著別個,偏打著二人,亦想不到文章也。此書此等筆法,與後文踢著襲人、誤打平兒,是一樣章法。\end{note}賈菌便跳出來,要揪打那一個飛硯的。金榮此時隨手抓了一根毛竹大板在手,地狹人多,那裏經得舞動長板。茗煙早吃了一下,亂嚷:“你們還不來動手!”寶玉還有三個小廝:一名鋤藥,一名掃紅,一名墨雨。這三個豈有不淘氣的,一齊亂嚷:“小婦養的!動了兵器了!”\begin{note}蒙雙夾:好聽之極,好看之極!\end{note}墨雨遂掇起一根門閂,掃紅鋤藥手中都是馬鞭子,蜂擁而上。賈瑞急攔一回這個,勸一回那個,誰聽他的話,肆行大鬧。衆頑童也有趁勢幫著打太平拳助樂的,也有膽小藏在一邊的,也有直立在桌上拍著手兒亂笑、喝著聲兒叫打的,登時間鼎沸起來。\begin{note}蒙側:燕青打擂臺,也不過如此。\end{note}
\end{parag}


\begin{parag}
    外邊李貴等幾個大僕人聽見裏邊作反起來,忙都進來一齊喝住。問是何原故。衆聲不一,這一個如此說,那一個又如彼說。\begin{note}蒙雙夾:妙!如聞其聲。\end{note}李貴且喝罵了茗煙四個一頓,攆了出去。\begin{note}蒙雙夾:處治得好。\end{note}秦鐘的頭早撞在金榮的板上,打去一層油皮,寶玉正拿褂襟子替他揉呢,見喝住了衆人,便命:“李貴,收書!拉馬來,我回去回太爺去!我們被人欺負了,不敢說別的,守禮來告訴瑞大爺,瑞大爺反倒派我們不是,聽人家罵我們,還調唆他們打我們茗煙,連秦鐘的頭也打破,這還在這裏念什麼書!茗煙他也是爲有人欺侮我的。不如散了罷。”李貴勸道:“哥兒不要性急。太爺既有事回家去了,這會子爲這點子事去聒噪他老人家,倒顯的咱們沒理。依我的主意,那裏的事那裏了結好,何必去驚動他老人家。這都是瑞大爺的不是,太爺不在這裏,你老人家就是這學裏的頭腦了,衆人看你著行事。\begin{note}蒙側:勸的心思,有個太爺得知,未必然之。故巧爲輾轉以結其局,而不失其體。\end{note}衆人有了不是,該打的打,該罰的罰,如何等鬧到這步田地不管?”賈瑞道:“我吆喝著都不聽。”\begin{note}蒙雙夾:如聞。\end{note}李貴笑道:“不怕你老人家惱我,素日你老人家到底有些不正經,所以這些兄弟纔不聽。就鬧到太爺跟前去,連你老人家也脫不過的。還不快作主意撕羅開了罷。”寶玉道:“撕羅什麼?我必是回去的!”秦鍾哭道:“有金榮,我是不在這裏唸書的。”寶玉道:“這是爲什麼?難道有人家來得的,咱們倒來不得?我必回明白衆人,攆了金榮去。”又問李貴:“金榮是那一房的親戚?”李貴想了一想:“也不用問了。若說起那一房的親戚,更傷了弟兄們的和氣了。”
\end{parag}


\begin{parag}
    茗煙在窗外道:“他是東胡同裏璜大奶奶的侄兒,那是什麼硬正仗腰子的,也來唬我們。璜大奶奶是他姑娘。你那姑媽只會打旋磨子,給我們璉二奶奶跪著借當頭。\begin{note}蒙側:可憐!開口告人,終身是玷。\end{note}我眼裏就看不起他那樣的主子奶奶!”李貴忙斷喝不止,說:“偏你這小狗肏的知道,有這些蛆嚼!”寶玉冷笑道:“我只當是誰的親戚,原來是璜嫂子的侄兒,我就去問問他來!”說著便要走,叫茗煙進來包書。茗煙包著書,又得意道:“爺也不用自己去見,等我去到他家,就說老太太有說的話問他呢,僱上一輛車拉進去,當著老太太問他,豈不省事?”\begin{note}蒙雙夾:又以賈母欺壓,更妙!\end{note}李貴忙喝道:“你要死!仔細回去我好不好先捶了你,然後再回老爺太太,就說寶玉全是你調唆的。我這裏好容易勸哄的好了一半了,你又來生個新法子。你鬧了學堂,不說變法兒壓息了纔是,倒要往大里鬧!”茗煙方不敢作聲兒了。
\end{parag}


\begin{parag}
    此時賈瑞也怕鬧大了,自己也不乾淨,只得委曲著來央告秦鍾,又央告寶玉。先是他二人不肯。後來寶玉說:“不回去也罷了,只叫金榮賠不是便罷。”金榮先是不肯,後來禁不得賈瑞也來逼他去賠不是,李貴等只得好勸金榮說:“原來是你起的端,你不這樣,怎得了局?”金榮強不得,只得與秦鍾作了揖。寶玉還不依,偏定要磕頭。賈瑞只要暫息此事,又悄悄的勸金榮說:“俗語說的好,‘殺人不過頭點地’。你既惹出事來,少不得下點氣兒,磕個頭就完事了。”金榮無奈,只得進前來與秦鍾磕頭。且聽下回分解。
\end{parag}


\begin{parag}
    \begin{note}蒙:此篇寫賈氏學中,非親即族,且學乃大衆之規範,人倫之根本。首先悖亂,以至於此極,其賈家之氣數,即此可知。挾用襲人之風流,羣小之惡逆,一揚一抑,作者自必有所取。\end{note}
\end{parag}


\begin{parag}
    \begin{note}夢:正是:\end{note}
\end{parag}


\begin{parag}
    \begin{note}忍得一時忿,終身無惱悶。\end{note}
\end{parag}

