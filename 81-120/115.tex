\chap{一百一十五}{惑偏私惜春矢素志 证同类宝玉失相知}



\begin{parag}
    话说宝玉为自己失言被宝钗问住,想要掩饰过去,只见秋纹进来说:“外头老爷叫二爷呢。”宝玉巴不得一声,便走了。去到贾政那里,贾政道:“我叫你来不为别的,现在你穿着孝,不便到学里去,你在家里,必要将你念过的文章温习温习。我这几天倒也闲着,隔两三日要做几篇文章我瞧瞧,看你这些时进益了没有。”宝玉只得答应着。贾政又道:“你环兄弟兰侄儿我也叫他们温习去了。倘若你作的文章不好,反倒不及他们,那可就不成事了。”宝玉不敢言语,答应了个“是”,站着不动。贾政道:“去罢。”宝玉退了出来,正撞见赖大诸人拿着些册子进来。
\end{parag}


\begin{parag}
    宝玉一溜烟回到自己房中,宝钗问了知道叫他作文章,倒也喜欢,惟有宝玉不愿意,也不敢怠慢。正要坐下静静心,见有两个姑子进来,宝玉看是地藏庵的,来和宝钗说:“请二奶奶安。”宝钗待理不理的说:“你们好?”因叫人来:“倒茶给师父们喝。”宝玉原要和那姑子说话,见宝钗似乎厌恶这些,也不好兜搭。那姑子知道宝钗是个冷人,也不久坐,辞了要去。宝钗道:“再坐坐去罢。”那姑子道:“我们因在铁槛寺做了功德,好些时没来请太太奶奶们的安,今日来了,见过了奶奶太太们,还要看四姑娘呢。”宝钗点头,由他去了。
\end{parag}


\begin{parag}
    那姑子便到惜春那里,见了彩屏,说:“姑娘在那里呢?”彩屏道:“不用提了。姑娘这几天饭都没吃,只是歪着。”那姑子道:“为什么?”彩屏道:“说也话长。你见了姑娘只怕他便和你说了。”惜春早已听见,急忙坐起来说:“你们两个人好啊?见我们家事差了,便不来了。”那姑子道:“阿弥陀佛!有也是施主,没也是施主,别说我们是本家庵里的,受过老太太多少恩惠呢。如今老太太的事,太太奶奶们都见了,只没有见姑娘,心里惦记,今儿是特特的来瞧姑娘来的。”惜春便问起水月庵的姑子来,那姑子道:“他们庵里闹了些事,如今门上也不肯常放进来了。”便问惜春道:“前儿听见说栊翠庵的妙师父怎么跟了人去了?”惜春道:“那里的话!说这个话的人堤防着割舌头。人家遭了强盗抢去,怎么还说这样的坏话。”那姑子道:“妙师父的为人怪僻,只怕是假惺惺罢。在姑娘面前我们也不好说的。那里象我们这些粗夯人,只知道讽经念佛,给人家忏悔,也为着自己修个善果。”惜春道:“怎么样就是善果呢?”那姑子道:“除了咱们家这样善德人家儿不怕,若是别人家,那些诰命夫人小姐也保不住一辈子的荣华。到了苦难来了,可就救不得了。只有个观世音菩萨大慈大悲,遇见人家有苦难的就慈心发动,设法儿救济。为什么如今都说大慈大悲救苦救难的观世音菩萨呢。我们修了行的人,虽说比夫人小姐们苦多着呢,只是没有险难的了。虽不能成佛作祖,修修来世或者转个男身,自己也就好了。不象如今脱生了个女人胎子,什么委屈烦难都说不出来。姑娘你还不知道呢,要是人家姑娘们出了门子,这一辈子跟着人是更没法儿的。若说修行,也只要修得真。那妙师父自为才情比我们强,他就嫌我们这些人俗,岂知俗的才能得善缘呢。他如今到底是遭了大劫了。”惜春被那姑子一番话说得合在机上,也顾不得丫头们在这里,便将尤氏待他怎样,前儿看家的事说了一遍。并将头发指给他瞧道:“你打谅我是什么没主意恋火坑的人么?早有这样的心,只是想不出道儿来。”那姑子听了,假作惊慌道:“姑娘再别说这个话!珍大奶奶听见还要骂杀我们,撵出庵去呢!姑娘这样人品,这样人家,将来配个好姑爷,享一辈子的荣华富贵。”惜春不等说完,便红了脸说:“珍大奶奶撵得你,我就撵不得么?”那姑子知是真心,便索性激他一激,说道:“姑娘别怪我们说错了话,太太奶奶们那里就依得姑娘的性子呢?那时闹出没意思来倒不好。我们倒是为姑娘的话。”惜春道:“这也瞧罢咧。”彩屏等听这话头不好,便使个眼色儿给姑子叫他去。那姑子会意,本来心里也害怕,不敢挑逗,便告辞出去。惜春也不留他,便冷笑道:“打谅天下就是你们一个地藏庵么!”那姑子也不敢答言去了。
\end{parag}


\begin{parag}
    彩屏见事不妥,恐担不是,悄悄的去告诉了尤氏说:“四姑娘绞头发的念头还没有息呢。他这几天不是病,竟是怨命。奶奶堤防些,别闹出事来,那会子归罪我们身上。”尤氏道:“他那里是为要出家,他为的是大爷不在家,安心和我过不去,也只好由他罢了。”彩屏等没法,也只好常常劝解。岂知惜春一天一天的不吃饭,只想绞头发。彩屏等吃不住,只得到各处告诉。邢王二夫人等也都劝了好几次,怎奈惜春执迷不解。
\end{parag}


\begin{parag}
    邢王二夫人正要告诉贾政,只听外头传进来说:“甄家的太太带了他们家的宝玉来了。”众人急忙接出,便在王夫人处坐下。众人行礼,叙些温寒,不必细述。只言王夫人提起甄宝玉与自己的宝玉无二,要请甄宝玉一见。传话出去,回来说道:“甄少爷在外书房同老爷说话,说的投了机了,打发人来请我们二爷三爷,还叫兰哥儿,在外头吃饭。吃了饭进来。”说毕,里头也便摆饭。不题。
\end{parag}


\begin{parag}
    且说贾政见甄宝玉相貌果与宝玉一样,试探他的文才,竟应对如流,甚是心敬,故叫宝玉等三人出来警励他们。再者倒底叫宝玉来比一比。宝玉听命,穿了素服,带了兄弟侄儿出来,见了甄宝玉,竟是旧相识一般。那甄宝玉也象那里见过的,两人行了礼,然后贾环贾兰相见。本来贾政席地而坐,要让甄宝玉在椅子上坐。甄宝玉因是晚辈,不敢上坐,就在地下铺了褥子坐下。如今宝玉等出来,又不能同贾政一处坐着,为甄宝玉又是晚一辈,又不好叫宝玉等站着。贾政知是不便,站着又说了几句话,叫人摆饭,说:“我失陪,叫小儿辈陪着,大家说说话儿,好叫他们领领大教。”甄宝玉逊谢道:“老伯大人请便。侄儿正欲领世兄们的教呢。”贾政回复了几句,便自往内书房去。那甄宝玉反要送出来,贾政拦住。宝玉等先抢了一步出了书房门坎,站立着看贾政进去,然后进来让甄宝玉坐下。彼此套叙了一回,诸如久慕竭想的话,也不必细述。
\end{parag}


\begin{parag}
    且说贾宝玉见了甄宝玉,想到梦中之景,并且素知甄宝玉为人必是和他同心,以为得了知己。因初次见面,不便造次。且又贾环贾兰在坐,只有极力夸赞说:“久仰芳名,无由亲炙。今日见面,真是谪仙一流的人物。”那甄宝玉素来也知贾宝玉的为人,今日一见,果然不差,”只是可与我共学,不可与你适道,他既和我同名同貌,也是三生石上的旧精魂了。既我略知了些道理,怎么不和他讲讲。但是初见,尚不知他的心与我同不同,只好缓缓的来。”便道:“世兄的才名,弟所素知的,在世兄是数万人的里头选出来最清最雅的,在弟是庸庸碌碌一等愚人,忝附同名,殊觉玷辱了这两个字。”贾宝玉听了,心想:“这个人果然同我的心一样的。但是你我都是男人,不比那女孩儿们清洁,怎么他拿我当作女孩儿看待起来?”便道:“世兄谬赞,实不敢当。弟是至浊至愚,只不过一块顽石耳,何敢比世兄品望高清,实称此两字。”甄宝玉道:“弟少时不知分量,自谓尚可琢磨。岂知家遭消索,数年来更比瓦砾犹残,虽不敢说历尽甘苦,然世道人情略略的领悟了好些。世兄是锦衣玉食,无不遂心的,必是文章经济高出人上,所以老伯钟爱,将为席上之珍。弟所以才说尊名方称。”贾宝玉听这话头又近了碌蠹的旧套,想话回答。贾环见未与他说话,心中早不自在。倒是贾兰听了这话甚觉合意,便说道:“世叔所言固是太谦,若论到文章经济,实在从历练中出来的,方为真才实学。在小侄年幼,虽不知文章为何物,然将读过的细味起来,那膏粱文绣比着令闻广誉,真是不啻百倍的了。”甄宝玉未及答言,贾宝玉听了兰儿的话心里越发不合,想道:“这孩子从几时也学了这一派酸论。”便说道:“弟闻得世兄也诋尽流俗,性情中另有一番见解。今日弟幸会芝范,想欲领教一番超凡入圣的道理,从此可以净洗俗肠,重开眼界,不意视弟为蠢物,所以将世路的话来酬应。”甄宝玉听说,心里晓得”他知我少年的性情,所以疑我为假。我索性把话说明,或者与我作个知心朋友也是好的。”便说道:“世兄高论,固是真切。但弟少时也曾深恶那些旧套陈言,只是一年长似一年,家君致仕在家,懒于酬应,委弟接待。后来见过那些大人先生尽都是显亲扬名的人,便是著书立说,无非言忠言孝,自有一番立德立言的事业,方不枉生在圣明之时,也不致负了父亲师长养育教诲之恩,所以把少时那一派迂想痴情渐渐的淘汰了些。如今尚欲访师觅友,教导愚蒙,幸会世兄,定当有以教我。适才所言,并非虚意。”贾宝玉愈听愈不耐烦,又不好冷淡,只得将言语支吾。幸喜里头传出话来说:“若是外头爷们吃了饭,请甄少爷里头去坐呢。”宝玉听了,趁势便邀甄宝玉进去。
\end{parag}


\begin{parag}
    那甄宝玉依命前行,贾宝玉等陪着来见王夫人。贾宝玉见是甄太太上坐,便先请过了安,贾环贾兰也见了。甄宝玉也请了王夫人的安。两母两子互相厮认。虽是贾宝玉是娶过亲的,那甄夫人年纪已老,又是老亲,因见贾宝玉的相貌身材与他儿子一般,不禁亲热起来。王夫人更不用说,拉着甄宝玉问长问短,觉得比自己家的宝玉老成些。回看贾兰,也是清秀超群的,虽不能象两个宝玉的形像,也还随得上。只有贾环粗夯,未免有偏爱之色。众人一见两个宝玉在这里,都来瞧看,说道:“真真奇事,名字同了也罢,怎么相貌身材都是一样的。亏得是我们宝玉穿孝,若是一样的衣服穿着,一时也认不出来。”内中紫鹃一时痴意发作,便想起黛玉来,心里说道:“可惜林姑娘死了,若不死时,就将那甄宝玉配了他,只怕也是愿意的。”正想着,只听得甄夫人道:“前日听得我们老爷回来说,我们宝玉年纪也大了,求这里老爷留心一门亲事。”王夫人正爱甄宝玉,顺口便说道:“我也想要与令郎作伐。我家有四个姑娘,那三个都不用说,死的死,嫁的嫁了,还有我们珍大侄儿的妹子,只是年纪过小几岁,恐怕难配。倒是我们大媳妇的两个堂妹子生得人才齐整,二姑娘呢,已经许了人家,三姑娘正好与令郎为配。过一天我给令郎作媒,但是他家的家计如今差些。”甄夫人道:“太太这话又客套了。如今我们家还有什么,只怕人家嫌我们穷罢了。”王夫人道:“现今府上复又出了差,将来不但复旧,必是比先前更要鼎盛起来。”甄夫人笑着道:“但愿依着太太的话更好。这么着就求太太作个保山。”甄宝玉听他们说起亲事,便告辞出来。贾宝玉等只得陪着来到书房,见贾政已在那里,复又立谈几句。听见甄家的人来回甄宝玉道:“太太要走了,请爷回去罢。”于是甄宝玉告辞出来。贾政命宝玉环兰相送。不题。
\end{parag}


\begin{parag}
    且说宝玉自那日见了甄宝玉之父,知道甄宝玉来京,朝夕盼望。今儿见面原想得一知己,岂知谈了半天,竟有些冰炭不投。闷闷的回到自己房中,也不言,也不笑,只管发怔。宝钗便问:“那甄宝玉果然象你么?”宝玉道:“相貌倒还是一样的。只是言谈间看起来并不知道什么,不过也是个禄蠹。”宝钗道:“你又编派人家了。怎么就见得也是个禄蠹呢?”宝玉道:“他说了半天,并没个明心见性之谈,不过说些什么文章经济,又说什么为忠为孝,这样人可不是个禄蠹么!只可惜他也生了这样一个相貌。我想来,有了他,我竟要连我这个相貌都不要了。”宝钗见他又发呆话,便说道:“你真真说出句话来叫人发笑,这相貌怎么能不要呢。况且人家这话是正理,做了一个男人原该要立身扬名的,谁象你一味的柔情私意。不说自己没有刚烈,倒说人家是禄蠹。”宝玉本听了甄宝玉的话甚不耐烦,又被宝钗抢白了一场,心中更加不乐,闷闷昏昏,不觉将旧病又勾起来了,并不言语,只是傻笑。宝钗不知,只道是“我的话错了,他所以冷笑”,也不理他。岂知那日便有些发呆,袭人等怄他也不言语。过了一夜,次日起来只是发呆,竟有前番病的样子。
\end{parag}


\begin{parag}
    一日,王夫人因为惜春定要绞发出家,尤氏不能拦阻,看着惜春的样子是若不依他必要自尽的,虽然昼夜着人看着,终非常事,便告诉了贾政。贾政叹气跺脚,只说:“东府里不知干了什么,闹到如此地位。”叫了贾蓉来说了一顿,叫他去和他母亲说,认真劝解劝解。”若是必要这样,就不是我们家的姑娘了。”岂知尤氏不劝还好,一劝了更要寻死,说:“做了女孩儿终不能在家一辈子的,若象二姐姐一样,老爷太太们倒要烦心,况且死了。如今譬如我死了似的,放我出了家,干干净净的一辈子,就是疼我了。况且我又不出门,就是栊翠庵,原是咱们家的基趾,我就在那里修行。我有什么,你们也照应得着。现在妙玉的当家的在那里。你们依我呢,我就算得了命了;若不依我呢,我也没法,只有死就完了。我如若遂了自己的心愿,那时哥哥回来我和他说,并不是你们逼着我的。若说我死了,未免哥哥回来倒说你们不容我。”尤氏本与惜春不合,听他的话也似乎有理,只得去回王夫人。
\end{parag}


\begin{parag}
    王夫人已到宝钗那里,见宝玉神魂失所,心下着忙,便说袭人道:“你们忒不留神,二爷犯了病也不来回我。”袭人道:“二爷的病原来是常有的,一时好,一时不好。天天到太太那里仍旧请安去,原是好好儿的,今儿才发糊涂些。二奶奶正要来回太太,恐防太太说我们大惊小怪。”宝玉听见王夫人说他们,心里一时明白,恐他们受委屈,便说道:“太太放心,我没什么病,只是心里觉着有些闷闷的。”王夫人道:“你是有这病根子,早说了好请大夫瞧瞧,吃两剂药好了不好!若再闹到头里丢了玉的时候似的,就费事了。”宝玉道:“太太不放心便叫个人来瞧瞧,我就吃药。”王夫人便叫丫头传话出来请大夫。这一个心思都在宝玉身上,便将惜春的事忘了。迟了一回,大夫看了,服药。王夫人回去。
\end{parag}


\begin{parag}
    过了几天,宝玉更糊涂了,甚至于饭食不进,大家着急起来。恰又忙着脱孝,家中无人,又叫了贾芸来照应大夫。贾琏家下无人,请了王仁来在外帮着料理。那巧姐儿是日夜哭母,也是病了。所以荣府中又闹得马仰人翻。
\end{parag}


\begin{parag}
    一日又当脱孝来家,王夫人亲身又看宝玉,见宝玉人事不醒,急得众人手足无措。一面哭着,一面告诉贾政说:“大夫回了,不肯下药,只好预备后事。”贾政叹气连连,只得亲自看视,见其光景果然不好,便又叫贾琏办去。贾琏不敢违拗,只得叫人料理。手头又短,正在为难,只见一个人跑进来说:“二爷,不好了,又有饥荒来了。”贾琏不知何事,这一唬非同小可,瞪着眼说道:“什么事?”那小厮道:“门上来了一个和尚,手里拿着二爷的这块丢的玉,说要一万赏银。”贾琏照脸啐道:“我打量什么事,这样慌张。前番那假的你不知道么!就是真的,现在人要死了,要这玉做什么!”小厮道:“奴才也说了,那和尚说给他银子就好了。”又听着外头嚷进来说:“这和尚撒野,各自跑进来了,众人拦他拦不住。”贾琏道:“那里有这样怪事,你们还不快打出去呢。”正闹着,贾政听见了,也没了主意了。里头又哭出来说:“宝二爷不好了!”贾政益发着急。只见那和尚嚷道:“要命拿银子来!”贾政忽然想起,头里宝玉的病是和尚治好的,这会子和尚来,或者有救星。但是这玉倘或是真,他要起银子来怎么样呢?想了一想,姑且不管他,果真人好了再说。
\end{parag}


\begin{parag}
    贾政叫人去请,那和尚已进来了,也不施礼,也不答话,便往里就跑。贾琏拉着道:“里头都是内眷,你这野东西混跑什么!”那和尚道:“迟了就不能救了。”贾琏急得一面走一面乱嚷道:“里头的人不要哭了,和尚进来了。”王夫人等只顾着哭,那里理会。贾琏走近来又嚷,王夫人等回过头来,见一个长大的和尚,唬了一跳,躲避不及。那和尚直走到宝玉炕前,宝钗避过一边,袭人见王夫人站着,不敢走开。只见那和尚道:“施主们,我是送玉来的。”说着,把那块玉擎着道:“快把银子拿出来,我好救他。”王夫人等惊惶无措,也不择真假,便说道:“若是救活了人,银子是有的。”那和尚笑道:“拿来。”王夫人道:“你放心,横竖折变的出来。”和尚哈哈大笑,手拿着玉在宝玉耳边叫道:“宝玉,宝玉,你的宝玉回来了。”说了这一句,王夫人等见宝玉把眼一睁。袭人说道:“好了。”只见宝玉便问道:“在那里呢?”那和尚把玉递给他手里。宝玉先前紧紧的攥着,后来慢慢的得过手来,放在自己眼前细细的一看说:“嗳呀,久违了!”里外众人都喜欢的念佛,连宝钗也顾不得有和尚了。贾琏也走过来一看,果见宝玉回过来了,心里一喜,疾忙躲出去了。
\end{parag}


\begin{parag}
    那和尚也不言语,赶来拉着贾琏就跑。贾琏只得跟着到了前头,赶着告诉贾政。贾政听了喜欢,即找和尚施礼叩谢。和尚还了礼坐下。贾琏心下狐疑:“必是要了银子才走。”贾政细看那和尚,又非前次见的,便问:“宝剎何方?法师大号?这玉是那里得的?怎么小儿一见便会活过来呢?”那和尚微微笑道:“我也不知道,只要拿一万银子来就完了。”贾政见这和尚粗鲁,也不敢得罪,便说:“有。”和尚道:“有便快拿来罢,我要走了。”贾政道:“略请少坐,待我进内瞧瞧。”和尚道:“你去快出来才好。”
\end{parag}


\begin{parag}
    贾政果然进去,也不及告诉便走到宝玉炕前。宝玉见是父亲来,欲要爬起,因身子虚弱起不来。王夫人按着说道:“不要动。”宝玉笑着拿这玉给贾政瞧道:“宝玉来了。”贾政略略一看,知道此事有些根源,也不细看,便和王夫人道:“宝玉好过来了。这赏银怎么样?”王夫人道:“尽着我所有的折变了给他就是了。”宝玉道:“只怕这和尚不是要银子的罢。”贾政点头道:“我也看来古怪,但是他口口声声的要银子。”王夫人道:“老爷出去先款留着他再说。”贾政出来,宝玉便嚷饿了,喝了一碗粥,还说要饭。婆子们果然取了饭来,王夫人还不敢给他吃。宝玉说:“不妨的,我已经好了。”便爬着吃了一碗,渐渐的神气果然好过来了,便要坐起来。麝月上去轻轻的扶起,因心里喜欢,忘了情说道:“真是宝贝,才看见了一会儿就好了。亏的当初没有砸破。”宝玉听了这话,神色一变,把玉一撂,身子往后一仰。未知死活,下回分解。
\end{parag}