\chap{六十二}{憨湘雲醉眠芍藥裀 呆香菱情解石榴裙}


\begin{parag}
    話說平兒出來吩咐林之孝家的道:“大事化爲小事,小事化爲沒事,方是興旺之家。若得不了一點子小事,便揚鈴打鼓的亂折騰起來,不成道理。如今將他母女帶回,照舊去當差。將秦顯家的仍舊退回。再不必提此事。只是每日小心巡察要緊。”說畢,起身走了。柳家的母女忙向上磕頭,林家的帶回園中,回了李紈探春,二人皆說:“知道了,能可無事,很好。”
\end{parag}


\begin{parag}
    司棋等人空興頭了一陣。那秦顯家的好容易等了這個空子鑽了來,只興頭上半天。在廚房內正亂著接收傢伙米糧煤炭等物,又查出許多虧空來,說:“粳米短了兩石,常用米又多支了一個月的,炭也欠著額數。”一面又打點送林之孝家的禮,悄悄的備了一簍炭,五百斤木柴,一擔粳米,在外邊就遣了子侄送入林家去了;又打點送帳房的禮;又預備幾樣菜蔬請幾位同事的人,說:“我來了,全仗列位扶持。自今以後都是一家人了。我有照顧不到的,好歹大家照顧些。”正亂著,忽有人來說與他:“看過這早飯就出去罷。柳嫂兒原無事,如今還交與他管了。”秦顯家的聽了。轟去魂魄,垂頭喪氣,登時掩旗息鼓,捲包而出。送人之物白丟了許多,自己倒要折變了賠補虧空。連司棋都氣了個倒仰,無計挽回,只得罷了。
\end{parag}


\begin{parag}
    趙姨娘正因彩雲私贈了許多東西,被玉釧兒吵出,生恐查詰出來,每日捏一把汗打聽信兒。忽見彩雲來告訴說:“都是寶玉應了,從此無事。”趙姨娘方把心放下來。誰知賈環聽如此說,便起了疑心,將彩雲凡私贈之物都拿了出來,照著彩雲的臉摔了去,說:“這兩面三刀的東西!我不稀罕。你不和寶玉好,他如何肯替你應。你既有擔當給了我,原該不與一個人知道。如今你既然告訴他,如今我再要這個,也沒趣兒。”彩雲見如此,急的發身賭誓,至於哭了,百般解說,賈環執意不信,說:“不看你素日之情,去告訴二嫂子,就說你偷來給我,我不敢要。你細想去。”說畢,摔手出去了。急的趙姨娘罵:“沒造化的種子,蛆心孽障。”氣的彩雲哭個淚乾腸斷。趙姨娘百般的安慰他:“好孩子,他辜負了你的心,我看的真。讓我收起來,過兩日他自然迴轉過來了。”說著,便要收東西。彩雲賭氣一頓包起來,乘人不見時,來至園中,都撇在河內,順水沉的沉漂的漂了。自己氣的夜間在被內暗哭。
\end{parag}


\begin{parag}
    當下又值寶玉生日已到,原來寶琴也是這日,二人相同。因王夫人不在家,也不曾象往年鬧熱。只有張道士送了四樣禮,換的寄名符兒;還有幾處僧尼廟的和尚姑子送了供尖兒,並壽星紙馬疏頭,並本命星官值年太歲週年換的鎖兒。家中常走的女先兒來上壽。王子騰那邊,仍是一套衣服,一雙鞋襪,一百壽桃,一百束上用銀絲掛麪。薛姨娘處減一等。其餘家中人,尤氏仍是一雙鞋襪;鳳姐兒是一個宮制四面和合荷包,裏面裝一個金壽星,一件波斯國所制玩器。各廟中遣人去放堂舍錢。又另有寶琴之禮,不能備述。姐妹中皆隨便,或有一扇的,或有一字的,或有一畫的,或有一詩的,聊復應景而已。
\end{parag}


\begin{parag}
    這日寶玉清晨起來,梳洗已畢,冠帶出來。至前廳院中,已有李貴等四五個人在那裏設下天地香燭,寶玉炷了香。行畢禮,奠茶焚紙後,便至寧府中宗祠祖先堂兩處行畢禮,出至月臺上,又朝上遙拜過賈母、賈政、王夫人等。一順到尤氏上房,行過禮,坐了一回,方回榮府。先至薛姨媽處,薛姨媽再三拉著,然後又遇見薛蝌,讓一回,方進園來。晴雯麝月二人跟隨,小丫頭夾著氈子,從李氏起,一一挨著,長的房中到過。復出二門,至李、趙、張、王四個奶媽家讓了一回,方進來。雖衆人要行禮,也不曾受。回至房中,襲人等只都來說一聲就是了。王夫人有言,不令年輕人受禮,恐折了福壽,故皆不磕頭。
\end{parag}


\begin{parag}
    歇一時,賈環賈蘭等來了,襲人連忙拉住,坐了一坐,便去了。寶玉笑說走乏了,便歪在牀上。方吃了半盞茶,只聽外面咭咭呱呱,一羣丫頭笑進來,原來是翠墨、小螺、翠縷、入畫,邢岫煙的丫頭篆兒,並奶子抱巧姐兒,綵鸞、繡鸞八九個人,都抱著紅氈笑著走來,說:“拜壽的擠破了門了,快拿面來我們喫。”剛進來時,探春、湘雲、寶琴、岫煙、惜春也都來了。寶玉忙迎出來,笑說:“不敢起動,快預備好茶。”進入房中,不免推讓一回,大家歸坐。襲人等捧過茶來,才吃了一口,平兒也打扮的花枝招展的來了。寶玉忙迎出來,笑說:“我方纔到鳳姐姐門上,回了進去,不能見,我又打發人進去讓姐姐的。”平兒笑道:“我正打發你姐姐梳頭,不得出來回你。後來聽見又說讓我,我那裏禁當的起,所以特趕來磕頭。”寶玉笑道:“我也禁當不起。”襲人早在外間安了坐,讓他坐。平兒便福下去,寶玉作揖不迭。平兒便跪下去,寶玉也忙還跪下,襲人連忙攙起來。又下了福,寶玉又還了一揖。襲人笑推寶玉:“你再作揖。”寶玉道:“已經完了,怎麼又作揖?”襲人笑道:“這是他來給你拜壽。今兒也是他的生日,你也該給他拜壽。”寶玉聽了,喜的忙作下揖去,說:“原來今兒也是姐姐的芳誕。”平兒還萬福不迭。湘雲拉寶琴岫煙說:“你們四個人對拜壽,直拜一天才是。”探春忙問:“原來邢妹妹也是今兒?我怎麼就忘了。”忙命丫頭:“去告訴二奶奶,趕著補了一分禮,與琴姑娘的一樣,送到二姑娘屋裏去。”丫頭答應著去了。岫煙見湘雲直口說出來,少不得要到各房去讓讓。
\end{parag}


\begin{parag}
    探春笑道:“倒有些意思,一年十二個月,月月有幾個生日。人多了,便這等巧,也有三個一日、兩個一日的。大年初一日也不白過,大姐姐佔了去。怨不得他福大,生日比別人就佔先。又是太祖太爺的生日。過了燈節,就是老太太和寶姐姐,他們孃兒兩個遇的巧。三月初一日是太太,初九日是璉二哥哥。二月沒人。”襲人道:“二月十二是林姑娘,怎麼沒人?就只不是咱家的人。”探春笑道:“我這個記性是怎麼了!”寶玉笑指襲人道:“他和林妹妹是一日,所以他記的。”探春笑道:“原來你兩個倒是一日。每年連頭也不給我們磕一個。平兒的生日我們也不知道,這也是才知道。”平兒笑道:“我們是那牌兒名上的人,生日也沒拜壽的福,又沒受禮職份,可吵鬧什麼,可不悄悄的過去。今兒他又偏吵出來了,等姑娘們回房,我再行禮去罷。”探春笑道:“也不敢驚動。只是今兒倒要替你過個生日,我心才過得去。”寶玉湘雲等一齊都說:“很是。”探春便吩咐了丫頭:“去告訴他奶奶,就說我們大家說了,今兒一日不放平兒出去,我們也大家湊了分子過生日呢。”丫頭笑著去了,半日,回來說:“二奶奶說了,多謝姑娘們給他臉。不知過生日給他些什麼喫,只別忘了二奶奶,就不來絮聒他了。”衆人都笑了。
\end{parag}


\begin{parag}
    探春因說道:“可巧今兒裏頭廚房不預備飯,一應下面弄菜都是外頭收拾。咱們就湊了錢叫柳家的來攬了去,只在咱們裏頭收拾倒好。”衆人都說是極。探春一面遣人去問李紈、寶釵、黛玉,一面遣人去傳柳家的進來,吩咐他內廚房中快收拾兩桌酒席。柳家的不知何意,因說外廚房都預備了。探春笑道:“你原來不知道,今兒是平姑娘的華誕。外頭預備的是上頭的,這如今我們私下又湊了分子,單爲平姑娘預備兩桌請他。你只管揀新巧的菜蔬預備了來,開了帳和我那裏領錢。”柳家的笑道:“原來今日也是平姑娘的千秋,我竟不知道。”說著,便向平兒磕下頭去,慌的平兒拉起他來。柳家的忙去預備酒席。
\end{parag}


\begin{parag}
    這裏探春又邀了寶玉,同到廳上去吃麪,等到李紈寶釵一齊來全,又遣人去請薛姨媽與黛玉。因天氣和暖,黛玉之疾漸愈,故也來了。花團錦簇,擠了一廳的人。
\end{parag}


\begin{parag}
    誰知薛蝌又送了巾扇香帛四色壽禮與寶玉,寶玉於是過去陪他吃麪。兩家皆治了壽酒,互相酬送,彼此同領。至午間,寶玉又陪薛蝌吃了兩杯酒。寶釵帶了寶琴過來與薛蝌行禮,把盞畢,寶釵因囑薛蝌:“家裏的酒也不用送過那邊去,這虛套竟可收了。你只請夥計們喫罷。我們和寶兄弟進去還要待人去呢,也不能陪你了。” 薛蝌忙說:“姐姐兄弟只管請,只怕夥計們也就好來了。”寶玉忙又告過罪,方同他姊妹回來。
\end{parag}


\begin{parag}
    一進角門,寶釵便命婆子將門鎖上,把鑰匙要了自己拿著。寶玉忙說:“這一道門何必關,又沒多的人走。況且姨娘、姐姐、妹妹都在裏頭,倘或家去取什麼,豈不費事。”寶釵笑道:“小心沒過逾的。你瞧你們那邊,這幾日七事八事,竟沒有我們這邊的人,可知是這門關的有功效了。若是開著,保不住那起人圖順腳,抄近路從這裏走,攔誰的是?不如鎖了,連媽和我也禁著些,大家別走。縱有了事,就賴不著這邊的人了。”寶玉笑道:“原來姐姐也知道我們那邊近日丟了東西?” 寶釵笑道:“你只知道玫瑰露和茯苓霜兩件,乃因人而及物。若非因人,你連這兩件還不知道呢。殊不知還有幾件比這兩件大的呢。若以後叨登不出來,是大家的造化;若叨登出來,不知裏頭連累多少人呢。你也是不管事的人,我才告訴你。平兒是個明白人,我前兒也告訴了他,皆因他奶奶不在外頭,所以使他明白了。若不出來,大家樂得丟開手。若犯出來,他心裏已有稿子,自有頭緒,就冤屈不著平人了。你只聽我說,以後留神小心就是了,這話也不可對第二個人講。”
\end{parag}


\begin{parag}
    說著,來到沁芳亭邊,只見襲人、香菱、待書、素雲、晴雯、麝月、芳官、蕊官、藕官等十來個人都在那裏看魚作耍。見他們來了,都說:“ 藥欄裏預備下了,快去上席罷。”寶釵等隨攜了他們同到了芍藥欄中紅香圃三間小敞廳內。連尤氏已請過來了,諸人都在那裏,只沒平兒。
\end{parag}


\begin{parag}
    原來平兒出去,有賴林諸家送了禮來,連三接四,上中下三等家人來拜壽送禮的不少,平兒忙著打發賞錢道謝,一面又色色的回明鳳姐兒,不過留下幾樣,也有不收的,也有收下即刻賞與人的。忙了一回,又直待鳳姐兒吃過麪,方換了衣裳往園裏來。
\end{parag}


\begin{parag}
    剛進了園,就有幾個丫鬟來找他,一同到了紅香圃中。只見筵開玳瑁瑁,褥設芙蓉。衆人都笑:“壽星全了。”上面四座定要讓他四個人坐,四人皆不肯。薛姨媽說:“我老天拔地,又不合你們的羣兒,我倒覺拘的慌,不如我到廳上隨便躺躺去倒好。我又喫不下什麼去,又不大喫酒,這裏讓他們倒便宜。”尤氏等執意不從。寶釵道:“這也罷了,倒是讓媽在廳上歪著自如些,有愛喫的送些過去,倒自在了。且前頭沒人在那裏,又可照看了。”探春等笑道:“既這樣,恭敬不如從命。” 因大家送了他到議事廳上,眼看著命丫頭們鋪了一個錦褥並靠背引枕之類,又囑咐:“好生給姨媽捶腿,要茶要水別推三扯四的。回來送了東西來,姨媽吃了就賞你們喫。只別離了這裏出去。”小丫頭們都答應了。
\end{parag}


\begin{parag}
    探春等方回來。終久讓寶琴岫煙二人在上,平兒面西坐,寶玉面東坐。探春又接了鴛鴦來,二人並肩對面相陪。西邊一桌,寶釵黛玉湘雲迎春惜春,一面又拉了香菱玉釧兒二人打橫。三桌上,尤氏李紈又拉了襲人彩雲陪坐。四桌上便是紫鵑、鶯兒、晴雯、小螺、司棋等人圍坐。當下探春等還要把盞,寶琴等四人都說:“這一鬧,一日都坐不成了。”方纔罷了。兩個女先兒要彈詞上壽,衆人都說:“我們沒人要聽那些野話,你廳上去說給姨太太解悶兒去罷。”一面又將各色喫食揀了,命人送與薛姨媽去。
\end{parag}


\begin{parag}
    寶玉便說:“雅坐無趣,須要行令纔好。”衆人有的說行這個令好,那個又說行那個令好。黛玉道:“依我說,拿了筆硯將各色全都寫了,拈成鬮兒,咱們抓出那個來,就是那個。”衆人都道妙。即拿了一副筆硯花箋。香菱近日學了詩,又天天學寫字,見了筆硯便圖不得,連忙起座說:“我寫。”大家想了一回,共得了十來個,念著,香菱一一的寫了,搓成鬮兒,擲在一個瓶中間。探春便命平兒揀,平兒向內攪了一攪,用箸拈了一個出來,打開看,上寫著“射覆”二字。寶釵笑道: “把個酒令的祖宗拈出來。‘射覆’從古有的,如今失了傳,這是後人纂的,比一切的令都難。這裏頭倒有一半是不會的,不如毀了,另拈一個雅俗共賞的。”探春笑道:“既拈了出來,如何又毀。如今再拈一個,若是雅俗共賞的,便叫他們行去。咱們行這個。”說著又著襲人拈了一個,卻是“拇戰”。史湘雲笑著說:“這個簡斷爽利,合了我的脾氣。我不行這個‘射覆’,沒的垂頭喪氣悶人,我只划拳去了。”探春道:“惟有他亂令,寶姐姐快罰他一鍾。”寶釵不容分說,便灌湘雲一杯。
\end{parag}


\begin{parag}
    探春道:“我喫一杯,我是令官,也不用宣,只聽我分派。”命取了令骰令盆來,“從琴妹擲起,挨下擲去,對了點的二人射覆。”寶琴一擲,是個三,岫煙寶玉等皆擲的不對,直到香菱方擲了一個三。寶琴笑道:“只好室內生春,若說到外頭去,可太沒頭緒了。”探春道:“自然。三次不中者罰一杯。你覆,他射。”寶琴想了一想,說了個“老”字。香菱原生於這令,一時想不到,滿室滿席都不見有與“老”字相連的成語。湘雲先聽了,便也亂看,忽見門斗上貼著“紅香圃”三個字,便知寶琴覆的是“吾不如老圃”的“圃”字。見香菱射不著,衆人擊鼓又催,便悄悄的拉香菱,教他說“藥”字。黛玉偏看見了,說“快罰他,又在那裏私相傳遞呢。”哄的衆人都知道了,忙又罰了一杯,恨的湘雲拿筷子敲黛玉的手。於是罰了香菱一杯。下則寶釵和探春對了點子。探春便覆了一個“人”字。寶釵笑道:“這個‘人’字泛的很。”探春笑道:“添一字,兩覆一射也不泛了。”說著,便又說了一個“窗”字。寶釵一想,因見席上有雞,便射著他是用“雞窗” “雞人”二典了,因射了一個“塒”字。探春知他射著,用了“雞棲於塒”的典,二人一笑,各飲一口門杯。
\end{parag}


\begin{parag}
    湘雲等不得,早和寶玉“三”“五”亂叫,划起拳來。那邊尤氏和鴛鴦隔著席也“七”“八”亂叫划起來。平兒襲人也作了一對划拳,叮叮噹噹只聽得腕上的鐲子響。一時湘雲贏了寶玉,襲人贏了平兒,尤氏贏了鴛鴦,三個人限酒底酒面,湘雲便說:“酒面要一句古文,一句舊詩,一句骨牌名,一句曲牌名,還要一句時憲書上的話,共總湊成一句話。酒底要關人事的果菜名。”衆人聽了,都笑說:“惟有他的令也比人嘮叨,倒也有意思。”便催寶玉快說。寶玉笑道:“誰說過這個,也等想一想兒。”黛玉便道:“你多喝一鍾,我替你說。”寶玉真個喝了酒,聽黛玉說道:
\end{parag}


\begin{poem}
    \begin{pl}落霞與孤鶩齊飛,風急江天過雁哀,卻是一隻折足雁,叫的人九迴腸,這是鴻雁來賓。\end{pl}
\end{poem}


\begin{parag}
    說的大家笑了,說:“這一串子倒有些意思。”黛玉又拈了一個榛穰,說酒底道:
\end{parag}


\begin{poem}
    \begin{pl}榛子非關隔院砧,何來萬戶搗衣聲。\end{pl}
\end{poem}


\begin{parag}
    令完,鴛鴦襲人等皆說的是一句俗語,都帶一個“壽”字的,不能多贅。
\end{parag}


\begin{parag}
    大家輪流亂劃了一陣,這上面湘雲又和寶琴對了手,李紈和岫煙對了點子。李紈便覆了一個“瓢”字,岫煙便射了一個“綠”字,二人會意,各飲一口。湘雲的拳卻輸了,請酒面酒底。寶琴笑道:“請君入甕。”大家笑起來,說:“這個典用的當。”湘雲便說道:
\end{parag}


\begin{poem}
    \begin{pl}奔騰而砰湃,江間波浪兼天湧,須要鐵鎖纜孤舟,既遇著一江風,不宜出行。\end{pl}

\end{poem}


\begin{parag}
    說的衆人都笑了,說:“好個謅斷了腸子的。怪道他出這個令,故意惹人笑。”又聽他說酒底。湘雲吃了酒,揀了一塊鴨肉呷口,忽見碗內有半個鴨頭,遂揀了出來喫腦子。衆人催他:“別隻顧喫,到底快說了。”湘雲便用箸子舉著說道:
\end{parag}


\begin{poem}
    \begin{pl}這鴨頭不是那丫頭,頭上那討桂花油。\end{pl}
\end{poem}


\begin{parag}
    衆人越發笑起來,引的晴雯、小螺、鶯兒等一干人都走過來說:“雲姑娘會開心兒,拿著我們取笑兒,快罰一杯才罷。怎見得我們就該擦桂花油的?倒得每人給一瓶子桂花油擦擦。”黛玉笑道:“他倒有心給你們一瓶子油,又怕掛誤著打盜竊的官司。”衆人不理論,寶玉卻明白,忙低了頭。彩雲有心病,不覺的紅了臉。寶釵忙暗暗的瞅了黛玉一眼。黛玉自悔失言,原是趣寶玉的,就忘了趣著彩雲。自悔不及,忙一頓行令划拳岔開了。
\end{parag}


\begin{parag}
    底下寶玉可巧和寶釵對了點子。寶釵覆了一個“寶”字,寶玉想了一想,便知是寶釵作戲指自己所佩通靈玉而言,便笑道:“姐姐拿我作雅謔,我卻射著了。說出來姐姐別惱,就是姐姐的諱‘釵’字就是了。”衆人道:“怎麼解?”寶玉道:“他說‘寶’,底下自然是‘玉’了。我射‘釵’字,舊詩曾有‘敲斷玉釵紅燭冷’,豈不射著了。”湘雲說道:“這用時事卻使不得,兩個人都該罰。”香菱忙道:“不止時事,這也有出處。”湘雲道:“‘寶玉’二字並無出處,不過是春聯上或有之,詩書紀載並無,算不得。”香菱道:“前日我讀岑嘉州五言律,現有一句說‘此鄉多寶玉’,怎麼你倒忘了?後來又讀李義山七言絕句,又有一句‘寶釵無日不生塵’,我還笑說他兩個名字都原來在唐詩上呢。”衆人笑說:“這可問住了,快罰一杯。”湘雲無語,只得飲了。大家又該對點的對點,划拳的划拳。這些人因賈母王夫人不在家,沒了管束,便任意取樂,呼三喝四,喊七叫八。滿廳中紅飛翠舞,玉動珠搖,真是十分熱鬧。頑了一回,大家方起席散了一散,倏然不見了湘雲,只當他外頭自便就來,誰知越等越沒了影響,使人各處去找,那裏找得著。
\end{parag}


\begin{parag}
    接著林之孝家的同著幾個老婆子來,生恐有正事呼喚,二者恐丫鬟們年青,乘王夫人不在家不服探春等約束,姿意痛飲,失了體統,故來請問有事無事。探春見他們來了,便知其意,忙笑道:“你們又不放心,來查我們來了。我們沒有多喫酒,不過是大家頑笑,將酒作個引子,媽媽們別耽心。”李紈尤氏都也笑說:“你們歇著去罷,我們也不敢叫他們多吃了。”林之孝家的等人笑說:“我們知道,連老太太叫姑娘喫酒姑娘們還不肯喫,何況太太們不在家,自然頑罷了。我們怕有事,來打聽打聽。二則天長了,姑娘們頑一回子還該點補些小食兒。素日又不大喫雜東西,如今喫一兩杯酒,若不多喫些東西,怕受傷。”探春笑道:“媽媽們說的是,我們也正要喫呢。”因回頭命取點心來。兩旁丫鬟們答應了,忙去傳點心。探春又笑讓:“你們歇著去罷,或是姨媽那裏說話兒去。我們即刻打發人送酒你們喫去。”林之孝家的等人笑回:“不敢領了。”又站了一回,方退了出來。平兒摸著臉笑道:“我的臉都熱了,也不好意思見他們。依我說竟收了罷,別惹他們再來,倒沒意思了。”探春笑道:“不相干,橫豎咱們不認真喝酒就罷了。”
\end{parag}


\begin{parag}
    正說著,只見一個小丫頭笑嘻嘻的走來:“姑娘們快瞧雲姑娘去,喫醉了圖涼快,在山子後頭一塊青板石凳上睡著了。”衆人聽說,都笑道:“快別吵嚷。”說著,都走來看時,果見湘雲臥于山石僻處一個石凳子上,業經香夢沉酣,四面芍藥花飛了一身,滿頭臉衣襟上皆是紅香散亂,手中的扇子在地下,也半被落花埋了,一羣蜂蝶鬧穰穰的圍著他,又用鮫帕包了一包芍藥花瓣枕著。衆人看了,又是愛,又是笑,忙上來推喚挽扶。湘雲口內猶作睡語說酒令,唧唧嘟嘟說:
\end{parag}


\begin{poem}
    \begin{pl}泉香而酒冽,玉盞盛來琥珀光,直飲到梅梢月上,醉扶歸,卻爲宜會親友。\end{pl}
\end{poem}


\begin{parag}
    衆人笑推他,說道:“快醒醒兒喫飯去,這潮凳上還睡出病來呢。”湘雲慢啓秋波,見了衆人,低頭看了一看自己,方知是醉了。原是來納涼避靜的,不覺的因多罰了兩杯酒,嬌嫋不勝,便睡著了,心中反覺自愧。連忙起身扎掙著同人來至紅香圃中,用過水,又吃了兩盞釅茶。探春忙命將醒酒石拿來給他銜在口內,一時又命他喝了一些酸湯,方纔覺得好了些。
\end{parag}


\begin{parag}
    當下又選了幾樣果菜與鳳姐送去,鳳姐兒也送了幾樣來。寶釵等喫過點心,大家也有坐的,也有立的,也有在外觀花的,也有扶欄觀魚的,各自取便說笑不一。探春便和寶琴下棋,寶釵岫煙觀局。林黛玉和寶玉在一簇花下唧唧噥噥不知說些什麼。只見林之孝家的和一羣女人帶了一個媳婦進來。那媳婦愁眉苦臉,也不敢進廳,只到了階下,便朝上跪下了,碰頭有聲。探春因一塊棋受了敵,算來算去總得了兩個眼,便折了官著,兩眼只瞅著棋枰,一隻手卻伸在盒內,只管抓弄棋子作想,林之孝家的站了半天,因回頭要茶時纔看見,問:“什麼事?”林之孝家的便指那媳婦說:“這是四姑娘屋裏的小丫頭彩兒的娘,現是園內伺候的人。嘴很不好,纔是我聽見了問著他,他說的話也不敢回姑娘,竟要攆出去纔是。”探春道:“怎麼不回大奶奶?”林之孝家的道:“方纔大奶奶都往廳上姨太太處去了,頂頭看見,我已回明白了,叫回姑娘來。”探春道:“怎麼不回二奶奶?”平兒道:“不回去也罷,我回去說一聲就是了。”探春點點頭,道:“既這麼著,就攆出他去,等太太來了,再回定奪。”說畢仍又下棋。這林之孝家的帶了那人去不提。
\end{parag}


\begin{parag}
    黛玉和寶玉二人站在花下,遙遙知意。黛玉便說道:“你家三丫頭倒是個乖人。雖然叫他管些事,倒也一步兒不肯多走。差不多的人就早作起威福來了。”寶玉道:“你不知道呢。你病著時,他幹了好幾件事。這園子也分了人管,如今多掐一草也不能了。又蠲了幾件事,單拿我和鳳姐姐作筏子禁別人。最是心裏有算計的人,豈只乖而已。”黛玉道:“要這樣纔好,咱們家裏也太花費了。我雖不管事,心裏每常閒了,替你們一算計,出的多進的少,如今若不省儉,必致後手不接。” 寶玉笑道:“憑他怎麼後手不接,也短不了咱們兩個人的。”黛玉聽了,轉身就往廳上尋寶釵說笑去了。
\end{parag}


\begin{parag}
    寶玉正欲走時,只見襲人走來,手內捧著一個小連環洋漆茶盤,裏面可式放著兩鍾新茶,因問:“他往那去了?我見你兩個半日沒喫茶,巴巴的倒了兩鍾來,他又走了。”寶玉道:“那不是他,你給他送去。”說著自拿了一鍾。襲人便送了那鍾去,偏和寶釵在一處,只得一鍾茶,便說:“那位渴了那位先接了,我再倒去。”寶釵笑道:“我卻不渴,只要一口漱一漱就夠了。”說著先拿起來喝了一口,剩下半杯遞在黛玉手內。襲人笑說:“我再倒去。”黛玉笑道:“你知道我這病,大夫不許我多喫茶,這半鍾儘夠了,難爲你想的到。”說畢,飲幹,將杯放下。襲人又來接寶玉的。寶玉因問:“這半日沒見芳官,他在那裏呢?”襲人四顧一瞧說:“纔在這裏幾個人鬥草的,這會子不見了。”
\end{parag}


\begin{parag}
    寶玉聽說,便忙回至房中,果見芳官面向裏睡在牀上。寶玉推他說道:“快別睡覺,咱們外頭頑去,一回兒好喫飯的。”芳官道:“你們喫酒不理我,教我悶了半日,可不來睡覺罷了。”寶玉拉了他起來,笑道:“咱們晚上家裏再喫,回來我叫襲人姐姐帶了你桌上喫飯,何如?”芳官道:“藕官蕊官都不上去,單我在那裏也不好。我也不慣喫那個麪條子,早起也沒好生喫。纔剛餓了,我已告訴了柳嫂子,先給我做一碗湯盛半碗粳米飯送來,我這裏吃了就完事。若是晚上喫酒,不許教人管著我,我要盡力喫夠了才罷。我先在家裏,喫二三斤好惠泉酒呢。如今學了這勞什子,他們說怕壞嗓子,這幾年也沒聞見。乘今兒我是要開齋了。”寶玉道: “這個容易。”
\end{parag}


\begin{parag}
    說著,只見柳家的果遣了人送了一個盒子來。小燕接著揭開,裏面是一碗蝦丸雞皮湯,又是一碗酒釀清蒸鴨子,一碟醃的胭脂鵝脯,還有一碟四個奶油松瓤卷酥,並一大碗熱騰騰碧熒熒蒸的綠畦香稻粳米飯。小燕放在案上,走去拿了小菜並碗箸過來,撥了一碗飯。芳官便說:“油膩膩的,誰喫這些東西。”只將湯泡飯吃了一碗,揀了兩塊醃鵝就不吃了。寶玉聞著,倒覺比往常之味有勝些似的,遂吃了一個卷酥,又命小燕也撥了半碗飯,泡湯一喫,十分香甜可口。小燕和芳官都笑了。喫畢,小燕便將剩的要交回。寶玉道:“你吃了罷,若不夠再要些來。”小燕道:“不用要,這就夠了。方纔麝月姐姐拿了兩盤子點心給我們吃了,我再吃了這個,盡不用再吃了。”說著,便站在桌旁一頓吃了,又留下兩個卷酥,說:“這個留著給我媽喫。晚上要喫酒,給我兩碗酒喫就是了。”寶玉笑道:“你也愛喫酒?等著咱們晚上痛喝一陣。你襲人姐姐和晴雯姐姐量也好,也要喝,只是每日不好意思。今兒大家開齋。還有一件事,想著囑咐你,我竟忘了,此刻纔想起來。以後芳官全要你照看他,他或有不到的去處,你提他,襲人照顧不過這些人來。”小燕道:“我都知道,都不用操心。但只這五兒怎麼樣?”寶玉道:“你和柳家的說去,明兒直叫他進來罷,等我告訴他們一聲就完了。”芳官聽了,笑道:“這倒是正經。”小燕又叫兩個小丫頭進來,伏侍洗手倒茶,自己收了傢伙,交與婆子,也洗了手,便去找柳家的,不在話下。
\end{parag}


\begin{parag}
    寶玉便出來,仍往紅香圃尋衆姐妹,芳官在後拿著巾扇。剛出了院門,只見襲人晴雯二人攜手回來。寶玉問:“你們做什麼?”襲人道:“擺下飯了,等你喫飯呢。”寶玉便笑著將方纔喫的飯一節告訴了他兩個。襲人笑道:“我說你是貓兒食,聞見了香就好,隔鍋飯兒香。雖然如此,也該上去陪他們多少應個景兒。”晴雯用手指戳在芳官額上,說道:“你就是個狐媚子,什麼空兒跑了去喫飯,兩個人怎麼就約下了,也不告訴我們一聲兒。”襲人笑道:“不過是誤打誤撞的遇見了,說約下了可是沒有的事。”晴雯道:“既這麼著,要我們無用。明兒我們都走了,讓芳官一個人就夠使了。”襲人笑道:“我們都去了使得,你卻去不得。”晴雯道: “惟有我是第一個要去,又懶又笨,性子又不好,又沒用。”襲人笑道:“倘或那孔雀褂子再燒個窟窿,你去了誰可會補呢。你倒別和我拿三撇四的,我煩你做個什麼,把你懶的橫針不拈,豎線不動。一般也不是我的私活煩你,橫豎都是他的,你就都不肯做。怎麼我去了幾天,你病的七死八活,一夜連命也不顧給他做了出來,這又是什麼原故?你到底說話,別隻佯憨,和我笑,也當不了什麼。”大家說著,來至廳上。薛姨媽也來了。大家依序坐下喫飯。寶玉只用茶泡半碗飯,應景而已。一時喫畢,大家喫茶閒話,又隨便頑笑。
\end{parag}


\begin{parag}
    外面小螺和香菱、芳官、蕊官、藕官、荳官等四五個人,都滿園中頑了一回,大家採了些花草來兜著,坐在花草堆中鬥草。這一個說:“我有觀音柳。”那一個說:“我有羅漢松。”那一個又說:“我有君子竹。”這一個又說:“我有美人蕉。”這個又說:“我有星星翠。”那個又說:“我有月月紅。”這個又說:“我有《牡丹亭》畔的牡丹葉。”那個又說:“我有《琵琶記》裏的枇杷果。”荳官便說:“我有姐妹花。”衆人沒了,香菱便說:“我有夫妻蕙。”荳官說:“從沒聽見有個夫妻蕙。”香菱道:“一箭一花爲蘭,一箭數花爲蕙。凡蕙有兩枝,上下結花者爲兄弟蕙,有並頭結花者爲夫妻蕙。我這枝並頭的,怎麼不是。”荳官沒的說了,便起身笑道:“依你說,若是這兩枝一大一小,就是老子兒子蕙了。若兩枝背面開的,就是仇人蕙了。你漢子去了大半年,你想夫妻了?便扯上蕙也有夫妻,好不害羞!”香菱聽了,紅了臉,忙要起身擰他,笑罵道:“我把你這個爛了嘴的小蹄子!滿嘴裏汗□的胡說了。等我起來打不死你這小蹄子!”荳官見他要勾來,怎容他起來,便忙連身將他壓倒。回頭笑著央告蕊官等:“你們來,幫著我擰他這謅嘴。”兩個人滾在草地下。衆人拍手笑說:“了不得了,那是一窪子水,可惜污了他的新裙子了。”荳官回頭看了一看,果見旁邊有一汪積雨,香菱的半扇裙子都污溼了,自己不好意思,忙奪了手跑了。衆人笑個不住,怕香菱拿他們出氣,也都鬨笑一散。
\end{parag}


\begin{parag}
    香菱起身低頭一瞧,那裙上猶滴滴點點流下綠水來。正恨罵不絕,可巧寶玉見他們鬥草,也尋了些花草來湊戲,忽見衆人跑了,只剩了香菱一個低頭弄裙,因問:“怎麼散了?”香菱便說:“我有一枝夫妻蕙,他們不知道,反說我謅,因此鬧起來,把我的新裙子也髒了。”寶玉笑道:“你有夫妻蕙,我這裏倒有一枝並蒂菱。”口內說,手內卻真個拈著一枝並蒂菱花,又拈了那枝夫妻蕙在手內。香菱道:“什麼夫妻不夫妻,並蒂不併蒂,你瞧瞧這裙子。”寶玉方低頭一瞧,便噯呀了一聲,說:“怎麼就拖在泥裏了?可惜這石榴紅綾最不經染。”香菱道:“這是前兒琴姑娘帶了來的。姑娘做了一條,我做了一條,今兒才上身。”寶玉跌腳嘆道: “若你們家,一日遭踏這一百件也不值什麼。只是頭一件既系琴姑娘帶來的,你和寶姐姐每人才一件,他的尚好,你的先髒了,豈不辜負他的心。二則姨媽老人家嘴碎,饒這麼樣,我還聽見常說你們不知過日子,只會遭踏東西,不知惜福呢。這叫姨媽看見了,又說一個不清。”香菱聽了這話,卻碰在心坎兒上,反倒喜歡起來了,因笑道:“就是這話了。我雖有幾條新裙子,都不和這一樣的,若有一樣的,趕著換了,也就好了。過後再說。”寶玉道:“你快休動,只站著方好,不然連小衣兒膝褲鞋面都要拖髒。我有個主意:襲人上月做了一條和這個一模一樣的,他因有孝,如今也不穿。竟送了你換下這個來,如何?”香菱笑著搖頭說:“不好。他們倘或聽見了倒不好。”寶玉道:“這怕什麼。等他們孝滿了,他愛什麼難道不許你送他別的不成。你若這樣,還是你素日爲人了!況且不是瞞人的事,只這告訴寶姐姐也可,只不過怕姨媽老人家生氣罷了。”香菱想了一想有理,便點頭笑道:“就是這樣罷了,別辜負了你的心。我等著你,千萬叫他親自送來纔好。”
\end{parag}


\begin{parag}
    寶玉聽了,喜歡非常,答應了忙忙的回來,一壁裏低頭心下暗算:“可惜這麼一個人,沒父母,連自己本姓都忘了,被人拐出來,偏又賣與了這個霸王。”因又想起上日平兒也是意外想不到的,今日更是意外之意外的事了。一壁胡思亂想,\begin{note}庚雙夾:又下此四字。\end{note}來至房中,拉了襲人,細細告訴了他原故。香菱之爲人,無人不憐愛的。襲人又本是個手中撒漫的,況與香菱素相交好,一聞此信,忙就開箱取了出來摺好,隨了寶玉來尋著香菱,他還站在那裏等呢。襲人笑道:“我說你太淘氣了,足的淘出個故事來才罷。”香菱紅了臉,笑說:“多謝姐姐了,誰知那起促狹鬼使黑心。”說著,接了裙子,展開一看,果然同自己的一樣。又命寶玉背過臉去,自己叉手向內解下來,將這條繫上。襲人道:“把這髒了的交與我拿回去,收拾了再給你送來。你若拿回去,看見了也是要問的。”香菱道:“好姐姐,你拿去不拘給那個妹妹罷。我有了這個,不要他了。”襲人道:“你倒大方的好。”香菱忙又萬福道謝,襲人拿了髒裙便走。
\end{parag}


\begin{parag}
    香菱見寶玉蹲在地下,將方纔的夫妻蕙與並蒂菱用樹枝兒摳了一個坑,先抓些落花來鋪墊了,將這菱蕙安放好,又將些落花來掩了,方撮土掩埋平服。香菱拉他的手,笑道:“這又叫做什麼?怪道人人說你慣會鬼鬼祟祟使人肉麻的事。你瞧瞧,你這手弄的泥烏苔滑的,還不快洗去。”寶玉笑著,方起身走了去洗手,香菱也自走開。二人已走遠了數步,香菱復轉身回來叫住寶玉。寶玉不知有何話,扎著兩隻泥手,笑嘻嘻的轉來問:“什麼?”香菱只顧笑。因那邊他的小丫頭臻兒走來說:“二姑娘等你說話呢。”香菱方向寶玉道:“裙子的事可別向你哥哥說纔好。”說畢,即轉身走了。寶玉笑道:“可不我瘋了,往虎口裏探頭兒去呢。”說著,也回去洗手去了。不知端詳,且聽下回分解。
\end{parag}