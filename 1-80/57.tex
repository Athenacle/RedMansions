\chap{五十七}{慧紫鵑情辭試忙玉 慈姨媽愛語慰癡顰}


\begin{parag}
    話說寶玉聽王夫人喚他,忙至前邊來,原來是王夫人要帶他拜甄夫人去。寶玉自是歡喜,忙去換衣服,跟了王夫人到那裏。見其家中形景,自與榮寧不甚差別,或有一二稍盛者。細問,果有一寶玉。甄夫人留席,竟日方回,寶玉方信。因晚間回家來,王夫人又吩咐預備上等的席面,定名班大戲,請過甄夫人母女。後二日,他母女便不作辭,回任去了,無話。
\end{parag}


\begin{parag}
    這日寶玉因見湘雲漸愈,然後去看黛玉。正值黛玉才歇午覺,寶玉不敢驚動,因紫鵑正在迴廊上手裏做針黹,便來問他:“昨日夜裏咳嗽可好了?”紫鵑道: “好些了。”寶玉笑道:“阿彌陀佛!寧可好了罷。”紫鵑笑道:“你也念起佛來,真是新聞!”寶玉笑道:“所謂‘病篤亂投醫’了。”一面說,一面見他穿著彈墨綾薄棉襖,外面只穿著青緞夾背心,寶玉便伸手向他身上摸了一摸,說:“穿這樣單薄,還在風口裏坐著,看天風饞,時氣又不好,你再病了,越發難了。”紫鵑便說道:“從此咱們只可說話,別動手動腳的。一年大二年小的,叫人看著不尊重。打緊的那起混帳行子們背地裏說你,你總不留心,還只管和小時一般行爲,如何使得。姑娘常常吩咐我們,不叫和你說笑。你近來瞧他遠著你還恐遠不及呢。”說著便起身,攜了針線進別房去了。
\end{parag}


\begin{parag}
    寶玉見了這般景況,心中忽澆了一盆冷水一般,只瞅著竹子,發了一回呆。因祝媽正來挖筍修竿,便怔怔的走出來,一時魂魄失守,心無所知,隨便坐在一塊山石上出神,不覺滴下淚來。直呆了五六頓飯工夫,千思萬想,總不知如何是可。偶值雪雁從王夫人房中取了人蔘來,從此經過,忽扭項看見桃花樹下石上一人手託著腮頰出神,不是別人,卻是寶玉。\begin{note}庚雙夾:畫出寶玉來,卻又不畫阿顰,何等筆力!便不從鵑寫,卻寫一雁,更奇。是仍歸寫鵑。\end{note}雪雁疑惑道:“怪冷的,他一個人在這裏作什麼?春天凡有殘疾的人都犯病,敢是他犯了呆病了?”\begin{note}庚雙夾:寫嬌憨女兒之心何等新巧。\end{note}一邊想,一邊便走過來蹲下笑道: “你在這裏作什麼呢?”寶玉忽見了雪雁,便說道:“你又作什麼來找我?你難道不是女兒?他既防嫌,不許你們理我,你又來尋我,倘被人看見,豈不又生口舌?你快家去罷了。”雪雁聽了,只當是他又受了黛玉的委屈,只得回至房中。
\end{parag}


\begin{parag}
    黛玉未醒,將人蔘交與紫鵑。紫鵑因問他:“太太做什麼呢?”雪雁道:“也歇中覺,所以等了這半日。姐姐你聽笑話兒:我因等太太的工夫,和玉釧兒姐姐坐在下房裏說話兒,誰知趙姨奶奶招手兒叫我。我只當有什麼話說,原來他和太太告了假,出去給他兄弟伴宿坐夜,明兒送殯去,跟他的小丫頭子小吉祥兒沒衣裳,要借我的月白緞子襖兒。我想他們一般也有兩件子的,往髒地方兒去恐怕弄髒了,自己的捨不得穿,故此借別人的。借我的弄髒了也是小事,只是我想,他素日有些什麼好處到咱們跟前,所以我說了:‘我的衣裳簪環都是姑娘叫紫鵑姐姐收著呢。如今先得去告訴他,還得回姑娘呢。姑娘身上又病著,更費了大事,誤了你老出門,不如再轉借罷。’”紫鵑笑道:“你這個小東西倒也巧。你不借給他,你往我和姑娘身上推,叫人怨不著你。他這會子就下去了,還是等明日一早纔去?”雪雁道: “這會子就去的,只怕此時已去了。”紫鵑點點頭。雪雁道:“姑娘還沒醒呢,是誰給了寶玉氣受,坐在那裏哭呢。”紫鵑聽了,忙問在那裏。雪雁道:“在沁芳亭後頭桃花底下呢。”
\end{parag}


\begin{parag}
    紫鵑聽說,忙放下針線,又囑咐雪雁好生聽叫:“若問我,答應我就來。”說著,便出了瀟湘館,一徑來尋寶玉,走至寶玉跟前,含笑說道:“我不過說了那兩句話,爲的是大家好,你就賭氣跑了這風地裏來哭,作出病來唬我。”寶玉忙笑道:“誰賭氣了!我因爲聽你說的有理,我想你們既這樣說,自然別人也是這樣說,將來漸漸的都不理我了,我所以想著自己傷心。”紫鵑也便挨他坐著。寶玉笑道:“方纔對面說話你尚走開,這會子如何又來挨我坐著?”紫鵑道:“你都忘了?幾日前你們姊妹兩個正說話,趙姨娘一頭走了進來,──我才聽見他不在家,所以我來問你。正是前日你和他才說了一句‘燕窩’就歇住了,總沒提起,我正想著問你。”寶玉道:“也沒什麼要緊。不過我想著寶姐姐也是客中,既喫燕窩,又不可間斷,若只管和他要,也太托實。雖不便和太太要,我已經在老太太跟前略露了個風聲,只怕老太太和鳳姐姐說了。我告訴他的,竟沒告訴完了他。如今我聽見一日給你們一兩燕窩,這也就完了。”紫鵑道:“原來是你說了,這又多謝你費心。我們正疑惑,老太太怎麼忽然想起來叫人每一日送一兩燕窩來呢?這就是了。”寶玉笑道:“這要天天喫慣了,喫上三二年就好了。”紫鵑道:“在這裏喫慣了,明年家去,那裏有這閒錢喫這個。”寶玉聽了,吃了一驚,忙問:“誰?往那個家去?”\begin{note}庚雙夾:這句不成話,細讀細嚼方有無限神情滋味。\end{note}紫鵑道:“你妹妹回蘇州家去。”寶玉笑道:\begin{note}庚雙夾:“笑”字奇甚。\end{note}“你又說白話。蘇州雖是原籍,因沒了姑父姑母,無人照看,才就了來的。明年回去找誰?可見是扯謊。”紫鵑冷笑道:“你太看小了人。你們賈家獨是大族人口多的,除了你家,別人只得一父一母,房族中真個再無人了不成?我們姑娘來時,原是老太太心疼他年小,雖有叔伯,不如親父母,故此接來住幾年。大了該出閣時,自然要送還林家的。終不成林家的女兒在你賈家一世不成?林家雖貧到沒飯喫,也是世代書宦人家,斷不肯將他家的人丟在親戚家,落人的恥笑。所以早則明年春天,遲則秋天。這裏縱不送去,林家亦必有人來接的。前日夜裏姑娘和我說了,叫我告訴你:將從前小時頑的東西,有他送你的,叫你都打點出來還他。他也將你送他的打疊了在那裏呢。”寶玉聽了,便如頭頂上響了一個焦雷一般。紫鵑看他怎樣回答,只不作聲。忽見晴雯找來說:“老太太叫你呢,誰知道在這裏。”紫鵑笑道:“他這裏問姑娘的病症。我告訴了他半日,他只不信。你倒拉他去罷。”說著,自己便走回房去了。
\end{parag}


\begin{parag}
    晴雯見他呆呆的,一頭熱汗,滿臉紫脹,忙拉他的手,一直到怡紅院中。襲人見了這般,慌起來,只說時氣所感,熱汗被風撲了。無奈寶玉發熱事猶小可,更覺兩個眼珠兒直直的起來,口角邊津液流出,皆不知覺。給他個枕頭,他便睡下;扶他起來,他便坐著;倒了茶來,他便喫茶。衆人見他這般,一時忙起來,又不敢造次去回賈母,先便差人出去請李嬤嬤。
\end{parag}


\begin{parag}
    一時李嬤嬤來了,看了半日,問他幾句話也無回答,用手向他脈門摸了摸,嘴脣人中上邊著力掐了兩下,掐的指印如許來深,竟也不覺疼。李嬤嬤只說了一聲 “可了不得了”,“呀”的一聲便摟著放聲大哭起來。急的襲人忙拉他說:“你老人家瞧瞧,可怕不怕?且告訴我們去回老太太、太太去。你老人家怎麼先哭起來?”李嬤嬤捶牀倒枕說:“這可不中用了!我白操了一世心了!”襲人等 他年老多知,所以請他來看,如今見他這般一說,都信以爲實,也都哭起來。
\end{parag}


\begin{parag}
    晴雯便告訴襲人,方纔如此這般。襲人聽了,便忙到瀟湘館來,見紫鵑正伏侍黛玉吃藥,也顧不得什麼,便走上來問紫鵑道:“你才和我們寶玉說了些什麼?你瞧他去,你回老太太去,我也不管了!”說著,便坐在椅上。黛玉忽見襲人滿面急怒,又有淚痕,舉止大變,便不免也慌了,忙問怎麼了。襲人定了一回,哭道: “不知紫鵑姑奶奶說了些什麼話,那個呆子眼也直了,手腳也冷了,話也不說了,李媽媽掐著也不疼了,已死了大半個了!\begin{note}庚雙夾:奇極之語。從急怒嬌憨口中描出不成話之話來,方是千古奇文。五字是一口氣來的。\end{note}連李媽媽都說不中用了,那裏放聲大哭。只怕這會子都死了!”黛玉一聽此言,李媽媽乃是經過的老嫗,說不中用了,可知必不中用。哇的一聲,將腹中之藥一概嗆出,抖腸搜肺,熾胃扇肝的痛聲大嗽了幾陣,一時面紅發亂,目腫筋浮,喘的抬不起頭來。紫鵑忙上來捶背,黛玉伏枕喘息半晌,推紫鵑道:“你不用捶,你竟拿繩子來勒死我是正經!”紫鵑哭道:“我並沒說什麼,不過是說了幾句頑話,他就認真了。”襲人道: “你還不知道他,那傻子每每頑話認了真。”黛玉道:“你說了什麼話,趁早兒去解說,他只怕就醒過來了。”紫鵑聽說,忙下了牀,同襲人到了怡紅院。
\end{parag}


\begin{parag}
    誰知賈母王夫人等已都在那裏了。賈母一見了紫鵑,眼內出火,罵道:“你這小蹄子,和他說了什麼?”紫鵑忙道:“並沒說什麼,不過說幾句頑話。”誰知寶玉見了紫鵑,方“噯呀”了一聲,哭出來了。衆人一見,方都放下心來。賈母便拉住紫鵑,只當他得罪了寶玉,所以拉紫鵑命他打。誰知寶玉一把拉住紫鵑,死也不放,說:“要去連我也帶了去。”衆人不解,細問起來,方知紫鵑說“要回蘇州去”一句頑話引出來的。賈母流淚道:“我當有什麼要緊大事,原來是這句頑話。” 又向紫鵑道:“你這孩子素日最是個伶俐聰敏的,你又知道他有個呆根子,平白的哄他作什麼?”薛姨媽勸道:“寶玉本來心實,可巧林姑娘又是從小兒來的,他姊妹兩個一處長了這麼大,比別的姊妹更不同。這會子熱剌剌的說一個去,別說他是個實心的傻孩子,便是冷心腸的大人也要傷心。這並不是什麼大病,老太太和姨太太只管萬安,喫一兩劑藥就好了。”
\end{parag}


\begin{parag}
    正說著,人回林之孝家的單大良家的都來瞧哥兒來了。賈母道:“難爲他們想著,叫他們來瞧瞧。”寶玉聽了一個“林”字,便滿牀鬧起來說:“了不得了,林家的人接他們來了,快打出去罷!”賈母聽了,也忙說:“打出去罷。”又忙安慰說:“那不是林家的人。林家的人都死絕了,沒人來接他的,你只放心罷。”寶玉哭道:“憑他是誰,除了林妹妹,都不許姓林的!”賈母道:“沒姓林的來,凡姓林的我都打走了。”一面吩咐衆人:“以後別叫林之孝家的進園來,你們也別說 ‘林’字。好孩子們,你們聽我這句話罷!”衆人忙答應,又不敢笑。一時寶玉又一眼看見了十錦格子上陳設的一隻金西洋自行船,便指著亂叫說:“那不是接他們來的船來了,灣在那裏呢。”賈母忙命拿下來。襲人忙拿下來,寶玉伸手要,襲人遞過,寶玉便掖在被中,笑道:“可去不成了!”一面說,一面死拉著紫鵑不放。
\end{parag}


\begin{parag}
    一時人回大夫來了,賈母忙命快進來。王夫人、薛姨媽、寶釵等暫避裏間,賈母便端坐在寶玉身旁。王太醫進來見許多的人,忙上去請了賈母的安,拿了寶玉的手診了一回。那紫鵑少不得低了頭。王大夫也不解何意,起身說道:“世兄這症乃是急痛迷心。古人曾雲:‘痰迷有別。有氣血虧柔,飲食不能熔化痰迷者;有怒惱中痰裹而迷者;有急痛壅塞者。’此亦痰迷之症,系急痛所致,不過一時壅蔽,較諸痰迷似輕。”賈母道:“你只說怕不怕,誰同你背醫書呢。”王太醫忙躬身笑說:“不妨,不妨。”賈母道:“果真不妨?”王太醫道:“實在不妨,都在晚生身上。”賈母道:“既如此,請到外面坐,開藥方。若喫好了,我另外預備好謝禮,叫他親自捧來送去磕頭;若耽誤了,打發人去拆了太醫院大堂。”王太醫只躬身笑說:“不敢,不敢。”他原聽了說“另具上等謝禮命寶玉去磕頭”,故滿口說 “不敢”,竟未聽見賈母后來說拆太醫院之戲語,猶說“不敢”,賈母與衆人反倒笑了。一時,按方煎了藥來服下,果覺比先安靜。無奈寶玉只不肯放紫鵑,只說他去了便是要回蘇州去了。賈母王夫人無法,只得命紫鵑守著他,另將琥珀去伏侍黛玉。
\end{parag}


\begin{parag}
    黛玉不時遣雪雁來探消息,這邊事務盡知,自己心中暗歎。幸喜衆人都知寶玉原有些呆氣,自幼是他二人親密。如今紫鵑之戲語亦是常情,寶玉之病亦非罕事,因不疑到別事去。
\end{parag}


\begin{parag}
    晚間寶玉稍安,賈母王夫人等方回房去。一夜還遣人來問訊幾次。李奶母帶領宋嬤嬤等幾個年老人用心看守,紫鵑、襲人、晴雯等日夜相伴。有時寶玉睡去,必從夢中驚醒,不是哭了說黛玉已去,便是有人來接。每一驚時,必得紫鵑安慰一番方罷。彼時賈母又命將祛邪守靈丹及開竅通神散各樣上方祕製諸藥,按方飲服。次日又服了王太醫藥,漸次好起來。寶玉心下明白,因恐紫鵑回去,故有時或作佯狂之態。紫鵑自那日也著實後悔,如今日夜辛苦,並沒有怨意。襲人等皆心安神定,因向紫鵑笑道:“都是你鬧的,還得你來治。也沒見我們這呆子聽了風就是雨,往後怎麼好。”暫且按下。
\end{parag}


\begin{parag}
    因此時湘雲之症已愈,天天過來瞧看,見寶玉明白了,便將他病中狂態形容了與他瞧,引的寶玉自己伏枕而笑。原來他起先那樣竟是不知的,如今聽人說還不信。無人時紫鵑在側,寶玉又拉他的手問道:“你爲什麼唬我?”紫鵑道:“不過是哄你頑的,你就認真了。”寶玉道:“你說的那樣有情有理,如何是頑話。”紫鵑笑道:“那些頑話都是我編的。林家實沒了人口,縱有也是極遠的。族中也都不在蘇州住,各省流寓不定。縱有人來接,老太太必不放去的。”寶玉道:“便老太太放去,我也不依。”紫鵑笑道:“果真的你不依?只怕是口裏的話。你如今也大了,連親也定下了,過二三年再娶了親,你眼裏還有誰了?”寶玉聽了,又驚問: “誰定了親?定了誰?”紫鵑笑道:“年裏我聽見老太太說,要定下琴姑娘呢。不然那麼疼他?”寶玉笑道:“人人只說我傻,你比我更傻。不過是句頑話,他已經許給梅翰林家了。果然定下了他,我還是這個形景了?先是我發誓賭咒砸這勞什子,你都沒勸過,說我瘋的?剛剛的這幾日纔好了,你又來慪我。”一面說,一面咬牙切齒的,又說道:“我只願這會子立刻我死了,把心迸出來你們瞧見了,然後連皮帶骨一概都化成一股灰,──灰還有形跡,不如再化一股煙,──煙還可凝聚,人還看見,須得一陣大亂風吹的四面八方都登時散了,這纔好!”一面說,一面又滾下淚來。紫鵑忙上來握他的嘴,替他擦眼淚,又忙笑解說道:“你不用著急。這原是我心裏著急,故來試你。”寶玉聽了,更又詫異,問道:“你又著什麼急?”紫鵑笑道:“你知道,我並不是林家的人,我也和襲人鴛鴦是一夥的,偏把我給了林姑娘使。偏生他又和我極好,比他蘇州帶來的還好十倍,一時一刻我們兩個離不開。我如今心裏卻愁,他倘或要去了,我必要跟了他去的。我是閤家在這裏,我若不去,辜負了我們素日的情常;若去,又棄了本家。所以我疑惑,故設出這謊話來問你,誰知你就傻鬧起來。”寶玉笑道:“原來是你愁這個,所以你是傻子。從此後再別愁了。我只告訴你一句躉話:活著,咱們一處活著;不活著,咱們一處化灰化煙。如何?”紫鵑聽了,心下暗暗籌劃。忽有人回:“環爺蘭哥兒問候。”寶玉道:“就說難爲他們,我才睡了,不必進來。” 婆子答應去了。紫鵑笑道:“你也好了,該放我回去瞧瞧我們那一個去了。”寶玉道:“正是這話。我昨日就要叫你去的,偏又忘了。我已經大好了,你就去罷。” 紫鵑聽說,方打疊鋪蓋妝奩之類。寶玉笑道:“我看見你文具裏頭有三兩面鏡子,你把那面小菱花的給我留下罷。我擱在枕頭旁邊,睡著好照,明兒出門帶著也輕巧。”紫鵑聽說,只得與他留下。先命人將東西送過去,然後別了衆人,自回瀟湘館來。
\end{parag}


\begin{parag}
    林黛玉近日聞得寶玉如此形景,未免又添些病症,多哭幾場。今見紫鵑來了,問其原故,已知大愈,仍遣琥珀去伏侍賈母。夜間人定後,紫鵑已寬衣臥下之時,悄向黛玉笑道:“寶玉的心倒實,聽見咱們去就那樣起來。”黛玉不答。紫鵑停了半晌,自言自語的說道:“一動不如一靜。我們這裏就算好人家,別的都容易,最難得的是從小兒一處長大,脾氣情性都彼此知道的了。”黛玉啐道:“你這幾天還不乏,趁這會子不歇一歇,還嚼什麼蛆。”紫鵑笑道:“倒不是白嚼蛆,我倒是一片真心爲姑娘。替你愁了這幾年了,無父母無兄弟,誰是知疼著熱的人?趁早兒老太太還明白硬朗的時節,作定了大事要緊。俗語說‘老健春寒秋後熱’,倘或老太太一時有個好歹,那時雖也完事,只怕耽誤了時光,還不得趁心如意呢。公子王孫雖多,那一個不是三房五妾,今兒朝東,明兒朝西?要一個天仙來,也不過三夜五夕,也丟在脖子後頭了,甚至於爲妾爲丫頭反目成仇的。若孃家有人有勢的還好些,若是姑娘這樣的人,有老太太一日還好一日,若沒了老太太,也只是憑人去欺負了。所以說,拿主意要緊。姑娘是個明白人,豈不聞俗語說:‘萬兩黃金容易得,知心一個也難求’。”黛玉聽了,便說道:“這丫頭今兒不瘋了?怎麼去了幾日,忽然變了一個人。我明兒必回老太太退回去,我不敢要你了。”紫鵑笑道:“我說的是好話,不過叫你心裏留神,並沒叫你去爲非作歹,何苦回老太太,叫我吃了虧,又有何好處?”說著,竟自睡了。黛玉聽了這話,口內雖如此說,心內未嘗不傷感,待他睡了,便直泣了一夜,至天明方打了一個盹兒。次日勉強盥漱了,吃了些燕窩粥,便有賈母等親來看視了,又囑咐了許多話。
\end{parag}


\begin{parag}
    目今是薛姨媽的生日,自賈母起,諸人皆有祝賀之禮。黛玉亦早備了兩色針線送去。是日也定了一本小戲請賈母王夫人等,獨有寶玉與黛玉二人不曾去得。至散時,賈母等順路又瞧他二人一遍,方回房去。次日,薛姨媽家又命薛蝌陪諸夥計吃了一天酒,連忙了三四天方完備。
\end{parag}


\begin{parag}
    因薛姨媽看見邢岫煙生得端雅穩重,且家道貧寒,是個釵荊裙布的女兒,便欲說與薛蟠爲妻。因薛蟠素習行止浮奢,又恐糟塌人家的女兒。正在躊躇之際,忽想起薛蝌未娶,看他二人恰是一對天生地設的夫妻,因謀之於鳳姐兒。鳳姐兒嘆道:“姑媽素知我們太太有些左性的,這事等我慢謀。”因賈母去瞧鳳姐兒時,鳳姐兒便和賈母說:“薛姑媽有件事求老祖宗,只是不好啓齒的。”賈母忙問何事,鳳姐便將求親一事說了。賈母笑道:“這有什麼不好啓齒?這是極好的事。等我和你婆婆說了,怕他不依?”因回房來,即刻就命人來請邢夫人過來,硬作保山。邢夫人想了一想:薛家根基不錯,且現今大富,薛蝌生得又好,且賈母硬作保山,將計就計便應了。賈母十分喜歡,忙命人請了薛姨媽來。二人見了,自然有許多謙辭。邢夫人即刻命人去告訴邢忠夫婦。他夫婦原是此來投靠邢夫人的,如何不依,早極口的說妙極。賈母笑道:“我愛管個閒事,今兒又管成了一件事,不知得多少謝媒錢?”薛姨媽笑道:“這是自然的。縱抬了十萬銀子來,只怕不希罕。但只一件,老太太既是主親,還得一位纔好。”賈母笑道:“別的沒有,我們家折腿爛手的人還有兩個。”說著,便命人去叫過尤氏婆媳二人來。賈母告訴他原故,彼此忙都道喜。賈母吩咐道:“咱們家的規矩你是盡知的,從沒有兩親家爭禮爭面的。如今你算替我在當中料理,也不可太嗇,也不可太費,把他兩家的事周全了回我。”尤氏忙答應了。薛姨媽喜之不盡,回家來忙命寫了請帖補送過寧府。尤氏深知邢夫人情性,本不欲管,無奈賈母親自囑咐,只得應了。惟有忖度邢夫人之意行事。薛姨媽是個無可無不可的人,倒還易說。這且不在話下。
\end{parag}


\begin{parag}
    如今薛姨媽既定了邢岫煙爲媳,合宅皆知。邢夫人本欲接出岫煙去住,賈母因說:“這又何妨,兩個孩子又不能見面,就是姨太太和他一個大姑,一個小姑,又何妨?況且都是女兒,正好親香呢。”邢夫人方罷。
\end{parag}


\begin{parag}
    蝌岫二人前次途中皆曾有一面之遇,大約二人心中也皆如意。只是邢岫煙未免比先時拘泥了些,不好與寶釵姊妹共處閒語;又兼湘雲是個愛取戲的,更覺不好意思。幸他是個知書達禮的,雖有女兒身分,還不是那種佯羞詐愧一味輕薄造作之輩。寶釵自見他時,見他家業貧寒,二則別人之父母皆年高有德之人,獨他父母偏是酒糟透之人,於女兒分中平常;邢夫人也不過是臉面之情,亦非真心疼愛;且岫煙爲人雅重,迎春是個有氣的死人,連他自己尚未照管齊全,如何能照管到他身上,凡閨閣中家常一應需用之物,或有虧乏,無人照管,他又不與人張口,寶釵倒暗中每相體貼接濟,也不敢與邢夫人知道,亦恐多心閒話之故耳。如今卻出人意料之外奇緣作成這門親事。岫煙心中先取中寶釵,然後方取薛蝌。有時岫煙仍與寶釵閒話,寶釵仍以姊妹相呼。
\end{parag}


\begin{parag}
    這日寶釵因來瞧黛玉,恰值岫煙也來瞧黛玉,二人在半路相遇。寶釵含笑喚他到跟前,二人同走至一塊石壁後,寶釵笑問他:“這天還冷的很,你怎麼倒全換了夾的?”岫煙見問,低頭不答。寶釵便知道又有了原故,因又笑問道:“必定是這個月的月錢又沒得。鳳丫頭如今也這樣沒心沒計了。”岫煙道:“他倒想著不錯日子給,因姑媽打發人和我說,一個月用不了二兩銀子,叫我省一兩給爹媽送出去,要使什麼,橫豎有二姐姐的東西,能著些兒搭著就使了。姐姐想,二姐姐也是個老實人,也不大留心,我使他的東西,他雖不說什麼,他那些媽媽丫頭,那一個是省事的,那一個是嘴裏不尖的?我雖在那屋裏,卻不敢很使他們,過三天五天,我倒得拿出錢來給他們打酒買點心喫纔好。因一月二兩銀子還不夠使,如今又去了一兩。前兒我悄悄的把綿衣服叫人當了幾吊錢盤纏。”寶釵聽了,愁眉嘆道:“偏梅家又閤家在任上,後年才進來。若是在這裏,琴兒過去了,好再商議你這事。離了這裏就完了。如今不先定了他妹妹的事,也斷不敢先娶親的。如今倒是一件難事。再遲兩年,又怕你熬煎出病來。等我和媽再商議,有人欺負你,你只管耐些煩兒,千萬別自己熬煎出病來。不如把那一兩銀子明兒也越性給了他們,倒都歇心。你以後也不用白給那些人東西喫,他尖刺讓他們去尖刺,很聽不過了,各人走開。倘或短了什麼,你別存那小家兒女氣,只管找我去。並不是作親後方如此,你一來時咱們就好的。便怕人閒話,你打發小丫頭悄悄的和我說去說是了。”岫煙低頭答應了。寶釵又指他裙上一個碧玉珮問道:“這是誰給你的?”岫煙道:“這是三姐姐給的。”寶釵點頭笑道:“他見人人皆有,獨你一個沒有,怕人笑話,故此送你一個。這是他聰明細緻之處。但還有一句話你也要知道,這些妝飾原出於大官富貴之家的小姐,你看我從頭至腳可有這些富麗閒妝?然七八年之先,我也是這樣來的,如今一時比不得一時了,所以我都自己該省的就省了。將來你這一到了我們家,這些沒有用的東西,只怕還有一箱子。咱們如今比不得他們了,總要一色從實守分爲主,不比他們纔是。”岫煙笑道:“姐姐既這樣說,我回去摘了就是了。”寶釵忙笑道:“你也太聽說了。這是他好意送你,你不佩著,他豈不疑心。我不過是偶然提到這裏,以後知道就是了。”岫煙忙又答應,又問:“姐姐此時那裏去?”寶釵道:“我到瀟湘館去。你且回去把那當票叫丫頭送來,我那裏悄悄的取出來,晚上再悄悄的送給你去,早晚好穿,不然風扇了事大。但不知當在那裏了?”岫煙道: “叫作‘恆舒典’,是鼓樓西大街的。”寶釵笑道:“這鬧在一家去了。夥計們倘或知道了,好說‘人沒過來,衣裳先過來’了。”岫煙聽說,便知是他家的本錢,也不覺紅了臉一笑,二人走開。
\end{parag}


\begin{parag}
    寶釵就往瀟湘館來。正值他母親也來瞧黛玉,正說閒話呢。寶釵笑道:“媽多早晚來的?我竟不知道。”薛姨媽道:“我這幾天連日忙,總沒來瞧瞧寶玉和他。所以今兒瞧他二個,都也好了。”黛玉忙讓寶釵坐了,因向寶釵道:“天下的事真是人想不到的,怎麼想的到姨媽和大舅母又作一門親家。”薛姨媽道:“我的兒,你們女孩家那裏知道,自古道:‘千里姻緣一線牽’。管姻緣的有一位月下老人,預先註定,暗裏只用一根紅絲把這兩個人的腳絆住,憑你兩家隔著海,隔著國,有世仇的,也終久有機會作了夫婦。這一件事都是出人意料之外,憑父母本人都願意了,或是年年在一處的,以爲是定了的親事,若月下老人不用紅線拴的,再不能到一處。比如你姐妹兩個的婚姻,此刻也不知在眼前,也不知在山南海北呢。”寶釵道:“惟有媽,說動話就拉上我們。”一面說,一面伏在他母親懷裏笑說:“咱們走罷。”黛玉笑道:“你瞧,這麼大了,離了姨媽他就是個最老道的,見了姨媽他就撒嬌兒。”薛姨媽用手摩弄著寶釵,嘆向黛玉道:“你這姐姐就和鳳哥兒在老太太跟前一樣,有了正經事就和他商量,沒了事幸虧他開開我的心。我見了他這樣,有多少愁不散的。”黛玉聽說,流淚嘆道:“他偏在這裏這樣,分明是氣我沒孃的人,故意來刺我的眼。”寶釵笑道:“媽瞧他輕狂,倒說我撒嬌兒。”薛姨媽道:“也怨不得他傷心,可憐沒父母,到底沒個親人。”又摩娑黛玉笑道:“好孩子別哭。你見我疼你姐姐你傷心了,你不知我心裏更疼你呢。你姐姐雖沒了父親,到底有我,有親哥哥,這就比你強了。我每每和你姐姐說,心裏很疼你,只是外頭不好帶出來的。你這裏人多口雜,說好話的人少,說歹話的人多,不說你無依無靠,爲人作人配人疼,只說我們看老太太疼你了,我們也洑上水去了。”黛玉笑道:“姨媽既這麼說,我明日就認姨媽做娘,姨媽若是棄嫌不認,便是假意疼我了。”薛姨媽道:“你不厭我,就認了纔好。”寶釵忙道:“認不得的。”黛玉道: “怎麼認不得?”寶釵笑問道:“我且問你,我哥哥還沒定親事,爲什麼反將邢妹妹先說與我兄弟了,是什麼道理?”黛玉道:“他不在家,或是屬相生日不對,所以先說與兄弟了。”寶釵笑道:“非也。我哥哥已經相準了,只等來家就下定了,也不必提出人來,我方纔說你認不得娘,你細想去。”說著,便和他母親擠眼兒發笑。黛玉聽了,便也一頭伏在薛姨媽身上,說道:“姨媽不打他我不依。”薛姨媽忙也摟他笑道:“你別信你姐姐的話,他是頑你呢。”寶釵笑道:“真個的,媽明兒和老太太求了他作媳婦,豈不比外頭尋的好?”黛玉便夠上來要抓他,口內笑說:“你越發瘋了。”薛姨媽忙也笑勸,用手分開方罷。又向寶釵道:“連邢女兒我還怕你哥哥糟踏了他,所以給你兄弟說了。別說這孩子,我也斷不肯給他。前兒老太太因要把你妹妹說給寶玉,偏生又有了人家,不然倒是一門好親。前兒我說定了邢女兒,老太太還取笑說:‘我原要說他的人,誰知他的人沒到手,倒被他說了我們的一個去了。’雖是頑話,細想來倒有些意思。我想寶琴雖有了人家,我雖沒人可給,難道一句話也不說。我想著,你寶兄弟老太太那樣疼他,他又生的那樣,若要外頭說去,斷不中意。不如竟把你林妹妹定與他,豈不四角俱全?”林黛玉先還怔怔的,聽後來見說到自己身上,便啐了寶釵一口,紅了臉,拉著寶釵笑道:“我只打你!你爲什麼招出姨媽這些老沒正經的話來?”寶釵笑道:“這可奇了!媽說你,爲什麼打我?”紫鵑忙也跑來笑道:“姨太太既有這主意,爲什麼不和太太說去?”薛姨媽哈哈笑道:“你這孩子,急什麼,想必催著你姑娘出了閣,你也要早些尋一個小女婿去了。”紫鵑聽了,也紅了臉,笑道:“姨太太真個倚老賣老的起來。”說著,便轉身去了。黛玉先罵:“又與你這蹄子什麼相干?”後來見了這樣,也笑起來說:“阿彌陀佛!該,該,該!也臊了一鼻子灰去了!”薛姨媽母女及屋內婆子丫鬟都笑起來。婆子們因也笑道:“姨太太雖是頑話,卻倒也不差呢。到閒了時和老太太一商議,姨太太竟做媒保成這門親事是千妥萬妥的。”薛姨媽道:“我一出這主意,老太太必喜歡的。”
\end{parag}


\begin{parag}
    一語未了,忽見湘雲走來,手裏拿著一張當票,口內笑道:“這是個帳篇子?”黛玉瞧了,也不認得。地下婆子們都笑道:“這可是一件奇貨,這個乖可不是白教人的。”寶釵忙一把接了,看時,就是岫煙才說的當票,忙折了起來。薛姨媽忙說:“那必定是那個媽媽的當票子失落了,回來急的他們找。那裏得的?”湘雲道:“什麼是當票子?”衆人都笑道:“真真是個呆子,連個當票子也不知道。”薛姨媽嘆道:“怨不得他,真真是侯門千金,而且又小,那裏知道這個?那裏去有這個?便是家下人有這個,他如何得見?別笑他呆子,若給你們家的小姐們看了,也都成了呆子。”衆婆子笑道:“林姑娘方纔也不認得,別說姑娘們。此刻寶玉他倒是外頭常走出去的,只怕也還沒見過呢。” 薛姨媽忙將原故講明。湘雲黛玉二人聽了方笑道:“原來爲此。人也太會想錢了,姨媽家的當鋪也有這個不成?”衆人笑道:“這又呆了。‘天下老鴰一般黑’,豈有兩樣的?”薛姨媽因又問是那裏拾的?湘雲方欲說時,寶釵忙說:“是一張死了沒用的,不知那年勾了帳的,香菱拿著哄他們頑的。”薛姨媽聽了此話是真,也就不問了。一時人來回:“那府裏大奶奶過來請姨太太說話呢。”薛姨媽起身去了。
\end{parag}


\begin{parag}
    這裏屋內無人時,寶釵方問湘云何處拾的。湘雲笑道:“我見你令弟媳的丫頭篆兒悄悄的遞與鶯兒。鶯兒便隨手夾在書裏,只當我沒看見。我等他們出去了,我偷著看,竟不認得。知道你們都在這裏,所以拿來大家認認。”黛玉忙問:“怎麼,他也當衣裳不成?既當了,怎麼又給你去?”寶釵見問,不好隱瞞他兩個,遂將方纔之事都告訴了他二人。黛玉便說“兔死狐悲,物傷其類”,不免感嘆起來。史湘雲便動了氣說:“等我問著二姐姐去!我罵那起老婆子丫頭一頓,給你們出氣何如?”說著,便要走。寶釵忙一把拉住,笑道:“你又發瘋了,還不給我坐著呢。”黛玉笑道:“你要是個男人,出去打一個報不平兒。你又充什麼荊軻聶政,真真好笑。”湘雲道:“既不叫我問他去,明兒也把他接到咱們苑裏一處住去,豈不好?”寶釵笑道:“明日再商量。”說著,人報:“三姑娘四姑娘來了。”三人聽了,忙掩了口不提此事。要知端的,且聽下回分解。
\end{parag}


\begin{parag}
    \begin{note}寫寶玉黛玉呼吸相關,不在字裏行間,全從無字句處,運鬼斧神工之筆,攝魄追魂,令我哭一回、嘆一回,渾身都是呆氣。\end{note}
\end{parag}


\begin{parag}
    \begin{note}寫寶釵岫煙相敘一段,真有英雄失路之悲,真有知己相逢之樂。時方午夜,燈影幢幢,讀書至此,掩卷出戶,見星月依稀,寒風微起,默立階除良久。\end{note}
\end{parag}
