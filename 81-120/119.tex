\chap{一百一十九}{中鄉魁寶玉卻塵緣 沐皇恩賈家延世澤}



\begin{parag}
    話說鶯兒見寶玉說話摸不着頭腦,正自要走,只聽寶玉又說道:“傻丫頭,我告訴你罷。你姑娘既是有造化的,你跟着他自然也是有造化的了。你襲人姐姐是靠不住的。只要往後你盡心伏侍他就是了。日後或有好處,也不枉你跟着他熬了一場。”鶯兒聽了前頭象話,後頭說的又有些不象了,便道:“我知道了。姑娘還等我呢。二爺要喫果子時,打發小丫頭叫我就是了。”寶玉點頭,鶯兒纔去了。一時寶釵襲人回來,各自房中去了。不題。
\end{parag}


\begin{parag}
    且說過了幾天便是場期,別人只知盼望他爺兒兩個作了好文章便可以高中的了,只有寶釵見寶玉的功課雖好,只是那有意無意之間,卻別有一種冷靜的光景。知他要進場了,頭一件,叔侄兩個都是初次赴考,恐人馬擁擠有什麼失閃,第二件,寶玉自和尚去後總不出門,雖然見他用功喜歡,只是改的太速太好了,反倒有些信不及,只怕又有什麼變故。所以進場的頭一天,一面派了襲人帶了小丫頭們同着素雲等給他爺兒兩個收拾妥當,自己又都過了目,好好的擱起預備着,一面過來同李紈回了王夫人,揀家裏的老成管事的多派了幾個,只說怕人馬擁擠碰了。
\end{parag}


\begin{parag}
    次日寶玉賈蘭換了半新不舊的衣服,欣然過來見了王夫人。王夫人囑咐道:“你們爺兒兩個都是初次下場,但是你們活了這麼大,並不曾離開我一天。就是不在我眼前,也是丫鬟媳婦們圍着,何曾自己孤身睡過一夜。今日各自進去,孤孤悽悽,舉目無親,須要自己保重。早些作完了文章出來,找着家人早些回來,也叫你母親媳婦們放心。”王夫人說着不免傷心起來。賈蘭聽一句答應一句。只見寶玉一聲不哼,待王夫人說完了,走過來給王夫人跪下,滿眼流淚,磕了三個頭,說道:“母親生我一世,我也無可答報,只有這一入場用心作了文章,好好的中個舉人出來。那時太太喜歡喜歡,便是兒子一輩的事也完了,一輩子的不好也都遮過去了。”王夫人聽了,更覺傷心起來,便道:“你有這個心自然是好的,可惜你老太太不能見你的面了!”一面說,一面拉他起來。那寶玉只管跪着不肯起來,便說道:“老太太見與不見,總是知道的,喜歡的,既能知道了,喜歡了,便不見也和見了的一樣。只不過隔了形質,並非隔了神氣啊。”李紈見王夫人和他如此,一則怕勾起寶玉的病來,二則也覺得光景不大吉祥,連忙過來說道:“太太,這是大喜的事,爲什麼這樣傷心?況且寶兄弟近來很知好歹,很孝順,又肯用功,只要帶了侄兒進去好好的作文章,早早的回來,寫出來請咱們的世交老先生們看了,等着爺兒兩個都報了喜就完了。”一面叫人攙起寶玉來。寶玉卻轉過身來給李紈作了個揖,說:“嫂子放心。我們爺兒兩個都是必中的。日後蘭哥還有大出息,大嫂子還要帶鳳冠穿霞帔呢。”李紈笑道:“但願應了叔叔的話,也不枉——”說到這裏,恐怕又惹起王夫人的傷心來,連忙嚥住了。寶玉笑道:“只要有了個好兒子能夠接續祖基,就是大哥哥不能見,也算他的後事完了。”李紈見天氣不早了,也不肯盡着和他說話,只好點點頭兒。此時寶釵聽得早已呆了,這些話不但寶玉,便是王夫人李紈所說,句句都是不祥之兆,卻又不敢認真,只得忍淚無言。寶玉走到跟前,深深的作了一個揖。衆人見他行事古怪,也摸不着是怎麼樣,又不敢笑他。只見寶釵的眼淚直流下來。衆人更是納罕。又聽寶玉說道:“姐姐,我要走了,你好生跟着太太聽我的喜信兒罷。”寶釵道:“是時候了,你不必說這些嘮叨話了。”寶玉道:“你倒催的我緊,我自己也知道該走了。”回頭見衆人都在這裏,只沒惜春紫鵑,便說道:“四妹妹和紫鵑姐姐跟前替我說一句罷,橫豎是再見就完了。”衆人見他的話又象有理,又象瘋話。大家只說他從沒出過門,都是太太的一套話招出來的,不如早早催他去了就完了事了,便說道:“外面有人等你呢,你再鬧就誤了時辰了。”寶玉仰面大笑道:“走了,走了!不用胡鬧了,完了事了!”衆人也都笑道:“快走罷。”獨有王夫人和寶釵孃兒兩個倒象生離死別的一般,那眼淚也不知從那裏來的,直流下來,幾乎失聲哭出。但見寶玉嘻天哈地,大有瘋傻之狀,遂從此出門走了。正是:
\end{parag}


\begin{poem}
    \begin{pl}
        走求名利無雙地,打出樊籠第一關。
    \end{pl}
\end{poem}


\begin{parag}
    不言寶玉賈蘭出門赴考。且說賈環見他們考去,自己又氣又恨,便自大爲王說:“我可要給母親報仇了。家裏一個男人沒有,上頭大太太依了我,還怕誰!”想定了主意,跑到邢夫人那邊請了安,說了些奉承的話。那邢夫人自然喜歡,便說道:“你這纔是明理的孩子呢。象那巧姐兒的事,原該我做主的,你璉二哥糊塗,放着親奶奶,倒託別人去!”賈環道:“人家那頭兒也說了,只認得這一門子。現在定了,還要備一分大禮來送太太呢。如今太太有了這樣的藩王孫女婿兒,還怕大老爺沒大官做麼!不是我說自己的太太,他們有了元妃姐姐,便欺壓的人難受。將來巧姐兒別也是這樣沒良心,等我去問問他。”邢夫人道:“你也該告訴他,他纔知道你的好處。只怕他父親在家也找不出這麼門子好親事來!但只平兒那個糊塗東西,他倒說這件事不好,說是你太太也不願意。想來恐怕我們得了意。若遲了你二哥回來,又聽人家的話,就辦不成了。”賈環道:“那邊都定了,只等太太出了八字。王府的規矩,三天就要來娶的。但是一件,只怕太太不願意,那邊說是不該娶犯官的孫女,只好悄悄的抬了去,等大老爺免了罪做了官,再大家熱鬧起來。”邢夫人道:“這有什麼不願意,也是禮上應該的。”賈環道:“既這麼着,這帖子太太出了就是了。”邢夫人道:“這孩子又糊塗了,裏頭都是女人,你叫芸哥兒寫了一個就是了。”賈環聽說,喜歡的了不得,連忙答應了出來,趕着和賈芸說了,邀着王仁到那外藩公館立文書兌銀子去了。
\end{parag}


\begin{parag}
    那知剛纔所說的話,早被跟邢夫人的丫頭聽見。那丫頭是求了平兒纔挑上的,便抽空兒趕到平兒那裏,一五一十的都告訴了。平兒早知此事不好,已和巧姐細細的說明。巧姐哭了一夜,必要等他父親回來作主,大太太的話不能遵。今兒又聽見這話,便大哭起來,要和太太講去。平兒急忙攔住道:“姑娘且慢着。大太太是你的親祖母,他說二爺不在家,大太太做得主的,況且還有舅舅做保山。他們都是一氣,姑娘一個人那裏說得過呢。我到底是下人,說不上話去。如今只可想法兒,斷不可冒失的。”邢夫人那邊的丫頭道:“你們快快的想主意,不然可就要抬走了。”說着,各自去了。平兒回過頭來見巧姐哭作一團,連忙扶着道:“姑娘,哭是不中用的,如今是二爺夠不着,聽見他們的話頭——”這句話還沒說完,只見邢夫人那邊打發人來告訴:“姑娘大喜的事來了。叫平兒將姑娘所有應用的東西料理出來。若是賠送呢,原說明了等二爺回來再辦。”平兒只得答應了。
\end{parag}


\begin{parag}
    回來又見王夫人過來,巧姐兒一把抱住,哭得倒在懷裏。王夫人也哭道:“妞兒不用着急,我爲你吃了大太太好些話,看來是扭不過來的。我們只好應着緩下去,即刻差個家人趕到你父親那裏去告訴。”平兒道:“太太還不知道麼?早起三爺在大太太跟前說了,什麼外藩規矩三日就要過去的。如今大太太已叫芸哥兒寫了名字年庚去了,還等得二爺麼?”王夫人聽說是“三爺”,便氣得說不出話來,呆了半天,一迭聲叫人找賈環。找了半日,人回:“今早同薔哥兒王舅爺出去了。”王夫人問:“芸哥呢?”衆人回說不知道。巧姐屋內人人瞪眼,一無方法。王夫人也難和邢夫人爭論,只有大家抱頭大哭。
\end{parag}


\begin{parag}
    有個婆子進來,回說:“後門上的人說,那個劉姥姥又來了。”王夫人道:“咱們家遭着這樣事,那有工夫接待人。不拘怎麼回了他去罷。”平兒道:“太太該叫他進來,他是姐兒的乾媽,也得告訴告訴他。”王夫人不言語,那婆子便帶了劉姥姥進來。各人見了問好。劉姥姥見衆人的眼圈兒都是紅的,也摸不着頭腦,遲了一會子,便問道:“怎麼了?太太姑娘們必是想二姑奶奶了。”巧姐兒聽見提起他母親,越發大哭起來。平兒道:“姥姥別說閒話,你既是姑娘的乾媽,也該知道的。”便一五一十的告訴了。把個劉姥姥也唬怔了,等了半天,忽然笑道:“你這樣一個伶俐姑娘,沒聽見過鼓兒詞麼,這上頭的方法多着呢。這有什麼難的。”平兒趕忙問道:“姥姥你有什麼法兒快說罷。”劉姥姥道:“這有什麼難的呢,一個人也不叫他們知道,扔崩一走,就完了事了。”平兒道:“這可是混說了。我們這樣人家的人,走到那裏去!”劉姥姥道:“只怕你們不走,你們要走,就到我屯裏去。我就把姑娘藏起來,即刻叫我女婿弄了人,叫姑娘親筆寫個字兒,趕到姑老爺那裏,少不得他就來了。可不好麼?”平兒道:“大太太知道呢?”劉姥姥道:“我來他們知道麼?”平兒道:“大太太住在後頭,他待人刻薄,有什麼信沒有送給他的。你若前門走來就知道了,如今是後門來的,不妨事。”劉姥姥道:“咱們說定了幾時,我叫女婿打了車來接了去。”平兒道:“這還等得幾時呢,你坐着罷。”急忙進去,將劉姥姥的話避了旁人告訴了。王夫人想了半天不妥當。平兒道:“只有這樣。爲的是太太纔敢說明,太太就裝不知道,回來倒問大太太。我們那裏就有人去,想二爺回來也快。”王夫人不言語,嘆了一口氣。巧姐兒聽見,便和王夫人道:“只求太太救我,橫豎父親回來只有感激的。”平兒道:“不用說了,太太回去罷。回來只要太太派人看屋子。”王夫人道:“掩密些。你們兩個人的衣服鋪蓋是要的。”平兒道:“要快走了纔中用呢,若是他們定了,回來就有了饑荒了。”一句話提醒了王夫人,便道:“是了,你們快辦去罷,有我呢。”於是王夫人回去,倒過去找邢夫人說閒話兒,把邢夫人先絆住了。平兒這裏便遣人料理去了,囑咐道:“倒別避人,有人進來看見,就說是大太太吩咐的,要一輛車子送劉姥姥去。”這裏又買囑了看後門的人僱了車來。平兒便將巧姐裝做青兒模樣,急急的去了。後來平兒只當送人,眼錯不見,也跨上車去了。原來近日賈府後門雖開,只有一兩個人看着,餘外雖有幾個家下人,因房大人少,空落落的,誰能照應。且邢夫人又是個不憐下人的,衆人明知此事不好,又都感念平兒的好處,所以通同一氣放走了巧姐。邢夫人還自和王夫人說話,那裏理會。只有王夫人甚不放心,說了一回話,悄悄的走到寶釵那裏坐下,心裏還是惦記着。寶釵見王夫人神色恍惚,便問:“太太的心裏有什麼事?”王夫人將這事背地裏和寶釵說了。寶釵道:“險得很!如今得快快兒的叫芸哥兒止住那裏纔妥當。”王夫人道:“我找不着環兒呢。”寶釵道:“太太總要裝作不知,等我想個人去叫大太太知道纔好。”王夫人點頭,一任寶釵想人。暫且不言。
\end{parag}


\begin{parag}
    且說外藩原是要買幾個使喚的女人,據媒人一面之辭,所以派人相看。相看的人回去稟明瞭藩王。藩王問起人家,衆人不敢隱瞞,只得實說。那外藩聽了,知是世代勳戚,便說:“了不得!這是有幹例禁的,幾乎誤了大事!況我朝覲已過,便要擇日起程,倘有人來再說,快快打發出去。”這日恰好賈芸王仁等遞送年庚,只見府門裏頭的人便說:“奉王爺的命,再敢拿賈府的人來冒充民女者,要拿住究治的。如今太平時候,誰敢這樣大膽!”這一嚷,唬得王仁等抱頭鼠竄的出來,埋怨那說事的人,大家掃興而散。
\end{parag}


\begin{parag}
    賈環在家候信,又聞王夫人傳喚,急得煩燥起來。見賈芸一人回來,趕着問道:“定了麼?”賈芸慌忙跺足道:“了不得,了不得!不知誰露了風了!”還把喫虧的話說了一遍。賈環氣得發怔說:“我早起在大太太跟前說的這樣好,如今怎麼樣處呢?這都是你們衆人坑了我了!”正沒主意,聽見裏頭亂嚷,叫着賈環等的名字說:“大太太二太太叫呢。”兩個人只得蹭進去。只見王夫人怒容滿面說:“你們乾的好事!如今逼死了巧姐和平兒了,快快的給我找還屍首來完事!”兩個人跪下。賈環不敢言語,賈芸低頭說道:“孫子不敢幹什麼,爲的是邢舅太爺和王舅爺說給巧妹妹作媒,我們纔回太太們的。大太太願意,纔叫孫子寫帖兒去的。人家還不要呢。怎麼我們逼死了妹妹呢!”王夫人道:“環兒在大太太那裏說的,三日內便要抬了走。說親作媒有這樣的麼!我也不問你們,快把巧姐兒還了我們,等老爺回來再說。”邢夫人如今也是一句話兒說不出了,只有落淚。王夫人便罵賈環說:“趙姨娘這樣混賬的東西,留的種子也是這混賬的!”說着,叫丫頭扶了回到自己房中。
\end{parag}


\begin{parag}
    那賈環賈芸邢夫人三個人互相埋怨,說道:“如今且不用埋怨,想來死是不死的,必是平兒帶了他到那什麼親戚家躲着去了。”邢夫人叫了前後的門人來罵着,問巧姐兒和平兒知道那裏去了。豈知下人一口同音說是:“大太太不必問我們,問當家的爺們就知道了。在大太太也不用鬧,等我們太太問起來我們有話說。要打大家打,要發大家都發。自從璉二爺出了門,外頭鬧的還了得!我們的月錢月米是不給了,賭錢喝酒鬧小旦,還接了外頭的媳婦兒到宅裏來。這不是爺嗎。”說得賈芸等頓口無言。王夫人那邊又打發人來催說:“叫爺們快找來。”那賈環等急得恨無地縫可鑽,又不敢盤問巧姐那邊的人。明知衆人深恨,是必藏起來了。但是這句話怎敢在王夫人面前說。只得各處親戚家打聽,毫無蹤跡。裏頭一個邢夫人,外頭環兒等,這幾天鬧的晝夜不寧。
\end{parag}


\begin{parag}
    看看到了出場日期,王夫人只盼着寶玉、賈蘭回來。等到晌午,不見回來,王夫人李紈寶釵着忙,打發人去到下處打聽。去了一起,又無消息,連去的人也不來了。回來又打發一起人去,又不見回來。三個人心裏如熱油熬煎,等到傍晚有人進來,見是賈蘭。衆人喜歡問道:“寶二叔呢?”賈蘭也不及請安,便哭道:“二叔丟了。”王夫人聽了這話便怔了,半天也不言語,便直挺挺的躺倒牀上。虧得彩雲等在後面扶着,下死的叫醒轉來哭着。見寶釵也是白瞪兩眼。襲人等已哭得淚人一般,只有哭着罵賈蘭道:“糊塗東西,你同二叔在一處,怎麼他就丟了?”賈蘭道:“我和二叔在下處,是一處喫一處睡。進了場,相離也不遠,刻刻在一處的。今兒一早,二叔的卷子早完了,還等我呢。我們兩個人一起去交了卷子,一同出來,在龍門口一擠,回頭就不見了。我們家接場的人都問我,李貴還說看見的,相離不過數步,怎麼一擠就不見了。現叫李貴等分頭的找去,我也帶了人各處號裏都找遍了,沒有,我所以這時候纔回來。”王夫人是哭的一句話也說不出來,寶釵心裏已知八九,襲人痛哭不已。賈薔等不等吩咐,也是分頭而去。可憐榮府的人個個死多活少,空備了接場的酒飯。賈蘭也忘卻了辛苦,還要自己找去。倒是王夫人攔住道:“我的兒,你叔叔丟了,還禁得再丟了你麼。好孩子,你歇歇去罷。”賈蘭那裏肯走。尤氏等苦勸不止。衆人中只有惜春心裏卻明白了,只不好說出來,便問寶釵道:“二哥哥帶了玉去了沒有?”寶釵道:“這是隨身的東西,怎麼不帶!”惜春聽了便不言語。襲人想起那日搶玉的事來,也是料着那和尚作怪,柔腸幾斷,珠淚交流,嗚嗚咽咽哭個不住。追想當年寶玉相待的情分,有時慪他,他便惱了,也有一種令人迴心的好處,那溫存體貼是不用說了。若慪急了他,便賭誓說做和尚。那知道今日卻應了這句話!看看那天已覺是四更天氣,並沒有個信兒。李紈又怕王夫人苦壞了,極力的勸着回房。衆人都跟着伺候,只有邢夫人回去。賈環躲着不敢出來。王夫人叫賈蘭去了,一夜無眠。次日天明,雖有家人回來,都說沒有一處不尋到,實在沒有影兒。於是薛姨媽、薛蝌、史湘雲、寶琴、李嬸等,連二連三的過來請安問信。
\end{parag}


\begin{parag}
    如此一連數日,王夫人哭得飲食不進,命在垂危。忽有家人回道:“海疆來了一人,口稱統制大人那裏來的,說我們家的三姑奶奶明日到京了。”王夫人聽說探春回京,雖不能解寶玉之愁,那個心略放了些。到了明日,果然探春回來。衆人遠遠接着,見探春出跳得比先前更好了,服採鮮明。見了王夫人形容枯槁,衆人眼腫腮紅,便也大哭起來,哭了一會,然後行禮。看見惜春道姑打扮,心裏很不舒服。又聽見寶玉心迷走失,家中多少不順的事,大家又哭起來。還虧得探春能言,見解亦高,把話來慢慢兒的勸解了好些時,王夫人等略覺好些。再明兒,三姑爺也來了。知有這樣的事,探春住下勸解。跟探春的丫頭老婆也與衆姐妹們相聚,各訴別後的事。從此上上下下的人,竟是無晝無夜專等寶玉的信。
\end{parag}


\begin{parag}
    那一夜五更多天,外頭幾個家人進來到二門口報喜。幾個小丫頭亂跑進來,也不及告訴大丫頭了,進了屋子便說:“太太奶奶們大喜。”王夫人打諒寶玉找着了,便喜歡的站起身來說:“在那裏找着的,快叫他進來。”那人道:“中了第七名舉人。”王夫人道:“寶玉呢?”家人不言語,王夫人仍舊坐下。探春便問:“第七名中的是誰?”家人回說“是寶二爺。”正說着,外頭又嚷道:“蘭哥兒中了。”那家人趕忙出去接了報單回稟,見賈蘭中了一百三十名。李紈心下喜歡,因王夫人不見了寶玉,不敢喜形於色。王夫人見賈蘭中了,心下也是喜歡,只想:“若是寶玉一回來,咱們這些人不知怎樣樂呢!”獨有寶釵心下悲苦,又不好掉淚。衆人道喜,說是“寶玉既有中的命,自然再不會丟的。況天下那有迷失了的舉人。”王夫人等想來不錯,略有笑容。衆人便趁勢勸王夫人等多進了些飲食。只見三門外頭焙茗亂嚷說:“我們二爺中了舉人,是丟不了的了。”衆人問道:“怎見得呢?”焙茗道:“‘一舉成名天下聞’,如今二爺走到那裏,那裏就知道的。誰敢不送來!”裏頭的衆人都說:“這小子雖是沒規矩,這句話是不錯的。”惜春道:“這樣大人了,那裏有走失的。只怕他勘破世情,入了空門,這就難找着他了。”這句話又招得王夫人等又大哭起來。李紈道:“古來成佛作祖成神仙的,果然把爵位富貴都拋了也多得很。”王夫人哭道:“他若拋了父母,這就是不孝,怎能成佛作祖。”探春道:“大凡一個人不可有奇處。二哥哥生來帶塊玉來,都道是好事,這麼說起來,都是有了這塊玉的不好。若是再有幾天不見,我不是叫太太生氣,就有些原故了,只好譬如沒有生這位哥哥罷了。果然有來頭成了正果,也是太太幾輩子的修積。”寶釵聽了不言語,襲人那裏忍得住,心裏一疼,頭上一暈便栽倒了。王夫人見了可憐,命人扶他回去。賈環見哥哥侄兒中了,又爲巧姐的事大不好意思,只報怨薔芸兩個,知道探春回來,此事不肯干休,又不敢躲開,這幾天竟是如在荊棘之中。
\end{parag}


\begin{parag}
    明日賈蘭只得先去謝恩,知道甄寶玉也中了,大家序了同年。提起賈寶玉心迷走失,甄寶玉嘆息勸慰。知貢舉的將考中的卷子奏聞,皇上一一的披閱,看取中的文章俱是平正通達的。見第七名賈寶玉是金陵籍貫,第一百三十名又是金陵賈蘭,皇上傳旨詢問,兩個姓賈的是金陵人氏,是否賈妃一族。大臣領命出來,傳賈寶玉賈蘭問話,賈蘭將寶玉場後迷失的話並將三代陳明,大臣代爲轉奏。皇上最是聖明仁德,想起賈氏功勳,命大臣查復,大臣便細細的奏明。皇上甚是憫恤,命有司將賈赦犯罪情由查案呈奏。皇上又看到海疆靖寇班師善後事宜一本,奏的是海宴河清、萬民樂業的事。皇上聖心大悅,命九卿敘功議賞,並大赦天下。賈蘭等朝臣散後拜了座師,並聽見朝內有大赦的信,便回了王夫人等。閤家略有喜色,只盼寶玉回來。薛姨媽更加喜歡,便要打算贖罪。
\end{parag}


\begin{parag}
    一日,人報甄老爺同三姑爺來道喜,王夫人便命賈蘭出去接待。不多一回,賈蘭進來笑嘻嘻的回王夫人道:“太太們大喜了。甄老伯在朝內聽見有旨意,說是大老爺的罪名免了,珍大爺不但免了罪,仍襲了寧國三等世職。榮國世職仍是老爺襲了,俟丁憂服滿,仍升工部郎中。所抄家產,全行賞還。二叔的文章,皇上看了甚喜,問知元妃兄弟,北靜王還奏說人品亦好,皇上傳旨召見,衆大臣奏稱據伊侄賈蘭回稱出場時迷失,現在各處尋訪,皇上降旨着五營各衙門用心尋訪。這旨意一下,請太太們放心,皇上這樣聖恩,再沒有找不着了。”王夫人等這纔大家稱賀,喜歡起來。只有賈環等心下着急,四處找尋巧姐。
\end{parag}


\begin{parag}
    那知巧姐隨了劉姥姥帶着平兒出了城,到了莊上,劉姥姥也不敢輕褻巧姐,便打掃上房讓給巧姐平兒住下。每日供給雖是鄉村風味,倒也潔淨。又有青兒陪着,暫且寬心。那莊上也有幾家富戶,知道劉姥姥家來了賈府姑娘,誰不來瞧,都道是天上神仙。也有送菜果的,也有送野味的,到也熱鬧。內中有個極富的人家,姓周,家財鉅萬,良田千頃。只有一子,生得文雅清秀,年紀十四歲,他父母延師讀書,新近科試中了秀才。那日他母親看見了巧姐,心裏羨慕,自想:“我是莊家人家,那能配得起這樣世家小姐!”呆呆的想着。劉姥姥知他心事,拉着他說:“你的心事我知道了,我給你們做個媒罷。”周媽媽笑道:“你別哄我,他們什麼人家,肯給我們莊家人麼。”劉姥姥道:“說着瞧罷。”於是兩人各自走開。
\end{parag}


\begin{parag}
    劉姥姥惦記着賈府,叫板兒進城打聽,那日恰好到寧榮街,只見有好些車轎在那裏。板兒便在鄰近打聽,說是:“寧榮兩府復了官,賞還抄的家產,如今府裏又要起來了。只是他們的寶玉中了官,不知走到那裏去了。”板兒心裏喜歡,便要回去,又見好幾匹馬到來,在門前下馬。只見門上打千兒請安說:“二爺回來了,大喜!大老爺身上安了麼?”那位爺笑着道:“好了。又遇恩旨,就要回來了。”還問:“那些人做什麼的?”門上回說:“是皇上派官在這裏下旨意,叫人領家產。”那位爺便喜歡進去。板兒便知是賈璉了。也不用打聽,趕忙回去告訴了他外祖母。劉姥姥聽說,喜的眉開眼笑,去和巧姐兒賀喜,將板兒的話說了一遍。平兒笑說道:“可不是,虧得姥姥這樣一辦,不然姑娘也摸不着那好時候。”巧姐更自歡喜。正說着,那送賈璉信的人也回來了,說是:“姑老爺感激得很,叫我一到家快把姑娘送回去。又賞了我好幾兩銀子。”劉姥姥聽了得意,便叫人趕了兩輛車,請巧姐平兒上車。巧姐等在劉姥姥家住熟了,反是依依不捨,更有青兒哭着,恨不能留下。劉姥姥知他不忍相別,便叫青兒跟了進城,一徑直奔榮府而來。
\end{parag}


\begin{parag}
    且說賈璉先前知道賈赦病重,趕到配所,父子相見,痛哭了一場,漸漸的好起來。賈璉接着家書,知道家中的事,稟明賈赦回來,走到中途,聽得大赦,又趕了兩天,今日到家,恰遇頒賞恩旨。裏面邢夫人等正愁無人接旨,雖有賈蘭,終是年輕,人報璉二爺回來,大家相見,悲喜交集,此時也不及敘話,即到前廳叩見了欽命大人。問了他父親好,說明日到內府領賞,寧國府第發交居住。衆人起身辭別,賈璉送出門去。見有幾輛屯車,家人們不許停歇,正在吵鬧。賈璉早知道是巧姐來的車,便罵家人道:“你們這班糊塗忘八崽子,我不在家,就欺心害主,將巧姐兒都逼走了。如今人家送來,還要攔阻,必是你們和我有什麼仇麼!”衆家人原怕賈璉回來不依,想來少時纔破,豈知賈璉說得更明,心下不懂,只得站着回道:“二爺出門,奴才們有病的,有告假的,都是三爺、薔大爺、芸大爺作主,不與奴才們相干。”賈璉道:“什麼混賬東西!我完了事再和你們說,快把車趕進來!”
\end{parag}


\begin{parag}
    賈璉進去見邢夫人,也不言語,轉身到了王夫人那裏,跪下磕了個頭,回道:“姐兒回來了,全虧太太。環兄弟太太也不用說他了。只是芸兒這東西,他上回看家就鬧亂兒,如今我去了幾個月,便鬧到這樣。回太太的話,這種人攆了他不往來也使得。”王夫人道:“你大舅子爲什麼也是這樣?”賈璉道:“太太不用說,我自有道理。”正說着,彩雲等回道:“巧姐兒進來了。”見了王夫人,雖然別不多時,想起這樣逃難的景況,不免落下淚來。巧姐兒也便大哭。賈璉謝了劉姥姥。王夫人便拉他坐下,說起那日的話來。賈璉見平兒,外面不好說別的,心裏感激,眼中流淚。自此賈璉心裏愈敬平兒,打算等賈赦等回來要扶平兒爲正。此是後話,暫且不題。
\end{parag}


\begin{parag}
    邢夫人正恐賈璉不見了巧姐,必有一番的周折,又聽見賈璉在王夫人那裏,心下更是着急,便叫丫頭去打聽。回來說是巧姐兒同着劉姥姥在那裏說話,邢夫人纔如夢初覺,知他們的鬼,還抱怨着王夫人“調唆我母子不和,到底是那個送信給平兒的?”正問着,只見巧姐同着劉姥姥帶了平兒,王夫人在後頭跟着進來,先把頭裏的話都說在賈芸王仁身上,說:“大太太原是聽見人說,爲的是好事,那裏知道外頭的鬼。”邢夫人聽了,自覺羞慚。想起王夫人主意不差,心裏也服。於是邢王夫人彼此心下相安。
\end{parag}


\begin{parag}
    平兒回了王夫人,帶了巧姐到寶釵那裏來請安,各自提各自的苦處。又說到“皇上隆恩,咱們家該興旺起來了。想來寶二爺必回來的。”正說到這話,只見秋紋急忙來說:“襲人不好了!”不知何事,且聽下回分解。
\end{parag}