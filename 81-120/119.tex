\chap{一百一十九}{中乡魁宝玉却尘缘 沐皇恩贾家延世泽}



\begin{parag}
    话说莺儿见宝玉说话摸不着头脑,正自要走,只听宝玉又说道:“傻丫头,我告诉你罢。你姑娘既是有造化的,你跟着他自然也是有造化的了。你袭人姐姐是靠不住的。只要往后你尽心伏侍他就是了。日后或有好处,也不枉你跟着他熬了一场。”莺儿听了前头象话,后头说的又有些不象了,便道:“我知道了。姑娘还等我呢。二爷要吃果子时,打发小丫头叫我就是了。”宝玉点头,莺儿纔去了。一时宝钗袭人回来,各自房中去了。不题。
\end{parag}


\begin{parag}
    且说过了几天便是场期,别人只知盼望他爷儿两个作了好文章便可以高中的了,只有宝钗见宝玉的功课虽好,只是那有意无意之间,却别有一种冷静的光景。知他要进场了,头一件,叔侄两个都是初次赴考,恐人马拥挤有什么失闪,第二件,宝玉自和尚去后总不出门,虽然见他用功喜欢,只是改的太速太好了,反倒有些信不及,只怕又有什么变故。所以进场的头一天,一面派了袭人带了小丫头们同着素云等给他爷儿两个收拾妥当,自己又都过了目,好好的搁起预备着,一面过来同李纨回了王夫人,拣家里的老成管事的多派了几个,只说怕人马拥挤碰了。
\end{parag}


\begin{parag}
    次日宝玉贾兰换了半新不旧的衣服,欣然过来见了王夫人。王夫人嘱咐道:“你们爷儿两个都是初次下场,但是你们活了这么大,并不曾离开我一天。就是不在我眼前,也是丫鬟媳妇们围着,何曾自己孤身睡过一夜。今日各自进去,孤孤凄凄,举目无亲,须要自己保重。早些作完了文章出来,找着家人早些回来,也叫你母亲媳妇们放心。”王夫人说着不免伤心起来。贾兰听一句答应一句。只见宝玉一声不哼,待王夫人说完了,走过来给王夫人跪下,满眼流泪,磕了三个头,说道:“母亲生我一世,我也无可答报,只有这一入场用心作了文章,好好的中个举人出来。那时太太喜欢喜欢,便是儿子一辈的事也完了,一辈子的不好也都遮过去了。”王夫人听了,更觉伤心起来,便道:“你有这个心自然是好的,可惜你老太太不能见你的面了!”一面说,一面拉他起来。那宝玉只管跪着不肯起来,便说道:“老太太见与不见,总是知道的,喜欢的,既能知道了,喜欢了,便不见也和见了的一样。只不过隔了形质,并非隔了神气啊。”李纨见王夫人和他如此,一则怕勾起宝玉的病来,二则也觉得光景不大吉祥,连忙过来说道:“太太,这是大喜的事,为什么这样伤心?况且宝兄弟近来很知好歹,很孝顺,又肯用功,只要带了侄儿进去好好的作文章,早早的回来,写出来请咱们的世交老先生们看了,等着爷儿两个都报了喜就完了。”一面叫人搀起宝玉来。宝玉却转过身来给李纨作了个揖,说:“嫂子放心。我们爷儿两个都是必中的。日后兰哥还有大出息,大嫂子还要带凤冠穿霞帔呢。”李纨笑道:“但愿应了叔叔的话,也不枉——”说到这里,恐怕又惹起王夫人的伤心来,连忙咽住了。宝玉笑道:“只要有了个好儿子能够接续祖基,就是大哥哥不能见,也算他的后事完了。”李纨见天气不早了,也不肯尽着和他说话,只好点点头儿。此时宝钗听得早已呆了,这些话不但宝玉,便是王夫人李纨所说,句句都是不祥之兆,却又不敢认真,只得忍泪无言。宝玉走到跟前,深深的作了一个揖。众人见他行事古怪,也摸不着是怎么样,又不敢笑他。只见宝钗的眼泪直流下来。众人更是纳罕。又听宝玉说道:“姐姐,我要走了,你好生跟着太太听我的喜信儿罢。”宝钗道:“是时候了,你不必说这些唠叨话了。”宝玉道:“你倒催的我紧,我自己也知道该走了。”回头见众人都在这里,只没惜春紫鹃,便说道:“四妹妹和紫鹃姐姐跟前替我说一句罢,横竖是再见就完了。”众人见他的话又象有理,又象疯话。大家只说他从没出过门,都是太太的一套话招出来的,不如早早催他去了就完了事了,便说道:“外面有人等你呢,你再闹就误了时辰了。”宝玉仰面大笑道:“走了,走了!不用胡闹了,完了事了!”众人也都笑道:“快走罢。”独有王夫人和宝钗娘儿两个倒象生离死别的一般,那眼泪也不知从那里来的,直流下来,几乎失声哭出。但见宝玉嘻天哈地,大有疯傻之状,遂从此出门走了。正是:
\end{parag}


\begin{poem}
    \begin{pl}
        走求名利无双地,打出樊笼第一关。
    \end{pl}
\end{poem}


\begin{parag}
    不言宝玉贾兰出门赴考。且说贾环见他们考去,自己又气又恨,便自大为王说:“我可要给母亲报仇了。家里一个男人没有,上头大太太依了我,还怕谁!”想定了主意,跑到邢夫人那边请了安,说了些奉承的话。那邢夫人自然喜欢,便说道:“你这纔是明理的孩子呢。象那巧姐儿的事,原该我做主的,你琏二哥糊涂,放着亲奶奶,倒托别人去!”贾环道:“人家那头儿也说了,只认得这一门子。现在定了,还要备一分大礼来送太太呢。如今太太有了这样的藩王孙女婿儿,还怕大老爷没大官做么!不是我说自己的太太,他们有了元妃姐姐,便欺压的人难受。将来巧姐儿别也是这样没良心,等我去问问他。”邢夫人道:“你也该告诉他,他纔知道你的好处。只怕他父亲在家也找不出这么门子好亲事来!但只平儿那个糊涂东西,他倒说这件事不好,说是你太太也不愿意。想来恐怕我们得了意。若迟了你二哥回来,又听人家的话,就办不成了。”贾环道:“那边都定了,只等太太出了八字。王府的规矩,三天就要来娶的。但是一件,只怕太太不愿意,那边说是不该娶犯官的孙女,只好悄悄的抬了去,等大老爷免了罪做了官,再大家热闹起来。”邢夫人道:“这有什么不愿意,也是礼上应该的。”贾环道:“既这么着,这帖子太太出了就是了。”邢夫人道:“这孩子又糊涂了,里头都是女人,你叫芸哥儿写了一个就是了。”贾环听说,喜欢的了不得,连忙答应了出来,赶着和贾芸说了,邀着王仁到那外藩公馆立文书兑银子去了。
\end{parag}


\begin{parag}
    那知刚纔所说的话,早被跟邢夫人的丫头听见。那丫头是求了平儿纔挑上的,便抽空儿赶到平儿那里,一五一十的都告诉了。平儿早知此事不好,已和巧姐细细的说明。巧姐哭了一夜,必要等他父亲回来作主,大太太的话不能遵。今儿又听见这话,便大哭起来,要和太太讲去。平儿急忙拦住道:“姑娘且慢着。大太太是你的亲祖母,他说二爷不在家,大太太做得主的,况且还有舅舅做保山。他们都是一气,姑娘一个人那里说得过呢。我到底是下人,说不上话去。如今只可想法儿,断不可冒失的。”邢夫人那边的丫头道:“你们快快的想主意,不然可就要抬走了。”说着,各自去了。平儿回过头来见巧姐哭作一团,连忙扶着道:“姑娘,哭是不中用的,如今是二爷够不着,听见他们的话头——”这句话还没说完,只见邢夫人那边打发人来告诉:“姑娘大喜的事来了。叫平儿将姑娘所有应用的东西料理出来。若是赔送呢,原说明了等二爷回来再办。”平儿只得答应了。
\end{parag}


\begin{parag}
    回来又见王夫人过来,巧姐儿一把抱住,哭得倒在怀里。王夫人也哭道:“妞儿不用着急,我为你吃了大太太好些话,看来是扭不过来的。我们只好应着缓下去,即刻差个家人赶到你父亲那里去告诉。”平儿道:“太太还不知道么?早起三爷在大太太跟前说了,什么外藩规矩三日就要过去的。如今大太太已叫芸哥儿写了名字年庚去了,还等得二爷么?”王夫人听说是“三爷”,便气得说不出话来,呆了半天,一迭声叫人找贾环。找了半日,人回:“今早同蔷哥儿王舅爷出去了。”王夫人问:“芸哥呢?”众人回说不知道。巧姐屋内人人瞪眼,一无方法。王夫人也难和邢夫人争论,只有大家抱头大哭。
\end{parag}


\begin{parag}
    有个婆子进来,回说:“后门上的人说,那个刘姥姥又来了。”王夫人道:“咱们家遭着这样事,那有工夫接待人。不拘怎么回了他去罢。”平儿道:“太太该叫他进来,他是姐儿的干妈,也得告诉告诉他。”王夫人不言语,那婆子便带了刘姥姥进来。各人见了问好。刘姥姥见众人的眼圈儿都是红的,也摸不着头脑,迟了一会子,便问道:“怎么了?太太姑娘们必是想二姑奶奶了。”巧姐儿听见提起他母亲,越发大哭起来。平儿道:“姥姥别说闲话,你既是姑娘的干妈,也该知道的。”便一五一十的告诉了。把个刘姥姥也唬怔了,等了半天,忽然笑道:“你这样一个伶俐姑娘,没听见过鼓儿词么,这上头的方法多着呢。这有什么难的。”平儿赶忙问道:“姥姥你有什么法儿快说罢。”刘姥姥道:“这有什么难的呢,一个人也不叫他们知道,扔崩一走,就完了事了。”平儿道:“这可是混说了。我们这样人家的人,走到那里去!”刘姥姥道:“只怕你们不走,你们要走,就到我屯里去。我就把姑娘藏起来,即刻叫我女婿弄了人,叫姑娘亲笔写个字儿,赶到姑老爷那里,少不得他就来了。可不好么?”平儿道:“大太太知道呢?”刘姥姥道:“我来他们知道么?”平儿道:“大太太住在后头,他待人刻薄,有什么信没有送给他的。你若前门走来就知道了,如今是后门来的,不妨事。”刘姥姥道:“咱们说定了几时,我叫女婿打了车来接了去。”平儿道:“这还等得几时呢,你坐着罢。”急忙进去,将刘姥姥的话避了旁人告诉了。王夫人想了半天不妥当。平儿道:“只有这样。为的是太太纔敢说明,太太就装不知道,回来倒问大太太。我们那里就有人去,想二爷回来也快。”王夫人不言语,叹了一口气。巧姐儿听见,便和王夫人道:“只求太太救我,横竖父亲回来只有感激的。”平儿道:“不用说了,太太回去罢。回来只要太太派人看屋子。”王夫人道:“掩密些。你们两个人的衣服铺盖是要的。”平儿道:“要快走了纔中用呢,若是他们定了,回来就有了饥荒了。”一句话提醒了王夫人,便道:“是了,你们快办去罢,有我呢。”于是王夫人回去,倒过去找邢夫人说闲话儿,把邢夫人先绊住了。平儿这里便遣人料理去了,嘱咐道:“倒别避人,有人进来看见,就说是大太太吩咐的,要一辆车子送刘姥姥去。”这里又买嘱了看后门的人雇了车来。平儿便将巧姐装做青儿模样,急急的去了。后来平儿只当送人,眼错不见,也跨上车去了。原来近日贾府后门虽开,只有一两个人看着,余外虽有几个家下人,因房大人少,空落落的,谁能照应。且邢夫人又是个不怜下人的,众人明知此事不好,又都感念平儿的好处,所以通同一气放走了巧姐。邢夫人还自和王夫人说话,那里理会。只有王夫人甚不放心,说了一回话,悄悄的走到宝钗那里坐下,心里还是惦记着。宝钗见王夫人神色恍惚,便问:“太太的心里有什么事?”王夫人将这事背地里和宝钗说了。宝钗道:“险得很!如今得快快儿的叫芸哥儿止住那里纔妥当。”王夫人道:“我找不着环儿呢。”宝钗道:“太太总要装作不知,等我想个人去叫大太太知道纔好。”王夫人点头,一任宝钗想人。暂且不言。
\end{parag}


\begin{parag}
    且说外藩原是要买几个使唤的女人,据媒人一面之辞,所以派人相看。相看的人回去禀明了藩王。藩王问起人家,众人不敢隐瞒,只得实说。那外藩听了,知是世代勋戚,便说:“了不得!这是有干例禁的,几乎误了大事!况我朝觐已过,便要择日起程,倘有人来再说,快快打发出去。”这日恰好贾芸王仁等递送年庚,只见府门里头的人便说:“奉王爷的命,再敢拿贾府的人来冒充民女者,要拿住究治的。如今太平时候,谁敢这样大胆!”这一嚷,唬得王仁等抱头鼠窜的出来,埋怨那说事的人,大家扫兴而散。
\end{parag}


\begin{parag}
    贾环在家候信,又闻王夫人传唤,急得烦燥起来。见贾芸一人回来,赶着问道:“定了么?”贾芸慌忙跺足道:“了不得,了不得!不知谁露了风了!”还把吃亏的话说了一遍。贾环气得发怔说:“我早起在大太太跟前说的这样好,如今怎么样处呢?这都是你们众人坑了我了!”正没主意,听见里头乱嚷,叫着贾环等的名字说:“大太太二太太叫呢。”两个人只得蹭进去。只见王夫人怒容满面说:“你们干的好事!如今逼死了巧姐和平儿了,快快的给我找还尸首来完事!”两个人跪下。贾环不敢言语,贾芸低头说道:“孙子不敢干什么,为的是邢舅太爷和王舅爷说给巧妹妹作媒,我们纔回太太们的。大太太愿意,纔叫孙子写帖儿去的。人家还不要呢。怎么我们逼死了妹妹呢!”王夫人道:“环儿在大太太那里说的,三日内便要抬了走。说亲作媒有这样的么!我也不问你们,快把巧姐儿还了我们,等老爷回来再说。”邢夫人如今也是一句话儿说不出了,只有落泪。王夫人便骂贾环说:“赵姨娘这样混账的东西,留的种子也是这混账的!”说着,叫丫头扶了回到自己房中。
\end{parag}


\begin{parag}
    那贾环贾芸邢夫人三个人互相埋怨,说道:“如今且不用埋怨,想来死是不死的,必是平儿带了他到那什么亲戚家躲着去了。”邢夫人叫了前后的门人来骂着,问巧姐儿和平儿知道那里去了。岂知下人一口同音说是:“大太太不必问我们,问当家的爷们就知道了。在大太太也不用闹,等我们太太问起来我们有话说。要打大家打,要发大家都发。自从琏二爷出了门,外头闹的还了得!我们的月钱月米是不给了,赌钱喝酒闹小旦,还接了外头的媳妇儿到宅里来。这不是爷吗。”说得贾芸等顿口无言。王夫人那边又打发人来催说:“叫爷们快找来。”那贾环等急得恨无地缝可钻,又不敢盘问巧姐那边的人。明知众人深恨,是必藏起来了。但是这句话怎敢在王夫人面前说。只得各处亲戚家打听,毫无踪迹。里头一个邢夫人,外头环儿等,这几天闹的昼夜不宁。
\end{parag}


\begin{parag}
    看看到了出场日期,王夫人只盼着宝玉、贾兰回来。等到晌午,不见回来,王夫人李纨宝钗着忙,打发人去到下处打听。去了一起,又无消息,连去的人也不来了。回来又打发一起人去,又不见回来。三个人心里如热油熬煎,等到傍晚有人进来,见是贾兰。众人喜欢问道:“宝二叔呢?”贾兰也不及请安,便哭道:“二叔丢了。”王夫人听了这话便怔了,半天也不言语,便直挺挺的躺倒床上。亏得彩云等在后面扶着,下死的叫醒转来哭着。见宝钗也是白瞪两眼。袭人等已哭得泪人一般,只有哭着骂贾兰道:“糊涂东西,你同二叔在一处,怎么他就丢了?”贾兰道:“我和二叔在下处,是一处吃一处睡。进了场,相离也不远,刻刻在一处的。今儿一早,二叔的卷子早完了,还等我呢。我们两个人一起去交了卷子,一同出来,在龙门口一挤,回头就不见了。我们家接场的人都问我,李贵还说看见的,相离不过数步,怎么一挤就不见了。现叫李贵等分头的找去,我也带了人各处号里都找遍了,没有,我所以这时候纔回来。”王夫人是哭的一句话也说不出来,宝钗心里已知八九,袭人痛哭不已。贾蔷等不等吩咐,也是分头而去。可怜荣府的人个个死多活少,空备了接场的酒饭。贾兰也忘却了辛苦,还要自己找去。倒是王夫人拦住道:“我的儿,你叔叔丢了,还禁得再丢了你么。好孩子,你歇歇去罢。”贾兰那里肯走。尤氏等苦劝不止。众人中只有惜春心里却明白了,只不好说出来,便问宝钗道:“二哥哥带了玉去了没有?”宝钗道:“这是随身的东西,怎么不带!”惜春听了便不言语。袭人想起那日抢玉的事来,也是料着那和尚作怪,柔肠几断,珠泪交流,呜呜咽咽哭个不住。追想当年宝玉相待的情分,有时怄他,他便恼了,也有一种令人回心的好处,那温存体贴是不用说了。若怄急了他,便赌誓说做和尚。那知道今日却应了这句话!看看那天已觉是四更天气,并没有个信儿。李纨又怕王夫人苦坏了,极力的劝着回房。众人都跟着伺候,只有邢夫人回去。贾环躲着不敢出来。王夫人叫贾兰去了,一夜无眠。次日天明,虽有家人回来,都说没有一处不寻到,实在没有影儿。于是薛姨妈、薛蝌、史湘云、宝琴、李婶等,连二连三的过来请安问信。
\end{parag}


\begin{parag}
    如此一连数日,王夫人哭得饮食不进,命在垂危。忽有家人回道:“海疆来了一人,口称统制大人那里来的,说我们家的三姑奶奶明日到京了。”王夫人听说探春回京,虽不能解宝玉之愁,那个心略放了些。到了明日,果然探春回来。众人远远接着,见探春出跳得比先前更好了,服采鲜明。见了王夫人形容枯槁,众人眼肿腮红,便也大哭起来,哭了一会,然后行礼。看见惜春道姑打扮,心里很不舒服。又听见宝玉心迷走失,家中多少不顺的事,大家又哭起来。还亏得探春能言,见解亦高,把话来慢慢儿的劝解了好些时,王夫人等略觉好些。再明儿,三姑爷也来了。知有这样的事,探春住下劝解。跟探春的丫头老婆也与众姐妹们相聚,各诉别后的事。从此上上下下的人,竟是无昼无夜专等宝玉的信。
\end{parag}


\begin{parag}
    那一夜五更多天,外头几个家人进来到二门口报喜。几个小丫头乱跑进来,也不及告诉大丫头了,进了屋子便说:“太太奶奶们大喜。”王夫人打谅宝玉找着了,便喜欢的站起身来说:“在那里找着的,快叫他进来。”那人道:“中了第七名举人。”王夫人道:“宝玉呢?”家人不言语,王夫人仍旧坐下。探春便问:“第七名中的是谁?”家人回说“是宝二爷。”正说着,外头又嚷道:“兰哥儿中了。”那家人赶忙出去接了报单回禀,见贾兰中了一百三十名。李纨心下喜欢,因王夫人不见了宝玉,不敢喜形于色。王夫人见贾兰中了,心下也是喜欢,只想:“若是宝玉一回来,咱们这些人不知怎样乐呢!”独有宝钗心下悲苦,又不好掉泪。众人道喜,说是“宝玉既有中的命,自然再不会丢的。况天下那有迷失了的举人。”王夫人等想来不错,略有笑容。众人便趁势劝王夫人等多进了些饮食。只见三门外头焙茗乱嚷说:“我们二爷中了举人,是丢不了的了。”众人问道:“怎见得呢?”焙茗道:“‘一举成名天下闻’,如今二爷走到那里,那里就知道的。谁敢不送来!”里头的众人都说:“这小子虽是没规矩,这句话是不错的。”惜春道:“这样大人了,那里有走失的。只怕他勘破世情,入了空门,这就难找着他了。”这句话又招得王夫人等又大哭起来。李纨道:“古来成佛作祖成神仙的,果然把爵位富贵都抛了也多得很。”王夫人哭道:“他若抛了父母,这就是不孝,怎能成佛作祖。”探春道:“大凡一个人不可有奇处。二哥哥生来带块玉来,都道是好事,这么说起来,都是有了这块玉的不好。若是再有几天不见,我不是叫太太生气,就有些原故了,只好譬如没有生这位哥哥罢了。果然有来头成了正果,也是太太几辈子的修积。”宝钗听了不言语,袭人那里忍得住,心里一疼,头上一晕便栽倒了。王夫人见了可怜,命人扶他回去。贾环见哥哥侄儿中了,又为巧姐的事大不好意思,只报怨蔷芸两个,知道探春回来,此事不肯干休,又不敢躲开,这几天竟是如在荆棘之中。
\end{parag}


\begin{parag}
    明日贾兰只得先去谢恩,知道甄宝玉也中了,大家序了同年。提起贾宝玉心迷走失,甄宝玉叹息劝慰。知贡举的将考中的卷子奏闻,皇上一一的披阅,看取中的文章俱是平正通达的。见第七名贾宝玉是金陵籍贯,第一百三十名又是金陵贾兰,皇上传旨询问,两个姓贾的是金陵人氏,是否贾妃一族。大臣领命出来,传贾宝玉贾兰问话,贾兰将宝玉场后迷失的话并将三代陈明,大臣代为转奏。皇上最是圣明仁德,想起贾氏功勋,命大臣查复,大臣便细细的奏明。皇上甚是悯恤,命有司将贾赦犯罪情由查案呈奏。皇上又看到海疆靖寇班师善后事宜一本,奏的是海宴河清、万民乐业的事。皇上圣心大悦,命九卿叙功议赏,并大赦天下。贾兰等朝臣散后拜了座师,并听见朝内有大赦的信,便回了王夫人等。合家略有喜色,只盼宝玉回来。薛姨妈更加喜欢,便要打算赎罪。
\end{parag}


\begin{parag}
    一日,人报甄老爷同三姑爷来道喜,王夫人便命贾兰出去接待。不多一回,贾兰进来笑嘻嘻的回王夫人道:“太太们大喜了。甄老伯在朝内听见有旨意,说是大老爷的罪名免了,珍大爷不但免了罪,仍袭了宁国三等世职。荣国世职仍是老爷袭了,俟丁忧服满,仍升工部郎中。所抄家产,全行赏还。二叔的文章,皇上看了甚喜,问知元妃兄弟,北静王还奏说人品亦好,皇上传旨召见,众大臣奏称据伊侄贾兰回称出场时迷失,现在各处寻访,皇上降旨着五营各衙门用心寻访。这旨意一下,请太太们放心,皇上这样圣恩,再没有找不着了。”王夫人等这纔大家称贺,喜欢起来。只有贾环等心下着急,四处找寻巧姐。
\end{parag}


\begin{parag}
    那知巧姐随了刘姥姥带着平儿出了城,到了庄上,刘姥姥也不敢轻亵巧姐,便打扫上房让给巧姐平儿住下。每日供给虽是乡村风味,倒也洁净。又有青儿陪着,暂且宽心。那庄上也有几家富户,知道刘姥姥家来了贾府姑娘,谁不来瞧,都道是天上神仙。也有送菜果的,也有送野味的,到也热闹。内中有个极富的人家,姓周,家财巨万,良田千顷。只有一子,生得文雅清秀,年纪十四岁,他父母延师读书,新近科试中了秀才。那日他母亲看见了巧姐,心里羡慕,自想:“我是庄家人家,那能配得起这样世家小姐!”呆呆的想着。刘姥姥知他心事,拉着他说:“你的心事我知道了,我给你们做个媒罢。”周妈妈笑道:“你别哄我,他们什么人家,肯给我们庄家人么。”刘姥姥道:“说着瞧罢。”于是两人各自走开。
\end{parag}


\begin{parag}
    刘姥姥惦记着贾府,叫板儿进城打听,那日恰好到宁荣街,只见有好些车轿在那里。板儿便在邻近打听,说是:“宁荣两府复了官,赏还抄的家产,如今府里又要起来了。只是他们的宝玉中了官,不知走到那里去了。”板儿心里喜欢,便要回去,又见好几匹马到来,在门前下马。只见门上打千儿请安说:“二爷回来了,大喜!大老爷身上安了么?”那位爷笑着道:“好了。又遇恩旨,就要回来了。”还问:“那些人做什么的?”门上回说:“是皇上派官在这里下旨意,叫人领家产。”那位爷便喜欢进去。板儿便知是贾琏了。也不用打听,赶忙回去告诉了他外祖母。刘姥姥听说,喜的眉开眼笑,去和巧姐儿贺喜,将板儿的话说了一遍。平儿笑说道:“可不是,亏得姥姥这样一办,不然姑娘也摸不着那好时候。”巧姐更自欢喜。正说着,那送贾琏信的人也回来了,说是:“姑老爷感激得很,叫我一到家快把姑娘送回去。又赏了我好几两银子。”刘姥姥听了得意,便叫人赶了两辆车,请巧姐平儿上车。巧姐等在刘姥姥家住熟了,反是依依不舍,更有青儿哭着,恨不能留下。刘姥姥知他不忍相别,便叫青儿跟了进城,一径直奔荣府而来。
\end{parag}


\begin{parag}
    且说贾琏先前知道贾赦病重,赶到配所,父子相见,痛哭了一场,渐渐的好起来。贾琏接着家书,知道家中的事,禀明贾赦回来,走到中途,听得大赦,又赶了两天,今日到家,恰遇颁赏恩旨。里面邢夫人等正愁无人接旨,虽有贾兰,终是年轻,人报琏二爷回来,大家相见,悲喜交集,此时也不及叙话,即到前厅叩见了钦命大人。问了他父亲好,说明日到内府领赏,宁国府第发交居住。众人起身辞别,贾琏送出门去。见有几辆屯车,家人们不许停歇,正在吵闹。贾琏早知道是巧姐来的车,便骂家人道:“你们这班糊涂忘八崽子,我不在家,就欺心害主,将巧姐儿都逼走了。如今人家送来,还要拦阻,必是你们和我有什么仇么!”众家人原怕贾琏回来不依,想来少时纔破,岂知贾琏说得更明,心下不懂,只得站着回道:“二爷出门,奴才们有病的,有告假的,都是三爷、蔷大爷、芸大爷作主,不与奴才们相干。”贾琏道:“什么混账东西!我完了事再和你们说,快把车赶进来!”
\end{parag}


\begin{parag}
    贾琏进去见邢夫人,也不言语,转身到了王夫人那里,跪下磕了个头,回道:“姐儿回来了,全亏太太。环兄弟太太也不用说他了。只是芸儿这东西,他上回看家就闹乱儿,如今我去了几个月,便闹到这样。回太太的话,这种人撵了他不往来也使得。”王夫人道:“你大舅子为什么也是这样?”贾琏道:“太太不用说,我自有道理。”正说着,彩云等回道:“巧姐儿进来了。”见了王夫人,虽然别不多时,想起这样逃难的景况,不免落下泪来。巧姐儿也便大哭。贾琏谢了刘姥姥。王夫人便拉他坐下,说起那日的话来。贾琏见平儿,外面不好说别的,心里感激,眼中流泪。自此贾琏心里愈敬平儿,打算等贾赦等回来要扶平儿为正。此是后话,暂且不题。
\end{parag}


\begin{parag}
    邢夫人正恐贾琏不见了巧姐,必有一番的周折,又听见贾琏在王夫人那里,心下更是着急,便叫丫头去打听。回来说是巧姐儿同着刘姥姥在那里说话,邢夫人纔如梦初觉,知他们的鬼,还抱怨着王夫人“调唆我母子不和,到底是那个送信给平儿的?”正问着,只见巧姐同着刘姥姥带了平儿,王夫人在后头跟着进来,先把头里的话都说在贾芸王仁身上,说:“大太太原是听见人说,为的是好事,那里知道外头的鬼。”邢夫人听了,自觉羞惭。想起王夫人主意不差,心里也服。于是邢王夫人彼此心下相安。
\end{parag}


\begin{parag}
    平儿回了王夫人,带了巧姐到宝钗那里来请安,各自提各自的苦处。又说到“皇上隆恩,咱们家该兴旺起来了。想来宝二爷必回来的。”正说到这话,只见秋纹急忙来说:“袭人不好了!”不知何事,且听下回分解。
\end{parag}