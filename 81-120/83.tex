\chap{八十三}{省宫闱贾元妃染恙 闹闺阃薛宝钗吞声}



\begin{parag}
    话说探春湘云才要走时,忽听外面一个人嚷道:“你这不成人的小蹄子!你是个什么东西,来这园子里头混搅!”黛玉听了,大叫一声道:“这里住不得了。”一手指着窗外,两眼反插上去。原来黛玉住在大观园中,虽靠着贾母疼爱,然在别人身上,凡事终是寸步留心。听见窗外老婆子这样骂着,在别人呢,一句是贴不上的,竟象专骂着自己的。自思一个千金小姐,只因没了爹娘,不知何人指使这老婆子来这般辱骂,那里委屈得来,因此肝肠崩裂,哭晕去了。紫鹃只是哭叫:“姑娘怎么样了,快醒转来罢。”探春也叫了一回。半晌,黛玉回过这口气,还说不出话来,那只手仍向窗外指着。
\end{parag}


\begin{parag}
    探春会意,开门出去,看见老婆子手中拿着拐棍赶着一个不干不净的毛丫头道:“我是为照管这园中的花果树木来到这里,你作什么来了!等我家去打你一个知道。”这丫头扭着头,把一个指头探在嘴里,瞅着老婆子笑。探春骂道:“你们这些人如今越发没了王法了,这里是你骂人的地方儿吗!”老婆子见是探春,连忙陪着笑脸儿说道:“刚才是我的外孙女儿,看见我来了他就跟了来。我怕他闹,所以才吆喝他回去,那里敢在这里骂人呢。”探春道:“不用多说了,快给我都出去。这里林姑娘身上不大好,还不快去么。”老婆子答应了几个“是”,说着一扭身去了。那丫头也就跑了。
\end{parag}


\begin{parag}
    探春回来,看见湘云拉着黛玉的手只管哭,紫鹃一手抱着黛玉,一手给黛玉揉胸口,黛玉的眼睛方渐渐的转过来了。探春笑道:“想是听见老婆子的话,你疑了心了么?”黛玉只摇摇头儿。探春道:“他是骂他外孙女儿,我才刚也听见了。这种东西说话再没有一点道理的,他们懂得什么避讳。”黛玉听了点点头儿,拉着探春的手道:“妹妹……。”叫了一声,又不言语了。探春又道:“你别心烦。我来看你是姊妹们应该的,你又少人伏侍。只要你安心肯吃药,心上把喜欢事儿想想,能够一天一天的硬朗起来,大家依旧结社做诗,岂不好呢。”湘云道:“可是三姐姐说的,那么着不乐?”黛玉哽咽道:“你们只顾要我喜欢,可怜我那里赶得上这日子,只怕不能够了!”探春道:“你这话说的太过了。谁没个病儿灾儿的,那里就想到这里来了。你好生歇歇儿罢,我们到老太太那边,回来再看你。你要什么东西,只管叫紫鹃告诉我。”黛玉流泪道:“好妹妹,你到老太太那里只说我请安,身上略有点不好,不是什么大病,也不用老太太烦心的。”探春答应道:“我知道,你只管养着罢。”说着,才同湘云出去了。
\end{parag}


\begin{parag}
    这里紫鹃扶着黛玉躺在床上,地下诸事,自有雪雁照料,自己只守着旁边,看着黛玉,又是心酸,又不敢哭泣。那黛玉闭着眼躺了半晌,那里睡得着?觉得园里头平日只见寂寞,如今躺在床上,偏听得风声,虫鸣声,鸟语声,人走的脚步声,又象远远的孩子们啼哭声,一阵一阵的聒噪的烦躁起来,因叫紫鹃放下帐子来。雪雁捧了一碗燕窝汤递与紫鹃,紫鹃隔着帐子轻轻问道:“姑娘喝一口汤罢?”黛玉微微应了一声。紫鹃复将汤递给雪雁,自己上来搀扶黛玉坐起,然后接过汤来,搁在唇边试了一试,一手搂着黛玉肩臂,一手端着汤送到唇边。黛玉微微睁眼喝了两三口,便摇摇头儿不喝了。紫鹃仍将碗递给雪雁,轻轻扶黛玉睡下。
\end{parag}


\begin{parag}
    静了一时,略觉安顿。只听窗外悄悄问道:“紫鹃妹妹在家么?”雪雁连忙出来,见是袭人,因悄悄说道:“姐姐屋里坐着。”袭人也便悄悄问道:“姑娘怎么着?”一面走,一面雪雁告诉夜间及方才之事。袭人听了这话,也唬怔了,因说道:“怪道刚才翠缕到我们那边,说你们姑娘病了,唬的宝二爷连忙打发我来看看是怎么样。”正说着,只见紫鹃从里间掀起帘子望外看,见袭人,点头儿叫他。袭人轻轻走过来问道:“姑娘睡着了吗?”紫鹃点点头儿,问道:“姐姐才听见说了?”袭人也点点头儿,蹙着眉道:“终久怎么样好呢!那一位昨夜也把我唬了个半死儿。”紫鹃忙问怎么了,袭人道:“昨日晚上睡觉还是好好儿的,谁知半夜里一迭连声的嚷起心疼来,嘴里胡说白道,只说好象刀子割了去的似的。直闹到打亮梆子以后才好些了。你说唬人不唬人。今日不能上学,还要请大夫来吃药呢。”正说着,只听黛玉在帐子里又咳嗽起来。紫鹃连忙过来捧痰盒儿接痰。黛玉微微睁眼问道:“你和谁说话呢?”紫鹃道:“袭人姐姐来瞧姑娘来了。”说着,袭人已走到床前。黛玉命紫鹃扶起,一手指着床边,让袭人坐下。袭人侧身坐了,连忙陪着笑劝道:“姑娘倒还是躺着罢。”黛玉道:“不妨,你们快别这样大惊小怪的。刚才是说谁半夜里心疼起来?”袭人道:是宝二爷偶然魇住了,不是认真怎么样。”黛玉会意,知道是袭人怕自己又悬心的原故,又感激,又伤心。因趁势问道:“既是魇住了,不听见他还说什么?”袭人道:“也没说什么。”黛玉点点头儿,迟了半日,叹了一声,才说道:“你们别告诉宝二爷说我不好,看耽搁了他的工夫,又叫老爷生气。”袭人答应了,又劝道:“姑娘还是躺躺歇歇罢。”黛玉点头,命紫鹃扶着歪下。袭人不免坐在旁边,又宽慰了几句,然后告辞,回到怡红院,只说黛玉身上略觉不受用,也没什么大病。宝玉才放了心。
\end{parag}


\begin{parag}
    且说探春湘云出了潇湘馆,一路往贾母这边来。探春因嘱咐湘云道:“妹妹,回来见了老太太,别象刚才那样冒冒失失的了。”湘云点头笑道:“知道了,我头里是叫他唬的忘了神了。”说着,已到贾母那边。探春因提起黛玉的病来。贾母听了自是心烦,因说道:“偏是这两个玉儿多病多灾的。林丫头一来二去的大了,他这个身子也要紧。我看那孩子太是个心细。”众人也不敢答言。贾母便向鸳鸯道:“你告诉他们,明儿大夫来瞧了宝玉,就叫他到林姑娘那屋里去。”鸳鸯答应着,出来告诉了婆子们,婆子们自去传话。这里探春湘云就跟着贾母吃了晚饭,然后同回园中去。不提。到了次日,大夫来了,瞧了宝玉,不过说饮食不调,着了点儿风邪,没大要紧,疏散疏散就好了。这里王夫人凤姐等一面遣人拿了方子回贾母,一面使人到潇湘馆告诉说大夫就过来。紫鹃答应了,连忙给黛玉盖好被窝,放下帐子。雪雁赶着收拾房里的东西。一时贾琏陪着大夫进来了,便说道:“这位老爷是常来的,姑娘们不用回避。”老婆子打起帘子,贾琏让着进入房中坐下。贾琏道”紫鹃姐姐,你先把姑娘的病势向王老爷说说。”王大夫道:“且慢说。等我诊了脉,听我说了看是对不对,若有不合的地方,姑娘们再告诉我。”紫鹃便向帐中扶出黛玉的一只手来,搁在迎手上。紫鹃又把镯子连袖子轻轻的搂起,不叫压住了脉息。那王大夫诊了好一回儿,又换那只手也诊了,便同贾琏出来,到外间屋里坐下,说道:“六脉皆弦,因平日郁结所致。”说着,紫鹃也出来站在里间门口。那王大夫便向紫鹃道:“这病时常应得头晕,减饮食,多梦,每到五更,必醒个几次。即日间听见不干自己的事,也必要动气,且多疑多惧。不知者疑为性情乖诞,其实因肝阴亏损,心气衰耗,都是这个病在那里作怪。不知是否?”紫鹃点点头儿,向贾琏道:“说的很是。”王太医道:“既这样就是了。”说毕起身,同贾琏往外书房去开方子。小厮们早已预备下一张梅红单帖,王太医吃了茶,因提笔先写道:
\end{parag}


\begin{qute2sp}
    六脉弦迟,素由积郁。左寸无力,心气已衰。关脉独洪,肝邪偏旺。木气不能疏达,势必上侵脾土,饮食无味,甚至胜所不胜,肺金定受其殃。气不流精,凝而为痰,血随气涌,自然咳吐。理宜疏肝保肺,涵养心脾。虽有补剂,未可骤施。姑拟黑逍遥以开其先,复用归肺固金以继其后。不揣固陋,俟高明裁服。
\end{qute2sp}


\begin{parag}
    又将七味药与引子写了。贾琏拿来看时,问道:“血势上冲,柴胡使得么?”王大夫笑道:“二爷但知柴胡是升提之品,为吐衄所忌。岂知用鳖血拌炒,非柴胡不足宣少阳甲胆之气。以鳖血制之,使其不致升提,且能培养肝阴,制遏邪火。所以《内经》说:‘通因通用,塞因塞用。’柴胡用鳖血拌炒,正是‘假周勃以安刘’的法子。”贾琏点头道:“原来是这么着,这就是了。”王夫人又道:“先请服两剂,再加减或再换方子罢。我还有一点小事,不能久坐,容日再来请安。”说着,贾琏送了出来,说道:“舍弟的药就是那么着了?”王大夫道:“宝二爷倒没什么大病,大约再吃一剂就好了。”说着,上车而去。
\end{parag}


\begin{parag}
    这里贾琏一面叫人抓药。一面回到房中告诉凤姐黛玉的病原与大夫用的药,述了一遍。只见周瑞家的走来回了几件没要紧的事,贾琏听到一半,便说道:“你回二奶奶罢,我还有事呢。”说着就走了。周瑞家的回完了这件事,又说道:“我方才到林姑娘那边,看他那个病,竟是不好呢。脸上一点血色也没有,摸了摸身上,只剩得一把骨头。问问他,也没有话说,只是淌眼泪。回来紫鹃告诉我说:‘姑娘现在病着,要什么自己又不肯要,我打算要问二奶奶那里支用一两个月的月钱。如今吃药虽是公中的,零用也得几个钱。’我答应了他,替他来回奶奶。”凤姐低了半日头,说道:“竟这么着罢:我送他几两银子使罢,也不用告诉林姑娘。这月钱却是不好支的,一个人开了例,要是都支起来,那如何使得呢。你不记得赵姨娘和三姑娘拌嘴了,也无非为的是月钱。况且近来你也知道,出去的多,进来的少,总绕不过弯儿来。不知道的,还说我打算的不好,更有那一种嚼舌根的,说我搬运到娘家去了。周嫂子,你倒是那里经手的人,这个自然还知道些。”周瑞家的道:“真正委屈死人!这样大门头儿,除了奶奶这样心计儿当家罢了。别说是女人当不来,就是三头六臂的男人,还撑不住呢。还说这些个混账话。”说着,又笑了一声,道:“奶奶还没听见呢,外头的人还更糊涂呢。前儿周瑞回家来,说起外头的人打谅着咱们府里不知怎么样有钱呢。也有说‘贾府里的银库几间,金库几间,使的家伙都是金子镶了玉石嵌了的。’也有说‘姑娘做了王妃,自然皇上家的东西分的了一半子给娘家。前儿贵妃娘娘省亲回来,我们还亲见他带了几车金银回来,所以家里收拾摆设的水晶宫似的。那日在庙里还愿,花了几万银子,只算得牛身上拔了一根毛罢咧。’有人还说‘他门前的狮子只怕还是玉石的呢。园子里还有金麒麟,叫人偷了一个去,如今剩下一个了。家里的奶奶姑娘不用说,就是屋里使唤的姑娘们,也是一点儿不动,喝酒下棋,弹琴画画,横竖有伏侍的人呢。单管穿罗罩纱,吃的戴的,都是人家不认得的。那些哥儿姐儿们更不用说了,要天上的月亮,也有人去拿下来给他顽。’还有歌儿呢,说是‘宁国府,荣国府,金银财宝如粪土。吃不穷,穿不穷,算来……’”说到这里,猛然咽住。原来那时歌儿说道是”算来总是一场空”。这周瑞家的说溜了嘴,说到这里,忽然想起这话不好,因咽住了。凤姐儿听了,已明白必是句不好的话了。也不便追问,因说道:“那都没要紧。只是这金麒麟的话从何而来?”周瑞家的笑道:“就是那庙里的老道士送给宝二爷的小金麒麟儿。后来丢了几天,亏了史姑娘捡着还了他,外头就造出这个谣言来了。奶奶说这些人可笑不可笑?”凤姐道:“这些话倒不是可笑,倒是可怕的。咱们一日难似一日,外面还是这么讲究。俗语儿说的,‘人怕出名猪怕壮’,况且又是个虚名儿,终久还不知怎么样呢。”周瑞家的道:“奶奶虑的也是。只是满城里茶坊酒铺儿以及各胡同儿都是这样说,并且不是一年了,那里握的住众人的嘴。”凤姐点点头儿,因叫平儿称了几两银子,递给周瑞家的,道:“你先拿去交给紫鹃,只说我给他添补买东西的。若要官中的,只管要去,别提这月钱的话。他也是个伶透人,自然明白我的话。我得了空儿,就去瞧姑娘去。”周瑞家的接了银子,答应着自去。不提。
\end{parag}


\begin{parag}
    且说贾琏走到外面,只见一个小厮迎上来回道:“大老爷叫二爷说话呢。”贾琏急忙过来,见了贾赦。贾赦道:“方才风闻宫里头传了一个太医院御医,两个吏目去看病,想来不是宫女儿下人了。这几天娘娘宫里有什么信儿没有?”贾琏道:“没有。”贾赦道:“你去问问二老爷和你珍大哥。不然,还该叫人去到太医院里打听打听才是。”贾琏答应了,一面吩咐人往太医院去,一面连忙去见贾政贾珍。贾政听了这话,因问道:“是那里来的风声?”贾琏道:“是大老爷才说的。”贾政道:“你索性和你珍大哥到里头打听打听。”贾琏道:“我已经打发人往太医院打听去了。”一面说着,一面退出来,去找贾珍。只见贾珍迎面来了,贾琏忙告诉贾珍。贾珍道:“我正为也听见这话,来回大老爷二老爷去的。”于是两个人同着来见贾政。贾政道:“如系元妃,少不得终有信的。”说着,贾赦也过来了。到了晌午,打听的人尚未回来。门上人进来,回说:“有两个内相在外要见二位老爷呢。”贾赦道:“请进来。”门上的人领了老公进来。贾赦贾政迎至二门外,先请了娘娘的安,一面同着进来,走至厅上让了坐。老公道:“前日这里贵妃娘娘有些欠安。昨日奉过旨意,宣召亲丁四人进里头探问。许各带丫头一人,余皆不用。亲丁男人只许在宫门外递个职名,请安听信,不得擅入。准于明日辰巳时进去,申酉时出来。”贾政贾赦等站着听了旨意,复又坐下,让老公吃茶毕,老公辞了出去。
\end{parag}


\begin{parag}
    贾赦贾政送出大门,回来先禀贾母。贾母道:“亲丁四人,自然是我和你们两位太太了。那一个人呢?”众人也不敢答言,贾母想了一想,道:“必得是凤姐儿,他诸事有照应。你们爷儿们各自商量去罢。”贾赦贾政答应了出来,因派了贾琏贾蓉看家外,凡文字辈至草字辈一应都去。遂吩咐家人预备四乘绿轿,十余辆大车,明儿黎明伺候。家人答应去了。贾赦贾政又进去回明老太太,辰巳时进去,申酉时出来,今日早些歇歇,明日好早些起来收拾进宫。贾母道:“我知道,你们去罢。”赦政等退出。这里邢夫人王夫人,凤姐儿也都说了一会子元妃的病,又说了些闲话,才各自散了。
\end{parag}


\begin{parag}
    次日黎明,各间屋子丫头们将灯火俱已点齐,太太们各梳洗毕,爷们亦各整顿好了。一到卯初,林之孝和赖大进来,至二门口回道:“轿车俱已齐备,在门外伺候着呢。”不一时,贾赦邢夫人也过来了。大家用了早饭。凤姐先扶老太太出来,众人围随,各带使女一人,缓缓前行。又命李贵等二人先骑马去外宫门接应,自己家眷随后。文字辈至草字辈各自登车骑马,跟着众家人,一齐去了。贾琏贾蓉在家中看家。
\end{parag}


\begin{parag}
    且说贾家的车辆轿马俱在外西垣门口歇下等着。一回儿,有两个内监出来说:“贾府省亲的太太奶奶们,着令入宫探问,爷们俱着令内宫门外请安,不得入见。”门上人叫快进去。贾府中四乘轿子跟着小内监前行,贾家爷们在轿后步行跟着,令众家人在外等候。走近宫门口,只见几个老公在门上坐着,见他们来了,便站起来说道:“贾府爷们至此。”贾赦贾政便捱次立定。轿子抬至宫门口,便都出了轿。早有几个小内监引路,贾母等各有丫头扶着步行。走至元妃寝宫,只见奎壁辉煌,琉璃照耀。又有两个小宫女儿传谕道:“只用请安,一概仪注都免。”贾母等谢了恩,来至床前请安毕,元妃都赐了坐。贾母等又告了坐。元妃便向贾母道:“近日身上可好?”贾母扶着小丫头,颤颤巍巍站起来,答应道:“托娘娘洪福,起居尚健。”元妃又向邢夫人王夫人问了好,邢王二夫人站着回了话。元妃又问凤姐家中过的日子若何,凤姐站起来回奏道:“尚可支持。”元妃道:“这几年来难为你操心。”凤姐正要站起来回奏,只见一个宫女传进许多职名,请娘娘龙目。元妃看时,就是贾赦贾政等若干人。那元妃看了职名,眼圈儿一红,止不住流下泪来。宫女儿递过绢子,元妃一面拭泪,一面传谕道:“今日稍安,令他们外面暂歇。”贾母等站起来,又谢了恩。元妃含泪道:“父女弟兄,反不如小家子得以常常亲近。”贾母等都忍着泪道:“娘娘不用悲伤,家中已托着娘娘的福多了。”元妃又问:“宝玉近来若何?”贾母道:“近来颇肯念书。因他父亲逼得严紧,如今文字也都做上来了。”元妃道:“这样才好。”遂命外宫赐宴,便有两个宫女儿,四个小太监引了到一座宫里,已摆得齐整,各按坐次坐了。不必细述。一时吃完了饭,贾母带着他婆媳三人谢过宴,又耽搁了一回。看看已近酉初,不敢羁留,俱各辞了出来。元妃命宫女儿引道,送至内宫门,门外仍是四个小太监送出。贾母等依旧坐着轿子出来,贾赦接着,大伙儿一齐回去。到家又要安排明后日进宫,仍令照应齐集。不题。
\end{parag}


\begin{parag}
    且说薛家夏金桂赶了薛蟠出去,日间拌嘴没有对头,秋菱又住在宝钗那边去了,只剩得宝蟾一人同住。既给与薛蟠作妾,宝蟾的意气又不比从前了。金桂看去更是一个对头,自己也后悔不来。一日,吃了几杯闷酒,躺在炕上,便要借那宝蟾做个醒酒汤儿,因问着宝蟾道:“大爷前日出门,到底是到那里去?你自然是知道的了。”宝蟾道:“我那里知道。他在奶奶跟前还不说,谁知道他那些事!”金桂冷笑道:“如今还有什么奶奶太太的,都是你们的世界了。别人是惹不得的,有人护庇着,我也不敢去虎头上捉虱子。你还是我的丫头,问你一句话,你就和我摔脸子,说塞话。你既这么有势力,为什么不把我勒死了,你和秋菱不拘谁做了奶奶,那不清净了么!偏我又不死,碍着你们的道儿。”宝蟾听了这话,那里受得住,便眼睛直直的瞅着金桂道:“奶奶这些闲话只好说给别人听去!我并没和奶奶说什么。奶奶不敢惹人家,何苦来拿着我们小软儿出气呢。正经的,奶奶又装听不见,‘没事人一大堆’了。”说着,便哭天哭地起来。金桂越发性起,便爬下炕来,要打宝蟾。宝蟾也是夏家的风气,半点儿不让。金桂将桌椅杯盏,尽行打翻,那宝蟾只管喊冤叫屈,那里理会他半点儿。岂知薛姨妈在宝钗房中听见如此吵嚷,叫香菱:“你去瞧瞧,且劝劝他。”宝钗道:“使不得,妈妈别叫他去。他去了岂能劝他,那更是火上浇了油了。”薛姨妈道:“既这么样,我自己过去。”宝钗道:“依我说妈妈也不用去,由着他们闹去罢。这也是没法儿的事了。”薛姨妈道:“这那里还了得!”说着,自己扶了丫头,往金桂这边来。宝钗只得也跟着过去,又嘱咐香菱道:“你在这里罢。”
\end{parag}


\begin{parag}
    母女同至金桂房门口,听见里头正还嚷哭不止。薛姨妈道:“你们是怎么着,又这样家翻宅乱起来,这还象个人家儿吗!矮墙浅屋的,难道都不怕亲戚们听见笑话了么。”金桂屋里接声道:“我倒怕人笑话呢!只是这里扫帚颠倒竖,也没有主子,也没有奴才,也没有妻,没有妾,是个混账世界了。我们夏家门子里没见过这样规矩,实在受不得你们家这样委屈了!”宝钗道:“大嫂子,妈妈因听见闹得慌,才过来的。就是问的急了些,没有分清‘奶奶’‘宝蟾’两字,也没有什么。如今且先把事情说开,大家和和气气的过日子,也省的妈妈天天为咱们操心。”那薛姨妈道:“是啊,先把事情说开了,你再问我的不是还不迟呢。”金桂道:“好姑娘,好姑娘,你是个大贤大德的。你日后必定有个好人家,好女婿,决不象我这样守活寡,举眼无亲,叫人家骑上头来欺负我的。我是个没心眼儿的人,只求姑娘我说话别往死里挑捡,我从小儿到如今,没有爹娘教导。再者我们屋里老婆汉子大女人小女人的事,姑娘也管不得!”宝钗听了这话,又是羞,又是气,见他母亲这样光景,又是疼不过。因忍了气说道:“大嫂子,我劝你少说句儿罢。谁挑捡你?又是谁欺负你?不要说是嫂子,就是秋菱,我也从来没有加他一点声气儿的。”金桂听了这几句话,更加拍着炕沿大哭起来,说:“我那里比得秋菱,连他脚底下的泥我还跟不上呢!他是来久了的,知道姑娘的心事,又会献勤儿,我是新来的,又不会献勤儿,如何拿我比他。何苦来,天下有几个都是贵妃的命,行点好儿罢!别修的象我嫁个糊涂行子守活寡,那就是活活儿的现了眼了!”薛姨妈听到这里,万分气不过,便站起身来道:“不是我护着自己的女孩儿,他句句劝你,你却句句怄他。你有什么过不去,不要寻他,勒死我倒也是希松的。”宝钗忙劝道:“妈妈,你老人家不用动气。咱们既来劝他,自己生气,倒多了层气。不如且出去,等嫂子歇歇儿再说。”因吩咐宝蟾道:“你可别再多嘴了。”跟了薛姨妈出得房来。
\end{parag}


\begin{parag}
    走过院子里,只见贾母身边的丫头同着秋菱迎面走来。薛姨妈道:“你从那里来,老太太身上可安?”那丫头道:“老太太身上好,叫来请姨太太安,还谢谢前儿的荔枝,还给琴姑娘道喜。”宝钗道:“你多早晚来的?”那丫头道:“来了好一会子了。”薛姨妈料他知道,红着脸说道:“这如今我们家里闹得也不象个过日子的人家了,叫你们那边听见笑话。”丫头道:“姨太太说那里的话,谁家没个碟大碗小磕着碰着的呢。那是姨太太多心罢咧。”说着,跟了回到薛姨妈房中,略坐了一回就去了。宝钗正嘱咐香菱些话,只听薛姨妈忽然叫道:“左肋疼痛的很。”说着,便向炕上躺下。唬得宝钗香菱二人手足无措。要知后事如何,下回分解。
\end{parag}