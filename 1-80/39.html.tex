\chap{三十九}{村老妪荒谈承色笑 痴情子实意觅踪迹}

\begin{parag}
    \begin{note}蒙回前总:只为贫寒不拣行,当家趋入且逢迎。岂知著意无名利,便是三才最上层。\end{note}
\end{parag}


\begin{parag}
    话说众人见平儿来了,都说:“你们奶奶作什么呢,怎么不来了?”平儿笑道:“他那里得空儿来。因为说没有好生吃得,又不得来,所以叫我来问还有没有,叫我要几个拿了家去吃罢。”湘云道:“有,多著呢。”忙令人拿了十个极大的。平儿道:“多拿几个团脐的。”众人又拉平儿坐,平儿不肯。李纨拉著他笑道: “偏要你坐。”拉著他身边坐下,端了一杯酒送到他嘴边。平儿忙喝了一口就要走。李纨道:“偏不许你去。显见得只有凤丫头,就不听我的话了。”说著又命嬷嬷们:“先送了盒子去,就说我留下平儿了。”那婆子一时拿了盒子回来说:“二奶奶说,叫奶奶和姑娘们别笑话要嘴吃。这个盒子里是方才舅太太那里送来的菱粉糕和鸡油卷儿,给奶奶姑娘们吃的。”又向平儿道:“说使你来你就贪住顽不去了。劝你少喝一杯儿罢。”平儿笑道:“多喝了又把我怎么样?”一面说,一面只管喝,又吃螃蟹。李纨揽著他笑道:“可惜这么个好体面模样儿,命却平常,只落得屋里使唤。不知道的人,谁不拿你当作奶奶太太看。”
\end{parag}


\begin{parag}
    平儿一面和宝钗湘云等吃喝,一面回头笑道:“奶奶,别只摸的我怪痒的。”李氏道:“嗳哟!这硬的是什么?”平儿道:“钥匙。”李氏道:“什么钥匙?要紧梯己东西怕人偷了去,却带在身上。我成日家和人说笑,有个唐僧取经,就有个白马来驮他;刘智远打天下,就有个瓜精来送盔甲;有个凤丫头,就有个你。你就是你奶奶的一把总钥匙,还要这钥匙作什么。”平儿笑道:“奶奶吃了酒,又拿了我来打趣著取笑儿了。”宝钗笑道:“这倒是真话。我们没事评论起人来,你们这几个都是百个里头挑不出一个来,妙在各人有各人的好处。”李纨道:“大小都有个天理。比如老太太屋里,要没那个鸳鸯如何使得。从太太起,那一个敢驳老太太的回,现在他敢驳回。偏老太太只听他一个人的话。老太太那些穿戴的,别人不记得,他都记得,要不是他经管著,不知叫人诓骗了多少去呢。那孩子心也公道,虽然这样,倒常替人说好话儿,还倒不依势欺人的。”惜春笑道:“老太太昨儿还说呢,他比我们还强呢。”平儿道:“那原是个好的,我们那里比的上他。”宝玉道:“太太屋里的彩霞,是个老实人。”探春道:“可不是,外头老实,心里有数儿。太太是那么佛爷似的,事情上不留心,他都知道。凡百一应事都是他提著太太行。连老爷在家出外去的一应大小事,他都知道。太太忘了,他背地里告诉太太。”李纨道:“那也罢了。”指著宝玉道:“这一个小爷屋里要不是袭人,你们度量到个什么田地!凤丫头就是楚霸王,也得这两只膀子好举千斤鼎。他不是这丫头,就得这么周到了!”平儿笑道:“先时陪了四个丫头,死的死,去的去,只剩下我一个孤鬼了。”李纨道:“你倒是有造化的。凤丫头也是有造化的。想当初你珠大爷在日,何曾也没两个人。你们看我还是那容不下人的?天天只见他两个不自在。所以你珠大爷一没了,趁年轻我都打发了。若有一个守得住,我倒有个膀臂。”说著滴下泪来。众人都道:“又何必伤心,不如散了倒好。”说著便都洗了手,大家约往贾母王夫人处问安。
\end{parag}


\begin{parag}
    众婆子丫头打扫亭子,收拾杯盘。袭人和平儿同往前去,让平儿到房里坐坐,再喝一杯茶。平儿说:“不喝茶了,再来吧。”说著便要出去。袭人又叫住问道: “这个月的月钱,连老太太和太太还没放呢,是为什么?”平儿见问,忙转身至袭人跟前,见方近无人,才悄悄说道:“你快别问,横竖再迟几天就放了。”袭人笑道:“这是为什么,唬得你这样?”平儿悄悄告诉他道:“这个月的月钱,我们奶奶早已支了,放给人使呢。等别处的利钱收了来,凑齐了才放呢。因为是你,我才告诉你,你可不许告诉一个人去。”袭人道:“难道他还短钱使,还没个足厌?何苦还操这心。”平儿笑道:“何曾不是呢。这几年拿著这一项银子,翻出有几百来了。他的公费月例又使不著,十两八两零碎攒了放出去,只他这梯己利钱,一年不到,上千的银子呢。”袭人笑道:“拿著我们的钱,你们主子奴才赚利钱,哄的我们呆呆的等著。”平儿道:“你又说没良心的话。你难道还少钱使?”袭人道:“我虽不少,只是我也没地方使去,就只预备我们那一个。”平儿道:“你倘若有要紧的事用钱使时,我那里还有几两银子,你先拿来使,明儿我扣下你的就是了。”袭人道:“此时也用不著,怕一时要用起来不够了,我打发人去取就是了。”
\end{parag}


\begin{parag}
    平儿答应著,一径出了园门,来至家内,只见凤姐儿不在房里。忽见上回来打抽丰的那刘姥姥和板儿又来了,坐在那边屋里,还有张材家的周瑞家的陪著,又有两三个丫头在地下倒口袋里的枣子倭瓜并些野菜。众人见他进来,都忙站起来了。\begin{note}庚双夹:妙文!上回是先见平儿后见凤姐,此则先见凤姐后见平儿也。何错综巧妙得情得理之至耶?\end{note}刘姥姥因上次来过,知道平儿的身分,忙跳下地来问“姑娘好”,又说:“家里都问好。早要来请姑奶奶的安看姑娘来的,因为庄家忙。好容易今年多打了两石粮食,瓜果菜蔬也丰盛。这是头一起摘下来的,并没敢卖呢,留的尖儿孝敬姑奶奶姑娘们尝尝。姑娘们天天山珍海味的也吃腻了,这个吃个野意儿,也算是我们的穷心。” 平儿忙道:“多谢费心。”又让坐,自己也坐了。又让“张婶子周大娘坐”,又令小丫头子倒茶去。周瑞张材两家的因笑道:“姑娘今儿脸上有些春色,眼圈儿都红了。”平儿笑道:“可不是。我原是不吃的,大奶奶和姑娘们只是拉著死灌,不得已喝了两盅,脸就红了。”张材家的笑道:“我倒想著要吃呢,又没人让我。明儿再有人请姑娘,可带了我去罢。”说著大家都笑了。周瑞家的道:“早起我就看见那螃蟹了,一斤只好秤两个三个。这么三大篓,想是有七八十斤呢。”周瑞家的道:“若是上上下下只怕还不够。”平儿道:“那里够,不过都是有名儿的吃两个子。那些散众的,也有摸得著的,也有摸不著的。”刘姥姥道:“这样螃蟹,今年就值五分一斤。十斤五钱,五五二两五,三五一十五,再搭上酒菜,一共倒有二十多两银子。阿弥陀佛!这一顿的钱够我们庄家人过一年了。”平儿因问:“想是见过奶奶了?”\begin{note}庚双夹:写平儿伶俐如此。\end{note}刘姥姥道:“见过了,叫我们等著呢。”说著又往窗外看天气,\begin{note}庚双夹:是八月中当开窗时,细致之甚。\end{note}说道:“天好早晚了,我们也去罢,别出不去城才是饥荒呢。”周瑞家的道:“这话倒是,我替你瞧瞧去。”说著一径去了,半日方来,笑道:“可是你老的福来了,竟投了这两个人的缘了。”平儿等问怎么样,周瑞家的笑道:“二奶奶在老太太的跟前呢。我原是悄悄的告诉二奶奶,‘刘姥姥要家去呢,怕晚了赶不出城去。’二奶奶说:‘大远的,难为他扛了那些沉东西来,晚了就住一夜明儿再去。’这可不是投上二奶奶的缘了。这也罢了,偏生老太太又听见了,问刘姥姥是谁。二奶奶便回明白了。老太太说:‘我正想个积古的老人家说话儿,请了来我见一见。’这可不是想不到天上缘分了。”说著,催刘姥姥下来前去。刘姥姥道:“我这生像儿怎好见的。好嫂子,你就说我去了罢。”平儿忙道:“你快去罢,不相干的。我们老太太最是惜老怜贫的,比不得那个狂三诈四的那些人。想是你怯上,我和周大娘送你去。”说著,同周瑞家的引了刘姥姥往贾母这边来。
\end{parag}


\begin{parag}
    二门口该班的小厮们见了平儿出来,都站起来了,又有两个跑上来,赶著平儿叫“姑娘”。\begin{note}庚双夹:想这一个“姑娘”非下称上之“姑娘”也,按北俗以姑母曰“姑姑”,南俗曰“娘娘”,此“姑娘”定是“姑姑”“娘娘”之称。每见大家风俗多有小童称少主妾曰“姑姑”“娘娘”者。按此书中若干人说话语气及动用前照饮食诸项,皆东南西北互相兼用,此“姑娘”之称亦南北相兼而用无疑矣。\end{note}平儿问:“又说什么?”那小厮笑道:“这会子也好早晚了,我妈病了,等著我去请大夫。好姑娘,我讨半日假可使的?”平儿道:“你们倒好,都商议定了,一天一个告假,又不回奶奶,只和我胡缠。前儿住儿去了,二爷偏生叫他,叫不著,我应起来了,还说我作了情。你今儿又来了。”\begin{note}庚双夹:分明几回没写到贾琏,今忽闲中一语便补得贾琏这边天天热闹,令人却如看见听见一般。所谓不写之写也。刘姥姥眼中耳中又一番识面,奇妙之甚!\end{note}周瑞家的道:“当真的他妈病了,姑娘也替他应著,放了他罢。”平儿道:“明儿一早来。听著,我还要使你呢,再睡的日头晒著屁股再来!你这一去,带个信儿给旺儿,就说奶奶的话,问著他那剩的利钱。明儿若不交了来,奶奶也不要了,就越性送他使罢。”\begin{note}庚双夹:交代过袭人的话,看他如此说,真比凤姐又甚一层。李纨之语不谬也。不知阿凤何等福得此一人。\end{note}那小厮欢天喜地答应去了。
\end{parag}


\begin{parag}
    平儿等来至贾母房中,彼时大观园中姊妹们都在贾母前承奉。\begin{note}庚双夹:妙极!连宝玉一并类入姊妹队中了。\end{note}刘姥姥进去,只见满屋里珠围翠绕,花枝招展,并不知都系何人。只见一张榻上歪著一位老婆婆,身后坐著一个纱罗裹的美人一般的一个丫鬟在那里捶腿,凤姐儿站著正说笑。\begin{note}庚双夹:奇奇怪怪文章。在刘姥姥眼中以为阿凤至尊至贵,普天下人独该站著说,阿凤独坐才是。如何今见阿凤独站哉?真妙文字。\end{note}刘姥姥便知是贾母了,忙上来陪著笑,福了几福,口里说:“请老寿星安。”\begin{note}庚双夹:更妙!贾母之号何其多耶?在诸人口中则曰“老太太”,在阿凤口中则曰“老祖宗”,在僧尼口中则曰“老菩萨”,在刘姥姥口中则曰“老寿星”者,却似有数人,想去则皆贾母,难得如此各尽其妙,刘姥姥亦善应接。\end{note}贾母亦欠身问好,又命周瑞家的端过椅子来坐著。那板儿仍是怯人,不知问候。\begin{note}庚双夹:“仍”字妙!盖有上文故也。不知教训者来看此句。\end{note}贾母道:“老亲家,你今年多大年纪了?”刘姥姥忙立身答道:“我今年七十五了。”贾母向众人道:“这么大年纪了,还这么健朗。比我大好几岁呢。我要到这么大年纪,还不知怎么动不得呢。”刘姥姥笑道:“我们生来是受苦的人,老太太生来是享福的。若我们也这样,那些庄家活也没人作了。”贾母道:“眼睛牙齿都还好?”刘姥姥道:“都还好,就是今年左边的槽牙活动了。”贾母道: “我老了,都不中用了,眼也花,耳也聋,记性也没了。你们这些老亲戚,我都不记得了。亲戚们来了,我怕人笑我,我都不会,不过嚼的动的吃两口,睡一觉,闷了时和这些孙子孙女儿顽笑一回就完了。”刘姥姥笑道:“这正是老太太的福了。我们想这么著也不能。”贾母道:“什么福,不过是个老废物罢了。”说的大家都笑了。贾母又笑道:“我才听见凤哥儿说,你带了好些瓜菜来,叫他快收拾去了,我正想个地里现撷的瓜儿菜儿吃。外头买的,不像你们田地里的好吃。”刘姥姥笑道:“这是野意儿,不过吃个新鲜。依我们想鱼肉吃,只是吃不起。”贾母又道:“今儿既认著了亲,别空空儿的就去。不嫌我这里,就住一两天再去。我们也有个园子,园子里头也有果子,你明日也尝尝,带些家去,你也算看亲戚一趟。”凤姐儿见贾母喜欢,也忙留道:“我们这里虽不比你们的场院大,空屋子还有两间。你住两天罢,把你们那里的新闻故事儿说些与我们老太太听听。”贾母笑道:“凤丫头别拿他取笑儿。他是乡屯里的人,老实,那里搁的住你打趣他。”说著,又命人去先抓果子与板儿吃。板儿见人多了,又不敢吃。贾母又命拿些钱给他,叫小幺儿们带他外头顽去。刘姥姥吃了茶,便把些乡村中所见所闻的事情说与贾母,贾母益发得了趣味。正说著,凤姐儿便令人来请刘姥姥吃晚饭。贾母又将自己的菜拣了几样,命人送过去与刘姥姥吃。
\end{parag}


\begin{parag}
    凤姐知道合了贾母的心,吃了饭便又打发过来。鸳鸯忙令老婆子带了刘姥姥去洗了澡,自己挑了两件随常的衣服令给刘姥姥换上。\begin{note}庚双夹:一段写鸳鸯身份权势心机,只写贾母也。\end{note}那刘姥姥那里见过这般行事,忙换了衣裳出来,坐在贾母榻前,又搜寻些话出来说。彼时宝玉姊妹们也都在这里坐著,他们何曾听见过这些话,自觉比那些瞽目先生说的书还好听。那刘姥姥虽是个村野人,却生来的有些见识,况且年纪老了,世情上经历过的,见头一个贾母高兴,第二见这些哥儿姐儿们都爱听,便没了说的也编出些话来讲。因说道:“我们村庄上种地种菜,每年每日,春夏秋冬,风里雨里,那有个坐著的空儿,天天都是在那地头子上作歇马凉亭,什么奇奇怪怪的事不见呢。就象去年冬天,接连下了几天雪,地下压了三四尺深。我那日起的早,还没出房门,只听外头柴草响。我想著必定是有人偷柴草来了。我爬著窗户眼儿一瞧,却不是我们村庄上的人。”贾母道:“必定是过路的客人们冷了,见现成的柴,抽些烤火去也是有的。”刘姥姥笑道:“也并不是客人,所以说来奇怪。老寿星当个什么人?原来是一个十七八岁的极标致的一个小姑娘,梳著溜油光的头,穿著大红袄儿,白绫裙子──”\begin{note}庚双夹:刘姥姥的口气如此。\end{note}刚说到这里,忽听外面人吵嚷起来,又说:“不相干的,别唬著老太太。”贾母等听了,忙问怎么了,丫鬟回说:“南院马棚里走了水,不相干,已经救下去了。”贾母最胆小的,听了这个话,忙起身扶了人出至廊上来瞧,只见东南上火光犹亮。贾母唬的口内念佛,忙命人去火神跟前烧香。王夫人等也忙都过来请安,又回说“已经下去了,老太太请进房去罢。”贾母足的看著火光息了方领众人进来。\begin{note}庚双夹:一段为后回作引,然偏于宝玉爱听时截住。\end{note}宝玉且忙著问刘姥姥:“那女孩儿大雪地作什么抽柴草?倘或冻出病来呢?”贾母道:“都是才说抽柴草惹出火来了,你还问呢。别说这个了,再说别的罢。”宝玉听说,心内虽不乐,也只得罢了。刘姥姥便又想了一篇,说道:“我们庄子东边庄上,有个老奶奶子,今年九十多岁了。他天天吃斋念佛,谁知就感动了观音菩萨夜里来托梦说: ‘你这样虔心,原来你该绝后的,如今奏了玉皇,给你个孙子。’原来这老奶奶只有一个儿子,这儿子也只一个儿子,好容易养到十七八岁上死了,哭的什么似的。后果然又养了一个,今年才十三四岁,生的雪团儿一般,聪明伶俐非常。可见这些神佛是有的。”这一夕话,实合了贾母王夫人的心事,连王夫人也都听住了。
\end{parag}


\begin{parag}
    宝玉心中只记挂著抽柴的故事,因闷闷的心中筹划。探春因问他:“昨日扰了史大妹妹,咱们回去商议著邀一社,又还了席,也请老太太赏菊花,何如?”宝玉笑道:“老太太说了,还要摆酒还史妹妹的席,叫咱们作陪呢。等著吃了老太太的,咱们再请不迟。”探春道:“越往前去越冷了,老太太未必高兴。”宝玉道: “老太太又喜欢下雨下雪的。不如咱们等下头场雪,请老太太赏雪岂不好?咱们雪下吟诗,也更有趣了。”林黛玉忙笑道:“咱们雪下吟诗?依我说,还不如弄一捆柴火,雪下抽柴,还更有趣儿呢。”说著,宝钗等都笑了。宝玉瞅了他一眼,也不答话。
\end{parag}


\begin{parag}
    一时散了,背地里宝玉足的拉了刘姥姥,细问那女孩儿是谁。刘姥姥只得编了告诉他道:“那原是我们庄北沿地埂子上有一个小祠堂里供的,不是神佛,当先有个什么老爷。”说著又想名姓。宝玉道:“不拘什么名姓,你不必想了,只说原故就是了。”刘姥姥道:“这老爷没有儿子,只有一位小姐,名叫茗玉。小姐知书识字,老爷太太爱如珍宝。可惜这茗玉小姐生到十七岁,一病死了。”宝玉听了,跌足叹惜,又问后来怎么样。刘姥姥道:“因为老爷太太思念不尽,便盖了这祠堂,塑了这茗玉小姐的像,派了人烧香拨火。如今日久年深的,人也没了,庙也烂了,那个像就成了精。”宝玉忙道:“不是成精,规矩这样人是虽死不死的。”刘姥姥道:“阿弥陀佛!原来如此。不是哥儿说,我们都当他成精。他时常变了人出来各村庄店道上闲逛。我才说这抽柴火的就是他了。我们村庄上的人还商议著要打了这塑像平了庙呢。”宝玉忙道:“快别如此。若平了庙,罪过不小。”刘姥姥道:“幸亏哥儿告诉我,我明儿回去告诉他们就是了。”宝玉道:“我们老太太、太太都是善人,合家大小也都好善喜舍,最爱修庙塑神的。我明儿做一个疏头,替你化些布施,你就做香头,攒了钱把这庙修盖,再装潢了泥像,每月给你香火钱烧香岂不好?”刘姥姥道:“若这样,我托那小姐的福,也有几个钱使了。”宝玉又问他地名庄名,来往远近,坐落何方。刘姥姥便顺口胡诌了出来。
\end{parag}


\begin{parag}
    宝玉信以为真,回至房中,盘算了一夜。次日一早,便出来给了茗烟几百钱,按著刘姥姥说的方向地名,著茗烟去先踏看明白,回来再做主意。那茗烟去后,宝玉左等也不来,右等也不来,急的热锅上的蚂蚁一般。好容易等到日落,方见茗烟兴兴头头的回来。宝玉忙道:“可有庙了?”茗烟笑道:“爷听的不明白,叫我好找。那地名座落不似爷说的一样,所以找了一日,找到东北上田埂子上才有一个破庙。”宝玉听说,喜的眉开眼笑,忙说道:“刘姥姥有年纪的人,一时错记了也是有的。你且说你见的。”茗烟道:“那庙门却倒是朝南开,也是稀破的。我找的正没好气,一见这个,我说‘可好了’,连忙进去。一看泥胎,唬的我跑出来了,活似真的一般。”宝玉喜的笑道:“他能变化人了,自然有些生气。”茗烟拍手道:“那里有什么女孩儿,竟是一位青脸红发的瘟神爷。”宝玉听了,啐了一口,骂道:“真是一个无用的杀才!这点子事也干不来。”茗烟道:“二爷又不知看了什么书,或者听了谁的混话,信真了,把这件没头脑的事派我去碰头,怎么说我没用呢?”宝玉见他急了,忙抚慰他道:“你别急。改日闲了你再找去。若是他哄我们呢,自然没了,若真是有的,你岂不也积了阴骘。我必重重的赏你。”正说著,只见二门上的小厮来说:“老太太房里的姑娘们站在二门口找二爷呢。”
\end{parag}


\begin{parag}
    \begin{note}蒙回末总:此回第一写势利之好财,第二写穷苦趋势之求财。且文章不得雷同,先既有杜诗,而今不得不用套坡公之遗事,以振其余响即此,以点染宝玉之痴。其文真如环转,无端倪可指。\end{note}
\end{parag}

