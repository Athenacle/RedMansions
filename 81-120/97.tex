\chap{九十七}{林黛玉焚稿断痴情 薛宝钗出闺成大礼}



\begin{parag}
    话说黛玉到潇湘馆门口,紫鹃说了一句话,更动了心,一时吐出血来,几乎晕倒。亏了还同着秋纹,两个人挽扶着黛玉到屋里来。那时秋纹去后,紫鹃雪雁守着,见他渐渐苏醒过来,问紫鹃道:“你们守着哭什么?”紫鹃见他说话明白,倒放了心了,因说:“姑娘刚才打老太太那边回来,身上觉着不大好,唬的我们没了主意,所以哭了。”黛玉笑道:“我那里就能够死呢。”这一句话没完,又喘成一处。原来黛玉因今日听得宝玉宝钗的事情,这本是他数年的心病,一时急怒,所以迷惑了本性。及至回来吐了这一口血,心中却渐渐的明白过来,把头里的事一字也不记得了。这会子见紫鹃哭,方模糊想起傻大姐的话来,此时反不伤心,惟求速死,以完此债。这里紫鹃雪雁只得守着,想要告诉人去,怕又象上次招得凤姐儿说他们失惊打怪的。
\end{parag}


\begin{parag}
    那知秋纹回去,神情慌遽。正值贾母睡起中觉来,看见这般光景,便问怎么了。秋纹吓的连忙把刚才的事回了一遍。贾母大惊说:“这还了得!”连忙着人叫了王夫人凤姐过来,告诉了他婆媳两个。凤姐道:“我都嘱咐到了,这是什么人走了风呢。这不更是一件难事了吗。贾母道:“且别管那些,先瞧瞧去是怎么样了。”说着便起身带着王夫人凤姐等过来看视。见黛玉颜色如雪,并无一点血色,神气昏沉,气息微细。半日又咳嗽了一阵,丫头递了痰盒,吐出都是痰中带血的。大家都慌了。只见黛玉微微睁眼,看见贾母在他旁边,便喘吁吁的说道:“老太太,你白疼了我了!”贾母一闻此言,十分难受,便道:“好孩子,你养着罢,不怕的。”黛玉微微一笑,把眼又闭上了。外面丫头进来回凤姐道:“大夫来了。”于是大家略避。王大夫同着贾琏进来,诊了脉,说道:“尚不妨事。这是郁气伤肝,肝不藏血,所以神气不定。如今要用敛阴止血的药,方可望好。”王大夫说完,同着贾琏出去开方取药去了。
\end{parag}


\begin{parag}
    贾母看黛玉神气不好,便出来告诉凤姐等道:“我看这孩子的病,不是我咒他,只怕难好。你们也该替他预备预备,冲一冲。或者好了,岂不是大家省心。就是怎么样,也不至临时忙乱。咱们家里这两天正有事呢。”凤姐儿答应了。贾母又问了紫鹃一回,到底不知是那个说的。贾母心里只是纳闷,因说:“孩子们从小儿在一处儿顽,好些是有的。如今大了懂的人事,就该要分别些,才是做女孩儿的本分,我才心里疼他。若是他心里有别的想头,成了什么人了呢!我可是白疼了他了。你们说了,我倒有些不放心。”因回到房中,又叫袭人来问。袭人仍将前日回王夫人的话并方才黛玉的光景述了一遍。贾母道:“我方才看他却还不至糊涂,这个理我就不明白了。咱们这种人家,别的事自然没有的,这心病也是断断有不得的。林丫头若不是这个病呢,我凭着花多少钱都使得。若是这个病,不但治不好,我也没心肠了。”凤姐道:“林妹妹的事老太太倒不必张心,横竖有他二哥哥天天同着大夫瞧看。倒是姑妈那边的事要紧。今日早起听见说,房子不差什么就妥当了,竟是老太太,太太到姑妈那边,我也跟了去,商量商量。就只一件,姑妈家里有宝妹妹在那里,难以说话,不如索性请姑妈晚上过来,咱们一夜都说结了,就好办了。”贾母王夫人都道:“你说的是。今日晚了,明日饭后咱们娘儿们就过去。”说着,贾母用了晚饭。凤姐同王夫人各自归房。不提。
\end{parag}


\begin{parag}
    且说次日凤姐吃了早饭过来,便要试试宝玉,走进里间说道:“宝兄弟大喜,老爷已择了吉日要给你娶亲了。你喜欢不喜欢?”宝玉听了,只管瞅着凤姐笑,微微的点点头儿。凤姐笑道:“给你娶林妹妹过来好不好?”宝玉却大笑起来。凤姐看着,也断不透他是明白是糊涂,因又问道:“老爷说你好了才给你娶林妹妹呢,若还是这么傻,便不给你娶了。”宝玉忽然正色道:“我不傻,你才傻呢。”说着,便站起来说:“我去瞧瞧林妹妹,叫他放心。”凤姐忙扶住了,说:“林妹妹早知道了。他如今要做新媳妇了,自然害羞,不肯见你的。”宝玉道:“娶过来他到底是见我不见?”凤姐又好笑,又着忙,心里想:“袭人的话不差。提了林妹妹,虽说仍旧说些疯话,却觉得明白些。若真明白了,将来不是林妹妹,打破了这个灯虎儿,那饥荒才难打呢。”便忍笑说道:“你好好儿的便见你,若是疯疯颠颠的,他就不见你了。”宝玉说道:“我有一个心,前儿已交给林妹妹了。他要过来,横竖给我带来,还放在我肚子里头。”凤姐听着竟是疯话,便出来看着贾母笑。贾母听了,又是笑,又是疼,便说道:“我早听见了。如今且不用理他,叫袭人好好的安慰他。咱们走罢。”
\end{parag}


\begin{parag}
    说着王夫人也来。大家到了薛姨妈那里,只说惦记着这边的事来瞧瞧。薛姨妈感激不尽,说些薛蟠的话。喝了茶,薛姨妈才要人告诉宝钗,凤姐连忙拦住说:“姑妈不必告诉宝妹妹。”又向薛姨妈陪笑说道:“老太太此来,一则为瞧姑妈,二则也有句要紧的话特请姑妈到那边商议。”薛姨妈听了,点点头儿说:“是了。”于是大家又说些闲话,便回来了。
\end{parag}


\begin{parag}
    当晚薛姨妈果然过来,见过了贾母,到王夫人屋里来,不免说起王子腾来,大家落了一回泪。薛姨妈便问道:“刚才我到老太太那里,宝哥儿出来请安还好好儿的,不过略瘦些,怎么你们说得很利害?”凤姐便道:“其实也不怎么样,只是老太太悬心。目今老爷又要起身外任去,不知几年才来。老太太的意思,头一件叫老爷看着宝兄弟成了家也放心,二则也给宝兄弟冲冲喜,借大妹妹的金琐压压邪气,只怕就好了。”薛姨妈心里也愿意,只虑着宝钗委屈,便道:“也使得,只是大家还要从长计较计较才好。”王夫人便按着凤姐的话和薛姨妈说,只说:“姨太太这会子家里没人,不如把装奁一概蠲免。明日就打发蝌儿去告诉蟠儿,一面这里过门,一面给他变法儿撕掳官事。”并不提宝玉的心事,又说:“姨太太,既作了亲,娶过来早早好一天,大家早放一天心。”正说着,只见贾母差鸳鸯过来候信。薛姨妈虽恐宝钗委屈,然也没法儿,又见这般光景,只得满口应承。鸳鸯回去回了贾母。贾母也甚喜欢,又叫鸳鸯过来求薛姨妈和宝钗说明原故,不叫他受委屈。薛姨妈也答应了。便议定凤姐夫妇作媒人。大家散了。王夫人姊妹不免又叙了半夜话儿。
\end{parag}


\begin{parag}
    次日,薛姨妈回家将这边的话细细的告诉了宝钗,还说:“我已经应承了。”宝钗始则低头不语,后来便自垂泪。薛姨妈用好言劝慰解释了好些话。宝钗自回房内,宝琴随去解闷。薛姨妈才告诉了薛蝌,叫他明日起身,”一则打听审详的事,二则告诉你哥哥一个信儿,你即便回来。”
\end{parag}


\begin{parag}
    薛蝌去了四日,便回来回复薛姨妈道:“哥哥的事上司已经准了误杀,一过堂就要题本了,叫咱们预备赎罪的银子。妹妹的事,说‘妈妈做主很好的,赶着办又省了好些银子,叫妈妈不用等我,该怎么着就怎么办罢。’”薛姨妈听了,一则薛蟠可以回家,二则完了宝钗的事,心瑞安放了好些。便是看着宝钗心里好象不愿意似的,”虽是这样,他是女儿家,素来也孝顺守礼的人,知我应了,他也没得说的。”便叫薛蝌:“办泥金庚帖,填上八字,即叫人送到琏二爷那边去。还问了过礼的日子来,你好预备。本来咱们不惊动亲友,哥哥的朋友是你说的‘都是混账人’,亲戚呢,就是贾王两家,如今贾家是男家,王家无人在京里。史姑娘放定的事,他家没有请咱们,咱们也不用通知。倒是把张德辉请了来,托他照料些,他上几岁年纪的人,到底懂事。”薛蝌领命,叫人送帖过去。
\end{parag}


\begin{parag}
    次日贾琏过来,见了薛姨妈,请了安,便说:“明日就是上好的日子,今日过来回姨太太,就是明日过礼罢。只求姨太太不要挑饬就是了。”说着,捧过通书来。薛姨妈也谦逊了几句,点头应允。贾琏赶着回去回明贾政。贾政便道:“你回老太太说,既不叫亲友们知道,诸事宁可简便些。若是东西上,请老太太瞧了就是了,不必告诉我。”贾琏答应,进内将话回明贾母。
\end{parag}


\begin{parag}
    这里王夫人叫了凤姐命人将过礼的物件都送与贾母过目,并叫袭人告诉宝玉。那宝玉又嘻嘻的笑道:“这里送到园里,回来园里又送到这里。咱们的人送,咱们的人收,何苦来呢。”贾母王夫人听了,都喜欢道:“说他糊涂,他今日怎么这么明白呢。”鸳鸯等忍不住好笑,只得上来一件一件的点明给贾母瞧,说:“这是金项圈,这是金珠首饰,共八十件。这是妆蟒四十匹。这是各色绸缎一百二十匹。这是四季的衣服共一百二十件。外面也没有预备羊酒,这是折羊酒的银子。”贾母看了都说“好”,轻轻的与凤姐说道:“你去告诉姨太太,说:不是虚礼,求姨太太等蟠儿出来慢慢的叫人给他妹妹做来就是了。那好日子的被褥还是咱们这里代办了罢。”凤姐答应了,出来叫贾琏先过去,又叫周瑞旺儿等,吩咐他们:“不必走大门,只从园里从前开的便门内送去,我也就过去。这门离潇湘馆还远,倘别处的人见了,嘱咐他们不用在潇湘馆里提起。”众人答应着送礼而去。宝玉认以为真,心里大乐,精神便觉得好些,只是语言总有些疯傻。那过礼的回来都不提名说姓,因此上下人等虽都知道,只因凤姐吩咐,都不敢走漏风声。
\end{parag}


\begin{parag}
    且说黛玉虽然服药,这病日重一日。紫鹃等在旁苦劝,说道:“事情到了这个分儿,不得不说了。姑娘的心事,我们也都知道。至于意外之事是再没有的。姑娘不信,只拿宝玉的身子说起,这样大病,怎么做得亲呢。姑娘别听瞎话,自己安心保重才好。”黛玉微笑一笑,也不答言,又咳嗽数声,吐出好些血来。紫鹃等看去,只有一息奄奄,明知劝不过来,惟有守着流泪,天天三四趟去告诉贾母。鸳鸯测度贾母近日比前疼黛玉的心差了些,所以不常去回。况贾母这几日的心都在宝钗宝玉身上,不见黛玉的信儿也不大提起,只请太医调治罢了。
\end{parag}


\begin{parag}
    黛玉向来病着,自贾母起,直到姊妹们的下人,常来问候。今见贾府中上下人等都不过来,连一个问的人都没有,睁开眼,只有紫鹃一人。自料万无生理,因扎挣着向紫鹃说道:“妹妹,你是我最知心的,虽是老太太派你伏侍我这几年,我拿你就当我的亲妹妹。”说到这里,气又接不上来。紫鹃听了,一阵心酸,早哭得说不出话来。迟了半日,黛玉又一面喘一面说道:“紫鹃妹妹,我躺着不受用,你扶起我来靠着坐坐才好。”紫鹃道:“姑娘的身上不大好,起来又要抖搂着了。”黛玉听了,闭上眼不言语了。一时又要起来。紫鹃没法,只得同雪雁把他扶起,两边用软枕靠住,自己却倚在旁边。
\end{parag}


\begin{parag}
    黛玉那里坐得住,下身自觉硌的疼,狠命的撑着,叫过雪雁来道:“我的诗本子。”说着又喘。雪雁料是要他前日所理的诗稿,因找来送到黛玉跟前。黛玉点点头儿,又抬眼看那箱子。雪雁不解,只是发怔。黛玉气的两眼直瞪,又咳嗽起来,又吐了一口血。雪雁连忙回身取了水来,黛玉漱了,吐在盒内。紫鹃用绢子给他拭了嘴。黛玉便拿那绢子指着箱子,又喘成一处,说不上来,闭了眼。紫鹃道:“姑娘歪歪儿罢。”黛玉又摇摇头儿。紫鹃料是要绢子,便叫雪雁开箱,拿出一块白绫绢子来。黛玉瞧了,撂在一边,使劲说道:“有字的。”紫鹃这才明白过来,要那块题诗的旧帕,只得叫雪雁拿出来递给黛玉。紫鹃劝道:“姑娘歇歇罢,何苦又劳神,等好了再瞧罢。”只见黛玉接到手里,也不瞧诗,扎挣着伸出那只手来狠命的撕那绢子,却是只有打颤的分儿,那里撕得动。紫鹃早已知他是恨宝玉,却也不敢说破,只说:“姑娘何苦自己又生气!”黛玉点点头儿,掖在袖里,便叫雪雁点灯。雪雁答应,连忙点上灯来。
\end{parag}


\begin{parag}
    黛玉瞧瞧,又闭了眼坐着,喘了一会子,又道:“笼上火盆。”紫鹃打谅他冷。因说道:“姑娘躺下,多盖一件罢。那炭气只怕耽不住。”黛玉又摇头儿。雪雁只得笼上,搁在地下火盆架上。黛玉点头,意思叫挪到炕上来。雪雁只得端上来,出去拿那张火盆炕桌。那黛玉却又把身子欠起,紫鹃只得两只手来扶着他。黛玉这才将方才的绢子拿在手中,瞅着那火点点头儿,往上一撂。紫鹃唬了一跳,欲要抢时,两只手却不敢动。雪雁又出去拿火盆桌子,此时那绢子已经烧着了。紫鹃劝道:“姑娘这是怎么说呢。”黛玉只作不闻,回手又把那诗稿拿起来,瞧了瞧又撂下了。紫鹃怕他也要烧,连忙将身倚住黛玉,腾出手来拿时,黛玉又早拾起,撂在火上。此时紫鹃却够不着,干急。雪雁正拿进桌子来,看见黛玉一撂,不知何物,赶忙抢时,那纸沾火就着,如何能够少待,早已烘烘的着了。雪雁也顾不得烧手,从火里抓起来撂在地下乱踩,却已烧得所余无几了。那黛玉把眼一闭,往后一仰,几乎不曾把紫鹃压倒。紫鹃连忙叫雪雁上来将黛玉扶着放倒,心里突突的乱跳。欲要叫人时,天又晚了,欲不叫人时,自己同着雪雁和鹦哥等几个小丫头,又怕一时有什么原故。好容易熬了一夜。到了次日早起,觉黛玉又缓过一点儿来。饭后,忽然又嗽又吐,又紧起来。紫鹃看着不祥了,连忙将雪雁等都叫进来看守,自己却来回贾母。那知到了贾母上房,静悄悄的,只有两三个老妈妈和几个做粗活的丫头在那里看屋子呢。紫鹃因问道:“老太太呢?”那些人都说不知道。紫鹃听这话诧异,遂到宝玉屋里去看,竟也无人。遂问屋里的丫头,也说不知。紫鹃已知八九,”但这些人怎么竟这样狠毒冷淡!”又想到黛玉这几天竟连一个人问的也没有,越想越悲,索性激起一腔闷气来,一扭身便出来了。自己想了一想,”今日倒要看看宝玉是何形状!看他见了我怎么样过的去!那一年我说了一句谎话他就急病了,今日竟公然做出这件事来!可知天下男子之心真真是冰寒雪冷,令人切齿的!”一面走,一面想,早已来到怡红院。只见院门虚掩,里面却又寂静的很。紫鹃忽然想到:“他要娶亲,自然是有新屋子的,但不知他这新屋子在何处?”正在那里徘徊瞻顾,看见墨雨飞跑,紫鹃便叫住他。墨雨过来笑嘻嘻的道:“姐姐在这里做什么?”紫鹃道:“我听见宝二爷娶亲,我要来看看热闹儿。谁知不在这里,也不知是几儿。”墨雨悄悄的道:“我这话只告诉姐姐,你可别告诉雪雁他们。上头吩咐了,连你们都不叫知道呢。就是今日夜里娶,那里是在这里,老爷派琏二爷另收拾了房子了。”说着又问:“姐姐有什么事么?”紫鹃道:“没什么事,你去罢。”墨雨仍旧飞跑去了。紫鹃自己也发了一回呆,忽然想起黛玉来,这时候还不知是死是活。因两泪汪汪,咬着牙发狠道:“宝玉,我看他明儿死了,你算是躲的过不见了!你过了你那如心如意的事儿,拿什么脸来见我!”一面哭,一面走,呜呜咽咽的自回去了。还未到潇湘馆,只见两个小丫头在门里往外探头探脑的,一眼看见紫鹃,那一个便嚷道:“那不是紫鹃姐姐来了吗。”紫鹃知道不好了,连忙摆手儿不叫嚷,赶忙进去看时,只见黛玉肝火上炎,两颧红赤。紫鹃觉得不妥,叫了黛玉的奶妈王奶奶来。一看,他便大哭起来。这紫鹃因王奶妈有些年纪,可以仗个胆儿,谁知竟是个没主意的人,反倒把紫鹃弄得心里七上八下。忽然想起一个人来,便命小丫头急忙去请。你道是谁,原来紫鹃想起李宫裁是个孀居,今日宝玉结亲,他自然回避。况且园中诸事向系李纨料理,所以打发人去请他。
\end{parag}


\begin{parag}
    李纨正在那里给贾兰改诗,冒冒失失的见一个丫头进来回说:“大奶奶,只怕林姑娘好不了,那里都哭呢。”李纨听了,吓了一大跳,也来不及问了,连忙站起身来便走,素云、碧月跟着,一头走着,一头落泪,想着:“姐妹在一处一场,更兼他那容貌才情真是寡二少双,惟有青女素娥可以仿佛一二,竟这样小小的年纪,就作了北邙乡女!偏偏凤姐想出一条偷梁换柱之计,自己也不好过潇湘馆来,竟未能少尽姊妹之情。真真可怜可叹。”一头想着,已走到潇湘馆的门口。里面却又寂然无声,李纨倒着起忙来,想来必是已死,都哭过了,那衣衾未知装裹妥当了没有?连忙三步两步走进屋子来。
\end{parag}


\begin{parag}
    里间门口一个小丫头已经看见,便说:“大奶奶来了。”紫鹃忙往外走,和李纨走了个对脸。李纨忙问:“怎么样?”紫鹃欲说话时,惟有喉中哽咽的分儿,却一字说不出。那眼泪一似断线珍珠一般,只将一只手回过去指着黛玉。李纨看了紫鹃这般光景,更觉心酸,也不再问,连忙走过来。看时,那黛玉已不能言。李纨轻轻叫了两声,黛玉却还微微的开眼,似有知识之状,但只眼皮嘴唇微有动意,口内尚有出入之息,却要一句话一点泪也没有了。李纨回身见紫鹃不在跟前,便问雪雁。雪雁道:“他在外头屋里呢。”李纨连忙出来,只见紫鹃在外间空床上躺着,颜色青黄,闭了眼只管流泪,那鼻涕眼泪把一个砌花锦边的褥子已湿了碗大的一片。李纨连忙唤他,那紫鹃才慢慢的睁开眼欠起身来。李纨道:“傻丫头,这是什么时候,且只顾哭你的!林姑娘的衣衾还不拿出来给他换上,还等多早晚呢。难道他个女孩儿家,你还叫他赤身露体精着来光着去吗!”紫鹃听了这句话,一发止不住痛哭起来。李纨一面也哭,一面着急,一面拭泪,一面拍着紫鹃的肩膀说:“好孩子,你把我的心都哭乱了,快着收拾他的东西罢,再迟一会子就了不得了。”正闹着,外边一个人慌慌张张跑进来,倒把李纨唬了一跳,看时却是平儿。跑进来看见这样,只是呆磕磕的发怔。李纨道:“你这会子不在那边,做什么来了?”说着,林之孝家的也进来了。平儿道:“奶奶不放心,叫来瞧瞧。既有大奶奶在这里,我们奶奶就只顾那一头儿了。”李纨点点头儿。平儿道:“我也见见林姑娘。”说着,一面往里走,一面早已流下泪来。这里李纨因和林之孝家的道:“你来的正好,快出去瞧瞧去。告诉管事的预备林姑娘的后事。妥当了叫他来回我,不用到那边去。”林之孝家的答应了,还站着。李纨道:“还有什么话呢?”林之孝家的道:“刚才二奶奶和老太太商量了,那边用紫鹃姑娘使唤使唤呢。”李纨还未答言,只见紫鹃道:“林奶奶,你先请罢。等着人死了我们自然是出去的,那里用这么……”说到这里却又不好说了,因又改说道:“况且我们在这里守着病人,身上也不洁净。林姑娘还有气儿呢,不时的叫我。”李纨在旁解说道:“当真这林姑娘和这丫头也是前世的缘法儿。倒是雪雁是他南边带来的,他倒不理会。惟有紫鹃,我看他两个一时也离不开。”林之孝家的头里听了紫鹃的话,未免不受用,被李纨这番一说,却也没的说,又见紫鹃哭得泪人一般,只好瞅着他微微的笑,因又说道:“紫鹃姑娘这些闲话倒不要紧,只是他却说得,我可怎么回老太太呢。况且这话是告诉得二奶奶的吗!”正说着,平儿擦着眼泪出来道:“告诉二奶奶什么事?”林之孝家的将方才的话说了一遍。平儿低了一回头,说:“这么着罢,就叫雪姑娘去罢。”李纨道:“他使得吗?”平儿走到李纨耳边说了几句,李纨点点头儿道:“既是这么着,就叫雪雁过去也是一样的。”林之孝家的因问平儿道:“雪姑娘使得吗?”平儿道:“使得,都是一样。”林家的道:“那么姑娘就快叫雪姑娘跟了我去。我先去回了老太太和二奶奶去,这可是大奶奶和姑娘的主意。回来姑娘再各自回二奶奶去。”李纨道:“是了。你这么大年纪,连这么点子事还不耽呢。”林家的笑道:“不是不耽,头一宗这件事老太太和二奶奶办的,我们都不能很明白,再者又有大奶奶和平姑娘呢。”说着,平儿已叫了雪雁出来。原来雪雁因这几日嫌他小孩子家懂得什么,便也把心冷淡了。况且听是老太太和二奶奶叫,也不敢不去。连忙收拾了头,平儿叫他换了新鲜衣服。跟着林家的去了。随后平儿又和李纨说了几句话。李纨又嘱咐平儿打那么催着林之孝家的叫他男人快办了来。平儿答应着出来,转了个弯子,看见林家的带着雪雁在前头走呢,赶忙叫住道:“我带了他去罢,你先告诉林大爷办林姑娘的东西去罢。奶奶那里我替回就是了。”那林家的答应着去了。这里平儿带了雪雁到了新房子里,回明了自去办事。
\end{parag}


\begin{parag}
    却说雪雁看见这般光景,想起他家姑娘,也未免伤心,只是在贾母凤姐跟前不敢露出。因又想道:“也不知用我作什么,我且瞧瞧。宝玉一日家和我们姑娘好的蜜里调油,这时候总不见面了,也不知是真病假病。怕我们姑娘不依,他假说丢了玉,装出傻子样儿来,叫我们姑娘寒了心。他好娶宝姑娘的意思。我看看他去,看他见了我傻不傻。莫不成今儿还装傻么!”一面想着,已溜到里间屋子门口,偷偷儿的瞧。这时宝玉虽因失玉昏愦,但只听见娶了黛玉为妻,真乃是从古至今天上人间第一件畅心满意的事了,那身子顿觉健旺起来,——只不过不似从前那般灵透,所以凤姐的妙计百发百中——巴不得即见黛玉,盼到今日完姻,真乐得手舞足蹈,虽有几句傻话,却与病时光景大相悬绝了。雪雁看了,又是生气又是伤心,他那里晓得宝玉的心事,便各自走开。
\end{parag}


\begin{parag}
    这里宝玉便叫袭人快快给他装新,坐在王夫人屋里。看见凤姐尤氏忙忙碌碌,再盼不到吉时,只管问袭人道:“林妹妹打园里来,为什么这么费事,还不来?”袭人忍着笑道:“等好时辰。”回来又听见凤姐与王夫人道:“虽然有服,外头不用鼓乐,咱们南边规矩要拜堂的,冷清清使不得。我传了家内学过音乐管过戏子的那些女人来吹打,热闹些。”王夫人点头说:“使得。”
\end{parag}


\begin{parag}
    一时大轿从大门进来,家里细乐迎出去,十二对宫灯,排着进来,倒也新鲜雅致。傧相请了新人出轿。宝玉见新人蒙着盖头,喜娘披着红扶着。下首扶新人的你道是谁,原来就是雪雁。宝玉看见雪雁,犹想:“因何紫鹃不来,倒是他呢?”又想道:“是了,雪雁原是他南边家里带来的,紫鹃仍是我们家的,自然不必带来。”因此见了雪雁竟如见了黛玉的一般欢喜。傧相赞礼拜了天地。请出贾母受了四拜,后请贾政夫妇登堂,行礼毕,送入洞房。还有坐床撒帐等事,俱是按金陵旧例。贾政原为贾母作主,不敢违拗,不信冲喜之说。那知今日宝玉居然象个好人一般,贾政见了,倒也喜欢,那新人坐了床便要揭起盖头的,凤姐早已防备,故请贾母王夫人等进去照应。
\end{parag}


\begin{parag}
    宝玉此时到底有些傻气,便走到新人跟前说道:“妹妹身上好了?好些天不见了,盖着这劳什子做什么!”欲待要揭去,反把贾母急出一身冷汗来。宝玉又转念一想道:“林妹妹是爱生气的,不可造次。”又歇了一歇,仍是按捺不住,只得上前揭了。喜娘接去盖头,雪雁走开,莺儿等上来伺候。宝玉睁眼一看,好象宝钗,心里不信,自己一手持灯,一手擦眼,一看,可不是宝钗么!只见他盛妆艳服,丰肩懦体,鬟低鬓亸,眼瞬息微,真是荷粉露垂,杏花烟润了。宝玉发了一回怔,又见莺儿立在旁边,不见了雪雁。宝玉此时心无主意,自己反以为是梦中了,呆呆的只管站着。众人接过灯去,扶了宝玉仍旧坐下,两眼直视,半语全无。贾母恐他病发,亲自扶他上床。凤姐尤氏请了宝钗进入里间床上坐下,宝钗此时自然是低头不语。宝玉定了一回神,见贾母王夫人坐在那边,便轻轻的叫袭人道:“我是在那里呢?这不是做梦么?”袭人道:“你今日好日子,什么梦不梦的混说。老爷可在外头呢。”宝玉悄悄儿的拿手指着道:“坐在那里这一位美人儿是谁?”袭人握了自己的嘴,笑的说不出话来,歇了半日才说道:“是新娶的二奶奶。”众人也都回过头去,忍不住的笑。宝玉又道:“好糊涂,你说二奶奶到底是谁?”袭人道:“宝姑娘。”宝玉道:“林姑娘呢?”袭人道:“老爷作主娶的是宝姑娘,怎么混说起林姑娘来。”宝玉道:“我才刚看见林姑娘了么,还有雪雁呢,怎么说没有。你们这都是做什么顽呢?”凤姐便走上来轻轻的说道:“宝姑娘在屋里坐着呢。别混说,回来得罪了他,老太太不依的。”宝玉听了,这会子糊涂更利害了。本来原有昏愦的病,加以今夜神出鬼没,更叫他不得主意,便也不顾别的了,口口声声只要找林妹妹去。贾母等上前安慰,无奈他只是不懂。又有宝钗在内,又不好明说。知宝玉旧病复发,也不讲明,只得满屋里点起安息香来,定住他的神魂,扶他睡下。众人鸦雀无闻,停了片时,宝玉便昏沉睡去。贾母等才得略略放心,只好坐以待旦,叫凤姐去请宝钗安歇。宝钗置若罔闻,也便和衣在内暂歇。贾政在外,未知内里原由,只就方才眼见的光景想来,心下倒宽了。恰是明日就是起程的吉日,略歇了一歇,众人贺喜送行。贾母见宝玉睡着,也回房去暂歇。
\end{parag}


\begin{parag}
    次早,贾政辞了宗祠,过来拜别贾母,禀称:“不孝远离,惟愿老太太顺时颐养。儿子一到任所,即修禀请安,不必挂念。宝玉的事,已经依了老太太完结,只求老太太训诲。”贾母恐贾政在路不放心,并不将宝玉复病的话说起,只说:“我有一句话,宝玉昨夜完姻,并不是同房。今日你起身,必该叫他远送才是。他因病冲喜,如今才好些,又是昨日一天劳乏,出来恐怕着了风。故此问你,你叫他送呢,我即刻去叫他,你若疼他,我就叫人带了他来,你见见,叫他给你磕头就算了。”贾政道:“叫他送什么,只要他从此以后认真念书,比送我还喜欢呢。”贾母听了,又放了一条心,便叫贾政坐着,叫鸳鸯去如此如此,带了宝玉,叫袭人跟着来。鸳鸯去了不多一会,果然宝玉来了,仍是叫他行礼。宝玉见了父亲,神志略敛些,片时清楚,也没什么大差。贾政吩咐了几句,宝玉答应了。贾政叫人扶他回去了,自己回到王夫人房中,又切实的叫王夫人管教儿子,断不可如前娇纵。明年乡试,务必叫他下场。王夫人一一的听了,也没提起别的。即忙命人扶了宝钗过来,行了新妇送行之礼,也不出房。其余内眷俱送至二门而回。贾珍等也受了一番训饬。大家举酒送行,一班子弟及晚辈亲友,直送至十里长亭而别。不言贾政起程赴任。且说宝玉回来,旧病陡发,更加昏愦,连饮食也不能进了。未知性命如何,下回分解。
\end{parag}
