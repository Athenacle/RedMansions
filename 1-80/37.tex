\chap{三十七}{秋爽齋偶結海棠社 蘅蕪苑夜擬菊花題}

\begin{parag}
    \begin{note}庚:美人用別號亦新奇花樣,且韻且雅,呼去覺滿口生香。結社出自探春意,作者已伏下回“興利除弊”之文也。\end{note}
\end{parag}


\begin{parag}
    \begin{note}庚:此回才放筆寫詩、寫詞、作扎,看他詩復詩、詞複詞、扎又扎,總不相犯。\end{note}
\end{parag}


\begin{parag}
    \begin{note}庚:湘雲,詩客也,前回寫之其今才起社,後用不即不離閒人數語數折,仍歸社中。何巧活之筆如此?\end{note}
\end{parag}


\begin{parag}
    \begin{note}蒙回前總:海棠名詩社,林史傲秋閨。縱有才八斗,不如富貴兒。\end{note}
\end{parag}


\begin{parag}
    這年賈政又點了學差,擇於八月二十日起身。是日拜過宗祠及賈母起身,寶玉諸子弟等送至灑淚亭。
\end{parag}


\begin{parag}
    卻說賈政出門去後,外面諸事不能多記。單表寶玉每日在園中任意縱性的逛蕩,真把光陰虛度,歲月空添。這日正無聊之際,只見翠墨進來,手裏拿著一副花箋送與他。寶玉因道:“可是我忘了,才說要瞧瞧三妹妹去的,可好些了,你偏走來。”翠墨道:“姑娘好了,今兒也不吃藥了,不過是涼著一點兒。”寶玉聽說,便展開花箋看時,上面寫道:
\end{parag}


\begin{qute}
    娣探謹奉二兄文幾:
    \begin{parag}
        前夕新霽,月色如洗,因惜清景難逢,詎忍就臥,時漏已三轉,猶徘徊於桐檻之下,未防風露所欺,致獲採薪之患。昨蒙親勞撫囑,復又數遣侍兒問切,兼以鮮荔並真卿墨跡見賜,何痌瘝惠愛之深哉!今因伏几憑牀處默之時,因思及歷來古人中處名攻利敵之場,猶置一些山滴水之區,遠招近揖,投轄攀轅,務結二三同志盤桓於其中,或豎詞壇,或開吟社,雖一時之偶興,遂成千古之佳談。娣雖不才,竊同叨棲處於泉石之間,而兼慕薛林之技。風庭月榭,惜未宴集詩人;簾杏溪桃,或可醉飛吟盞。孰謂蓮社之雄才,獨許鬚眉;直以東山之雅會,讓餘脂粉。若蒙棹雪而來,娣則掃花以待。
    \end{parag}
    \begin{parag}
        此謹奉。
    \end{parag}
\end{qute}


\begin{parag}
    寶玉看了,不覺喜的拍手笑道:“倒是三妹妹的高雅,我如今就去商議。”一面說,一面就走,翠墨跟在後面。剛到了沁芳亭,只見園中後門上值日的婆子手裏拿著一個字帖走來,見了寶玉,便迎上去,口內說道:“芸哥兒請安,在後門只等著,叫我送來的。”寶玉打開看時,寫道是:
\end{parag}


\begin{qute2sp}
    \begin{parag}
        不肖男芸恭請父親大人萬福金安。男思自蒙天恩,認於膝下,日夜思一孝順,竟無可孝順之處。前因買辦花草,上託大人金福,竟認得許多花兒匠,\begin{note}庚雙夾:直欲噴飯,真好新鮮文字。\end{note}並認得許多名園。因忽見有白海棠一種,不可多得。故變盡方法,只弄得兩盆。大人若視男是親男一般,\begin{note}庚雙夾:皆千古未有之奇文,初讀令人不解,思之則噴飯。\end{note}便留下賞玩。因天氣暑熱,恐園中姑娘們不便,故不敢面見。奉書恭啓,並叩臺安。
    \end{parag}

    \begin{parag}
        男芸跪書。\begin{note}蒙雙夾:一笑\end{note}
    \end{parag}
\end{qute2sp}


\begin{parag}
    寶玉看了,笑道:“獨他來了,還有什麼人?”婆子道:“還有兩盆花兒。”寶玉道:“你出去說,我知道了,難爲他想著。你便把花兒送到我屋裏去就是了。”一面說,一面同翠墨往秋爽齋來,只見寶釵、黛玉、迎春、惜春已都在那裏了。\begin{note}蒙雙夾:卻因芸之一字工夫,已將諸豔請來,省卻多少閒文。不然必雲如何請如何來,則必至齊犯寶玉,終成重複之文矣。\end{note}
\end{parag}


\begin{parag}
    衆人見他進來,都笑說:“又來了一個。”探春笑道:“我不算俗,偶然起個念頭,寫了幾個帖兒試一試,誰知一招皆到。”寶玉笑道:“可惜遲了,早該起個社的。”黛玉道:“你們只管起社,可別算上我,我是不敢的。”迎春笑道:“你不敢誰還敢呢。”\begin{note}庚雙夾:必得如此方是妙文。若也如寶玉說興頭說,則不是黛玉矣。\end{note}寶玉道:“這是一件正經大事,大家鼓舞起來,不要你謙我讓的。各有主意自管說出來大家平章。\begin{note}庚雙夾:這是“正經大事”已妙,且曰 “平章”,更妙!的是寶玉的口角。\end{note}寶姐姐也出個主意,林妹妹也說個話兒。”寶釵道:“你忙什麼,人還不全呢。”\begin{note}庚雙夾:妙!寶釵自有主見,真不誣也。\end{note}一語未了,李紈也來了,進門笑道:“雅的緊!要起詩社,我自薦我掌壇。前兒春天我原有這個意思的。我想了一想,我又不會作詩,瞎亂些什麼,因而也忘了,就沒有說得。既是三妹妹高興,我就幫你作興起來。”\begin{note}庚雙夾:看他又是一篇文字,分敘單傳之法也。\end{note}
\end{parag}


\begin{parag}
    黛玉道:“既然定要起詩社,咱們都是詩翁了,先把這些姐妹叔嫂的字樣改了纔不俗。”\begin{note}庚雙夾:看他寫黛玉,真可人也。\end{note}李紈道:“極是,何不大家起個別號,彼此稱呼則雅。\begin{note}庚雙夾:未起詩社,先起別號。\end{note}我是定了‘稻香老農’,再無人佔的。”\begin{note}庚雙夾:最妙!一個花樣。\end{note}探春笑道: “我就是‘秋爽居士’罷。”寶玉道:“居士,主人到底不恰,且又瘰贅。這裏梧桐芭蕉盡有,或指梧桐芭蕉起個倒好。”探春笑道:“有了,我最喜芭蕉,就稱 ‘蕉下客’罷。”衆人都道別致有趣。黛玉笑道:“你們快牽了他去,燉了脯子喫酒。”衆人不解。黛玉笑道:“古人曾雲‘蕉葉覆鹿’。他自稱‘蕉下客 ’,可不是一隻鹿了?快做了鹿脯來。”衆人聽了都笑起來。探春因笑道:“你別忙中使巧話來罵人,我已替你想了個極當的美號了。”又向衆人道:“當日娥皇女英灑淚在竹上成斑,故今斑竹又名湘妃竹。如今他住的是瀟湘館,他又愛哭,將來他想林姐夫,那些竹子也是要變成斑竹的。以後都叫他作‘瀟湘妃子’就完了。” 大家聽說,都拍手叫妙。林黛玉低了頭方不言語。\begin{note}庚雙夾:妙極趣極!所謂“夫人必自侮然後人侮之”,看因一謔便勾出一美號來,何等妙文哉!另一花樣。\end{note}李紈笑道:“我替薛大妹妹也早已想了個好的,也只三個字。”惜春迎春都問是什麼。\begin{note}庚雙夾:妙文!迎春惜春固不能答言,然不便撕之不敘,故插他二人問。試思近日諸豪宴集雄語偉辯之時,座上或有一二愚夫不敢接談,然偏好問,亦真可厭之事。\end{note}李紈道:“我是封他‘蘅蕪君’了,不知你們如何。”探春笑道:“這個封號極好。”寶玉道:“我呢?你們也替我想一個。”\begin{note}庚雙夾:必有是問。\end{note}寶釵笑道:“你的號早有了,‘無事忙’三字恰當的很。”\begin{note}庚雙夾:真恰當,形容得盡。\end{note}李紈道:“你還是你的舊號‘絳洞花王’就好。”\begin{note}庚雙夾:妙極!又點前文。通部中從頭至末,前文已過者恐去之冷落,使人忘懷,得便一點。未來者恐來之突然,或先伏一線。皆行文之妙訣也。\end{note}寶玉笑道:“小時候乾的營生,還提他作什麼。”\begin{note}庚雙夾:赧言如聞,不知大時又有何營生。\end{note}探春道:“你的號多的很,又起什麼。我們愛叫你什麼,你就答應著就是了。”\begin{note}庚雙夾:更妙!若只管挨次一個一個亂起,則成何文字?另一花樣。\end{note}寶釵道:“還得我送你個號罷。有最俗的一個號,卻於你最當。天下難得的是富貴,又難得的是閒散,這兩樣再不能兼有,不想你兼有了,就叫你 ‘富貴閒人’也罷了。”寶玉笑道:“當不起,當不起,倒是隨你們混叫去罷。”李紈道:“二姑娘四姑娘起個什麼號?”迎春道:“我們又不大會詩,白起個號作什麼?”\begin{note}庚雙夾:假斯文、守錢虜來看這句。\end{note}探春道:“雖如此,也起個纔是。”寶釵道:“他住的是紫菱洲,就叫他‘菱洲’;四丫頭在藕香榭,就叫他‘藕榭’就完了。”
\end{parag}


\begin{parag}
    李紈道:“就是這樣好。但序齒我大,你們都要依我的主意,管情說了大家合意。我們七個人起社,我和二姑娘四姑娘都不會作詩,須得讓出我們三個人去。我們三個各分一件事。”探春笑道:“已有了號,還只管這樣稱呼,不如不有了。以後錯了,也要立個罰約纔好。”李紈道:“立定了社,再定罰約。我那裏地方大,竟在我那裏作社。我雖不能作詩,這些詩人竟不厭俗客,我作個東道主人,我自然也清雅起來了。若是要推我作社長,我一個社長自然不夠,必要再請兩位副社長,就請菱洲藕榭二位學究來,一位出題限韻,一位謄錄監場。亦不可拘定了我們三個人不作,若遇見容易些的題目韻腳,我們也隨便作一首。你們四個卻是要限定的。若如此便起,若不依我,我也不敢附驥了。”迎春惜春本性懶於詩詞,又有薛林在前,聽了這話便深合己意,二人皆說:“極是。”探春等也知此意,見他二人悅服,也不好強,只得依了。因笑道:“這話也罷了,只是自想好笑,好好的我起了個主意,反叫你們三個來管起我來了。”寶玉道:“既這樣,咱們就往稻香村去。”李紈道:“都是你忙,今日不過商議了,等我再請。”寶釵道:“也要議定幾日一會纔好。”探春道:“若只管會的多,又沒趣了。一月之中,只可兩三次纔好。”寶釵點頭道:“一月只要兩次就夠了。擬定日期,風雨無阻。除這兩日外,倘有高興的,他情願加一社的,或情願到他那裏去,或附就了來,亦可使得,豈不活潑有趣。”衆人都道:“這個主意更好。”
\end{parag}


\begin{parag}
    探春道:“只是原系我起的意,我須得先作個東道主人,方不負我這興。”李紈道:“既這樣說,明日你就先開一社如何?”探春道:“明日不如今日,此刻就很好。你就出題,菱洲限韻,藕榭監場。”迎春道:“依我說,也不必隨一人出題限韻,竟是拈鬮公道。”李紈道:“方纔我來時,看見他們抬進兩盆白海棠來,倒是好花。你們何不就詠起他來?”\begin{note}庚雙夾:真正好題。妙在未起詩社先得了題目。\end{note}迎春道:“都還未賞,先倒作詩。”寶釵道:“不過是白海棠,又何必定要見了才作。古人的詩賦,也不過都是寄興寫情耳。若都是等見了作,如今也沒這些詩了。”\begin{note}辰夾:真詩人語。\end{note}迎春道:“既如此,待我限韻。”說著,走到書架前抽出一本詩來,隨手一揭,這首竟是一首七言律,遞與衆人看了,都該作七言律。迎春掩了詩,又向一個小丫頭道:“你隨口說一個字來。”那丫頭正倚門立著,便說了個“門”字。迎春笑道:“就是門字韻,‘十三元’了。頭一個韻定要這‘門’字。”說著,又要了韻牌匣子過來,抽出“十三元”一屜,又命那小丫頭隨手拿四塊。那丫頭便拿了“盆”“魂”“痕”“昏”四塊來。寶玉道:“這‘盆’‘門’兩個字不大好作呢!”
\end{parag}


\begin{parag}
    侍書一樣預備下四份紙筆,便都悄然各自思索起來。獨黛玉或撫梧桐,或看秋色,或又和丫鬟們嘲笑。\begin{note}庚雙夾:看他單寫黛玉。\end{note}迎春又令丫鬟炷了一支“夢甜香”。原來這“夢甜香”只有三寸來長,有燈草粗細,以其易燼,故以此燼爲限,如香燼未成便要罰。\begin{note}庚雙夾:好香!專能撰此新奇字樣。\end{note}一時探春便先有了,自提筆寫出,又改抹了一回,遞與迎春。因問寶釵:“蘅蕪君,你可有了?”寶釵道:“有卻有了,只是不好。”寶玉背著手,在迴廊上踱來踱去,因向黛玉說道:“你聽,他們都有了。”黛玉道:“你別管我。”寶玉又見寶釵已謄寫出來,因說道:“了不得!香只剩了一寸了,我纔有了四句。”又向黛玉道: “香就完了,只管蹲在那潮地下作什麼?”黛玉也不理。寶玉道:“可顧不得你了,好歹也寫出來罷。”說著也走在案前寫了。李紈道:“我們要看詩了,若看完了還不交卷是必罰的。”寶玉道:“稻香老農雖不善作卻善看,又最公道,\begin{note}庚雙夾:理豈不公。\end{note}你就評閱優劣,我們都服的。”衆人都道:“自然。”於是先看探春的稿上寫道是:
\end{parag}


\begin{poem}
    \begin{pl}詠白海棠限門盆魂痕昏\end{pl}

    \begin{pl}斜陽寒草帶重門,苔翠盈鋪雨後盆。\end{pl}

    \begin{pl}玉是精神難比潔,雪爲肌骨易消魂。\end{pl}

    \begin{pl}芳心一點嬌無力,倩影三更月有痕。\end{pl}

    \begin{pl}莫謂縞仙能羽化,多情伴我詠黃昏。\end{pl}
\end{poem}


\begin{parag}
    次看寶釵的是:
\end{parag}


\begin{poem}
    \begin{pl}珍重芳姿晝掩門,\end{pl}
    \begin{note}蒙雙夾:寶釵詩全是自寫身份,諷刺時事。只以品行爲先,才技爲末。……最恨近日小說中一百美人詩詞語氣只得一個豔稿。\end{note}

    \begin{pl}自攜手甕灌苔盆。\end{pl}

    \begin{pl}胭脂洗出秋階影,\end{pl}

    \begin{pl}氷雪招來露砌魂。\end{pl}
    \begin{note}庚雙夾:看他清潔自厲,終不肯作一輕浮語。\end{note}

    \begin{pl}淡極始知花更豔,\end{pl}
    \begin{note}庚雙夾:好極!高情巨眼能幾人哉!正“鳥鳴山更幽”也。\end{note}

    \begin{pl}愁多焉得玉無痕。\end{pl}
    \begin{note}庚雙夾:看他諷刺林寶二人著手。\end{note}

    \begin{pl}欲償白帝憑清潔,\end{pl}
    \begin{note}庚雙夾:看他收到自己身上來,是何等身份。\end{note}

    \begin{pl}不語婷婷日又昏。\end{pl}
\end{poem}


\begin{parag}
    李紈笑道:“到底是蘅蕪君。”說著又看寶玉的,道是:
\end{parag}


\begin{poem}
    \begin{pl}秋容淺淡映重門,七節攢成雪滿盆。\end{pl}

    \begin{pl}出浴太真氷作影,捧心西子玉爲魂。\end{pl}

    \begin{pl}曉風不散愁千點,\end{pl}
    \begin{note}庚雙夾:這句直是自己一生心事。\end{note}
    \begin{pl}宿雨還添淚一痕。\end{pl}
    \begin{note}庚雙夾:妙在終不忘黛玉。\end{note}

    \begin{pl}獨倚畫欄如有意,清砧怨笛送黃昏。\end{pl}
    \begin{note}庚雙夾:寶玉再細心作,只怕還有好的。只是一心掛著黛玉,故手妥不警也。\end{note}
\end{poem}


\begin{parag}
    大家看了,寶玉說探春的好,李紈纔要推寶釵這詩有身分,因又催黛玉。黛玉道:“你們都有了。”說著提筆一揮而就,擲與衆人。李紈等看他寫道是:
\end{parag}


\begin{poem}
    \begin{pl}半卷湘簾半掩門,\end{pl}\begin{note}庚雙夾:且不說花,且說看花的人,起得突然別緻。\end{note}

    \begin{pl}碾氷爲土玉爲盆。\end{pl}\begin{note}庚雙夾:妙極!料定他自與別人不同。\end{note}
\end{poem}


\begin{parag}
    看了這句,寶玉先喝起彩來,只說“從何處想來!”又看下面道:
\end{parag}


\begin{poem}
    \begin{pl}偷來梨蕊三分白,借得梅花一縷魂。\end{pl}
\end{poem}


\begin{parag}
    衆人看了也都不禁叫好,說“果然比別人又是一樣心腸。”又看下面道是:
\end{parag}


\begin{poem}
    \begin{pl}月窟仙人縫縞袂,秋閨怨女拭啼痕。\end{pl}\begin{note}庚雙夾:虛敲旁比,真逸才也。且不脫落自己。\end{note}

    \begin{pl}嬌羞默默同誰訴,倦倚西風夜已昏。\end{pl}\begin{note}庚雙夾:看他終結道自己,一人是一人口氣。逸才仙品固讓顰兒,溫雅沉著終是寶釵。今日之作寶玉自應居末。\end{note}
\end{poem}


\begin{parag}
    衆人看了,都道是這首爲上。李紈道:“若論風流別致,自是這首;若論含蓄渾厚,終讓蘅稿。”探春道:“這評的有理,瀟湘妃子當居第二。”李紈道:“怡紅公子是壓尾,你服不服?”寶玉道:“我的那首原不好了,這評的最公。”\begin{note}庚雙夾:話內細思則似有不服先評之意。\end{note}又笑道:“只是蘅瀟二首還要斟酌。” 李紈道:“原是依我評論,不與你們相干,再有多說者必罰。”寶玉聽說,只得罷了。李紈道:“從此後我定於每月初二、十六這兩日開社,出題限韻都要依我。這其間你們有高興的,你們只管另擇日子補開,那怕一個月每天都開社,我只不管。只是到了初二、十六這兩日,是必往我那裏去。”寶玉道:“到底要起個社名纔是。”探春道:“俗了又不好,特新了,刁鑽古怪也不好。可巧纔是海棠詩開端,就叫個海棠社罷。雖然俗些,因真有此事,也就不礙了。”說畢大家又商議了一回,略用些酒果,方各自散去。也有回家的,也有往賈母王夫人處去的。當下別人無話。\begin{note}庚雙夾:一路總不大寫薛林興頭,可見他二人並不著意於此。不寫薛林正是大手筆,獨他二人長於詩,必使他二人爲之則板腐矣。全是錯綜法。\end{note}
\end{parag}


\begin{parag}
    且說襲人\begin{note}庚雙夾:忽然寫到襲人,真令人不解。看他如何終此詩社之文。\end{note}因見寶玉看了字貼兒便慌慌張張的同翠墨去了,也不知是何事。後來又見後門上婆子送了兩盆海棠花來。襲人問是那裏來的,婆子便將寶玉前一番緣故說了。襲人聽說便命他們擺好,讓他們在下房裏坐了,自己走到自己房內秤了六錢銀子封好,又拿了三百錢走來,都遞與那兩個婆子道:“這銀子賞那抬花來的小子們,這錢你們打酒喫罷。”那婆子們站起來,眉開眼笑,千恩萬謝的不肯受,見襲人執意不收,方領了。襲人又道:“後門上外頭可有該班的小子們?”婆子忙應道:“天天有四個,原預備裏面差使的。姑娘有什麼差使,我們吩咐去。”襲人笑道:“有什麼差使?今兒寶二爺要打發人到小侯爺家與史大姑娘送東西去,可巧你們來了,順便出去叫後門小子們僱輛車來。回來你們就往這裏拿錢,不用叫他們又往前頭混碰去。”婆子答應著去了。
\end{parag}


\begin{parag}
    襲人回至房中,拿碟子盛東西與史湘雲送去,\begin{note}庚雙夾:線頭卻牽出,觀者猶不理。不知是何碟何物,令人犯思度。\end{note}卻見槅子上碟槽空著。\begin{note}庚雙夾:妙極細極!因此處系依古董式樣摳成槽子,故無此件此槽遂空。若忘卻前文,此句不解。\end{note}因回頭見晴雯、秋紋、麝月等都在一處做針黹,襲人問道:“這一個纏絲白瑪瑙碟子那去了?”衆人見問,都你看我我看你,都想不起來。半日,晴雯笑道:“給三姑娘送荔枝去的,還沒送來呢。”襲人道:“家常送東西的傢伙也多,巴巴的拿這個去。”晴雯道:“我何嘗不也這樣說。他說這個碟子配上鮮荔枝纔好看。\begin{note}庚雙夾:自然好看,原該如此。可恨今之有一二好花者不背像景而用。\end{note}我送去,三姑娘見了也說好看,叫連碟子放著,就沒帶來。你再瞧,那槅子盡上頭的一對聯珠瓶還沒收來呢。”秋紋笑道:“提起瓶來,我又想起笑話。我們寶二爺說聲孝心一動,也孝敬到二十分。因那日見園裏桂花,折了兩枝,原是自己要插瓶的,忽然想起來說,這是自己園裏的纔開的新鮮花,不敢自己先頑,巴巴的把那一對瓶拿下來,親自灌水插好了,叫個人拿著,親自送一瓶進老太太,又進一瓶與太太。誰知他孝心一動,連跟的人都得了福了。可巧那日是我拿去的。老太太見了這樣,喜的無可無不可,見人就說:‘到底是寶玉孝順我,連一枝花兒也想的到。別人還只抱怨我疼他。’你們知道,老太太素日不大同我說話的,有些不入他老人家的眼的。那日竟叫人拿幾百錢給我,說我可憐見的,生的單柔。這可是再想不到的福氣。幾百錢是小事,難得這個臉面。及至到了太太那裏,太太正和二奶奶、趙姨奶奶、周姨奶奶好些人翻箱子,找太太當日年輕的顏色衣裳,不知給那一個。一見了,連衣裳也不找了,且看花兒。又有二奶奶在旁邊湊趣兒,誇寶玉又是怎麼孝敬,又是怎樣知好歹,有的沒的說了兩車話。當著衆人,太太自爲又增了光,堵了衆人的嘴。太太越發喜歡了,現成的衣裳就賞了我兩件。衣裳也是小事,年年橫豎也得,卻不象這個彩頭。”晴雯笑道:“呸!沒見世面的小蹄子!那是把好的給了人,挑剩下的纔給你,你還充有臉呢。”秋紋道:“憑他給誰剩的,到底是太太的恩典。”晴雯道:“要是我,我就不要。若是給別人剩下的給我,也罷了。一樣這屋裏的人,難道誰又比誰高貴些?把好的給他,剩下的纔給我,我寧可不要,衝撞了太太,我也不受這口軟氣。”秋紋忙問:“給這屋裏誰的?我因爲前兒病了幾天,家去了,不知是給誰的。好姐姐,你告訴我知道知道。”晴雯道:“我告訴了你,難道你這會退還太太去不成?”秋紋笑道:“胡說。我白聽了喜歡喜歡。那怕給這屋裏的狗剩下的,我只領太太的恩典,也不犯管別的事。”衆人聽了都笑道:“罵的巧,可不是給了那西洋花點子哈巴兒了。”襲人笑道:“你們這起爛了嘴的!得了空就拿我取笑打牙兒。一個個不知怎麼死呢。”秋紋笑道:“原來姐姐得了,我實在不知道。我陪個不是罷。”襲人笑道:“少輕狂罷。你們誰取了碟子來是正經。”\begin{note}庚雙夾:看他忽然夾寫女兒喁喁一段,總不脫落正事。所謂此書一回是兩段,兩段中卻有無限事體,或有一語透至一回者,或有反補上回者,錯綜穿插,從不一氣直起直瀉至終爲了。\end{note}麝月道:“那瓶得空兒也該收來了。老太太屋裏還罷了,太太屋裏人多手雜。別人還可以,趙姨奶奶一夥的人見是這屋裏的東西,又該使黑心弄壞了才罷。太太也不大管這些,不如早些收來正經。”晴雯聽說,便擲下針黹道:“這話倒是,等我取去。”秋紋道:“還是我取去罷,你取你的碟子去。”晴雯笑道:“我偏取一遭兒去。是巧宗兒你們都得了,難道不許我得一遭兒?”麝月笑道:“通共秋丫頭得了一遭兒衣裳,那裏今兒又巧,你也遇見找衣裳不成。”晴雯冷笑道:“雖然碰不見衣裳,或者太太看見我勤謹,一個月也把太太的公費裏分出二兩銀子來給我,也定不得。”說著,又笑道:“你們別和我裝神弄鬼的,什麼事我不知道。”一面說,一面往外跑了。秋紋也同他出來,自去探春那裏取了碟子來。
\end{parag}


\begin{parag}
    襲人打點齊備東西,叫過本處的一個老宋媽媽來,\begin{note}庚雙夾:“宋”,送也。隨事生文,妙!\end{note}向他說道:“你先好生梳洗了,換了出門的衣裳來,如今打發你與史姑娘送東西去。”那嬤嬤道:“姑娘只管交給我,有話說與我,我收拾了就好一順去的。”襲人聽說,便端過兩個小掐絲盒子來。先揭開一個,裏面裝的是紅菱和雞頭兩樣鮮果;又那一個,是一碟子桂花糖蒸新慄粉糕。又說道:“這都是今年咱們這裏園裏新結的果子,寶二爺送來與姑娘嚐嚐。再前日姑娘說這瑪瑙碟子好,姑娘就留下頑罷。\begin{note}庚雙夾:妙!隱這一件公案。餘想襲人必要瑪瑙碟子盛去,何必嬌奢輕□如是耶?固有此一案,則無怪矣。\end{note}這絹包兒裏頭是姑娘上日叫我作的活計,姑娘別嫌粗糙,能著用罷。替我們請安,替二爺問好就是了。”宋嬤嬤道:“寶二爺不知還有什麼說的,姑娘再問問去,回來又別說忘了。”襲人因問秋紋:“方纔可見在三姑娘那裏?”秋紋道:“他們都在那裏商議起什麼詩社呢,又都作詩。想來沒話,你只去罷。” 嬤嬤聽了,便拿了東西出去,另外穿戴了。襲人又囑咐他:“從後門出去,有小子和車等著呢。”宋媽去後,不在話下。
\end{parag}


\begin{parag}
    寶玉回來,先忙著看了一回海棠,至房內告訴襲人起詩社的事。襲人也把打發宋媽媽與史湘雲送東西去的話告訴了寶玉。寶玉聽了,拍手道:“偏忘了他。我自覺心裏有件事,只是想不起來,虧你提起來,正要請他去。這詩社裏若少了他還有什麼意思。”襲人勸道:“什麼要緊,不過玩意兒。他比不得你們自在,家裏又作不得主兒。告訴他,他要來又由不得他;不來,他又牽腸掛肚的,沒的叫他不受用。”寶玉道:“不妨事,我回老太太打發人接他去。”正說著,宋媽媽已經回來,回覆道生受,與襲人道乏,又說:“問二爺作什麼呢,我說和姑娘們起什麼詩社作詩呢。史姑娘說,他們作詩也不告訴他去,急的了不的。”寶玉聽了立身便往賈母處來,立逼著叫人接去。賈母因說:“今兒天晚了,明日一早再去。”寶玉只得罷了,回來悶悶的。
\end{parag}


\begin{parag}
    次日一早,便又往賈母處來催逼人接去。直到午後,史湘雲纔來,寶玉方放了心,見面時就把始末原由告訴他,又要與他詩看。李紈等因說道:“且別給他詩看,先說與他韻。他後來,先罰他和了詩:若好,便請入社;若不好,還要罰他一個東道再說。”史湘雲道:“你們忘了請我,我還要罰你們呢。就拿韻來,我雖不能,只得勉強出醜。容我入社,掃地焚香我也情願。”衆人見他這般有趣,越發喜歡,都埋怨昨日怎麼忘了他,遂忙告訴他韻。史湘雲一心興頭,等不得推敲刪改,一面只管和人說著話,心內早已和成,即用隨便的紙筆錄出,\begin{note}庚雙夾:可見定是好文字,不管怎樣就有了。越用工夫越講完筆墨終成塗雅。\end{note}先笑說道: “我卻依韻和了兩首,\begin{note}庚雙夾:更奇!想前四律已將形容盡矣,一首猶恐重犯,不知二首又從何處著筆。\end{note}好歹我卻不知,不過應命而已。”說著遞與衆人。衆人道:“我們四首也算想絕了,再一首也不能了。你倒弄了兩首,那裏有許多話說,必要重了我們。”一面說,一面看時,只見那兩首詩寫道:
\end{parag}


\begin{poem}
    \begin{pl}神仙昨日降都門,\end{pl}
    \begin{note}庚雙夾:落想便新奇,不落彼四套。\end{note}
    \begin{pl}種得藍田玉一盆。\end{pl}
    \begin{note}庚雙夾:好!“盆”字押得更穩,不落彼四套。\end{note}

    \begin{pl}自是霜娥偏愛冷,\end{pl}
    \begin{note}庚雙夾:又不脫自己將來形景。\end{note}
    \begin{pl} 非關倩女亦離魂。\end{pl}

    \begin{pl}秋陰捧出何方雪,\end{pl}
    \begin{note}庚雙夾:拍案叫絕!壓倒羣芳在此一句。\end{note}
    \begin{pl}雨漬添來隔宿痕。\end{pl}

    \begin{pl}卻喜詩人吟不倦,豈令寂寞度朝昏。\end{pl}
    \begin{note}庚雙夾:真好!\end{note}
    \emptypl

    \begin{pl}蘅芷階通蘿薜門,也宜牆角也宜盆。\end{pl}
    \begin{note}庚雙夾:更好!\end{note}

    \begin{pl}花因喜潔難尋偶,人爲題秋易斷魂。\end{pl}

    \begin{pl}玉燭滴乾風裏淚,晶簾隔破月中痕。\end{pl}

    \begin{pl}幽情慾向嫦娥訴,無奈虛廊夜色昏。\end{pl}
    \begin{note}庚雙夾:二首真可壓卷。詩是好詩,文是奇奇怪怪之文,總令人想不到忽有二首來壓卷。\end{note}
\end{poem}


\begin{parag}
    衆人看一句,驚訝一句,看到了,贊到了,都說:“這個不枉作了海棠詩,真該要起海棠社了。”史湘雲道:“明日先罰我個東道,就讓我先邀一社可使得?” 衆人道:“這更妙了。”因又將昨日的與他評論了一回。\begin{note}該批:觀湘雲作海棠詩,如見其嬌憨之態。是乃實有,非作書者杜撰也。\end{note}
\end{parag}


\begin{parag}
    至晚,寶釵將湘雲邀往蘅蕪苑安歇去。湘雲燈下計議如何設東擬題。寶釵聽他說了半日,皆不妥當,\begin{note}庚雙夾:卻於此刻方寫寶釵。\end{note}因向他說道:“既開社,便要作東。雖然是頑意兒,也要瞻前顧後,又要自己便宜,又要不得罪了人,然後方大家有趣。你家裏你又作不得主,一個月通共那幾串錢,你還不夠盤纏呢。這會子又幹這沒要緊的事,你嬸子聽見了,越發抱怨你了。況且你就都拿出來,做這個東道也是不夠。難道爲這個家去要不成?還是往這裏要呢?”一席話提醒了湘雲,倒躊躕起來。寶釵道:“這個我已經有個主意。我們當鋪裏有個夥計,他家田上出的很好的肥螃蟹,前兒送了幾斤來。現在這裏的人,從老太太起連上園裏的人,有多一半都是愛喫螃蟹的。前日姨娘還說要請老太太在園裏賞桂花喫螃蟹,因爲有事還沒有請呢。你如今且把詩社別提起,只管普通一請。等他們散了,咱們有多少詩作不得的。我和我哥哥說,要幾簍極肥極大的螃蟹來,再往鋪子裏取上幾罈好酒,再備上四五桌果碟,豈不又省事又大家熱鬧了。”湘雲聽了,心中自是感服,極贊他想的周到。寶釵又笑道:“我是一片真心爲你的話。你千萬別多心,想著我小看了你,咱們兩個就白好了。你若不多心,我就好叫他們辦去的。”湘雲忙笑道:“好姐姐,你這樣說,倒多心待我了。憑他怎麼糊塗,連個好歹也不知,還成個人了?我若不把姐姐當親姐姐一樣看,上回那些家常話煩難事也不肯盡情告訴你了。”寶釵聽說,便叫一個婆子來:“出去和大爺說,依前日的大螃蟹要幾簍來,明日飯後請老太太姨娘賞桂花。你說大爺好歹別忘了,我今兒已請下人了。”\begin{note}庚雙夾:必得如此叮嚀,阿呆兄方記得。\end{note}那婆子出去說明,回來無話。
\end{parag}


\begin{parag}
    這裏寶釵又向湘雲道:“詩題也不要過於新巧了。你看古人詩中那些刁鑽古怪的題目和那極險的韻了,若題過於新巧,韻過於險,再不得有好詩,終是小家氣。詩固然怕說熟話,更不可過於求生,只要頭一件立意清新,自然措詞就不俗了。究竟這也算不得什麼,還是紡績針黹是你我的本等。一時閒了,倒是於你我深有益的書看幾章是正經。”湘雲只答應著,因笑道:“我如今心裏想著,昨日作了海棠詩,我如今要作個菊花詩如何?”寶釵道:“菊花倒也合景,只是前人太多了。”湘雲道:“我也是如此想著,恐怕落套。”寶釵想了一想,說道:“有了,如今以菊花爲賓,以人爲主,竟擬出幾個題目來,都是兩個字:一個虛字,一個實字,實字便用‘菊’字,虛字就用通用門的。如此又是詠菊,又是賦事,前人也沒作過,也不能落套。賦景詠物兩關著,又新鮮,又大方。”湘雲笑道:“這卻很好。只是不知用何等虛字纔好。你先想一個我聽聽。”寶釵想了一想,笑道:“《菊夢》就好。”湘雲笑道:“果然好。我也有一個,《菊影》可使得?”寶釵道:“也罷了。只是也有人作過,若題目多,這個也夾的上。我又有了一個。”湘雲道:“快說出來。”寶釵道:“《問菊》如何?”湘雲拍案叫妙,因接說道:“我也有了,《訪菊》如何?”寶釵也贊有趣,因說道:“越性擬出十個來,寫上再來。”說著,二人研墨蘸筆,湘雲便寫,寶釵便念,一時湊了十個。湘雲看了一遍,又笑道:“十個還不成幅,越性湊成十二個便全了,也如人家的字畫冊頁一樣。”寶釵聽說,又想了兩個,一共湊成十二。又說道:“既這樣,越性編出他個次序先後來。”湘雲道:“如此更妙,竟弄成個菊譜了。”寶釵道:“起首是《憶菊》;憶之不得,故訪,第二是《訪菊》;訪之既得,便種,第三是《種菊》;種既盛開,故相對而賞,第四是《對菊》;相對而興有餘,故折來供瓶爲玩,第五是《供菊》;既供而不吟,亦覺菊無彩色,第六便是《詠菊》;既入詞章,不可不供筆墨,第七便是《畫菊》;既爲菊如是碌碌,究竟不知菊有何妙處,不禁有所問,第八便是《問菊》;菊如解語,使人狂喜不禁,第九便是《簪菊》;如此人事雖盡,猶有菊之可詠者,《菊影》《菊夢》二首續在第十第十一;末卷便以《殘菊》總收前題之盛。這便是三秋的妙景妙事都有了。”湘雲依說將題錄出,又看了一回,又問“該限何韻?”寶釵道:“我平生最不喜限韻的,分明有好詩,何苦爲韻所縛。咱們別學那小家派,只出題不拘韻。原爲大家偶得了好句取樂,並不爲此而難人。”湘雲道: “這話很是。這樣大家的詩還進一層。但只咱們五個人,這十二個題目,難道每人作十二首不成?”寶釵道:“那也太難人了。將這題目謄好,都要七言律,明日貼在牆上。他們看了,誰作那一個就作那一個。有力量者,十二首都作也可;不能的,一首不成也可。高才捷足者爲尊。若十二首已全,便不許他後趕著又作,罰他就完了。”湘雲道:“這倒也罷了。”二人商議妥貼,方纔息燈安寢。要知端的,且聽下回分解。
\end{parag}


\begin{parag}
    \begin{note}蒙回末總:薛家女子何貞俠,總因富貴不須誇。發言行事何其嘉,居心用意不狂奢。世人若可平心度,便解雲釵兩不暇。\end{note}
\end{parag}

