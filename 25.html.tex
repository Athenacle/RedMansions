\chap{二十五}{魇魔法叔嫂逢五鬼 红楼梦通灵遇双真}
\begin{parag}
    \begin{note}蒙回前总:有缘的,推不开;如心的,死不改。纵然是,通灵神玉也遭尘败。梦里徘徊,醒后疑猜,时时兜(左手右兜)底上心来。怕人窥破笑盈腮,独自无言偷打噜。这的是,前生造定今生债。\end{note}
\end{parag}


\begin{parag}
    话说红玉心神恍惚,情思缠绵,忽朦胧睡去,遇见贾芸要拉他,却回身一跑,被门槛绊了一跤,唬醒过来,方知是梦。因此翻来覆去,一夜无眠。至次日天明,方才起来,就有几个丫头子来会他去打扫房子地面,提洗脸水。这红玉也不梳洗,向镜中胡乱挽了一挽头发,洗了洗手,腰内束了一条汗巾子,便来打扫房屋。谁知宝玉昨儿见了红玉,也就留了心。若要直点名唤他来使用,一则怕袭人等寒心;\begin{note}甲戌侧:是宝玉心中想,不是袭人拈酸。\end{note}二则又不知红玉是何等行为,若好还罢了,\begin{note}甲戌侧:不知“好”字是如何讲?答曰:在“何等行为”四字上看便知,玉儿每情不情,况有情者乎?\end{note}若不好起来,那时倒不好退送的。因此心下闷闷的,早起来也不梳洗,只坐著出神。一时下了窗子,隔著纱屉子,向外看的真切,只见好几个丫头在那里扫地,都擦胭抹粉,簪花插柳的,\begin{note}甲戌侧:八字写尽蠢鬟,是为衬红玉,亦如用豪贵人家浓妆艳饰插金戴银的衬宝钗、黛玉也。\end{note}独不见昨儿那一个。宝玉便靸了鞋晃出了房门,只装著看花儿,这里瞧瞧,那里望望,\begin{note}庚辰侧:文字有层次。\end{note}一抬头,只见西南角上游廊底下栏杆上似有一个人倚在那里,却恨面前有一株海棠花遮著,看不真切。\begin{note}甲戌双夹:余所谓此书之妙皆从诗词句中翻出者,皆系此等笔墨也。试问观者,此非“隔花人远天涯近”乎?可知上几回非余妄拟也。\end{note}只得又转了一步,仔细一看,可不是昨儿那个丫头在那里出神。待要迎上去,又不好去的。正想著,忽见碧痕来催他洗脸,只得进去了。不在话下。
\end{parag}


\begin{parag}
    却说红玉正自出神,忽见袭人招手叫他,\begin{note}甲戌侧:此处方写出袭人来,是衬贴法。\end{note}只得走上前来。袭人笑道:“我们这里的喷壶还没有收拾了来呢,你到林姑娘那里去,把他们的借来使使。”红玉答应了,便走出来往潇湘馆去。正走上翠烟桥,抬头一望,只见山坡上高处都是拦著帏幙,方想起今儿有匠役在里头种树。因转身一望,只见那边远远一簇人在那里掘土,贾芸正坐在那山子石上。红玉待要过去,又不敢过去,只得闷闷的向潇湘馆取了喷壶回来,无精打彩自向房内倒著。众人只说他一时身上不爽快,都不理论。\begin{note}甲戌侧:文字到此一顿,狡猾之甚。\end{note}
\end{parag}


\begin{parag}
    展眼过了一日,\begin{note}甲戌侧:必云“展眼过了一日”者,是反衬红玉“挨一刻似一夏”也,知乎?\end{note}原来次日就是王子腾夫人的寿诞,那里原打发人来请贾母王夫人的,王夫人见贾母不自在,也便不去了。\begin{note}甲戌侧:所谓一笔两用也!\end{note}倒是薛姨妈同凤姐儿并贾家几个姊妹、宝钗、宝玉一齐都去了,至晚方回。
\end{parag}


\begin{parag}
    可巧王夫人见贾环下了学,便命他来抄个《金刚咒》\begin{note}甲戌侧:用《金刚咒》引五鬼法。\end{note}唪诵唪诵。那贾环正在王夫人炕上坐著,命人点灯,拿腔作势的抄写。\begin{note}甲戌侧:小人乍得意者齐来一玩。\end{note}一时又叫彩云倒杯茶来,一时又叫玉钏儿来剪剪蜡花,一时又说金钏儿挡了灯影。众丫鬟们素日厌恶他,都不答理。只有彩霞还和他合的来,\begin{note}甲戌侧:暗中又伏一风月之隙。\end{note}倒了一钟茶来递与他。因见王夫人和人说话儿,他便悄悄的向贾环说道:“你安些分罢,何苦讨这个厌那个厌的。”贾环道:“我也知道了,你别哄我。如今你和宝玉好,把我不答理,我也看出来了。”彩霞咬著嘴唇,向贾环头上戳了一指头,说道:“没良心的!狗咬吕洞宾,不识好人心。”\begin{note}甲戌双夹:风月之情,皆系彼此业障所牵。虽云“惺惺惜惺惺”,但亦从业障而来。蠢妇配才郎,世间固不少,然俏女慕村夫者尤多,所谓业障牵魔,不在才貌之论。\end{note}\begin{note}庚辰眉:此等世俗之言,亦因人而用,妥极当极!壬午孟夏,雨窗。畸笏。\end{note}
\end{parag}


\begin{parag}
    两人正说著,只见凤姐来了,拜见过王夫人。王夫人便一长一短的问他,今儿是那几位堂客,戏文好歹,酒席如何等语。说了不多几句话,宝玉也来了,进门见了王夫人,不过规规矩矩说了几句,\begin{note}甲戌侧:是大家子弟模样。\end{note}便命人除去抹额,脱了袍服,拉了靴子,便一头滚在王夫人怀里。\begin{note}甲戌侧:余几几失声哭出。\end{note}王夫人便用手满身满脸摩挲抚弄他,\begin{note}甲戌侧:普天下幼年丧母者齐来一哭。\end{note}宝玉也搬著王夫人的脖子说长道短的。\begin{note}甲戌侧:慈母娇儿写尽矣。\end{note}王夫人道:“我的儿,你又吃多了酒,脸上滚热。你还只是揉搓,一会闹上酒来。还不在那里静静的倒一会子呢。”说著,便叫人拿个枕头来。宝玉听说便下来,在王夫人身后倒下,又叫彩霞来替他拍著。宝玉便和彩霞说笑,只见彩霞淡淡的,不大答理,两眼睛只向贾环处看。宝玉便拉他的手笑道:“好姐姐,你也理我理儿呢。”一面说,一面拉他的手,彩霞夺手不肯,便说:“再闹,我就嚷了。”
\end{parag}


\begin{parag}
    二人正闹著,原来贾环听的见,素日原恨宝玉,如今又见他和彩霞闹,心中越发按不下这口毒气。虽不敢明言,却每每暗中算计,\begin{note}甲戌侧:已伏金钏回矣。\end{note}只是不得下手,今见相离甚近,便要用热油烫瞎他的眼睛。因而故意装作失手,把那一盏油汪汪的蜡灯向宝玉脸上只一推。只听宝玉“嗳哟”了一声,满屋里众人都唬了一跳。连忙将地下的戳灯挪过来,又将里外间屋的灯拿了三四盏看时,只见宝玉满脸满头都是油。王夫人又急又气,一面命人来替宝玉擦洗,一面又骂贾环。凤姐三步两步的上炕去替宝玉收拾著,\begin{note}甲戌侧:阿凤活现纸上。\end{note}一面笑道:“老三还是这么慌脚鸡似的,我说你上不得高台盘。赵姨娘时常也该教导教导他。”\begin{note}庚辰侧:为下文紧一步。\end{note}一句话提醒了王夫人,那王夫人不骂贾环,便叫过赵姨娘来骂道:“养出这样黑心不知道理下流种子来,也不管管!几番几次我都不理论,\begin{note}甲戌侧:补出素日来。\end{note}你们得了意了,越发上来了!”
\end{parag}


\begin{parag}
    那赵姨娘素日虽然常怀嫉妒之心,不忿凤姐宝玉两个,也不敢露出来;如今贾环又生了事,受这场恶气,不但吞声承受,而且还要走去替宝玉收拾。只见宝玉左边脸上烫了一溜燎泡出来,幸而眼睛竟没动。王夫人看了,又是心疼,又怕明日贾母问怎么回答,急的又把赵姨娘数落一顿。\begin{note}甲戌侧:总是为楔紧“五鬼”一回文字。\end{note}然后又安慰了宝玉一回,又命取败毒消肿药来敷上。宝玉道:“有些疼,还不妨事。明儿老太太问,就说是我自己烫的罢了。”凤姐笑\begin{note}甲戌侧:两笑,坏极。庚辰眉:为五鬼法作耳,非泛文也。雨窗。\end{note}道:“便说是自己烫的,\begin{note}甲戌侧:玉兄自是悌弟之心性,一叹。\end{note}也要骂人为什么不小心看著,叫你烫了!横竖有一场气生的,到明儿凭你怎么说去罢。”\begin{note}甲戌侧:坏极!总是调唆口吻,赵氏宁不觉乎?\end{note}王夫人命人好生送了宝玉回房去后,袭人等见了,都慌的了不得。
\end{parag}


\begin{parag}
    林黛玉见宝玉出了一天门,就觉闷闷的,没个可说话的人。至晚正打发人来问了两三遍回来不曾,这遍方才回来,又偏生烫了。林黛玉便赶著来瞧,只见宝玉正拿镜子照呢,左边脸上满满的敷了一脸的药。林黛玉只当烫的十分利害,忙上来问怎么烫了,要瞧瞧。宝玉见他来了,忙把脸遮著,摇手叫他出去,不肯叫他看── 知道他的癖性喜洁,见不得这些东西。\begin{note}甲戌双夹:写宝玉文字,此等方是正紧笔墨。\end{note}林黛玉自己也知道自己也有这件癖性,\begin{note}甲戌双夹:写林黛玉文字,此等方是正经笔墨。故二人文字虽多,如此等暗伏淡写处亦不少,观者实实看不出者。\end{note}知道宝玉的心内怕他嫌脏,\begin{note}甲戌侧:二人纯用体贴功夫。\end{note}\begin{note}甲戌双夹:将二人一并,真真写他二人之心玲珑七窍。\end{note}因笑道:“我瞧瞧烫了那里了,有什么遮著藏著的。”一面说,一面就凑上来,强搬著脖子瞧了一瞧,问他疼的怎么样。宝玉道:“也不很疼,养一两日就好了。”林黛玉坐了一回,闷闷的回房去了。一宿无话。次日,宝玉见了贾母,虽然自己承认是自己烫的,不与别人相干,免不得那贾母又把跟从的人骂一顿。\begin{note}甲戌侧:此原非正文,故草草写去。\end{note}
\end{parag}


\begin{parag}
    过了一日,就有宝玉寄名的干娘马道婆进荣国府来请安。见了宝玉,唬一大跳,问起原由,说是烫的,便点头叹息一回,向宝玉脸上用指头画了一画,口内嘟囔囔的又持诵了一回,说道:“管保就好了,这不过是一时飞灾。”又向贾母道:“祖宗老菩萨那里知道,那经典佛法上说的利害,\begin{note}甲戌侧:一段无伦无理信口开河的混话,却句句都是耳闻目睹者,并非杜撰而有。作者与余实实经过。\end{note}大凡那王公卿相人家的子弟,只一生长下来,暗里便有许多促狭鬼跟著他,得空便拧他一下,或掐他一下,或吃饭时打下他的饭碗来,或走著推他一跤,所以往往的那些大家子孙多有长不大的。”贾母听如此说,便赶著问:“这有什么佛法解释没有呢?”马道婆道:“这个容易,只是替他多作些因果善事也就罢了。再那经上还说,西方有位大光明普照菩萨,专管照耀阴暗邪祟,若有善男子善女子虔心供奉者,可以永佑儿孙康宁安静,再无惊恐邪祟撞客之灾。”贾母道:“倒不知怎么个供奉这位菩萨?”马道婆道:“也不值些什么,不过除香烛供养之外,一天多添几斤香油,点上个大海灯。这海灯,便是菩萨现身法像,昼夜不敢息的。”贾母道:“ 一天一夜也得多少油?明白告诉我,我也好作这件功德的。”马道婆听如此说,便笑道:“这也不拘,随施主菩萨们随心愿舍罢了。像我们庙里,就有好几处的王妃诰命供奉的:南安郡王府里的太妃,他许的多,愿心大,一天是四十八斤油,一斤灯草,\begin{note}甲戌侧:贼婆先用大铺排试之。\end{note}那海灯也只比缸略小些;锦田侯的诰命次一等,一天不过二十四斤油;再还有几家也有五斤的、三斤的、一斤的,都不拘数。那小家子穷人家舍不起这些,就是四两半斤,也少不得替他点。”贾母听了,点头思忖。\begin{note}甲戌眉:“点头思忖”是量事之大小,非吝啬也。日费香油四十八斤,每月油二百五十余斤,合钱三百余串。为一小儿,如何服众?太君细心若是。\end{note}马道婆又道:“还有一件,若是为父母尊亲长上的,多舍些不妨;若是象老祖宗如今为宝玉,若舍多了倒不好,\begin{note}甲戌侧:贼道婆!是自“太君思忖”上来,后用如此数语收之,使太君必心悦诚服愿行。贼婆,贼婆,费我作者许多心机摹写也。\end{note}还怕哥儿禁不起,倒折了福。也不当家花花的,要舍,大则七斤,小则五斤,也就是了。”贾母说:“既是这样说,你便一日五斤合准了,每月打趸来关了去。”马道婆念了一声“阿弥陀佛慈悲大菩萨”。贾母又命人来吩咐:“以后大凡宝玉出门的日子,拿几串钱交给他的小子们带著,遇见僧道穷苦人好舍。”说毕,那马道婆又坐了一回,便又往各院各房问安,闲逛了一回。一时来至赵姨娘房内,\begin{note}甲戌侧:有“各院各房”,接此方不觉突然。\end{note}二人见过,赵姨娘命小丫头倒了茶来与他吃。
\end{parag}


\begin{parag}
    马道婆因见炕上堆著些零碎绸缎湾角,赵姨娘正粘鞋呢。马道婆道:“可是我正没了鞋面子了。\begin{note}甲戌侧:见者有分是也。\end{note}赵奶奶你有零碎缎子,不拘什么颜色的,弄一双鞋面给我。”赵姨娘听说,便叹口气说道:“你瞧瞧那里头,还有那一块是成样的?成了样的东西,也不能到我手里来!有的没的都在这里,你不嫌,就挑两块子去。”马道婆见说,果真便挑了两块袖将起来。
\end{parag}


\begin{parag}
    赵姨娘问道:“前日我送了五百钱去,在药王跟前上供,你可收了没有?”马道婆道:“早已替你上了供了。”赵姨娘叹口气道:“阿弥陀佛!我手里但凡从容些,也时常的上个供,只是心有余力量不足。”马道婆道:“你只管放心,将来熬的环哥儿大了,得个一官半职,那时你要作多大的功德不能?”赵姨娘听说,鼻子里笑了一声,说道:“罢,罢,再别说起。如今就是个样儿,我们娘儿们跟的上这屋里那一个儿!也不是有了宝玉,竟是得了活龙。他还是小孩子家,长的得人意儿,大人偏疼他些也还罢了;\begin{note}甲戌侧:赵妪数语,可知玉兄之身份,况在背后之言。\end{note}我只不伏这个主儿。”\begin{note}甲戌侧:活现赵妪。\end{note}一面说,一面伸出两个指头儿来。\begin{note}甲戌侧:活现阿凤。\end{note}马道婆会意,便问道:“可是琏二奶奶?”赵姨娘唬的忙摇手儿,走到门前,掀帘子向外看看无人,\begin{note}甲戌侧:是心胆俱怕破。\end{note}方进来向马道婆悄悄说道:“了不得,了不得!提起这个主儿,这一分家私要不都叫他搬送到娘家去,我也不是个人。”\begin{note}庚辰侧:这是妒心正题目。\end{note}
\end{parag}


\begin{parag}
    马道婆见他如此说,便探他口气说道:\begin{note}庚辰侧:有隙即入,所谓贼婆,是极!\end{note}“我还用你说,难道都看不出来。也亏你们心里也不理论,只凭他去。倒也妙。”赵姨娘道:“我的娘,不凭他去,难道谁还敢把他怎么样呢?”马道婆听说,鼻子里一笑,\begin{note}庚辰侧:二笑。\end{note}半晌说道:“不是我说句造孽的话,你们没有本事!──也难怪别人。明不敢怎样,暗里也就算计了,\begin{note}甲戌侧:贼婆操必胜之券,赵妪已堕术中,故敢直出明言。可畏可怕!\end{note}还等到这如今!”赵姨娘闻听这话里有道理,心内暗暗的欢喜,便说道:“怎么暗里算计?我倒有这个意思,只是没这样的能干人。你若教给我这法子,我大大的谢你。”马道婆听说这话打拢了一处,便又故意说道:“阿弥陀佛!你快休问我,我那里知道这些事。罪过,罪过。”\begin{note}甲戌侧:远一步却是近一步。贼婆,贼婆!\end{note}赵姨娘道:“你又来了。你是最肯济困扶危的人,难道就眼睁睁的看人家来摆布死了我们娘儿两个不成?难道还怕我不谢你?”马道婆听说如此,便笑道:“若说我不忍叫你娘儿们受人委曲还犹可,若说谢我的这两个字,可是你错打算盘了。就便是我希图你谢,靠你有些什么东西能打动我?”\begin{note}甲戌侧:探谢礼大小是如此说法,可怕可畏!\end{note}赵姨娘听这话口气松动了,便说道:“你这么个明白人,怎么糊涂起来了。你若果然法子灵验,把他两个绝了,明日这家私不怕不是我环儿的。那时你要什么不得?” 马道婆听了,低了头,半晌说道:“那时候事情妥了,又无凭据,你还理我呢!”赵姨娘道:“这又何难。如今我虽手里没什么,也零碎攒了几两梯己,还有几件衣服簪子,你先拿些去。下剩的,我写个欠银子文契给你,你要什么保人也有,那时我照数给你。”马道婆道:“果然这样?”赵姨娘道:“这如何还撒得谎。”说著便叫过一个心腹婆子来,耳根底下嘁嘁喳喳喳说了几句话。\begin{note}甲戌侧:所谓狐群狗党大家难免,看官著眼。\end{note}那婆子出去了,一时回来,果然写了个五百两欠契来。赵姨娘便印了手模,\begin{note}甲戌侧:痴妇,痴妇!\end{note}走到橱柜里将梯己拿了出来,与马道婆看看,道:“这个你先拿了去做香烛供奉使费,可好不好?”马道婆看看白花花的一堆银子,又有欠契,并不顾青红皂白,\begin{note}甲戌侧:有道婆作干娘者来看此句。“并不顾”三字怕杀人。千万件恶事皆从三字生出来。可怕可畏可警,可长存戒之。\end{note}满口里应著,伸手先去抓了银子掖起来,然后收了欠契。又向裤腰里掏了半晌,掏出十个纸铰的青面白发的鬼来,并两个纸人,\begin{note}甲戌侧:如此现成,更可怕。庚辰侧:如此现成,想贼婆所害之人岂止宝玉、阿凤二人哉?大家太君夫人诫之慎之。\end{note}递与赵姨娘,又悄悄的教他道:“把他两个的年庚八字写在这两个纸人身上,一并五个鬼都掖在他们各人的床上就完了。我只在家里作法,自有效验。千万小心,不要害怕!”\begin{note}甲戌眉:宝玉乃贼婆之寄名干儿,一样下此毒手,况阿凤乎?三姑六婆之害如此,即贾母之神明,在所不免。其他只知吃斋念佛之夫人太君,岂能防范的来?此系老太君一大病。作者一片婆心,不避嫌疑,特为写出,使看官再四著眼,吾家儿孙慎之戒之!\end{note}正才说著,只见王夫人的丫鬟进来找道:“奶奶可在这里,太太等你呢。”二人方散了,不在话下。
\end{parag}


\begin{parag}
    却说林黛玉因见宝玉近日烫了脸,总不出门,倒时常在一处说说话儿。这日饭后看了两篇书,自觉无趣,便同紫鹃雪雁做了一回针线,更觉烦闷。便倚著房门出了一回神,\begin{note}甲戌侧:所谓“闲倚绣房吹柳絮”是也。\end{note}信步出来,看阶下新迸出的稚笋,\begin{note}甲戌侧:妙妙!“笋根稚子无人见”,今得颦儿一见,何幸如之。\end{note}
\end{parag}


\begin{parag}
    不觉出了院门。一望园中,四顾无人,\begin{note}甲戌侧:恐冷落圆亭花柳,故有是十数字也。\end{note}惟见花光柳影,鸟语溪声。\begin{note}甲戌侧:纯用画家笔写。\end{note}林黛玉信步便往怡红院中来,只见几个丫头舀水,都在回廊上围著看画眉洗澡呢。\begin{note}甲戌侧:闺中女儿乐事。\end{note}听见房内有笑声,林黛玉便入房中看时,原来是李宫裁、凤姐、宝钗都在这里呢,一见他进来都笑道:“这不又来了一个。”林黛玉笑道:“今儿齐全,谁下帖子请来的?”凤姐道:“前儿我打发了丫头送了两瓶茶叶去,\begin{note}庚辰侧:有照应。\end{note}你往那去了?”林黛玉笑道:“哦,可是倒忘了,\begin{note}甲戌侧:该云“我正看《会真记》呢”。一笑。\end{note}多谢多谢。”凤姐儿又道:“你尝了可还好不好?”没有说完,宝玉便说道:“论理可倒罢了,只是我说不大甚好,也不知别人尝著怎么样。”宝钗道:“味倒轻,只是颜色不大好些。”\begin{note}庚辰眉:二宝答言是补出诸艳俱领过之文。乙酉冬,雪窗。畸笏老人。\end{note}凤姐道:“那是暹罗进贡来的。我尝著也没什么趣儿,还不如我每日吃的呢。”林黛玉道:“我吃著好,\begin{note}甲戌侧:卿爱因味轻也。卿如何担的起味厚之物耶?\end{note}不知你们的脾胃是怎样?”宝玉道:“你果然爱吃,把我这个也拿了去吃罢。”凤姐笑道:“你要爱吃,我那里还有呢。”林黛玉道:“果真的,我就打发丫头取去了。”凤姐道:“不用取去,我打发人送来就是了。我明儿还有一件事求你,一同打发人送来。”林黛玉听了笑道:“你们听听,这是吃了他们家一点子茶叶,就来使唤人了。”凤姐笑道:“倒求你,你倒说这些闲话,吃茶吃水的。你既吃了我们家的茶,怎么还不给我们家作媳妇?”\begin{note}甲戌侧:二玉事在贾府上下诸人即看书人批书人皆信定一段好夫妻,书中常常每每道及,岂具不然,叹叹!\end{note}\begin{note}庚辰侧:二玉之配偶在贾府上下诸人即观者批者作者皆为无疑,故常常有此等点题语。我也要笑。\end{note}众人听了一齐都笑起来。
\end{parag}


\begin{parag}
    林黛玉红了脸,一声儿不言语,便回过头去了。李宫裁笑向宝钗道:“真真我们二婶子的诙谐是好的。”\begin{note}庚辰侧:好赞!该他赞。\end{note}林黛玉道:“什么诙谐,不过是贫嘴贱舌讨人厌恶罢了。”\begin{note}甲戌侧:此句还要候查。\end{note}说著便啐了一口。
\end{parag}


\begin{parag}
    凤姐笑道:“你别作梦!你给我们家作了媳妇,少什么?”指宝玉道:“你瞧瞧,人物儿、门第配不上,\begin{note}甲戌侧:大大一泄,好接后文。\end{note}根基配不上,家私配不上?那一点还玷辱了谁呢?”林黛玉抬身就走。宝钗便叫:“颦儿急了,还不回来坐著。走了倒没意思。”说著便站起来拉住。刚至房门前,只见赵姨娘和周姨娘两个人进来瞧宝玉。李宫裁、宝钗、宝玉等都让他两个坐。独凤姐只和林黛玉说笑,正眼也不看他们。宝钗方欲说话时,只见王夫人房内的丫头来说:“舅太太来了,请奶奶姑娘们出去呢。”李宫裁听了,连忙叫著凤姐等走了。赵、周两个忙辞了宝玉出去。宝玉道:“我也不能出去,你们好歹别叫舅母进来。”又道:“林妹妹,你先略站一站,我说一句话。”凤姐听了,回头向林黛玉笑道:“有人叫你说话呢。”说著便把林黛玉往里一推,和李纨一同去了。
\end{parag}


\begin{parag}
    这里宝玉拉著林黛玉的袖子,只是嘻嘻的笑,\begin{note}庚辰侧:此刻好看之至!\end{note}心里有话,只是口里说不出来。\begin{note}甲戌侧:是已受镇,“说不出来”。勿得错会了意。\end{note}此时林黛玉只是禁不住把脸红涨了,挣著要走。宝玉忽然“嗳哟”了一声,说:“好头疼!”\begin{note}甲戌侧:自黛玉看书起分三段写来,真无容针之空。如夏日乌云四起,疾闪长雷不绝,不知雨落何时,忽然霹雳一声,倾盆大注,何快如之,何乐如之,其令人宁不叫绝!\end{note}林黛玉道:“该,阿弥陀佛!”\begin{note}庚辰眉:黛玉念佛,是吃茶之语在心故也。然摹写神妙,一丝不漏如此。己卯冬夜。\end{note}只见宝玉大叫一声:“我要死!”将身一纵,离地跳有三四尺高,口内乱嚷乱叫,说起胡话来了。林黛玉并丫头们都唬慌了,忙去报知王夫人、贾母等。此时王子腾的夫人也在这里,都一齐来时,宝玉益发拿刀弄杖,寻死觅活的,闹得天翻地覆。贾母、王夫人见了,唬的抖衣而颤,且“儿”一声“肉”一声放声恸哭。于是惊动诸人,连贾赦、邢夫人、贾珍、贾政、贾琏、贾蓉、贾芸、贾萍、薛姨妈、薛蟠并周瑞家的一干家中上上下下里里外外众媳妇丫头等,都来园内看视。登时园内乱麻一般。\begin{note}甲戌侧:写玉兄惊动若许人忙乱,正写太君一人之钟爱耳。看官勿被作者瞒过。\end{note}正没个主见,只见凤姐手持一把明晃晃刚刀砍进园来,见鸡杀鸡,见狗杀狗,见人就要杀人。\begin{note}甲戌双夹:此处焉用鸡犬?然辉煌富丽非处家之常也,鸡犬闲闲始为儿孙千年之业,故于此处必用鸡犬二字,方时一簇腾腾大舍。\end{note}众人越发慌了。周瑞媳妇忙带著几个有力量的胆壮的婆娘上去抱住,夺下刀来,抬回房去。平儿、丰儿等哭的泪天泪地。贾政等心中也有些烦难,顾了这里,丢不下那里。
\end{parag}


\begin{parag}
    别人慌张自不必讲,独有薛蟠更比诸人忙到十分去:\begin{note}甲戌侧:写呆兄忙是愈觉忙中之愈忙,且避正文之絮烦。好笔仗,写得出。\end{note}\begin{note}庚辰侧:写呆兄是躲烦碎文字法。好想头,好笔力。《石头记》最得力处在此。\end{note}又恐薛姨妈被人挤倒,又恐薛宝钗被人瞧见,又恐香菱被人臊皮──知道贾珍等是在女人身上做功夫的,\begin{note}甲戌侧:从阿呆兄意中,又写贾珍一笔,妙!\end{note}因此忙的不堪。忽一眼瞥见了林黛玉风流婉转,已酥倒在那里。\begin{note}甲戌侧:忙到容针不能。此似唐突颦儿,却是写情字万不能禁止者,又可知颦儿之丰神若仙子也。\end{note}\begin{note}甲戌双夹:忙中写闲,真大手眼,大章法。\end{note}
\end{parag}


\begin{parag}
    当下众人七言八语,有的说请端公送祟的,有的说请巫婆跳神的,有的又荐玉皇阁的张真人,种种喧腾不一。也曾百般医治祈祷,问卜求神,总无效验。堪堪日落。王子腾夫人告辞去后,次日王子腾也来瞧问。\begin{note}甲戌侧:写外戚,亦避正文之繁。\end{note}接著小史侯家、邢夫人弟兄辈并各亲戚眷属都来瞧看,也有送符水的,也有荐僧道的,总不见效。他叔嫂二人愈发糊涂,不省人事,睡在床上,浑身火炭一般,口内无般不说。到夜晚间,那些婆娘媳妇丫头们都不敢上前。因此把他二人都抬到王夫人的上房内,\begin{note}甲戌侧:收拾得干净有著落。庚辰侧:收拾得得体正大。\end{note}夜间派了贾芸带著小厮们挨次轮班看守。贾母、王夫人、邢夫人、薛姨妈等寸地不离,只围著干哭。
\end{parag}


\begin{parag}
    此时贾赦、贾政又恐哭坏了贾母,日夜熬油费火,闹的人口不安,也都没了主意。贾赦还各处去寻僧觅道。贾政见不灵效,著实懊恼,\begin{note}甲戌侧:四字写尽政老矣。\end{note}因阻贾赦道:“儿女之数,皆由天命,非人力可强者。他二人之病出于不意,百般医治不效,想天意该当如此,也只好由他们去罢。”\begin{note}甲戌侧:念书人自应如是语。\end{note}贾赦也不理此话,仍是百般忙乱,那里见些效验。看看三日光阴,那凤姐和宝玉躺在床上,亦发连气都将没了。合家人口无不惊慌,都说没了指望,忙著将他二人的后世的衣履都治备下了。贾母、王夫人、贾琏、平儿、袭人这几个人更比诸人哭的忘餐废寝,觅死寻活。赵姨娘、贾环等自是称愿。\begin{note}甲戌侧:补明赵妪进怡红为作法也。\end{note}
\end{parag}


\begin{parag}
    到了第四日早晨,贾母等正围著宝玉哭时,只见宝玉睁开眼说道:\begin{note}甲戌侧:“语不惊人死不休”,此之谓也。\end{note}“从今以后,我可不在你家了!快收拾了,打发我走罢。”贾母听了这话,如同摘心去肝一般。赵姨娘在旁劝道:“老太太也不必过于悲痛。\begin{note}庚辰侧:断不可少此句。\end{note}哥儿已是不中用了,不如把哥儿的衣服穿好,让他早些回去,也免些苦;只管舍不得他,这口气不断,他在那世里也受罪不安生。”\begin{note}庚辰侧:大遂心人必有是语。\end{note}这些话没说完,被贾母照脸啐了一口唾沫,骂道:“烂了舌头的混帐老婆,谁叫你来多嘴多舌的!你怎么知道他在那世里受罪不安生?怎么见得不中用了?你愿他死了,有什么好处?你别做梦!他死了,我只和你们要命。素日都不是你们调唆著逼他写字念书,\begin{note}甲戌双夹:奇语,所谓溺爱者不明,然天生必有是一段文字的。\end{note}把胆子唬破了,见了他老子不象个避猫鼠儿?都不是你们这起淫妇调唆的!这会子逼死了,你们遂了心,我饶那一个!”一面骂,一面哭。贾政在旁听见这些话,心里越发难过,便喝退赵姨娘,自己上来委婉解劝。一时又有人来回说:“两口棺椁都做齐了,\begin{note}甲戌侧:偏写一头不了又一头之文,真步步紧之文。\end{note}请老爷出去看。”贾母听了,如火上浇油一般,便骂:“是谁做了棺椁?”一叠声只叫把做棺椁的拉来打死。
\end{parag}


\begin{parag}
    正闹的天翻地覆,没个开交,只闻得隐隐的木鱼声响,\begin{note}甲戌侧:不费丝毫勉强,轻轻收住数百言文字,《石头记》得力处全在此处。以幻作真,以真作幻,看书人亦要如是看法为幸。\end{note}念了一句:“南无解冤孽菩萨。有那人口不利,家宅颠倾,或逢凶险,或中邪祟者,我们善能医治。”贾母、王夫人听见这些话,那里还耐得住,便命人去快请进来。贾政虽不自在,奈贾母之言如何违拗;想如此深宅,何得听的这样真切,\begin{note}甲戌侧:作者是幻笔,合屋俱是幻耳,焉能无闻?\end{note}心中亦希罕,\begin{note}甲戌侧:政老亦落幻中。\end{note}命人请了进来。众人举目看时,原来是一个癞头和尚与一个跛足道人。\begin{note}甲戌双夹:僧因凤姐,道因宝玉,一丝不乱。\end{note}
\end{parag}


\begin{parag}
    见那和尚是怎的模样:
\end{parag}


\begin{poem}
    \begin{pl}鼻如悬胆两眉长,目似明星蓄宝光,\end{pl}

    \begin{pl}破衲芒鞋无住迹,腌臜更有满头疮。\end{pl}
\end{poem}


\begin{parag}
    那道人又是怎生模样:
\end{parag}


\begin{poem}
    \begin{pl}一足高来一足低,浑身带水又拖泥。\end{pl}

    \begin{pl}相逢若问家何处,却在蓬莱弱水西。\end{pl}
\end{poem}


\begin{parag}
    贾政问道:“你道友二人在那庙里焚修。”那僧笑道:“长官不须多话。\begin{note}甲戌侧:避俗套法。\end{note}因闻得府上人口不利,故特来医治。”贾政道:“倒有两个人中邪,不知你们有何符水?”那道人笑道:“你家现有希世奇珍,如何还问我们有符水?”贾政听这话有意思,心中便动了,因说道:“小儿落草时虽带了一块宝玉下来,上面说能除邪祟,\begin{note}庚辰侧:点题。\end{note}谁知竟不灵验。”那僧道:“长官你那里知道那物的妙用。只因他如今被声色货利所迷,\begin{note}甲戌双夹:石皆能迷,可知其害不小。观者著眼,方可读《石头记》。\end{note}故不灵验了。\begin{note}甲戌侧:读书者观之。\end{note}你今且取他出来,待我们持诵持诵,只怕就好了。”\begin{note}庚辰侧: “只怕”二字,是不知此石肯听持诵否?\end{note}
\end{parag}


\begin{parag}
    贾政听说,便向宝玉项上取下那玉来递与他二人。那和尚接了过来,擎在掌上,长叹一声道:“青埂峰一别,展眼已过十三载矣!\begin{note}庚辰侧:正点题,大荒山手捧时语。\end{note}人世光阴,如此迅速,尘缘满日,若似弹指!\begin{note}甲戌双夹:见此一句,令人可叹可惊,不忍往后再看矣!\end{note}可羡你当时的那段好处:
\end{parag}


\begin{poem}
    \begin{pl}   天不拘兮地不羁,心头无喜亦无悲;\end{pl}
    \begin{note}甲戌侧:所谓越不聪明越快活。\end{note}

    \begin{pl}   却因锻炼通灵后,便向人间觅是非。\end{pl}
\end{poem}


\begin{parag}
    可叹你今日这番经历:
\end{parag}


\begin{poem}
    \begin{pl} 粉渍脂痕污宝光,绮栊昼夜困鸳鸯。\end{pl}

    \begin{pl} 沉酣一梦终须醒,\end{pl}\begin{note}甲戌侧:无百年的筵席。\end{note}\begin{pl}冤孽偿清好散场!”\end{pl}\begin{note}甲戌侧:三次锻炼,焉得不成佛作祖?\end{note}
\end{poem}


\begin{parag}
    念毕,又摩弄一回,说了些疯话,递与贾政道:“此物已灵,不可亵渎,悬于卧室上槛,将他二人安在一室之内,除亲身妻母外,不可使外人冲犯。\begin{note}庚辰侧:是要紧语,是不可不写之套语。\end{note}三十三日之后,包管身安病退,复旧如初。”说著回头便走了。\begin{note}庚辰眉:通灵玉除邪,全部百回只此一见,何得再言?僧道踪迹虚实,幻笔幻想,写幻人于幻文也。壬午孟夏,雨窗。\end{note}贾政赶著还说话,让二人坐了吃茶,要送谢礼,他二人早已出去了。贾母等还只管著人去赶,那里有个踪影。少不得依言将他二人就安放在王夫人卧室之内,将玉悬在门上。王夫人亲身守著,不许别个人进来。
\end{parag}


\begin{parag}
    至晚间他二人竟渐渐醒来,\begin{note}甲戌侧:能领持诵,故如此灵效。\end{note}说腹中饥饿。贾母、王夫人如得了珍宝一般,\begin{note}甲戌侧:昊天罔极之恩如何报得?哭杀幼而丧亲者。\end{note}旋熬了米汤与他二人吃了,精神渐长,邪祟稍退,一家子才把心放下来。\begin{note}甲戌眉:通灵玉听癞和尚二偈即刻灵应,抵却前回若干《庄子》及语录机锋偈子。正所谓物各有所主也。叹不得见玉兄“悬崖撒手”文字为恨。\end{note}李宫裁并贾府三艳、薛宝钗、林黛玉、平儿、袭人等在外间听信息。闻得吃了米汤,省了人事,别人未开口,林黛玉先就念了一声“阿弥陀佛”。\begin{note}甲戌侧:针对得病时那一声。\end{note}薛宝钗便回头看了他半日,嗤的一声笑。众人都不会意,贾惜春道:“宝姐姐,好好的笑什么?”宝钗笑道:“我笑如来佛比人还忙:\begin{note}庚辰侧:这一句作正意看,余皆雅谑,但此一 实 颦儿半部之谑。\end{note}又要讲经说法,又要普渡众生;这如今宝玉,凤姐姐病了,又烧香还愿,赐福消灾;今才好些,又管林姑娘的姻缘了。你说忙的可笑不可笑。”林黛玉不觉的红了脸,啐了一口道:“你们这起人不是好人,不知怎么死!再不跟著好人学,只跟那些贫嘴恶舌的人学。”一面说,一面摔帘子出去了。不知端详,且听下回分解。
\end{parag}


\begin{parag}
    \begin{note}甲戌:先写红玉数行引接正文,是不作开门见山文字。\end{note}
\end{parag}


\begin{parag}
    \begin{note}甲戌:灯油引大光明普照菩萨,大光明普照菩萨引五鬼魇魔法是一线贯成。\end{note}
\end{parag}


\begin{parag}
    \begin{note}甲戌:通灵玉除邪,全部只此一见,却又不灵,遇癞和尚、跛道人一点方灵应矣。写利欲之害如此。\end{note}
\end{parag}


\begin{parag}
    \begin{note}甲戌:此回本意是为禁三姑六婆进门之害,难以防范。\end{note}
\end{parag}


\begin{parag}
    \begin{note}庚辰:此回书因才干乖觉太露,引出事来,作者婆心为世之乖觉人为鉴。\end{note}
\end{parag}


\begin{parag}
    \begin{note}蒙回末总评:欲深魔重复可疑,苦海冤河解者谁?结不休时冤日盛,井天甚小性难移。\end{note}
\end{parag}

