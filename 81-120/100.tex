\chap{一百}{破好事香菱結深恨 悲遠嫁寶玉感離情}



\begin{parag}
    話說賈政去見了節度,進去了半日不見出來,外頭議論不一。李十兒在外也打聽不出什麼事來,便想到報上的饑荒,實在也着急,好容易聽見賈政出來,便迎上來跟着,等不得回去,在無人處便問:“老爺進去這半天,有什麼要緊的事?”賈政笑道:“並沒有事。只爲鎮海總制是這位大人的親戚,有書來囑託照應我,所以說了些好話。又說我們如今也是親戚了。”李十兒聽得,心內喜歡,不免又壯了些膽子,便竭力縱恿賈政許這親事。賈政心想薛蟠的事到底有什麼掛礙,在外頭信息不早,難以打點,故回到本任來便打發家人進京打聽,順便將總制求親之事回明賈母,如若願意,即將三姑娘接到任所。家人奉命趕到京中,回明瞭王夫人,便在吏部打聽得賈政並無處分,惟將署太平縣的這位老爺革職,即寫了稟帖安慰了賈政,然後住着等信。
\end{parag}


\begin{parag}
    且說薛姨媽爲着薛蟠這件人命官司,各衙門內不知花了多少銀錢,才定了誤殺具題。原打量將當鋪折變給人,備銀贖罪。不想刑部駁審,又託人花了好些錢,總不中用,依舊定了個死罪,監着守候秋天大審。薛姨媽又氣又疼,日夜啼哭。寶釵雖時常過來勸解,說是:“哥哥本來沒造化。承受了祖父這些家業,就該安安頓頓的守着過日子。在南邊已經鬧的不象樣,便是香菱那件事情就了不得,因爲仗着親戚們的勢力,花了些銀錢,這算白打死了一個公子。哥哥就該改過做起正經人來,也該奉養母親纔是,不想進了京仍是這樣。媽媽爲他不知受了多少氣,哭掉了多少眼淚。給他娶了親,原想大家安安逸逸的過日子,不想命該如此,偏偏娶的嫂子又是一個不安靜的,所以哥哥躲出門的。真正俗語說的‘冤家路兒狹’,不多幾天就鬧出人命來了。媽媽和二哥哥也算不得不盡心的了,花了銀錢不算,自己還求三拜四的謀幹。無奈命裏應該,也算自作自受。大凡養兒女是爲着老來有靠,便是小戶人家還要掙一碗飯養活母親,那裏有將現成的鬧光了反害的老人家哭的死去活來的?不是我說,哥哥的這樣行爲,不是兒子,竟是個冤家對頭。媽媽再不明白,明哭到夜,夜哭到明,又受嫂子的氣。我呢,又不能常在這裏勸解,我看見媽媽這樣,那裏放得下心。他雖說是傻,也不肯叫我回去。前兒老爺打發人回來說,看見京報唬的了不得,所以才叫人來打點的。我想哥哥鬧了事,擔心的人也不少。幸虧我還是在跟前的一樣,若是離鄉調遠聽見了這個信,只怕我想媽媽也就想殺了。我求媽媽暫且養養神,趁哥哥的活口現在,問問各處的賬目。人家該咱們的,咱們該人家的,亦該請個舊夥計來算一算,看看還有幾個錢沒有。”薛姨媽哭着說道:“這幾天爲鬧你哥哥的事,你來了,不是你勸我,便是我告訴你衙門的事。你還不知道,京裏的官商名字已經退了,兩個當鋪已經給了人家,銀子早拿來使完了。還有一個當鋪,管事的逃了,虧空了好幾千兩銀子,也夾在裏頭打官司。你二哥哥天天在外頭要帳,料着京裏的帳已經去了幾萬銀子,只好拿南邊公分裏銀子並住房折變纔夠。前兩天還聽見一個荒信,說是南邊的公當鋪也因爲折了本兒收了。若是這麼着,你孃的命可就活不成的了。”說着,又大哭起來。寶釵也哭着勸道:“銀錢的事,媽媽操心也不中用,還有二哥哥給我們料理。單可恨這些夥計們,見咱們的勢頭兒敗了,各自奔各自的去也罷了,我還聽見說幫着人家來擠我們的訛頭。可見我哥哥活了這麼大,交的人總不過是些個酒肉弟兄,急難中是一個沒有的。媽媽若是疼我,聽我的話,有年紀的人,自己保重些。媽媽這一輩子。想來還不致挨凍受餓。家裏這點子衣裳傢伙,只好聽憑嫂子去,那是沒法兒的了。所有的家人婆子,瞧他們也沒心在這裏,該去的叫他們去。就可憐香菱苦了一輩子,只好跟着媽媽過去。實在短什麼,我要是有的,還可以拿些個來,料我們那個也沒有不依的。就是襲姑娘也是心術正道的,他聽見我哥哥的事,他倒提起媽媽來就哭。我們那一個還道是沒事的,所以不大着急,若聽見了也是要唬個半死兒的。”薛姨媽不等說完,便說:“好姑娘,你可別告訴他。他爲一個林姑娘幾乎沒要了命,如今纔好了些。要是他急出個原故來,不但你添一層煩惱,我越發沒了依靠了。”寶釵道:“我也是這麼想,所以總沒告訴他。”正說着,只聽見金桂跑來外間屋裏哭喊道:“我的命是不要的了!男人呢,已經是沒有活的分兒了。咱們如今索性鬧一鬧,大夥兒到法場上去拼一拼。”說着。便將頭往隔斷板上亂撞,撞的披頭散髮。氣得薛姨媽白瞪着兩隻眼,一句話也說不出來。還虧得寶釵嫂子長,嫂子短,好一句,歹一句的勸他。金桂道:“姑奶奶,如今你是比不得頭裏的了。你兩口兒好好的過日子,我是個單身人兒,要臉做什麼!”說着,便要跑到街上回孃家去,虧得人還多,扯住了,又勸了半天方住。把個寶琴唬的再不敢見他。若是薛蝌在家,他便抹粉施脂,描眉畫鬢,奇情異致的打扮收拾起來,不時打從薛蝌住房前過,或故意咳嗽一聲,或明知薛蝌在屋,特問房裏何人。有時遇見薛蝌,他便妖妖喬喬,嬌嬌癡癡的問寒問熱,忽喜忽嗔。丫頭們看見,都趕忙躲開。他自己也不覺得,只是一意一心要弄得薛蝌感情時,好行寶蟾之計。那薛蝌卻只躲着,有時遇見,也不敢不周旋一二,只怕他撒潑放刁的意思。更加金桂一則爲色迷心,越瞧越愛,越想越幻,那裏還看得出薛蝌的真假來。只有一宗,他見薛蝌有什麼東西都是託香菱收着,衣服縫洗也是香菱,兩個人偶然說話,他來了,急忙散開,一發動了一個醋字。欲待發作薛蝌,卻是捨不得,只得將一腔隱恨都擱在香菱身上。卻又恐怕鬧了香菱得罪了薛蝌,倒弄得隱忍不發。
\end{parag}


\begin{parag}
    一日,寶蟾走來笑嘻嘻的向金桂道:“奶奶看見了二爺沒有?”金桂道:“沒有。”寶蟾笑道:“我說二爺的那種假正經是信不得的。咱們前日送了酒去,他說不會喝,剛纔我見他到太太那屋裏去,那臉上紅撲撲兒的一臉酒氣。奶奶不信,回來只在咱們院門口等他,他打那邊過來時奶奶叫住他問問,看他說什麼。”金桂聽了,一心的怒氣,便道:“他那裏就出來了呢。他既無情義,問他作什麼!”寶蟾道:“奶奶又迂了。他好說,咱們也好說,他不好說,咱們再另打主意。”金桂聽着有理,因叫寶蟾瞧着他,看他出去了。寶蟾答應着出來。金桂卻去打開鏡奩,又照了一照,把嘴脣兒又抹了一抹,然後拿一條灑花絹子,纔要出來,又似忘了什麼的,心裏倒不知怎麼是好了。只聽寶蟾外面說道:“二爺今日高興呵,那裏喝了酒來了?”金桂聽了,明知是叫他出來的意思,連忙掀起簾子出來。只見薛蝌和寶蟾說道:“今日是張大爺的好日子,所以被他們強不過吃了半鍾,到這時候臉還發燒呢。”一句話沒說完,金桂早接口道:“自然人家外人的酒比咱們自己家裏的酒是有趣兒的。”薛蝌被他拿話一激,臉越紅了,連忙走過來陪笑道:“嫂子說那裏的話。”寶蟾見他二人交談,便躲到屋裏去了。
\end{parag}


\begin{parag}
    這金桂初時原要假意發作薛蝌兩句,無奈一見他兩頰微紅,雙眸帶澀,別有一種謹願可憐之意,早把自己那驕悍之氣感化到爪窪國去了,因笑說道:“這麼說,你的酒是硬強着才肯喝的呢。”薛蝌道:“我那裏喝得來。”金桂道:“不喝也好,強如象你哥哥喝出亂子來,明兒娶了你們奶奶兒,象我這樣守活寡受孤單呢!”說到這裏,兩個眼已經乜斜了,兩腮上也覺紅暈了。薛蝌見這話越發邪僻了,打算着要走。金桂也看出來了,那裏容得,早已走過來一把拉住。薛蝌急了道:“嫂子放尊重些。”說着渾身亂顫。金桂索性老着臉道:“你只管進來,我和你說一句要緊的話。”正鬧着,忽聽背後一個人叫道:“奶奶,香菱來了。”把金桂唬了一跳,回頭瞧時,卻是寶蟾掀着簾子看他二人的光景,一抬頭見香菱從那邊來了,趕忙知會金桂。金桂這一驚不小,手已鬆了。薛蝌得便脫身跑了。那香菱正走着,原不理會,忽聽寶蟾一嚷,才瞧見金桂在那裏拉住薛蝌往裏死拽。香菱卻唬的心頭亂跳,自己連忙轉身回去。這裏金桂早已連嚇帶氣,呆呆的瞅着薛蝌去了。怔了半天,恨了一聲,自己掃興歸房,從此把香菱恨入骨髓。那香菱本是要到寶琴那裏,剛走出腰門,看見這般,嚇回去了。
\end{parag}


\begin{parag}
    是日,寶釵在賈母屋裏聽得王夫人告訴老太太要聘探春一事。賈母說道:“既是同鄉的人,很好。只是聽見那孩子到過我們家裏,怎麼你老爺沒有提起?”王夫人道:“連我們也不知道。”賈母道:“好便好,但是道兒太遠。雖然老爺在那裏,倘或將來老爺調任,可不是我們孩子太單了嗎。”王夫人道:“兩家都是做官的,也是拿不定。或者那邊還調進來,即不然,終有個葉落歸根。況且老爺既在那裏做官,上司已經說了,好意思不給麼?想來老爺的主意定了,只是不做主,故遣人來回老太太的。”賈母道:“你們願意更好。只是三丫頭這一去了,不知三年兩年那邊可能回家?若再遲了,恐怕我趕不上再見他一面了。”說着,掉下淚來。王夫人道:“孩子們大了,少不得總要給人家的。就是本鄉本土的人,除非不做官還使得,若是做官的,誰保得住總在一處。只要孩子們有造化就好。譬如迎姑娘倒配得近呢,偏是時常聽見他被女婿打鬧,甚至不給飯喫。就是我們送了東西去,他也摸不着。近來聽見益發不好了,也不放他回來。兩口子拌起來就說咱們使了他家的銀錢。可憐這孩子總不得個出頭的日子。前兒我惦記他,打發人去瞧他,迎丫頭藏在耳房裏不肯出來。老婆子們必要進去,看見我們姑娘這樣冷天還穿着幾件舊衣裳。他一包眼淚的告訴婆子們說:‘回去別說我這麼苦,這也是命裏所招,也不用送什麼衣服東西來,不但摸不着,反要添一頓打。說是我告訴的。’老太太想想,這倒是近處眼見的,若不好更難受。倒虧了大太太也不理會他,大老爺也不出個頭!如今迎姑娘實在比我們三等使喚的丫頭還不如。我想探丫頭雖不是我養的,老爺既看見過女婿,定然是好才許的。只請老太太示下,擇個好日子,多派幾個人送到他老爺任上。該怎麼着,老爺也不肯將就。”賈母道:“有他老子作主,你就料理妥當,揀個長行的日子送去,也就定了一件事。”王夫人答應着“是”。寶釵聽得明白,也不敢則聲,只是心裏叫苦:“我們家裏姑娘們就算他是個尖兒,如今又要遠嫁,眼看着這裏的人一天少似一天了。”見王夫人起身告辭出去,他也送了出來,一徑回到自己房中,並不與寶玉說話。見襲人獨自一個做活,便將聽見的話說了。襲人也很不受用。
\end{parag}


\begin{parag}
    卻說趙姨娘聽見探春這事,反歡喜起來,心裏說道:“我這個丫頭在家忒瞧不起我,我何從還是個娘,比他的丫頭還不濟。況且洑上水護着別人。他擋在頭裏,連環兒也不得出頭。如今老爺接了去,我倒乾淨。想要他孝敬我,不能夠了。只願意他象迎丫頭似的,我也稱稱願。”一面想着,一面跑到探春那邊與他道喜說:“姑娘,你是要高飛的人了,到了姑爺那邊自然比家裏還好。想來你也是願意的。便是養了你一場,並沒有借你的光兒。就是我有七分不好,也有三分的好,總不要一去了把我擱在腦杓子後頭。”探春聽着毫無道理,只低頭作活,一句也不言語。趙姨娘見他不理,氣忿忿的自己去了。
\end{parag}


\begin{parag}
    這裏探春又氣又笑,又傷心,也不過自己掉淚而已。坐了一回,悶悶的走到寶玉這邊來。寶玉因問道:“三妹妹,我聽見林妹妹死的時候你在那裏來着。我還聽見說,林妹妹死的時候遠遠的有音樂之聲。或者他是有來歷的也未可知。”探春笑道:“那是你心裏想着罷了。只是那夜卻怪,不似人家鼓樂之音。你的話或者也是。”寶玉聽了,更以爲實。又想前日自己神魂飄蕩之時,曾見一人,說是黛玉生不同人,死不同鬼,必是那裏的仙子臨凡。忽又想起那年唱戲做的嫦娥,飄飄豔豔,何等風致。過了一回,探春去了。因必要紫鵑過來,立即回了賈母去叫他。無奈紫鵑心裏不願意,雖經賈母王夫人派了過來,也就沒法,只是在寶玉跟前,不是噯聲,就是嘆氣的。寶玉背地裏拉着他,低聲下氣要問黛玉的話,紫鵑從沒好話回答。寶釵倒背底裏誇他有忠心,並不嗔怪他。那雪雁雖是寶玉娶親這夜出過力的,寶釵見他心地不甚明白,便回了賈母王夫人,將他配了一個小廝,各自過活去了。王奶媽養着他,將來好送黛玉的靈柩回南。鸚哥等小丫頭仍伏侍了老太太。寶玉本想念黛玉,因此及彼,又想跟黛玉的人已經雲散,更加納悶。悶到無可如何,忽又想起黛玉死得這樣清楚,必是離凡返仙去了,反又喜歡。忽然聽見襲人和寶釵那裏講究探春出嫁之事,寶玉聽了,啊呀的一聲,哭倒在炕上。唬得寶釵襲人都來扶起說:“怎麼了?”寶玉早哭的說不出來,定了一回子神,說道:“這日子過不得了!我姊妹們都一個一個的散了!林妹妹是成了仙去了。大姐姐呢已經死了,這也罷了,沒天天在一塊。二姐姐呢,碰着了一個混賬不堪的東西。三妹妹又要遠嫁,總不得見的了。史妹妹又不知要到那裏去。薛妹妹是有了人家的。這些姐姐妹妹,難道一個都不留在家裏,單留我做什麼!”襲人忙又拿話解勸。寶釵擺着手說:“你不用勸他,讓我來問他。”因問着寶玉道:“據你的心裏,要這些姐妹都在家裏陪到你老了,都不要爲終身的事嗎?若說別人,或者還有別的想頭。你自己的姐姐妹妹,不用說沒有遠嫁的,就是有,老爺作主,你有什麼法兒!打量天下獨是你一個人愛姐姐妹妹呢,若是都象你,就連我也不能陪你了。大凡人唸書,原爲的是明理,怎麼你益發糊塗了。這麼說起來,我同襲姑娘各自一邊兒去,讓你把姐姐妹妹們都邀了來守着你。”寶玉聽了,兩隻手拉住寶釵襲人道:“我也知道。爲什麼散的這麼早呢?等我化了灰的時候再散也不遲。”襲人掩着他的嘴道:“又胡說。才這兩天身上好些,二奶奶才喫些飯。若是你又鬧翻了,我也不管了。”寶玉慢慢的聽他兩個人說話都有道理,只是心上不知道怎麼纔好,只得強說道:“我卻明白,但只是心裏鬧的慌。”寶釵也不理他,暗叫襲人快把定心丸給他吃了,慢慢的開導他。襲人便欲告訴探春說臨行不必來辭,寶釵道:“這怕什麼。等消停幾日,待他心裏明白,還要叫他們多說句話兒呢。況且三姑娘是極明白的人,不象那些假惺惺的人,少不得有一番箴諫。他以後便不是這樣了。”正說着,賈母那邊打發過鴛鴦來說,知道寶玉舊病又發,叫襲人勸說安慰,叫他不要胡思亂想。襲人等應了。鴛鴦坐了一會子去了。那賈母又想起探春遠行,雖不備妝奩,其一應動用之物俱該預備,便把鳳姐叫來,將老爺的主意告訴了一遍,即叫他料理去。鳳姐答應,不知怎麼辦理,下回分解。
\end{parag}