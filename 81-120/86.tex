\chap{八十六}{受私贿老官翻案牍 寄闲情淑女解琴书}



\begin{parag}
    话说薛姨妈听了薛蝌的来书,因叫进小厮问道:“你听见你大爷说,到底是怎么就把人打死了呢?”小厮道:“小的也没听真切。那一日大爷告诉二爷说。”说着回头看了一看,见无人,才说道:“大爷说自从家里闹的特利害,大爷也没心肠了,所以要到南边置货去。这日想着约一个人同行,这人在咱们这城南二百多地住。大爷找他去了,遇见在先和大爷好的那个蒋玉菡带着些小戏子进城。大爷同他在个铺子里吃饭喝酒,因为这当槽儿的尽着拿眼瞟蒋玉菡,大爷就有了气了。后来蒋玉菡走了。第二天,大爷就请找的那个人喝酒,酒后想起头一天的事来,叫那当槽儿的换酒,那当槽儿的来迟了,大爷就骂起来了。那个人不依,大爷就拿起酒碗照他打去。谁知那个人也是个泼皮,便把头伸过来叫大爷打。大爷拿碗就砸他的脑袋一下,他就冒了血了,躺在地下,头里还骂,后头就不言语了。”薛姨妈道:“怎么也没人劝劝吗?”那小厮道:“这个没听见大爷说,小的不敢妄言。”薛姨妈道:“你先去歇歇罢。”小厮答应出来。这里薛姨妈自来见王夫人,托王夫人转求贾政。贾政问了前后,也只好含糊应了,只说等薛蝌递了呈子,看他本县怎么批了再作道理。
\end{parag}


\begin{parag}
    这里薛姨妈又在当铺里兑了银子,叫小厮赶着去了。三日后果有回信。薛姨妈接着了,即叫小丫头告诉宝钗,连忙过来看了。只见书上写道:
\end{parag}


\begin{parag}
    带去银两做了衙门上下使费。哥哥在监也不大吃苦,请太太放心。独是这里的人很刁,尸亲见证都不依,连哥哥请的那个朋友也帮着他们。我与李祥两个俱系生地生人,幸找着一个好先生,许他银子,才讨个主意,说是须得拉扯着同哥哥喝酒的吴良,弄人保出他来,许他银两,叫他撕掳。他若不依,便说张三是他打死,明推在异乡人身上,他吃不住,就好办了。我依着他,果然吴良出来。现在买嘱尸亲见证,又做了一张呈子。前日递的,今日批来,请看呈底便知。
\end{parag}


\begin{parag}
    因又念呈底道:
\end{parag}


\begin{qute2sp}
    具呈人某,呈为兄遭飞祸代伸冤抑事。窃生胞兄薛蟠,本籍南京,寄寓西京。于某年月日备本往南贸易。去未数日,家奴送信回家,说遭人命。生即奔宪治,知兄误伤张姓,及至囹圄。据兄泣告,实与张姓素不相认,并无仇隙。偶因换酒角口,生兄将酒泼地,恰值张三低头拾物,一时失手,酒碗误碰卤门身死。蒙恩拘讯,兄惧受刑,承认斗殴致死。仰蒙宪天仁慈,知有冤抑,尚未定案。生兄在禁,具呈诉辩,有干例禁。生念手足,冒死代呈,伏乞宪慈恩准,提证质讯,开恩莫大。生等举家仰戴鸿仁,永永无既矣。激切上呈。
\end{qute2sp}


\begin{parag}
    批的是:
\end{parag}


\begin{qute2sp}
    尸场检验,证据确凿。且并未用刑,尔兄自认斗杀,招供在案。今尔远来,并非目睹,何得捏词妄控。理应治罪,姑念为兄情切,且恕。不准。
\end{qute2sp}


\begin{parag}
    薛姨妈听到那里,说道:“这不是救不过来了么。这怎么好呢!”宝钗道:“二哥的书还没看完,后面还有呢。”因又念道:“有要紧的问来使便知。”薛姨妈便问来人,因说道:“县里早知我们的家当充足,须得在京里谋干得大情,再送一分大礼,还可以复审,从轻定案。太太此时必得快办,再迟了就怕大爷要受苦了。”
\end{parag}


\begin{parag}
    薛姨妈听了,叫小厮自去,即刻又到贾府与王夫人说明原故,恳求贾政。贾政只肯托人与知县说情,不肯提及银物。薛姨妈恐不中用,求凤姐与贾琏说了,花上几千银子,才把知县买通。薛蝌那里也便弄通了。然后知县挂牌坐堂,传齐了一干邻保证见尸亲人等,监里提出薛蟠。刑房书吏俱一一点名。知县便叫地保对明初供,又叫尸亲张王氏并尸叔张二问话。张王氏哭禀道:“小的的男人是张大,南乡里住,十八年前死了。大儿子二儿子也都死了,光留下这个死的儿子叫张三,今年二十三岁,还没有娶女人呢。为小人家里穷,没得养活,在李家店里做当槽儿的。那一天晌午,李家店里打发人来叫俺,说‘你儿子叫人打死了。”我的青天老爷,小的就唬死了。跑到那里,看见我儿子头破血出的躺在地下喘气儿,问他话也说不出来,不多一会儿就死了。小人就要揪住这个小杂种拼命。”众衙役吆喝一声。张王氏便磕头道:“求青天老爷伸冤,小人就只这一个儿子了。”知县便叫下去,又叫李家店的人问道:“那张三是你店内佣工的么?”那李二回道:“不是佣工,是做当槽儿的。”知县道:“那日尸场上你说张三是薛蟠将碗砸死的,你亲眼见的么。”李二说道:“小的在柜上,听见说客房里要酒。不多一回,便听见说‘不好了,打伤了。’小的跑进去,只见张三躺在地下,也不能言语。小的便喊禀地保,一面报他母亲去了。他们到底怎样打的,实在不知道,求太爷问那喝酒的便知道了。”知县喝道:“初审口供,你是亲见的,怎么如今说没有见?”李二道:“小的前日唬昏了乱说。”衙役又吆喝了一声。知县便叫吴良问道:“你是同在一处喝酒的么?薛蟠怎么打的,据实供来。”吴良说:“小的那日在家,这个薛大爷叫我喝酒。他嫌酒不好要换,张三不肯。薛大爷生气把酒向他脸上泼去,不晓得怎么样就碰在那脑袋上了。这是亲眼见的。”知县道:“胡说。前日尸场上薛蟠自己认拿碗砸死的,你说你亲眼见的,怎么今日的供不对?掌嘴。”衙役答应着要打,吴良求着说:“薛蟠实没有与张三打架,酒碗失手碰在脑袋上的。求老爷问薛蟠便是恩典了。”知县叫提薛蟠,问道:“你与张三到底有什么仇隙?毕竟是如何死的,实供上来。”薛蟠道:“求太老爷开恩,小的实没有打他。为他不肯换酒,故拿酒泼他,不想一时失手,酒碗误碰在他的脑袋上。小的即忙掩他的血,那里知道再掩不住,血淌多了,过一回就死了。前日尸场上怕太老爷要打,所以说是拿碗砸他的。只求太爷开恩。”知县便喝道:“好个糊涂东西!本县问你怎么砸他的,你便供说恼他不换酒才砸的,今日又供是失手碰的。”知县假作声势,要打要夹,薛蟠一口咬定。知县叫仵作将前日尸场填写伤痕据实报来。仵作禀报说:“前日验得张三尸身无伤,惟卤门有磁器伤长一寸七分,深五分,皮开,卤门骨脆裂破三分。实系磕碰伤。”知县查对尸格相符,早知书吏改轻,也不驳诘,胡乱便叫画供。张王氏哭喊道:“青天老爷!前日听见还有多少伤,怎么今日都没有了?”知县道:“这妇人胡说,现有尸格,你不知道么。”叫尸叔张二便问道:“你侄儿身死,你知道有几处伤?”张二忙供道:“脑袋上一伤。”知县道:“可又来。”叫书吏将尸格给张王氏瞧去,并叫地保、尸叔指明与他瞧,现有尸场亲押证见俱供并未打架,不为鬪殴。只依误伤吩咐画供。将薛蟠监禁候详,余令原保领出,退堂。张王氏哭着乱嚷,知县叫众衙役撵他出去。张二也劝张王氏道:“实在误伤,怎么赖人。现在太老爷断明,不要胡闹了。”薛蝌在外打听明白,心内喜欢,便差人回家送信。等批详回来,便好打点赎罪,且住着等信。只听路上三三两两传说,有个贵妃薨了,皇上辍朝三日。这里离陵寝不远,知县办差垫道,一时料着不得闲,住在这里无益,不如到监告诉哥哥安心等着,“我回家去,过几日再来。”薛蟠也怕母亲痛苦,带信说:“我无事,必须衙门再使费几次,便可回家了。只是不要可惜银钱。”
\end{parag}


\begin{parag}
    薛蝌留下李祥在此照料,一径回家,见了薛姨妈,陈说知县怎样徇情,怎样审断,终定了误伤,将来尸亲那里再花些银子,一准赎罪,便没事了。薛姨妈听说,暂且放心,说:“正盼你来家中照应。贾府里本该谢去,况且周贵妃薨了,他们天天进去,家里空落落的。我想着要去替姨太太那边照应照应作伴儿,只是咱们家又没人。你这来的正好。”薛蝌道:“我在外头原听见说是贾妃薨了,这么才赶回来的。我们元妃好好儿的,怎么说死了?”薛姨妈道:“上年原病过一次,也就好了。这回又没听见元妃有什么病。只闻那府里头几天老太太不大受用,合上眼便看见元妃娘娘。众人都不放心,直至打听起来,又没有什么事。到了大前儿晚上,老太太亲口说是‘怎么元妃独自一个人到我这里?’众人只道是病中想的话,总不信。老太太又说:‘你们不信,元妃还与我说是荣华易尽,须要退步抽身。’众人都说:‘谁不想到?这是有年纪的人思前想后的心事。’所以也不当件事。恰好第二天早起,里头吵嚷出来说娘娘病重,宣各诰命进去请安。他们就惊疑的了不得,赶着进去。他们还没有出来,我们家里已听见周贵妃薨逝了。你想外头的讹言,家里的疑心,恰碰在一处,可奇不奇!”宝钗道:“不但是外头的讹言舛错,便在家里的,一听见‘娘娘’两个字,也就都忙了,过后才明白。这两天那府里这些丫头婆子来说,他们早知道不是咱们家的娘娘。我说:‘你们那里拿得定呢?’他说道:‘前几年正月,外省荐了一个算命的,说是很准。那老太太叫人将元妃八字夹在丫头们八字里头,送出去叫他推算。他独说这正月初一日生日的那位姑娘只怕时辰错了,不然真是个贵人,也不能在这府中。老爷和众人说,不管他错不错,照八字算去。那先生便说,甲申年正月丙寅这四个字内有伤官败财,惟申字内有正官禄马,这就是家里养不住的,也不见什么好。这日子是乙卯,初春木旺,虽是比肩,那里知道愈比愈好,就象那个好木料,愈经斫削,才成大器。独喜得时上什么辛金为贵,什么巳中正官禄马独旺,这叫作飞天禄马格。又说什么日禄归时,贵重的很,天月二德坐本命,贵受椒房之宠。这位姑娘若是时辰准了,定是一位主子娘娘。这不是算准了么!我们还记得说,可惜荣华不久,只怕遇着寅年卯月,这就是比而又比,劫而又劫,譬如好木,太要做玲珑剔透,本质就不坚了。他们把这些话都忘记了,只管瞎忙。我才想起来告诉我们大奶奶,今年那里是寅年卯月呢。”宝钗尚未说完,薛蝌急道:“且不要管人家的事,既有这样个神仙算命的,我想哥哥今年什么恶星照命,遭这么横祸,快开八字与我给他算去,看有妨碍么。”宝钗道:“他是外省来的,不知如今在京不在了。”
\end{parag}


\begin{parag}
    说着,便打点薛姨妈往贾府去。到了那里,只有李纨探春等在家接着,便问道:“大爷的事怎么样了?”薛姨妈道:“等详上司才定,看来也到不了死罪了。”这才大家放心。探春便道:“昨晚太太想着说,上回家里有事,全仗姨太太照应,如今自己有事,也难提了。心里只是不放心。”薛姨妈道:“我在家里也是难过。只是你大哥遭了事,你二兄弟又办事去了,家里你姐姐一个人,中什么用?况且我们媳妇儿又是个不大晓事的,所以不能脱身过来。目今那里知县也正为预备周贵妃的差事,不得了结案件,所以你二兄弟回来了,我才得过来看看。”李纨便道:“请姨太太这里住几天更好。”薛姨妈点头道:“我也要在这边给你们姐妹们作作伴儿,就只你宝妹妹冷静些。”惜春道:“姨妈要惦着,为什么不把宝姐姐也请过来?”薛姨妈笑着说道:“使不得。”惜春道:“怎么使不得?他先怎么住着来呢?”李纨道:“你不懂的,人家家里如今有事,怎么来呢。”惜春也信以为实,不便再问。正说着,贾母等回来。见了薛姨妈,也顾不得问好,便问薛蟠的事。薛姨妈细述了一遍。宝玉在旁听见什么蒋玉菡一段,当着众人不问,心里打量是”他既回了京,怎么不来瞧我?”又见宝钗也不过来,不知是怎么个原故。心内正自呆呆的想呢,恰好黛玉也来请安。宝玉稍觉心里喜欢,便把想宝钗的念头打断,同着姊妹们在老太太那里吃了晚饭。大家散了,薛姨妈将就住在老太太的套间屋里。
\end{parag}


\begin{parag}
    宝玉回到自己房中,换了衣服,忽然想起蒋玉菡给的汗巾,便向袭人道:“你那一年没有系的那条红汗巾子还有没有?”袭人道:“我搁着呢。问他做什么?”宝玉道:“我白问问。”袭人道:“你没有听见,薛大爷相与这些混账人,所以闹到人命关天。你还提那些作什么?有这样白操心,倒不如静静儿的念念书,把这些个没要紧的事撂开了也好。”宝玉道:“我并不闹什么,偶然想起,有也罢,没也罢,我白问一声,你们就有这些话。”袭人笑道:“并不是我多话。一个人知书达理,就该往上巴结才是。就是心爱的人来了,也叫他瞧着喜欢尊敬啊。”宝玉被袭人一提,便说:“了不得,方才我在老太太那边,看见人多,没有与妹妹说话。他也不曾理我,散的时候他先走了,此时必在屋里。我去就来。”说着就走。袭人道:“快些回来罢,这都是我提头儿,倒招起你的高兴来了。”
\end{parag}


\begin{parag}
    宝玉也不答言,低着头,一径走到潇湘馆来。只见黛玉靠在桌上看书。宝玉走到跟前,笑说道:“妹妹早回来了。”黛玉也笑道:“你不理我,我还在那里做什么!”宝玉一面笑说:“他们人多说话,我插不下嘴去,所以没有和你说话。”一面瞧着黛玉看的那本书。书上的字一个也不认得,有的象“芍”字,有的象“茫”字,也有一个“大”字旁边“九”字加上一勾,中间又添个“五”字,也有上头“五”字“六”字又添一个“木”字,底下又是一个“五”字,看着又奇怪,又纳闷,便说:“妹妹近日愈发进了,看起天书来了。”黛玉嗤的一声笑道:“好个念书的人,连个琴谱都没有见过。”宝玉道:“琴谱怎么不知道,为什么上头的字一个也不认得。妹妹你认得么?”黛玉道:“不认得瞧他做什么?”宝玉道:“我不信,从没有听见你会抚琴。我们书房里挂着好几张,前年来了一个清客先生叫做什么嵇好古,老爷烦他抚了一曲。他取下琴来说,都使不得,还说:‘老先生若高兴,改日携琴来请教。’想是我们老爷也不懂,他便不来了。怎么你有本事藏着?”黛玉道:“我何尝真会呢。前日身上略觉舒服,在大书架上翻书,看有一套琴谱,甚有雅趣,上头讲的琴理甚通,手法说的也明白,真是古人静心养性的工夫。我在扬州也听得讲究过,也曾学过,只是不弄了,就没有了。这果真是‘三日不弹,手生荆棘。’前日看这几篇没有曲文,只有操名。我又到别处找了一本有曲文的来看着,才有意思。究竟怎么弹得好,实在也难。书上说的师旷鼓琴能来风雷龙凤,孔圣人尚学琴于师襄,一操便知其为文王,高山流水,得遇知音。”说到这里,眼皮儿微微一动,慢慢的低下头去。宝玉正听得高兴,便道:“好妹妹,你才说的实在有趣,只是我才见上头的字都不认得,你教我几个呢。”黛玉道:“不用教的,一说便可以知道的。”宝玉道:“我是个糊涂人,得教我那个‘大’字加一勾,中间一个‘五’字的。”黛玉笑道:“这‘大’字‘九’字是用左手大拇指按琴上的九徽,这一勾加‘五’字是右手钩五弦。并不是一个字,乃是一声,是极容易的。还有吟,揉,绰,注,撞,走,飞,推等法,是讲究手法的。”宝玉乐得手舞足蹈的说:“好妹妹,你既明琴理,我们何不学起来。”黛玉道:“琴者,禁也。古人制下,原以治身,涵养性情,抑其淫荡,去其奢侈。若要抚琴,必择静室高斋,或在层楼的上头,在林石的里面,或是山巅上,或是水涯上。再遇着那天地清和的时候,风清月朗,焚香静坐,心不外想,气血和平,才能与神合灵,与道合妙。所以古人说‘知音难遇’。若无知音,宁可独对着那清风明月,苍松怪石,野猿老鹤,抚弄一番,以寄兴趣,方为不负了这琴。还有一层,又要指法好,取音好。若必要抚琴,先须衣冠整齐,或鹤氅,或深衣,要如古人的像表,那才能称圣人之器,然后盥了手,焚上香,方才将身就在榻边,把琴放在案上,坐在第五徽的地方儿,对着自己的当心,两手方从容抬起,这才心身俱正。还要知道轻重疾徐,卷舒自若,体态尊重方好。”宝玉道:“我们学着顽,若这么讲究起来,那就难了。”
\end{parag}


\begin{parag}
    两个人正说着,只见紫鹃进来,看见宝玉笑说道:“宝二爷,今日这样高兴。”宝玉笑道:“听见妹妹讲究的叫人顿开茅塞,所以越听越爱听。”紫鹃道:“不是这个高兴,说的是二爷到我们这边来的话。”宝玉道:“先时妹妹身上不舒服,我怕闹的他烦。再者我又上学,因此显著就疏远了似的。”紫鹃不等说完,便道:“姑娘也是才好,二爷既这么说,坐坐也该让姑娘歇歇儿了,别叫姑娘只是讲究劳神了。”宝玉笑道:“可是我只顾爱听,也就忘了妹妹劳神了。”黛玉笑道:“说这些倒也开心,也没有什么劳神的。只是怕我只管说,你只管不懂呢。”宝玉道:“横竖慢慢的自然明白了。”说着,便站起来道:“当真的妹妹歇歇儿罢。明儿我告诉三妹妹和四妹妹去,叫他们都学起来,让我听。”黛玉笑道:“你也太受用了。即如大家学会了抚起来,你不懂,可不是对——”黛玉说到那里,想起心上的事,便缩住口,不肯往下说了。宝玉便笑道:“只要你们能弹,我便爱听,也不管牛不牛的了。”黛玉红了脸一笑,紫鹃雪雁也都笑了。于是走出门来,只见秋纹带着小丫头捧着一盆兰花来说:“太太那边有人送了四盆兰花来,因里头有事没有空儿顽他,叫给二爷一盆,林姑娘一盆。”黛玉看时,却有几枝双朵儿的,心中忽然一动,也不知是喜是悲,便呆呆的呆看。那宝玉此时却一心只在琴上,便说:“妹妹有了兰花,就可以做《猗兰操》了。”黛玉听了,心里反不舒服。回到房中,看着花,想到”草木当春,花鲜叶茂,想我年纪尚小,便象三秋蒲柳。若是果能随愿,或者渐渐的好来,不然,只恐似那花柳残春,怎禁得风催雨送。”想到那里,不禁又滴下泪来。紫鹃在旁看见这般光景,却想不出原故来。方才宝玉在这里那么高兴,如今好好的看花,怎么又伤起心来。正愁着没法儿解,只见宝钗那边打发人来。未知何事,下回分解。
\end{parag}