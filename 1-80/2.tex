\chap{二}{賈夫人仙逝揚州城 冷子興演說榮國府}

\begin{parag}
    \begin{note}甲庚己:此回亦非正文,本旨只在冷子興一人,即俗謂“冷中出熱,無中生有”也。其演說榮府一篇者,蓋因族大人多,若從作者筆下一一敘出,盡一二回不能得明\begin{subnote}按:己作“明白”\end{subnote},則成何文字?故借用冷字一人,略出其大半\begin{subnote}按:庚“略出其文半”;己作“略出其文”\end{subnote},使閱者心中,已有一榮府隱隱在心,然後用黛玉、寶釵等兩三次皴染,則耀然於心中、眼中矣。此即畫家三染法也。\end{note}
\end{parag}


\begin{parag}
    \begin{note}甲庚己:未寫榮府正人,先寫外戚,是由遠及近\begin{subnote}按:己作“由近及遠”\end{subnote},由小至大也。若使先敘出榮府,然後一一敘及外戚,又一一至朋友、至奴僕,其死板、拮据之筆,豈作十二釵人,手中之物也?今先寫外戚者,正是寫榮國一府也。故又怕閒文贅瘰,開筆即寫賈夫人已死,是特使黛玉入榮\begin{subnote}按:庚、己皆作“榮府”\end{subnote}之速也。\end{note}
\end{parag}


\begin{parag}
    \begin{note}甲庚己:通靈寶玉於士隱夢中一出,今又於子興口中一出,閱者已洞然矣。然後於黛玉、寶釵二人目中極精、極細一描,則是文章鎖合處。蓋不肯一筆直下,有若放閘之水、然信之爆\begin{subnote}按:己作“燃信之爆竹”\end{subnote},使其精華一泄而無餘也。究竟此玉原應出自釵黛目中,方有照應。今預從子興口中說出,實雖寫,而卻未寫。觀其後文,可知此一回則是虛敲傍擊之文,筆則是,反逆隱回\begin{subnote}按:庚、己皆作“反逆隱曲”\end{subnote}之筆。\end{note}
\end{parag}


\begin{parag}
    \begin{note}蒙:以百回之大文,先以此回作兩大筆以帽之,誠是大觀。世態人情,盡盤旋於其間,而一絲不亂,非聚龍象力者,其孰能哉?\end{note}
\end{parag}


\begin{parag}
    詩云:\begin{note}甲行夾:只此一詩便妙極!此等才情,自是雪芹平生所長,餘自謂評書非關評詩也。\end{note}
\end{parag}


\begin{poem}
    \begin{pl}一局輸嬴料不真,香銷茶盡尚逡巡。\end{pl}

    \begin{pl}欲知目下興衰兆,須問傍觀冷眼人。\end{pl}\begin{note}甲眉:故用冷子興演說。\end{note}
\end{poem}


\begin{parag}
    卻說封肅因聽見公差傳喚,忙出來陪笑啓問。那些人只嚷:“快請出甄爺來!”\begin{note}甲側:一絲不亂。\end{note}封肅忙陪笑道:“小人姓封,幷不姓甄。只有當日小婿姓甄,今已出家一二年了,不知可是問他?”那些公人道:“我們也不知什麼真假,\begin{note}甲側:點睛妙筆。\end{note}因奉太爺之命來問。他既是你女婿,便帶了你去親見太爺面稟,省得亂跑。”說著,不容封肅多言,大家推擁他去了。封家人個個都驚慌,不知何兆。
\end{parag}


\begin{parag}
    那天約二更時,只見封肅方回來,歡天喜地。\begin{note}甲側:出自封肅口內,便省卻多少閒文。\end{note}衆人忙問端的。他乃說道:“原來本府新升的太爺姓賈名化,本貫胡州人氏,曾與女婿舊日相交。方纔在咱門前過去,因見嬌杏\begin{note}甲側:僥倖也。託言當日丫頭回顧,故有今日,亦不過偶然僥倖耳,非真實得塵中英傑也。非近日小說中滿紙紅拂紫煙之可比。甲眉:餘批重出。餘閱此書,偶有所得,即筆錄之。非從首至尾閱過復從首加批者,故偶有復處。且諸公之批,自是諸公眼界;脂齋之批,亦有脂齋取樂處。後每一閱,亦必有一語半言,重加批評於側,故又有於前後照應之說等批。\end{note}那丫頭買線,所以他只當女婿移住於此。我一一將原故回明,那太爺倒傷感嘆息了一回,又問外孫女兒,\begin{note}甲側:細。\end{note}我說看燈丟了。太爺說:‘不妨,我自使番役,務必探訪回來。’\begin{note}甲側:爲葫蘆案伏線。\end{note}說了一回話,臨走倒送了我二兩銀子。”甄家娘子聽了,不免心中傷感。\begin{note}甲側:所謂“舊事淒涼不可聞”也。\end{note}一宿無話。
\end{parag}


\begin{parag}
    至次日,早有雨村遣人送了兩封銀子、四匹錦緞,答謝甄家娘子,\begin{note}甲側:雨村已是下流人物,看此,今之如雨村者亦未有矣。\end{note}又寄一封密書與封肅,轉託問甄家娘子要那嬌杏作二房。\begin{note}甲側:謝禮卻爲此。險哉,人之心也!\end{note}封肅喜的屁滾尿流,巴不得去奉承,便在女兒前一力攛掇成了,\begin{note}甲側:一語道盡。\end{note}乘夜只用一乘小轎,便把嬌杏送進去了。雨村歡喜,自不必說,乃封百金贈封肅,外謝甄家娘子許多物事,令其好生養贍,以待尋訪女兒下落。\begin{note}甲側:找前伏後。士隱家一段小枯榮至此結住,所謂真不去假焉來也!\end{note}封肅回家無話。
\end{parag}


\begin{parag}
    卻說嬌杏這丫鬟,便是那年回顧雨村者。因偶然一顧,便弄出這段事來,亦是自己意料不到之奇緣。\begin{note}甲側:註明一筆,更妥當。\end{note}誰想他命運兩濟,\begin{note}甲眉:好極!與英蓮“有命無運”四字,遙遙相映射。蓮,主也;杏,僕也。今蓮反無運,而杏則兩全,可知世人原在運數,不在眼下之高低也。此則大有深意存焉。\end{note}不承望自到雨村身邊,只一年便生了一子,又半載,雨村嫡妻忽染疾下世,雨村便將他扶冊作正室夫人了。正是:
\end{parag}


\begin{poem}
    \begin{pl}偶因一著錯,\end{pl}\begin{note}甲側:妙極!蓋女兒原不應私顧外人之謂。\end{note}

    \begin{pl}便爲人上人。\end{pl}\begin{note}甲側:更妙!可知守禮俟命,終爲俄莩。其調侃寓意不小。甲眉:從來只見集古集唐等句,未見集俗語者。此又更奇之至!\end{note}

\end{poem}


\begin{parag}
    原來,雨村因那年士隱贈銀之後,他於十六日便起身入都。至大比之期,不料他十分得意,已會了進士,選入外班,今已升了本府知府。雖才幹優長,未免有些貪酷之弊,且又恃才侮上,那些官員皆側目而視。\begin{note}甲側:此亦奸雄必有之理。\end{note}不上一年,便被上司尋了個空隙,作成一本,參他“生情狡猾,擅纂禮儀,且沽清正之名,而暗結虎狼之屬,致使地方多事,民命不堪”\begin{note}甲側:此亦奸雄必有之事。\end{note}等語。龍顏大怒,即批革職。該部文書一到,本府官員無不喜悅。那雨村心中雖十分慚恨,卻面上全無一點怨色,仍是嘻笑自若。\begin{note}甲側:此亦奸雄必有之態。\end{note}交代過公事,將歷年做官積的些資本幷家小人屬送至原籍,安排妥協,\begin{note}甲側:先雲“根基已盡”,故今用此四字,細甚!\end{note}卻是自已擔風袖月,遊覽天下勝蹟。\begin{note}甲側:已伏下至金陵一節矣。\end{note}
\end{parag}


\begin{parag}
    那日,偶又遊至維揚地面,因聞得今歲鹽政點的是林如海。這林如海姓林名海,表字如海。\begin{note}甲側:蓋雲“學海文林”也。總是暗寫黛玉。\end{note}乃是前科的探花,今已升至蘭臺寺大夫,\begin{note}甲眉:官制半遵古名亦好。餘最喜此等半有半無,半古半今,事之所無,理之必有,極玄極幻,荒唐不經之處。\end{note}本貫姑蘇\begin{note}甲側:十二釵正出之地,故用真。\end{note}人氏,今欽點出爲巡鹽御史,到任方一月有餘。
\end{parag}


\begin{parag}
    原來這林如海之祖,曾襲過列侯,今到如海,業經五世。起初時,只封襲三世,因當今隆恩盛德,遠邁前代,\begin{note}甲眉:可笑近時小說中,無故極力稱揚浪子淫女,臨收結時,還必致感動朝廷,使君父同入其情慾之界,明遂其意,何無人心之至!不知彼作者有何好處,有何謝!報到朝廷高廟之上,直將半生淫朽穢資睿德,又苦拉君父作一干證護身符,強媒硬保,得遂其淫慾哉!\end{note}額外加恩,至如海之父,又襲了一代;至如海,便從科第出身。雖系鐘鼎之家,卻亦是書香\begin{note}甲側:要緊二字,蓋鐘鼎亦必有書香方至美。\end{note}之族。只可惜這林家支庶不盛,子孫有限,雖有幾門,卻與如海俱是堂族而已,沒甚親支嫡派的。\begin{note}甲側:總爲黛玉極力一寫。\end{note}今如海年已四十,只有一個三歲之子,偏又於去歲死了。雖有幾房姬妾,\begin{note}甲側:帶寫賢妻。\end{note}奈他命中無子,亦無可如何之事。今只有嫡妻賈氏,生得一女,乳名黛玉,年方五歲。夫妻無子,故愛如珍寶,且又見他聰明清秀,\begin{note}甲側:看他寫黛玉,只用此四字。可笑近來小說中,滿紙“天下無二”“古今無雙”等字。\end{note}便也欲使他讀書識得幾個字,不過假充養子之意,聊解膝下荒涼之嘆。\begin{note}甲眉:如此敘法,方是至情至理之妙文。最可笑者,近小說中滿紙班昭蔡琰、文君道韞。\end{note}
\end{parag}


\begin{parag}
    雨村正值偶感風寒,病在旅店,將一月光景方漸愈。一因身體勞倦,二因盤費不繼,也正欲尋個合式之處,暫且歇下。幸有兩個舊友,亦在此境居住,\begin{note}甲側:寫雨村自得意後之交識也。又爲冷子興作引。\end{note}因聞得鹽政欲聘一西賓,雨村便相托友力,謀了進去,且作安身之計。妙在只一個女學生,幷兩個伴讀丫鬟,這女學生年又小,身體又極怯弱,工課不限多寡,故十分省力。
\end{parag}


\begin{parag}
    堪堪又是一載的光陰,誰知女學生之母賈氏夫人一疾而終。女學生侍湯奉藥,守喪盡哀,遂又將辭館別圖。林如海意欲令女學生守制讀書,故又將他留下。近因女學生哀痛過傷,本自怯弱多病,\begin{note}甲側:又一染。\end{note}觸犯舊症,遂連日不曾上學。\begin{note}甲眉:上半回已終,寫“仙逝”正爲黛玉也。故一句帶過,恐閒文有妨正筆。\end{note}雨村閒居無聊,每當風日晴和,飯後便出來閒步。
\end{parag}


\begin{parag}
    這日,偶至郭外,意欲賞鑑那村野風光。\begin{note}甲眉:大都世人意料此,終不能此;不及彼者,而反及彼。故特書意在村野風光,卻忽遇見子興一篇榮國繁華氣象。\end{note}忽信步至一山環水旋、茂林深竹之處,隱隱的有座廟宇,門巷傾頹,牆垣朽敗,門前有額,題著“智通寺”三字,\begin{note}甲側:誰爲智者?又誰能通?一嘆。\end{note}門旁又有一副舊破的對聯,曰:
\end{parag}


\begin{poem}
    \begin{pl}身後有餘忘縮手,眼前無路想回頭。\end{pl}\begin{note}甲夾:先爲寧、榮諸人當頭一喝,卻是爲餘一喝。\end{note}
\end{poem}


\begin{parag}
    雨村看了,因想到:這兩句話,文雖淺近,其意則深。\begin{note}甲側:一部書之總。\end{note}我也曾遊過些名山大剎,倒不曾見過這話頭,其中想必有個翻過筋斗來的亦未可知,\begin{note}甲側:隨筆帶出禪機,又爲後文多少語錄不落空。\end{note}何不進去試試?想著走入,只有一個龍鍾老僧在那裏煮粥。\begin{note}甲側:是雨村火氣。\end{note}雨村見了,便不在意。\begin{note}甲側:火氣。\end{note}及至問他兩句話,那老僧既聾且昏,\begin{note}甲側:是翻過來的。\end{note}齒落舌鈍,\begin{note}甲側:是翻過來的。\end{note}所答非所問。
\end{parag}


\begin{parag}
    雨村不耐煩,便仍出來,\begin{note}甲眉:畢竟雨村還是俗眼,只能識得阿鳳、寶玉、黛玉等未覺之先,卻不識得既證之後。甲眉:未出寧、榮繁華盛處,卻先寫一荒涼小景;未寫通部入世迷人,卻先寫一出世醒人。迴風舞雪,倒峽逆波,別小說中所無之法。\end{note}意欲到那村肆中沽飲三杯,以助野趣,於是款步行來,將入肆門,只見座上喫酒之客有一人起身大笑,接了出來,口內說:“奇遇,奇遇!”雨村忙看時,此人是都中在古董行中貿易的號冷子興者,\begin{note}甲側:此人不過借爲引繩,不必細寫。\end{note}舊日在都相識。雨村最贊這冷子興是個有作爲大本領的人,這子興又借雨村斯文之名,故二人說話投機,最相契合。雨村忙笑問道:“老兄何日到此?弟竟不知。今日偶遇,真奇緣也。”子興道:“去年歲底到家,今因還要入都,從此順路找個敝友說一句話,承他之情,留我多住兩日。我也無緊事,且盤桓兩日,待月半時也就起身了。今日敝友有事,我因閒步至此,且歇歇腳。不期這樣巧遇!”一面說,一面讓雨村同席坐了,另整上酒餚來。二人閒談漫飲,敘些別後之事。\begin{note}甲側:好!若多談則累贅。\end{note}
\end{parag}


\begin{parag}
    雨村因問:“近日都中可有新聞沒有?”\begin{note}甲側:不突然,亦常問常答之言。\end{note}子興道:“倒沒有什麼新聞,倒是老先生你貴同宗家,\begin{note}甲側:雨村已無族中矣,何及此耶?看他下文。\end{note}出了一件小小的異事。”雨村笑道:“弟族中無人在都,何談及此?”子興笑道:“你們同姓,豈非同宗一族?”雨村問是誰家。
\end{parag}


\begin{parag}
    子興道:“榮國府賈府中,可也不玷辱了先生的門楣了?”\begin{note}甲側:刳小人之心肺,聞小人之口角。\end{note}雨村笑道:“原來是他家。若論起來,寒族人丁卻不少,自東漢賈復以來,支派繁盛,各省皆有,\begin{note}甲側:此話縱真,亦必謂是雨村欺人語。\end{note}誰逐細考查得來?若論榮國一支,卻是同譜。但他那等榮耀,我們不便去攀扯,至今故越發生疏難認了。”子興嘆\begin{note}甲側:嘆得怪。\end{note}道:“老先生休如此說。如今的這寧、榮兩門,也都蕭疏了,不比先時的光景。”\begin{note}甲側:記清此句。可知書中之榮府已是末世了。\end{note}雨村道:“當日寧榮兩宅的人口也極多,如何就蕭疏了?”\begin{note}甲側:作者之意原只寫末世,此已是賈府之末世了。\end{note}冷子興道:“正是,說來也話長。”雨村道:“去歲我到金陵地界,因欲遊覽六朝遺蹟,那日進了石頭城,\begin{note}甲側:點睛神妙。\end{note}從他老宅門前經過。街東是寧國府,街西是榮國府,二宅相連,竟將大半條街佔了。大門前雖冷落無人,\begin{note}甲側:好!寫出空宅。\end{note}隔著圍牆一望,裏面廳殿樓閣,也還都崢嶸軒峻,就是後\begin{note}甲側:“後”字何不直用“西”字?甲側:恐先生墮淚,故不敢用“西”字。\end{note}一帶花園子裏面樹木山石,也還都有蓊蔚洇潤之氣,那裏像個衰敗之家?”
\end{parag}


\begin{parag}
    冷子興笑道:“虧你是進士出身,原來不通!古人有云:‘百足之蟲,死而不僵。’如今雖說不及先年那樣興盛,較之平常仕宦之家,到底氣象不同。如今生齒日繁,事務日盛,主僕上下,安富尊榮者盡多,運籌謀畫者無一,\begin{note}甲側:二語乃今古富貴世家之大病。\end{note}其日用排場費用,又不能將就省儉,如今外面的架子雖未甚倒,\begin{note}甲側:“甚”字好!蓋已半倒矣。\end{note}內囊卻也盡上來了。這還是小事,更有一件大事。誰知這樣鐘鳴鼎食之家,翰墨詩書之族,\begin{note}甲側:兩句寫出榮府。\end{note}如今的兒孫,竟一代不如一代了!”\begin{note}甲眉:文是極好之文,理是必有之理,話則極痛極悲之話。\end{note}雨村聽說,也納罕道:“這樣詩禮之家,豈有不善教育之理?別門不知,只說這寧、榮二宅,是最教子有方的。”\begin{note}甲側:一轉有力。\end{note}
\end{parag}


\begin{parag}
    子興嘆道:“正說的是這兩門呢。待我告訴你。當日寧國公\begin{note}甲側:演。\end{note}與榮國公\begin{note}甲側:源。\end{note}是一母同胞弟兄兩個。寧公居長,生了四個兒子。\begin{note}甲側:賈薔、賈菌之祖,不言可知矣。\end{note}寧公死後,賈代化襲了官,\begin{note}甲側:第二代。\end{note}也養了兩個兒子。長名賈敷,至八九歲上便死了,只剩了次子賈敬襲了官,\begin{note}甲側:第三代。\end{note}如今一味好道,只愛燒丹鍊汞,\begin{note}甲側:亦是大族末世常有之事。嘆嘆!\end{note}餘者一概不在心上。幸而早年留下一子,名喚賈珍,\begin{note}甲側:第四代。\end{note}因他父親一心想作神仙,把官倒讓他襲了。他父親又不肯回原籍來,只在都中城外和道士們胡羼。這位珍爺倒生了一個兒子,今年才十六歲,名叫賈蓉。\begin{note}甲側:至蓉五代。\end{note}如今敬老爹一概不管。這珍爺那裏肯讀書,只一味高樂不了,把寧國府竟翻了過來,也沒有人敢來管他。\begin{note}甲側:伏後文。\end{note}再說榮府你聽,方纔所說異事,就出在這裏。自榮公死後,長子賈代善襲了官,\begin{note}甲側:第二代。\end{note}娶的也是金陵世勳史侯家的小姐\begin{note}甲側:因湘雲,故及之。\end{note}爲妻,生了兩個兒子:長子賈赦,次子賈政。\begin{note}甲側:第三代。\end{note}如今代善早已去世,太夫人\begin{note}甲側:記真,湘雲祖姑史氏太君也。\end{note}尚在。長子賈赦襲著官。\begin{note}[伏下賈璉鳳姐當家之文。]\end{note}次子賈政,自幼酷喜讀書,祖父最疼。原欲以科甲出身的,不料代善臨終時遺本一上,皇上因恤先臣,即時令長子襲官外,問還有几子,立刻引見,遂額外賜了這政老爹一個主事之銜,\begin{note}甲側:嫡真實事,非妄擬也。\end{note}令其入部習學,如今現已升了員外郎了。\begin{note}甲側:總是稱功頌德。\end{note}這政老爹的夫人王氏,\begin{note}甲側:記清。\end{note}頭胎生的公子,名喚賈珠,十四歲進學,不到二十歲就娶了妻生了子,\begin{note}甲側:此即賈蘭也。至蘭第五代。\end{note}一病死了。\begin{note}甲側:略可望者即死,嘆嘆!\end{note}第二胎生了一位小姐,生在大年初一,這就奇了,不想後來又生一位公子,\begin{note}甲眉:一部書中第一人卻如此淡淡帶出,故不見後來玉兄文字繁難。\end{note}說來更奇,一落胎胞,嘴裏便銜下一塊五彩晶瑩的玉來,上面還有許多字跡,\begin{note}甲側:青埂頑石已得下落。\end{note}就取名叫作寶玉。你道是新奇異事不是?”\begin{note}正是寧、榮二處支譜。\end{note}
\end{parag}


\begin{parag}
    雨村笑道:“果然奇異。只怕這人來歷不小。”
\end{parag}


\begin{parag}
    子興冷笑道:“萬人皆如此說,因而乃祖母便先愛如珍寶。那年週歲時,政老爹便要試他將來的志向,便將那世上所有之物擺了無數,與他抓取。誰知他一概不取,伸手只把些脂粉釵環抓來。政老爹便大怒了,說:‘將來酒色之徒耳!’因此便大不喜悅。獨那史老太君還是命根一樣。說來又奇,如今長了七八歲,雖然淘氣異常,但其聰明乖覺處,百個不及他一個。說起孩子話來也奇怪,他說:‘女兒是水作的骨肉,男人是泥作的骨肉。\begin{note}甲側:真千古奇文奇情。\end{note}我見了女兒,我便清爽;見了男子,便覺濁臭逼人。’你道好笑不好笑?將來色鬼無移了!”\begin{note}甲側:沒有這一句,雨村如何罕然厲色,幷後奇奇怪怪之論?\end{note}雨村罕然厲色忙止道:“非也!可惜你們不知道這人來歷。大約政老前輩也錯以淫魔色鬼看待了。若非多讀書識事,加以致知格物之功,悟道參玄之力,不能知也。”
\end{parag}


\begin{parag}
    子興見他說得這樣重大,忙請教其端。雨村道:“天地生人,除大仁大惡兩種,餘者皆無大異。若大仁者,則應運而生,大惡者,則應劫而生。運生世治,劫生世危。堯,舜,禹,湯,文,武,周,召,孔,孟,董,韓,周,程,張,朱,皆應運而生者。蚩尤,共工,桀,紂,始皇,王莽,曹操,桓溫,安祿山,秦檜等,皆應劫而生者。\begin{note}甲側:此亦略舉大概幾人而言。\end{note}大仁者,修治天下;大惡者,撓亂天下。清明靈秀,天地之正氣,仁者之所秉也;殘忍乖僻,天地之邪氣,惡者之所秉也。今當運隆祚永之朝,太平無爲之世,清明靈秀之氣所秉者,上至朝廷,下及草野,比比皆是。所餘之秀氣,漫無所歸,遂爲甘露,爲和風,洽然溉及四海。彼殘忍乖僻之邪氣,不能蕩溢於光天化日之中,遂凝結充塞於深溝大壑之內,偶因風蕩,或被雲催,略有搖動感發之意,一絲半縷誤而泄出者,偶值靈秀之氣適過,正不容邪,邪復妒正,\begin{note}甲側:譬得好。\end{note}兩不相下,亦如風水雷電,地中既遇,既不能消,又不能讓,必至搏擊掀發後始盡。故其氣亦必賦人,發泄一盡始散。使男女偶秉此氣而生者,在上則不能成仁人君子,下亦不能爲大凶大惡。\begin{note}甲側:恰極,是確論。\end{note}置之於萬萬人中,其聰俊靈秀之氣,則在萬萬人之上,其乖僻邪謬不近人情之態,又在萬萬人之下。若生於公侯富貴之家,則爲情癡情種,若生於詩書清貧之族,則爲逸士高人,縱再偶生於薄祚寒門,斷不能爲走卒健僕,甘遭庸人驅制駕馭,必爲奇優名倡。如前代之許由、陶潛、阮籍、嵇康、劉伶、王謝二族、顧虎頭、陳後主、唐明皇、宋徽宗、劉庭芝、溫飛卿、米南宮、石曼卿、柳耆卿、秦少游,近日之倪雲林、唐伯虎、祝枝山,再如李龜年、黃幡綽、敬新磨、卓文君、紅拂、薛濤、崔鶯、朝雲之流。此皆易地則同之人也。”
\end{parag}


\begin{parag}
    子興道:“依你說,成則王侯敗則賊了?\begin{note}甲側:《女仙外史》中論魔道已奇,此又非《外史》之立意,故覺愈奇。\end{note}”雨村道:“正是這意。你還不知,我自革職以來,這兩年遍遊各省,也曾遇見兩個異樣孩子。\begin{note}甲側:先虛陪一個。\end{note}所以,方纔你一說這寶玉,我就猜著了八九亦是這一派人物。不用遠說,只金陵城內,欽差金陵省體仁院總裁\begin{note}甲側:此銜無考,亦因寓懷而設,置而勿論。\end{note}甄家,\begin{note}甲眉:又一真正之家,特與假家遙對,故寫假則知真。\end{note}你可知麼?”子興道:“誰人不知!這甄府和賈府就是老親,又繫世交。兩家來往,極其親熱的。便在下也和他家來往非止一日了。”\begin{note}甲側:說大話之走狗,畢真。\end{note}雨村笑道:“去歲我在金陵,也曾有人薦我到甄府處館。我進去看其光景,誰知他家那等顯貴,卻是個富而好禮之家,\begin{note}甲側:如聞其聲。甲眉:只一句便是一篇世家傳,與子興口中是兩樣。\end{note}倒是個難得之館。但這一個學生,雖是啓蒙,卻比一個舉業的還勞神。說起來更可笑,他說:‘必得兩個女兒伴著我讀書,我方能認得字,心裏也明白,不然我自己心裏糊塗。’\begin{note}甲側:甄家之寶玉乃上半部不寫者,故此處極力表明,以遙照賈家之寶玉,凡寫賈家之寶玉,則正爲真寶玉傳影。蒙側:靈玉卻只一塊,而寶玉有兩個,情性如一,亦如六耳、悟空之意耶?\end{note}又常對跟他的小廝們說:‘這女兒兩個字,極尊貴,極清淨的,比那阿彌陀佛,元始天尊的這兩個寶號還更尊榮無對的呢!\begin{note}甲眉:如何只以釋、老二號爲譬,略不敢及我先師儒聖等人?餘則不敢以頑劣目之。\end{note}你們這濁口臭舌,萬不可唐突了這兩個字,要緊。但凡要說時,必須先用清水香茶\begin{note}甲側:恭敬。\end{note}漱了口才可,設若失錯,\begin{note}甲側:罪過。\end{note}便要鑿牙穿腮等事。’其暴虐浮躁,頑劣憨癡,種種異常。只一放了學,進去見了那些女兒們,其溫厚和平,聰敏文雅,\begin{note}甲側:與前八個字嫡對。\end{note}竟又變了一個。因此,他令尊也曾下死笞楚過幾次,無奈竟不能改。每打的喫疼不過時,他便‘姐姐’‘妹妹’亂叫起來。\begin{note}甲眉:以自古未聞之奇語,故寫成自古未有之奇文。此是一部書中大調侃寓意處。蓋作者實因鶺鴒之悲、棠棣之威,故撰此閨閣庭幃之傳。\end{note}後來聽得裏面女兒們拿他取笑:‘因何打急了只管叫姐妹做甚?莫不是求姐妹去說情討饒?你豈不愧些!’他回答的最妙。他說:‘急疼之時,只叫“姐姐”“妹妹”字樣,或可解疼也未可知,因叫了一聲,便果覺不疼了,遂得了祕法。每疼痛之極,便連叫姐妹起來了。’你說可笑不可笑?也因祖母溺愛不明,每因孫辱師責子,因此我就辭了館出來。如今在這巡鹽御史林家做館了。你看,這等子弟,必不能守祖父之根基,從師長之規諫的。只可惜他家幾個姊妹都是少有的。”\begin{note}甲側:實點一筆,餘謂作者必有。\end{note}
\end{parag}


\begin{parag}
    子興道:“便是賈府中,現有的三個也不錯。政老爹的長女,名元\begin{note}甲側:原也。\end{note}春,現因賢孝才德,選入宮作女史\begin{note}甲側:因漢以前例,妙!\end{note}去了。二小姐乃赦老爹之妾所出,名迎\begin{note}甲側:應也。\end{note}春,三小姐乃政老爹之庶出,名探\begin{note}甲側:嘆也。\end{note}春,四小姐乃寧府珍爺之胞妹,名喚惜\begin{note}甲側:息也。\end{note}春。因史老夫人極愛孫女,都跟在祖母這邊一處讀書,聽得個個不錯。”\begin{note}復接前文未及,正詞源三疊。\end{note}雨村道:“更妙在甄家的風俗,女兒之名,亦皆從男子之名命字,不似別家另外用這些春紅香玉等艶字的,何得賈府亦樂此俗套?”
\end{parag}


\begin{parag}
    子興道:“不然,只因現今大小姐是正月初一日所生,故名元春,餘者方從了春字。上一輩的,卻也是從兄弟而來的。現有對證:目今你貴東家林公之夫人,即榮府中赦、政二公之胞妹,在家時名喚賈敏。不信時,你回去細訪可知。”雨村拍案笑道:“怪道這女學生讀至凡書中有‘敏’字,皆唸作‘密’字,每每如是;寫字遇著‘敏’字,又減一二筆,我心中就有些疑惑。今聽你說的,是爲此無疑矣。怪道我這女學生言語舉止另是一樣,不與近日女子相同,度其母必不凡,方得其女,今知爲榮府之孫,又不足罕矣。可傷上月竟亡故了。”子興嘆道:“老姊妹四個,這一個是極小的,又沒了。長一輩的姊妹,一個也沒了。只看這小一輩的,將來之東牀如何呢。”
\end{parag}


\begin{parag}
    雨村道:“正是,方纔說這政公,已有銜玉之兒,又有長子所遺一個弱孫。這赦老竟無一個不成?”子興道:“政公既有玉兒之後,其妾又生了一個,\begin{note}甲側:帶出賈環。\end{note}倒不知其好歹。隻眼前現有二子一孫,卻不知將來如何。若問那赦公,也有二子。長名賈璉,今已二十來往了。親上作親,娶的就是政老爹夫人王氏之內侄女,\begin{note}甲側:另出熙鳳一人。\end{note}今已娶了二年。這位璉爺身上現捐的是個同知,也是不肯讀書,於世路上好機變,言談去的,所以如今只在乃叔政老爺家住著,幫著料理些家務。誰知自娶了他令夫人之後,倒上下無一人不稱頌他夫人的,璉爺倒退了一射之地。說模樣又極標緻,言談又爽利,心機又極深細,竟是個男人萬不及一的。”\begin{note}甲側:未見其人,先已有照。甲眉:非警幻案下而來爲誰?\end{note}
\end{parag}


\begin{parag}
    雨村聽了,笑道:“可知我前言不謬。\begin{note}甲側:略一總住。\end{note}你我方纔所說的這幾個人,都只怕是那正邪兩賦而來一路之人,未可知也。”子興道:“邪也罷,正也罷,只顧算別人家的帳,你也喫一杯酒纔好。”雨村道:“正是,只顧說話,竟多吃了幾杯。”子興笑道:“說著別人家的閒話,正好下酒,\begin{note}甲側:蓋雲此一段話亦爲世人茶酒之笑談耳。\end{note}即多喫幾杯何妨。”雨村向窗外看\begin{note}甲側:畫。\end{note}道:“天也晚了,仔細關了城。我們慢慢的進城再談,未爲不可。”於是,二人起身,算還酒帳。\begin{note}甲側:不得謂此處收得索然,蓋原非正文也。\end{note}
\end{parag}


\begin{parag}
    方欲走時,又聽得後面有人叫道:“雨村兄,恭喜了!特來報個喜信的。”
\end{parag}


\begin{parag}
    \begin{note}甲側:此等套頭,亦不得不用。\end{note}雨村忙回頭看時——\begin{note}己夾:語言太煩,令人不耐。古人云“惜墨如金”,看此視墨如土矣,雖演至千萬回亦可也。\end{note}
\end{parag}


\begin{parag}
    \begin{note}蒙、戚:先自寫幸遇之情於前,而敘藉口談幻境之情於後。世上不平事,道路口如碑。雖作者之苦心,亦人情之必有。雨村之遇嬌杏,是此文之總帽,故在前。冷子興之談,是事蹟之總帽,故敘寫於後。冷暖世情,比比如畫。\end{note}
\end{parag}


\begin{parag}
    \begin{note}蒙、戚:有情原比無情苦,生死相關總在心。也是前緣天作合,何妨黛玉淚淋淋。\end{note}
\end{parag}

