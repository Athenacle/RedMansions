\chap{九十三}{甄家僕投靠賈家門 水月庵掀翻風月案}



\begin{parag}
    卻說馮紫英去後,賈政叫門上人來吩咐道:“今兒臨安伯那裏來請喫酒,知道是什麼事?”門上的人道:“奴才曾問過,並沒有什麼喜慶事。不過南安王府裏到了一班小戲子,都說是個名班。伯爺高興,唱兩天戲請相好的老爺們瞧瞧,熱鬧熱鬧。大約不用送禮的。”說着,賈赦過來問道:“明兒二老爺去不去?”賈政道:“承他親熱,怎麼好不去的。”說着,門上進來回道:“衙門裏書辦來請老爺明日上衙門,有堂派的事,必得早些去。”賈政道:“知道了。”說着,只見兩個管屯裏地租子的家人走來,請了安,磕了頭,旁邊站着。賈政道:“你們是郝家莊的?”兩個答應了一聲。賈政也不往下問,竟與賈赦各自說了一回話兒散了。家人等秉着手燈送過賈赦去。
\end{parag}


\begin{parag}
    這裏賈璉便叫那管租的人道:“說你的。”那人說道:“十月裏的租子奴才已經趕上來了,原是明兒可到。誰知京外拿車,把車上的東西不由分說都掀在地下。奴才告訴他說是府裏收租子的車,不是買賣車。他更不管這些。奴才叫車伕只管拉着走,幾個衙役就把車伕混打了一頓,硬扯了兩輛車去了。奴才所以先來回報,求爺打發個人到衙門裏去要了來纔好。再者,也整治整治這些無法無天的差役纔好。爺還不知道呢,更可憐的是那買賣車,客商的東西全不顧,掀下來趕着就走。那些趕車的但說句話,打的頭破血出的。”賈璉聽了,罵道:“這個還了得!”立刻寫了一個帖兒,叫家人:“拿去向拿車的衙門裏要車去,並車上東西。若少了一件,是不依的。快叫周瑞。”周瑞不在家。又叫旺兒,旺兒晌午出去了,還沒有回來。賈璉道:“這些忘八羔子,一個都不在家!他們終年家喫糧不管事。”因吩咐小廝們:“快給我找去。”說着,也回到自己屋裏睡下。不提。
\end{parag}


\begin{parag}
    且說臨安伯第二天又打發人來請。賈政告訴賈赦道:“我是衙門裏有事,璉兒要在家等候拿車的事情,也不能去,倒是大老爺帶寶玉應酬一天也罷了。”賈赦點頭道:“也使得。”賈政遣人去叫寶玉,說”今兒跟大爺到臨安伯那裏聽戲去。”寶玉喜歡的了不得,便換上衣服,帶了焙茗,掃紅,鋤藥三個小子出來,見了賈赦,請了安,上了車,來到臨安伯府裏。門上人回進去,一會子出來說:“老爺請。”於是賈赦帶着寶玉走入院內,只見賓客喧闐。賈赦寶玉見了臨安伯,又與衆賓客都見過了禮。大家坐着說笑了一回。只見一個掌班的拿着一本戲單,一個牙笏,向上打了一個千兒,說道:“求各位老爺賞戲。”先從尊位點起,挨至賈赦,也點了一出。那人回頭見了寶玉,便不向別處去,竟搶步上來打個千兒道:“求二爺賞兩出。”寶玉一見那人,面如傅粉,脣若塗朱,鮮潤如出水芙蕖,飄揚似臨風玉樹。原來不是別人,就是蔣玉菡。前日聽得他帶了小戲兒進京,也沒有到自己那裏。此時見了,又不好站起來,只得笑道:“你多早晚來的?”蔣玉菡把手在自己身子上一指,笑道:“怎麼二爺不知道麼?”寶玉因衆人在坐,也難說話,只得胡亂點了一出。蔣玉菡去了,便有幾個議論道:“此人是誰?”有的說:“他向來是唱小旦的,如今不肯唱小旦,年紀也大了,就在府裏掌班。頭裏也改過小生。他也攢了好幾個錢,家裏已經有兩三個鋪子,只是不肯放下本業,原舊領班。”有的說:“想必成了家了。”有的說:“親還沒有定。他倒拿定一個主意,說是人生配偶關係一生一世的事,不是混鬧得的,不論尊卑貴賤,總要配的上他的才能。所以到如今還並沒娶親。”寶玉暗忖度道:“不知日後誰家的女孩兒嫁他。要嫁着這樣的人材兒,也算是不辜負了。”那時開了戲,也有崑腔,也有高腔,也有弋腔梆子腔,做得熱鬧。
\end{parag}


\begin{parag}
    過了晌午,便擺開桌子喫酒。又看了一回,賈赦便欲起身。臨安伯過來留道:“天色尚早,聽見說蔣玉菡還有一出《佔花魁》,他們頂好的首戲。”寶玉聽了,巴不得賈赦不走。於是賈赦又坐了一會。果然蔣玉菡扮着秦小官伏侍花魁醉後神情,把這一種憐香惜玉的意思,做得極情盡致。以後對飲對唱,纏綿繾綣。寶玉這時不看花魁,只把兩隻眼睛獨射在秦小官身上。更加蔣玉菡聲音響亮,口齒清楚,按腔落板,寶玉的神魂都唱了進去了。直等這齣戲進場後,更知蔣玉菡極是情種,非尋常戲子可比。因想着《樂記》上說的是”情動於中,故形於聲。聲成文謂之音。”所以知聲,知音,知樂,有許多講究。聲音之原,不可不察。詩詞一道,但能傳情,不能入骨,自後想要講究講究音律。寶玉想出了神,忽見賈赦起身,主人不及相留。寶玉沒法,只得跟了回來。到了家中,賈赦自回那邊去了,寶玉來見賈政。
\end{parag}


\begin{parag}
    賈政才下衙門,正向賈璉問起拿車之事。賈璉道:“今兒門人拿帖兒去,知縣不在家。他的門上說了:這是本官不知道的,並無牌票出去拿車,都是那些混賬東西在外頭撒野擠訛頭。既是老爺府裏的,我便立刻叫人去追辦,包管明兒連車連東西一併送來,如有半點差遲,再行稟過本官,重重處治。此刻本官不在家,求這裏老爺看破些,可以不用本官知道更好。”賈政道:“既無官票,到底是何等樣人在那裏作怪?”賈璉道:“老爺不知,外頭都是這樣。想來明兒必定送來的。”賈璉說完下來,寶玉上去見了。賈政問了幾句,便叫他往老太太那裏去。
\end{parag}


\begin{parag}
    賈璉因爲昨夜叫空了家人,出來傳喚,那起人多已伺候齊全。賈璉罵了一頓,叫大管家賴升:“將各行檔的花名冊子拿來,你去查點查點。寫一張諭帖,叫那些人知道:若有並未告假,私自出去,傳喚不到,貽誤公事的,立刻給我打了攆出去!”賴升連忙答應了幾個”是”,出來吩咐了一回。家人各自留意。
\end{parag}


\begin{parag}
    過不幾時,忽見有一個人頭上載着氈帽,身上穿着一身青布衣裳,腳下穿着一雙撒鞋,走到門上向衆人作了個揖。衆人拿眼上上下下打諒了他一番,便問他是那裏來的。那人道:“我自南邊甄府中來的。並有家老爺手書一封,求這裏的爺們呈上尊老爺。”衆人聽見他是甄府來的,才站起來讓他坐下道:“你乏了,且坐坐,我們給你回就是了。”門上一面進來回明賈政,呈上來書。賈政拆書看時,上寫着:
\end{parag}


\begin{qute2sp}
    世交夙好,氣誼素敦。遙仰襜帷,不勝依切。弟因菲材獲譴,自分萬死難償,幸邀寬宥,待罪邊隅,迄今門戶凋零,家人星散。所有奴子包勇,向曾使用,雖無奇技,人尚愨實。倘使得備奔走,餬口有資,屋烏之愛,感佩無涯矣。專此奉達,餘容再敘。不宣。
\end{qute2sp}


\begin{parag}
    賈政看完,笑道:“這裏正因人多,甄家倒薦人來,又不好卻的。”吩咐門上:“叫他見我。且留他住下,因材使用便了。”門上出去,帶進人來。見賈政便磕了三個頭,起來道:“家老爺請老爺安。”自己又打個千兒說:“包勇請老爺安。”賈政回問了甄老爺的好,便把他上下一瞧。但見包勇身長五尺有零,肩背寬肥,濃眉爆眼,磕額長髯,氣色粗黑,垂着手站着。便問道:“你是向來在甄家的,還是住過幾年的?”包勇道:“小的向在甄家的。”賈政道:“你如今爲什麼要出來呢?”包勇道:“小的原不肯出來。只是家爺再四叫小的出來,說是別處你不肯去,這裏老爺家裏只當原在自己家裏一樣的,所以小的來的。”賈政道:“你們老爺不該有這事情,弄到這樣的田地。”包勇道:“小的本不敢說,我們老爺只是太好了,一味的真心待人,反倒招出事來。”賈政道:“真心是最好的了。”包勇道:“因爲太真了,人人都不喜歡,討人厭煩是有的。”賈政笑了一笑道:“既這樣,皇天自然不負他的。”包勇還要說時,賈政又問道:“我聽見說你們家的哥兒不是也叫寶玉麼?”包勇道:“是。”賈政道:“他還肯向上巴結麼?”包勇道:“老爺若問我們哥兒,倒是一段奇事。哥兒的脾氣也和我家老爺一個樣子,也是一味的誠實。從小兒只管和那些姐妹們在一處頑,老爺太太也狠打過幾次,他只是不改。那一年太太進京的時候兒,哥兒大病了一場,已經死了半日,把老爺幾乎急死,裝裹都預備了。幸喜後來好了,嘴裏說道,走到一座牌樓那裏,見了一個姑娘領着他到了一座廟裏,見了好些櫃子,裏頭見了好些冊子。又到屋裏,見了無數女子,說是多變了鬼怪似的,也有變做骷髏兒的。他嚇急了,便哭喊起來。老爺知他醒過來了,連忙調治,漸漸的好了。老爺仍叫他在姐妹們一處頑去,他竟改了脾氣了,好着時候的頑意兒一概都不要了,惟有唸書爲事。就有什麼人來引誘他,他也全不動心。如今漸漸的能夠幫着老爺料理些家務了。”賈政默然想了一回,道:“你去歇歇去罷。等這裏用着你時,自然派你一個行次兒。”包勇答應着退下來,跟着這裏人出去歇息。不提。
\end{parag}


\begin{parag}
    一日賈政早起剛要上衙門,看見門上那些人在那裏交頭接耳,好象要使賈政知道的似的,又不好明回,只管咕咕唧唧的說話。賈政叫上來問道:“你們有什麼事,這麼鬼鬼祟祟的?”門上的人回道:“奴才們不敢說。”賈政道:“有什麼事不敢說的?”門上的人道:“奴才今兒起來開門出去,見門上貼着一張白紙,上寫着許多不成事體的字。”賈政道:“那裏有這樣的事,寫的是什麼?”門上的人道:“是水月庵裏的醃髒話。”賈政道:“拿給我瞧。”門上的人道:“奴才本要揭下來,誰知他貼得結實,揭不下來,只得一面抄一面洗。剛纔李德揭了一張給奴才瞧,就是那門上貼的話。奴才們不敢隱瞞。”說着呈上那帖兒。賈政接來看時,上面寫着:
\end{parag}


\begin{poem}
    \begin{pl}
        西貝草斤年紀輕,水月庵裏管尼僧。
    \end{pl}


    \begin{pl}
        一個男人多少女,窩娼聚賭是陶情。
    \end{pl}


    \begin{pl}
        不肖子弟來辦事,榮國府內出新聞。
    \end{pl}
\end{poem}


\begin{parag}
    賈政看了,氣得頭昏目暈,趕着叫門上的人不許聲張,悄悄叫人往寧榮兩府靠近的夾道子牆壁上再去找尋。隨即叫人去喚賈璉出來。
\end{parag}


\begin{parag}
    賈璉即忙趕至。賈政忙問道:“水月庵中寄居的那些女尼女道,向來你也查考查考過沒有?”賈璉道:“沒有。一向都是芹兒在那裏照管。”賈政道:“你知道芹兒照管得來照管不來?”賈璉道:“老爺既這麼說,想來芹兒必有不妥當的地方兒。”賈政嘆道:“你瞧瞧這個帖兒寫的是什麼。”賈璉一看,道:“有這樣事麼。”正說着,只見賈蓉走來,拿着一封書子,寫着“二老爺密啓”。打開看時,也是無頭榜一張,與門上所貼的話相同。賈政道:“快叫賴大帶了三四輛車子到水月庵裏去,把那些女尼女道士一齊拉回來。不許泄漏,只說裏頭傳喚。”賴大領命去了。
\end{parag}


\begin{parag}
    且說水月庵中小女尼女道士等初到庵中,沙彌與道士原系老尼收管,日間教他些經懺。以後元妃不用,也便習學得懶怠了。那些女孩子們年紀漸漸的大了,都也有個知覺了。更兼賈芹也是風流人物,打量芳官等出家只是小孩子性兒,便去招惹他們。那知芳官竟是真心,不能上手,便把這心腸移到女尼女道士身上。因那小沙彌中有個名叫沁香的和女道士中有個叫做鶴仙的,長得都甚妖嬈,賈芹便和這兩個人勾搭上了。閒時便學些絲絃,唱個曲兒。那時正當十月中旬,賈芹給庵中那些人領了月例銀子,便想起法兒來,告訴衆人道:“我爲你們領月錢不能進城,又只得在這裏歇着。怪冷的,怎麼樣?我今兒帶些果子酒,大家喫着樂一夜好不好?”那些女孩子都高興,便擺起桌子,連本庵的女尼也叫了來,惟有芳官不來。賈芹喝了幾杯,便說道要行令。沁香等道:“我們都不會,到不如搳拳罷。誰輸了喝一杯,豈不爽快。”本庵的女尼道:“這天剛過晌午,混嚷混喝的不象。且先喝幾盅,愛散的先散去,誰愛陪芹大爺的,回來晚上儘子喝去,我也不管。”正說着,只見道婆急忙進來說:“快散了罷,府裏賴大爺來了。”衆女尼忙亂收拾,便叫賈芹躲開。賈芹因多喝了幾杯,便道:“我是送月錢來的,怕什麼!”話猶未完,已見賴大進來,見這般樣子,心裏大怒。爲的是賈政吩咐不許聲張,只得含糊裝笑道:“芹大爺也在這裏呢麼。”賈芹連忙站起來道:“賴大爺,你來作什麼?”賴大說:“大爺在這裏更好。快快叫沙彌道士收拾上車進城,宮裏傳呢。”賈芹等不知原故,還要細問。賴大說:“天已不早了,快快的好趕進城。”衆女孩子只得一齊上車,賴大騎着大走騾押着趕進城。不題。
\end{parag}


\begin{parag}
    卻說賈政知道這事,氣得衙門也不能上了,獨坐在內書房嘆氣。賈璉也不敢走開。忽見門上的進來稟道:“衙門裏今夜該班是張老爺,因張老爺病了,有知會來請老爺補一班。”賈政正等賴大回來要辦賈芹,此時又要該班,心裏納悶,也不言語。賈璉走上去說道:“賴大是飯後出去的,水月庵離城二十來裏,就趕進城也得二更天。今日又是老爺的幫班,請老爺只管去。賴大來了,叫他押着,也別聲張,等明兒老爺回來再發落。倘或芹兒來了,也不用說明,看他明兒見了老爺怎麼樣說。”賈政聽來有理,只得上班去了。
\end{parag}


\begin{parag}
    賈璉抽空纔要回到自己房中,一面走着,心裏抱怨鳳姐出的主意,欲要埋怨,因他病着,只得隱忍,慢慢的走着。且說那些下人一人傳十傳到裏頭。先是平兒知道,即忙告訴鳳姐。鳳姐因那一夜不好,懨懨的總沒精神,正是惦記鐵檻寺的事情。聽說外頭貼了匿名揭帖的一句話,嚇了一跳,忙問貼的是什麼。平兒隨口答應,不留神就錯說了道:“沒要緊,是饅頭庵裏的事情。”鳳姐本是心虛,聽見饅頭庵的事情,這一唬直唬怔了,一句話沒說出來,急火上攻,眼前發暈,咳嗽了一陣,哇的一聲,吐出一口血來。平兒慌了,說道:“水月庵裏不過是女沙彌女道士的事,奶奶着什麼急。”鳳姐聽是水月庵,才定了定神,說道:“呸,糊塗東西,到底是水月庵呢,是饅頭庵?”平兒笑道:“是我頭裏錯聽了是饅頭庵,後來聽見不是饅頭庵,是水月庵。我剛纔也就說溜了嘴,說成饅頭庵了。”鳳姐道:“我就知道是水月庵,那饅頭庵與我什麼相干。原是這水月庵是我叫芹兒管的,大約克扣了月錢。”平兒道:“我聽着不象月錢的事,還有些醃髒話呢。”鳳姐道:“我更不管那個。你二爺那裏去了?”平兒說:“聽見老爺生氣,他不敢走開。我聽見事情不好,我吩咐這些人不許吵嚷,不知太太們知道了麼。但聽見說老爺叫賴大拿這些女孩子去了。且叫個人前頭打聽打聽。奶奶現在病着,依我竟先別管他們的閒事。”正說着,只見賈璉進來。鳳姐欲待問他,見賈璉一臉的怒氣,暫且裝作不知。賈璉飯沒喫完,旺兒來說:“外頭請爺呢,賴大回來了。”賈璉道:“芹兒來了沒有?”旺兒道:“也來了。”賈璉便道:“你去告訴賴大,說老爺上班兒去了。把這些個女孩子暫且收在園裏,明日等老爺回來送進宮去。只叫芹兒在內書房等着我。”旺兒去了。
\end{parag}


\begin{parag}
    賈芹走進書房,只見那些下人指指點點,不知說什麼。看起這個樣兒來,不象宮裏要人。想着問人,又問不出來。正在心裏疑惑,只見賈璉走出來。賈芹便請了安,垂手侍立,說道:“不知道娘娘宮裏即刻傳那些孩子們做什麼,叫侄兒好趕。幸喜侄兒今兒送月錢去還沒有走,便同着賴大來了。二叔想來是知道的。”賈璉道:“我知道什麼!你纔是明白的呢。”賈芹摸不着頭腦兒,也不敢再問。賈璉道:“你幹得好事,把老爺都氣壞了。”賈芹道:“侄兒沒有幹什麼。庵裏月錢是月月給的,孩子們經懺是不忘記的。”賈璉見他不知,又是平素常在一處頑笑的,便嘆口氣道:“打嘴的東西,你各自去瞧瞧罷!”便從靴掖兒裏頭拿出那個揭帖來,扔與他瞧。賈芹拾來一看,嚇的面如土色,說道:“這是誰幹的!我並沒得罪人,爲什麼這麼坑我!我一月送錢去,只走一趟,並沒有這些事。若是老爺回來打着問我,侄兒便死了。我母親知道,更要打死。”說着,見沒人在旁邊,便跪下去說道:“好叔叔,救我一救兒罷!”說着,只管磕頭,滿眼淚流。賈璉想道:“老爺最惱這些,要是問準了有這些事,這場氣也不小。鬧出去也不好聽,又長那個貼帖兒的人的志氣了。將來咱們的事多着呢。倒不如趁着老爺上班兒,和賴大商量着,若混過去,就可以沒事了。現在沒有對證。”想定主意,便說:“你別瞞我,你乾的鬼鬼祟祟的事,你打諒我都不知道呢。若要完事,就是老爺打着問你,你一口咬定沒有才好。沒臉的,起去罷!”叫人去喚賴大。不多時,賴大來了。賈璉便與他商量。賴大說:“這芹大爺本來鬧的不象了。奴才今兒到庵裏的時候,他們正在那裏喝酒呢。帖兒上的話是一定有的。”賈璉道:“芹兒你聽,賴大還賴你不成。”賈芹此時紅漲了臉,一句也不敢言語。還是賈璉拉着賴大,央他:“護庇護庇罷,只說是芹哥兒在家裏找來的。你帶了他去,只說沒有見我。明日你求老爺也不用問那些女孩子了,竟是叫了媒人來,領了去一賣完事。果然娘娘再要的時候兒咱們再買。”賴大想來,鬧也無益,且名聲不好,就應了。賈璉叫賈芹:“跟了賴大爺去罷,聽着他教你。你就跟着他。”說罷,賈芹又磕了一個頭,跟着賴大出去。到了沒人的地方兒,又給賴大磕頭。賴大說:“我的小爺,你太鬧的不象了。不知得罪了誰,鬧出這個亂兒。你想想誰和你不對罷。”賈芹想了一想,忽然想起一個人來。未知是誰,下回分解。
\end{parag}