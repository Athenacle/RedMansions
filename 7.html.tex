\chap{七}{送宫花周瑞叹英莲 谈肄业秦钟结宝玉}
\begin{parag}
    \begin{note}蒙:苦尽甘来递转,正强忽弱谁明?惺惺自古惜惺惺,时运文章操劲。无缝机关难见,多少笔墨偏精。有情情处特无情,何是人人不醒?\end{note}
\end{parag}


\begin{parag}
    \begin{note}靖:他小说中一笔作两三笔者、一事启两事者均曾见之。岂有似“送花”一回间三带四攒花簇锦之文哉?\end{note}
\end{parag}


\begin{parag}
    题曰:
\end{parag}


\begin{poem}
    \begin{pl}十二花容色最新,不知谁是惜花人?\end{pl}

    \begin{pl}相逢若问名何氏,家住江南本姓秦。\end{pl}
\end{poem}


\begin{parag}
    话说周瑞家的送了刘姥姥去后,便上来回王夫人话。\begin{note}甲戌侧:不回凤姐,却回王夫人,不交代处,正交代得清楚。\end{note}谁知王夫人不在上房,问丫鬟们时,方知往薛姨妈那边闲话去了。\begin{note}甲戌侧:文章只是随笔写来,便有流离生动之妙。\end{note}周瑞家的听说,便转出东角门至东院,往梨香院来。刚至院门前,只见王夫人的丫鬟名金钏儿\begin{note}甲戌侧:金钏、宝钗互相映射。妙!\end{note}者,和一个才留了头的小女孩儿\begin{note}甲戌侧:莲卿别来无恙否?\end{note}站在台阶坡上顽。见周瑞家的来了,便知有话回,因向内努嘴儿。\begin{note}甲戌侧:画。\end{note}周瑞家的轻轻掀帘进去,只见王夫人和薛姨妈长篇大套的说些家务人情等语。
\end{parag}


\begin{parag}
    周瑞家的不敢惊动,遂进里间来。\begin{note}甲戌双夹:总用双歧岔路之笔,令人估料不到之文。\end{note}只见薛宝钗\begin{note}甲戌侧:自入梨香院,至此方写。\end{note}穿著家常衣服,\begin{note}甲戌双夹:好!写一人换一副笔墨,另出一花样。甲戌眉:“家常爱著旧衣裳”是也。\end{note}头上只散挽著䰖儿,坐在炕边里,伏在小炕桌上同丫鬟莺儿正描花样子呢。\begin{note}甲戌侧:一幅《绣窗仕女图》,亏想得周到。\end{note}见他进来,宝钗才放下笔,转过身来,满面堆笑让:“周姐姐坐。”周瑞家的也忙陪笑问:“姑娘好?”一面炕沿上坐了,因说:“这有两三天也没见姑娘到那边逛逛去,只怕是你宝兄弟冲撞了你不成?”\begin{note}甲戌侧:一人不漏,一笔不板。\end{note}宝钗笑道:“那里的话。只因我那种病又发了,\begin{note}甲戌眉:“那种病”“那”字,与前二玉“不知因何”二“又”字,皆得天成地设之体;且省却多少闲文,所谓“惜墨如金”是也。\end{note}所以这两天没出屋子。”\begin{note}甲戌侧:得空便入。\end{note}周瑞家的道:“正是呢,姑娘到底有什么病根儿,也该趁早儿请个大夫来,好生开个方子,认真吃几剂,一势儿除了根才是。小小的年纪倒作下个病根儿,也不是顽的。”宝钗听了便笑道:“再不要提吃药,为这病请大夫吃药,也不知白花了多少银子钱呢。凭你什么名医仙药,从不见一点儿效。后来还亏了一个秃头和尚,\begin{note}甲戌侧:奇奇怪怪,真云龙作雨,忽隐忽见,使人逆料不到。\end{note}说专治无名之症,因请他看了。他说我这是从胎里带来的一股热毒,\begin{note}甲戌侧:凡心偶炽,是以孽火齐攻。\end{note}幸而先天壮,还不相干。\begin{note}甲戌侧:浑厚故也,假使颦、凤辈,不知又何如治之。\end{note}若吃寻常药,是不中用的。他就说了一个海上方,又给了一包药末子作引子,异香异气的。不知是那里弄了来的。他说发了时吃一丸就好。倒也奇怪,吃他的药倒效验些。”\begin{note}甲戌双夹:卿不知从那里弄来,余则深知是从放春山采来,以灌愁海水和成,烦广寒玉兔捣碎,在太虚幻境空灵殿上炮制配合者也。\end{note}
\end{parag}


\begin{parag}
    周瑞家的因问:“不知是个什么海上方儿?姑娘说了,我们也记著,说与人知道,倘遇见这样病,也是行好的事。”宝钗见问,乃笑道:“不用这方儿还好,若用了这方儿,真真把人琐碎死。东西药料一概都有限,只难得‘可巧’二字:要春天开的白牡丹花蕊十二两,\begin{note}甲戌侧:凡用“十二”字样,皆照应十二钗。\end{note}夏天开的白荷花蕊十二两,秋天的白芙蓉蕊十二两,冬天的白梅花蕊十二两。将这四样花蕊,于次年春分这日晒干,和在药末子一处,一齐研好。又要雨水这日的雨水十二钱,……”周瑞家的忙道:“嗳哟!这么说来,这就得三年的工夫。倘或雨水这日竟不下雨,这却怎处呢?”宝钗笑道:“所以说那里有这样可巧的雨,便没雨也只好再等罢了。白露这日的露水十二钱,霜降这日的霜十二钱,小雪这日的雪十二钱。把这四样水调匀,和了药,再加十二钱蜂蜜,十二钱白糖,丸了龙眼大的丸子,盛在旧磁坛内,埋在花根底下。若发了病时,拿出来吃一丸,用十二分黄柏煎汤送下。”\begin{note}甲戌双夹:末用黄柏更妙。可知“甘苦”二字,不独十二钗,世皆同有者。\end{note}
\end{parag}


\begin{parag}
    周瑞家的听了笑道:“阿弥陀佛,真坑死人的事儿!等十年未必都这样巧的呢。”宝钗道:“竟好,自他说了去后,一二年间可巧都得了,好容易配成一料。如今从南带至北,现在就埋在梨花树底下呢。”\begin{note}甲戌侧:“梨香”二字有著落,并未白白虚设。\end{note}周瑞家的又问道:“这药可有名子没有呢?”宝钗道:“有。\begin{note}甲戌侧:一字句。\end{note}这也是那癞头和尚说下的。叫作‘冷香丸’。”\begin{note}甲戌侧:新雅奇甚。\end{note}周瑞家的听了点头儿,因又说:“这病发了时到底觉怎么著?”宝钗道:“也不觉甚怎么著,只不过喘嗽些,吃一丸下去也就好些了。”\begin{note}甲戌双夹:以花为药,可是吃烟火人想得出者?诸公且不必问其事之有无,只据此新奇妙文悦我等心目,便当浮一大白。\end{note}
\end{parag}


\begin{parag}
    周瑞家的还欲说话时,忽听王夫人问:“谁在房里呢?”周瑞家的忙出去答应了,趁便回了刘姥姥之事。略待半刻,见王夫人无语,方欲退出,\begin{note}甲戌双夹:行文原只在一二字,便有许多省力处。不得此窍者,便在窗下百般扭捏。\end{note}薛姨妈忽又笑道:\begin{note}甲戌双夹:“忽”字“又”字与“方欲”二字对射。\end{note}“你且站住。我有一宗东西,你带了去罢。”说著便叫香菱。\begin{note}甲戌双夹:二字仍从“莲”上起来。盖“英莲”者,“应怜”也,“香菱”者亦“相怜”之意。此是改名之“英莲”也。\end{note}只听帘栊响处,方才和金钏顽的那个小丫头进来了,问:“奶奶叫我作什么?”\begin{note}甲戌双夹:这是英莲天生成的口气,妙甚!\end{note}薛姨妈道:“把匣子里的花儿拿来。”香菱答应了,向那边捧了个小锦匣来。薛姨妈道:“这是宫里头的新鲜样法,拿纱堆的花儿十二支。昨儿我想起来,白放著可惜了儿的,何不给他们姊妹们戴去。昨儿要送去,偏又忘了。你今儿来的巧,就带了去罢。你家的三位姑娘,每人一对,剩下的六枝,送林姑娘两枝,那四枝给了凤哥罢。”\begin{note}甲戌侧:妙文!今古小说中可有如此口吻者?\end{note}王夫人道:“留著给宝丫头戴罢了,又想著他们。”薛姨妈道:“姨娘不知道,宝丫头古怪\begin{note}甲戌侧:“古怪”二字,正是宝卿身份。\end{note}著呢,他从来不爱这些花儿粉儿的。”\begin{note}甲戌双夹:可知周瑞一回,正为宝菱二人所有,正《石头记》得力处也。\end{note}
\end{parag}


\begin{parag}
    说著,周瑞家的拿了匣子,走出房门,见金钏仍在那里晒日阳儿。周瑞家的因问他道:“那香菱小丫头子,可就是常说临上京时买的,为他打人命官司的那个小丫头子么?”金钏道:“可不就是。”\begin{note}甲戌侧:出明英莲。\end{note}正说著,只见香菱笑嘻嘻的走来。周瑞家的便拉了他的手,细细的看了一会,因向金钏儿笑道:“倒好个模样儿,竟有些象咱们东府里蓉大奶奶的品格儿。”\begin{note}甲戌双夹:一击两鸣法,二人之美,并可知矣。再忽然想到秦可卿,何玄幻之极。假使说像荣府中所有之人,则死板之至,故远远以可卿之貌为譬,似极扯淡,然却是天下必有之情事。\end{note}金钏儿笑道:“我也是这们说呢。”周瑞家的又问香菱:“你几岁投身到这里?”又问:“你父母今在何处?今年十几岁了?本处是那里人?”香菱听问,都摇头说:“不记得了。”\begin{note}甲戌双夹:伤痛之极,亦必如此收住方妙。不然,则又将作出香菱思乡一段文字矣。\end{note}周瑞家的和金钏儿听了,倒反为叹息伤感一回。
\end{parag}


\begin{parag}
    一时间周瑞家的携花至王夫人正房后头来。原来近日贾母说孙女儿们太多了,一处挤著倒不方便,只留宝玉、黛玉二人这边解闷,却将迎、探、惜三人移到王夫人这边房后三间小抱厦内居住,令李纨陪伴照管。\begin{note}甲戌侧:不作一笔安逸之笔矣。\end{note}如今周瑞家的故顺路先往这里来,只见几个小丫头子都在抱厦内听呼唤呢。迎春的丫鬟司棋与探春的丫鬟侍书\begin{note}甲戌双夹:妙名。贾家四钗之鬟,暗以琴、棋、书、画四字列名,省力之甚,醒目之甚,却是俗中不俗处。\end{note}二人正掀帘子出来,手里都捧著茶钟,周瑞家的便知他们姊妹在一处坐著呢,遂进入内房,只见迎春探春二人正在窗下围棋。周瑞家的将花送上,说明缘故。二人忙住了棋,都欠身道谢,命丫鬟们收了。
\end{parag}


\begin{parag}
    周瑞家的答应了,因说:“四姑娘不在房里?只怕在老太太那边呢。”丫鬟们道:“在这屋里不是?”\begin{note}甲戌双夹:用画家三五聚散法写来,方不死板。\end{note}周瑞家的听了,便往这边屋里来。只见惜春正同水月庵\begin{note}[即馒头庵。]\end{note}的小姑子智能儿一处顽笑,\begin{note}甲戌双夹:总是得空便入。百忙中又带出王夫人喜施舍等事,可知一支笔作千百支用。又伏后文。甲戌眉:闲闲一笔,却将后半部线索提动。\end{note}见周瑞家的进来,惜春便问他何事。周瑞家的便将花匣打开,说明原故。惜春笑道:“我这里正和智能儿说,我明儿也剃了头同他作姑子去呢,可巧又送了花儿来,若剃了头,可把这花儿戴在那里呢?”说著,大家取笑一回,惜春命丫鬟入画\begin{note}甲戌侧:曰司棋,曰侍书,曰入画;后文补抱琴。琴、棋、书、画四字最俗,上添一虚字则觉新雅。\end{note}来收了。
\end{parag}


\begin{parag}
    周瑞家的因问智能儿:“你是什么时候来的?你师父那秃歪剌往那里去了?”智能儿道:“我们一早就来了,我师父见了太太,就往于老爷府内去了,叫我在这里等他呢。”\begin{note}甲戌双夹:又虚贴一个于老爷,可知尚僧尼者,悉愚人也。\end{note}周瑞家的又道:“十五的月例香供银子可曾得了没有?”智能儿摇头儿说:“我不知道。”\begin{note}甲戌双夹:妙!年轻未任事也。一应骗布施、哄斋供诸恶,皆是老秃贼设局。写一种人,一种人活像。\end{note}惜春听了,便问周瑞家的:“如今各庙月例银子是谁管著?”周瑞家的道:“是余信\begin{note}甲戌侧:明点“愚信”二字。\end{note}管著。”惜春听了笑道:“这就是了。他师父一来,余信家的就赶上来,和他师父咕唧了半日,想是就为这事了。”\begin{note}甲戌双夹:一人不落,一事不忽,伏下多少后文,岂真为送花哉!\end{note}
\end{parag}


\begin{parag}
    那周瑞家的又和智能儿劳叨了一会,便往凤姐儿处来。穿夹道从李纨后窗下过,\begin{note}甲戌双夹:细极!李纨虽无花,岂可失而不写者?故用此顺笔便墨,间三带四,使观者不忽。\end{note}越过西花墙,出西角门进入凤姐院中。走至堂屋,只见小丫头丰儿坐在凤姐房中门槛上,见周瑞家的来了,连忙\begin{note}甲戌侧:二字著紧。\end{note}摆手儿叫他往东屋里去。周瑞家的会意,忙蹑手蹑足往东边房里来,只见奶子正拍著大姐儿睡觉呢。\begin{note}甲戌侧:总不重犯,写一次有一次的新样文法。\end{note}周瑞家的悄问奶子道:“奶奶睡中觉呢?也该请醒了。”奶子摇头儿。\begin{note}甲戌侧:有神理。\end{note}正说著,只听那边一阵笑声,却有贾琏的声音。接著房门响处,平儿拿著大铜盆出来,叫丰儿舀水进去。\begin{note}甲戌双夹:妙文奇想!阿凤之为人,岂有不著意于“风月”二字之理哉?若直以明笔写之,不但唐突阿凤身价,亦且无妙文可赏。若不写之,又万万不可。故只用 “柳藏鹦鹉语方知”之法,略一皴染,不独文字有隐微,亦且不至污渎阿凤之英风俊骨。所谓此书无一不妙。甲戌眉:余素所藏仇十洲《幽窗听莺暗春图》,其心思笔墨,已是无双,今见此阿凤一传,则觉画工太板。\end{note}平儿便到这边来,一见了周瑞家的便问:“你老人家又跑了来作什么?”周瑞家的忙起身,拿匣子与他,说送花儿一事。平儿听了,便打开匣子,拿了四枝,转身去了。半刻工夫,手里拿出两枝来,\begin{note}甲戌侧:攒花簇锦之文,故使人耳目眩乱。\end{note}先叫彩明吩咐道:“送到那边府里给小蓉大奶奶戴去。”\begin{note}甲戌侧:忙中更忙,又曰“密处不容针”,此等处是也。\end{note}次后方命周瑞家的回去道谢。
\end{parag}


\begin{parag}
    周瑞家的这才往贾母这边来。穿过了穿堂,抬头忽见他女儿打扮著才从他婆家来。周瑞家的忙问:“你这会跑来作什么?”他女儿笑道:“妈一向身上好?我在家里等了这半日,妈竟不出去,什么事情这样忙的不回家?我等烦了,自己先到了老太太跟前请了安了,这会子请太太的安去。妈还有什么不了的差事,手里是什么东西?”周瑞家的笑道:“嗳!今儿偏偏的来了个刘姥姥,我自己多事,为他跑了半日,这会子又被姨太太看见了,送这几枝花儿与姑娘奶奶们。这会子还没送清楚呢。你这会子跑了来,一定有什么事。”他女儿笑道:“你老人家倒会猜。实对你老人家说,你女婿前儿因多吃了两杯酒,和人分争,不知怎的被人放了一把邪火,说他来历不明,告到衙门里,要递解还乡。所以我来和你老人家商议商议,这个情分,求那一个可了事呢?”周瑞家的听了道:“我就知道呢。这有什么大不了的!你且家去等我,我给林姑娘送了花儿去就回家去。此时太太二奶奶都不得闲儿,你回去等我。这有什么,忙的如此。”女儿听说,便回去了,又说:“妈,好歹快来。”周瑞家的道:“是了。小人儿家没经过什么事,就急得你这样了。”说著。便到黛玉房中去了。\begin{note}甲戌双夹:又生出一小段来,是荣、宁中常事,亦是阿凤正文,若不如此穿插,直用一送花到底,亦太死板,不是《石头记》笔墨矣。\end{note}
\end{parag}


\begin{parag}
    谁知此时黛玉不在自己房中,却在宝玉房中大家解九连环顽呢。\begin{note}甲戌侧:妙极!又一花样。此时二玉已隔房矣。\end{note}周瑞家的进来笑道:“林姑娘,姨太太著我送花儿与姑娘带。”宝玉听说,便先问:“什么花儿?拿来给我。”一面早伸手接过来了。\begin{note}甲戌侧:瞧他夹写宝玉。\end{note}开匣看时,原来是宫制堆纱新巧的假花儿。\begin{note}甲戌侧:此处方一细写花形。\end{note}黛玉只就宝玉手中看了一看,\begin{note}甲戌侧:妙!看他写黛玉。\end{note}便问道:“还是单送我一人的,还是别的姑娘们都有呢?”\begin{note}甲戌双夹:在黛玉心中,不知有何丘壑。\end{note}周瑞家的道:“各位都有了,这两枝是姑娘的了。”黛玉冷笑道:“我就知道,别人不挑剩下的也不给我。”\begin{note}甲戌侧:吾实不知黛卿胸中有何丘壑,在“看一看”上传神。\end{note}周瑞家的听了,一声儿不言语。\begin{note}甲戌眉:余阅送花一回,薛姨妈云“宝丫头不喜这些花儿粉儿的”,则谓是宝钗正传。又出阿凤、惜春一段,则又知是阿凤正传。今又到颦儿一段,却又将阿颦之天性,从骨中一写,方知亦系颦儿正传。小说中一笔作两三笔者有之,一事启两事者有之,未有如此恒河沙数之笔也。\end{note}宝玉便问道:“周姐姐,你作什么到那边去了。”周瑞家的因说:“太太在那里,因回话去了,姨太太就顺便叫我带来了。”宝玉道:“宝姐姐在家作什么呢?怎么这几日也不过这边来?”周瑞家的道:“身上不大好呢。”宝玉听了,便和丫头说:“谁去瞧瞧?只说我和林姑娘\begin{note}甲戌侧:“和林姑娘”四字著眼。\end{note}打发了来请姨太太姐姐安,问姐姐是什么病,现吃什么药。论理我该亲自来的,就说才从学里来,也著了些凉,异日再亲自来看罢。”\begin{note}甲戌眉:余观“才从学里来”几句,忽追思昔日情景,可叹!想纨绔小儿,自开口云“学里”,亦如市俗人开口便云“有些小事”,然何尝真有事哉!此掩饰推托之词耳。宝玉若不云“从学房里来凉著”,然则便云“因憨顽时凉著”者哉?写来一笑,继之一叹。\end{note}说著,茜雪便答应去了。周瑞家的自去,无话。
\end{parag}


\begin{parag}
    原来这周瑞的女婿,便是雨村的好友冷子兴,\begin{note}甲戌侧:著眼。\end{note}近因卖古董和人打官司,故教女人来讨情分。周瑞家的仗著主子的势利,把这些事也不放在心上,晚间只求求凤姐儿便完了。
\end{parag}


\begin{parag}
    至掌灯时分,凤姐已卸了妆,来见王夫人回话:“今儿甄家\begin{note}甲戌侧:又提甄家。\end{note}送了来的东西,我已收了。\begin{note}甲戌侧:不必细说方妙。\end{note}咱们送他的,趁著他家有年下进鲜的船回去,一并都交给他们带了去罢?”王夫人点头。凤姐又道:“临安伯老太太生日的礼已经打点了,派谁送去呢?”\begin{note}甲戌侧:阿凤一生尖处。\end{note}王夫人道:“你瞧谁闲著,就叫他们去四个女人就是了,又来当什么正经事问我。”\begin{note}甲戌双夹:虚描二事,真真千头万绪,纸上虽一回两回中或有不能写到阿凤之事,然亦有阿凤在彼处手忙心忙矣,观此回可知。\end{note}凤姐又笑道:“今日珍大嫂子来,请我明日过去逛逛,明日倒没有什么事情。”王夫人道:“有事没事都害不著什么。每常他来请,有我们,你自然不便意,他既不请我们,单请你,可知是他诚心叫你散淡散淡,别辜负了他的心,便有事也该过去才是。”凤姐答应了。当下李纨、迎、探等姐妹们亦来定省毕,各自归房无话。
\end{parag}


\begin{parag}
    次日凤姐梳洗了,先回王夫人毕,方来辞贾母。宝玉听了,也要跟了逛去。凤姐只得答应,立等著换了衣服,姐儿两个坐了车,一时进入宁府。早有贾珍之妻尤氏与贾蓉之妻秦氏婆媳两个,引了多少姬妾丫鬟媳妇等接出仪门。那尤氏一见了凤姐,必先笑嘲一阵,一手携了宝玉同入上房来归坐。秦氏献茶毕,凤姐因说:“你们请我来作什么?有什么好东西孝敬我,就快献上来,我还有事呢。”尤氏秦氏未及答话,地下几个姬妾先就笑说:“二奶奶今儿不来就罢,既来了就依不得二奶奶了。”正说著,只见贾蓉进来请安。宝玉因问:“大哥哥今日不在家么?”尤氏道:“出城与老爷请安去了。可是你怪闷的,坐在这里作什么?何不也去逛逛?”
\end{parag}


\begin{parag}
    秦氏笑道:“今儿巧,上回宝叔立刻要见的我那兄弟,他今儿也在这里,\begin{note}甲戌眉:欲出鲸卿,却先小妯娌闲闲一聚,随笔带出,不见一丝作造。\end{note}想在书房里呢,宝叔何不去瞧一瞧?”宝玉听了,即便下炕要走。尤氏、凤姐都忙说:“好生著,忙什么?”一面便吩咐,“好生小心跟著,别委屈著他,倒比不得跟了老太太过来就罢了。”\begin{note}甲戌双夹:“委屈”二字极不通,却是至情,写愚妇至矣!\end{note}凤姐说道:“既这么著,何不请进这秦小爷来,我也瞧一瞧。难道我见不得他不成?”尤氏笑道:“罢,罢!可以不必见他,比不得咱们家的孩子们,胡打海摔的惯了。\begin{note}甲戌双夹:卿家“胡打海摔”,不知谁家方珍怜珠惜?此极相矛盾却极入情,盖大家妇人口吻如此。\end{note}人家的孩子都是斯斯文文的惯了,乍见了你这破落户,还被人笑话死了呢。”凤姐笑\begin{note}甲戌侧:自负得起。\end{note}道:“普天下的人,我不笑话就罢了,竟叫这小孩子笑话我不成?”贾蓉笑道:“不是这话,他生的腼腆,没见过大阵仗儿,婶子见了,没的生气。”凤姐啐道:“他是哪吒,我也要见一见!别放你娘的屁了。再不带我看看,给你一顿好嘴巴。”贾蓉笑嘻嘻的说:“我不敢扭著,就带他来。”\begin{note}甲戌眉:此等处写阿凤之放纵,是为后回伏线。\end{note}
\end{parag}


\begin{parag}
    说著,果然出去带进一个小后生来,较宝玉略瘦些,清眉秀目,粉面朱唇,身材俊俏,举止风流,似在宝玉之上,只是羞羞怯怯,有女儿之态,腼腆含糊,慢向凤姐作揖问好。凤姐喜的先推宝玉,笑道:“比下去了!”\begin{note}甲戌侧:不知从何处想来。\end{note}便探身一把携了这孩子的手,就命他身傍坐了,慢慢的问他年纪读书等事,\begin{note}甲戌侧:分明写宝玉,却先偏写阿凤。\end{note}方知他学名唤秦钟。\begin{note}甲戌双夹:设云“情钟”。古诗云:“未嫁先名玉,来时本姓秦。”二语便是此书大纲目、大比托、大讽刺处。\end{note}早有凤姐的丫鬟媳妇们见凤姐初会秦钟,并未备得表礼来,遂忙过那边去告诉平儿。平儿知道凤姐与秦氏厚密,虽是小后生家,亦不可太俭,遂自作主意,拿了一匹尺头,两个“状元及第”的小金锞(kè)子,交付与来人送过去。凤姐犹笑说太简薄等语。秦氏等谢毕。一时吃过饭,尤氏、凤姐、秦氏等抹骨牌,不在话下。\begin{note}甲戌双夹:一人不落,又带出强将手下无弱兵。\end{note}
\end{parag}


\begin{parag}
    宝玉秦钟二人随便起坐说话。\begin{note}甲戌侧:淡淡写来。\end{note}那宝玉只一见了秦钟的人品出众,心中便有所失,痴了半日,自己心中又起了呆意,乃自思道:“天下竟有这等人物!如今看来,我竟成了泥猪癞狗了。可恨我为什么生在这侯门公府之家,若也生在寒门薄宦之家,早得与他交结,也不枉生了一世。我虽如此比他尊贵,\begin{note}甲戌双夹:这一句不是宝玉本意中语,却是古今历来膏粱纨绔之意。\end{note}可知锦绣纱罗,也不过裹了我这根死木头;美酒羊羔,也不过填了我这粪窟泥沟。‘富贵’二字,不料遭我荼毒了!”\begin{note}甲戌双夹:一段痴情,翻“贤贤易色”一句筋斗,使此后朋友中无复再敢假谈道义,虚论情常。蒙侧:此是作者一大发泄处。\end{note}秦钟自见了宝玉形容出众,举止不浮,\begin{note}甲戌双夹:“不浮”二字妙,秦卿目中所取正在此。\end{note}更兼金冠绣服,骄婢侈童,\begin{note}甲戌双夹:这二句是贬,不是奖。此八字遮饰过多少魑魅纨绮秦卿目中所鄙者。\end{note}秦钟心中亦自思道:“果然这宝玉怨不得人溺爱他。可恨我偏生于清寒之家,不能与他耳鬓交接,可知‘贫富’二字限人,亦世间之大不快事。”\begin{note}甲戌双夹:“贫富”二字中,失却多少英雄朋友!蒙侧:总是作者大发泄处,借此以伸多少不乐。\end{note}二人一样的胡思乱想。\begin{note}甲戌双夹:作者又欲瞒过众人。\end{note}忽又\begin{note}甲戌双夹:二字写小儿得神。\end{note}宝玉问他读什么书。\begin{note}甲戌双夹:宝玉问读书,亦想不到之大奇事。\end{note}秦钟见问,便因实而答。\begin{note}甲戌双夹:四字普天下朋友来看。\end{note}二人你言我语,十来句后,越觉亲密起来。
\end{parag}


\begin{parag}
    一时摆上茶果,宝玉便说:“我两个又不吃酒,把果子摆在里间小炕上,我们那里坐去,省得闹你们。”\begin{note}甲戌双夹:眼见得二人一身一体矣。\end{note}于是二人进里间来吃茶。秦氏一面张罗与凤姐摆酒果,一面忙进来嘱宝玉道:“宝叔,你侄儿倘或言语不防头,你千万看著我,不要理他。他虽腼腆,却性子左强,不大随和些是有的。”\begin{note}甲戌侧:实写秦钟,又映宝玉。\end{note}宝玉笑道:“你去罢,我知道了。”秦氏又嘱了他兄弟一回,方去陪凤姐。
\end{parag}


\begin{parag}
    一时凤姐尤氏又打发人来问宝玉:“要吃什么,外面有,只管要去。”宝玉只答应著,也无心在饮食上,只问秦钟近日家务等事。\begin{note}甲戌双夹:宝玉问读书已奇,今又问家务,岂不更奇?\end{note}秦钟因说:“业师于去年病故,家父又年纪老迈,残疾在身,公务繁冗,因此尚未议及再延师一事,目下不过在家温习旧课而已。再读书一事,必须有一二知己为伴,时常大家讨论,才能进益。”宝玉不待说完,便答道:“正是呢,我们却有个家塾,合族中有不能延师的,便可入塾读书,子弟们中亦有亲戚在内可以附读。我因业师上年回家去了,也现荒废著呢。家父之意,亦欲暂送我去温习旧书,待明年业师上来,再各自在家里读。家祖母因说:一则家学里之子弟太多,生恐大家淘气,反不好,二则也因我病了几天,遂暂且耽搁著。如此说来,尊翁如今也为此事悬心。今日回去,何不禀明,就往我们敝塾中来,我亦相伴,彼此有益,岂不是好事?” 秦钟笑道:\begin{note}甲戌眉:真是可儿之弟。\end{note}“家父前日在家提起延师一事,也曾提起这里的义学倒好,原要来和这里的亲翁商议引荐。因这里又事忙,不便为这点小事来聒絮的。宝叔果然度小侄或可磨墨涤砚,何不速速的作成,\begin{note}甲戌眉:真是可卿之弟。\end{note}又彼此不致荒废,又可以常相谈聚,又可以慰父母之心,又可以得朋友之乐,岂不是美事?”宝玉道:“放心,放心。咱们回来告诉你姐夫、姐姐和琏二嫂子。你今日回家就禀明令尊,我回去再禀明祖母,再无不速成之理。”二人计议一定。那天气已是掌灯时候,出来又看他们顽了一回牌。算帐时,却又是秦氏、尤氏二人输了戏酒的东道,\begin{note}甲戌侧:自然是二人输。\end{note}言定后日吃这东道,一面就叫送饭。
\end{parag}


\begin{parag}
    吃毕晚饭,因天黑了,尤氏说:“先派两个小子送了这秦相公家去。”媳妇们传出去半日,秦钟告辞起身。尤氏问:“派了谁送去?”媳妇们回说:“外头派了焦大,谁知焦大醉了,又骂呢。”\begin{note}甲戌双夹:可见骂非一次矣。\end{note}尤氏、秦氏都说道:“偏又派他作什么!放著这些小子们,那一个派不得?偏要惹他去。”\begin{note}甲戌侧:便奇。\end{note}凤姐道:“我成日家说你太软弱了,纵的家里人这样还了得了。”尤氏叹道:“你难道不知这焦大的?连老爷都不理他的,你珍大哥哥也不理他。只因他从小儿跟著太爷们出过三四回兵,从死人堆里把太爷背了出来,得了命,自己挨著饿,却偷了东西来给主子吃。两日没得水,得了半碗水给主子喝,他自己喝马溺。不过仗著这些功劳情分,有祖宗时都另眼相待,如今谁肯难为他去。他自己又老了,又不顾体面,一味吃酒,吃醉了,无人不骂。我常说给管事的,不要派他差事,全当一个死的就完了。今儿又派了他。”\begin{note}蒙侧:有此功劳,实不可轻易摧折,亦当处之道,厚其赡养,尊其等次。送人回家,原非酬功之事。所谓汉之功臣不得保其首领者,我知之矣。\end{note}凤姐道:“我何曾不知这焦大。倒是你们没主意,有这样的,何不打发他远远的庄子上去就完了。”\begin{note}甲戌眉:这是为后协理宁国伏线。\end{note}说著,因问:“我们的车可齐备了?”地下众人都应道:“伺候齐了。”
\end{parag}


\begin{parag}
    凤姐起身告辞,和宝玉携手同行。尤氏等送至大厅,只见灯烛辉煌,众小厮在丹墀侍立。那焦大又恃贾珍不在家,即在家亦不好怎样他,更可以任意洒落洒落。因趁著酒兴,先骂大总管赖二,\begin{note}甲戌双夹:记清,荣府中则是赖大,又故意综错的妙。\end{note}说他不公道,欺软怕硬:“有了好差事就派别人,象这等黑更半夜送人的事,就派我。没良心的王八羔子!瞎充管家!你也不想想,焦大太爷跷跷脚,比你的头还高呢。二十年头里的焦大太爷眼里有谁?别说你们这把子的杂种王八羔子们!”
\end{parag}


\begin{parag}
    正骂的兴头上,贾蓉送凤姐的车出去,众人喝他不听,贾蓉忍不得,便骂了他两句,使人捆起来,“等明日酒醒了,问他还寻死不寻死了!”那焦大那里把贾蓉放在眼里,反大叫起来,赶著贾蓉叫:“蓉哥儿,你别在焦大跟前使主子性儿。别说你这样儿的,就是你爹,你爷爷,也不敢和焦大挺腰子!不是焦大一个人,你们就做官儿,享荣华,受富贵?你祖宗九死一生挣下这家业,到如今了,不报我的恩,反和我充起主子来了。\begin{note}甲戌侧:忽接此焦大一段,真可惊心骇目,一字化一泪,一泪化一血珠。\end{note}不和我说别的还可,若再说别的,咱们红刀子进去白刀子出来!”\begin{note}甲戌双夹:是醉人口中文法。一段借醉奴口角闲闲补出宁荣往事近故,特为天下世家一笑。\end{note}凤姐在车上说与贾蓉道:“以后还不早打发了这个没王法的东西!留在这里岂不是祸害?倘或亲友知道了,岂不笑话咱们这样的人家,连个王法规矩都没有。”贾蓉答应“是”。
\end{parag}


\begin{parag}
    众小厮见他太撒野了,只得上来几个,揪翻捆倒,拖往马圈里去。焦大越发连贾珍都说出来,乱嚷乱叫说:“我要往祠堂里哭太爷去。那里承望到如今生下这些畜牲来!每日家偷狗戏鸡,爬灰的爬灰,养小叔子的养小叔子,我什么不知道?咱们‘胳膊折了往袖子里藏’!”\begin{note}甲戌眉:“不如意事常八九,可与人言无二三。”以二句批是段,聊慰石兄。\end{note}\begin{note}蒙侧;放笔痛骂一回,富贵之家,每罹此祸。\end{note}众小厮听他说出这些没天日的话来,唬的魂飞魄散,也不顾别的了,便把他捆起来,用土和马粪满满的填了他一嘴。
\end{parag}


\begin{parag}
    凤姐和贾蓉等也遥遥的闻得,便都装作没听见。宝玉在车上见这般醉闹,倒也有趣,因问凤姐道:“姐姐,你听他说‘爬灰的爬灰’,什么是‘爬灰’?”\begin{note}蒙侧:暗伏后来史湘云之问。\end{note}凤姐听了,连忙立眉嗔目断喝道:“少胡说!那是醉汉嘴里混唚。你是什么样的人,不说没听见,还倒细问!等我回去回了太太,仔细捶你不捶你!”唬的宝玉忙央告道:“好姐姐,我再不敢了。”凤姐亦忙回色哄道:“这才是呢。等到了家,咱们回了老太太,打发你同秦家侄儿学里念书去要紧。”说著,却自回往荣府而来。正是:
\end{parag}


\begin{poem}
    \begin{pl} 不因俊俏难为友,正为风流始读书。\end{pl}
    \begin{note}甲戌侧:原来不读书即蠢物矣。\end{note}
\end{poem}
