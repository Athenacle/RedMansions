\chap{七十三}{痴丫头误拾绣春囊 懦小姐不问累金凤}


\begin{parag}
    \begin{note}蒙回前总:贾母一席话隐隐照起,全文便可一直叙去,接笔却置贼不论,转出赌钱,接笔却置赌钱不论,转出奸证,接笔却置奸证不论,转出讨情,一波未平,一波又起,势如怒蛇出穴,蜿蜒不就捕。\end{note}
\end{parag}


\begin{parag}
    话说那赵姨娘和贾政说话,忽听外面一声响,不知何物。忙问时,原来是外间窗屉不曾扣好,塌了屈戍了吊下来。赵姨娘骂了丫头几句,自己带领丫鬟上好,方进来打发贾政安歇。不在话下。
\end{parag}


\begin{parag}
    却说怡红院中宝玉正才睡下,丫鬟们正欲各散安歇,忽听有人击院门。老婆子开了门,见是赵姨娘房内的丫鬟名唤小鹊的。问他什么事,小鹊不答,直往房内来找宝玉。\begin{note}庚双夹:奇,从未见此婢也。\end{note}只见宝玉才睡下,晴雯等犹在床边坐著,大家顽笑,见他来了,都问:“什么事,这时候又跑了来作什么?”\begin{note}庚双夹:又是补出前文矣,非只张一回也。\end{note}小鹊笑向宝玉道:“我来告诉你一个信儿。方才我们奶奶这般如此在老爷前说了。你仔细明儿老爷问你话。”说著回身就去了。袭人命留他吃茶,因怕关门,遂一直去了。
\end{parag}


\begin{parag}
    这里宝玉听了,便如孙大圣听见了紧箍咒一般,登时四肢五内一齐皆不自在起来。想来想去,别无他法,且理熟了书预备明儿盘考。口内不舛错,便有他事,也可搪塞一半。想罢,忙披衣起来要读书。心中又自后悔,这些日子只说不提了,偏又丢生,早知该天天好歹温习些的。如今打算打算,肚子内现可背诵的,不过只有《学》《庸》《二论》是带注背得出的。至上本《孟子》,就有一半是夹生的,若凭空提一句,断不能接背的,至《下孟》,就有一大半忘了。算起五经来,因近来作诗,常把《诗经》读些,虽不甚精阐,还可塞责。\begin{note}庚双夹:妙!宝玉读书原系从问中□而有。\end{note}别的虽不记得,素日贾政也幸未吩咐过读的,纵不知,也还不妨。至于古文,这是那几年所读过的几篇,连《左传》《国策》《公羊》《谷粱》汉唐等文,不过几十篇,这几年竟未曾温得半篇片语,虽闲时也曾遍阅,不过一时之兴,随看随忘,未下苦工夫,如何记得。这是断难塞责的。更有时文八股一道,因平素深恶此道,原非圣贤之制撰,焉能阐发圣贤之微奥,不过作后人饵名钓禄之阶。虽贾政当日起身时选了百十篇命他读的,不过偶因见其中或一二股内,或承起之中,有作的或精致,或流荡,或游戏,或悲感,稍能动性者,偶一读之,不过供一时之兴趣,究竟何曾成篇潜心玩索。\begin{note}庚双夹:妙!写宝玉读书非为功名也。\end{note}如今若温习这个,又恐明日盘诘那个,若温习那个,又恐盘驳这个。况一夜之功,亦不能全然温习,因此越添了焦燥。自己读书不致紧要,却带累著一房丫鬟们皆不能睡。袭人麝月晴雯等几个大的是不用说,在旁剪烛斟茶,那些小的,都困眼朦胧,前仰后合起来。晴雯因骂道:“什么蹄子们,一个个黑日白夜挺尸挺不够,偶然一次睡迟了些,就装出这腔调来了。再这样,我拿针戳给你们两下子!”
\end{parag}


\begin{parag}
    话犹未了,只听外间咕咚一声,急忙看时,原来是一个小丫头子坐著打盹,一头撞到壁上了,从梦中惊醒,恰正是晴雯说这话之时,他怔怔的只当是晴雯打了他一下,遂哭央说:“好姐姐,我再不敢了。”众人都发起笑来。宝玉忙劝道:“饶他去罢,原该叫他们都睡去才是。你们也该替换著睡去。”袭人忙道:“小祖宗,你只顾你的罢。通共这一夜的功夫,你把心暂且用在这几本书上,等过了这一关,由你再张罗别的去,也不算误了什么。”宝玉听他说的恳切,只得又读。读了没有几句,麝月又斟了一杯茶来润舌,宝玉接茶吃了。因见麝月只穿著短袄,解了裙子,宝玉道:“夜静了,冷,到底穿一件大衣裳才是。”麝月笑指著书道:“你暂且把我们忘了,把心且略对著他些罢。”\begin{note}庚双夹:此处岂是读书之处,又岂是伴读之人?古今天下误尽多少纨绔!何况又是此等时之怡红院,此等之鬟婢,又是此等一个宝玉哉!\end{note}
\end{parag}


\begin{parag}
    话犹未了,只听金星玻璃从后房门跑进来,口内喊说:“不好了,一个人从墙上跳下来了!”众人听说,忙问在那里,即喝起人来,各处寻找。晴雯因见宝玉读书苦恼,劳费一夜神思,明日也未必妥当,心下正要替宝玉想出一个主意来脱此难,正好忽然逢此一惊,即便生计,向宝玉道:“趁这个机会快装病,只说唬著了。” 此话正中宝玉心怀,因而遂传起上夜人等来,打著灯笼,各处搜寻,并无踪迹,都说:“小姑娘们想是睡花了眼出去,风摇的树枝儿,错认作人了。”晴雯便道: “别放诌屁!你们查的不严,怕得不是,还拿这话来支吾。才刚并不是一个人见的,宝玉和我们出去有事,大家亲见的。如今宝玉唬的颜色都变了,满身发热,我如今还要上房里取安魂丸药去。太太问起来,是要回明白的,难道依你说就罢了不成。”众人听了,吓的不敢则声,只得又各处去找。雯和玻璃二人果出去要药,故意闹的众人皆知宝玉吓著了。王夫人听了,忙命人来看视给药,又吩咐各上夜人仔细搜查,又一面叫查二门外邻园墙上夜的小厮们。于是园内灯笼火把,直闹了一夜。至五更天,就传管家男女,命仔细查一查,拷问内外上夜男女等人。
\end{parag}


\begin{parag}
    贾母闻知宝玉被吓,细问原由,不敢再隐,只得回明。贾母道:“我必料到有此事。如今各处上夜都不小心,还是小事,只怕他们就是贼也未可知。”当下邢夫人并尤氏等都过来请安,凤姐及李纨姊妹等皆陪侍,听贾母如此说,都默无所答。独探春出位笑道:“近因凤姐姐身子不好,几日园内的人比先放肆了许多。先前不过是大家偷著一时半刻,或夜里坐更时,三四个人聚在一处,或掷骰或斗牌,小小的顽意,不过为熬困。近来渐次放诞,竟开了赌局,甚至有头家局主,或三十吊五十吊三百吊的大输赢。半月前竟有争斗相打之事。”贾母听了,忙说:“你既知道,为何不早回我们来?”探春道:“我因想著太太事多,且连日不自在,所以没回。只告诉了大嫂子和管事的人们,戒饬过几次,近日好些。”贾母忙道:“你姑娘家,如何知道这里头的利害。你自为耍钱常事,不过怕起争端。殊不知夜间既耍钱,就保不住不吃酒,既吃酒,就免不得门户任意开锁。或买东西,寻张觅李,其中夜静人稀,趋便藏贼引奸引盗,何等事作不出来。况且园内的姊妹们起居所伴者皆系丫头媳妇们,贤愚混杂,贼盗事小,再有别事,倘略沾带些,关系不小。这事岂可轻恕。”探春听说,便默然归坐。凤姐虽未大愈,精神因此比常稍减,\begin{note}庚双夹:看他渐次写来,从不作一笔安逸之笔,况阿凤之文哉。\end{note}今见贾母如此说,便忙道:“偏生我又病了。”遂回头命人速传林之孝家的等总理家事四个媳妇到来,当著贾母申饬了一顿。贾母命即刻查了头家赌家来,有人出首者赏,隐情不告者罚。
\end{parag}


\begin{parag}
    林之孝家的等见贾母动怒,谁敢徇私,忙至园内传齐人,一一盘查。虽不免大家赖一回,终不免水落石出。查得大头家三人,小头家八人,聚赌者通共二十多人,都带来见贾母,跪在院内磕响头求饶。贾母先问大头家名姓和钱之多少。原来这三个大头家,一个就是林之孝家的两姨亲家,一个就是园内厨房内柳家媳妇之妹,一个就是迎春之乳母。这是三个为首的,余者不能多记。贾母便命将骰子牌一并烧毁,所有的钱入官分散与众人,将为首者每人四十大板,撵出,总不许再入;从者每人二十大板,革去三月月钱,拨入圊厕行内。又将林之孝家的申饬了一番。林之孝家的见他的亲戚又与他打嘴,自己也觉没趣。迎春在坐,也觉没意思。黛玉、宝钗、探春等见迎春的乳母如此,也是物伤其类的意思,遂都起身笑向贾母讨情说:“这个妈妈素日原不顽的,不知怎么也偶然高兴。求看二姐姐面上,饶他这次罢。”贾母道:“你们不知。大约这些奶子们,一个个仗著奶过哥儿姐儿,原比别人有些体面,他们就生事,比别人更可恶,专管调唆主子护短偏向。我都是经过的。况且要拿一个作法,恰好果然就遇见了一个。你们别管,我自有道理。”宝钗等听说,只得罢了。
\end{parag}


\begin{parag}
    一时贾母歇晌,大家散出,都知贾母今日生气,皆不敢各散回家,只得在此暂候。尤氏便往凤姐处来闲话了一回,因他也不自在,只得往园内寻众姑嫂闲谈。邢夫人在王夫人处坐了一回,也就往园内散散心来。刚至园门前,只见贾母房内的小丫头子名唤傻大姐的笑嘻嘻走来,手内拿著个花红柳绿的东西,低头一壁瞧著,一壁只管走,不防迎头撞见邢夫人,抬头看见,方才站住。邢夫人因说:“这痴丫头,又得了个什么狗不识儿这么欢喜?拿来我瞧瞧。”原来这傻大姐年方十四五岁,是新挑上来的与贾母这边提水桶扫院子专作粗活的一个丫头。只因他生得体肥面阔,两只大脚作粗活简捷爽利,且心性愚顽,一无知识,行事出言,常在规矩之外。贾母因喜欢他爽利便捷,又喜他出言可以发笑,便起名为“呆大姐”,常闷来便引他取笑一回,毫无避忌,因此又叫他作“痴丫头”。他纵有失礼之处,见贾母喜欢他,众人也就不去苛责。这丫头也得了这个力,若贾母不唤他时,便入园内来顽耍。今日正在园内掏促织,忽在山石背后得了一个五彩绣香囊,其华丽精致,固是可爱,但上面绣的并非花鸟等物,一面却是两个人赤条条的盘踞相抱,一面是几个字。这痴丫头原不认得是春意,便心下盘算:“敢是两个妖精打架?不然必是两口子相打。”左右猜解不来,正要拿去与贾母看,\begin{note}庚双夹:险极妙极!荣府堂堂诗礼之家,且大观园又何等严肃清幽之地,金闺玉阁尚有此等秽物,天下浅浦募之家宁不慎乎!虽然,但此等偏出大官世族之中者,盖因其房室香宵、鬟婢混杂,焉保其个个守礼持节哉?此正为大官世族而告诫。浅浦募之处毋如主婢日夕耳鬓交磨,一止一动悉在耳目之中,又何必谆谆再四焉!\end{note}是以笑嘻嘻的一壁看,一壁走,忽见了邢夫人如此说,便笑道:“太太真个说的巧,真个是狗不识呢。\begin{note}庚双夹:妙!寓言也,大凡知此交媾之情者真狗畜之说耳,非肆言恶詈凡识此事者即狗矣。然则云先与贾母看,则先骂贾母矣。此处邢夫人亦看,然则又骂邢夫人乎?故作者又难。\end{note}太太请瞧一瞧。”说著,便送过去。邢夫人接来一看,吓得连忙死紧攥住,\begin{note}庚双夹:妙!这一“吓”字方是写邢夫人之笔,虽前文明写邢夫人之为人稍劣,然不在情理之中,若不用慎重之笔,则邢夫人直系一小家卑污极轻贼极轻之人矣,岂得与荣府赐房哉?所谓此书针绵慎密处全在无意中一字一句之间耳,看者细心方得。\end{note}忙问:“你是那里得的?”傻大姐道:“我掏促织儿在山石上拣的。”邢夫人道:“快休告诉一人。这不是好东西,连你也要打死。皆因你素日是傻子,以后再别提起了。”这傻大姐听了,反吓的黄了脸,说:“再不敢了。”磕了个头,呆呆而去。邢夫人回头看时,都是些女孩儿,不便递与,自己便塞在袖内,心内十分罕异,揣摩此物从何而至,且不形于声色,且来至迎春室中。
\end{parag}


\begin{parag}
    迎春正因他乳母获罪,自觉无趣,心中不自在,忽报母亲来了,遂接入内室。奉茶毕,邢夫人因说道:“你这么大了,你那奶妈子行此事,你也不说说他。如今别人都好好的,偏咱们的人做出这事来,什么意思。”\begin{note}庚双夹:“咱们”二字便见自怀异心,从上文生离异发沥而来,谨密之至。更有人于此者君未知也,一笑。\end{note}迎春低著头弄衣带,半晌答道:“我说他两次,他不听也无法。况且他是妈妈,只有他说我的,没有我说他的。”\begin{note}庚双夹:妙极!直画出一个懦弱小姐来。\end{note}邢夫人道:“胡说!你不好了他原该说,如今他犯了法,你就该拿出小姐的身分来。他敢不从,你就回我去才是。如今直等外人共知,是什么意思。\begin{note}庚双夹:我竟问:外人为谁?\end{note}再者,只他去放头儿,还恐怕他巧言花语的和你借贷些簪环衣履作本钱,你这心活面软,未必不周接他些。若被他骗去,我是一个钱没有的,看你明日怎么过节。”迎春不语,只低头弄衣带。邢夫人见他这般,因冷笑道:“总是你那好哥哥好嫂子,一对儿赫赫扬扬,琏二爷凤奶奶,两口子遮天盖日,百事周到,竟通共这一个妹子,全不在意。\begin{note}庚双夹:加在于琏凤的是父母常情,极是何必又如此说来便见私意。\end{note}但凡是我身上吊下来的,又有一话说──只好凭他们罢了。\begin{note}庚双夹:如何?此皆妇女私假之意,大不可者。\end{note}况且你又不是我养的,\begin{note}庚双夹:更不好。\end{note}你虽然不是同他一娘所生,到底是同出一父,也该彼此瞻顾些,也免别人笑话。\begin{note}庚双夹:又问:别人为谁?又问:彼二人虽不同母,终是同父。彼二人既系同父,其父又系君之何人?吁!妇人私心,今古有之。\end{note}我想天下的事也难较定,你是大老爷跟前人养的,这里探丫头也是二老爷跟前人养的,出身一样。如今你娘死了,从前看来你两个的娘,只有你娘比如今赵姨娘强十倍的,你该比探丫头强才是。怎么反不及他一半!谁知竟不然,这可不是异事。倒是我一生无儿无女的,一生干净,也不能惹人笑话议论为高。”\begin{note}庚双夹:最可恨妇人无嗣者引此话是说。\end{note}旁边伺侯的媳妇们便趁机道:“我们的姑娘老实仁德,那里像他们三姑娘伶牙俐齿,会要姊妹们的强。他们明知姐姐这样,他竟不顾恤一点儿。”\begin{note}庚双夹:杀杀杀!此辈专生离异。余因实受其蛊,今读此文,直欲拔剑劈纸。又不知作者多少眼泪洒出此回也。又问:不知如何顾恤些?又不知有何可顾恤之处?直令人不解愚奴贱婢之言。酷肖之至。\end{note}邢夫人道:“连他哥哥嫂子还如是,别人又作什么呢。”一言未了,人回:“琏二奶奶来了。”邢夫人听了,冷笑两声,命人出去说:“请他自去养病,我这里不用他伺候。”接著又有探春的小丫头来报说:“老太太醒了。”邢夫人方起身前边来。迎春送至院外方回。
\end{parag}


\begin{parag}
    绣桔因说道:“如何,前儿我回姑娘,那一个攒珠累丝金凤竟不知那里去了。回了姑娘,姑娘竟不问一声儿。我说必是老奶奶拿去典了银子放头儿的,姑娘不信,只说司棋收著呢。问司棋,司棋虽病著,心里却明白。我去问他,他说没有收起来,还在书架上匣内暂放著,预备八月十五日恐怕要戴呢。姑娘就该问老奶奶一声,只是脸软怕人恼。如今竟怕无著,明儿要都戴时,独咱们不戴,是何意思呢。”\begin{note}庚双夹:这个“咱们”使得恰,是女儿 喁私语,非前文之一例可比。写得出,批得出。\end{note}迎春道:“何用问,自然是他拿去暂时借一肩儿。我只说他悄悄的拿了出去,不过一时半晌,仍旧悄悄的送来就完了,谁知他就忘了。今日偏又闹出来,问他想也无益。”绣桔道:“何曾是忘记!他是试准了姑娘的性格,所以才这样。如今我有个主意:我竟走到二奶奶房里将此事回了他,或他著人去要,或他省事拿几吊钱来替他赔补。如何?”\begin{note}庚双夹:写女儿各有机变,个个不同。\end{note}迎春忙道:“罢,罢,罢,省些事罢。宁可没有了,又何必生事。”\begin{note}庚双夹:总是懦语。\end{note}绣桔道:“姑娘怎么这样软弱。都要省起事来,将来连姑娘还骗了去呢,我竟去的是。”说著便走。迎春便不言语,只好由他。
\end{parag}


\begin{parag}
    谁知迎春乳母子媳王住儿媳妇正因他婆婆得了罪,来求迎春去讨情,听他们正说金凤一事,且不进去。也因素日迎春懦弱,他们都不放在心上。如今见绣桔立意去回凤姐,估著这事脱不去的,且又有求迎春之事,只得进来,陪笑先向绣桔说:“姑娘,你别去生事。姑娘的金丝凤,原是我们老奶奶老糊涂了,输了几个钱,没的捞梢,所以暂借了去。原说一日半晌就赎的,因总未捞过本儿来,就迟住了。可巧今儿又不知是谁走了风声,弄出事来。虽然这样,到底主子的东西,我们不敢迟误下,终久是要赎的。如今还要求姑娘看从小儿吃奶的情常,往老太太那边去讨个情面,救出他老人家来才好。”迎春先便说道:“好嫂子,你趁早儿打了这妄想,要等我去说情儿,等到明年也不中用的。方才连宝姐姐林妹妹大伙儿说情,老太太还不依,何况是我一个人。我自己愧还愧不来,反去讨臊去。”绣桔便说:“赎金凤是一件事,说情是一件事,别绞在一处说。难道姑娘不去说情,你就不赎了不成?嫂子且取了金凤来再说。”王住儿家的听见迎春如此拒绝他,绣桔的话又锋利无可回答,一时脸上过不去,也明欺迎春素日好性儿,乃向绣桔发话道:“姑娘,你别太仗势了。你满家子算一算,谁的妈妈奶子不仗著主子哥儿多得些益,偏咱们就这样丁是丁卯是卯的,只许你们偷偷摸摸的哄骗了去。自从邢姑娘来了,太太吩咐一个月俭省出一两银子来与舅太太去,这里饶添了邢姑娘的使费,反少了一两银子。常时短了这个,少了那个,那不是我们供给?谁又要去?不过大家将就些罢了。算到今日,少说些也有三十两了。我们这一向的钱,岂不白填了限呢。”绣桔不待说完,便啐了一口,道:“作什么的白填了三十两,我且和你算算帐,姑娘要了些什么东西?”迎春听见这媳妇发邢夫人之私意,\begin{note}庚双夹:大书此句,诛心之笔。\end{note}忙止道:“罢,罢,罢。你不能拿了金凤来,不必牵三扯四乱嚷。我也不要那凤了。便是太太们问时,我只说丢了,也妨碍不著你什么的,出去歇息歇息倒好。”一面叫绣桔倒茶来。绣桔又气又急,因说道:“姑娘虽不怕,我们是作什么的,把姑娘的东西丢了。他倒赖说姑娘使了他们的钱,这如今竟要准折起来。倘或太太问姑娘为什么使了这些钱,敢是我们就中取势了?这还了得!”一行说,一行就哭了。司棋听不过,只得勉强过来,帮著绣桔问著那媳妇。迎春劝止不住,自拿了一《太上感应篇》来看。\begin{note}庚双夹:神妙之至!从纸上跳出一位懦弱小姐,且书又有奇,大妙!\end{note}
\end{parag}


\begin{parag}
    三人正没开交,可巧宝钗、黛玉、宝琴、探春等因恐迎春今日不自在,都约来安慰他。走至院中,听得两三个人较口。探春从纱窗内一看,只见迎春倚在床上看书,若有不闻之状。\begin{note}庚双夹:看他写迎春,虽稍劣,然亦大家千金之格也。\end{note}探春也笑了。小丫鬟们忙打起帘子,报导:“姑娘们来了。”迎春方放下书起身。那媳妇见有人来,且又有探春在内,不劝而自止了,遂趁便要去。探春坐下,便问:“才刚谁在这里说话?倒象拌嘴似的。”\begin{note}庚双夹:瞧他写探春气宇。\end{note}迎春笑道:“没有说什么,左不过是他们小题大作罢了。何必问他。”探春笑道:“我才听见什么‘金凤’,又是什么‘没有钱只和我们奴才要’,谁和奴才要钱了?难道姐姐和奴才要钱了不成?难道姐姐不是和我们一样有月钱的,一样有用度不成?”司棋绣桔道:“姑娘说的是了。姑娘们都是一样的,那一位姑娘的钱不是由著奶奶妈妈们使,连我们也不知道怎么是算帐,不过要东西只说得一声儿。如今他偏要说姑娘使过了头儿,他赔出许多来了。究竟姑娘何曾和他要什么了。” 探春笑道:“姐姐既没有和他要,必定是我们或者和他们要了不成!你叫他进来,我倒要问问他。”迎春笑道:“这话又可笑。你们又无沾碍,何得带累于他。”探春笑道:“这倒不然。我和姐姐一样,姐姐的事和我的也是一般,他说姐姐就是说我。我那边的人有怨我的,姐姐听见也即同怨姐姐是一理。咱们是主子,自然不理论那些钱财小事,只知想起什么要什么,也是有的事。但不知金累丝凤因何又夹在里头?”那王住儿媳妇生恐绣桔等告出他来,遂忙进来用话饰。探春深知其意,因笑道:“你们所以糊涂。如今你奶奶已得了不是,趁此求求二奶奶,把方才的钱尚未散人的拿出些来赎取了就完了。比不得没闹出来,大家都藏著留脸面,如今既是没了脸,趁此时纵有十个罪,也只一人受罚,没有砍两颗头的理。你依我,竟是和二奶奶说说。在这里大声小气,如何使得。”这媳妇被探春说出真病,也无可赖了,只不敢往凤姐处自首。探春笑道:“我不听见便罢,既听见,少不得替你们分解分解。”谁知探春早使个眼色与待书出去了。
\end{parag}


\begin{parag}
    这里正说话,忽见平儿进来。宝琴拍手笑说道:“三姐姐敢是有驱神召将的符术?”黛玉笑道:“这倒不是道家玄术,倒是用兵最精的,所谓‘守如处女,脱如狡兔’,出其不备之妙策也。”二人取笑。宝钗便使眼色与二人,令其不可,遂以别话岔开。探春见平儿来了,遂问:“你奶奶可好些了?真是病糊涂了,事事都不在心上,叫我们受这样的委曲。”平儿忙道:“姑娘怎么委曲?谁敢给姑娘气受,姑娘快吩咐我。”当时住儿媳妇儿方慌了手脚,遂上来赶著平儿叫:“姑娘坐下,让我说原故请听。”平儿正色道:“姑娘这里说话,也有你我混插口的礼!你但凡知礼,只该在外头伺候。不叫你进不来的地方,几曾有外头的媳妇子们无故到姑娘们房里来的例。”绣桔道:“你不知我们这屋里是没礼的,谁爱来就来。”平儿道:“都是你们的不是。姑娘好性儿,你们就该打出去,然后再回太太去才是。”王住儿媳妇见平儿出了言,红了脸方退出去。探春接著道:“我且告诉你,若是别人得罪了我,倒还罢了。如今那住儿媳妇和他婆婆仗著是妈妈,又瞅著二姐姐好性儿,如此这般私自拿了首饰去赌钱,而且还捏造假帐妙算,威逼著还要去讨情,和这两个丫头在卧房里大嚷大叫,二姐姐竟不能辖治,所以我看不过,才请你来问一声:还是他原是天外的人,不知道理?还是谁主使他如此,先把二姐姐制伏,然后就要治我和四姑娘了?”平儿忙陪笑道:“姑娘怎么今日说这话出来?我们奶奶如何当得起!”探春冷笑道:“俗语说的,‘物伤其类’,‘齿竭唇亡’,我自然有些惊心。”平儿道:“若论此事,还不是大事,极好处置。但他现是姑娘的奶嫂,据姑娘怎么样为是?”当下迎春只和宝钗阅《感应篇》故事,究竟连探春之语亦不曾闻得,忽见平儿如此说,乃笑道:“问我,我也没什么法子。他们的不是,自作自受,我也不能讨情,我也不去苛责就是了。至于私自拿去的东西,送来我收下,不送来我也不要了。太太们要问,我可以隐瞒遮饰过去,是他的造化,若瞒不住,我也没法,没有个为他们反欺枉太太们的理,少不得直说。你们若说我好性儿,没个决断,竟有好主意可以八面周全,不使太太们生气,任凭你们处治,我总不知道。”众人听了,都好笑起来。黛玉笑道:“真是‘虎狼屯于阶陛尚谈因果’。若使二姐姐是个男人,这一家上下若许人,又如何裁治他们。”迎春笑道:“正是。多少男人尚如此,何况我哉?”一语未了,只见又有一个人进来。正不知道是那个,且听下回分解。
\end{parag}


\begin{parag}
    \begin{note}蒙回末总:一篇奸盗淫邪文字,反以四子五经《公羊》《谷梁》秦汉诸作起,以《太上感应篇》结后,何心哉!他深见“书中自有黄金屋”“书中有女美如雨”等语误尽天下苍生,而大奸大盗从此出。故特作此一起结,为五淫浊世顶门一声棒喝也。眼空似箕,笔大如椽,何得以寻行数墨绳之。\end{note}
\end{parag}


\begin{parag}
    \begin{note}蒙回末总:探春处处出头,人谓其能,吾谓其苦;迎春处处藏舌,谓其怯,吾谓其超。探春运符咒,因及役鬼驱神;迎春说因果,更可降狼伏虎。\end{note}
\end{parag}
