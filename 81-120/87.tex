\chap{八十七}{感深秋抚琴悲往事 坐禅寂走火入邪魔}



\begin{parag}
    却说黛玉叫进宝钗家的女人来,问了好,呈上书子。黛玉叫他去喝茶,便将宝钗来书打开看时,只见上面写着:
\end{parag}


\begin{qute2sp}
    妹生辰不偶,家运多艰,姊妹伶仃,萱亲衰迈。兼之猇声狺语,旦暮无休。更遭惨祸飞灾,不啻惊风密雨。夜深辗侧,愁绪何堪。属在同心,能不为之愍恻乎?回忆海棠结社,序属清秋,对菊持螯,同盟欢洽。犹记“孤标傲世偕谁隐,一样花开为底迟”之句,未尝不叹冷节遗芳,如吾两人也。感怀触绪,聊赋四章,匪曰无故呻吟,亦长歌当哭之意耳。悲时序之递嬗兮,又属清秋。感遭家之不造兮,独处离愁。北堂有萱兮,何以忘忧?无以解忧兮,我心咻咻。一解。云凭凭兮秋风酸,步中庭兮霜叶干。何去何从兮,失我故欢。静言思之兮恻肺肝!二解。惟鲔有潭兮,惟鹤有梁。鳞甲潜伏兮,羽毛何长!搔首问兮茫茫,高天厚地兮,谁知余之永伤。三解。银河耿耿兮寒气侵,月色横斜兮玉漏沉。忧心炳炳兮发我哀吟,吟复吟兮寄我知音。四解。
\end{qute2sp}


\begin{parag}
    黛玉看了,不胜伤感。又想:“宝姐姐不寄与别人,单寄与我,也是惺惺惜惺惺的意思。”正在沉吟,只听见外面有人说道:“林姐姐在家里呢么?”黛玉一面把宝钗的书迭起,口内便答应道:“是谁?”正问着,早见几个人进来,却是探春,湘云,李纹,李绮。彼此问了好,雪雁倒上茶来,大家喝了,说些闲话。因想起前年的菊花诗来,黛玉便道:“宝姐姐自从挪出去,来了两遭,如今索性有事也不来了,真真奇怪。我看他终久还来我们这里不来。”探春微笑道:“怎么不来,横竖要来的。如今是他们尊嫂有些脾气,姨妈上了年纪的人,又兼有薛大哥的事,自然得宝姐姐照料一切,那里还比得先前有工夫呢。”正说着,忽听得忽喇喇一片风声,吹了好些落叶,打在窗纸上。停了一回儿,又透过一阵清香来。众人闻着,都说道:“这是何处来的香风?这象什么香?”黛玉道:“好象木樨香。”探春笑道:“林姐姐终不脱南边人的话,这大九月里的,那里还有桂花呢。”黛玉笑道:“原是啊,不然怎么不竟说是桂花香只说似乎象呢。”湘云道:“三姐姐,你也别说。你可记得‘十里荷花,三秋桂子’?在南边,正是晚桂开的时候了。你只没有见过罢了,等你明日到南边去的时候,你自然也就知道了。”探春笑道:“我有什么事到南边去?况且这个也是我早知道的,不用你们说嘴。”李纹李绮只抿着嘴儿笑。黛玉道:“妹妹,这可说不齐。俗语说,‘人是地行仙’,今日在这里,明日就不知在那里。譬如我,原是南边人,怎么到了这里呢?”湘云拍着手笑道:“今儿三姐姐可叫林姐姐问住了。不但林姐姐是南边人到这里,就是我们这几个人就不同。也有本来是北边的,也有根子是南边,生长在北边的,也有生长在南边,到这北边的,今儿大家都凑在一处。可见人总有一个定数,大凡地和人总是各自有缘分的。”众人听了都点头,探春也只是笑。又说了一会子闲话儿,大家散出。黛玉送到门口,大家都说:“你身上才好些,别出来了,看着了风。”于是黛玉一面说着话儿,一面站在门口又与四人殷勤了几句,便看着他们出院去了。进来坐着,看看已是林鸟归山,夕阳西坠。因史湘云说起南边的话,便想着“父母若在,南边的景致,春花秋月,水秀山明,二十四桥,六朝遗迹。不少下人伏侍,诸事可以任意,言语亦可不避。香车画舫,红杏青帘,惟我独尊。今日寄人篱下,纵有许多照应,自己无处不要留心。不知前生作了什么罪孽,今生这样孤凄。真是李后主说的‘此间日中只以眼泪洗面’矣!”一面思想,不知不觉神往那里去了。
\end{parag}


\begin{parag}
    紫鹃走来,看见这样光景,想着必是因刚才说起南边北边的话来,一时触着黛玉的心事了,便问道:“姑娘们来说了半天话,想来姑娘又劳了神了。刚才我叫雪雁告诉厨房里给姑娘作了一碗火肉白菜汤,加了一点儿虾米儿,配了点青笋紫菜。姑娘想着好么?”黛玉道:“也罢了。”紫鹃道:“还熬了一点江米粥。”黛玉点点头儿,又说道:“那粥该你们两个自己熬了,不用他们厨房里熬才是。”紫鹃道:“我也怕厨房里弄的不干净,我们各自熬呢。就是那汤,我也告诉雪雁和柳嫂儿说了,要弄干净着。柳嫂儿说了,他打点妥当,拿到他屋里叫他们五儿瞅着炖呢。”黛玉道:“我倒不是嫌人家肮赃,只是病了好些日子,不周不备,都是人家。这会子又汤儿粥儿的调度,未免惹人厌烦。”说着,眼圈儿又红了。紫鹃道:“姑娘这话也是多想。姑娘是老太太的外孙女儿,又是老太太心坎儿上的。别人求其在姑娘跟前讨好儿还不能呢,那里有抱怨的。”黛玉点点头儿,因又问道:“你才说的五儿,不是那日和宝二爷那边的芳官在一处的那个女孩儿?”紫鹃道:“就是他。”黛玉道:“不听见说要进来么?”紫鹃道:“可不是,因为病了一场,后来好了才要进来,正是晴雯他们闹出事来的时候,也就耽搁住了。”黛玉道:“我看那丫头倒也还头脸儿干净。”说着,外头婆子送了汤来。雪雁出来接时,那婆子说道:“柳嫂儿叫回姑娘,这是他们五儿作的,没敢在大厨房里作,怕姑娘嫌肮赃。”雪雁答应着接了进来。黛玉在屋里已听见了,吩咐雪雁告诉那老婆子回去说,叫他费心。雪雁出来说了,老婆子自去。这里雪雁将黛玉的碗箸安放在小几儿上,因问黛玉道:“还有咱们南来的五香大头菜,拌些麻油醋可好么?”黛玉道:“也使得,只不必累赘了。”一面盛上粥来,黛玉吃了半碗,用羹匙舀了两口汤喝,就搁下了。两个丫鬟撤了下来,拭净了小几端下去,又换上一张常放的小几。黛玉漱了口,盥了手,便道:“紫鹃,添了香了没有?”紫鹃道:“就添去。”黛玉道:“你们就把那汤和粥吃了罢,味儿还好,且是干净。待我自己添香罢。”两个人答应了,在外间自吃去了。
\end{parag}


\begin{parag}
    这里黛玉添了香,自己坐着。才要拿本书看,只听得园内的风自西边直透到东边,穿过树枝,都在那里唏哩哗喇不住的响。一回儿,檐下的铁马也只管叮叮当当的乱敲起来。一时雪雁先吃完了,进来伺候。黛玉便问道:“天气冷了,我前日叫你们把那些小毛儿衣服晾晾,可曾晾过没有?”雪雁道:“都晾过了。”黛玉道:“你拿一件来我披披。”雪雁走去将一包小毛衣服抱来,打开毡包,给黛玉自拣。只见内中夹着个绢包儿,黛玉伸手拿起打开看时,却是宝玉病时送来的旧手帕,自己题的诗,上面泪痕犹在,里头却包着那剪破了的香囊扇袋并宝玉通灵玉上的穗子。原来晾衣服时从箱中捡出,紫鹃恐怕遗失了,遂夹在这毡包里的。这黛玉不看则已,看了时也不说穿那一件衣服,手里只拿着那两方手帕,呆呆的看那旧诗。看了一回,不觉的簌簌泪下。紫鹃刚从外间进来,只见雪雁正捧着一毡包衣裳在旁边呆立,小几上却搁着剪破的香囊,两三截儿扇袋和那铰折了的穗子,黛玉手中自拿着两方旧帕,上边写着字迹,在那里对着滴泪。正是:
\end{parag}


\begin{poem}
    \begin{pl}
        失意人逢失意事,新啼痕间旧啼痕。
    \end{pl}
\end{poem}


\begin{parag}
    紫鹃见了这样,知是他触物伤情,感怀旧事,料道劝也无益,只得笑着道:“姑娘还看那些东西作什么,那都是那几年宝二爷和姑娘小时一时好了,一时恼了,闹出来的笑话儿。要象如今这样斯抬斯敬,那里能把这些东西白遭塌了呢。”紫鹃这话原给黛玉开心,不料这几句话更提起黛玉初来时和宝玉的旧事来,一发珠泪连绵起来。紫鹃又劝道:“雪雁这里等着呢,姑娘披上一件罢。”那黛玉才把手帕撂下。紫鹃连忙拾起,将香袋等物包起拿开。这黛玉方披了一件皮衣,自己闷闷的走到外间来坐下。回头看见案上宝钗的诗启尚未收好,又拿出来瞧了两遍,叹道:“境遇不同,伤心则一。不免也赋四章,翻入琴谱,可弹可歌,明日写出来寄去,以当和作。”便叫雪雁将外边桌上笔砚拿来,濡墨挥毫,赋成四迭。又将琴谱翻出,借他《猗兰》《思贤》两操,合成音韵,与自己做的配齐了,然后写出,以备送与宝钗。又即叫雪雁向箱中将自己带来的短琴拿出,调上弦,又操演了指法。黛玉本是个绝顶聪明人,又在南边学过几时,虽是手生,到底一理就熟。抚了一番,夜已深了,便叫紫鹃收拾睡觉。不题。
\end{parag}


\begin{parag}
    却说宝玉这日起来梳洗了,带着焙茗正往书房中来,只见墨雨笑嘻嘻的跑来迎头说道:“二爷今日便宜了,太爷不在书房里,都放了学了。”宝玉道:“当真的么?”墨雨道:“二爷不信,那不是三爷和兰哥儿来了。”宝玉看时,只见贾环贾兰跟着小厮们,两个笑嘻的嘴里咭咭呱呱不知说些什么,迎头来了。见了宝玉,都垂手站住。宝玉问道:“你们两个怎么就回来了?”贾环道:“今日太爷有事,说是放一天学,明儿再去呢。”宝玉听了,方回身到贾母贾政处去禀明了,然后回到怡红院中。袭人问道:“怎么又回来了?”宝玉告诉了他,只坐了一坐儿,便往外走。袭人道:“往那里去,这样忙法?就放了学,依我说也该养养神儿了。”宝玉站住脚,低了头,说道:“你的话也是。但是好容易放一天学,还不散散去,你也该可怜我些儿了。”袭人见说的可怜,笑道:“由爷去罢。”正说着,端了饭来。宝玉也没法儿,只得且吃饭,三口两口忙忙的吃完,漱了口,一溜烟往黛玉房中去了。
\end{parag}


\begin{parag}
    走到门口,只见雪雁在院中晾绢子呢。宝玉因问:“姑娘吃了饭了么?”雪雁道:“早起喝了半碗粥,懒待吃饭。这时候打盹儿呢。二爷且到别处走走,回来再来罢。”宝玉只得回来。
\end{parag}


\begin{parag}
    无处可去,忽然想起惜春有好几天没见,便信步走到蓼风轩来。刚到窗下,只见静悄悄一无人声。宝玉打谅他也睡午觉,不便进去。才要走时,只听屋里微微一响,不知何声。宝玉站住再听,半日又拍的一响。宝玉还未听出,只见一个人道:“你在这里下了一个子儿,那里你不应么?”宝玉方知是下大棋,但只急切听不出这个人的语音是谁。底下方听见惜春道:“怕什么,你这么一吃我,我这么一应,你又这么吃,我又这么应。还缓着一着儿呢,终久连得上。”那一个又道:“我要这么一吃呢?”惜春道:“阿嗄,还有一着‘反扑’在里头呢!我倒没防备。”宝玉听了,听那一个声音很熟,却不是他们姊妹。料着惜春屋里也没外人,轻轻的掀帘进去。看时不是别人,却是那栊翠庵的槛外人妙玉。这宝玉见是妙玉,不敢惊动。妙玉和惜春正在凝思之际,也没理会。宝玉却站在旁边看他两个的手段。只见妙玉低着头问惜春道:“你这个‘畸角儿’不要了么?”惜春道:“怎么不要。你那里头都是死子儿,我怕什么。”妙玉道:“且别说满话,试试看。”惜春道:“我便打了起来,看你怎么样。”妙玉却微微笑着,把边上子一接,却搭转一吃,把惜春的一个角儿都打起来了,笑着说道:“这叫做‘倒脱靴势’。”
\end{parag}


\begin{parag}
    惜春尚未答言,宝玉在旁情不自禁,哈哈一笑,把两个人都唬了一大跳。惜春道:“你这是怎么说,进来也不言语,这么使促狭唬人。你多早晚进来的?”宝玉道:“我头里就进来了,看着你们两个争这个‘畸角儿’。”说着,一面与妙玉施礼,一面又笑问道:“妙公轻易不出禅关,今日何缘下凡一走?”妙玉听了,忽然把脸一红,也不答言,低了头自看那棋。宝玉自觉造次,连忙陪笑道:“倒是出家人比不得我们在家的俗人,头一件心是静的。静则灵,灵则慧。”宝玉尚未说完,只见妙玉微微的把眼一抬,看了宝玉一眼,复又低下头去,那脸上的颜色渐渐的红晕起来。宝玉见他不理,只得讪讪的旁边坐了。惜春还要下子,妙玉半日说道:“再下罢。”便起身理理衣裳,重新坐下,痴痴的问着宝玉道:“你从何处来?”宝玉巴不得这一声,好解释前头的话,忽又想道:“或是妙玉的机锋。”转红了脸答应不出来。妙玉微微一笑,自和惜春说话。惜春也笑道:“二哥哥,这什么难答的,你没的听见人家常说的‘从来处来’么。这也值得把脸红了,见了生人的似的。”妙玉听了这话,想起自家,心上一动,脸上一热,必然也是红的,倒觉不好意思起来。因站起来说道:“我来得久了,要回庵里去了。”惜春知妙玉为人,也不深留,送出门口。妙玉笑道:“久已不来这里,弯弯曲曲的,回去的路头都要迷住了。”宝玉道:“这倒要我来指引指引何如?”妙玉道:“不敢,二爷前请。”于是二人别了惜春,离了蓼风轩,弯弯曲曲,走近潇湘馆,忽听得叮咚之声。妙玉道:“那里的琴声?”宝玉道:“想必是林妹妹那里抚琴呢。”妙玉道:“原来他也会这个,怎么素日不听见提起?”宝玉悉把黛玉的事述了一遍,因说:“咱们去看他。”妙玉道:“从古只有听琴,再没有‘看琴’的。”宝玉笑道:“我原说我是个俗人。”说着,二人走至潇湘馆外,在山子石坐着静听,甚觉音调清切。只听得低吟道:
\end{parag}


\begin{poem}
    \begin{pl}
        风萧萧兮秋气深,美人千里兮独沉吟。望故乡兮何处,倚栏杆兮涕沾襟。
    \end{pl}
\end{poem}


\begin{parag}
    歇了一回,听得又吟道:
\end{parag}


\begin{poem}
    \begin{pl}
        山迢迢兮水长,照轩窗兮明月光。耿耿不寐兮银河渺茫,罗衫怯怯兮风露凉。
    \end{pl}
\end{poem}


\begin{parag}
    又歇了一歇。妙玉道:“刚才‘侵’字韵是第一迭,如今‘阳’字韵是第二迭了。咱们再听。”里边又吟道:
\end{parag}

\begin{poem}
    \begin{pl}
        子之遭兮不自由,予之遇兮多烦忧。之子与我兮心焉相投,思古人兮俾无尤。
    \end{pl}
\end{poem}


\begin{parag}
    妙玉道:“这又是一拍。何忧思之深也!”宝玉道:“我虽不懂得,但听他音调,也觉得过悲了。”里头又调了一回弦。妙玉道:“君弦太高了,与无射律只怕不配呢。”里边又吟道:
\end{parag}

\begin{poem}
    \begin{pl}
        人生斯世兮如轻尘,天上人间兮感夙因。感夙因兮不可惙,素心如何天上月。
    \end{pl}
\end{poem}


\begin{parag}
    妙玉听了,呀然失色道:“如何忽作变征之声?音韵可裂金石矣。只是太过。”宝玉道:“太过便怎么?”妙玉道:“恐不能持久。”正议论时,听得君弦蹦的一声断了。妙玉站起来连忙就走。宝玉道:“怎么样?”妙玉道:“日后自知,你也不必多说。”竟自走了。弄得宝玉满肚疑团,没精打彩的归至怡红院中,不表。单说妙玉归去,早有道婆接着,掩了庵门,坐了一回,把“禅门日诵”念了一遍。吃了晚饭,点上香拜了菩萨,命道婆自去歇着,自己的禅床靠背俱已整齐,屏息垂帘,跏趺坐下,断除妄想,趋向真如。坐到三更过后,听得屋上骨碌碌一片瓦响,妙玉恐有贼来,下了禅床,出到前轩,但见云影横空,月华如水。那时天气尚不很凉,独自一个凭栏站了一回,忽听房上两个猫儿一递一声厮叫。那妙玉忽想起日间宝玉之言,不觉一阵心跳耳热。自己连忙收慑心神,走进禅房,仍到禅床上坐了。怎奈神不守舍,一时如万马奔驰,觉得禅床便恍荡起来,身子已不在庵中。便有许多王孙公子要求娶他,又有些媒婆扯扯拽拽扶他上车,自己不肯去。一回儿又有盗贼劫他,持刀执棍的逼勒,只得哭喊求救。早惊醒了庵中女尼道婆等众,都拿火来照看。只见妙玉两手撒开,口中流沫。急叫醒时,只见眼睛直竖,两颧鲜红,骂道:“我是有菩萨保佑,你们这些强徒敢要怎么样!”众人都唬的没了主意,都说道:“我们在这里呢,快醒转来罢。”妙玉道:“我要回家去,你们有什么好人送我回去罢。”道婆道:“这里就是你住的房子。”说着,又叫别的女尼忙向观音前祷告,求了签,翻开签书看时,是触犯了西南角上的阴人。就有一个说:“是了。大观园中西南角上本来没有人住,阴气是有的。”一面弄汤弄水的在那里忙乱。那女尼原是自南边带来的,伏侍妙玉自然比别人尽心,围着妙玉,坐在禅床上。妙玉回头道:“你是谁?”女尼道:“是我。”妙玉仔细瞧了一瞧,道:“原来是你。”便抱住那女尼呜呜咽咽的哭起来,说道:“你是我的妈呀,你不救我,我不得活了。”那女尼一面唤醒他,一面给他揉着。道婆倒上茶来喝了,直到天明才睡了。
\end{parag}


\begin{parag}
    女尼便打发人去请大夫来看脉,也有说是思虑伤脾的,也有说是热入血室的,也有说是邪祟触犯的,也有说是内外感冒的,终无定论。后请得一个大夫来看了,问:“曾打坐过没有?”道婆说道:“向来打坐的。”大夫道:“这病可是昨夜忽然来的么?”道婆道:“是。”大夫道:“这是走魔入火的原故。”众人问:“有碍没有?”大夫道:“幸亏打坐不久,魔还入得浅,可以有救。”写了降伏心火的药,吃了一剂,稍稍平复些。外面那些游头浪子听见了,便造作许多谣言说:“这样年纪,那里忍得住。况且又是很风流的人品,很乖觉的性灵,以后不知飞在谁手里,便宜谁去呢。”过了几日,妙玉病虽略好,神思未复,终有些恍惚。
\end{parag}


\begin{parag}
    一日惜春正坐着,彩屏忽然进来回道:“姑娘知道妙玉师父的事吗?”惜春道:“他有什么事?”彩屏道:“我昨日听见邢姑娘和大奶奶那里说呢。他自从那日和姑娘下棋回去,夜间忽然中了邪,嘴里乱嚷说强盗来抢他来了,到如今还没好。姑娘你说这不是奇事吗。”惜春听了,默默无语,因想:“妙玉虽然洁净,毕竟尘缘未断。可惜我生在这种人家不便出家。我若出了家时,那有邪魔缠扰,一念不生,万缘俱寂。”想到这里,蓦与神会,若有所得,便口占一偈云:
\end{parag}

\begin{poem}
    \begin{pl}
        大造本无方,云何是应住。
    \end{pl}


    \begin{pl}
        既从空中来,应向空中去。
    \end{pl}
\end{poem}


\begin{parag}
    占毕,即命丫头焚香。自己静坐了一回,又翻开那棋谱来,把孔融王积薪等所著看了几篇。内中“荷叶包蟹势”、“黄莺搏兔势”都不出奇,“三十六局杀角势”一时也难会难记,独看到“八龙走马”,觉得甚有意思。正在那里作想,只听见外面一个人走进院来,连叫彩屏。未知是谁,下回分解。
\end{parag}