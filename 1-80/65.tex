\chap{六十五}{賈二舍偷娶尤二姨 尤三姐思嫁柳二郎}


\begin{parag}
    \begin{note}蒙回前總:筆筆敘二姐溫柔和順高鳳姐十倍,言語行事勝鳳姐五分,堪爲賈璉二房,所以深著鳳姐不念宗祠血食,爲賈宅第一罪人。綱目書法!\end{note}
\end{parag}


\begin{parag}
    \begin{note}蒙回前總:文有雙管齊下法,此文是也。事在寧府,卻把鳳姐之尖酸刻薄、平兒之任俠直鯁、李紈之號“菩薩”、探春之號“玫瑰”、林姑娘之“怕倒”、薛姑娘之“怕化”一時齊現,是何等妙文!\end{note}
\end{parag}


\begin{parag}
    話說賈璉賈珍賈蓉等三人商議,事事妥貼,至初二日,先將尤老和三姐送入新房。尤老一看,雖不似賈蓉口內之言,也十分齊備,母女二人已稱了心。鮑二夫婦見了如一盆火,趕著尤老一口一聲喚老孃,又或是老太太;趕著三姐喚三姨,或是姨娘。至次日五更天,一乘素轎,將二姐抬來。各色香燭紙馬,並鋪蓋以及酒飯,早已備得十分妥當。一時,賈璉素服坐了小轎而來,拜過天地,焚了紙馬。那尤老見二姐身上頭上煥然一新,不是在家模樣,十分得意。攙入洞房。是夜賈璉同他顛鸞倒鳳,百般恩愛,不消細說。
\end{parag}


\begin{parag}
    那賈璉越看越愛,越瞧越喜,不知怎生奉承這二姐,乃命鮑二等人不許提三說二的,直以奶奶稱之,自己也稱奶奶,竟將鳳姐一筆勾倒。有時回家中,只說在東府有事羈絆,鳳姐輩因知他和賈珍相得,自然是或有事商議,也不疑心。再家下人雖多,都不管這些事。便有那遊手好閒專打聽小事的人,也都去奉承賈璉,乘機討些便宜,誰肯去露風。於是賈璉深感賈珍不盡。賈璉一月出五兩銀子做天天的供給。若不來時,他母女三人一處喫飯;若賈璉來了,他夫妻二人一處喫,他母女便回房自喫。賈璉又將自己積年所有的梯己,一併搬了與二姐收著,又將鳳姐素日之爲人行事,枕邊衾內盡情告訴了他,只等一死,便接他進去。二姐聽了,自是願意。當下十來個人,倒也過起日子來,十分豐足。
\end{parag}


\begin{parag}
    眼見已是兩個月光景。這日賈珍在鐵檻寺作完佛事,晚間回家時,因與他姨妹久別,竟要去探望探望。先命小廝去打聽賈璉在與不在,小廝回來說不在。賈珍歡喜,將左右一概先遣回去,只留兩個心腹小童牽馬。一時,到了新房,已是掌燈時分,悄悄入去。兩個小廝將馬拴在圈內,自往下房去聽候。
\end{parag}


\begin{parag}
    賈珍進來,屋內才點燈,先看過了尤氏母女,然後二姐出見,賈珍仍喚二姨。大家喫茶,說了一回閒話。賈珍因笑說:“我作的這保山如何?若錯過了,打著燈籠還沒處尋,過日你姐姐還備了禮來瞧你們呢。”說話之間,尤二姐已命人預備下酒饌,關起門來,都是一家人,原無避諱。那鮑二來請安,賈珍便說:“你還是個有良心的小子,所以叫你來伏侍。日後自有大用你之處,不可在外頭喫酒生事。我自然賞你。倘或這裏短了什麼,你璉二爺事多,那裏人雜,你只管去回我。我們弟兄不比別人。”鮑二答應道:“是,小的知道。若小的不盡心,除非不要這腦袋了。”賈珍點頭說:“要你知道。”當下四人一處喫酒。尤二姐知局,便邀他母親說:“我怪怕的,媽同我到那邊走走來。”尤老也會意,便真個同他出來,只剩小丫頭們。賈珍便和三姐挨肩擦臉,百般輕薄起來。小丫頭子們看不過,也都躲了出去,憑他兩個自在取樂,不知作些什麼勾當。
\end{parag}


\begin{parag}
    跟的兩個小廝都在廚下和鮑二飲酒,鮑二女人上竈。忽見兩個丫頭也走了來嘲笑,要喫酒。鮑二因說:“姐兒們不在上頭伏侍,也偷來了。一時叫起來沒人,又是事。”他女人罵道:“胡塗渾嗆了的忘八!你撞喪那黃湯罷。撞喪醉了,夾著你那膫子挺你的屍去。叫不叫,與你屄相干!一應有我承當,風雨橫豎灑不著你頭上來。”這鮑二原因妻子發跡的,近日越發虧他。自己除賺錢喫酒之外,一概不管,賈璉等也不肯責備他,故他視妻如母,百依百隨,且喫夠了便去睡覺。這裏鮑二家的陪著這些丫鬟小廝喫酒,討他們的好,準備在賈珍前上好。
\end{parag}


\begin{parag}
    四人正喫的高興,忽聽扣門之聲,鮑二家的忙出來開門,看見是賈璉下馬,問有事無事。鮑二女人便悄悄告他說:“大爺在這裏西院裏呢。”賈璉聽了,便回至臥房。只見尤二姐和他母親都在房中,見他來了,二人面上便有些訕訕的。賈璉反推不知,只命:“快拿酒來,咱們喫兩杯好睡覺。我今日很乏了。”尤二姐忙上來陪笑接衣奉茶,問長問短。賈璉喜的心癢難受。一時鮑二家的端上酒來,二人對飲。他丈母不喫,自回房中睡去了。兩個小丫頭分了一個過來伏侍。
\end{parag}


\begin{parag}
    賈璉的心腹小童隆兒拴馬去,見已有了一匹馬,細瞧一瞧,知是賈珍的,心下會意,也來廚下。只見喜兒壽兒兩個正在那裏坐著喫酒,見他來了,也都會意,故笑道:“你這會子來的巧。我們因趕不上爺的馬,恐怕犯夜,往這裏來借宿一宵的。”隆兒便笑道:“有的是炕,只管睡。我是二爺使我送月銀的,交給了奶奶,我也不回去了。”喜兒便說:“我們喫多了,你來喫一鍾。”隆兒才坐下,端起杯來,忽聽馬棚內鬧將起來。原來二馬同槽,不能相容,互相蹶踢起來。隆兒等慌的忙放下酒杯,出來喝馬,好容易喝住,另拴好了,方進來。鮑二家的笑說:“你三人就在這裏罷,茶也現成了,我可去了。”說著,帶門出去。這裏喜兒喝了幾杯,已是楞子眼了。隆兒壽兒關了門,回頭見喜兒直挺挺的仰臥炕上,二人便推他說:“好兄弟,起來好生睡,只顧你一個人,我們就苦了。”那喜兒便說道:“咱們今兒可要公公道道的貼一爐子燒餅,要有一個充正經的人,我痛把你媽一肏。”隆兒壽兒見他醉了,也不必多說,只得吹了燈,將就睡下。
\end{parag}


\begin{parag}
    尤二姐聽見馬鬧,心下便不自安,只管用言語混亂賈璉。那賈璉吃了幾杯,春興發作,便命收了酒果,掩門寬衣。尤二姐只穿著大紅小襖,散挽烏雲,滿臉春色,比白日更增了顏色。賈璉摟他笑道:“人人都說我們那夜叉婆齊整,如今我看來,給你拾鞋也不要。”尤二姐道:“我雖標緻,卻無品行。看來到底是不標緻的好。”賈璉忙問道:“這話如何說?我卻不解。”尤二姐滴淚說道:“你們拿我作愚人待,什麼事我不知。我如今和你作了兩個月夫妻,日子雖淺,我也知你不是愚人。我生是你的人,死是你的鬼,如今既作了夫妻,我終身靠你,豈敢瞞藏一字。我算是有靠,將來我妹子卻如何結果?據我看來,這個形景恐非長策,要作長久之計方可。”賈璉聽了,笑道:“你且放心,我不是拈酸喫醋之輩。前事我已盡知,你也不必驚慌。你因妹夫倒是作兄的,自然不好意思,不如我去破了這例。”說著走了,便至西院中來,只見窗內燈燭輝煌,二人正喫酒取樂。
\end{parag}


\begin{parag}
    賈璉便推門進去,笑說:“大爺在這裏,兄弟來請安。”賈珍羞的無話,只得起身讓坐。賈璉忙笑道:“何必又作如此景象,咱們弟兄從前是如何樣來!大哥爲我操心,我今日粉身碎骨,感激不盡。大哥若多心,我意何安。從此以後,還求大哥如昔方好;不然,兄弟能可絕後,再不敢到此處來了。”說著,便要跪下。慌的賈珍連忙攙起,只說:“兄弟怎麼說,我無不領命。”賈璉忙命人:“看酒來,我和大哥喫兩杯。”又拉尤三姐說:“你過來,陪小叔子一杯。”賈珍笑著說:“老二,到底是你,哥哥必要喫幹這鍾。”說著,一揚脖。尤三姐站在炕上,指賈璉笑道:“你不用和我花馬弔嘴的,清水下雜麪,你喫我看見。見提著影戲人子上場,好歹別戳破這層紙兒。你別油蒙了心,打諒我們不知道你府上的事。這會子花了幾個臭錢,你們哥兒倆拿著我們姐兒兩個權當粉頭來取樂兒,你們就打錯了算盤了。我也知道你那老婆太難纏,如今把我姐姐拐了來做二房,偷的鑼兒敲不得。我也要會會那鳳奶奶去,看他是幾個腦袋幾隻手。若大家好取和便罷;倘若有一點叫人過不去,我有本事先把你兩個的牛黃狗寶掏了出來,再和那潑婦拼了這命,也不算是尤三姑奶奶!喝酒怕什麼,咱們就喝!”說著,自己綽起壺來斟了一杯,自己先喝了半杯,摟過賈璉的脖子來就灌,說:“我和你哥哥已經喫過了,咱們來親香親香。”唬的賈璉酒都醒了。賈珍也不承望尤三姐這等無恥老辣。弟兄兩個本是風月場中耍慣的,不想今日反被這閨女一席話說住。尤三姐一疊聲又叫:“將姐姐請來,要樂咱們四個一處同樂。俗語說‘便宜不過當家’,他們是弟兄,咱們是姊妹,又不是外人,只管上來。”尤二姐反不好意思起來。賈珍得便就要一溜,尤三姐那裏肯放。賈珍此時方後悔,不承望他是這種爲人,與賈璉反不好輕薄起來。
\end{parag}


\begin{parag}
    這尤三姐鬆鬆挽著頭髮,大紅襖子半掩半開,露著蔥綠抹胸,一痕雪脯。底下綠褲紅鞋,一對金蓮或翹或並,沒半刻斯文。兩個墜子卻似打鞦韆一般,燈光之下,越顯得柳眉籠翠霧,檀口點丹砂。本是一雙秋水眼,再吃了酒,又添了餳澀淫浪,不獨將他二姊壓倒,據珍璉評去,所見過的上下貴賤若干女子,皆未有此綽約風流者。二人已酥麻如醉,不禁去招他一招,他那淫態風情,反將二人禁住。那尤三姐放出手眼來略試了一試,他弟兄兩個竟全然無一點別識別見,連口中一句響亮話都沒了,不過是酒色二字而已。自己高談闊論,任意揮霍灑落一陣,拿他弟兄二人嘲笑取樂,竟真是他嫖了男人,並非男人淫了他。一時他的酒足興盡,也不容他弟兄多坐,攆了出去,自己關門睡去了。
\end{parag}


\begin{parag}
    自此後,或略有丫鬟婆娘不到之處,便將賈璉、賈珍、賈蓉三個潑聲厲言痛罵,說他爺兒三個誆騙了他寡婦孤女。賈珍回去之後,以後亦不敢輕易再來,有時尤三姐自己高了興悄命小廝來請,方敢去一會,到了這裏,也只好隨他的便。誰知這尤三姐天生脾氣不堪,仗著自己風流標致,偏要打扮的出色,另式作出許多萬人不及的淫情浪態來,哄的男子們垂涎落魄,欲近不能,欲遠不捨,迷離顛倒,他以爲樂。他母姊二人也十分相勸,他反說:“姐姐糊塗。咱們金玉一般的人,白叫這兩個現世寶沾污了去,也算無能。而且他家有一個極利害的女人,如今瞞著他不知,咱們方安。倘或一日他知道了,豈有干休之理,勢必有一場大鬧,不知誰生誰死。趁如今我不拿他們取樂作踐准折,到那時白落個臭名,後悔不及。”因此一說,他母女見不聽勸,也只得罷了。那尤三姐天天挑揀穿喫,打了銀的,又要金的;有了珠子,又要寶石;喫的肥鵝,又宰肥鴨。或不趁心,連桌一推;衣裳不如意,不論綾緞新整,便用剪刀剪碎,撕一條,罵一句,究竟賈珍等何曾隨意了一日,反花了許多昧心錢。
\end{parag}


\begin{parag}
    賈璉來了,只在二姐房內,心中也悔上來。無奈二姐倒是個多情人,以爲賈璉是終身之主了,凡事倒還知疼著癢。若論起溫柔和順,凡事必商必議,不敢恃才自專,實較鳳姐高十倍;若論標緻,言談行事,也勝五分。雖然如今改過,但已經失了腳,有了一個“淫”字,憑他有甚好處也不算了。偏這賈璉又說:“誰人無錯,知過必改就好。”故不提已往之淫,只取現今之善,便如膠授漆,似水如魚,一心一計,誓同生死,那裏還有鳳平二人在意了?二姐在枕邊衾內,也常勸賈璉說: “你和珍大哥商議商議,揀個熟的人,把三丫頭聘了罷。留著他不是常法子,終久要生出事來,怎麼處?”賈璉道:“前日我曾回過大哥的,他只是捨不得。我說 ‘是塊肥羊肉,只是燙的慌;玫瑰花兒可愛,刺大扎手。咱們未必降的住,正經揀個人聘了罷。’他只意意思思,就丟開手了。你叫我有何法。”二姐道:“你放心。咱們明日先勸三丫頭,他肯了,讓他自己鬧去。鬧的無法,少不得聘他。”賈璉聽了說:“這話極是。”
\end{parag}


\begin{parag}
    至次日,二姐另備了酒,賈璉也不出門,至午間特請他小妹過來,與他母親上坐。尤三姐便知其意,\begin{note}庚雙夾:全用醍醐灌頂,全是大翻身大解悟法。\end{note}酒過三巡,不用姐姐開口,先便滴淚泣道:\begin{note}庚雙夾:全用如是等語一洗孽障。\end{note}“姐姐今日請我,自有一番大禮要說。但妹子不是那愚人,也不用絮絮叨叨提那從前醜事,我已盡知,說也無益。既如今姐姐也得了好處安身,媽也有了安身之處,我也要自尋歸結去,方是正理。但終身大事,一生至一死,非同兒戲。我如今改過守分,只要我揀一個素日可心如意的人方跟他去。若憑你們揀擇,雖是富比石崇,才過子建,貌比潘安的,我心裏進不去,也白過了一世。”賈璉笑道:“這也容易。憑你說是誰就是誰,一應彩禮都有我們置辦,母親也不用操心。”尤三姐泣道:“姐姐知道,不用我說。”賈璉笑問二姐是誰,二姐一時也想不起來。大家想來,賈璉便道:“定是此人無移了!”便拍手笑道:“我知道了。這人原不差,果然好眼力。”二姐笑問是誰,賈璉笑道:“別人他如何進得去,一定是寶玉。” 二姐與尤老聽了,亦以爲然。尤三姐便啐了一口,道:\begin{note}庚雙夾:奇,不知何爲。\end{note}“我們有姊妹十個,也嫁你弟兄十個不成?\begin{note}庚雙夾:有理之極!\end{note}難道除了你家,天下就沒了好男子了不成!”\begin{note}庚雙夾:一罵反有理。\end{note}衆人聽了都詫異:“除去他,還有那一個?”\begin{note}庚雙夾:餘亦如此想。\end{note}尤三姐笑道:“別隻在眼前想,姐姐只在五年前想就是了。”\begin{note}庚雙夾:奇甚!\end{note}
\end{parag}


\begin{parag}
    正說著,忽見賈璉的心腹小廝興兒走來請賈璉說:“老爺那邊緊等著叫爺呢。小的答應往舅老爺那邊去了,小的連忙來請。”賈璉又忙問:“昨日家裏沒人問?” 興兒道:“小的回奶奶說,爺在家廟裏同珍大爺商議作百日的事,只怕不能來家。”賈璉忙命拉馬,隆兒跟隨去了,留下興兒答應人來事務。
\end{parag}


\begin{parag}
    尤二姐拿了兩碟菜,命拿大杯斟了酒,就命興兒在炕沿下蹲著喫,一長一短向他說話兒。問他家裏奶奶多大年紀,怎個利害的樣子,老太太多大年紀,太太多大年紀,姑娘幾個,各樣家常等語。興兒笑嘻嘻的在炕沿下一頭喫,一頭將榮府之事備細告訴他母女。又說:“我是二門上該班的人。我們共是兩班,一班四個,共是八個。這八個人有幾個是奶奶的心腹,有幾個是爺的心腹。奶奶的心腹我們不敢惹,爺的心腹奶奶的就敢惹。提起我們奶奶來,心裏歹毒,口裏尖快。我們二爺也算是個好的,那裏見得他。倒是跟前的平姑娘爲人很好,雖然和奶奶一氣,他倒背著奶奶常作些個好事。小的們凡有了不是,奶奶是容不過的,只求求他去就完了。如今合家大小除了老太太、太太兩個人,沒有不恨他的,只不過面子情兒怕他。皆因他一時看的人都不及他,只一味哄著老太太、太太兩個人喜歡。他說一是一,說二是二,沒人敢攔他。又恨不得把銀子錢省下來堆成山,好叫老太太、太太說他會過日子,殊不知苦了下人,他討好兒。估著有好事,他就不等別人去說,他先抓尖兒;或有了不好事或他自己錯了,他便一縮頭推到別人身上來,他還在旁邊撥火兒。如今連他正經婆婆大太太都嫌了他,說他‘雀兒揀著旺處飛,黑母雞一窩兒,自家的事不管,倒替人家去瞎張羅’。若不是老太太在頭裏,早叫過他去了。”尤二姐笑道:“你背著他這等說他,將來你又不知怎麼說我呢。我又差他一層兒,越發有的說了。”興兒忙跪下說道:“奶奶要這樣說,小的不怕雷打!但凡小的們有造化起來,先娶奶奶時若得了奶奶這樣的人,小的們也少挨些打罵,也少提心吊膽的。如今跟爺的這幾個人,誰不背前背後稱揚奶奶聖德憐下。我們商量著叫二爺要出來,情願來答應奶奶呢。”尤二姐笑道:“猴兒肏的,還不起來呢。說句頑話,就唬的那樣起來。你們作什麼來,我還要找了你奶奶去呢。”興兒連忙搖手說:“奶奶千萬不要去。我告訴奶奶,一輩子別見他纔好。嘴甜心苦,兩面三刀;上頭一臉笑,腳下使絆子;明是一盆火,暗是一把刀:都佔全了。只怕三姨的這張嘴還說他不過。好,奶奶這樣斯文良善人,那裏是他的對手!”尤氏笑道:“我只以禮待他,他敢怎麼樣!”興兒道:“不是小的吃了酒放肆胡說,奶奶便有禮讓,他看見奶奶比他標緻,又比他得人心,他怎肯幹休善罷?人家是醋罐子,他是醋缸醋甕。凡丫頭們二爺多看一眼,他有本事當著爺打個爛羊頭。雖然平姑娘在屋裏,大約一年二年之間兩個有一次到一處,他還要口裏掂十個過子呢,氣的平姑娘性子發了,哭鬧一陣,說:‘又不是我自己尋來的,你又浪著勸我,我原不依,你反說我反了,這會子又這樣。’他一般的也罷了,倒央告平姑娘。”尤二姐笑道:“可是扯謊?這樣一個夜叉,怎麼反怕屋裏的人呢?”興兒道:“這就是俗語說的‘天下逃不過一個理字去’了。這平兒是他自幼的丫頭,陪了過來一共四個,嫁人的嫁人,死的死了,只剩了這個心腹。他原爲收了屋裏,一則顯他賢良名兒,二則又叫拴爺的心,好不外頭走邪的。又還有一段因果:我們家的規矩,凡爺們大了,未娶親之先都先放兩個人伏侍的。二爺原有兩個,誰知他來了沒半年,都尋出不是來,都打發出去了。別人雖不好說,自己臉上過不去,所以強逼著平姑娘作了房裏人。那平姑娘又是個正經人,從不把這一件事放在心上,也不會挑妻窩夫的,倒一味忠心赤膽伏侍他,才容下了。”
\end{parag}


\begin{parag}
    尤二姐笑道:“原來如此。但我聽見你們家還有一位寡婦奶奶和幾位姑娘。他這樣利害,這些人如何依得?”興兒拍手笑道:“原來奶奶不知道。我們家這位寡婦奶奶,他的渾名叫作‘大菩薩’,第一個善德人。我們家的規矩又大,寡婦奶奶們不管事,只宜清淨守節。妙在姑娘又多,只把姑娘們交給他,看書寫字,學針線,學道理,這是他的責任。除此問事不知,說事不管。只因這一向他病了,事多,這大奶奶暫管幾日。究竟也無可管,不過是按例而行,不象他多事逞才。我們大姑娘不用說,但凡不好也沒這段大福了。二姑娘的渾名是‘二木頭’,戳一針也不知噯喲一聲。三姑娘的渾名是‘玫瑰花’。”尤氏姊妹忙笑問何意。興兒笑道: “玫瑰花又紅又香,無人不愛的,只是刺戳手。也是一位神道,可惜不是太太養的,‘老鴰窩裏出鳳凰’。四姑娘小,他正經是珍大爺親妹子,因自幼無母,老太太命太太抱過來養這麼大,也是一位不管事的。奶奶不知道,我們家的姑娘不算,另外有兩個姑娘,真是天上少有,地下無雙。一個是咱們姑太太的女兒,姓林,小名兒叫什麼黛玉,面龐身段和三姨不差什麼,一肚子文章,只是一身多病,這樣的天,還穿夾的,出來風兒一吹就倒了。我們這起沒王法的嘴都悄悄的叫他‘多病西施 ’。還有一位姨太太的女兒,姓薛,叫什麼寶釵,竟是雪堆出來的。每常出門或上車。或一時院子裏瞥見一眼,我們鬼使神差,見了他兩個,不敢出氣兒。”尤二姐笑道:“你們大家規矩,雖然你們小孩子進的去,然遇見小姐們,原該遠遠藏開。”興兒搖手道:“不是,不是。那正經大禮,自然遠遠的藏開,自不必說。就藏開了,自己不敢出氣,是生怕這氣大了,吹倒了姓林的;氣暖了,吹化了姓薛的。”說的滿屋裏都笑起來了。不知端詳,且聽下回分解。
\end{parag}


\begin{parag}
    \begin{note}蒙回後總評:房內兄弟聚麀,棚內兩馬相鬧;小廝與賈母飲酒,小姨與姐夫同牀。可見有是主必有奴(是)奴,有是兄必有是弟,有是姐必有是妹,有是人必有是馬。\end{note}
\end{parag}
