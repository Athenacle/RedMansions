\chap{一百一十七}{阻超凡佳人双护玉 欣聚党恶子独承家}



\begin{parag}
    话说王夫人打发人来叫宝钗过去商量,宝玉听见说是和尚在外头,赶忙的独自一人走到前头,嘴里乱嚷道:“我的师父在那里?”叫了半天,并不见有和尚,只得走到外面。见李贵将和尚拦住,不放他进来。宝玉便说道:“太太叫我请师父进去。”李贵听了松了手,那和尚便摇摇摆摆的进去。宝玉看见那僧的形状与他死去时所见的一般,心里早有些明白了,便上前施礼,连叫:“师父,弟子迎候来迟。”那僧说:“我不要你们接待,只要银子,拿了来我就走。”宝玉听来又不象有道行的话,看他满头癞疮,混身腌臜破烂,心里想道:“自古说‘真人不露相,露相不真人’,也不可当面错过,我且应了他谢银,并探探他的口气。”便说道:“师父不必性急,现在家母料理,请师父坐下略等片刻。弟子请问,师父可是从‘太虚幻境’而来?”那和尚道:“什么幻境,不过是来处来去处去罢了!我是送还你的玉来的。我且问你,那玉是从那里来的?”宝玉一时对答不来。那僧笑道:“你自己的来路还不知,便来问我!”宝玉本来颖悟,又经点化,早把红尘看破,只是自己的底里未知,一闻那僧问起玉来,好象当头一棒,便说道:“你也不用银子了,我把那玉还你罢。”那僧笑道:“也该还我了。”
\end{parag}


\begin{parag}
    宝玉也不答言,往里就跑,走到自己院内,见宝钗袭人等都到王夫人那里去了,忙向自己床边取了那玉便走出来。迎面碰见了袭人,撞了一个满怀,把袭人唬了一跳,说道:“太太说,你陪着和尚坐着很好,太太在那里打算送他些银两。你又回来做什么?”宝玉道:“你快去回太太,说不用张罗银两了,我把这玉还了他就是了。”袭人听说,即忙拉住宝玉道:“这断使不得的!那玉就是你的命,若是他拿去了,你又要病着了。”宝玉道:“如今不再病的了,我已经有了心了,要那玉何用!”摔脱袭人,便要想走。袭人急得赶着嚷道:“你回来,我告诉你一句话。”宝玉回过头来道:“没有什么说的了。”袭人顾不得什么,一面赶着跑,一面嚷道:“上回丢了玉,几乎没有把我的命要了!刚刚儿的有了,你拿了去,你也活不成,我也活不成了!你要还他,除非是叫我死了!”说着,赶上一把拉住。宝玉急了道:“你死也要还,你不死也要还!”狠命的把袭人一推,抽身要走。怎奈袭人两只手绕着宝玉的带子不放松,哭喊着坐在地下。里面的丫头听见连忙赶来,瞧见他两个人的神情不好,只听见袭人哭道:“快告诉太太去,宝二爷要把那玉去还和尚呢!”丫头赶忙飞报王夫人。那宝玉更加生气,用手来掰开了袭人的手,幸亏袭人忍痛不放。紫鹃在屋里听见宝玉要把玉给人,这一急比别人更甚,把素日冷淡宝玉的主意都忘在九霄云外了,连忙跑出来帮着抱住宝玉。那宝玉虽是个男人,用力摔打,怎奈两个人死命的抱住不放,也难脱身,叹口气道:“为一块玉这样死命的不放,若是我一个人走了,又待怎么样呢?”袭人紫鹃听到那里,不禁嚎啕大哭起来。正在难分难解,王夫人宝钗急忙赶来,见是这样形景,便哭着喝道:“宝玉,你又疯了吗!”宝玉见王夫人来了,明知不能脱身,只得陪笑说道:“这当什么,又叫太太着急。他们总是这样大惊小怪的,我说那和尚不近人情,他必要一万银子,少一个不能。我生气进来拿这玉还他,就说是假的,要这玉干什么。他见得我们不希罕那玉,便随意给他些就过去了。”王夫人道:“我打谅真要还他,这也罢了。为什么不告诉明白了他们,叫他们哭哭喊喊的象什么。”宝钗道:“这么说呢倒还使得。要是真拿那玉给他,那和尚有些古怪,倘或一给了他,又闹到家口不宁,岂不是不成事了么?至于银钱呢,就把我的头面折变了,也还够了呢。”王夫人听了道:“也罢了,且就这么办罢。”宝玉也不回答。只见宝钗走上来在宝玉手里拿了这玉,说道:“你也不用出去,我合太太给他钱就是了。”宝玉道:“玉不还他也使得,只是我还得当面见他一见才好。”袭人等仍不肯放手,到底宝钗明决,说:“放了手由他去就是了。”袭人只得放手。宝玉笑道:“你们这些人原来重玉不重人哪。你们既放了我,我便跟着他走了,看你们就守着那块玉怎么样!”袭人心里又着急起来,仍要拉他,只碍着王夫人和宝钗的面前,又不好太露轻薄。恰好宝玉一撒手就走了。袭人忙叫小丫头在三门口传了焙茗等,”告诉外头照应着二爷,他有些疯了。”小丫头答应了出去。
\end{parag}


\begin{parag}
    王夫人宝钗等进来坐下,问起袭人来由,袭人便将宝玉的话细细说了。王夫人宝钗甚是不放心,又叫人出去吩咐众人伺候,听着和尚说些什么。回来小丫头传话进来回王夫人道:“二爷真有些疯了。外头小厮们说,里头不给他玉,他也没法,如今身子出来了,求着那和尚带了他去。”王夫人听了说道:“这还了得!那和尚说什么来着?”小丫头回道:“和尚说要玉不要人。”宝钗道:“不要银子了么?”小丫头道:“没听见说,后来和尚和二爷两个人说着笑着,有好些话外头小厮们都不大懂。”王夫人道:“糊涂东西,听不出来,学是自然学得来的。”便叫小丫头:“你把那小厮叫进来。”小丫头连忙出去叫进那小厮,站在廊下,隔着窗户请了安。王夫人便问道:“和尚和二爷的话你们不懂,难道学也学不来吗?”那小厮回道:“我们只听见说什么‘大荒山’,什么‘青埂峰’,又说什么‘太虚境’,‘斩断尘缘’这些话。”王夫人听了也不懂。宝钗听了,唬得两眼直瞪,半句话都没有了。正要叫人出去拉宝玉进来,只见宝玉笑嘻嘻的进来说:“好了,好了。”宝钗仍是发怔。王夫人道:“你疯疯颠颠的说的是什么?”宝玉道:“正经话又说我疯颠。那和尚与我原是认得的,他不过也是要来见我一见。他何尝是真要银子呢,也只当化个善缘就是了。所以说明了他自己就飘然而去了。这可不是好了么!”王夫人不信,又隔着窗户问那小厮。那小厮连忙出去问了门上的人,进来回说:“果然和尚走了。说请太太们放心,我原不要银子,只要宝二爷时常到他那里去去就是了。诸事只要随缘,自有一定的道理。”王夫人道:“原来是个好和尚,你们曾问住在那里?”门上道:“奴才也问来着,他说我们二爷是知道的。”王夫人问宝玉道:“他到底住在那里?”宝玉笑道:“这个地方说远就远,说近就近。”宝钗不待说完,便道:“你醒醒儿罢,别尽着迷在里头。现在老爷太太就疼你一个人,老爷还吩咐叫你干功名长进呢。”宝玉道:“我说的不是功名么!你们不知道,‘一子出家,七祖升天’呢。”王夫人听到那里,不觉伤心起来,说:“我们的家运怎么好,一个四丫头口口声声要出家,如今又添出一个来了。我这样个日子过他做什么!”说着,大哭起来。宝钗见王夫人伤心,只得上前苦劝。宝玉笑道:“我说了这一句顽话,太太又认起真来了。”王夫人止住哭声道:“这些话也是混说的么!”正闹着,只见丫头来回话:“琏二爷回来了,颜色大变,说请太太回去说话。”王夫人又吃了一惊,说道:“将就些,叫他进来罢,小婶子也是旧亲,不用回避了。”贾琏进来,见了王夫人请了安。宝钗迎着也问了贾琏的安。回说道:“刚才接了我父亲的书信,说是病重的很,叫我就去,若迟了恐怕不能见面。”说到那里,眼泪便掉下来了。王夫人道:“书上写的是什么病?”贾琏道:“写的是感冒风寒起来的,如今成了痨病了。现在危急,专差一个人连日连夜赶来的,说如若再耽搁一两天就不能见面了。故来回太太,侄儿必得就去才好。只是家里没人照管。蔷儿芸儿虽说糊涂,到底是个男人,外头有了事来还可传个话。侄儿家里倒没有什么事,秋桐是天天哭着喊着不愿意在这里,侄儿叫了他娘家的人来领了去了,倒省了平儿好些气。虽是巧姐没人照应,还亏平儿的心不很坏。妞儿心里也明白,只是性气比他娘还刚硬些,求太太时常管教管教他。”说着眼圈儿一红,连忙把腰里拴槟榔荷包的小绢子拉下来擦眼。王夫人道:“放着他亲祖母在那里,托我做什么。”贾琏轻轻的说道:“太太要说这个话,侄儿就该活活儿的打死了。没什么说的,总求太太始终疼侄儿就是了。”说着,就跪下来了。王夫人也眼圈儿红了,说:“你快起来,娘儿们说话儿,这是怎么说。只是一件,孩子也大了,倘或你父亲有个一差二错又耽搁住了,或者有个门当户对的来说亲,还是等你回来,还是你太太作主?”贾琏道:“现在太太们在家,自然是太太们做主,不必等我。”王夫人道:“你要去,就写了禀帖给二老爷送个信,说家下无人,你父亲不知怎样,快请二老爷将老太太的大事早早的完结,快快回来。”贾琏答应了“是”,正要走出去,复转回来回说道:“咱们家的家下人家里还够使唤,只是园里没有人太空了。包勇又跟了他们老爷去了。姨太太住的房子,薛二爷已搬到自己的房子内住了。园里一带屋子都空着,忒没照应,还得太太叫人常查看查看。那栊翠庵原是咱们家的地基,如今妙玉不知那里去了,所有的根基他的当家女尼不敢自己作主,要求府里一个人管理管理。”王夫人道:“自己的事还闹不清,还搁得住外头的事么。这句话好歹别叫四丫头知道,若是他知道了,又要吵着出家的念头出来了。你想咱们家什么样的人家,好好的姑娘出了家,还了得!”贾琏道:“太太不提起侄儿也不敢说,四妹妹到底是东府里的,又没有父母,他亲哥哥又在外头,他亲嫂子又不大说的上话。侄儿听见要寻死觅活了好几次。他既是心里这么着的了,若是牛着他,将来倘或认真寻了死,比出家更不好了。”王夫人听了点头道:“这件事真真叫我也难担。我也做不得主,由他大嫂子去就是了。”
\end{parag}


\begin{parag}
    贾琏又说了几句才出来,叫了众家人来交待清楚,写了书,收拾了行装,平儿等不免叮咛了好些话。只有巧姐儿惨伤的了不得,贾琏又欲托王仁照应,巧姐到底不愿意,听见外头托了芸蔷二人,心里更不受用,嘴里却说不出来,只得送了他父亲,谨谨慎慎的随着平儿过日子。丰儿小红因凤姐去世,告假的告假,告病的告病,平儿意欲接了家中一个姑娘来,一则给巧姐作伴,二则可以带量他。遍想无人,只有喜鸾四姐儿是贾母旧日钟爱的,偏偏四姐儿新近出了嫁了,喜鸾也有了人家儿,不日就要出阁,也只得罢了。
\end{parag}


\begin{parag}
    且说贾芸贾蔷送了贾琏,便进来见了邢王二夫人。他两个倒替着在外书房住下,日间便与家人厮闹,有时找了几个朋友吃个车箍辘会,甚至聚赌,里头那里知道。一日邢大舅王仁来,瞧见了贾芸贾蔷住在这里,知他热闹,也就借着照看的名儿时常在外书房设局赌钱喝酒。所有几个正经的家人,贾政带了几个去,贾琏又跟去了几个,只有那赖林诸家的儿子侄儿。那些少年托着老子娘的福吃喝惯了的,那知当家立计的道理。况且他们长辈都不在家,便是没笼头的马了,又有两个旁主人怂恿,无不乐为。这一闹,把个荣国府闹得没上没下,没里没外。那贾蔷还想勾引宝玉,贾芸拦住道:“宝二爷那个人没运气的,不用惹他。那一年我给他说了一门子绝好的亲,父亲在外头做税官,家里开几个当铺,姑娘长的比仙女儿还好看。我巴巴儿的细细的写了一封书子给他,谁知他没造化,——”说到这里,瞧了瞧左右无人,又说:“他心里早和咱们这个二婶娘好上了。你没听见说,还有一个林姑娘呢,弄的害了相思病死的,谁不知道。这也罢了,各自的姻缘罢咧。谁知他为这件事倒恼了我了,总不大理。他打谅谁必是借谁的光儿呢。”贾蔷听了点点头,才把这个心歇了。
\end{parag}


\begin{parag}
    他两个还不知道宝玉自会那和尚以后,他是欲断尘缘。一则在王夫人跟前不敢任性,已与宝钗袭人等皆不大款洽了。那些丫头不知道,还要逗他,宝玉那里看得到眼里。他也并不将家事放在心里。时常王夫人宝钗劝他念书,他便假作攻书,一心想着那个和尚引他到那仙境的机关。心目中触处皆为俗人,却在家难受,闲来倒与惜春闲讲。他们两个人讲得上了,那种心更加准了几分,那里还管贾环贾兰等。那贾环为他父亲不在家,赵姨娘已死,王夫人不大理会他,便入了贾蔷一路。倒是彩云时常规劝,反被贾环辱骂。玉钏儿见宝玉疯颠更甚,早和他娘说了要求着出去。如今宝玉贾环他哥儿两个各有一种脾气,闹得人人不理。独有贾兰跟着他母亲上紧攻书,作了文字送到学里请教代儒。因近来代儒老病在床,只得自己刻苦。李纨是素来沉静,除了请王夫人的安,会会宝钗,余者一步不走,只有看着贾兰攻书。所以荣府住的人虽不少,竟是各自过各自的,谁也不肯做谁的主。贾环贾蔷等愈闹的不象事了,甚至偷典偷卖,不一而足。贾环更加宿娼滥赌,无所不为。
\end{parag}


\begin{parag}
    一日邢大舅王仁都在贾家外书房喝酒,一时高兴,叫了几个陪酒的来唱着喝着劝酒。贾蔷便说:“你们闹的太俗。我要行个令儿。”众人道:“使得。”贾蔷道:“咱们‘月’字流觞罢。我先说起‘月’字,数到那个便是那个喝酒,还要酒面酒底。须得依着令官,不依者罚三大杯。”众人都依了。贾蔷喝了一杯令酒,便说:“飞羽觞而醉月。”顺饮数到贾环。贾蔷说:“酒面要个‘桂’字。”贾环便说道“‘冷露无声湿桂花’。酒底呢?”贾蔷道:“说个‘香’字。”贾环道:“天香云外飘。”大舅说道:“没趣,没趣。你又懂得什么字了,也假斯文起来!这不是取乐,竟是怄人了。咱们都蠲了,倒是搳搳拳,输家喝输家唱,叫做‘苦中苦’。若是不会唱的,说个笑话儿也使得,只要有趣。”众人都道:“使得。”于是乱搳起来。王仁输了,喝了一杯,唱了一个。众人道好,又搳起来了。是个陪酒的输了,唱了一个什么“小姐小姐多丰彩”。以后邢大舅输了,众人要他唱曲儿,他道:“我唱不上来的,我说个笑话儿罢。”贾蔷道:“若说不笑仍要罚的。”邢大舅就喝了杯,便说道:“诸位听着:村庄上有一座元帝庙,旁边有个土地祠。那元帝老爷常叫土地来说闲话儿。一日元帝庙里被了盗,便叫土地去查访。土地禀道:‘这地方没有贼的,必是神将不小心,被外贼偷了东西去。’元帝道:‘胡说,你是土地,失了盗不问你问谁去呢?你倒不去拿贼,反说我的神将不小心吗?’土地禀道:‘虽说是不小心,到底是庙里的风水不好。’元帝道:‘你倒会看风水么?’土地道:‘待小神看看。’那土地向各处瞧了一会,便来回禀道:‘老爷坐的身子背后两扇红门就不谨慎。小神坐的背后是砌的墙,自然东西丢不了。以后老爷的背后亦改了墙就好了。’元帝老爷听来有理,便叫神将派人打墙。众神将叹口气道:‘如今香火一炷也没有,那里有砖灰人工来打墙!’元帝老爷没法,叫众神将作法,却都没有主意。那元帝老爷脚下的龟将军站起来道:‘你们不中用,我有主意。你们将红门拆下来,到了夜里拿我的肚子垫住这门口,难道当不得一堵墙么?’众神将都说道:‘好,又不花钱,又便当结实。’于是龟将军便当这个差使,竟安静了。岂知过了几天,那庙里又丢了东西。众神将叫了土地来说道:‘你说砌了墙就不丢东西,怎么如今有了墙还要丢?’那土地道:‘这墙砌的不结实。’众神将道:‘你瞧去。’土地一看,果然是一堵好墙,怎么还有失事?把手摸了一摸道:‘我打谅是真墙,那里知道是个假墙!’”众人听了大笑起来。贾蔷也忍不住的笑,说道:“傻大舅,你好!我没有骂你,你为什么骂我!快拿杯来罚一大杯。”邢大舅喝了,已有醉意。众人又喝了几杯,都醉起来。邢大舅说他姐姐不好,王仁说他妹妹不好,都说的狠狠毒毒的。贾环听了,趁着酒兴也说凤姐不好,怎样苛刻我们,怎么样踏我们的头。众人道:“大凡做个人,原要厚道些。看凤姑娘仗着老太太这样的利害,如今焦了尾巴梢子了,只剩了一个姐儿,只怕也要现世现报呢。”贾芸想着凤姐待他不好,又想起巧姐儿见他就哭,也信着嘴儿混说。还是贾蔷道:“喝酒罢,说人家做什么。”那两个陪酒的道:“这位姑娘多大年纪了?长得怎么样?”贾蔷道:“模样儿是好的很的。年纪也有十三四岁了。”那陪酒的说道:“可惜这样人生在府里这样人家,若生在小户人家,父母兄弟都做了官,还发了财呢。”众人道:“怎么样?”那陪酒的说:“现今有个外藩王爷,最是有情的,要选一个妃子。若合了式,父母兄弟都跟了去。可不是好事儿吗?”众人都不大理会,只有王仁心里略动了一动,仍旧喝酒。
\end{parag}


\begin{parag}
    只见外头走进赖林两家的子弟来,说:“爷们好乐呀!”众人站起来说道:“老大老三怎么这时候才来?叫我们好等!”那两个人说道:“今早听见一个谣言,说是咱们家又闹出事来了,心里着急,赶到里头打听去,并不是咱们。”众人道:“不是咱们就完了,为什么不就来?”那两个说道:“虽不是咱们,也有些干系。你们知道是谁,就是贾雨村老爷。我们今儿进去,看见带着锁子,说要解到三法司衙门里审问去呢。我们见他常在咱们家里来往,恐有什么事,便跟了去打听。”贾芸道:“到底老大用心,原该打听打听。你且坐下喝一杯再说。”两人让了一回,便坐下,喝着酒道:“这位雨村老爷人也能干,也会钻营,官也不小了,只是贪财,被人家参了个婪索属员的几款。如今的万岁爷是最圣明最仁慈的,独听了一个‘贪’字,或因糟蹋了百姓,或因恃势欺良,是极生气的,所以旨意便叫拿问。若是问出来了,只怕搁不住。若是没有的事,那参的人也不便。如今真真是好时候,只要有造化做个官儿就好。”众人道:“你的哥哥就是有造化的,现做知县还不好么。”赖家的说道:“我哥哥虽是做了知县,他的行为只怕也保不住怎么样呢。”众人道:“手也长么?”赖家的点点头儿,便举起杯来喝酒。众人又道:“里头还听见什么新闻?”两人道:“别的事没有,只听见海疆的贼寇拿住了好些,也解到法司衙门里审问。还审出好些贼寇,也有藏在城里的,打听消息,抽空儿就劫抢人家,如今知道朝里那些老爷们都是能文能武,出力报效,所到之处早就消灭了。”众人道:“你听见有在城里的,不知审出咱们家失盗了一案来没有?”两人道:“倒没有听见。恍惚有人说是有个内地里的人,城里犯了事,抢了一个女人下海去了。那女人不依,被这贼寇杀了。那贼寇正要跳出关去,被官兵拿住了,就在拿获的地方正了法了。”众人道:“咱们栊翠庵的什么妙玉不是叫人抢去,不要就是他罢?”贾环道:“必是他!”众人道:“你怎么知道?”贾环道:“妙玉这个东西是最讨人嫌的。他一日家捏酸,见了宝玉就眉开眼笑了。我若见了他,他从不拿正眼瞧我一瞧。真要是他,我才趁愿呢!”众人道:“抢的人也不少,那里就是他。”贾芸道:“有点信儿。前日有个人说,他庵里的道婆做梦,说看见是妙玉叫人杀了。”众人笑道:“梦话算不得。”邢大舅道:“管他梦不梦,咱们快吃饭罢。今夜做个大输赢。”众人愿意,便吃毕了饭,大赌起来。
\end{parag}


\begin{parag}
    赌到三更多天,只听见里头乱嚷,说是四姑娘合珍大奶奶拌嘴,把头发都绞掉了,赶到邢夫人王夫人那里去磕了头,说是要求容他做尼姑呢,送他一个地方,若不容他他就死在眼前。那邢王两位太太没主意,叫请蔷大爷芸二爷进去。贾芸听了,便知是那回看家的时候起的念头,想来是劝不过来的了,便合贾蔷商议道:“太太叫我们进去,我们是做不得主的。况且也不好做主,只好劝去。若劝不住,只好由他们罢。咱们商量了写封书给琏二叔,便卸了我们的干系了。”两人商量定了主意,进去见了邢王两位太太,便假意的劝了一回。无奈惜春立意必要出家,就不放他出去,只求一两间净屋子给他诵经拜佛。尤氏见他两个不肯作主,又怕惜春寻死,自己便硬做主张,说是:“这个不是索性我耽了罢。说我做嫂子的容不下小姑子,逼他出了家了就完了。若说到外头去呢,断断使不得。若在家里呢,太太们都在这里,算我的主意罢。叫蔷哥儿写封书子给你珍大爷琏二叔就是了。”贾蔷等答应了。不知邢王二夫人依与不依,下回分解。
\end{parag}