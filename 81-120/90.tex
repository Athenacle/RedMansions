\chap{九十}{失绵衣贫女耐嗷嘈 送果品小郎惊叵测}


\begin{parag}
    却说黛玉自立意自戕之后,渐渐不支,一日竟至绝粒。从前十几天内,贾母等轮流看望,他有时还说几句话,这两日索性不大言语。心里虽有时昏晕,却也有时清楚。贾母等见他这病不似无因而起,也将紫鹃雪雁盘问过两次,两个那里敢说。便是紫鹃欲向侍书打听消息,又怕越闹越真,黛玉更死得快了,所以见了侍书,毫不提起。那雪雁是他传话弄出这样缘故来,此时恨不得长出百十个嘴来说“我没说”,自然更不敢提起。到了这一天黛玉绝粒之日,紫鹃料无指望了,守着哭了会子,因出来偷向雪雁道:“你进屋里来好好儿的守着他。我去回老太太,太太和二奶奶去,今日这个光景大非往常可比了。”雪雁答应,紫鹃自去。
\end{parag}


\begin{parag}
    这里雪雁正在屋里伴着黛玉,见他昏昏沉沉,小孩子家那里见过这个样儿,只打谅如此便是死的光景了,心中又痛又怕,恨不得紫鹃一时回来才好。正怕着,只听窗外脚步走响,雪雁知是紫鹃回来,才放下心了,连忙站起来掀着里间帘子等他。只见外面帘子响处,进来了一个人,却是侍书。那侍书是探春打发来看黛玉的,见雪雁在那里掀着帘子,便问道:“姑娘怎么样?”雪雁点点头儿叫他进来。侍书跟进来,见紫鹃不在屋里,瞧了瞧黛玉,只剩得残喘微延,唬的惊疑不止,因问:“紫鹃姐姐呢?”雪雁道:“告诉上屋里去了。”那雪雁此时只打谅黛玉心中一无所知了,又见紫鹃不在面前,因悄悄的拉了侍书的手问道:“你前日告诉我说的什么王大爷给这里宝二爷说了亲,是真话么?”侍书道:“怎么不真。”雪雁道:“多早晚放定的?”侍书道:“那里就放定了呢。那一天我告诉你时,是我听见小红说的。后来我到二奶奶那边去,二奶奶正和平姐姐说呢,说那都是门客们借着这个事讨老爷的喜欢,往后好拉拢的意思。别说大太太说不好,就是大太太愿意,说那姑娘好,那大太太眼里看的出什么人来!再者老太太心里早有了人了,就在咱们园子里的。大太太那里摸的着底呢。老太太不过因老爷的话,不得不问问罢咧。又听见二奶奶说,宝玉的事,老太太总是要亲上作亲的,凭谁来说亲,横竖不中用。”雪雁听到这里,也忘了神了,因说道:“这是怎么说,白白的送了我们这一位的命了!”侍书道:“这是从那里说起?”雪雁道:“你还不知道呢。前日都是我和紫鹃姐姐说来着,这一位听见了,就弄到这步田地了。”侍书道:“你悄悄儿的说罢,看仔细他听见了。”雪雁道:“人事都不省了,瞧瞧罢,左不过在这一两天了。”正说着,只见紫鹃掀帘进来说:“这还了得!你们有什么话,还不出去说,还在这里说。索性逼死他就完了。”侍书道:“我不信有这样奇事。”紫鹃道:“好姐姐,不是我说,你又该恼了。你懂得什么呢!懂得也不传这些舌了。”
\end{parag}


\begin{parag}
    这里三个人正说着,只听黛玉忽然又嗽了一声。紫鹃连忙跑到炕沿前站着,侍书雪雁也都不言语了。紫鹃弯着腰,在黛玉身后轻轻问道:“姑娘喝口水罢。”黛玉微微答应了一声。雪雁连忙倒了半钟滚白水,紫鹃接了托着,侍书也走近前来。紫鹃和他摇头儿,不叫他说话,侍书只得咽住了。站了一回,黛玉又嗽了一声。紫鹃趁势问道:“姑娘喝水呀?”黛玉又微微应了一声,那头似有欲抬之意,那里抬得起。紫鹃爬上炕去,爬在黛玉旁边,端着水试了冷热,送到唇边,扶了黛玉的头,就到碗边,喝了一口。紫鹃才要拿时,黛玉意思还要喝一口,紫鹃便托着那碗不动。黛玉又喝了一口,摇摇头儿不喝了,喘了一口气,仍旧躺下。半日,微微睁眼说道:“刚才说话不是侍书么?”紫鹃答应道:“是。”侍书尚未出去,因连忙过来问候。黛玉睁眼看了,点点头儿,又歇了一歇,说道:“回去问你姑娘好罢。”侍书见这番光景,只当黛玉嫌烦,只得悄悄的退出去了。原来那黛玉虽则病势沉重,心里却还明白。起先侍书雪雁说话时,他也模糊听见了一半句,却只作不知,也因实无精神答理。及听了雪雁侍书的话,才明白过前头的事情原是议而未成的,又兼侍书说是凤姐说的,老太太的主意亲上作亲,又是园中住着的,非自己而谁?因此一想,阴极阳生,心神顿觉清爽许多,所以才喝了两口水,又要想问侍书的话。恰好贾母,王夫人,李纨,凤姐听见紫鹃之言,都赶着来看。黛玉心中疑团已破,自然不似先前寻死之意了。虽身体软弱,精神短少,却也勉强答应一两句了。凤姐因叫过紫鹃问道:“姑娘也不至这样,这是怎么说,你这样唬人。”紫鹃道:“实在头里看着不好,才敢去告诉的,回来见姑娘竟好了许多,也就怪了。”贾母笑道:“你也别怪他,他懂得什么。看见不好就言语,这倒是他明白的地方,小孩子家,不嘴懒脚懒就好。”说了一回,贾母等料着无妨,也就去了。正是:
\end{parag}


\begin{poem}
    \begin{pl}
        心病终须心药治,解铃还是系铃人。
    \end{pl}
\end{poem}


\begin{parag}
    不言黛玉病渐减退,且说雪雁紫鹃背地里都念佛。雪雁向紫鹃说道:“亏他好了,只是病的奇怪,好的也奇怪。”紫鹃道:“病的倒不怪,就只好的奇怪。想来宝玉和姑娘必是姻缘,人家说的‘好事多磨’,又说道‘是姻缘棒打不回’。这样看起来,人心天意,他们两个竟是天配的了。再者,你想那一年我说了林姑娘要回南去,把宝玉没急死了,闹得家翻宅乱。如今一句话,又把这一个弄得死去活来。可不说的三生石上百年前结下的么。”说着,两个悄悄的抿着嘴笑了一回。雪雁又道:“幸亏好了。咱们明儿再别说了,就是宝玉娶了别的人家儿的姑娘,我亲见他在那里结亲,我也再不露一句话了。”紫鹃笑道:“这就是了。”不但紫鹃和雪雁在私下里讲究,就是众人也都知道黛玉的病也病得奇怪,好也好得奇怪,三三两两,唧唧哝哝议论著。不多几时,连凤姐儿也知道了,邢王二夫人也有些疑惑,倒是贾母略猜着了八九。
\end{parag}


\begin{parag}
    那时正值邢王二夫人凤姐等在贾母房中说闲话,说起黛玉的病来。贾母道:“我正要告诉你们,宝玉和林丫头是从小儿在一处的,我只说小孩子们,怕什么?以后时常听得林丫头忽然病,忽然好,都为有了些知觉了。所以我想他们若尽着搁在一块儿,毕竟不成体统。你们怎么说?”王夫人听了,便呆了一呆,只得答应道:“林姑娘是个有心计儿的。至于宝玉,呆头呆恼,不避嫌疑是有的,看起外面,却还都是个小孩儿形象。此时若忽然或把那一个分出园外,不是倒露了什么痕迹了么。古来说的:‘男大须婚,女大须嫁。’老太太想,倒是赶着把他们的事办办也罢了。”贾母皱了一皱眉,说道:“林丫头的乖僻,虽也是他的好处,我的心里不把林丫头配他,也是为这点子。况且林丫头这样虚弱,恐不是有寿的。只有宝丫头最妥。”王夫人道:“不但老太太这么想,我们也是这样。但林姑娘也得给他说了人家儿才好,不然女孩儿家长大了,那个没有心事?倘或真与宝玉有些私心,若知道宝玉定下宝丫头,那倒不成事了。”贾母道:“自然先给宝玉娶了亲,然后给林丫头说人家,再没有先是外人后是自己的。况且林丫头年纪到底比宝玉小两岁。依你们这样说,倒是宝玉定亲的话不许叫他知道倒罢了。”凤姐便吩咐众丫头们道:“你们听见了,宝二爷定亲的话,不许混吵嚷。若有多嘴的,堤防着他的皮。”贾母又向凤姐道:“凤哥儿,你如今自从身上不大好,也不大管园里的事了。我告诉你,须得经点儿心。不但这个,就象前年那些人喝酒耍钱,都不是事。你还精细些,少不得多分点心儿,严紧严紧他们才好。况且我看他们也就只还服你。”凤姐答应了。娘儿们又说了一回话,方各自散了。从此凤姐常到园中照料。一日,刚走进大观园,到了紫菱洲畔,只听见一个老婆子在那里嚷。凤姐走到跟前,那婆子才瞧见了,早垂手侍立,口里请了安。凤姐道:“你在这里闹什么?”婆子道:“蒙奶奶们派我在这里看守花果,我也没有差错,不料邢姑娘的丫头说我们是贼。”凤姐道:“为什么呢?”婆子道:“昨儿我们家的黑儿跟着我到这里顽了一回,他不知道,又往邢姑娘那边去瞧了一瞧,我就叫他回去了。今儿早起听见他们丫头说丢了东西了。我问他丢了什么,他就问起我来了。”凤姐道:“问了你一声,也犯不着生气呀。”婆子道:“这里园子到底是奶奶家里的,并不是他们家里的。我们都是奶奶派的,贼名儿怎么敢认呢。”凤姐照脸啐了一口,厉声道:“你少在我跟前唠唠叨叨的!你在这里照看,姑娘丢了东西,你们就该问哪,怎么说出这些没道理的话来。把老林叫了来,撵出他去。”丫头们答应了。只见邢岫烟赶忙出来,迎着凤姐陪笑道:“这使不得,没有的事,事情早过去了。”凤姐道:“姑娘,不是这个话。倒不讲事情,这名分上太岂有此理了。”岫烟见婆子跪在地下告饶,便忙请凤姐到里边去坐。凤姐道:“他们这种人我知道,他除了我,其余都没上没下的了。”岫烟再三替他讨饶,只说自己的丫头不好。凤姐道:“我看着邢姑娘的分上,饶你这一次。”婆子才起来,磕了头,又给岫烟磕了头,才出去了。
\end{parag}


\begin{parag}
    这里二人让了坐。凤姐笑问道:“你丢了什么东西了?”岫烟笑道:“没有什么要紧的,是一件红小袄儿,已经旧了的。我原叫他们找,找不着就罢了。这小丫头不懂事,问了那婆子一声,那婆子自然不依了。这都是小丫头糊涂不懂事,我也骂了几句,已经过去了,不必再提了。”凤姐把岫烟内外一瞧,看见虽有些皮绵衣服,已是半新不旧的,未必能暖和。他的被窝多半是薄的。至于房中桌上摆设的东西,就是老太太拿来的,却一些不动,收拾的干干净净。凤姐心上便很爱敬他,说道:“一件衣服原不要紧,这时候冷,又是贴身的,怎么就不问一声儿呢。这撒野的奴才了不得了!”说了一回,凤姐出来,各处去坐了一坐,就回去了。到了自己房中,叫平儿取了一件大红洋绉的小袄儿,一件松花色绫子一斗珠儿的小皮袄,一条宝蓝盘锦镶花绵裙,一件佛青银鼠褂子,包好叫人送去。
\end{parag}


\begin{parag}
    那时岫烟被那老婆子聒噪了一场,虽有凤姐来压住,心上终是不安。想起“许多姊妹们在这里,没有一个下人敢得罪他的,独自我这里,他们言三语四,刚刚凤姐来碰见。”想来想去,终是没意思,又说不出来。正在吞声饮泣,看见凤姐那边的丰儿送衣服过来。岫烟一看,决不肯受。丰儿道:“奶奶吩咐我说,姑娘要嫌是旧衣裳,将来送新的来。”岫烟笑谢道:“承奶奶的好意,只是因我丢了衣服,他就拿来,我断不敢受。你拿回去千万谢你们奶奶,承你奶奶的情,我算领了。”倒拿个荷包给了丰儿。那丰儿只得拿了去了。不多时,又见平儿同着丰儿过来,岫烟忙迎着问了好,让了坐。平儿笑说道:“我们奶奶说,姑娘特外道的了不得。”岫烟道:“不是外道,实在不过意。”平儿道:“奶奶说,姑娘要不收这衣裳,不是嫌太旧,就是瞧不起我们奶奶。刚才说了,我要拿回去,奶奶不依我呢。”岫烟红着脸笑谢道:“这样说了,叫我不敢不收。”又让了一回茶。
\end{parag}


\begin{parag}
    平儿同丰儿回去,将到凤姐那边,碰见薛家差来的一个老婆子,接着问好。平儿便问道:“你那里来的?”婆子道:“那边太太姑娘叫我来请各位太太,奶奶,姑娘们的安。我才刚在奶奶前问起姑娘来,说姑娘到园中去了。可是从邢姑娘那里来么?”平儿道:“你怎么知道?”婆子道:“方才听见说。真真的二奶奶和姑娘们的行事叫人感念。”平儿笑了一笑说:“你回来坐着罢。”婆子道:“我还有事,改日再过来瞧姑娘罢。”说着走了。平儿回来,回复了凤姐。不在话下。
\end{parag}


\begin{parag}
    且说薛姨妈家中被金桂搅得翻江倒海,看见婆子回来,述起岫烟的事,宝钗母女二人不免滴下泪来。宝钗道:“都为哥哥不在家,所以叫邢姑娘多吃几天苦。如今还亏凤姐姐不错。咱们底下也得留心,到底是咱们家里人。”说着,只见薛蝌进来说道:“大哥哥这几年在外头相与的都是些什么人,连一个正经的也没有,来一起子,都是些狐群狗党。我看他们那里是不放心,不过将来探探消息儿罢咧。这两天都被我干出去了。以后吩咐了门上,不许传进这种人来。”薛姨妈道:“又是蒋玉菡那些人哪?”薛蝌道:“蒋玉菡却倒没来,倒是别人。”薛姨妈听了薛蝌的话,不觉又伤心起来,说道:“我虽有儿,如今就象没有的了,就是上司准了,也是个废人。你虽是我侄儿,我看你还比你哥哥明白些,我这后辈子全靠你了。你自己从今更要学好。再者,你聘下的媳妇儿,家道不比往时了。人家的女孩儿出门子不是容易,再没别的想头,只盼着女婿能干,他就有日子过了。若邢丫头也象这个东西,”说着把手往里头一指,道:“我也不说了。邢丫头实在是个有廉耻有心计儿的,又守得贫,耐得富。只是等咱们的事情过去了,早些把你们的正经事完结了,也了我一宗心事。”薛蝌道:“琴妹妹还没有出门子,这倒是太太烦心的一件事。至于这个,可算什么呢。”大家又说了一回闲话。
\end{parag}


\begin{parag}
    薛蝌回到自己房中,吃了晚饭,想起邢岫烟住在贾府园中,终是寄人篱下,况且又穷,日用起居,不想可知。况兼当初一路同来,模样儿性格儿都知道的。可知天意不均:如夏金桂这种人,偏教他有钱,娇养得这般泼辣,邢岫烟这种人,偏教他这样受苦。阎王判命的时候,不知如何判法的。想到闷来也想吟诗一首,写出来出出胸中的闷气。又苦自己没有工夫,只得混写道:
\end{parag}


\begin{poem}
    \begin{pl}
        蛟龙失水似枯鱼,两地情怀感索居。
    \end{pl}


    \begin{pl}
        同在泥涂多受苦,不知何日向清虚。
    \end{pl}
\end{poem}


\begin{parag}
    写毕看了一回,意欲拿来粘在壁上,又不好意思。自己沉吟道:“不要被人看见笑话。”又念了一遍,道:“管他呢,左右粘上自己看着解闷儿罢。”又看了一回,到底不好,拿来夹在书里。又想自己年纪可也不小了,家中又碰见这样飞灾横祸,不知何日了局,致使幽闺弱质,弄得这般凄凉寂寞。正在那里想时,只见宝蟾推门进来,拿着一个盒子,笑嘻嘻放在桌上。薛蝌站起来让坐。宝蟾笑着向薛蝌道:“这是四碟果子,一小壶儿酒,大奶奶叫给二爷送来的。”薛蝌陪笑道:“大奶奶费心。但是叫小丫头们送来就完了,怎么又劳动姐姐呢。”宝蟾道:“好说。自家人,二爷何必说这些套话。再者我们大爷这件事,实在叫二爷操心,大奶奶久已要亲自弄点什么儿谢二爷,又怕别人多心。二爷是知道的,咱们家里都是言合意不合,送点子东西没要紧,倒没的惹人七嘴八舌的讲究。所以今日些微的弄了一两样果子,一壶酒,叫我亲自悄悄儿的送来。”说着,又笑瞅了薛蝌一眼,道:“明儿二爷再别说这些话,叫人听着怪不好意思的。我们不过也是底下的人,伏侍的着大爷就伏侍的着二爷,这有何妨呢。”薛蝌一则秉性忠厚,二则到底年轻,只是向来不见金桂和宝蟾如此相待,心中想到刚才宝蟾说为薛蟠之事也是情理,因说道:“果子留下罢,这个酒儿,姐姐只管拿回去。我向来的酒上实在很有限,挤住了偶然喝一钟,平日无事是不能喝的。难道大奶奶和姐姐还不知道么。”宝蟾道:“别的我作得主,独这一件事,我可不敢应。大奶奶的脾气儿,二爷是知道的,我拿回去,不说二爷不喝,倒要说我不尽心了。”薛蝌没法,只得留下。宝蟾方才要走,又到门口往外看看,回过头来向着薛蝌一笑,又用手指着里面说道:“他还只怕要来亲自给你道乏呢。”薛蝌不知何意,反倒讪讪的起来,因说道:“姐姐替我谢大奶奶罢。天气寒,看凉着。再者,自己叔嫂,也不必拘这些个礼。”宝蟾也不答言,笑着走了。
\end{parag}


\begin{parag}
    薛蝌始而以为金桂为薛蟠之事,或者真是不过意,备此酒果给自己道乏,也是有的。及见了宝蟾这种鬼鬼祟祟不尴不尬的光景,也觉了几分。却自己回心一想:“他到底是嫂子的名分,那里就有别的讲究了呢。或者宝蟾不老成,自己不好意思怎么样,却指着金桂的名儿,也未可知。然而到底是哥哥的屋里人,也不好。”忽又一转念:“那金桂素性为人毫无闺阁理法,况且有时高兴,打扮得妖调非常,自以为美,又焉知不是怀着坏心呢?不然,就是他和琴妹妹也有了什么不对的地方儿,所以设下这个毒法儿,要把我拉在浑水里,弄一个不清不白的名儿,也未可知。”想到这里,索性倒怕起来。正在不得主意的时候,忽听窗外扑哧的笑了一声,把薛蝌倒唬了一跳。未知是谁,下回分解。
\end{parag}