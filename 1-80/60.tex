\chap{六十}{茉莉粉替去薔薇硝 玫瑰露引來茯苓霜}

\begin{parag}
    話說襲人因問平兒,何事這等忙亂。平兒笑道:“都是世人想不到的,說來也好笑,等幾日告訴你,如今沒頭緒呢,且也不得閒兒。”一語未了,只見李紈的丫鬟來了,說:“平姐姐可在這裏,奶奶等你,你怎麼不去了?”平兒忙轉身出來,口內笑說:“來了,來了。”襲人等笑道:“他奶奶病了,他又成了香餑餑了,都搶不到手。”平兒去了不提。
\end{parag}


\begin{parag}
    寶玉便叫春燕:“你跟了你媽去,到寶姑娘房裏給鶯兒幾句好話聽聽,也不可白得罪了他。”春燕答應了,和他媽出去。寶玉又隔窗說道:“不可當著寶姑娘說,仔細反叫鶯兒受教導。”
\end{parag}


\begin{parag}
    孃兒兩個應了出來,一壁走著,一面說閒話兒。春燕因向他娘道:“我素日勸你老人家再不信,何苦鬧出沒趣來才罷。”他娘笑道:“小蹄子,你走罷,俗語道:‘不經一事,不長一智。’我如今知道了。你又該來支問著我。”春燕笑道:“媽,你若安分守己,在這屋裏長久了,自有許多的好處。我且告訴你句話:寶玉常說,將來這屋裏的人,無論家裏外頭的,一應我們這些人,他都要回太太全放出去,與本人父母自便呢。你只說這一件可好不好?”他娘聽說,喜的忙問:“這話果真?”春燕道:“誰可扯這謊做什麼?”婆子聽了,便唸佛不絕。
\end{parag}


\begin{parag}
    當下來至蘅蕪苑中,正值寶釵、黛玉、薛姨媽等喫飯。鶯兒自去泡茶,春燕便和他媽一徑到鶯兒前,陪笑說“方纔言語冒撞了,姑娘莫嗔莫怪,特來陪罪”等語。鶯兒忙笑讓坐,又倒茶。他孃兒兩個說有事,便作辭回來。忽見蕊官趕出叫:“媽媽姐姐,略站一站。”一面走上來,遞了一個紙包與他們,說是薔薇硝,帶與芳官去擦臉。春燕笑道:“你們也太小氣了,還怕那裏沒這個與他,巴巴的你又弄一包給他去。”蕊官道:“他是他的,我送的是我的。好姐姐,千萬帶回去罷。” 春燕只得接了。孃兒兩個回來,正值賈環賈琮二人來問候寶玉,也才進去。春燕便向他娘說:“只我進去罷,你老不用去。”他娘聽了,自此便百依百隨的,不敢倔強了。
\end{parag}


\begin{parag}
    春燕進來,寶玉知道回覆,便先點頭。春燕知意,便不再說一語,略站了一站,便轉身出來,使眼色與芳官。芳官出來,春燕方悄悄的說與他蕊官之事,並與了他硝。寶玉並無與琮環可談之語,因笑問芳官手裏是什麼。芳官便忙遞與寶玉瞧,又說是擦春癬的薔薇硝。寶玉笑道:“虧他想得到。”賈環聽了,便伸著頭瞧了一瞧,又聞得一股清香,便彎著腰向靴桶內掏出一張紙來託著,笑說:“好哥哥,給我一半兒。”寶玉只得要與他。芳官心中因是蕊官之贈,不肯與別人,連忙攔住,笑說道:“別動這個,我另拿些來。”寶玉會意,忙笑包上,說道:“快取來。”
\end{parag}


\begin{parag}
    芳官接了這個,自去收好,便從奩中去尋自己常使的。啓奩看時,盒內已空,心中疑惑,早間還剩了些,如何沒了?因問人時,都說不知。麝月便說:“這會子且忙著問這個,不過是這屋裏人一時短了。你不管拿些什麼給他們,他們那裏看得出來?快打發他們去了,咱們好喫飯。”芳官聽了,便將些茉莉粉包了一包拿來。賈環見了就伸手來接。芳官便忙向炕上一擲。賈環只得向炕上拾了,揣在懷內,方作辭而去。
\end{parag}


\begin{parag}
    原來賈政不在家,且王夫人等又不在家,賈環連日也便裝病逃學。如今得了硝,興興頭頭來找彩雲。正值彩雲和趙姨娘閒談,賈環嘻嘻向彩雲道:“我也得了一包好的,送你擦臉。你常說,薔薇硝擦癬,比外頭的銀硝強。你且看看,可是這個?”彩雲打開一看,嗤的一聲笑了,說道:“你是和誰要來的?”賈環便將方纔之事說了。彩雲笑道:“這是他們哄你這鄉老呢。這不是硝,這是茉莉粉。”賈環看了一看,果然比先的帶些紅色,聞聞也是噴香,因笑道:“這也是好的,硝粉一樣,留著擦罷,自是比外頭買的高便好。”彩雲只得收了。趙姨娘便說:“有好的給你!誰叫你要去了,怎怨他們耍你!依我,拿了去照臉摔給他去,趁著這回子撞屍的撞屍去了,挺牀的便挺牀,吵一齣子,大家別心淨,也算是報仇。莫不是兩個月之後,還找出這個碴兒來問你不成?便問你,你也有話說。寶玉是哥哥,不敢衝撞他罷了。難道他屋裏的貓兒狗兒,也不敢去問問不成!”賈環聽說,便低了頭。彩雲忙說:“這又何苦生事,不管怎樣,忍耐些罷了。”趙姨娘道:“你快休管,橫豎與你無干。乘著抓住了理,罵給那些浪淫婦們一頓也是好的。”又指賈環道:“呸!你這下流沒剛性的,也只好受這些毛崽子的氣!平白我說你一句兒,或無心中錯拿了一件東西給你,你倒會扭頭暴筋瞪著眼蹾摔娘。這會子被那起屄崽子耍弄也罷了。你明兒還想這些家裏人怕你呢。你沒有屄本事,我也替你羞。”賈環聽了,不免又愧又急,又不敢去,只摔手說道:“你這麼會說,你又不敢去,指使了我去鬧。倘或往學裏告去捱了打,你敢自不疼呢?遭遭兒調唆了我鬧去,鬧出了事來,我捱了打罵,你一般也低了頭。這會子又調唆我和毛丫頭們去鬧。你不怕三姐姐,你敢去,我就伏你。”只這一句話,便戳了他孃的肺,便喊說:“我腸子爬出來的,我再怕不成!這屋裏越發有的說了。”一面說,一面拿了那包子,便飛也似往園中去。彩雲死勸不住,只得躲入別房。賈環便也躲出儀門,自去頑耍。
\end{parag}


\begin{parag}
    趙姨娘直進園子,正是一頭火,頂頭正遇見藕官的乾孃夏婆子走來。見趙姨娘氣恨恨的走來,因問:“姨奶奶那去?”趙姨娘又說:“你瞧瞧,這屋裏連三日兩日進來的唱戲的小粉頭們,都三般兩樣掂人分兩放小菜碟兒了。若是別一個,我還不惱,若叫這些小娼婦捉弄了,還成個什麼!”夏婆子聽了,正中己懷,忙問因何。趙姨娘悉將芳官以粉作硝輕侮賈環之事說了。夏婆子道:“我的奶奶,你今日才知道,這算什麼事。連昨日這個地方他們私自燒紙錢,寶玉還攔到頭裏。人家還沒拿進個什麼兒來,就說使不得,不乾不淨的忌諱。這燒紙倒不忌諱?你老想一想,這屋裏除了太太,誰還大似你?你老自己撐不起來;但凡撐起來的,誰還不怕你老人家?如今我想,乘著這幾個小粉頭兒恰不是正頭貨,得罪了他們也有限的,快把這兩件事抓著理扎個筏子,我在旁作證據,你老把威風抖一抖,以後也好爭別的禮。便是奶奶姑娘們,也不好爲那起小粉頭子說你老的。”趙姨娘聽了這話,益發有理,便說:“燒紙的事不知道,你卻細細的告訴我。”夏婆子便將前事一一的說了,又說:“你只管說去。倘或鬧起,還有我們幫著你呢。”趙姨娘聽了越發得了意,仗著膽子便一徑到了怡紅院中。
\end{parag}


\begin{parag}
    可巧寶玉聽見黛玉在那裏,便往那裏去了。芳官正與襲人等喫飯,見趙姨娘來了,便都起身笑讓:“姨奶奶喫飯,有什麼事這麼忙?”趙姨娘也不答話,走上來便將粉照著芳官臉上撒來,指著芳官罵道:“小淫婦!你是我銀子錢買來學戲的,不過娼婦粉頭之流!我家裏下三等奴才也比你高貴些的,你都會看人下菜碟兒。寶玉要給東西,你攔在頭裏,莫不是要了你的了?拿這個哄他,你只當他不認得呢!好不好,他們是手足,都是一樣的主子,那裏有你小看他的!”芳官那裏禁得住這話,一行哭,一行說:“沒了硝我才把這個給他的。若說沒了,又恐他不信,難道這不是好的?我便學戲,也沒往外頭去唱。我一個女孩兒家,知道什麼是粉頭面頭的!姨奶奶犯不著來罵我,我又不是姨奶奶家買的。‘梅香拜把子--都是奴幾’呢!”襲人忙拉他說:“休胡說!”趙姨娘氣的便上來打了兩個耳刮子。襲人等忙上來拉勸,說:“姨奶奶別和他小孩子一般見識,等我們說他。”芳官捱了兩下打,那裏肯依,便拾頭打滾,潑哭潑鬧起來。口內便說:“你打得起我麼?你照照那模樣兒再動手!我叫你打了去,我還活著!”便撞在懷裏叫他打。衆人一面勸,一面拉他。晴雯悄拉襲人說:“別管他們,讓他們鬧去,看怎麼開交!如今亂爲王了,什麼你也來打,我也來打,都這樣起來還了得呢!”
\end{parag}


\begin{parag}
    外面跟著趙姨娘來的一干的人聽見如此,心中各各稱願,都念佛說:“也有今日!”又有那一干懷怨的老婆子見打了芳官,也都稱願。
\end{parag}


\begin{parag}
    當下藕官蕊官等正在一處作耍,湘雲的大花面葵官,寶琴的豆官,兩個聞了此信,慌忙找著他兩個說:“芳官被人欺侮,咱們也沒趣,須得大家破著大鬧一場,方爭過氣來。”四人終是小孩子心性,只顧他們情分上義憤,便不顧別的,一齊跑入怡紅院中。豆官先便一頭,幾乎不曾將趙姨娘撞了一跌。那三個也便擁上來,放聲大哭,手撕頭撞,把個趙姨娘裹住。晴雯等一面笑,一面假意去拉。急的襲人拉起這個,又跑了那個,口內只說:“你們要死!有委曲只好說,這沒理的事如何使得!”趙姨娘反沒了主意,只好亂罵。蕊官藕官兩個一邊一個,抱住左右手;葵官豆官前後頭頂住。四人只說:“你只打死我們四個就罷!”芳官直挺挺躺在地下,哭得死過去。
\end{parag}


\begin{parag}
    正沒開交,誰知晴雯早遣春燕回了探春。當下尤氏、李紈、探春三人帶著平兒與衆媳婦走來,忙忙將四個喝住。問起原故,趙姨娘便氣的瞪著眼粗了筋,一五一十說個不清。尤李兩個不答言,只喝禁他四人。探春便嘆氣說:“這是什麼大事,姨娘也太肯動氣了!我正有一句話要請姨娘商議,怪道丫頭說不知在那裏,原來在這裏生氣呢,快同我來。”尤氏李氏都笑說:“姨娘請到廳上來,咱們商量。”
\end{parag}


\begin{parag}
    趙姨娘無法,只得同他三人出來,口內猶說長說短。探春便說:“那些小丫頭子們原是些頑意兒,喜歡呢,和他說說笑笑;不喜歡便可以不理他。便他不好了,也如同貓兒狗兒抓咬了一下子,可恕就恕,不恕時也該叫了管家媳婦們去說給他去責罰,何苦自己不尊重,大吆小喝失了體統。你瞧周姨娘,怎不見人欺他,他也不尋人去。我勸姨娘且回房去煞煞性兒,別聽那些混帳人的調唆,沒的惹人笑話,自己呆白給人作粗活。心裏有二十分的氣,也忍耐這幾天,等太太回來自然料理。” 一席話說得趙姨娘閉口無言,只得回房去了。
\end{parag}


\begin{parag}
    這裏探春氣的和尤氏李紈說:“這麼大年紀,行出來的事總不叫人敬伏。這是什麼意思,值得吵一吵,並不留體統,耳朵又軟,心裏又沒有計算。這又是那起沒臉面的奴才們的調停,作弄出個呆人替他們出氣。”越想越氣,因命人查是誰調唆的。媳婦們只得答應著,出來相視而笑,都說是“大海里那裏尋針去?”只得將趙姨娘的人並園中喚來盤詰,都說不知道。衆人沒法,只得回探春:“一時難查,慢慢訪查,凡有口舌不妥的,一總來回了責罰。”
\end{parag}


\begin{parag}
    探春氣漸漸平服方罷。可巧艾官便悄悄的回探春說:“都是夏媽和我們素日不對,每每的造言生事。前兒賴藕官燒紙,幸虧是寶玉叫他燒的,寶玉自己應了,他纔沒話說。今兒我與姑娘送手帕去,看見他和姨奶奶在一處說了半天,嘁嘁喳喳的,見了我才走開了。”探春聽了,雖知情弊,亦料定他們皆是一黨,本皆淘氣異常,便只答應,也不肯據此爲實。
\end{parag}


\begin{parag}
    誰知夏婆子的外孫女兒蟬姐兒便是探春處當役的,時常與房中丫鬟們買東西呼喚人,衆女孩兒都和他好。這日飯後,探春正上廳理事,翠墨在家看屋子,因命蟬姐出去叫小幺兒買糕去。蟬兒便說:“我才掃了個大園子,腰腿生疼的,你叫個別的人去罷。”翠墨笑說:“我又叫誰去?你趁早兒去,我告訴你一句好話,你到後門順路告訴你老孃防著些兒。”說著,便將艾官告他老孃話告訴了他。蟬姐聽了,忙接了錢道:“這個小蹄子也要捉弄人,等我告訴去。”說著,便起身出來。至後門邊,只見廚房內此刻手閒之時,都坐在階砌上說閒話呢,他老孃亦在內。蟬兒便命一個婆子出去買糕。他且一行罵,一行說,將方纔之話告訴與夏婆子。夏婆子聽了,又氣又怕,便欲去找艾官問他,又欲往探春前去訴冤。蟬兒忙攔住說:“你老人家去怎麼說呢?這話怎得知道的,可又叨登不好了。說給你老防著就是了,那裏忙到這一時兒。”
\end{parag}


\begin{parag}
    正說著,忽見芳官走來,扒著院門,笑向廚房中柳家媳婦說道:“柳嫂子,寶二爺說了:晚飯的素菜要一樣涼涼的酸酸的東西,只別擱上香油弄膩了。”柳家的笑道:“知道。今兒怎遣你來了告訴這麼一句要緊話。你不嫌髒,進來逛逛兒不是?”芳官才進來,忽有一個婆子手裏託了一碟糕來。芳官便戲道:“誰買的熱糕?我先嚐一塊兒。”蟬兒一手接了道:“這是人家買的,你們還希罕這個。”柳家的見了,忙笑道:“芳姑娘,你喜喫這個?我這裏有纔買下給你姐姐喫的,他不曾喫,還收在那裏,乾乾淨淨沒動呢。”說著,便拿了一碟出來,遞與芳官,又說:“你等我進去替你頓口好茶來。”一面進去,現通開火頓茶。芳官便拿著熱糕,問到蟬兒臉上說:“稀罕喫你那糕,這個不是糕不成?我不過說著頑罷了,你給我磕個頭,我也不喫。”說著,便將手內的糕一塊一塊的掰了,擲著打雀兒頑,口內笑說:“柳嫂子,你別心疼,我回來買二斤給你。”小蟬氣的怔怔的,瞅著冷笑道:“雷公老爺也有眼睛,怎不打這作孽的!他還氣我呢。我可拿什麼比你們,又有人進貢,又有人作乾奴才,溜你們好上好兒,幫襯著說句話兒。”衆媳婦都說:“姑娘們,罷呀,天天見了就咕唧。”有幾個伶透的,見了他們對了口,怕又生事,都拿起腳來各自走開了。當下蟬兒也不敢十分說他,一面咕嘟著去了。
\end{parag}


\begin{parag}
    這裏柳家的見人散了,忙出來和芳官說:“前兒那話兒說了不曾?”芳官道:“說了。等一二日再提這事。偏那趙不死的又和我鬧了一場。前兒那玫瑰露姐姐吃了不曾,他到底可好些?”柳家的道:“可不都吃了。他愛的什麼似的,又不好問你再要的。”芳官道:“不值什麼,等我再要些來給他就是了。”
\end{parag}


\begin{parag}
    原來這柳家的有個女兒,今年才十六歲,雖是廚役之女,卻生的人物與平、襲、紫、鴛皆類。因他排行第五,便叫他是五兒。\begin{note}庚雙夾:五月之柳,春色可知。\end{note}因素有弱疾,故沒得差。近因柳家的見寶玉房中的丫鬟差輕人多,且又聞得寶玉將來都要放他們,故如今要送他到那裏應名兒。正無頭路,可巧這柳家的是梨香院的差役,他最小意殷勤,伏侍得芳官一干人比別的乾孃還好。芳官等亦待他們極好,如今便和芳官說了,央芳官去與寶玉說。寶玉雖是依允,只是近日病著,又見事多,尚未說得。
\end{parag}


\begin{parag}
    前言少述,且說當下芳官回至怡紅院中,回覆了寶玉。寶玉正在聽見趙姨娘廝吵,心中自是不悅,說又不是,不說又不是,只得等吵完了,打聽著探春勸了他去後方從蘅蕪苑回來,勸了芳官一陣,方大家安妥。今見他回來,又說還要些玫瑰露與柳五兒喫去,寶玉忙道:“有的,我又不大喫,你都給他去罷。”說著命襲人取了出來,見瓶中亦不多,遂連瓶與了他。
\end{parag}


\begin{parag}
    芳官便自攜了瓶與他去。正值柳家的帶進他女兒來散悶,在那邊犄角子上一帶地方逛了一回,便回到廚房內,正喫茶歇腳兒。芳官拿了一個五寸來高的小玻璃瓶來,迎亮照看,裏面小半瓶胭脂一般的汁子,還道是寶玉喫的西洋葡萄酒。母女兩個忙說:“快拿旋子燙滾水,你且坐下。”芳官笑道:“就剩了這些,連瓶子都給你們罷。”五兒聽了,方知是玫瑰露,忙接了,謝了又謝。芳官又問他“好些?”五兒道:“今兒精神些,進來逛逛。這後邊一帶,也沒什麼意思,不過見些大石頭大樹和房子後牆,正經好景緻也沒看見。”芳官道:“你爲什麼不往前去?”柳家的道:“我沒叫他往前去。姑娘們也不認得他,倘有不對眼的人看見了,又是一番口舌。明兒託你攜帶他有了房頭,怕沒有人帶著逛呢,只怕逛膩了的日子還有呢。”芳官聽了,笑道:“怕什麼,有我呢。”柳家的忙道:“噯喲喲,我的姑娘,我們的頭皮兒薄,比不得你們。”說著,又倒了茶來。芳官那裏喫這茶,只漱了一口就走了。柳家的說道:“我這裏佔著手,五丫頭送送。”
\end{parag}


\begin{parag}
    五兒便送出來,因見無人,又拉著芳官說道:“我的話到底說了沒有?”芳官笑道:“難道哄你不成?我聽見屋裏正經還少兩個人的窩兒,並沒補上。一個是紅玉的,璉二奶奶要去還沒給人來;一個是墜兒的,也還沒補。如今要你一個也不算過分。皆因平兒每每的和襲人說,凡有動人動錢的事,得挨的且挨一日更好。如今三姑娘正要拿人扎筏子呢,連他屋裏的事都駁了兩三件,如今正要尋我們屋裏的事沒尋著,何苦來往網裏碰去。倘或說些話駁了,那時老了,倒難迴轉。不如等冷一冷,老太太、太太心閒了,憑是天大的事先和老的一說,沒有不成的。”五兒道:“雖如此說,我卻性急等不得了。趁如今挑上來了,一則給我媽爭口氣,也不枉養我一場;\begin{note}庚雙夾:爲母。\end{note}二則添了月錢,家裏又從容些;\begin{note}庚雙夾:二爲家中。\end{note}三則我的心開一開,只怕這病就好了。--便是請大夫吃藥,也省了家裏的錢。”芳官道:“我都知道了,你只放心。”二人別過,芳官自去不提。
\end{parag}


\begin{parag}
    單表五兒回來,與他娘深謝芳官之情。他娘因說:“再不承望得了這些東西,雖然是個珍貴物兒,卻是喫多了也最動熱。竟把這個倒些送個人去,也是個大情。” 五兒問:“送誰?”他娘道:“送你舅舅的兒子,昨日熱病,也想這些東西喫。如今我倒半盞與他去。”五兒聽了,半日沒言語,隨他媽倒了半盞子去,將剩的連瓶便放在傢伙廚內。五兒冷笑道:“依我說,竟不給他也罷了。倘或有人盤問起來,倒又是一場事了。”他娘道:“那裏怕起這些來,還了得了。我們辛辛苦苦的,裏頭賺些東西,也是應當的。難道是賊偷的不成?”說著,一徑去了。直至外邊他哥哥家中,他侄子正躺著,一見了這個,他哥嫂侄男無不歡喜。現從井上取了涼水,和吃了一碗,心中一暢,頭目清涼。剩的半盞,用紙覆著,放在桌上。
\end{parag}


\begin{parag}
    可巧又有家中幾個小廝同他侄兒素日相好的,走來問候他的病。內中有一小夥叫喚錢槐者,乃系趙姨娘之內侄。他父母現在庫上管賬,他本身又派跟賈環上學。因他有些錢勢,尚未娶親,素日看上了柳家的五兒標緻,和父母說了,欲娶他爲妻。也曾央中保媒人再四求告。柳家父母卻也情願,爭奈五兒執意不從,雖未明言,卻行止中已帶出,父母未敢應允。近日又想往園內去,越發將此事丟開,只等三五年後放出來,自向外邊擇婿了。錢家見他如此,也就罷了。怎奈錢槐不得五兒,心中又氣又愧,發恨定要弄取成配,方了此願。今日也同人來瞧望柳侄,不期柳家的在內。
\end{parag}


\begin{parag}
    柳家的忽見一羣人來了,內中有錢槐,便推說不得閒,起身便走了。他哥嫂忙說:“姑媽怎麼不喫茶就走?倒難爲姑媽記掛。”柳家的因笑道:“只怕裏面傳飯,再閒了出來瞧侄子罷。”他嫂子因向抽屜內取了一個紙包出來,拿在手內送了柳家的出來,至牆角邊遞與柳家的,又笑道:“這是你哥哥昨兒在門上該班兒,誰知這五日一班,竟偏冷淡,一個外財沒發。只有昨兒有粵東的官兒來拜,送了上頭兩小簍子茯苓霜。餘外給了門上人一簍作門禮,你哥哥分了這些。這地方千年松柏最多,所以單取了這茯苓的精液和了藥,不知怎麼弄出這怪俊的白霜兒來。說第一用人乳和著,每日早起喫一鍾,最補人的;第二用牛奶子;萬不得,滾白水也好。我們想著,正宜外甥女兒喫。原是上半日打發小丫頭子送了家去的,他說鎖著門,連外甥女兒也進去了。本來我要瞧瞧他去,給他帶了去的,又想主子們不在家,各處嚴緊,我又沒什甚麼差使,有要沒緊跑些什麼。況且這兩日風聲,聞得裏頭家反宅亂的,倘或沾帶了倒值多的。姑娘來的正好,親自帶去罷。”
\end{parag}


\begin{parag}
    柳氏道了生受,作別回來。剛到了角門前,只見一個小幺兒笑道:“你老人家那裏去了?裏頭三次兩趟叫人傳呢,我們三四個人都找你老去了,還沒來。你老人家卻從那裏來了?這條路又不是家去的路,我倒疑心起來。”那柳家的笑罵道:“好猴兒崽子,……”要知端的,且聽下回分解。
\end{parag}
