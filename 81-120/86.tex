\chap{八十六}{受私賄老官翻案牘 寄閒情淑女解琴書}



\begin{parag}
    話說薛姨媽聽了薛蝌的來書,因叫進小廝問道:“你聽見你大爺說,到底是怎麼就把人打死了呢?”小廝道:“小的也沒聽真切。那一日大爺告訴二爺說。”說着回頭看了一看,見無人,才說道:“大爺說自從家裏鬧的特利害,大爺也沒心腸了,所以要到南邊置貨去。這日想着約一個人同行,這人在咱們這城南二百多地住。大爺找他去了,遇見在先和大爺好的那個蔣玉菡帶着些小戲子進城。大爺同他在個鋪子裏喫飯喝酒,因爲這當槽兒的盡着拿眼瞟蔣玉菡,大爺就有了氣了。後來蔣玉菡走了。第二天,大爺就請找的那個人喝酒,酒後想起頭一天的事來,叫那當槽兒的換酒,那當槽兒的來遲了,大爺就罵起來了。那個人不依,大爺就拿起酒碗照他打去。誰知那個人也是個潑皮,便把頭伸過來叫大爺打。大爺拿碗就砸他的腦袋一下,他就冒了血了,躺在地下,頭裏還罵,後頭就不言語了。”薛姨媽道:“怎麼也沒人勸勸嗎?”那小廝道:“這個沒聽見大爺說,小的不敢妄言。”薛姨媽道:“你先去歇歇罷。”小廝答應出來。這裏薛姨媽自來見王夫人,託王夫人轉求賈政。賈政問了前後,也只好含糊應了,只說等薛蝌遞了呈子,看他本縣怎麼批了再作道理。
\end{parag}


\begin{parag}
    這裏薛姨媽又在當鋪裏兌了銀子,叫小廝趕着去了。三日後果有回信。薛姨媽接着了,即叫小丫頭告訴寶釵,連忙過來看了。只見書上寫道:
\end{parag}


\begin{parag}
    帶去銀兩做了衙門上下使費。哥哥在監也不大喫苦,請太太放心。獨是這裏的人很刁,屍親見證都不依,連哥哥請的那個朋友也幫着他們。我與李祥兩個俱系生地生人,幸找着一個好先生,許他銀子,才討個主意,說是須得拉扯着同哥哥喝酒的吳良,弄人保出他來,許他銀兩,叫他撕擄。他若不依,便說張三是他打死,明推在異鄉人身上,他喫不住,就好辦了。我依着他,果然吳良出來。現在買囑屍親見證,又做了一張呈子。前日遞的,今日批來,請看呈底便知。
\end{parag}


\begin{parag}
    因又念呈底道:
\end{parag}


\begin{qute2sp}
    具呈人某,呈爲兄遭飛禍代伸冤抑事。竊生胞兄薛蟠,本籍南京,寄寓西京。於某年月日備本往南貿易。去未數日,家奴送信回家,說遭人命。生即奔憲治,知兄誤傷張姓,及至囹圄。據兄泣告,實與張姓素不相認,並無仇隙。偶因換酒角口,生兄將酒潑地,恰值張三低頭拾物,一時失手,酒碗誤碰滷門身死。蒙恩拘訊,兄懼受刑,承認鬥毆致死。仰蒙憲天仁慈,知有冤抑,尚未定案。生兄在禁,具呈訴辯,有幹例禁。生念手足,冒死代呈,伏乞憲慈恩准,提證質訊,開恩莫大。生等舉家仰戴鴻仁,永永無既矣。激切上呈。
\end{qute2sp}


\begin{parag}
    批的是:
\end{parag}


\begin{qute2sp}
    屍場檢驗,證據確鑿。且並未用刑,爾兄自認鬥殺,招供在案。今爾遠來,並非目睹,何得捏詞妄控。理應治罪,姑念爲兄情切,且恕。不準。
\end{qute2sp}


\begin{parag}
    薛姨媽聽到那裏,說道:“這不是救不過來了麼。這怎麼好呢!”寶釵道:“二哥的書還沒看完,後面還有呢。”因又念道:“有要緊的問來使便知。”薛姨媽便問來人,因說道:“縣裏早知我們的家當充足,須得在京裏謀幹得大情,再送一分大禮,還可以複審,從輕定案。太太此時必得快辦,再遲了就怕大爺要受苦了。”
\end{parag}


\begin{parag}
    薛姨媽聽了,叫小廝自去,即刻又到賈府與王夫人說明原故,懇求賈政。賈政只肯託人與知縣說情,不肯提及銀物。薛姨媽恐不中用,求鳳姐與賈璉說了,花上幾千銀子,才把知縣買通。薛蝌那裏也便弄通了。然後知縣掛牌坐堂,傳齊了一干鄰保證見屍親人等,監裏提出薛蟠。刑房書吏俱一一點名。知縣便叫地保對明初供,又叫屍親張王氏並屍叔張二問話。張王氏哭稟道:“小的的男人是張大,南鄉里住,十八年前死了。大兒子二兒子也都死了,光留下這個死的兒子叫張三,今年二十三歲,還沒有娶女人呢。爲小人家裏窮,沒得養活,在李家店裏做當槽兒的。那一天晌午,李家店裏打發人來叫俺,說‘你兒子叫人打死了。”我的青天老爺,小的就唬死了。跑到那裏,看見我兒子頭破血出的躺在地下喘氣兒,問他話也說不出來,不多一會兒就死了。小人就要揪住這個小雜種拼命。”衆衙役吆喝一聲。張王氏便磕頭道:“求青天老爺伸冤,小人就只這一個兒子了。”知縣便叫下去,又叫李家店的人問道:“那張三是你店內傭工的麼?”那李二回道:“不是傭工,是做當槽兒的。”知縣道:“那日屍場上你說張三是薛蟠將碗砸死的,你親眼見的麼。”李二說道:“小的在櫃上,聽見說客房裏要酒。不多一回,便聽見說‘不好了,打傷了。’小的跑進去,只見張三躺在地下,也不能言語。小的便喊稟地保,一面報他母親去了。他們到底怎樣打的,實在不知道,求太爺問那喝酒的便知道了。”知縣喝道:“初審口供,你是親見的,怎麼如今說沒有見?”李二道:“小的前日唬昏了亂說。”衙役又吆喝了一聲。知縣便叫吳良問道:“你是同在一處喝酒的麼?薛蟠怎麼打的,據實供來。”吳良說:“小的那日在家,這個薛大爺叫我喝酒。他嫌酒不好要換,張三不肯。薛大爺生氣把酒向他臉上潑去,不曉得怎麼樣就碰在那腦袋上了。這是親眼見的。”知縣道:“胡說。前日屍場上薛蟠自己認拿碗砸死的,你說你親眼見的,怎麼今日的供不對?掌嘴。”衙役答應着要打,吳良求着說:“薛蟠實沒有與張三打架,酒碗失手碰在腦袋上的。求老爺問薛蟠便是恩典了。”知縣叫提薛蟠,問道:“你與張三到底有什麼仇隙?畢竟是如何死的,實供上來。”薛蟠道:“求太老爺開恩,小的實沒有打他。爲他不肯換酒,故拿酒潑他,不想一時失手,酒碗誤碰在他的腦袋上。小的即忙掩他的血,那裏知道再掩不住,血淌多了,過一回就死了。前日屍場上怕太老爺要打,所以說是拿碗砸他的。只求太爺開恩。”知縣便喝道:“好個糊塗東西!本縣問你怎麼砸他的,你便供說惱他不換酒才砸的,今日又供是失手碰的。”知縣假作聲勢,要打要夾,薛蟠一口咬定。知縣叫仵作將前日屍場填寫傷痕據實報來。仵作稟報說:“前日驗得張三尸身無傷,惟滷門有磁器傷長一寸七分,深五分,皮開,滷門骨脆裂破三分。實系磕碰傷。”知縣查對屍格相符,早知書吏改輕,也不駁詰,胡亂便叫畫供。張王氏哭喊道:“青天老爺!前日聽見還有多少傷,怎麼今日都沒有了?”知縣道:“這婦人胡說,現有屍格,你不知道麼。”叫屍叔張二便問道:“你侄兒身死,你知道有幾處傷?”張二忙供道:“腦袋上一傷。”知縣道:“可又來。”叫書吏將屍格給張王氏瞧去,並叫地保、屍叔指明與他瞧,現有屍場親押證見俱供並未打架,不爲鬪毆。只依誤傷吩咐畫供。將薛蟠監禁候詳,餘令原保領出,退堂。張王氏哭着亂嚷,知縣叫衆衙役攆他出去。張二也勸張王氏道:“實在誤傷,怎麼賴人。現在太老爺斷明,不要胡鬧了。”薛蝌在外打聽明白,心內喜歡,便差人回家送信。等批詳回來,便好打點贖罪,且住着等信。只聽路上三三兩兩傳說,有個貴妃薨了,皇上輟朝三日。這裏離陵寢不遠,知縣辦差墊道,一時料着不得閒,住在這裏無益,不如到監告訴哥哥安心等着,“我回家去,過幾日再來。”薛蟠也怕母親痛苦,帶信說:“我無事,必須衙門再使費幾次,便可回家了。只是不要可惜銀錢。”
\end{parag}


\begin{parag}
    薛蝌留下李祥在此照料,一徑回家,見了薛姨媽,陳說知縣怎樣徇情,怎樣審斷,終定了誤傷,將來屍親那裏再花些銀子,一準贖罪,便沒事了。薛姨媽聽說,暫且放心,說:“正盼你來家中照應。賈府裏本該謝去,況且周貴妃薨了,他們天天進去,家裏空落落的。我想着要去替姨太太那邊照應照應作伴兒,只是咱們家又沒人。你這來的正好。”薛蝌道:“我在外頭原聽見說是賈妃薨了,這麼才趕回來的。我們元妃好好兒的,怎麼說死了?”薛姨媽道:“上年原病過一次,也就好了。這回又沒聽見元妃有什麼病。只聞那府裏頭幾天老太太不大受用,合上眼便看見元妃娘娘。衆人都不放心,直至打聽起來,又沒有什麼事。到了大前兒晚上,老太太親口說是‘怎麼元妃獨自一個人到我這裏?’衆人只道是病中想的話,總不信。老太太又說:‘你們不信,元妃還與我說是榮華易盡,須要退步抽身。’衆人都說:‘誰不想到?這是有年紀的人思前想後的心事。’所以也不當件事。恰好第二天早起,裏頭吵嚷出來說娘娘病重,宣各誥命進去請安。他們就驚疑的了不得,趕着進去。他們還沒有出來,我們家裏已聽見周貴妃薨逝了。你想外頭的訛言,家裏的疑心,恰碰在一處,可奇不奇!”寶釵道:“不但是外頭的訛言舛錯,便在家裏的,一聽見‘娘娘’兩個字,也就都忙了,過後才明白。這兩天那府裏這些丫頭婆子來說,他們早知道不是咱們家的娘娘。我說:‘你們那裏拿得定呢?’他說道:‘前幾年正月,外省薦了一個算命的,說是很準。那老太太叫人將元妃八字夾在丫頭們八字裏頭,送出去叫他推算。他獨說這正月初一日生日的那位姑娘只怕時辰錯了,不然真是個貴人,也不能在這府中。老爺和衆人說,不管他錯不錯,照八字算去。那先生便說,甲申年正月丙寅這四個字內有傷官敗財,惟申字內有正官祿馬,這就是家裏養不住的,也不見什麼好。這日子是乙卯,初春木旺,雖是比肩,那裏知道愈比愈好,就象那個好木料,愈經斫削,才成大器。獨喜得時上什麼辛金爲貴,什麼巳中正官祿馬獨旺,這叫作飛天祿馬格。又說什麼日祿歸時,貴重的很,天月二德坐本命,貴受椒房之寵。這位姑娘若是時辰準了,定是一位主子娘娘。這不是算準了麼!我們還記得說,可惜榮華不久,只怕遇着寅年卯月,這就是比而又比,劫而又劫,譬如好木,太要做玲瓏剔透,本質就不堅了。他們把這些話都忘記了,只管瞎忙。我纔想起來告訴我們大奶奶,今年那裏是寅年卯月呢。”寶釵尚未說完,薛蝌急道:“且不要管人家的事,既有這樣個神仙算命的,我想哥哥今年什麼惡星照命,遭這麼橫禍,快開八字與我給他算去,看有妨礙麼。”寶釵道:“他是外省來的,不知如今在京不在了。”
\end{parag}


\begin{parag}
    說着,便打點薛姨媽往賈府去。到了那裏,只有李紈探春等在家接着,便問道:“大爺的事怎麼樣了?”薛姨媽道:“等詳上司才定,看來也到不了死罪了。”這才大家放心。探春便道:“昨晚太太想着說,上回家裏有事,全仗姨太太照應,如今自己有事,也難提了。心裏只是不放心。”薛姨媽道:“我在家裏也是難過。只是你大哥遭了事,你二兄弟又辦事去了,家裏你姐姐一個人,中什麼用?況且我們媳婦兒又是個不大曉事的,所以不能脫身過來。目今那裏知縣也正爲預備周貴妃的差事,不得了結案件,所以你二兄弟回來了,我才得過來看看。”李紈便道:“請姨太太這裏住幾天更好。”薛姨媽點頭道:“我也要在這邊給你們姐妹們作作伴兒,就只你寶妹妹冷靜些。”惜春道:“姨媽要惦着,爲什麼不把寶姐姐也請過來?”薛姨媽笑着說道:“使不得。”惜春道:“怎麼使不得?他先怎麼住着來呢?”李紈道:“你不懂的,人家家裏如今有事,怎麼來呢。”惜春也信以爲實,不便再問。正說着,賈母等回來。見了薛姨媽,也顧不得問好,便問薛蟠的事。薛姨媽細述了一遍。寶玉在旁聽見什麼蔣玉菡一段,當着衆人不問,心裏打量是”他既回了京,怎麼不來瞧我?”又見寶釵也不過來,不知是怎麼個原故。心內正自呆呆的想呢,恰好黛玉也來請安。寶玉稍覺心裏喜歡,便把想寶釵的念頭打斷,同着姊妹們在老太太那裏吃了晚飯。大家散了,薛姨媽將就住在老太太的套間屋裏。
\end{parag}


\begin{parag}
    寶玉回到自己房中,換了衣服,忽然想起蔣玉菡給的汗巾,便向襲人道:“你那一年沒有系的那條紅汗巾子還有沒有?”襲人道:“我擱着呢。問他做什麼?”寶玉道:“我白問問。”襲人道:“你沒有聽見,薛大爺相與這些混賬人,所以鬧到人命關天。你還提那些作什麼?有這樣白操心,倒不如靜靜兒的念念書,把這些個沒要緊的事撂開了也好。”寶玉道:“我並不鬧什麼,偶然想起,有也罷,沒也罷,我白問一聲,你們就有這些話。”襲人笑道:“並不是我多話。一個人知書達理,就該往上巴結纔是。就是心愛的人來了,也叫他瞧着喜歡尊敬啊。”寶玉被襲人一提,便說:“了不得,方纔我在老太太那邊,看見人多,沒有與妹妹說話。他也不曾理我,散的時候他先走了,此時必在屋裏。我去就來。”說着就走。襲人道:“快些回來罷,這都是我提頭兒,倒招起你的高興來了。”
\end{parag}


\begin{parag}
    寶玉也不答言,低着頭,一徑走到瀟湘館來。只見黛玉靠在桌上看書。寶玉走到跟前,笑說道:“妹妹早回來了。”黛玉也笑道:“你不理我,我還在那裏做什麼!”寶玉一面笑說:“他們人多說話,我插不下嘴去,所以沒有和你說話。”一面瞧着黛玉看的那本書。書上的字一個也不認得,有的象“芍”字,有的象“茫”字,也有一個“大”字旁邊“九”字加上一勾,中間又添個“五”字,也有上頭“五”字“六”字又添一個“木”字,底下又是一個“五”字,看着又奇怪,又納悶,便說:“妹妹近日愈發進了,看起天書來了。”黛玉嗤的一聲笑道:“好個唸書的人,連個琴譜都沒有見過。”寶玉道:“琴譜怎麼不知道,爲什麼上頭的字一個也不認得。妹妹你認得麼?”黛玉道:“不認得瞧他做什麼?”寶玉道:“我不信,從沒有聽見你會撫琴。我們書房裏掛着好幾張,前年來了一個清客先生叫做什麼嵇好古,老爺煩他撫了一曲。他取下琴來說,都使不得,還說:‘老先生若高興,改日攜琴來請教。’想是我們老爺也不懂,他便不來了。怎麼你有本事藏着?”黛玉道:“我何嘗真會呢。前日身上略覺舒服,在大書架上翻書,看有一套琴譜,甚有雅趣,上頭講的琴理甚通,手法說的也明白,真是古人靜心養性的工夫。我在揚州也聽得講究過,也曾學過,只是不弄了,就沒有了。這果真是‘三日不彈,手生荊棘。’前日看這幾篇沒有曲文,只有操名。我又到別處找了一本有曲文的來看着,纔有意思。究竟怎麼彈得好,實在也難。書上說的師曠鼓琴能來風雷龍鳳,孔聖人尚學琴於師襄,一操便知其爲文王,高山流水,得遇知音。”說到這裏,眼皮兒微微一動,慢慢的低下頭去。寶玉正聽得高興,便道:“好妹妹,你才說的實在有趣,只是我才見上頭的字都不認得,你教我幾個呢。”黛玉道:“不用教的,一說便可以知道的。”寶玉道:“我是個糊塗人,得教我那個‘大’字加一勾,中間一個‘五’字的。”黛玉笑道:“這‘大’字‘九’字是用左手大拇指按琴上的九徽,這一勾加‘五’字是右手鉤五絃。並不是一個字,乃是一聲,是極容易的。還有吟,揉,綽,注,撞,走,飛,推等法,是講究手法的。”寶玉樂得手舞足蹈的說:“好妹妹,你既明琴理,我們何不學起來。”黛玉道:“琴者,禁也。古人制下,原以治身,涵養性情,抑其淫蕩,去其奢侈。若要撫琴,必擇靜室高齋,或在層樓的上頭,在林石的裏面,或是山巔上,或是水涯上。再遇着那天地清和的時候,風清月朗,焚香靜坐,心不外想,氣血和平,才能與神合靈,與道合妙。所以古人說‘知音難遇’。若無知音,寧可獨對着那清風明月,蒼松怪石,野猿老鶴,撫弄一番,以寄興趣,方爲不負了這琴。還有一層,又要指法好,取音好。若必要撫琴,先須衣冠整齊,或鶴氅,或深衣,要如古人的像表,那才能稱聖人之器,然後盥了手,焚上香,方纔將身就在榻邊,把琴放在案上,坐在第五徽的地方兒,對着自己的當心,兩手方從容抬起,這才心身俱正。還要知道輕重疾徐,卷舒自若,體態尊重方好。”寶玉道:“我們學着頑,若這麼講究起來,那就難了。”
\end{parag}


\begin{parag}
    兩個人正說着,只見紫鵑進來,看見寶玉笑說道:“寶二爺,今日這樣高興。”寶玉笑道:“聽見妹妹講究的叫人頓開茅塞,所以越聽越愛聽。”紫鵑道:“不是這個高興,說的是二爺到我們這邊來的話。”寶玉道:“先時妹妹身上不舒服,我怕鬧的他煩。再者我又上學,因此顯著就疏遠了似的。”紫鵑不等說完,便道:“姑娘也是纔好,二爺既這麼說,坐坐也該讓姑娘歇歇兒了,別叫姑娘只是講究勞神了。”寶玉笑道:“可是我只顧愛聽,也就忘了妹妹勞神了。”黛玉笑道:“說這些倒也開心,也沒有什麼勞神的。只是怕我只管說,你只管不懂呢。”寶玉道:“橫豎慢慢的自然明白了。”說着,便站起來道:“當真的妹妹歇歇兒罷。明兒我告訴三妹妹和四妹妹去,叫他們都學起來,讓我聽。”黛玉笑道:“你也太受用了。即如大家學會了撫起來,你不懂,可不是對——”黛玉說到那裏,想起心上的事,便縮住口,不肯往下說了。寶玉便笑道:“只要你們能彈,我便愛聽,也不管牛不牛的了。”黛玉紅了臉一笑,紫鵑雪雁也都笑了。於是走出門來,只見秋紋帶着小丫頭捧着一盆蘭花來說:“太太那邊有人送了四盆蘭花來,因裏頭有事沒有空兒頑他,叫給二爺一盆,林姑娘一盆。”黛玉看時,卻有幾枝雙朵兒的,心中忽然一動,也不知是喜是悲,便呆呆的呆看。那寶玉此時卻一心只在琴上,便說:“妹妹有了蘭花,就可以做《猗蘭操》了。”黛玉聽了,心裏反不舒服。回到房中,看着花,想到”草木當春,花鮮葉茂,想我年紀尚小,便象三秋蒲柳。若是果能隨願,或者漸漸的好來,不然,只恐似那花柳殘春,怎禁得風催雨送。”想到那裏,不禁又滴下淚來。紫鵑在旁看見這般光景,卻想不出原故來。方纔寶玉在這裏那麼高興,如今好好的看花,怎麼又傷起心來。正愁着沒法兒解,只見寶釵那邊打發人來。未知何事,下回分解。
\end{parag}