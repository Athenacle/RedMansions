\chap{一百二十}{甄士隱詳說太虛情 賈雨村歸結紅樓夢}



\begin{parag}
    話說寶釵聽秋紋說襲人不好,連忙進去瞧看。巧姐兒同平兒也隨着走到襲人炕前。只見襲人心痛難禁,一時氣厥。寶釵等用開水灌了過來,仍舊扶他睡下,一面傳請大夫。巧姐兒問寶釵道:“襲人姐姐怎麼病到這個樣?”寶釵道:“大前兒晚上哭傷了心了,一時發暈栽倒了。太太叫人扶他回來,他就睡倒了。因外頭有事,沒有請大夫瞧他,所以致此。”說着,大夫來了,寶釵等略避。大夫看了脈,說是急怒所致,開了方子去了。原來襲人模糊聽見說寶玉若不回來,便要打發屋裏的人都出去,一急越發不好了。到大夫瞧後,秋紋給他煎藥。他各自一人躺着,神魂未定,好象寶玉在他面前,恍惚又象是個和尚,手裏拿着一本冊子揭着看,還說道:“你別錯了主意,我是不認得你們的了。”襲人似要和他說話,秋紋走來說:“藥好了,姐姐喫罷。”襲人睜眼一瞧,知是個夢,也不告訴人。吃了藥,便自己細細的想:“寶玉必是跟了和尚去。上回他要拿玉出去,便是要脫身的樣子,被我揪住,看他竟不象往常,把我混推混揉的,一點情意都沒有。後來待二奶奶更生厭煩。在別的姊妹跟前,也是沒有一點情意。這就是悟道的樣子。但是你悟了道,拋了二奶奶怎麼好!我是太太派我服侍你,雖是月錢照着那樣的分例,其實我究竟沒有在老爺太太跟前回明就算了你的屋裏人。若是老爺太太打發我出去,我若死守着,又叫人笑話,若是我出去,心想寶玉待我的情分,實在不忍。”左思右想,實在難處。想到剛纔的夢“好象和我無緣”的話,“倒不如死了乾淨。”豈知吃藥以後,心痛減了好些,也難躺着,只好勉強支持。過了幾日,起來服侍寶釵。寶釵想念寶玉,暗中垂淚,自嘆命苦。又知他母親打算給哥哥贖罪,很費張羅,不能不幫着打算。暫且不表。
\end{parag}


\begin{parag}
    且說賈政扶賈母靈柩,賈蓉送了秦氏鳳姐鴛鴦的棺木,到了金陵,先安了葬。賈蓉自送黛玉的靈也去安葬。賈政料理墳基的事。一日接到家書,一行一行的看到寶玉賈蘭得中,心裏自是喜歡。後來看到寶玉走失,復又煩惱,只得趕忙回來。在道兒上又聞得有恩赦的旨意,又接家書,果然赦罪復職,更是喜歡,便日夜趲行。
\end{parag}


\begin{parag}
    一日,行到毘陵驛地方,那天乍寒下雪,泊在一個清靜去處。賈政打發衆人上岸投帖辭謝朋友,總說即刻開船,都不敢勞動。船中只留一個小廝伺候,自己在船中寫家書,先要打發人起旱到家。寫到寶玉的事,便停筆。抬頭忽見船頭上微微的雪影裏面一個人,光着頭,赤着腳,身上披着一領大紅猩猩氈的斗篷,向賈政倒身下拜。賈政尚未認清,急忙出船,欲待扶住問他是誰。那人已拜了四拜,站起來打了個問訊。賈政纔要還揖,迎面一看,不是別人,卻是寶玉。賈政喫一大驚,忙問道:“可是寶玉麼?”那人只不言語,似喜似悲。賈政又問道:“你若是寶玉,如何這樣打扮,跑到這裏?”寶玉未及回言,只見舡頭上來了兩人,一僧一道,夾住寶玉說道:“俗緣已畢,還不快走。”說着,三個人飄然登岸而去。賈政不顧地滑,疾忙來趕。見那三人在前,那裏趕得上。只聽見他們三人口中不知是那個作歌曰:
\end{parag}


\begin{poem}

    \begin{pl}
        我所居兮,青埂之峯。
    \end{pl}


    \begin{pl}
        我所遊兮,鴻蒙太空。
    \end{pl}


    \begin{pl}
        誰與我遊兮,吾誰與從。
    \end{pl}


    \begin{pl}
        渺渺茫茫兮,歸彼大荒。
    \end{pl}
\end{poem}


\begin{parag}
    賈政一面聽着,一面趕去,轉過一小坡,倏然不見。賈政已趕得心虛氣喘,驚疑不定,回過頭來,見自己的小廝也是隨後趕來。賈政問道:“你看見方纔那三個人麼?”小廝道:“看見的。奴才爲老爺追趕,故也趕來。後來只見老爺,不見那三個人了。”賈政還欲前走,只見白茫茫一片曠野,並無一人。賈政知是古怪,只得回來。
\end{parag}


\begin{parag}
    衆家人回舡,見賈政不在艙中,問了舡夫,說是“老爺上岸追趕兩個和尚一個道士去了。”衆人也從雪地裏尋蹤迎去,遠遠見賈政來了,迎上去接着,一同回船。賈政坐下,喘息方定,將見寶玉的話說了一遍。衆人回稟,便要在這地方尋覓。賈政嘆道:“你們不知道,這是我親眼見的,並非鬼怪。況聽得歌聲大有元妙。那寶玉生下時銜了玉來,便也古怪,我早知不祥之兆,爲的是老太太疼愛,所以養育到今。便是那和尚道士,我也見了三次:頭一次是那僧道來說玉的好處,第二次便是寶玉病重,他來了將那玉持誦了一番,寶玉便好了,第三次送那玉來坐在前廳,我一轉眼就不見了。我心裏便有些詫異,只道寶玉果真有造化,高僧仙道來護佑他的。豈知寶玉是下凡歷劫的,竟哄了老太太十九年!如今叫我纔明白。”說到那裏,掉下淚來。衆人道:“寶二爺果然是下凡的和尚,就不該中舉人了。怎麼中了纔去?”賈政道:“你們那裏知道,大凡天上星宿,山中老僧,洞裏的精靈,他自有一種性情。你看寶玉何嘗肯唸書,他若略一經心,無有不能的。他那一種脾氣也是各別另樣。”說着,又嘆了幾聲。衆人便拿“蘭哥得中,家道復興”的話解了一番。賈政仍舊寫家書,便把這事寫上,勸諭閤家不必想念了。寫完封好,即着家人回去。賈政隨後趕回。暫且不題。
\end{parag}


\begin{parag}
    且說薛姨媽得了赦罪的信,便命薛蝌去各處借貸。並自己湊齊了贖罪銀兩。刑部準了,收兌了銀子,一角文書將薛蟠放出。他們母子姊妹弟兄見面,不必細述,自然是悲喜交集了。薛蟠自己立誓說道:“若是再犯前病,必定犯殺犯剮!”薛姨媽見他這樣,便要握他嘴說:“只要自己拿定主意,必定還要妄口巴舌血淋淋的起這樣惡誓麼!只香菱跟了你受了多少的苦處,你媳婦已經自己治死自己了,如今雖說窮了,這碗飯還有得喫,據我的主意,我便算他是媳婦了,你心裏怎麼樣?”薛蟠點頭願意。寶釵等也說:“很該這樣。”倒把香菱急得臉脹通紅,說是:“伏侍大爺一樣的,何必如此。”衆人便稱起大奶奶來,無人不服。薛蟠便要去拜謝賈家,薛姨媽寶釵也都過來。見了衆人,彼此聚首,又說了一番的話。正說着,恰好那日賈政的家人回家,呈上書子,說:“老爺不日到了。”王夫人叫賈蘭將書子念給聽。賈蘭唸到賈政親見寶玉的一段,衆人聽了都痛哭起來,王夫人寶釵襲人等更甚。大家又將賈政書內叫家內“不必悲傷,原是借胎”的話解說了一番。“與其作了官,倘或命運不好,犯了事壞家敗產,那時倒不好了。寧可咱們家出一位佛爺,倒是老爺太太的積德,所以纔投到咱們家來。不是說句不顧前後的話,當初東府裏太爺倒是修煉了十幾年,也沒有成了仙。這佛是更難成的。太太這麼一想,心裏便開豁了。”王夫人哭着和薛姨媽道:“寶玉拋了我,我還恨他呢。我嘆的是媳婦的命苦,纔成了一二年的親,怎麼他就硬着腸子都撂下了走了呢!”薛姨媽聽了也甚傷心。寶釵哭得人事不知。所有爺們都在外頭,王夫人便說道:“我爲他擔了一輩子的驚,剛剛兒的娶了親,中了舉人,又知道媳婦作了胎,我纔喜歡些,不想弄到這樣結局!早知這樣,就不該娶親害了人家的姑娘!”薛姨媽道:“這是自己一定的,咱們這樣人家,還有什麼別的說的嗎?幸喜有了胎,將來生個外孫子必定是有成立的,後來就有了結果了。你看大奶奶,如今蘭哥兒中了舉人,明年成了進士,可不是就做了官了麼。他頭裏的苦也算吃盡的了,如今的甜來,也是他爲人的好處。我們姑娘的心腸兒姊姊是知道的,並不是刻薄輕佻的人,姊姊倒不必耽憂。”王夫人被薛姨媽一番言語說得極有理,心想:“寶釵小時候更是廉靜寡慾極愛素淡的,他所以纔有這個事,想人生在世真有一定數的。看着寶釵雖是痛哭,他端莊樣兒一點不走,卻倒來勸我,這是真真難得的!不想寶玉這樣一個人,紅塵中福分竟沒有一點兒!”想了一回,也覺解了好些。又想到襲人身上:“若說別的丫頭呢,沒有什麼難處的,大的配了出去,小的伏侍二奶奶就是了。獨有襲人可怎麼處呢?”此時人多,也不好說,且等晚上和薛姨媽商量。
\end{parag}


\begin{parag}
    那日薛姨媽並未回家,因恐寶釵痛哭,所以在寶釵房中解勸。那寶釵卻是極明理,思前想後,“寶玉原是一種奇異的人。夙世前因,自有一定,原無可怨天尤人。”更將大道理的話告訴他母親了。薛姨媽心裏反倒安了,便到王夫人那裏先把寶釵的話說了。王夫人點頭嘆道:“若說我無德,不該有這樣好媳婦了。”說着,更又傷心起來。薛姨媽倒又勸了一會子,因又提起襲人來,說:“我見襲人近來瘦的了不得,他是一心想着寶哥兒。但是正配呢理應守的,屋裏人願守也是有的。惟有這襲人,雖說是算個屋裏人,到底他和寶哥兒並沒有過明路兒的。”王夫人道:“我纔剛想着,正要等妹妹商量商量。若說放他出去,恐怕他不願意,又要尋死覓活的,若要留着他也罷,又恐老爺不依。所以難處。”薛姨媽道:“我看姨老爺是再不肯叫守着的。再者姨老爺並不知道襲人的事,想來不過是個丫頭,那有留的理呢?只要姊姊叫他本家的人來,狠狠的吩咐他,叫他配一門正經親事,再多多的陪送他些東西。那孩子心腸兒也好,年紀兒又輕,也不枉跟了姐姐會子,也算姐姐待他不薄了。襲人那裏還得我細細勸他。就是叫他家的人來也不用告訴他,只等他家裏果然說定了好人家兒,我們還去打聽打聽,若果然足衣足食,女婿長的象個人兒,然後叫他出去。”王夫人聽了道:“這個主意很是。不然叫老爺冒冒失失的一辦,我可不是又害了一個人了麼!”薛姨媽聽了點頭道:“可不是麼!”又說了幾句,便辭了王夫人,仍到寶釵房中去了。
\end{parag}


\begin{parag}
    看見襲人淚痕滿面,薛姨媽便勸解譬喻了一會。襲人本來老實,不是伶牙利齒的人,薛姨媽說一句,他應一句,回來說道:“我是做下人的人,姨太太瞧得起我,纔和我說這些話,我是從不敢違拗太太的。”薛姨媽聽他的話,“好一個柔順的孩子!”心裏更加喜歡。寶釵又將大義的話說了一遍,大家各自相安。
\end{parag}


\begin{parag}
    過了幾日,賈政回家,衆人迎接。賈政見賈赦賈珍已都回家,弟兄叔侄相見,大家歷敘別來的景況。然後內眷們見了,不免想起寶玉來,又大家傷了一會子心。賈政喝住道:“這是一定的道理。如今只要我們在外把持家事,你們在內相助,斷不可仍是從前這樣的散慢。別房的事,各有各家料理,也不用承總。我們本房的事,裏頭全歸於你,都要按理而行。”王夫人便將寶釵有孕的話也告訴了,將來丫頭們都勸放出去。賈政聽了,點頭無語。
\end{parag}


\begin{parag}
    次日賈政進內,請示大臣們,說是:“蒙恩感激,但未服闋,應該怎麼謝恩之處,望乞大人們指教。”衆朝臣說是代奏請旨。於是聖恩浩蕩,即命陛見。賈政進內謝了恩,聖上又降了好些旨意,又問起寶玉的事來。賈政據實回奏。聖上稱奇,旨意說,寶玉的文章固是清奇,想他必是過來人,所以如此。若在朝中,可以進用。他既不敢受聖朝的爵位,便賞了一個“文妙真人”的道號。賈政又叩頭謝恩而出。
\end{parag}


\begin{parag}
    回到家中,賈璉賈珍接着,賈政將朝內的話述了一遍,衆人喜歡。賈珍便回說:“寧國府第收拾齊全,回明瞭要搬過去。櫳翠庵圈在園內,給四妹妹靜養。”賈政並不言語,隔了半日,卻吩咐了一番仰報天恩的話。賈璉也趁便回說:“巧姐親事,父親太太都願意給周家爲媳。”賈政昨晚也知巧姐的始末,便說:“大老爺大太太作主就是了。莫說村居不好,只要人家清白,孩子肯唸書,能夠上進。朝裏那些官兒難道都是城裏的人麼?”賈璉答應了“是”,又說:“父親有了年紀,況且又有痰症的根子,靜養幾年,諸事原仗二老爺爲主。”賈政道:“提起村居養靜,甚合我意。只是我受恩深重,尚未酬報耳。”賈政說畢進內。賈璉打發請了劉姥姥來,應了這件事。劉姥姥見了王夫人等,便說些將來怎樣升官,怎樣起家,怎樣子孫昌盛。正說着,丫頭回道:“花自芳的女人進來請安。”王夫人問幾句話,花自芳的女人將親戚作媒,說的是城南蔣家的,現在有房有地,又有鋪面,姑爺年紀略大了幾歲,並沒有娶過的,況且人物兒長的是百裏挑一的。王夫人聽了願意,說道:“你去應了,隔幾日進來再接你妹子罷。”王夫人又命人打聽,都說是好。王夫人便告訴了寶釵,仍請了薛姨媽細細的告訴了襲人。襲人悲傷不已,又不敢違命的,心裏想起寶玉那年到他家去,回來說的死也不回去的話,“如今太太硬作主張。若說我守着,又叫人說我不害臊,若是去了,實不是我的心願”,便哭得咽哽難鳴,又被薛姨媽寶釵等苦勸,回過念頭想道:“我若是死在這裏,倒把太太的好心弄壞了。我該死在家裏纔是。”於是,襲人含悲叩辭了衆人,那姐妹分手時自然更有一番不忍說。襲人懷着必死的心腸上車回去,見了哥哥嫂子,也是哭泣,但只說不出來。那花自芳悉把蔣家的娉禮送給他看,又把自己所辦妝奩一一指給他瞧,說那是太太賞的,那是置辦的。襲人此時更難開口,住了兩天,細想起來:“哥哥辦事不錯,若是死在哥哥家裏,豈不又害了哥哥呢。”千思萬想,左右爲難,真是一縷柔腸,幾乎牽斷,只得忍住。
\end{parag}


\begin{parag}
    那日已是迎娶吉期,襲人本不是那一種潑辣人,委委屈屈的上轎而去,心裏另想到那裏再作打算。豈知過了門,見那蔣家辦事極其認真,全都按着正配的規矩。一進了門,丫頭僕婦都稱奶奶。襲人此時欲要死在這裏,又恐害了人家,辜負了一番好意。那夜原是哭着不肯俯就的,那姑爺卻極柔情曲意的承順。到了第二天開箱,這姑爺看見一條猩紅汗巾,方知是寶玉的丫頭。原來當初只知是賈母的侍兒,益想不到是襲人。此時蔣玉菡念着寶玉待他的舊情,倒覺滿心惶愧,更加周旋,又故意將寶玉所換那條松花綠的汗巾拿出來。襲人看了,方知這姓蔣的原來就是蔣玉菡,始信姻緣前定。襲人纔將心事說出,蔣玉菡也深爲嘆息敬服,不敢勉強,並越發溫柔體貼,弄得個襲人真無死所了。看官聽說:雖然事有前定,無可奈何。但孽子孤臣,義夫節婦,這“不得已”三字也不是一概推委得的。此襲人所以在又一副冊也。正是前人過那桃花廟的詩上說道:
\end{parag}


\begin{poem}
    \begin{pl}
        千古艱難惟一死,傷心豈獨息夫人!
    \end{pl}
\end{poem}


\begin{parag}
    不言襲人從此又是一番天地。且說那賈雨村犯了婪索的案件,審明定罪,今遇大赦,褫籍爲民。雨村因叫家眷先行,自己帶了一個小廝,一車行李,來到急流津覺迷渡口。只見一個道者從那渡頭草棚裏出來,執手相迎。雨村認得是甄士隱,也連忙打恭,士隱道:“賈先生別來無恙?”雨村道:“老仙長到底是甄老先生!何前次相逢覿面不認?後知火焚草亭,下鄙深爲惶恐。今日幸得相逢,益嘆老仙翁道德高深。奈鄙人下愚不移,致有今日。”甄士隱道:“前者老大人高官顯爵,貧道怎敢相認!原因故交,敢贈片言,不意老大人相棄之深。然而富貴窮通,亦非偶然,今日復得相逢,也是一樁奇事。這裏離草菴不遠,暫請膝談,未知可否?”
\end{parag}


\begin{parag}
    雨村欣然領命,兩人攜手而行,小廝驅車隨後,到了一座茅庵。士隱讓進雨村坐下,小童獻上茶來。雨村便請教仙長超塵的始末。士隱笑道:“一念之間,塵凡頓易。老先生從繁華境中來,豈不知溫柔富貴鄉中有一寶玉乎?”雨村道:“怎麼不知。近聞紛紛傳述,說他也遁入空門。下愚當時也曾與他往來過數次,再不想此人竟有如是之決絕。”士隱道:“非也。這一段奇緣,我先知之。昔年我與先生在仁清巷舊宅門口敘話之前,我已會過他一面。”雨村驚訝道:“京城離貴鄉甚遠,何以能見?”士隱道:“神交久矣。”雨村道:“既然如此,現今寶玉的下落,仙長定能知之。”士隱道:“寶玉,即寶玉也。那年榮寧查抄之前,釵黛分離之日,此玉早已離世。一爲避禍,二爲撮合,從此夙緣一了,形質歸一,又復稍示神靈,高魁貴子,方顯得此玉那天奇地靈之寶,非凡間可比。前經茫茫大士渺渺真人攜帶下凡,如今塵緣已滿,仍是此二人攜歸本處,這便是寶玉的下落。”雨村聽了,雖不能全然明白,卻也十知四五,便點頭嘆道:“原來如此,下愚不知。但那寶玉既有如此的來歷,又何以情迷至此,復又豁悟如此?還要請教。”士隱笑道:“此事說來,老先生未必盡解。太虛幻境即是真如福地。一番閱冊,原始要終之道,歷歷生平,如何不悟?仙草歸真,焉有通靈不復原之理呢!”雨村聽着,卻不明白了。知仙機也不便更問,因又說道:“寶玉之事既得聞命,但是敝族閨秀如此之多,何元妃以下算來結局俱屬平常呢?”士隱嘆息道:“老先生莫怪拙言,貴族之女俱屬從情天孽海而來。大凡古今女子,那‘淫’字固不可犯,只這‘情’字也是沾染不得的。所以崔鶯蘇小,無非仙子塵心,宋玉相如,大是文人口孽。凡是情思纏綿的,那結果就不可問了。”雨村聽到這裏,不覺拈鬚長嘆,因又問道:“請教老仙翁,那榮寧兩府,尚可如前否?”士隱道:“福善禍淫,古今定理。現今榮寧兩府,善者修緣,惡者悔禍,將來蘭桂齊芳,家道復初,也是自然的道理。”雨村低了半日頭,忽然笑道:“是了,是了。現在他府中有一個名蘭的已中鄉榜,恰好應着‘蘭’字。適間老仙翁說‘蘭桂齊芳’,又道寶玉‘高魁子貴’,莫非他有遺腹之子,可以飛黃騰達的麼?”士隱微微笑道:“此係後事,未便預說。”雨村還要再問,士隱不答,便命人設俱盤飧,邀雨村共食。
\end{parag}


\begin{parag}
    食畢,雨村還要問自己的終身,士隱便道:“老先生草菴暫歇,我還有一段俗緣未了,正當今日完結。”雨村驚訝道:“仙長純修若此,不知尚有何俗緣?”士隱道:“也不過是兒女私情罷了。”雨村聽了益發驚異:“請問仙長,何出此言?”士隱道:“老先生有所不知,小女英蓮幼遭塵劫,老先生初任之時曾經判斷。今歸薛姓,產難完劫,遺一子於薛家以承宗祧。此時正是塵緣脫盡之時,只好接引接引。”士隱說着拂袖而起。雨村心中恍恍惚惚,就在這急流津覺迷渡口草菴中睡着了。
\end{parag}


\begin{parag}
    這士隱自去度脫了香菱,送到太虛幻境,交那警幻仙子對冊,剛過牌坊,見那一僧一道,縹渺而來。士隱接着說道:“大士,真人,恭喜,賀喜!情緣完結,都交割清楚了麼?”那僧道說:“情緣尚未全結,倒是那蠢物已經回來了。還得把他送還原所,將他的後事敘明,不枉他下世一回。”士隱聽了,便供手而別。那僧道仍攜了玉到青埂峯下,將寶玉安放在女媧煉石補天之處,各自雲遊而去。從此後,“天外書傳天外事,兩番人作一番人。”
\end{parag}


\begin{parag}
    這一日空空道人又從青埂峯前經過,見那補天未用之石仍在那裏,上面字跡依然如舊,又從頭的細細看了一遍,見後面偈文後又歷敘了多少收緣結果的話頭,便點頭嘆道:“我從前見石兄這段奇文,原說可以聞世傳奇,所以曾經抄錄,但未見返本還原。不知何時復有此一佳話,方知石兄下凡一次,磨出光明,修成圓覺,也可謂無復遺憾了。只怕年深日久,字跡模糊,反有舛錯,不如我再抄錄一番,尋個世上清閒無事的人,託他傳遍,知道奇而不奇,俗而不俗,真而不真,假而不假。或者塵夢勞人,聊倩鳥呼歸去,山靈好客,更從石化飛來,亦未可知。”想畢,便又抄了,仍袖至那繁華昌盛的地方,遍尋了一番,不是建功立業之人,即系饒口謀衣之輩,那有閒情更去和石頭饒舌。直尋到急流津覺迷渡口,草菴中睡着一個人,因想他必是閒人,便要將這抄錄的《石頭記》給他看看。那知那人再叫不醒。空空道人復又使勁拉他,纔慢慢的開眼坐起,便草草一看,仍舊擲下道:“這事我早已親見盡知。你這抄錄的尚無舛錯,我只指與你一個人,託他傳去,便可歸結這一新鮮公案了。”空空道人忙問何人,那人道:“你須待某年某月某日到一個悼紅軒中,有個曹雪芹先生,只說賈雨村言託他如此如此。”說畢,仍舊睡下了。
\end{parag}


\begin{parag}
    那空空道人牢牢記着此言,又不知過了幾世幾劫,果然有個悼紅軒,見那曹雪芹先生正在那裏翻閱歷來的古史。空空道人便將賈雨村言了,方把這《石頭記》示看。那雪芹先生笑道:“果然是‘賈雨村言’了!”空空道人便問:“先生何以認得此人,便肯替他傳述?”曹雪芹先生笑道:“說你空,原來你肚裏果然空空。既是假語村言,但無魯魚亥豕以及背謬矛盾之處,樂得與二三同志,酒餘飯飽,雨夕燈窗之下,同消寂寞,又不必大人先生品題傳世,似你這樣尋根問底,便是刻舟求劍,膠柱鼓瑟了。”那空空道人聽了,仰天大笑,擲下抄本,飄然而去。一面走着,口中說道:“果然是敷衍荒唐!不但作者不知,抄者不知,並閱者也不知。不過遊戲筆墨,陶情適性而已!”後人見了這本奇傳,亦曾題過四句偈語,爲作者緣起之言更轉一竿頭雲:
\end{parag}


\begin{poem}
    \begin{pl}
        說到辛酸處,荒唐愈可悲。
    \end{pl}

    \begin{pl}
        由來同一夢,休笑世人癡!
    \end{pl}
\end{poem}