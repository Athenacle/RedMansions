\chap{八十五}{賈存週報升郎中任 薛文起復惹放流刑}



\begin{parag}
    話說趙姨娘正在屋裏抱怨賈環,只聽賈環在外間屋裏發話道:“我不過弄倒了藥吊子,灑了一點子藥,那丫頭子又沒就死了,值的他也罵我,你也罵我,賴我心壞,把我往死裏糟踏。等着我明兒還要那小丫頭子的命呢,看你們怎麼着!只叫他們提防着就是了。”那趙姨娘趕忙從裏間出來,握住他的嘴說道:“你還只管信口胡唚,還叫人家先要了我的命呢!”孃兒兩個吵了一回。趙姨娘聽見鳳姐的話,越想越氣,也不着人來安慰鳳姐一聲兒。過了幾天,巧姐兒也好了。因此兩邊結怨比從前更加一層了。
\end{parag}


\begin{parag}
    一日林之孝進來回道:“今日是北靜郡王生日,請老爺的示下。”賈政吩咐道:“只按向年舊例辦了,回大老爺知道,送去就是了。”林之孝答應了,自去辦理。不一時,賈赦過來同賈政商議,帶了賈珍,賈璉,寶玉去與北靜王拜壽。別人還不理論,惟有寶玉素日仰慕北靜王的容貌威儀,巴不得常見纔好,遂連忙換了衣服,跟着來到北府。賈赦賈政遞了職名候諭。不多時,裏面出來了一個太監,手裏掐着數珠兒,見了賈赦賈政,笑嘻嘻的說道:“二位老爺好?”賈赦賈政也都趕忙問好。他兄弟三人也過來問了好。那太監道:“王爺叫請進去呢。”於是爺兒五個跟着那太監進入府中,過了兩層門,轉過一層殿去,裏面方是內宮門。剛到門前,大家站住,那太監先進去回王爺去了。這裏門上小太監都迎着問了好。一時那太監出來,說了個“請”字,爺兒五個肅敬跟入。只見北靜郡王穿着禮服,已迎到殿門廊下。賈赦賈政先上來請安,捱次便是珍,璉,寶玉請安。那北靜郡王單拉着寶玉道:“我久不見你,很惦記你。”因又笑問道:“你那塊玉兒好?”寶玉躬着身打着一半千兒回道:“蒙王爺福庇,都好。”北靜王道:“今日你來,沒有什麼好東西給你喫的,倒是大家說說話兒罷。”說着,幾個老公打起簾子,北靜王說“請”,自己卻先進去,然後賈赦等都躬着身跟進去。先是賈赦請北靜王受禮,北靜王也說了兩句謙辭,那賈赦早已跪下,次及賈政等捱次行禮,自不必說。
\end{parag}


\begin{parag}
    那賈赦等復肅敬退出。北靜王吩咐太監等讓在衆戚舊一處好生款待,卻單留寶玉在這裏說話兒,又賞了坐。寶玉又磕頭謝了恩,在挨門邊繡墩上側坐,說了一回讀書作文諸事。北靜王甚加愛惜,又賞了茶,因說道:“昨兒巡撫吳大人來陛見,說起令尊翁前任學政時,秉公辦事,凡屬生童,俱心服之至。他陛見時,萬歲爺也曾問過,他也十分保舉,可知是令尊翁的喜兆。”寶玉連忙站起,聽畢這一段話,纔回啓道:“此是王爺的恩典,吳大人的盛情。”正說着,小太監進來回道:“外面諸位大人老爺都在前殿謝王爺賞宴。”說着,呈上謝宴並請午安的帖子來。北靜王略看了一看,仍遞給小太監,笑了一笑說道:“知道了,勞動他們。”那小太監又回道:“這賈寶玉王爺單賞的飯預備了。”北靜王便命那太監帶了寶玉到一所極小巧精緻的院裏,派人陪着吃了飯,又過來謝了恩。北靜王又說了些好話兒,忽然笑說道:“我前次見你那塊玉倒有趣兒,回來說了個式樣,叫他們也作了一塊來。今日你來得正好,就給你帶回去頑罷。”因命小太監取來,親手遞給寶玉。寶玉接過來捧着,又謝了,然後退出。北靜王又命兩個小太監跟出來,才同着賈赦等回來了。賈赦便各自回院裏去。
\end{parag}


\begin{parag}
    這裏賈政帶着他三人回來見過賈母,請過了安,說了一回府裏遇見的人。寶玉又回了賈政吳大人陛見保舉的話。賈政道:“這吳大人本來咱們相好,也是我輩中人,還倒是有骨氣的。”又說了幾句閒話兒,賈母便叫”歇着去罷。”賈政退出,珍,璉,寶玉都跟到門口。賈政道:“你們都回去陪老太太坐着去罷。”說着,便回房去。剛坐了一坐,只見一個小丫頭回道:“外面林之孝請老爺回話。”說着,遞上個紅單帖來,寫着吳巡撫的名字。賈政知是來拜,便叫小丫頭叫林之孝進來。賈政出至廊檐下。林之孝進來回道:“今日巡撫吳大人來拜,奴才回了去了。再奴才還聽見說,現今工部出了一個郎中缺,外頭人和部裏都吵嚷是老爺擬正呢。”賈政道:“瞧罷咧。”林之孝又回了幾句話,纔出去了。
\end{parag}


\begin{parag}
    且說珍,璉,寶玉三人回去,獨有寶玉到賈母那邊,一面述說北靜王待他的光景,並拿出那塊玉來。大家看着笑了一回。賈母因命人:“給他收起去罷,別丟了。”因問:“你那塊玉好生帶着罷?別鬧混了。”寶玉在項上摘了下來,說:“這不是我那一塊玉,那裏就掉了呢。比起來,兩塊玉差遠着呢,那裏混得過。我正要告訴老太太,前兒晚上我睡的時候把玉摘下來掛在帳子裏,他竟放起光來了,滿帳子都是紅的。”賈母說道:“又胡說了,帳子的檐子是紅的,火光照着,自然紅是有的。”寶玉道:“不是。那時候燈已滅了,屋裏都漆黑的了,還看得見他呢。”邢王二夫人抿着嘴笑。鳳姐道:“這是喜信發動了。”寶玉道:“什麼喜信?”賈母道:“你不懂得。今兒個鬧了一天,你去歇歇兒去罷,別在這裏說呆話了。”寶玉又站了一回兒,纔回園中去了。
\end{parag}


\begin{parag}
    這裏賈母問道:“正是。你們去看薛姨媽說起這事沒有?”王夫人道:“本來就要去看的,因鳳丫頭爲巧姐兒病着,耽擱了兩天,今日纔去的。這事我們都告訴了,姨媽倒也十分願意,只說蟠兒這時侯不在家,目今他父親沒了,只得和他商量商量再辦。”賈母道:“這也是情理的話。既這麼樣,大家先別提起,等姨太太那邊商量定了再說。”不說賈母處談論親事,且說寶玉回到自己房中,告訴襲人道:“老太太與鳳姐姐方纔說話含含糊糊,不知是什麼意思。”襲人想了想,笑了一笑道:“這個我也猜不着。但只剛纔說這些話時,林姑娘在跟前沒有?”寶玉道:“林姑娘才病起來,這些時何曾到老太太那邊去呢。”正說着,只聽外間屋裏麝月與秋紋拌嘴。襲人道:“你兩個又鬧什麼?”麝月道:“我們兩個鬥牌,他贏了我的錢他拿了去,他輸了錢就不肯拿出來。這也罷了,他倒把我的錢都搶了去了。”寶玉笑道:“幾個錢什麼要緊,傻丫頭,不許鬧了。”說的兩個人都咕嘟着嘴坐着去了。這裏襲人打發寶玉睡下。不提。
\end{parag}


\begin{parag}
    卻說襲人聽了寶玉方纔的話,也明知是給寶玉提親的事。因恐寶玉每有癡想,這一提起不知又招出他多少呆話來,所以故作不知,自己心上卻也是頭一件關切的事。夜間躺着想了個主意,不如去見見紫鵑,看他有什麼動靜,自然就知道了。次日一早起來,打發寶玉上了學,自己梳洗了,便慢慢的去到瀟湘館來。只見紫鵑正在那裏掐花兒呢,見襲人進來,便笑嘻嘻的道:“姐姐屋裏坐着。”襲人道:“坐着,妹妹掐花兒呢嗎?姑娘呢?”紫鵑道:“姑娘才梳洗完了,等着溫藥呢。”紫鵑一面說着,一面同襲人進來。見了黛玉正在那裏拿着一本書看。襲人陪着笑道:“姑娘怨不得勞神,起來就看書。我們寶二爺唸書若能象姑娘這樣,豈不好了呢。”黛玉笑着把書放下。雪雁已拿着個小茶盤裏託着一鍾藥,一鍾水,小丫頭在後面捧着痰盒漱盂進來。原來襲人來時要探探口氣,坐了一回,無處入話,又想着黛玉最是心多,探不成消息再惹着了他倒是不好,又坐了坐,搭訕着辭了出來了。將到怡紅院門口,只見兩個人在那裏站着呢。襲人不便往前走,那一個早看見了,連忙跑過來。襲人一看,卻是鋤藥,因問”你作什麼?”鋤藥道:“剛纔芸二爺來了,拿了個帖兒,說給咱們寶二爺瞧的,在這裏候信。”襲人道:“寶二爺天天上學,你難道不知道,還候什麼信呢。”鋤藥笑道:“我告訴他了。他叫告訴姑娘,聽姑娘的信呢。”襲人正要說話,只見那一個也慢慢的蹭了過來,細看時,就是賈芸,溜溜湫湫往這邊來了。襲人見是賈芸,連忙向鋤藥道:“你告訴說知道了,回來給寶二爺瞧罷。”那賈芸原要過來和襲人說話,無非親近之意,又不敢造次,只得慢慢踱來。相離不遠,不想襲人說出這話,自己也不好再往前走,只好站住。這裏襲人已掉背臉往回裏去了。賈芸只得怏怏而回,同鋤藥出去了。
\end{parag}


\begin{parag}
    晚間寶玉回房,襲人便回道:“今日廊下小芸二爺來了。”寶玉道:“作什麼?”襲人道:“他還有個帖兒呢。”寶玉道:“在那裏?拿來我看看。”麝月便走去在裏間屋裏書槅子上頭拿了來。寶玉接過看時,上面皮兒上寫着“叔父大人安稟”。寶玉道:“這孩子怎麼又不認我作父親了?”襲人道:“怎麼?”寶玉道:“前年他送我白海棠時稱我作‘父親大人’今日這帖子封皮上寫着‘叔父’,可不是又不認了麼。”襲人道:“他也不害臊,你也不害臊。他那麼大了,倒認你這麼大兒的作父親,可不是他不害臊?你正經連個——”剛說到這裏,臉一紅,微微的一笑。寶玉也覺得了,便道:“這倒難講。俗語說:‘和尚無兒,孝子多着呢。’只是我看着他還伶俐得人心兒,才這麼着,他不願意,我還不希罕呢。”說着,一面拆那帖兒,襲人也笑道:“那小芸二爺也有些鬼鬼頭頭的。什麼時候又要看人,什麼時侯又躲躲藏藏的,可知也是個心術不正的貨。”寶玉只顧拆開看那字兒,也不理會襲人這些話。襲人見他看那帖兒,皺一回眉,又笑一笑兒,又搖搖頭兒,後來光景竟大不耐煩起來。襲人等他看完了,問道:“是什麼事情?”寶玉也不答言,把那帖子已經撕作幾段,襲人見這般光景,也不便再問,便問寶玉吃了飯還看書不看。寶玉道:“可笑芸兒這孩子竟這樣的混賬。”襲人見他所答非所問,便微微的笑着問道:“到底是什麼事?”寶玉道:“問他作什麼,咱們喫飯罷。吃了飯歇着罷,心裏鬧的怪煩的。”說着叫小丫頭子點了一個火兒來,把那撕的帖兒燒了。
\end{parag}


\begin{parag}
    一時小丫頭們擺上飯來。寶玉只是怔怔的坐着,襲人連哄帶慪催着吃了一口兒飯,便擱下了,仍是悶悶的歪在牀上。一時間,忽然掉下淚來。此時襲人麝月都摸不着頭腦。麝月道:“好好兒的,這又是爲什麼?都是什麼芸兒雨兒的,不知什麼事弄了這麼個浪帖子來,惹的這麼傻了的似的,哭一會子,笑一會子。要天長日久鬧起這悶葫蘆來,可叫人怎麼受呢。”說着,竟傷起心來。襲人旁邊由不得要笑,便勸道:“好妹妹,你也別慪人了。他一個人就夠受了,你又這麼着。他那帖子上的事難道與你相干?”麝月道:“你混說起來了。知道他帖兒上寫的是什麼混賬話,你混往人身上扯。要那麼說,他帖兒上只怕倒與你相干呢。”襲人還未答言,只聽寶玉在牀上噗哧的一聲笑了,爬起來抖了抖衣裳,說:“咱們睡覺罷,別鬧了。明日我還起早唸書呢。”說着便躺下睡了。一宿無話。
\end{parag}


\begin{parag}
    次日寶玉起來梳洗了,便往家塾裏去。走出院門,忽然想起,叫焙茗略等,急忙轉身回來叫:“麝月姐姐呢?”麝月答應着出來問道:“怎麼又回來了?”寶玉道:“今日芸兒要來了,告訴他別在這裏鬧,再鬧我就回老太太和老爺去了。”麝月答應了,寶玉才轉身去了。剛往外走着,只見賈芸慌慌張張往裏來,看見寶玉連忙請安,說:“叔叔大喜了。”那寶玉估量着是昨日那件事,便說道:“你也太冒失了,不管人心裏有事沒事,只管來攪。”賈芸陪笑道:“叔叔不信只管瞧去,人都來了,在咱們大門口呢。”寶玉越發急了,說:“這是那裏的話!”正說着,只聽外邊一片聲嚷起來。賈芸道:“叔叔聽這不是?”寶玉越發心裏狐疑起來,只聽一個人嚷道:“你們這些人好沒規矩,這是什麼地方,你們在這裏混嚷。”那人答道:“誰叫老爺升了官呢,怎麼不叫我們來吵喜呢。別人家盼着吵還不能呢。”寶玉聽了,才知道是賈政升了郎中了,人來報喜的。心中自是甚喜。連忙要走時,賈芸趕着說道:“叔叔樂不樂?叔叔的親事要再成了,不用說是兩層喜了。”寶玉紅了臉,啐了一口道:“呸!沒趣兒的東西!還不快走呢。”賈芸把臉紅了道:“這有什麼的,我看你老人家就不——”寶玉沉着臉道:“就不什麼?”賈芸未及說完,也不敢言語了。
\end{parag}


\begin{parag}
    寶玉連忙來到家塾中,只見代儒笑着說道:“我纔剛聽見你老爺升了。你今日還來了麼?”寶玉陪笑道:“過來見了太爺,好到老爺那邊去。”代儒道:“今日不必來了,放你一天假罷。可不許回園子裏頑去。你年紀不小了,雖不能辦事,也當跟着你大哥他們學學纔是。”寶玉答應着回來。剛走到二門口,只見李貴走來迎着,旁邊站住笑道:“二爺來了麼,奴才纔要到學裏請去。”寶玉笑道:“誰說的?”李貴道:“老太太纔打發人到院裏去找二爺,那邊的姑娘們說二爺學裏去了。剛纔老太太打發人出來叫奴才去給二爺告幾天假,聽說還要唱戲賀喜呢,二爺就來了。”說着,寶玉自己進去。進了二門,只見滿院裏丫頭老婆都是笑容滿面,見他來了,笑道:“二爺這早晚纔來,還不快進去給老太太道喜去呢。”
\end{parag}


\begin{parag}
    寶玉笑着進了房門,只見黛玉挨着賈母左邊坐着呢,右邊是湘雲。地下邢王二夫人。探春,惜春,李紈,鳳姐,李紋,李綺,邢岫煙一干姐妹,都在屋裏,只不見寶釵,寶琴,迎春三人。寶玉此時喜的無話可說,忙給賈母道了喜,又給邢王二夫人道喜,一一見了衆姐妹,便向黛玉笑道:“妹妹身體可大好了?”黛玉也微笑道:“大好了。聽見說二哥哥身上也欠安,好了麼?”寶玉道:“可不是,我那日夜裏忽然心裏疼起來,這幾天剛好些就上學去了,也沒能過去看妹妹。”黛玉不等他說完,早扭過頭和探春說話去了。鳳姐在地下站着笑道:“你兩個那裏象天天在一處的,倒象是客一般,有這些套話,可是人說的‘相敬如賓’了。”說的大家一笑。林黛玉滿臉飛紅,又不好說,又不好不說,遲了一回兒,才說道:“你懂得什麼?”衆人越發笑了。鳳姐一時回過味來,才知道自己出言冒失,正要拿話岔時,只見寶玉忽然向黛玉道:“林妹妹,你瞧芸兒這種冒失鬼。”說了一句,方想起來,便不言語了。招的大家又都笑起來,說:“這從那裏說起。”黛玉也摸不着頭腦,也跟着訕訕的笑。寶玉無可搭訕,因又說道:“可是剛纔我聽見有人要送戲,說是幾兒?”大家都瞅着他笑。鳳姐兒道:“你在外頭聽見,你來告訴我們。你這會子問誰呢?”寶玉得便說道:“我外頭再去問問去。”賈母道:“別跑到外頭去,頭一件看報喜的笑話,第二件你老子今日大喜,回來碰見你,又該生氣了。”寶玉答應了個“是”,纔出來了。
\end{parag}


\begin{parag}
    這裏賈母因問鳳姐誰說送戲的話,鳳姐道:“說是舅太爺那邊說,後兒日子好,送一班新出的小戲兒給老太太,老爺,太太賀喜。”因又笑着說道:“不但日子好,還是好日子呢。”說着這話,卻瞅着黛玉笑。黛玉也微笑。王夫人因道:“可是呢,後日還是外甥女兒的好日子呢。”賈母想了一想,也笑道:“可見我如今老了,什麼事都糊塗了。虧了有我這鳳丫頭是我個‘給事中’。既這麼着,很好,他舅舅家給他們賀喜,你舅舅家就給你做生日,豈不好呢。”說的大家都笑起來,說道:“老祖宗說句話兒都是上篇上論的,怎麼怨得有這麼大福氣呢。”說着,寶玉進來,聽見這些話,越發樂的手舞足蹈了。一時,大家都在賈母這邊喫飯,甚熱鬧,自不必說。飯後,那賈政謝恩回來,給宗祠裏磕了頭,便來給賈母磕頭,站着說了幾句話,便出去拜客去了。這裏接連着親戚族中的人來來去去,鬧鬧穰穰,車馬填門,貂蟬滿座,真是:
\end{parag}


\begin{poem}
    \begin{pl}
        花到正開蜂蝶鬧,月逢十足海天寬。
    \end{pl}
\end{poem}


\begin{parag}
    如此兩日,已是慶賀之期。這日一早,王子騰和親戚家已送過一班戲來,就在賈母正廳前搭起行臺。外頭爺們都穿着公服陪侍,親戚來賀的約有十餘桌酒。裏面爲着是新戲,又見賈母高興,便將琉璃戲屏隔在後廈,裏面也擺下酒席。上首薛姨媽一桌,是王夫人寶琴陪着,對面老太太一桌,是邢夫人岫煙陪着,下面尚空兩桌,賈母叫他們快來,一回兒,只見鳳姐領着衆丫頭,都簇擁着林黛玉來了。黛玉略換了幾件新鮮衣服,打扮得宛如嫦娥下界,含羞帶笑的出來見了衆人。湘雲,李紋,李紈都讓他上首座,黛玉只是不肯。賈母笑道:“今日你坐了罷。”薛姨媽站起來問道:“今日林姑娘也有喜事麼?”賈母笑道:“是他的生日。”薛姨媽道:“咳,我倒忘了。”走過來說道:“恕我健忘,回來叫寶琴過來拜姐姐的壽。”黛玉笑說“不敢”。大家坐了。那黛玉留神一看,獨不見寶釵,便問道:“寶姐姐可好麼?爲什麼不過來?”薛姨媽道:“他原該來的,只因無人看家,所以不來。”黛玉紅着臉微笑道:“姨媽那裏又添了大嫂子,怎麼倒用寶姐姐看起家來?大約是他怕人多熱鬧,懶待來罷。我倒怪想他的。”薛姨媽笑道:“難得你惦記他。他也常想你們姊妹們,過一天我叫他來,大家敘敘。”
\end{parag}


\begin{parag}
    說着,丫頭們下來斟酒上菜,外面已開戲了。出場自然是一兩出吉慶戲文,乃至第三齣,只見金童玉女,旗幡寶幢,引着一個霓裳羽衣的小旦,頭上披着一條黑帕,唱了一回兒進去了。衆皆不識,聽見外面人說:“這是新打的《蕊珠記》裏的《冥升》。小旦扮的是嫦娥,前因墮落人寰,幾乎給人爲配,幸虧觀音點化,他就未嫁而逝,此時升引月宮。不聽見曲裏頭唱的‘人間只道風情好,那知道秋月春花容易拋,幾乎不把廣寒宮忘卻了!’”第四齣是《喫糠》,第五齣是達摩帶着徒弟過江回去,正扮出些海市蜃樓,好不熱鬧。
\end{parag}


\begin{parag}
    衆人正在高興時,忽見薛家的人滿頭汗闖進來,向薛蝌說道:“二爺快回去,並裏頭回明太太也請速回去,家中有要事。”薛蝌道:“什麼事?”家人道:“家去說罷。”薛蝌也不及告辭就走了。薛姨媽見裏頭丫頭傳進話去,更駭得面如土色,即忙起身,帶着寶琴,別了一聲,即刻上車回去了。弄得內外愕然。賈母道:“咱們這裏打發人跟過去聽聽,到底是什麼事,大家都關切的。”衆人答應了個“是”。不說賈府依舊唱戲,單說薛姨媽回去,只見有兩個衙役站在二門口,幾個當鋪裏夥計陪着,說:“太太回來自有道理。”正說着,薛姨媽已進來了。那衙役們見跟從着許多男婦簇擁着一位老太太,便知是薛蟠之母。看見這個勢派,也不敢怎麼,只得垂手侍立,讓薛姨媽進去了。
\end{parag}


\begin{parag}
    那薛姨媽走到廳房後面,早聽見有人大哭,卻是金桂。薛姨媽趕忙走來,只見寶釵迎出來,滿面淚痕,見了薛姨媽,便道:“媽媽聽了先彆着急,辦事要緊。”薛姨媽同着寶釵進了屋子,因爲頭裏進門時已經走着聽見家人說了,嚇的戰戰兢兢的了,一面哭着,因問:“到底是和誰?”只見家人回道:“太太此時且不必問那些底細,憑他是誰,打死了總是要償命的,且商量怎麼辦纔好。”薛姨媽哭着出來道:“還有什麼商議?”家人道:“依小的們的主見,今夜打點銀兩同着二爺趕去和大爺見了面,就在那裏訪一個有斟酌的刀筆先生,許他些銀子,先把死罪撕擄開,回來再求賈府去上司衙門說情。還有外面的衙役,太太先拿出幾兩銀子來打發了他們。我們好趕着辦事。”薛姨媽道:“你們找着那家子,許他發送銀子,再給他些養濟銀子,原告不追,事情就緩了。”寶釵在簾內說道:“媽媽,使不得。這些事越給錢越鬧的兇,倒是剛纔小廝說的話是。”薛姨媽又哭道:“我也不要命了,趕到那裏見他一面,同他死在一處就完了。”寶釵急的一面勸,一面在簾子裏叫人”快同二爺辦去罷。”丫頭們攙進薛姨媽來。薛蝌才往外走,寶釵道:“有什麼信打發人即刻寄了來,你們只管在外頭照料。”薛蝌答應着去了。這寶釵方勸薛姨媽,那裏金桂趁空兒抓住香菱,又和他嚷道:“平常你們只管誇他們家裏打死了人一點事也沒有,就進京來了的,如今攛掇的真打死人了。平日裏只講有錢有勢有好親戚,這時侯我看着也是唬的慌手慌腳的了。大爺明兒有個好歹兒不能回來時,你們各自幹你們的去了,撂下我一個人受罪!”說着,又大哭起來。這裏薛姨媽聽見,越發氣的發昏。寶釵急的沒法。正鬧着,只見賈府中王夫人早打發大丫頭過來打聽來了。寶釵雖心知自己是賈府的人了,一則尚未提明,二則事急之時,只得向那大丫頭道:“此時事情頭尾尚未明白,就只聽見說我哥哥在外頭打死了人被縣裏拿了去了,也不知怎麼定罪呢。剛纔二爺纔去打聽去了,一半日得了準信,趕着就給那邊太太送信去。你先回去道謝太太惦記着,底下我們還有多少仰仗那邊爺們的地方呢。”那丫頭答應着去了。薛姨媽和寶釵在家抓摸不着。
\end{parag}


\begin{parag}
    過了兩日,只見小廝回來,拿了一封書交給小丫頭拿進來。寶釵拆開看時,書內寫着:
\end{parag}


\begin{qute2sp}
    大哥人命是誤傷,不是故殺。今早用蝌出名補了一張呈紙進去,尚未批出。大哥前頭口供甚是不好,待此紙批准後再錄一堂,能夠翻供得好,便可得生了。快向當鋪內再取銀五百兩來使用。千萬莫遲。並請太太放心。餘事問小廝。
\end{qute2sp}


\begin{parag}
    寶釵看了,一一念給薛姨媽聽了。薛姨媽拭着眼淚說道:“這麼看起來,竟是死活不定了。”寶釵道:“媽媽先別傷心,等着叫進小廝來問明瞭再說。”一面打發小丫頭把小廝叫進來。薛姨媽便問小廝道:“你把大爺的事細說與我聽聽。”小廝道:“我那一天晚上聽見大爺和二爺說的,把我唬糊塗了。”未知小廝說出什麼話來,下回分解。
\end{parag}