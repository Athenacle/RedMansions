\chap{一百一十一}{鸳鸯女殉主登太虚 狗彘奴欺天招伙盗}



\begin{parag}
    话说凤姐听了小丫头的话,又气又急又伤心,不觉吐了一口血,便昏晕过去,坐在地下。平儿急来靠着,忙叫了人来搀扶着,慢慢的送到自己房中,将凤姐轻轻的安放在炕上,立刻叫小红斟上一杯开水送到凤姐唇边。凤姐呷了一口,昏迷仍睡。秋桐过来略瞧了一瞧,却便走开,平儿也不叫他。只见丰儿在旁站着,平儿叫他快快的去回明白了二奶奶吐血发晕不能照应的话,告诉了邢王二夫人。邢夫人打谅凤姐推病藏躲,因这时女亲在内不少,也不好说别的,心里却不全信,只说:“叫他歇着去罢。”众人也并无言语。只说这晚人客来往不绝,幸得几个内亲照应。家下人等见凤姐不在,也有偷闲歇力的,乱乱吵吵,已闹的七颠八倒,不成事体了。到二更多天远客去后,便预备辞灵。孝幕内的女眷大家都哭了一阵。只见鸳鸯已哭的昏晕过去了,大家扶住捶闹了一阵才醒过来,便说“老太太疼我一场我跟了去”的话。众人都打谅人到悲哭俱有这些言语,也不理会。到了辞灵之时,上上下下也有百十余人,只鸳鸯不在。众人忙乱之时,谁去捡点。到了琥珀等一干的人哭奠之时,却不见鸳鸯,想来是他哭乏了,暂在别处歇着,也不言语。辞灵以后,外头贾政叫了贾琏问明送殡的事,便商量着派人看家。贾琏回说:“上人里头派了芸儿在家照应,不必送殡,下人里头派了林之孝的一家子照应拆棚等事。但不知里头派谁看家?”贾政道:“听见你母亲说是你媳妇病了不能去,就叫他在家的。你珍大嫂子又说你媳妇病得利害,还叫四丫头陪着,带领了几个丫头婆子照看上屋里才好。”贾琏听了,心想:“珍大嫂子与四丫头两个不合,所以撺掇着不叫他去,若是上头就是他照应,也是不中用的。我们那一个又病着,也难照应。”想了一回,回贾政道:“老爷且歇歇儿,等进去商量定了再回。”贾政点了点头,贾琏便进去了。
\end{parag}


\begin{parag}
    谁知此时鸳鸯哭了一场,想到“自己跟着老太太一辈子,身子也没有着落。如今大老爷虽不在家,大太太的这样行为我也瞧不上。老爷是不管事的人,以后便乱世为王起来了,我们这些人不是要叫他们掇弄了么。谁收在屋子里,谁配小子,我是受不得这样折磨的,倒不如死了干净。但是一时怎么样的个死法呢?”一面想,一面走回老太太的套间屋内。刚跨进门,只见灯光惨淡,隐隐有个女人拿着汗巾子好似要上吊的样子。鸳鸯也不惊怕,心里想道:“这一个是谁?和我的心事一样,倒比我走在头里了。”便问道:“你是谁?咱们两个人是一样的心,要死一块儿死。”那个人也不答言。鸳鸯走到跟前一看,并不是这屋子的丫头,仔细一看,觉得冷气侵人时就不见了。鸳鸯呆了一呆,退出在炕沿上坐下,细细一想道:“哦,是了,这是东府里的小蓉大奶奶啊!他早死了的了,怎么到这里来?必是来叫我来了。他怎么又上吊呢?”想了一想道:“是了,必是教给我死的法儿。”鸳鸯这么一想,邪侵入骨,便站起来,一面哭,一面开了妆匣,取出那年绞的一绺头发,揣在怀里,就在身上解下一条汗巾,按着秦氏方才比的地方拴上。自己又哭了一回,听见外头人客散去,恐有人进来,急忙关上屋门,然后端了一个脚凳自己站上,把汗巾拴上扣儿套在咽喉,便把脚凳蹬开。可怜咽喉气绝,香魂出窍,正无投奔,只见秦氏隐隐在前,鸳鸯的魂魄疾忙赶上说道:“蓉大奶奶,你等等我。”那个人道:“我并不是什么蓉大奶奶,乃警幻之妹可卿是也。”鸳鸯道:“你明明是蓉大奶奶,怎么说不是呢?”那人道:“这也有个缘故,待我告诉你,你自然明白了。我在警幻宫中原是个钟情的首坐,管的是风情月债,降临尘世,自当为第一情人,引这些痴情怨女早早归入情司,所以该当悬粱自尽的。因我看破凡情,超出情海,归入情天,所以太虚幻境痴情一司竟自无人掌管。今警幻仙子已经将你补入,替我掌管此司,所以命我来引你前去的。”鸳鸯的魂道:“我是个最无情的,怎么算我是个有情的人呢?”那人道:“你还不知道呢。世人都把那淫欲之事当作‘情’字,所以作出伤风败化的事来,还自谓风月多情,无关紧要。不知‘情’之一字,喜怒哀乐未发之时便是个性,喜怒哀乐已发便是情了。至于你我这个情,正是未发之情,就如那花的含苞一样,欲待发泄出来,这情就不为真情了。”鸳鸯的魂听了点头会意,便跟了秦氏可卿而去。
\end{parag}


\begin{parag}
    这里琥珀辞了灵,听邢王二夫人分派看家的人,想着去问鸳鸯明日怎样坐车的,在贾母的外间屋里找了一遍不见,便找到套间里头。刚到门口,见门儿掩着,从门缝里望里看时,只见灯光半明不灭的,影影绰绰,心里害怕,又不听见屋里有什么动静,便走回来说道:“这蹄子跑到那里去了?”劈头见了珍珠,说:“你见鸳鸯姐姐来着没有?”珍珠道:“我也找他,太太们等他说话呢。必在套间里睡着了罢。”琥珀道:“我瞧了,屋里没有。那灯也没人夹蜡花儿,漆黑怪怕的,我没进去。如今咱们一块儿进去瞧,看有没有。”琥珀等进去正夹蜡花,珍珠说:“谁把脚凳撂在这里,几乎绊我一跤。”说着往上一瞧,唬的嗳哟一声,身子往后一仰,咕咚的栽在琥珀身上。琥珀也看见了,便大嚷起来,只是两只脚挪不动。
\end{parag}


\begin{parag}
    外头的人也都听见了,跑进来一瞧,大家嚷着报与邢王二夫人知道。王夫人宝钗等听了,都哭着去瞧。邢夫人道:“我不料鸳鸯倒有这样志气,快叫人去告诉老爷。”只有宝玉听见此信,便唬的双眼直竖。袭人等慌忙扶着,说道:“你要哭就哭,别憋着气。”宝玉死命的才哭出来了,心想“鸳鸯这样一个人偏又这样死法,”又想“实在天地间的灵气独钟在这些女子身上了。他算得了死所,我们究竟是一件浊物,还是老太太的儿孙,谁能赶得上他。”复又喜欢起来。那时宝钗听见宝玉大哭,也出来了,及到跟前,见他又笑。袭人等忙说:“不好了,又要疯了。”宝钗道:“不妨事,他有他的意思。”宝玉听了,更喜欢宝钗的话,“倒是他还知道我的心,别人那里知道。”正在胡思乱想,贾政等进来,着实的嗟叹着,说道:“好孩子,不枉老太太疼他一场!”即命贾琏出去吩咐人连夜买棺盛殓,“明日便跟着老太太的殡送出,也停在老太太棺后,全了他的心志。”贾琏答应出去。这里命人将鸳鸯放下,停放里间屋内。平儿也知道了,过来同袭人莺儿等一干人都哭的哀哀欲绝。内中紫鹃也想起自己终身一无着落,“恨不跟了林姑娘去,又全了主仆的恩义,又得了死所。如今空悬在宝玉屋内,虽说宝玉仍是柔情蜜意,究竟算不得什么?”于是更哭得哀切。
\end{parag}


\begin{parag}
    王夫人即传了鸳鸯的嫂子进来,叫他看着入殓。逐与邢夫人商量了,在老太太项内赏了他嫂子一百两银子,还说等闲了将鸳鸯所有的东西俱赏他们。他嫂子磕了头出去,反喜欢说:“真真的我们姑娘是个有志气的,有造化的,又得了好名声,又得了好发送。”旁边一个婆子说道:“罢呀嫂子,这会子你把一个活姑娘卖了一百银子便这么喜欢了,那时候儿给了大老爷,你还不知得多少银钱呢,你该更得意了。”一句话戳了他嫂子的心,便红了脸走开了。刚走到二门上,见林之孝带了人抬进棺材来了,他只得也跟进去帮着盛殓,假意哭嚎了几声。贾政因他为贾母而死,要了香来上了三炷,作了一个揖,说:“他是殉葬的人,不可作丫头论。你们小一辈都该行个礼。”宝玉听了,喜不自胜,走上来恭恭敬敬磕了几个头。贾琏想他素日的好处,也要上来行礼,被邢夫人说道:“有了一个爷们便罢了,不要折受他不得超生。”贾琏就不便过来了。宝钗听了,心中好不自在,便说道:“我原不该给他行礼,但只老太太去世,咱们都有未了之事,不敢胡为,他肯替咱们尽孝,咱们也该托托他好好的替咱们伏侍老太太西去,也少尽一点子心哪。”说着扶了莺儿走到灵前,一面奠酒,那眼泪早扑簌簌流下来了,奠毕拜了几拜,狠狠的哭了他一场。众人也有说宝玉的两口子都是傻子,也有说他两个心肠儿好的,也有说他知礼的。贾政反倒合了意。一面商量定了看家的仍是凤姐惜春,余者都遣去伴灵。一夜谁敢安眠,一到五更,听见外面齐人。到了辰初发引,贾政居长,衰麻哭泣,极尽孝子之礼。灵柩出了门,便有各家的路祭,一路上的风光不必细述。走了半日,来至铁槛寺安灵,所有孝男等俱应在庙伴宿,不题。
\end{parag}


\begin{parag}
    且说家中林之孝带领拆了棚,将门窗上好,打扫净了院子,派了巡更的人到晚打更上夜。只是荣府规例,一二更,三门掩上,男人便进不去了,里头只有女人们查夜。凤姐虽隔了一夜渐渐的神气清爽了些,只是那里动得。只有平儿同着惜春各处走了一走,咐吩了上夜的人,也便各自归房。却说周瑞的干儿子何三,去年贾珍管事之时,因他和鲍二打架,被贾珍打了一顿,撵在外头,终日在赌场过日。近知贾母死了,必有些事情领办,岂知探了几天的信,一些也没有想头,便嗳声叹气的回到赌场中,闷闷的坐下。那些人便说道:“老三,你怎么样?不下来捞本了么?”何三道:“倒想要捞一捞呢,就只没有钱么。”那些人道:“你到你们周大太爷那里去了几日,府里的钱你也不知弄了多少来,又来和我们装穷儿了。”何三道:“你们还说呢,他们的金银不知有几百万,只藏着不用。明儿留着不是火烧了就是贼偷了,他们才死心呢。”那些人道:“你又撒谎,他家抄了家,还有多少金银?”何三道:“你们还不知道呢,抄去的是撂不了的。如今老太太死还留了好些金银,他们一个也不使,都在老太太屋里搁着,等送了殡回来才分呢。”内中有一个人听在心里,掷了几骰,便说:“我输了几个钱,也不翻本儿了,睡去了。”说着,便走出来拉了何三道:“老三,我和你说句话。”何三跟他出来。那人道:“你这样一个伶俐人,这样穷,为你不服这口气。”何三道:“我命里穷,可有什么法儿呢。”那人道:“你才说荣府的银子这么多,为什么不去拿些使唤使唤?”何三道:“我的哥哥,他家的金银虽多,你我去白要一二钱他们给咱们吗!”那人笑道:“他不给咱们,咱们就不会拿吗!”何三听了这话里有话,便问道:“依你说怎么样拿呢?”那人道:“我说你没有本事,若是我,早拿了来了。”何三道:“你有什么本事?”那人便轻轻的说道:“你若要发财,你就引个头儿。我有好些朋友都是通天的本事,不要说他们送殡去了,家里剩下几个女人,就让有多少男人也不怕。只怕你没这么大胆子罢咧。”何三道:“什么敢不敢!你打谅我怕那个干老子么,我是瞧着干妈的情儿上头才认他作干老子罢咧,他又算了人了!你刚才的话,就只怕弄不来倒招了饥荒。他们那个衙门不熟?别说拿不来,倘或拿了来也要闹出来的。”那人道:“这么说你的运气来了。我的朋友还有海边上的呢,现今都在这里看个风头,等个门路。若到了手,你我在这里也无益,不如大家下海去受用不好么?你若撂不下你干妈,咱们索性把你干妈也带了去,大家伙儿乐一乐好不好?”何三道:“老大,你别是醉了罢,这些话混说的什么。”说着,拉了那人走到一个僻静地方,两个人商量了一回,各人分头而去。暂且不题。
\end{parag}


\begin{parag}
    且说包勇自被贾政吆喝派去看园,贾母的事出来也忙了,不曾派他差使,他也不理会,总是自做自吃,闷来睡一觉,醒时便在园里耍刀弄棍,倒也无拘无束。那日贾母一早出殡,他虽知道,因没有派他差事,他任意闲游。只见一个女尼带了一个道婆来到园内腰门那里扣门,包勇走来说道:“女师父那里去?”道婆道:“今日听得老太太的事完了,不见四姑娘送殡,想必是在家看家。想他寂寞,我们师父来瞧他一瞧。”包勇道:“主子都不在家,园门是我看的,请你们回去罢。要来呢,等主子们回来了再来。”婆子道:“你是那里来的个黑炭头,也要管起我们的走动来了。”包勇道:“我嫌你们这些人,我不叫你们来,你们有什么法儿!”婆子生了气,嚷道:“这都是反了天的事了!连老太太在日还不能拦我们的来往走动呢,你是那里的这么个横强盗,这样没法没天的。我偏要打这里走!”说着,便把手在门环上狠狠的打了几下。妙玉已气的不言语,正要回身便走,不料里头看二门的婆子听见有人拌嘴似的,开门一看,见是妙玉,已经回身走去,明知必是包勇得罪了走了。近日婆子们都知道上头太太们四姑娘都亲近得很,恐他日后说出门上不放他进来,那时如何担得住,赶忙走来说:“不知师父来,我们开门迟了。我们四姑娘在家里还正想师父呢,快请回来。看园子的小子是个新来的,他不知咱们的事,回来回了太太,打他一顿撵出去就完了。”妙玉虽是听见,总不理他。那经得看腰门的婆子赶上再四央求,后来才说出怕自己担不是,几乎急的跪下,妙玉无奈,只得随了那婆子过来。包勇见这般光景,自然不好拦他,气得瞪眼叹气而回。
\end{parag}


\begin{parag}
    这里妙玉带了道婆走到惜春那里,道了恼,叙了些闲话。说起“在家看家,只好熬个几夜。但是二奶奶病着,一个人又闷又是害怕,能有一个人在这里我就放心。如今里头一个男人也没有,今儿你既光降,肯伴我一宵,咱们下棋说话儿,可使得么?”妙玉本自不肯,见惜春可怜,又提起下棋,一时高兴应了,打发道婆回去取了他的茶具衣褥,命侍儿送了过来,大家坐谈一夜。惜春欣幸异常,便命彩屏去开上年蠲的雨水,预备好茶。那妙玉自有茶具。那道婆去了不多一时,又来了个侍者,带了妙玉日用之物。惜春亲自烹茶。两人言语投机,说了半天,那时已是初更时候,彩屏放下棋枰,两人对弈。惜春连输两盘,妙玉又让了四个子儿,惜春方赢了半子。这时已到四更,天空地阔,万籁无声。妙玉道:“我到五更须得打坐一回,我自有人伏侍,你自去歇息。”惜春犹是不舍,见妙玉要自己养神,不便扭他。正要歇去,猛听得东边上屋内上夜的人一片声喊起,惜春那里的老婆子们也接着声嚷道:“了不得了!有了人了!”唬得惜春彩屏等心胆俱裂,听见外头上夜的男人便声喊起来。妙玉道:“不好了,必是这里有了贼了。”正说着,这里不敢开门,便掩了灯光。在窗户眼内往外一瞧,只是几个男人站在院内,唬得不敢作声,回身摆着手轻轻的爬下来说:“了不得,外头有几个大汉站着。”说犹未了,又听得房上响声不绝,便有外头上夜的人进来吆喝拿贼。一个人说道:“上屋里的东西都丢了,并不见人。东边有人去了,咱们到西边去。”惜春的老婆子听见有自己的人,便在外间屋里说道:“这里有好些人上了房了。”上夜的都道:“你瞧,这可不是吗。”大家一齐嚷起来。只听房上飞下好些瓦来,众人都不敢上前。正在没法,只听园门腰门一声大响,打进门来,见一个梢长大汉,手执木棍。众人唬得藏躲不及,听得那人喊说道:“不要跑了他们一个!你们都跟我来。”这些家人听了这话,越发唬得骨软筋酥,连跑也跑不动了。只见这人站在当地只管乱喊,家人中有一个眼尖些的看出来了,你道是谁,正是甄家荐来的包勇。这些家人不觉胆壮起来,便颤巍巍的说道:“有一个走了,有的在房上呢。”包勇便向地下一扑,耸身上房追赶那贼。这些贼人明知贾家无人,先在院内偷看惜春房内,见有个绝色女尼,便顿起淫心,又欺上屋俱是女人,且又畏惧,正要踹进门去,因听外面有人进来追赶,所以贼众上房。见人不多,还想抵挡,猛见一人上房赶来,那些贼见是一人,越发不理论了,便用短兵抵住。那经得包勇用力一棍打去,将贼打下房来。那些贼飞奔而逃,从园墙过去,包勇也在房上追捕。岂知园内早藏下了几个在那里接赃,已经接过好些,见贼伙跑回,大家举械保护,见追的只有一人,明欺寡不敌众,反倒迎上来。包勇一见,生气道:“这些毛贼!敢来和我鬪鬪!”那伙贼便说:“我们有一个伙计被他们打倒了,不知死活,咱们索性抢了他出来。”这里包勇闻声即打,那伙贼便抡起器械,四五个人围住包勇乱打起来。外头上夜的人也都仗着胆子,只顾赶了来。众贼见鬪他不过,只得跑了。包勇还要赶时,被一个箱子一绊,立定看时,心想东西未丢,众贼远逃,也不追赶。便叫众人将灯照着,地下只有几个空箱,叫人收拾,他便欲跑回上房。因路径不熟,走到凤姐那边,见里面灯烛辉煌,便问:“这里有贼没有?”里头的平儿战兢兢的说道:“这里也没开门,只听上屋叫喊说有贼呢。你到那里去罢。”包勇正摸不着路头,遥见上夜的人过来,才跟着一齐寻到上屋。见是门开户启,那些上夜的在那里啼哭。
\end{parag}


\begin{parag}
    一时贾芸林之孝都进来了,见是失盗。大家着急进内查点,老太太的房门大开,将灯一照,锁头拧折,进内一瞧,箱柜已开,便骂那些上夜女人道:“你们都是死人么!贼人进来你们不知道的么!”那些上夜的人啼哭着说道:“我们几个人轮更上夜,是管二三更的,我们都没有住脚前后走的。他们是四更五更,我们的下班儿。只听见他们喊起来,并不见一个人,赶着照看,不知什么时候把东西早已丢了。求爷们问管四五更的。”林之孝道:“你们个个要死,回来再说。咱们先到各处看去。”上夜的男人领着走到尤氏那边,门儿关紧,有几个接音说:“唬死我们了。”林之孝问道:“这里没有丢东西?”里头的人方开了门道:“这里没丢东西。”林之孝带着人走到惜春院内,只听得里面说道:“了不得了!唬死了姑娘了,醒醒儿罢。”林之孝便叫人开门,问是怎样了。里头婆子开门说:“贼在这里打仗,把姑娘都唬坏了,亏得妙师父和彩屏才将姑娘救醒。东西是没失。”林之孝道:“贼人怎么打仗?”上夜的男人说:“幸亏包大爷上了房把贼打跑了去了,还听见打倒一个人呢。”包勇道:“在园门那里呢。”贾芸等走到那边,果见一人躺在地下死了。细细一瞧,好象周瑞的干儿子。众人见了诧异,派一个人看守着,又派两个人照看前后门,俱仍旧关锁着。
\end{parag}


\begin{parag}
    林之孝便叫人开了门,报了营官,立刻到来查勘。踏察贼迹是从后夹道上屋的,到了西院房上,见那瓦破碎不堪,一直过了后园去了。众上夜的齐声说道:“这不是贼,是强盗。”营官着急道:“并非明火执杖,怎算是盗。”上夜的道:“我们赶贼,他在房上掷瓦,我们不能近前,幸亏我们家的姓包的上房打退。赶到园里,还有好几个贼竟与姓包的打仗,打不过姓包的才都跑了。”营官道:“可又来,若是强盗,倒打不过你们的人么。不用说了,你们快查清了东西,递了失单,我们报就是了。”
\end{parag}


\begin{parag}
    贾芸等又到上屋,已见凤姐扶病过来,惜春也来。贾芸请了凤姐的安,问了惜春的好。大家查看失物,因鸳鸯已死,琥珀等又送灵去了,那些东西都是老太太的,并没见数,只用封锁,如今打从那里查去。众人都说:“箱柜东西不少,如今一空,偷的时候不小,那些上夜的人管什么的!况且打死的贼是周瑞的干儿子,必是他们通同一气的。”凤姐听了,气的眼睛直瞪瞪的便说:“把那些上夜的女人都拴起来,交给营里审问。”众人叫苦连天,跪地哀求。不知怎生发放,并失去的物有无着落,下回分解。
\end{parag}