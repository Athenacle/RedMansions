\chap{二十七}{滴翠亭楊妃戲彩蝶 埋香冢飛燕泣殘紅}

\begin{parag}
    \begin{note}庚:《葬花吟》是大觀園諸豔之歸源小引,故用在踐花日諸豔畢集之期。踐花日不論其典與不典,只取其韻耳。\end{note}
\end{parag}


\begin{parag}
    話說林黛玉正自悲泣,忽聽院門響處,只見寶釵出來了,寶玉襲人一羣人送了出來。待要上去問著寶玉,又恐當著衆人問羞了寶玉不便,因而閃過一旁,讓寶釵去了,寶玉等進去關了門,方轉過來,猶望著門灑了幾點淚。\begin{note}庚側:四字閃煞顰兒也。\end{note}自覺無味,方轉身回來,無精打彩的卸了殘妝。
\end{parag}


\begin{parag}
    紫鵑雪雁素日知道林黛玉的情性:無事悶坐,不是愁眉,\begin{note}庚側:畫美人之祕訣。\end{note}便是長嘆,且好端端的不知爲了什麼,常常的便自淚道不幹的。\begin{note}庚側:補寫,卻是避繁文法。\end{note}先時還有人解勸,怕他思父母,想家鄉,受了委曲,只得用話寬慰解勸。誰知後來一年一月的竟常常的如此,\begin{note}甲側:補瀟湘館常文也。\end{note}把這個樣兒看慣,也都不理論了。所以也沒人理,由他去悶坐,\begin{note}庚側:所謂“久病牀前少孝子”是也。\end{note}只管睡覺去了。那林黛玉倚著牀欄杆,兩手抱著膝,\begin{note}甲側:畫美人祕訣。\end{note}眼睛含著淚,\begin{note}庚側:前批的畫美人祕訣,今竟畫出《金閨夜坐圖》來了。\end{note}好似木雕泥塑\begin{note}甲側:木是旃檀,泥是金沙方可。\end{note}的一般,直坐到二更多天方纔睡了。一宿無話。
\end{parag}


\begin{parag}
    至次日乃是四月二十六日,原來這日未時交芒種節。尚古風俗:凡交芒種節的這日,都要設擺各色禮物,祭餞花神,言芒種一過,便是夏日了,衆花皆卸,花神退位,\begin{note}庚側:無論事之有無,看去有理。\end{note}須要餞行。然閨中更興這件風俗,所以大觀園中之人都早起來了。那些女孩子們,或用花瓣柳枝編成轎馬的,或用綾錦紗羅疊成幹旄旌幢的,都用綵線繫了。每一顆樹上,每一枝花上,都繫了這些物事。滿園裏繡帶飄颻,花枝招展,\begin{note}甲側:數句大觀園景倍勝省親一回,在一園人俱得閒閒尋樂上看,彼時只有元春一人閒耳。\end{note}\begin{note}庚側:數句抵省親一回文字,反覺閒閒有趣有味的領略。\end{note}更兼這些人打扮得桃羞杏讓,燕妒鶯慚,\begin{note}甲側:桃、杏、燕、鶯是這樣用法。\end{note}一時也道不盡。
\end{parag}


\begin{parag}
    且說寶釵、迎春、探春、惜春、李紈、鳳姐\begin{note}庚眉:不寫鳳姐隨大衆一筆,見紅玉一段則認爲泛文矣。何一絲不漏若此。畸笏。\end{note}等並巧姐、大姐、香菱與衆丫鬟們在園內玩耍,獨不見林黛玉。迎春因說道:“林妹妹怎麼不見?好個懶丫頭!這會子還睡覺不成?”寶釵道:“你們等著,我去鬧了他來。”說著便丟下了衆人,一直往瀟湘館來。正走著,只見文官等十二個女孩子也來了,\begin{note}庚側:一人不漏。\end{note}上來問了好,說了一回閒話。寶釵回身指道:“他們都在那裏呢,你們找他們去罷。我叫林姑娘去就來。”說著便逶迤往瀟湘館來。\begin{note}甲側:安插一處,好寫一處,正一張口難說兩家話也。\end{note}忽然抬頭見寶玉進去了,寶釵便站住低頭想了想:寶玉和林黛玉是從小兒一處長大,他兄妹間多有不避嫌疑之處,嘲笑喜怒無常;\begin{note}庚側:道盡二玉連日事。\end{note}況且林黛玉素習猜忌,好弄小性兒的。此刻自己也跟了進去,一則寶玉不便,二則黛玉嫌疑。\begin{note}甲側:道盡黛玉每每小性,全不在寶釵身上。\end{note}罷了,倒是回來的妙。想畢抽身回來。
\end{parag}


\begin{parag}
    剛要尋別的姊妹去,忽見前面一雙玉色蝴蝶,大如團扇,一上一下迎風翩躚,十分有趣。寶釵意欲撲了來玩耍,遂向袖中取出扇子來,向草地下來撲。\begin{note}甲側:可是一味知書識禮女夫子行止?寫寶釵無不相宜。\end{note}只見那一雙蝴蝶忽起忽落,來來往往,穿花度柳,將欲過河去了。倒引的寶釵躡手躡腳的,一直跟到池中滴翠亭上,香汗淋漓,嬌喘細細。\begin{note}庚側:若玉兄在,必有許多張羅。\end{note}寶釵也無心撲了,\begin{note}庚側:原是無可無不可。\end{note}剛欲回來,只聽滴翠亭裏邊嘁嘁喳喳有人說話。\begin{note}甲側:無閒紙閒筆之文如此。\end{note}原來這亭子四面俱是遊廊曲橋,蓋造在池中水上,四面雕鏤槅子糊著紙。
\end{parag}


\begin{parag}
    寶釵在亭外聽見說話,便煞住腳往裏細聽,\begin{note}庚眉:這樁風流案,又一體寫法,甚當。己冬夜。\end{note}只聽說道:“你瞧瞧這手帕子,果然是你丟的那塊,你就拿著;要不是,就還芸二爺去。”又有一人說話:“可不是我那塊!拿來給我罷。”又聽道:“你拿什麼謝我呢?難道白尋了來不成。”又答道:“我既許了謝你,自然不哄你。”又聽說道:“我尋了來給你,自然謝我;但只是揀的人,你就不拿什麼謝他?”又回道:“你別胡說。他是個爺們家,揀了我的東西,自然該還的。我拿什麼謝他呢?”又聽說道:“你不謝他,我怎麼回他呢?況且他再三再四的和我說了,若沒謝的,不許我給你呢。”半晌,又聽答道:“也罷,拿我這個給他,算謝他的罷。──你要告訴別人呢?須說個誓來。”又聽說道:“我要告訴一個人,就長一個疔,日後不得好死!”又聽說道:“噯呀!咱們只顧說話,看有人來悄悄在外頭聽見。\begin{note}庚側:豈敢。\end{note}\begin{note}庚眉:這是自難自法,好極好極!慣用險筆如此。壬午夏,雨窗。\end{note}不如把這槅子都推開了,\begin{note}庚側:賊起飛志,不假。\end{note}便是有人見咱們在這裏,他們只當我們說頑話呢。若走到跟前,咱們也看的見,就別說了。”
\end{parag}


\begin{parag}
    寶釵在外面聽見這話,心中喫驚,\begin{note}甲側:四字寫寶釵守身如此。\end{note}想道:“怪道從古至今那些姦淫狗盜的人,心機都不錯。\begin{note}庚側:道盡矣。\end{note}這一開了,見我在這裏,他們豈不臊了。況才說話的語音,大似寶玉房裏的紅兒的言語。他素昔眼空心大,是個頭等刁鑽古怪東西。今兒我聽了他的短兒,一時人急造反,狗急跳牆,不但生事,而且我還沒趣。如今便趕著躲了,料也躲不及,少不得要使個‘金蟬脫殼’的法子。”猶未想完,只聽“咯吱”一聲,寶釵便故意放重了腳步,\begin{note}庚側:閨中弱女機變,如此之便,如此之急。\end{note}笑著叫道:“顰兒,我看你往那裏藏!”一面說,一面故意往前趕。那亭內的紅玉墜兒剛一推窗,只聽寶釵如此說著往前趕,\begin{note}庚眉:此句實借紅玉反寫寶釵也,勿得認錯作者章法。\end{note}兩個人都唬怔了。寶釵反向他二人笑道:“你們把林姑娘藏在那裏了?”\begin{note}庚側:像極!好煞,妙煞!焉的不拍案叫絕!\end{note}墜兒道:“何曾見林姑娘了。”寶釵道:“我纔在河那邊看著林姑娘在這裏蹲著弄水兒的。我要悄悄的唬他一跳,還沒有走到跟前,他倒看見我了,朝東一繞就不見了。別是藏在這裏頭了。”\begin{note}庚側:像極!是極!\end{note}一面說,一面故意進去尋了一尋,抽身就走,口內說道: “一定是又鑽在山子洞裏去了。遇見蛇,咬一口也罷了。”一面說一面走,心中又好笑:\begin{note}甲側:真弄嬰兒,輕便如此,即餘至此亦要發笑。\end{note}這件事算遮過去了,不知他二人是怎樣。
\end{parag}


\begin{parag}
    誰知紅玉聽了寶釵的話,便信以爲真,\begin{note}甲側:寶釵身份。\end{note}\begin{note}庚側:實有這一句的。\end{note}讓寶釵去遠,便拉墜兒道:“了不得了!林姑娘蹲在這裏,一定聽了話去了!”\begin{note}庚側:移東挪西,任意寫去,卻是真有的。\end{note}墜兒聽說,也半日不言語。紅玉又道:“這可怎麼樣呢?”\begin{note}甲側:二句系黛玉身份。\end{note}墜兒道:“便是聽了,管誰筋疼,各人幹各人的就完了。”\begin{note}庚側:勉強話。\end{note}紅玉道:“若是寶姑娘聽見,還倒罷了。林姑娘嘴裏又愛刻薄人,心裏又細,他一聽見了,倘或走露了風聲,怎麼樣呢?”二人正說著,只見文官、香菱、司棋、侍書等上亭子來了。二人只得掩住這話,且和他們頑笑。
\end{parag}


\begin{parag}
    只見鳳姐兒站在山坡上招手叫,紅玉連忙棄了衆人,跑至鳳姐前,笑問:“奶奶使喚作什麼?”鳳姐打諒了一打諒,見他生的乾淨俏麗,說話知趣,因笑道: “我的丫頭今兒沒跟進來。我這會子想起一件事來,使喚個人出去,可不知你能幹不能幹,說的齊全不齊全?”紅玉笑道:“奶奶有什麼話,只管吩咐我說去。若說不齊全,誤了奶奶的事,憑奶奶責罰罷了。”\begin{note}甲側:操必勝之券。紅兒機括志量,自知能應阿鳳使令意。\end{note}鳳姐笑道:“你是那位小姐房裏的?\begin{note}庚側:反如此問。\end{note}我使出去,他回來找你,我好替你答應。”\begin{note}庚側:問那小姐爲此。\end{note}紅玉道:“我是寶二爺房裏的。”鳳姐聽了笑道:“噯喲!你原來是寶玉房裏的,怪道呢,\begin{note}甲側:“哎喲”“怪道”四字,一是玉兄手下無能爲者。前文打量生的“乾淨俏麗”四字,合而觀之,小紅則活現於紙上矣。\end{note}\begin{note}庚側:誇讚語也。\end{note}也罷了。你到我們家,告訴你平姐姐:外頭屋裏桌子上汝窯盤子架兒底下放著一卷銀子,那是一百六十兩,給繡匠的工價,等張材家的來要,當面稱給他瞧了,再給他拿去。\begin{note}庚側:一件。\end{note}再裏頭牀頭間有一個小荷包拿了來。”\begin{note}庚側:二件。\end{note}
\end{parag}


\begin{parag}
    紅玉聽說撤身去了,回來只見鳳姐不在這山坡子上了。因見司棋從山洞裏出來,站著系裙子,\begin{note}庚側:小點綴。一笑。\end{note}便趕上來問道:“姐姐,不知道二奶奶往那裏去了?”司棋道:“沒理論。”\begin{note}庚側:妙極!\end{note}紅玉聽了,抽身又往四下裏一看,只見那邊探春寶釵在池邊看魚。紅玉上來陪笑問道:“姑娘們可知道二奶奶那去了?”探春道:“往你大奶奶院裏找去。”紅玉聽了,才往稻香村來,頂頭只見\begin{note}庚側:又一折。\end{note} 晴雯、綺霰、碧痕、紫綃、麝月、侍書、入畫、鶯兒等一羣人來了。晴雯一見了紅玉,便說道:“你只是瘋罷!院子裏花兒也不澆,雀兒也不喂,茶爐子也不爖,就在外頭逛。”\begin{note}庚側:必有此數句,方引出稱心得意之語來。再不用本院人見小紅,此差只幾分遂心。\end{note} 紅玉道:“昨兒二爺說了,今兒不用澆花,過一日澆一回罷。我喂雀兒的時侯,姐姐還睡覺呢。” 碧痕道:“茶爐子呢?”\begin{note}甲側:岔一人問,俱是不受用意。\end{note}紅玉道:“今兒不該我爖的班兒,有茶沒茶別問我。”綺霰道:“你聽聽他的嘴!你們別說了,讓他逛去罷。”紅玉道:“你們再問問我逛了沒有。二奶奶使喚我說話取東西的。”\begin{note}甲側:非小紅誇耀,系爾等逼出來的,離怡紅意已定矣。\end{note}說著將荷包舉給他們看,\begin{note}庚側:得意!稱心如意,在此一舉荷包。\end{note}方沒言語了, \begin{note}甲側:衆女兒何苦自討之。\end{note}大家分路走開。晴雯冷笑道:“怪道呢!原來爬上高枝兒去了,把我們不放在眼裏。不知說了一句話半句話,名兒姓兒知道了不曾呢,就把他興的這樣!這一遭半遭兒的算不得什麼,過了後兒還得聽呵!有本事從今兒出了這園子,長長遠遠的在高枝兒上纔算得。”\begin{note}庚側:雖是醋語,卻與下無痕。\end{note}一面說著去了。
\end{parag}


\begin{parag}
    這裏紅玉聽說,不便分證,只得忍著氣來找鳳姐兒。到了李氏房中,果見鳳姐兒在這裏和李氏說話兒呢。紅玉上來回道:“平姐姐說,奶奶剛出來了,他就把銀子收了起來,\begin{note}甲側:交代不在盤架下了。\end{note}才張材家的來討,當面稱了給他拿去了。”說著將荷包遞了上去,\begin{note}庚側:兩件完了。\end{note}又道:“平姐姐教我回奶奶:才旺兒進來討奶奶的示下,好往那家子去。平姐姐就把那話按著奶奶的主意打發他去了。”鳳姐笑道:“他怎麼按我的主意打發去了?”\begin{note}甲側:可知前紅玉雲“就把那按奶奶的主意”是欲儉,但恐累贅耳,故阿鳳有是問,彼能細答。\end{note}紅玉道:“平姐姐說:我們奶奶問這裏奶奶好。原是我們二爺不在家,雖然遲了兩天,只管請奶奶放心。等五奶奶\begin{note}甲側:又一門。\end{note}好些,我們奶奶還會了五奶奶來瞧奶奶呢。五奶奶前兒打發了人來說,舅奶奶\begin{note}甲側:又一門。\end{note}帶了信來了,問奶奶好,還要和這裏的姑奶奶尋兩丸延年神驗萬全丹。若有了,奶奶\begin{note}甲側:又一門。\end{note}打發人來,只管送在我們奶奶這裏。明兒有人去,就順路給那邊舅奶奶帶去的。”
\end{parag}


\begin{parag}
    話未說完,\begin{note}庚側:又一潤色。\end{note}李氏道:“噯喲!\begin{note}甲側:紅玉今日方遂心如意,卻爲寶玉後伏線。\end{note}這些話我就不懂了。什麼‘奶奶’‘爺爺’的一大堆。”鳳姐笑道:“怨不得你不懂,這是四五門子的話呢。”說著又向紅玉笑道:“好孩子,難爲你說的齊全。別像他們扭扭捏捏的蚊子似的。\begin{note}庚側:寫死假斯文。\end{note}嫂子不知道,如今除了我隨手使的幾個人之外,我就怕和人說話。他們必定把一句話拉長了作兩三截兒,咬文咬字,拿著腔兒,哼哼唧唧的,急的我冒火,他們那裏知道!先時我們平兒也是這麼著,我就問著他:難道必定裝蚊子哼哼就是美人了?\begin{note}庚側:貶殺,罵殺。\end{note}說了幾遭纔好些兒了。”李宮裁笑道: “都像你潑皮破落戶纔好。”鳳姐又道:“這一個丫頭就好。\begin{note}甲側:紅玉聽見了嗎?\end{note}方纔兩遭,說話雖不多,聽那口聲就簡斷。”\begin{note}甲側:紅玉此刻心內想:可惜晴雯等不在傍。\end{note}說著又向紅玉笑道:“你明兒伏侍我去罷。我認你作女兒,我一調理你就出息了。”\begin{note}庚側:不假。\end{note}
\end{parag}


\begin{parag}
    紅玉聽了,撲哧一笑。鳳姐道:“你怎麼笑?你說我年輕,比你能大幾歲,就作你的媽了?你別作春夢呢!你打聽打聽,這些人頭比你大的大的,趕著我叫媽,我還不理。今兒抬舉了你呢!”紅玉笑道:“我不是笑這個,我笑奶奶認錯了輩數了。我媽是奶奶的女兒,\begin{note}庚側:所以說“比你大的大的”。\end{note}這會子又認我作女兒。”鳳姐道:“誰是你媽?”\begin{note}庚側:晴雯說過。\end{note}李宮裁笑道:“你原來不認得他?他是林之孝之女。”\begin{note}甲側:管家之女,而晴卿輩擠之,招禍之媒也。\end{note}鳳姐聽了十分詫異,說道:“哦!原來是他的丫頭。”\begin{note}甲側:傳神。\end{note}又笑道:“林之孝兩口子都是錐子扎不出一聲兒來的。我成日家說,他們倒是配就了的一對夫妻,一對天聾地啞。\begin{note}甲側:用的是阿鳳口角。\end{note}那裏承望養出這麼個伶俐丫頭來!你十幾歲了?”紅玉道:“十七歲了。”又問名字,\begin{note}甲側:真真不知名,可嘆!\end{note}紅玉道:“原叫紅玉的,因爲重了寶二爺,如今只叫紅兒了。”
\end{parag}


\begin{parag}
    鳳姐聽說將眉一皺,把頭一回,說道:“討人嫌的很!\begin{note}庚側:又一下針。\end{note}得了玉的益似的,你也玉,我也玉。”因說道:“既這麼著肯跟,我還和他媽說,‘賴大家的如今事多,也不知這府裏誰是誰,你替我好好的挑兩個丫頭我使’,他一般答應著。他饒不挑,倒把這女孩子送了別處去。難道跟我必定不好?”李氏笑道:“你可是又多心了。他進來在先,你說話在後,怎麼怨的他媽!”鳳姐道:“既這麼著,明兒我和寶玉說,叫他再要人,\begin{note}甲側:有悌弟之心。\end{note}叫這丫頭跟我去。可不知本人願意不願意?”\begin{note}甲側:總是追寫紅玉十分心事。\end{note}紅玉笑道:“願意不願意,我們也不敢說。\begin{note}甲側:好答!可知兩處俱是主見。\end{note}只是跟著奶奶,我們也學些眉眼高低,\begin{note}庚側:千願意萬願意之言。\end{note}出入上下,大小的事也得見識見識。”\begin{note}甲側:且系本心本意,“獄神廟”回內方見。\end{note}\begin{note}庚眉:奸邪婢豈是怡紅應答者,故即逐之。前良兒,後篆兒,便是確證。作者又不得有也。己冬夜。\end{note}\begin{note}庚眉:此係未見“抄沒”、“獄神廟”諸事,故有是批。丁亥夏。畸笏。\end{note}剛說著,只見王夫人的丫頭來請,\begin{note}庚側:截得真好。\end{note}鳳姐便辭了李宮裁去了。紅玉回怡紅院去,\begin{note}庚側:好,接得更好。\end{note}不在話下。
\end{parag}


\begin{parag}
    如今且說林黛玉因夜間失寐,次日起來遲了,聞得衆姊妹都在園中作餞花會,恐人笑他癡懶,連忙梳洗了出來。剛到了院中,只見寶玉進門來了,笑道:“好妹妹,你昨兒可告我了不曾?\begin{note}甲側:明知無是事,不得不作開談。\end{note}教我懸了一夜心。”\begin{note}庚側:並不爲告懸心。\end{note}林黛玉便回頭叫紫鵑道:\begin{note}甲側:不見寶玉,阿顰斷無此一段閒言,總在欲言不言難禁之意,了卻“情情”之正文也。\end{note}\begin{note}庚側:倒像不曾聽見的。\end{note}“把屋子收拾了,撂下一扇紗屜;看那大燕子回來,把簾子放下來,拿獅子倚住;燒了香就把爐罩上。”一面說一面又往外走。寶玉見他這樣,還認作是昨日中晌的事,\begin{note}甲側:畢真不錯。\end{note}那知晚間的這段公案,還打恭作揖的。林黛玉正眼也不看,各自出了院門,一直找別的姊妹去了。寶玉心中納悶,自己猜疑:看起這個光景來,不象是爲昨日的事;但只昨日我回來的晚了,又沒見他,再沒有衝撞了他的去處。\begin{note}庚側:畢真不錯。\end{note}一面想,一面由不得隨後追了來。
\end{parag}


\begin{parag}
    只見寶釵探春正在那邊看仙鶴,\begin{note}庚側:二玉文字豈是容易寫的,故有此截。\end{note}\begin{note}庚眉:《石頭記》用截法、岔法、突然法、伏線法、由近漸遠法、將繁改簡法、重作輕抹法、虛敲實應法種種諸法,總在人意料之外,且不曾見一絲牽強,所謂“信手拈來無不是”是也。\end{note}見黛玉來了,三個一同站著說話兒。又見寶玉來了,探春便笑道:“寶哥哥,身上好?我整整三天沒見了。”\begin{note}甲側:橫雲截嶺,好極,妙極!二玉文原不易寫,《石頭記》得力處在茲。\end{note}寶玉笑道: “妹妹身上好?我前兒還在大嫂子跟前問你呢。”探春道:“哥哥往這裏來,我和你說話。”\begin{note}庚側:是移一處語。\end{note}寶玉聽說,便跟了他來到一棵石榴樹下。探春因說道:“這幾天老爺可叫你沒有?”\begin{note}甲側:老爺叫寶玉再無喜事,故園中合宅皆知。\end{note}寶玉笑道:“沒有叫。”探春說:“昨兒我恍惚聽見說老爺叫你出去的。”寶玉笑道:“那想是別人聽錯了,並沒叫的。”\begin{note}甲側:非謊也,避繁也。\end{note}\begin{note}庚批:怕文繁。\end{note}探春又笑道:“這幾個月,我又攢下有十來吊錢了。你還拿了去,明兒出門逛去的時候,或是好字畫,好輕巧頑意兒,替我帶些來。”\begin{note}庚眉:若無此一岔,二玉和合則成嚼蠟文字。《石頭記》得力處正此。丁亥夏。畸笏叟。\end{note}寶玉道:“我這麼城裏城外、大廊小廟的逛,也沒見個新奇精緻東西,左不過是那些金玉銅磁沒處撂的古董,再就是綢緞喫食衣服了。”探春道:“誰要這些。怎麼像你上回買的那柳枝兒編的小籃子,整竹子根摳的香盒兒,泥垛的風爐兒,這就好了。我喜歡的什麼似的,誰知他們都愛上了,都當寶貝似的搶了去了。”寶玉笑道:“原來要這個。這不值什麼,拿五百錢出去給小子們,管拉一車來。”\begin{note}庚批:不知物理艱難,公子口氣也。\end{note}探春道:“小廝們知道什麼。你揀那樸而不俗、直而不拙者,\begin{note}甲側:是論物?是論人?看官著眼。\end{note}這些東西,你多多的替我帶了來。我還象上回的鞋作一雙你穿,比那一雙還加工夫,如何呢?”
\end{parag}


\begin{parag}
    寶玉笑道:“你提起鞋來,我想起個故事:那一回我穿著,可巧遇見了老爺,\begin{note}庚側:補遺法。\end{note}老爺就不受用,問是誰作的。我那裏敢提‘三妹妹’三個字,我就回說是前兒我生日,是舅母給的。老爺聽了是舅母給的,纔不好說什麼,半日還說:‘何苦來!虛耗人力,作踐綾羅,作這樣的東西。’我回來告訴了襲人,襲人說這還罷了,趙姨娘氣的抱怨的了不得:‘正經兄弟,\begin{note}庚側:指環哥。\end{note}鞋搭拉襪搭拉的\begin{note}甲側:何至如此,寫妒婦信口逗。\end{note}沒人看的見,且作這些東西!’”探春聽說,登時沉下臉來,道:“這話糊塗到什麼田地!怎麼我是該作鞋的人麼?環兒難道沒有分例的,沒有人的?一般的衣裳是衣裳,鞋襪是鞋襪,丫頭老婆一屋子,怎麼抱怨這些話!給誰聽呢!我不過是閒著沒事兒,作一雙半雙,愛給那個哥哥兄弟,隨我的心。誰敢管我不成!這也是白氣。”寶玉聽了,點頭笑道:“你不知道,他心裏自然又有個想頭了。”探春聽說,益發動了氣,將頭一扭,說道:“連你也糊塗了!他那想頭自然是有的,不過是那陰微鄙賤的見識。他只管這麼想,我只管認得老爺、太太兩個人,別人我一概不管。就是姊妹弟兄跟前,誰和我好,我就和誰好,什麼偏的庶的,我也不知道。論理我不該說他,但忒昏憒的不象了!還有笑話呢:\begin{note}甲側:開一步,妙妙!\end{note}就是上回我給你那錢,替我帶那頑的東西。過了兩天,他見了我,也是說沒錢使,怎麼難,我也不理論。誰知後來丫頭們出去了,他就抱怨起來,說我攢的錢爲什麼給你使,倒不給環兒使呢。我聽見這話,又好笑又好氣,我就出來往太太跟前去了。”\begin{note}庚眉:這一節特爲“興利除弊”一回伏線。\end{note}正說著,只見寶釵那邊笑道:\begin{note}庚側:截得好。\end{note}“說完了,來罷。顯見的是哥哥妹妹了,丟下別人,且說梯己去。我們聽一句兒就使不得了!”說著,探春寶玉二人方笑著來了。
\end{parag}


\begin{parag}
    寶玉因不見了林黛玉,\begin{note}甲側:兄妹話雖久長,心事總未少歇,接得好。\end{note}便知他躲了別處去了,想了一想,索性遲兩日,\begin{note}甲側:作書人調侃耶?\end{note}等他的氣消一消再去也罷了。因低頭看見許多鳳仙石榴等各色落花,錦重重的落了一地,\begin{note}庚眉:不因見落花,寶玉如何突至埋香冢?不至埋香冢,如何寫《葬花吟》?《石頭記》無閒文閒字正此。丁亥夏。畸笏叟。\end{note}因嘆道:“這是他心裏生了氣,也不收拾這花兒來了。待我送了去,明兒再問著他。”\begin{note}甲側:至埋香冢方不牽強,好情理。\end{note}說著,只見寶釵約著他們往外頭去。\begin{note}甲側:收拾的乾淨。\end{note}寶玉道:“我就來。”說畢,等他二人去遠了,\begin{note}甲側:怕人笑說。\end{note}便把那花兜了起來,登山渡水,過樹穿花,一直奔了那日同林黛玉葬桃花的去處來。將已到了花冢,\begin{note}庚側:新鮮。\end{note}猶未轉過山坡,只聽山坡那邊有嗚咽之聲,一行數落著,哭的好不傷感。\begin{note}甲側:奇文異文,俱出《石頭記》上,且愈出愈奇文。\end{note}寶玉心下想道:“這不知是那房裏的丫頭,受了委曲,\begin{note}甲側:岔開線絡,活潑之至!\end{note}跑到這個地方來哭。”一面想,一面煞住腳步,聽他哭道是:\begin{note}甲側:詩詞歌賦,如此章法寫於書上者乎?\end{note}\begin{note}庚側:詩詞文章,試問有如此行筆者乎?\end{note}\begin{note}甲眉:開生面,立新場,是書多多矣,惟此回處[更]生更新。非顰兒斷無是佳吟,非石兄斷無是情聆,難爲了作者了,故留數字以慰之。\end{note}\begin{note}庚眉:開生面,立新場,是書不止“紅樓夢”一回,惟是回更生更新,且讀去非阿顰無是且(佳)吟,非石兄斷無是章法行文,愧殺古今小說家也。畸笏。\end{note}
\end{parag}


\begin{poem}
    \begin{pl} 花謝花飛花滿天,紅消香斷有誰憐?遊絲軟系飄春榭,落絮輕沾撲繡簾。\end{pl}

    \begin{pl} 簾中女兒惜春暮,愁緒滿懷無處訴,手把花鋤出繡簾,忍踏落花來複去?\end{pl}

    \begin{pl} 柳絲榆莢自芳菲,不管桃飄與李飛。桃李明年能再發,明歲閨中知有誰?\end{pl}

    \begin{pl} 三月香巢已壘成,梁間燕子太無情!明年花發雖可啄,卻不道人去梁空巢也傾。\end{pl}

    \begin{pl} 一年三百六十日,風刀霜劍嚴相逼,明媚鮮妍能幾時?一朝飄泊難尋覓。\end{pl}

    \begin{pl} 花開易見落難尋,階前悶殺葬花人,獨倚花鋤淚暗灑,灑上花枝見血痕。\end{pl}

    \begin{pl} 杜鵑無語正黃昏,荷鋤歸去掩重門。青燈照壁人初睡,冷雨敲窗被未溫。\end{pl}

    \begin{pl} 怪奴底事倍傷神,半爲憐春半惱春。憐春忽至惱忽去,至又無言去不聞。\end{pl}

    \begin{pl} 昨宵庭外悲歌發,知是花魂與鳥魂。花魂鳥魂總難留,鳥自無言花自羞。\end{pl}

    \begin{pl} 願奴脅下生雙翼,隨花飛到天盡頭。天盡頭,何處有香丘?\end{pl}

    \begin{pl} 未若錦囊收豔骨,一抔淨土掩風流。質本潔來還潔去,強於污淖陷渠溝。\end{pl}

    \begin{pl} 爾今死去儂收葬,未卜儂身何日喪?儂今葬花人笑癡,他年葬儂知是誰?\end{pl}

    \begin{pl} 試看春殘花漸落,便是紅顏老死時。一朝春盡紅顏老,花落人亡兩不知!\end{pl}
\end{poem}


\begin{parag}
    寶玉不覺癡倒。要知端底,再看下回。
\end{parag}


\begin{parag}
    \begin{note}甲回尾:餘讀《葬花吟》,至再至三四,其悽楚感慨,令人身世兩忘,舉筆再四,不能下批。有客曰:“先生身非寶玉,何能下筆?即字字雙圈,批詞通仙,料難遂顰兒之意。俟看玉兄之後文再批。”噫唏!阻餘者想亦《石頭記》來的?故擲筆以待。\end{note}
\end{parag}


\begin{parag}
    \begin{note}甲:餞花辰不論典與不典,只取其韻致生趣耳。\end{note}
\end{parag}


\begin{parag}
    \begin{note}甲:池邊戲蝶,偶爾適興;亭外急智脫殼。明寫寶釵非拘拘然一女夫子。\end{note}
\end{parag}


\begin{parag}
    \begin{note}甲:鳳姐用小紅,可知晴雯等埋沒其人久矣,無怪有私心私情。且紅玉後有寶玉大得力處,此於千里外伏線也。\end{note}
\end{parag}


\begin{parag}
    \begin{note}甲:《石頭記》用截法、岔法、突然法、伏線法、由近漸遠法、將繁改簡法、重作輕抹法、虛敲實應法種種諸法,總在人意料之外,且不曾見一絲牽強,所謂“信手拈來無不是”是也。\end{note}
\end{parag}


\begin{parag}
    \begin{note}甲:不因見落花,寶玉如何突至埋香冢;不至埋香冢又如何寫《葬花吟》。\end{note}
\end{parag}


\begin{parag}
    \begin{note}甲:埋香冢葬花乃諸豔歸源,《葬花吟》又系諸豔一偈也。\end{note}
\end{parag}


\begin{parag}
    \begin{note}蒙回後總評:幸逢知己無迴避,審語歌窗怕有人。總是關心渾不了,叮嚀囑咐爲輕春。\end{note}
\end{parag}


\begin{parag}
    \begin{note}蒙回後總評:心事將誰告,花飛動我悲。埋香吟哭後,日日斂雙眉。\end{note}
\end{parag}

