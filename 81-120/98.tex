\chap{九十八}{苦絳珠魂歸離恨天 病神瑛淚灑相思地}



\begin{parag}
    話說寶玉見了賈政,回至房中,更覺頭昏腦悶,懶待動彈,連飯也沒喫,便昏沉睡去。仍舊延醫診治,服藥不效,索性連人也認不明白了。大家扶着他坐起來,還是象個好人。一連鬧了幾天,那日恰是回九之期,若不過去,薛姨媽臉上過不去,若說去呢,寶玉這般光景。賈母明知是爲黛玉而起,欲要告訴明白,又恐氣急生變。寶釵是新媳婦,又難勸慰,必得姨媽過來纔好。若不回九,姨媽嗔怪。便與王夫人鳳姐商議道:“我看寶玉竟是魂不守舍,起動是不怕的。用兩乘小轎叫人扶着從園裏過去,應了回九的吉期,以後請姨媽過來安慰寶釵,咱們一心一意的調治寶玉,可不兩全?”王夫人答應了,即刻預備。幸虧寶釵是新媳婦,寶玉是個瘋傻的,由人掇弄過去了。寶釵也明知其事,心裏只怨母親辦得糊塗,事已至此,不肯多言。獨有薛姨媽看見寶玉這般光景,心裏懊悔,只得草草完事。
\end{parag}


\begin{parag}
    到家,寶玉越加沉重,次日連起坐都不能了。日重一日,甚至湯水不進。薛姨媽等忙了手腳,各處遍請名醫,皆不識病源。只有城外破寺中住着個窮醫,姓畢,別號知庵的,診得病源是悲喜激射,冷暖失調,飲食失時,憂忿滯中,正氣壅閉:此內傷外感之症。於是度量用藥,至晚服了,二更後果然省些人事,便要水喝。賈母王夫人等才放了心,請了薛姨媽帶了寶釵都到賈母那裏暫且歇息。
\end{parag}


\begin{parag}
    寶玉片時清楚,自料難保,見諸人散後,房中只有襲人,因喚襲人至跟前,拉着手哭道:“我問你,寶姐姐怎麼來的?我記得老爺給我娶了林妹妹過來,怎麼被寶姐姐趕了去了?他爲什麼霸佔住在這裏?我要說呢,又恐怕得罪了他。你們聽見林妹妹哭得怎麼樣了?”襲人不敢明說,只得說道:“林姑娘病着呢。”寶玉又道:“我瞧瞧他去。”說着,要起來。豈知連日飲食不進,身子那能動轉,便哭道:“我要死了!我有一句心裏的話,只求你回明老太太:橫豎林妹妹也是要死的,我如今也不能保。兩處兩個病人都要死的,死了越發難張羅。不如騰一處空房子,趁早將我同林妹妹兩個抬在那裏,活着也好一處醫治伏侍,死了也好一處停放。你依我這話,不枉了幾年的情分。”襲人聽了這些話,便哭的哽嗓氣噎。寶釵恰好同了鶯兒過來,也聽見了,便說道:“你放着病不保養,何苦說這些不吉利的話。老太太才安慰了些,你又生出事來。老太太一生疼你一個,如今八十多歲的人了,雖不圖你的封誥,將來你成了人,老太太也看着樂一天,也不枉了老人家的苦心。太太更是不必說了,一生的心血精神,撫養了你這一個兒子,若是半途死了,太太將來怎麼樣呢。我雖是命薄,也不至於此。據此三件看來,你便要死,那天也不容你死的,所以你是不得死的。只管安穩着,養個四五天後,風邪散了,太和正氣一足,自然這些邪病都沒有了。”寶玉聽了,竟是無言可答,半晌方纔嘻嘻的笑道:“你是好些時不和我說話了,這會子說這些大道理的話給誰聽?”寶釵聽了這話,便又說道:“實告訴你說罷,那兩日你不知人事的時候,林妹妹已經亡故了。”寶玉忽然坐起來,大聲詫異道:“果真死了嗎?”寶釵道:“果真死了。豈有紅口白舌咒人死的呢。老太太,太太知道你姐妹和睦,你聽見他死了自然你也要死,所以不肯告訴你。”寶玉聽了,不禁放聲大哭,倒在牀上。
\end{parag}


\begin{parag}
    忽然眼前漆黑,辨不出方向,心中正自恍惚,只見眼前好象有人走來,寶玉茫然問道:“借問此是何處?”那人道:“此陰司泉路。你壽未終,何故至此?”寶玉道:“適聞有一故人已死,遂尋訪至此,不覺迷途。”那人道:“故人是誰?”寶玉道:“姑蘇林黛玉。”那人冷笑道:“林黛玉生不同人,死不同鬼,無魂無魄,何處尋訪!凡人魂魄,聚而成形,散而爲氣,生前聚之,死則散焉。常人尚無可尋訪,何況林黛玉呢。汝快回去罷。”寶玉聽了,呆了半晌道:“既雲死者散也,又如何有這個陰司呢?”那人冷笑道:“那陰司說有便有,說無就無。皆爲世俗溺於生死之說,設言以警世,便道上天深怒愚人,或不守分安常,或生祿未終自行夭折,或嗜淫慾尚氣逞兇無故自隕者,特設此地獄,囚其魂魄,受無邊的苦,以償生前之罪。汝尋黛玉,是無故自陷也。且黛玉已歸太虛幻境,汝若有心尋訪,潛心修養,自然有時相見。如不安生,即以自行夭折之罪囚禁陰司,除父母外,欲圖一見黛玉,終不能矣。”那人說畢,袖中取出一石,向寶玉心口擲來。寶玉聽了這話,又被這石子打着心窩,嚇的即欲回家,只恨迷了道路。正在躊躇,忽聽那邊有人喚他。回首看時,不是別人,正是賈母,王夫人,寶釵,襲人等圍繞哭泣叫着。自己仍舊躺在牀上。見案上紅燈,窗前皓月,依然錦鏽叢中,繁華世界。定神一想,原來竟是一場大夢。渾身冷汗,覺得心內清爽。仔細一想,真正無可奈何,不過長嘆數聲而已。寶釵早知黛玉已死,因賈母等不許衆人告訴寶玉知道,恐添病難治。自己卻深知寶玉之病實因黛玉而起,失玉次之,故趁勢說明,使其一痛決絕,神魂歸一,庶可療治。賈母王夫人等不知寶釵的用意,深怪他造次。後來見寶玉醒了過來,方纔放心。立即到外書房請了畢大夫進來診視。那大夫進來診了脈,便道:“奇怪,這回脈氣沉靜,神安鬱散,明日進調理的藥,就可以望好了。”說着出去。衆人各自安心散去。
\end{parag}


\begin{parag}
    襲人起初深怨寶釵不該告訴,惟是口中不好說出。鶯兒背地也說寶釵道:“姑娘忒性急了。”寶釵道:“你知道什麼!好歹橫豎有我呢。”那寶釵任人誹謗,並不介意,只窺察寶玉心病,暗下鍼砭。一日,寶玉漸覺神志安定,雖一時想起黛玉,尚有糊塗。更有襲人緩緩的將“老爺選定的寶姑娘爲人和厚,嫌林姑娘秉性古怪,原恐早夭,老太太恐你不知好歹,病中着急,所以叫雪雁過來哄你”的話時常勸解。寶玉終是心酸落淚。欲待尋死,又想着夢中之言,又恐老太太,太太生氣,又不能撩開。又想黛玉已死,寶釵又是第一等人物,方信金石姻緣有定,自己也解了好些。寶釵看來不妨大事,於是自己心也安了,只在賈母王夫人等前盡行過家庭之禮後,便設法以釋寶玉之憂。寶玉雖不能時常坐起,亦常見寶釵坐在牀前,禁不住生來舊病。寶釵每以正言勸解,以”養身要緊,你我既爲夫婦,豈在一時”之語安慰他。那寶玉心裏雖不順遂,無奈日裏賈母王夫人及薛姨媽等輪流相伴,夜間寶釵獨去安寢,賈母又派人服侍,只得安心靜養。又見寶釵舉動溫柔,也就漸漸的將愛慕黛玉的心腸略移在寶釵身上,此是後話。
\end{parag}


\begin{parag}
    卻說寶玉成家的那一日,黛玉白日已昏暈過去,卻心頭口中一絲微氣不斷,把個李紈和紫鵑哭的死去活來。到了晚間,黛玉去又緩過來了,微微睜開眼,似有要水要湯的光景。此時雪雁已去,只有紫鵑和李紈在旁。紫鵑便端了一盞桂圓湯和的梨汁,用小銀匙灌了兩三匙。黛玉閉着眼靜養了一會子,覺得心裏似明似暗的。此時李紈見黛玉略緩,明知是回光反照的光景,卻料着還有一半天耐頭,自己回到稻香村料理了一回事情。
\end{parag}


\begin{parag}
    這裏黛玉睜開眼一看,只有紫鵑和奶媽並幾個小丫頭在那裏,便一手攥了紫鵑的手,使着勁說道:“我是不中用的人了。你伏侍我幾年,我原指望咱們兩個總在一處。不想我。……”說着,又喘了一會子,閉了眼歇着。紫鵑見他攥着不肯鬆手,自己也不敢挪動,看他的光景比早半天好些,只當還可以迴轉,聽了這話,又寒了半截。半天,黛玉又說道:“妹妹,我這裏並沒親人。我的身子是乾淨的,你好歹叫他們送我回去。”說到這裏又閉了眼不言語了。那手卻漸漸緊了,喘成一處,只是出氣大入氣小,已經促疾的很了。
\end{parag}


\begin{parag}
    紫鵑忙了,連忙叫人請李紈,可巧探春來了。紫鵑見了,忙悄悄的說道:“三姑娘,瞧瞧林姑娘罷。”說着,淚如雨下。探春過來,摸了摸黛玉的手已經涼了,連目光也都散了。探春紫鵑正哭着叫人端水來給黛玉擦洗,李紈趕忙進來了。三個人才見了,不及說話。剛擦着,猛聽黛玉直聲叫道:“寶玉,寶玉,你好……”說到“好”字,便渾身冷汗,不作聲了。紫鵑等急忙扶住,那汗愈出,身子便漸漸的冷了。探春李紈叫人亂着攏頭穿衣,只見黛玉兩眼一翻,嗚呼,香魂一縷隨風散,愁緒三更入夢遙!
\end{parag}


\begin{parag}
    當時黛玉氣絕,正是寶玉娶寶釵的這個時辰。紫鵑等都大哭起來。李紈探春想他素日的可疼,今日更加可憐,也便傷心痛哭。因瀟湘館離新房子甚遠,所以那邊並沒聽見。一時大家痛哭了一陣,只聽得遠遠一陣音樂之聲,側耳一聽,卻又沒有了。探春李紈走出院外再聽時,惟有竹梢風動,月影移牆,好不淒涼冷淡!一時叫了林之孝家的過來,將黛玉停放畢,派人看守,等明早去回鳳姐。
\end{parag}


\begin{parag}
    鳳姐因見賈母王夫人等忙亂,賈政起身,又爲寶玉惛憒更甚,正在着急異常之時,若是又將黛玉的凶信一回,恐賈母王夫人愁苦交加,急出病來,只得親自到園。到了瀟湘館內,也不免哭了一場。見了李紈探春,知道諸事齊備,便說:“很好。只是剛纔你們爲什麼不言語,叫我着急?”探春道:“剛纔送老爺,怎麼說呢。”鳳姐道:“還倒是你們兩個可憐他些。這麼着,我還得那邊去招呼那個冤家呢。但是這件事好累墜,若是今日不回,使不得,若回了,恐怕老太太擱不住。”李紈道:“你去見機行事,得回再回方好。”鳳姐點頭,忙忙的去了。
\end{parag}


\begin{parag}
    鳳姐到了寶玉那裏,聽見大夫說不妨事,賈母王夫人略覺放心,鳳姐便背了寶玉,緩緩的將黛玉的事回明瞭。賈母王夫人聽得都唬了一大跳。賈母眼淚交流說道:“是我弄壞了他了。但只是這個丫頭也忒傻氣!”說着,便要到園裏去哭他一場,又惦記着寶玉,兩頭難顧。王夫人等含悲共勸賈母不必過去,“老太太身子要緊。”賈母無奈,只得叫王夫人自去。又說:“你替我告訴他的陰靈:‘並不是我忍心不來送你,只爲有個親疏。你是我的外孫女兒,是親的了,若與寶玉比起來,可是寶玉比你更親些。倘寶玉有些不好,我怎麼見他父親呢。’”說着,又哭起來。王夫人勸道:“林姑娘是老太太最疼的,但只壽夭有定。如今已經死了,無可盡心,只是葬禮上要上等的發送。一則可以少盡咱們的心,二則就是姑太太和外甥女兒的陰靈兒,也可以少安了。”賈母聽到這裏,越發痛哭起來。鳳姐恐怕老人家傷感太過,明仗着寶玉心中不甚明白,便偷偷的使人來撒個謊兒哄老太太道:“寶玉那裏找老太太呢。”賈母聽見,才止住淚問道:“不是又有什麼緣故?”鳳姐陪笑道:“沒什麼緣故,他大約是想老太太的意思。”賈母連忙扶了珍珠兒,鳳姐也跟着過來。
\end{parag}


\begin{parag}
    走至半路,正遇王夫人過來,一一回明瞭賈母。賈母自然又是哀痛的,只因要到寶玉那邊,只得忍淚含悲的說道:“既這麼着,我也不過去了。由你們辦罷,我看着心裏也難受,只別委屈了他就是了。”王夫人鳳姐一一答應了。賈母才過寶玉這邊來,見了寶玉,因問:“你做什麼找我?”寶玉笑道:“我昨日晚上看見林妹妹來了,他說要回南去。我想沒人留的住,還得老太太給我留一留他。”賈母聽着,說:“使得,只管放心罷。”襲人因扶寶玉躺下。
\end{parag}


\begin{parag}
    賈母出來到寶釵這邊來。那時寶釵尚未回九,所以每每見了人倒有些含羞之意。這一天見賈母滿面淚痕,遞了茶,賈母叫他坐下。寶釵側身陪着坐了,才問道:“聽得林妹妹病了,不知他可好些了?”賈母聽了這話,那眼淚止不住流下來,因說道:“我的兒,我告訴你,你可別告訴寶玉。都是因你林妹妹,才叫你受了多少委屈。你如今作媳婦了,我才告訴你。這如今你林妹妹沒了兩三天了,就是娶你的那個時辰死的。如今寶玉這一番病還是爲着這個,你們先都在園子裏,自然也都是明白的。”寶釵把臉飛紅了,想到黛玉之死,又不免落下淚來。賈母又說了一回話去了。自此寶釵千回萬轉,想了一個主意,只不肯造次,所以過了回九纔想出這個法子來。如今果然好些,然後大家說話纔不至似前留神。獨是寶玉雖然病勢一天好似一天,他的癡心總不能解,必要親去哭他一場。賈母等知他病未除根,不許他胡思亂想,怎奈他鬱悶難堪,病多反覆。倒是大夫看出心病,索性叫他開散了,再用藥調理,倒可好得快些。寶玉聽說,立刻要往瀟湘館來。賈母等只得叫人抬了竹椅子過來,扶寶玉坐上。賈母王夫人即便先行。到了瀟湘館內,一見黛玉靈柩,賈母已哭得淚乾氣絕。鳳姐等再三勸住。王夫人也哭了一場。李紈便請賈母王夫人在裏間歇着,猶自落淚。
\end{parag}


\begin{parag}
    寶玉一到,想起未病之先來到這裏,今日屋在人亡,不禁嚎啕大哭。想起從前何等親密,今日死別,怎不更加傷感。衆人原恐寶玉病後過哀,都來解勸,寶玉已經哭得死去活來,大家攙扶歇息。其餘隨來的,如寶釵,俱極痛哭。獨是寶玉必要叫紫鵑來見,問明姑娘臨死有何話說。紫鵑本來深恨寶玉,見如此,心裏已回過來些,又見賈母王夫人都在這裏,不敢灑落寶玉,便將林姑娘怎麼復病,怎麼燒燬帕子,焚化詩稿,並將臨死說的話,一一的都告訴了。寶玉又哭得氣噎喉幹。探春趁便又將黛玉臨終囑咐帶柩回南的話也說了一遍。賈母王夫人又哭起來。多虧鳳姐能言勸慰,略略止些,便請賈母等回去。寶玉那裏肯舍,無奈賈母逼着,只得勉強回房。
\end{parag}


\begin{parag}
    賈母有了年紀的人,打從寶玉病起,日夜不寧,今又大痛一陣,已覺頭暈身熱。雖是不放心惦着寶玉,卻也掙扎不住,回到自己房中睡下。王夫人更加心痛難禁,也便回去,派了彩雲幫着襲人照應,並說:“寶玉若再悲慼,速來告訴我們。”寶釵是知寶玉一時必不能捨,也不相勸,只用諷刺的話說他。寶玉倒恐寶釵多心,也便飲泣收心。歇了一夜,倒也安穩。明日一早,衆人都來瞧他,但覺氣虛身弱,心病倒覺去了幾分。於是加意調養,漸漸的好起來。賈母幸不成病,惟是王夫人心痛未痊。那日薛姨媽過來探望,看見寶玉精神略好,也就放心,暫且住下。
\end{parag}


\begin{parag}
    一日,賈母特請薛姨媽過去商量說:“寶玉的命都虧姨太太救的,如今想來不妨了,獨委屈了你的姑娘。如今寶玉調養百日,身體復舊,又過了姑娘的功服,正好圓房。要求姨太太作主,另擇個上好的吉日。”薛姨媽便道:“老太太主意很好,何必問我。寶丫頭雖生的粗笨,心裏卻還是極明白的。他的性情老太太素日是知道的。但願他們兩口兒言和意順,從此老太太也省好些心,我姐姐也安慰些,我也放了心了。老太太便定個日子。還通知親戚不用呢?”賈母道:“寶玉和你們姑娘生來第一件大事,況且費了多少周折,如今才得安逸,必要大家熱鬧幾天。親戚都要請的。一來酬願,二則咱們喫杯喜酒,也不枉我老人家操了好些心。”薛姨媽聽說,自然也是喜歡的,便將要辦妝奩的話也說了一番。賈母道:“咱們親上做親,我想也不必這些。若說動用的,他屋裏已經滿了。必定寶丫頭他心愛的要你幾件,姨太太就拿了來。我看寶丫頭也不是多心的人,不比的我那外孫女兒的脾氣,所以他不得長壽。”說着,連薛姨媽也便落淚。恰好鳳姐進來,笑道:“老太太姑媽又想着什麼了?”薛姨媽道:“我和老太太說起你林妹妹來,所以傷心。”鳳姐笑道:“老太太和姑媽且別傷心,我剛纔聽了個笑話兒來了,意思說給老太太和姑媽聽。”賈母拭了拭眼淚,微笑道:“你又不知要編派誰呢,你說來我和姨太太聽聽。說不笑我們可不依。”只見那鳳姐未從張口,先用兩隻手比着,笑彎了腰了。未知他說出些什麼來,下回分解。
\end{parag}