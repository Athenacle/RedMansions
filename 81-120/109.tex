\chap{一百零九}{候芳魂五儿承错爱 还孽债迎女返真元}



\begin{parag}
    话说宝钗叫袭人问出原故,恐宝玉悲伤成疾,便将黛玉临死的话与袭人假作闲谈,说是:“人生在世,有意有情,到了死后各自干各自的去了,并不是生前那样个人死后还是这样。活人虽有痴心,死的竟不知道。况且林姑娘既说仙去,他看凡人是个不堪的浊物,那里还肯混在世上。只是人自己疑心,所以招些邪魔外祟来缠扰了。”宝钗虽是与袭人说话,原说给宝玉听的。袭人会意,也说是“没有的事。若说林姑娘的魂灵儿还在园里,我们也算好的,怎么不曾梦见了一次。”宝玉在外闻听得,细细的想道:“果然也奇。我知道林妹妹死了,那一日不想几遍,怎么从没梦过。想是他到天上去了,瞧我这凡夫俗子不能交通神明,所以梦都没有一个儿。我就在外间睡着,或者我从园里回来,他知道我的实心,肯与我梦里一见。我必要问他实在那里去了,我也时常祭奠。若是果然不理我这浊物,竟无一梦,我便不想他了。”主意已定,便说:“我今夜就在外间睡了,你们也不用管我。”宝钗也不强他,只说:“你不要胡思乱想。你不瞧瞧,太太因你园里去了急得话都说不出来。若是知道还不保养身子,倘或老太太知道了,又说我们不用心。”宝玉道:“白这么说罢咧,我坐一会子就进来。你也乏了,先睡罢。”宝钗知他必进来的,假意说道:“我睡了,叫袭姑娘伺候你罢。”宝玉听了,正合机宜。候宝钗睡了,他便叫袭人麝月另铺设下一副被褥,常叫人进来瞧二奶奶睡着了没有。宝钗故意装睡,也是一夜不宁。那宝玉知是宝钗睡着,便与袭人道:“你们各自睡罢,我又不伤感。你若不信,你就伏侍我睡了再进去,只要不惊动我就是了。”袭人果然伏侍他睡下,便预备下了茶水,关好了门,进里间去照应一回,各自假寐,宝玉若有动静,再为出来。宝玉见袭人等进来,便将坐更的两个婆子支到外头,他轻轻的坐起来,暗暗的祝了几句,便睡下了,欲与神交。起初再睡不着,以后把心一静,便睡去了。岂知一夜安眠,直到天亮。宝玉醒来,拭眼坐起来想了一回,并无有梦,便叹口气道:“正是‘悠悠生死别经年,魂魄不曾来入梦’。”宝钗却一夜反没有睡着,听宝玉在外边念这两句,便接口道:“这句又说莽撞了,如若林妹妹在时,又该生气了。”宝玉听了,反不好意思,只得起来搭讪着往里间走来,说:“我原要进来的,不觉得一个盹儿就打着了。”宝钗道:“你进来不进来与我什么相干。”袭人等本没有睡,眼见他们两个说话,即忙倒上茶来。已见老太太那边打发小丫头来,问:“宝二爷昨睡得安顿么?若安顿时,早早的同二奶奶梳洗了就过去。”袭人便说:“你去回老太太,说宝玉昨夜很安顿,回来就过来。”小丫头去了。
\end{parag}


\begin{parag}
    宝钗起来梳洗了,莺儿袭人等跟着先到贾母那里行了礼,便到王夫人那边起至凤姐都让过了,仍到贾母处,见他母亲也过来了。大家问起:“宝玉晚上好么?”宝钗便说:“回去就睡了,没有什么。”众人放心,又说些闲话。只见小丫头进来说:“二姑奶奶要回去了。听见说孙姑爷那边人来到大太太那里说了些话,大太太叫人到四姑娘那边说不必留了,让他去罢。如今二姑奶奶在大太太那边哭呢,大约就过来辞老太太。”贾母众人听了,心中好不自在,都说:“二姑娘这样一个人,为什么命里遭着这样的人,一辈子不能出头。这便怎么好!”说着,迎春进来,泪痕满面,因为是宝钗的好日子,只得含着泪,辞了众人要回去。贾母知道他的苦处,也不便强留,只说道:“你回去也罢了。但是不要悲伤,碰着了这样人,也是没法儿的。过几天我再打发人接你去。”迎春道:“老太太始终疼我,如今也疼不来了。可怜我只是没有再来的时候了。”说着,眼泪直流。众人都劝道:“这有什么不能回来的?比不得你三妹妹,隔得远,要见面就难了。”贾母等想起探春,不觉也大家落泪,只为是宝钗的生日,即转悲为喜说:“这也不难,只要海疆平静,那边亲家调进京来,就见的着了。”大家说:“可不是这么着呢。”说着,迎春只得含悲而别。众人送了出来,仍回贾母那里。从早至暮,又闹了一天。
\end{parag}


\begin{parag}
    众人见贾母劳乏,各自散了。独有薛姨妈辞了贾母,到宝钗那里,说道:“你哥哥是今年过了,直要等到皇恩大赦的时候减了等才好赎罪。这几年叫我孤苦伶仃怎么处!我想要与你二哥哥完婚,你想想好不好?”宝钗道:“妈妈是为着大哥哥娶了亲唬怕的了,所以把二哥哥的事犹豫起来。据我说很该就办。邢姑娘是妈妈知道的,如今在这里也很苦,娶了去虽说我家穷,究竟比他傍人门户好多着呢。”薛姨妈道:“你得便的时候就去告诉老太太,说我家没人,就要拣日子了。”宝钗道:“妈妈只管同二哥哥商量,挑个好日子,过来和老太太,大太太说了,娶过去就完了一宗事。这里大太太也巴不得娶了去才好。”薛姨妈道:“今日听见史姑娘也就回去了,老太太心里要留你妹妹在这里住几天,所以他住下了。我想他也是不定多早晚就走的人了,你们姊妹们也多叙几天话儿。”宝钗道:“正是呢。”于是薛姨妈又坐了一坐,出来辞了众人回去了。
\end{parag}


\begin{parag}
    却说宝玉晚间归房,因想昨夜黛玉竟不入梦,“或者他已经成仙,所以不肯来见我这种浊人也是有的,不然就是我的性儿太急了,也未可知。”便想了个主意,向宝钗说道:“我昨夜偶然在外间睡着,似乎比在屋里睡的安稳些,今日起来心里也觉清静些。我的意思还要在外间睡两夜,只怕你们又来拦我。”宝钗听了,明知早晨他嘴里念诗是为着黛玉的事了。想来他那个呆性是不能劝的,倒好叫他睡两夜,索性自己死了心也罢了,况兼昨夜听他睡的倒也安静,便道:“好没来由,你只管睡去,我们拦你作什么!但只不要胡思乱想,招出些邪魔外祟来。”宝玉笑道:“谁想什么!”袭人道:“依我劝二爷竟还是屋里睡罢,外边一时照应不到,着了风倒不好。”宝玉未及答言,宝钗却向袭人使了个眼色。袭人会意,便道:“也罢,叫个人跟着你罢,夜里好倒茶倒水的。”宝玉便笑道:“这么说,你就跟了我来。”袭人听了倒没意思起来,登时飞红了脸,一声也不言语。宝钗素知袭人稳重,便说道:“他是跟惯了我的,还叫他跟着我罢。叫麝月五儿照料着也罢了。况且今日他跟着我闹了一天也乏了,该叫他歇歇了。”宝玉只得笑着出来。宝钗因命麝月五儿给宝玉仍在外间铺设了,又嘱咐两个人醒睡些,要茶要水都留点神儿。
\end{parag}


\begin{parag}
    两个答应着出来,看见宝玉端然坐在床上,闭目合掌,居然象个和尚一般,两个也不敢言语,只管瞅着他笑。宝钗又命袭人出来照应。袭人看见这般却也好笑,便轻轻的叫道:“该睡了,怎么又打起坐来了!”宝玉睁开眼看见袭人,便道:“你们只管睡罢,我坐一坐就睡。”袭人道:“因为你昨日那个光景,闹的二奶奶一夜没睡。你再这么着,成何事体。”宝玉料着自己不睡都不肯睡,便收拾睡下。袭人又嘱咐了麝月等几句,才进去关门睡了。这里麝月五儿两个人也收拾了被褥,伺候宝玉睡着,各自歇下。
\end{parag}


\begin{parag}
    那知宝玉要睡越睡不着,见他两个人在那里打铺,忽然想起那年袭人不在家时晴雯麝月两个人伏侍,夜间麝月出去,晴雯要唬他,因为没穿衣服着了凉,后来还是从这个病上死的。想到这里,一心移在晴雯身上去了。忽又想起凤姐说五儿给晴雯脱了个影儿,因又将想晴雯的心肠移在五儿身上。自己假装睡着,偷偷的看那五儿,越瞧越象晴雯,不觉呆性复发。听了听,里间已无声息,知是睡了。却见麝月也睡着了,便故意叫了麝月两声,却不答应。五儿听见宝玉唤人,便问道:“二爷要什么?”宝玉道:“我要漱漱口。”五儿见麝月已睡,只得起来重新剪了蜡花,倒了一钟茶来,一手托着漱盂。却因赶忙起来的,身上只穿着一件桃红绫子小袄儿,松松的挽着一个纂儿。宝玉看时,居然晴雯复生。忽又想起晴雯说的“早知担个虚名,也就打个正经主意了”,不觉呆呆的呆看,也不接茶。
\end{parag}


\begin{parag}
    那五儿自从芳官去后,也无心进来了。后来听见凤姐叫他进来伏侍宝玉,竟比宝玉盼他进来的心还急。不想进来以后,见宝钗袭人一般尊贵稳重,看着心里实在敬慕,又见宝玉疯疯傻傻,不似先前风致,又听见王夫人为女孩子们和宝玉顽笑都撵了:所以把这件事搁在心上,倒无一毫的儿女私情了。怎奈这位呆爷今晚把他当作晴雯,只管爱惜起来。那五儿早已羞得两颊红潮,又不敢大声说话,只得轻轻的说道:“二爷漱口啊。”宝玉笑着接了茶在手中,也不知道漱了没有,便笑嘻嘻的问道:“你和晴雯姐姐好不是啊?”五儿听了摸不着头脑,便道:“都是姐妹,也没有什么不好的。”宝玉又悄悄的问道:“晴雯病重了我看他去,不是你也去了么?”五儿微微笑着点头儿。宝玉道:“你听见他说什么了没有?”五儿摇着头儿道:“没有。”宝玉已经忘神,便把五儿的手一拉。五儿急得红了脸,心里乱跳,便悄悄说道:“二爷有什么话只管说,别拉拉扯扯的。”宝玉才放了手,说道:“他和我说来着,‘早知担了个虚名,也就打正经主意了。’你怎么没听见么?”五儿听了这话明明是轻薄自己的意思,又不敢怎么样,便说道:“那是他自己没脸,这也是我们女孩儿家说得的吗。”宝玉着急道:“你怎么也是这么个道学先生!我看你长的和他一模一样,我才肯和你说这个话,你怎么倒拿这些话来糟踏他!”此时五儿心中也不知宝玉是怎么个意思,便说道:“夜深了,二爷也睡罢,别紧着坐着,看凉着。刚才奶奶和袭人姐姐怎么嘱咐了?”宝玉道:“我不凉。”说到这里,忽然想起五儿没穿着大衣服,就怕他也象晴雯着了凉,便说道:“你为什么不穿上衣服就过来!”五儿道:“爷叫的紧,那里有尽着穿衣裳的空儿。要知道说这半天话儿时,我也穿上了。”宝玉听了,连忙把自己盖的一件月白绫子绵袄儿揭起来递给五儿,叫他披上。五儿只不肯接,说:“二爷盖着罢,我不凉。我凉我有我的衣裳。”说着,回到自己铺边,拉了一件长袄披上。又听了听,麝月睡的正浓,才慢慢过来说:“二爷今晚不是要养神呢吗?”宝玉笑道:“实告诉你罢,什么是养神,我倒是要遇仙的意思。”五儿听了,越发动了疑心,便问道:“遇什么仙?”宝玉道:“你要知道,这话长着呢。你挨着我来坐下,我告诉你。”五儿红了脸笑道:“你在那里躺着,我怎么坐呢。”宝玉道:“这个何妨。那一年冷天,也是你麝月姐姐和你晴雯姐姐顽,我怕冻着他,还把他揽在被里渥着呢。这有什么的!大凡一个人总不要酸文假醋才好。”五儿听了,句句都是宝玉调戏之意。那知这位呆爷却是实心实意的话儿。五儿此时走开不好,站着不好,坐下不好,倒没了主意了,因微微的笑着道:“你别混说了,看人家听见这是什么意思。怨不得人家说你专在女孩儿身上用工夫,你自己放着二奶奶和袭人姐姐都是仙人儿似的,只爱和别人胡缠。明儿再说这些话,我回了二奶奶,看你什么脸见人。”正说着,只听外面咕咚一声,把两个人吓了一跳。里间宝钗咳嗽了一声。宝玉听见,连忙呶嘴儿。五儿也就忙忙的息了灯悄悄的躺下了。原来宝钗袭人因昨夜不曾睡,又兼日间劳乏了一天,所以睡去,都不曾听见他们说话。此时院中一响,早已惊醒,听了听,也无动静。宝玉此时躺在床上,心里疑惑:“莫非林妹妹来了,听见我和五儿说话故意吓我们的?”翻来覆去,胡思乱想,五更以后,才朦胧睡去。
\end{parag}


\begin{parag}
    却说五儿被宝玉鬼混了半夜,又兼宝钗咳嗽,自己怀着鬼胎,生怕宝钗听见了,也是思前想后,一夜无眠。次日一早起来,见宝玉尚自昏昏睡着,便轻轻的收拾了屋子。那时麝月已醒,便道:“你怎么这么早起来了,你难道一夜没睡吗?”五儿听这话又似麝月知道了的光景,便只是讪笑,也不答言。不一时,宝钗袭人也都起来,开了门见宝玉尚睡,却也纳闷:“怎么外边两夜睡得倒这般安稳?”及宝玉醒来,见众人都起来了,自己连忙爬起,揉着眼睛,细想昨夜又不曾梦见,可是仙凡路隔了。慢慢的下了床,又想昨夜五儿说的宝钗袭人都是天仙一般,这话却也不错,便怔怔的瞅着宝钗。宝钗见他发怔,虽知他为黛玉之事,却也定不得梦不梦,只是瞅的自己倒不好意思,便道:“二爷昨夜可真遇见仙了么?”宝玉听了,只道昨晚的话宝钗听见了,笑着勉强说道:“这是那里的话!”那五儿听了这一句,越发心虚起来,又不好说的,只得且看宝钗的光景。只见宝钗又笑着问五儿道:“你听见二爷睡梦中和人说话来着么?”宝玉听了,自己坐不住,搭讪着走开了。五儿把脸飞红,只得含糊道:“前半夜倒说了几句,我也没听真。什么‘担了虚名’,又什么‘没打正经主意’,我也不懂,劝着二爷睡了,后来我也睡了,不知二爷还说来着没有。”宝钗低头一想:“这话明是为黛玉了。但尽着叫他在外头,恐怕心邪了招出些花妖月姊来。况兼他的旧病原在姊妹上情重,只好设法将他的心意挪移过来,然后能免无事。”想到这里,不免面红耳热起来,也就讪讪的进房梳洗去了。
\end{parag}


\begin{parag}
    且说贾母两日高兴,略吃多了些,这晚有些不受用,第二天便觉着胸口饱闷。鸳鸯等要回贾政。贾母不叫言语,说:“我这两日嘴馋些吃多了点子,我饿一顿就好了。你们快别吵嚷。”于是鸳鸯等并没有告诉人。
\end{parag}


\begin{parag}
    这日晚间,宝玉回到自己屋里,见宝钗自贾母王夫人处才请了晚安回来。宝玉想着早起之事,未免赧颜抱惭。宝钗看他这样,也晓得是个没意思的光景,因想着:“他是个痴情人,要治他的这病,少不得仍以痴情治之。”想了一回,便问宝玉道:“你今夜还在外间睡去罢咧?”宝玉自觉没趣,便道:“里间外间都是一样的。”宝钗意欲再说,反觉不好意思。袭人道:“罢呀,这倒是什么道理呢。我不信睡得那么安稳!”五儿听见这话,连忙接口道:“二爷在外间睡,别的倒没什么,只是爱说梦话,叫人摸不着头脑儿,又不敢驳他的回。”袭人便道:“我今日挪到床上睡睡,看说梦话不说?你们只管把二爷的铺盖铺在里间就完了。”宝钗听了,也不作声。宝玉自己惭愧不来,那里还有强嘴的分儿,便依着搬进里间来。一则宝玉负愧,欲安慰宝钗之心,二则宝钗恐宝玉思郁成疾,不如假以词色,使得稍觉亲近,以为移花接木之计。于是当晚袭人果然挪出去。宝玉因心中愧悔,宝钗欲拢络宝玉之心,自过门至今日,方才如鱼得水,恩爱缠绵,所谓二五之精妙合而凝的了。此是后话。
\end{parag}


\begin{parag}
    且说次日宝玉宝钗同起,宝玉梳洗了先过贾母这边来。这里贾母因疼宝玉,又想宝钗孝顺,忽然想起一件东西,便叫鸳鸯开了箱子,取出祖上所遗一个汉玉玦,虽不及宝玉他那块玉石,挂在身上却也稀罕。鸳鸯找出来递与贾母,便说道:“这件东西我好象从没见的,老太太这些年还记得这样清楚,说是那一箱什么匣子里装着,我按着老太太的话一拿就拿出来了。老太太怎么想着拿出来做什么?”贾母道:“你那里知道,这块玉还是祖爷爷给我们老太爷,老太爷疼我,临出嫁的时候叫了我去亲手递给我的。还说:‘这玉是汉时所佩的东西,很贵重,你拿着就象见了我的一样。’我那时还小,拿了来也不当什么,便撩在箱子里。到了这里,我见咱们家的东西也多,这算得什么,从没带过,一撩便撩了六十多年。今儿见宝玉这样孝顺,他又丢了一块玉,故此想着拿出来给他,也象是祖上给我的意思。”一时宝玉请了安,贾母便喜欢道:“你过来,我给你一件东西瞧瞧。”宝玉走到床前,贾母便把那块汉玉递给宝玉。宝玉接来一瞧,那玉有三寸方圆,形似甜瓜,色有红晕,甚是精致。宝玉口口称赞。贾母道:“你爱么?这是我祖爷爷给我的,我传了你罢。”宝玉笑着请了个安谢了,又拿了要送给他母亲瞧。贾母道:“你太太瞧了告诉你老子,又说疼儿子不如疼孙子了。他们从没见过。”宝玉笑着去了。宝钗等又说了几句话,也辞了出来。自此贾母两日不进饮食,胸口仍是结闷,觉得头晕目眩,咳嗽。邢王二夫人凤姐等请安,见贾母精神尚好,不过叫人告诉贾政,立刻来请了安。贾政出来,即请大夫看脉。不多一时,大夫来诊了脉,说是有年纪的人停了些饮食,感冒些风寒,略消导发散些就好了。开了方子,贾政看了,知是寻常药品,命人煎好进服。以后贾政早晚进来请安,一连三日,不见稍减。贾政又命贾琏:“打听好大夫,快去请来瞧老太太的病。咱们家常请的几个大夫,我瞧着不怎么好,所以叫你去。”贾琏想了一想,说道:“记得那年宝兄弟病的时候,倒是请了一个不行医的来瞧好了的,如今不如找他。”贾政道:“医道却是极难的,愈是不兴时的大夫倒有本领。你就打发人去找来罢。”贾琏即忙答应去了,回来说道:“这刘大夫新近出城教书去了,过十来天进城一次。这时等不得,又请了一位,也就来了。”贾政听了,只得等着。不题。
\end{parag}


\begin{parag}
    且说贾母病时,合宅女眷无日不来请安。一日,众人都在那里,只见看园内腰门的老婆子进来,回说:“园里的栊翠庵的妙师父知道老太太病了,特来请安。”众人道:“他不常过来,今儿特地来,你们快请进来。”凤姐走到床前回贾母。岫烟是妙玉的旧相识,先走出去接他。只见妙玉头带妙常髻,身上穿一件月白素绸袄儿,外罩一件水田青缎镶边长背心,拴着秋香色的丝绦,腰下系一条淡墨画的白绫裙,手执麈尾念珠,跟着一个侍儿,飘飘拽拽的走来。岫烟见了问好,说是“在园内住的日子,可以常常来瞧瞧你。近来因为园内人少,一个人轻易难出来。况且咱们这里的腰门常关着,所以这些日子不得见你。今儿幸会。”妙玉道:“头里你们是热闹场中,你们虽在外园里住,我也不便常来亲近。如今知道这里的事情也不大好,又听说是老太太病着,又掂记你,并要瞧瞧宝姑娘。我那管你们的关不关,我要来就来,我不来你们要我来也不能啊。”岫烟笑道:“你还是那种脾气。”一面说着,已到贾母房中。众人见了都问了好。妙玉走到贾母床前问候,说了几句套话。贾母便道:“你是个女菩萨,你瞧瞧我的病可好得了好不了?”妙玉道:“老太太这样慈善的人,寿数正有呢。一时感冒,吃几贴药想来也就好了。有年纪人只要宽心些。”贾母道:“我倒不为这些,我是极爱寻快乐的。如今这病也不觉怎样,只是胸隔闷饱,刚才大夫说是气恼所致。你是知道的,谁敢给我气受,这不是那大夫脉理平常么。我和琏儿说了,还是头一个大夫说感冒伤食的是,明儿仍请他来。”说着,叫鸳鸯吩咐厨房里办一桌净素菜来,请他在这里便饭。妙玉道:“我已吃过午饭了,我是不吃东西的。”王夫人道:“不吃也罢,咱们多坐一会说些闲话儿罢。”妙玉道:“我久已不见你们,今儿来瞧瞧。”又说了一回话便要走,回头见惜春站着,便问道:“四姑娘为什么这样瘦?不要只管爱画劳了心。”惜春道:“我久不画了。如今住的房屋不比园里的显亮,所以没兴画。”妙玉道:“你如今住在那一所了?”惜春道:“就是你才进来的那个门东边的屋子。你要来很近。”妙玉道:“我高兴的时候来瞧你。”惜春等说着送了出去,回身过来,听见丫头们回说大夫在贾母那边呢。众人暂且散去。
\end{parag}


\begin{parag}
    那知贾母这病日重一日,延医调治不效,以后又添腹泻。贾政着急,知病难医,即命人到衙门告假,日夜同王夫人亲视汤药。一日,见贾母略进些饮食,心里稍宽。只见老婆子在门外探头,王夫人叫彩云看去,问问是谁。彩云看了是陪迎春到孙家去的人,便道:“你来做什么?”婆子道:“我来了半日,这里找不着一个姐姐们,我又不敢冒撞,我心里又急。”彩云道:“你急什么?又是姑爷作践姑娘不成么?”婆子道:“姑娘不好了。前儿闹了一场,姑娘哭了一夜,昨日痰堵住了。他们又不请大夫,今日更利害了。”彩云道:“老太太病着呢,别大惊小怪的。”王夫人在内已听见了,恐老太太听见不受用,忙叫彩云带他外头说去。岂知贾母病中心静,偏偏听见,便道:“迎丫头要死了么?”王夫人便道:“没有。婆子们不知轻重,说是这两日有些病,恐不能就好,到这里问大夫。”贾母道:“瞧我的大夫就好,快请了去。”王夫人便叫彩云叫这婆子去回大太太去,那婆子去了。这里贾母便悲伤起来,说是:“我三个孙女儿,一个享尽了福死了,三丫头远嫁不得见面,迎丫头虽苦,或者熬出来,不打量他年轻轻儿的就要死了。留着我这么大年纪的人活着做什么!”王夫人鸳鸯等解劝了好半天。那时宝钗李氏等不在房中,凤姐近来有病,王夫人恐贾母生悲添病,便叫人叫了他们来陪着,自己回到房中,叫彩云来埋怨这婆子不懂事,”以后我在老太太那里,你们有事不用来回。”丫头们依命不言。岂知那婆子刚到邢夫人那里,外头的人已传进来说:“二姑奶奶死了。”邢夫人听了,也便哭了一场。现今他父亲不在家中,只得叫贾琏快去瞧看。知贾母病重,众人都不敢回。可怜一位如花似月之女,结褵年余,不料被孙家揉搓以致身亡。又值贾母病笃,众人不便离开,竟容孙家草草完结。
\end{parag}


\begin{parag}
    贾母病势日增,只想这些好女儿。一时想起湘云,便打发人去瞧他。回来的人悄悄的找鸳鸯,因鸳鸯在老太太身旁,王夫人等都在那里,不便上去,到了后头找了琥珀,告诉他道:“老太太想史姑娘,叫我们去打听。那里知道史姑娘哭得了不得,说是姑爷得了暴病,大夫都瞧了,说这病只怕不能好,若变了个痨病,还可捱过四五年。所以史姑娘心里着急。又知道老太太病,只是不能过来请安,还叫我不要在老太太面前提起。倘或老太太问起来,务必托你们变个法儿回老太太才好。”琥珀听了,咳了一声,就也不言语了,半日说道:“你去罢。”琥珀也不便回,心里打算告诉鸳鸯,叫他撒谎去,所以来到贾母床前,只见贾母神色大变,地下站着一屋子的人,嘁嘁的说“瞧着是不好了”,也不敢言语了。这里贾政悄悄的叫贾琏到身旁,向耳边说了几句话。贾琏轻轻的答应出去了,便传齐了现在家的一干家人说:“老太太的事待好出来了,你们快快分头派人办去。头一件先请出板来瞧瞧,好挂里子。快到各处将各人的衣服量了尺寸,都开明了,便叫裁缝去做孝衣。那棚杠执事都去讲定。厨房里还该多派几个人。”赖大等回道:“二爷,这些事不用爷费心,我们早打算好了。只是这项银子在那里打算?”贾琏道:“这种银子不用打算了,老太太自己早留下了。刚才老爷的主意只要办的好,我想外面也要好看。”赖大等答应,派人分头办去。
\end{parag}


\begin{parag}
    贾琏复回到自己房中,便问平儿:“你奶奶今儿怎么样?”平儿把嘴往里一努说:“你瞧去。”贾琏进内,见凤姐正要穿衣,一时动不得,暂且靠在炕桌儿上。贾琏道:“你只怕养不住了。老太太的事今儿明儿就要出来了,你还脱得过么。快叫人将屋里收拾收拾就该扎挣上去了。若有了事,你我还能回来么。”凤姐道:“咱们这里还有什么收拾的,不过就是这点子东西,还怕什么!你先去罢,看老爷叫你。我换件衣裳就来。”
\end{parag}


\begin{parag}
    贾琏先回到贾母房里,向贾政悄悄的回道:“诸事已交派明白了。”贾政点头。外面又报太医进来了,贾琏接入,又诊了一回,出来悄悄的告诉贾琏:“老太太的脉气不好,防着些。”贾琏会意,与王夫人等说知。王夫人即忙使眼色叫鸳鸯过来,叫他把老太太的装裹衣服预备出来。鸳鸯自去料理。贾母睁眼要茶喝,邢夫人便进了一杯参汤。贾母刚用嘴接着喝,便道:“不要这个,倒一钟茶来我喝。”众人不敢违拗,即忙送上来,一口喝了,还要,又喝一口,便说:“我要坐起来。”贾政等道:“老太太要什么只管说,可以不必坐起来才好。”贾母道:“我喝了口水,心里好些,略靠着和你们说说话。”珍珠等用手轻轻的扶起,看见贾母这回精神好些。未知生死,下回分解。
\end{parag}