\chap{四十二}{蘅蕪君蘭言解疑語 瀟湘子雅謬補餘香}


\begin{parag}
    \begin{note}庚:釵玉名雖兩個,人卻一身,此幻筆也。今書至三十八回時已過三分之一有餘,故寫是回使二人合而爲一。請看黛玉逝後寶釵之文字便知餘言不謬矣。\end{note}
\end{parag}


\begin{parag}
    \begin{note}蒙回前總:誰謂詩書鮮誤人,豪華相尚失天真。見得古人原立意,不正心身總莫論。\end{note}
\end{parag}


\begin{parag}
    話說他姊妹復進園來,喫過飯,大家散出,都無別話。
\end{parag}


\begin{parag}
    且說劉姥姥帶著板兒,先來見鳳姐兒,說:“明日一早定要家去了。雖住了兩三天,日子卻不多,把古往今來沒見過的,沒喫過的,沒聽見過的,都經驗了。難得老太太和姑奶奶並那些小姐們,連各房裏的姑娘們,都這樣憐貧惜老照看我。我這一回去後沒別的報答,惟有請些高香天天給你們唸佛,保佑你們長命百歲的,就算我的心了。”鳳姐兒笑道:“你別喜歡。都是爲你,老太太也被風吹病了,睡著說不好過;我們大姐兒也著了涼,在那裏發熱呢。”劉姥姥聽了,忙嘆道:“老太太有年紀的人,不慣十分勞乏的。”鳳姐兒道:“從來沒象昨兒高興。往常也進園子逛去,不過到一二處坐坐就回來了。昨兒因爲你在這裏,要叫你逛逛,一個園子倒走了多半個。大姐兒因爲找我去,太太遞了一塊糕給他,誰知風地裏吃了,就發起熱來。”劉姥姥道:“小姐兒只怕不大進園子,生地方兒,小人兒家原不該去。比不得我們的孩子,會走了,那個墳圈子裏不跑去。一則風撲了也是有的;二則只怕他身上乾淨,眼睛又淨,或是遇見什麼神了。依我說,給他瞧瞧祟書本子,仔細撞客著了。”一語提醒了鳳姐兒,便叫平兒拿出《玉匣記》著彩明來唸。彩明翻了一回念道:“八月二十五日,病者在東南方得遇花神。用五色紙錢四十張,向東南方四十步送之,大吉。”鳳姐兒笑道:“果然不錯,園子裏頭可不是花神!只怕老太太也是遇見了。”一面命人請兩分紙錢來,著兩個人來,一個與賈母送祟,一個與大姐兒送祟。果見大姐兒安穩睡了。\begin{note}庚雙夾:豈真送了就安穩哉?蓋婦人之心意皆如此,即不送豈有一夜不睡之理?作者正描愚人之見耳。\end{note}
\end{parag}


\begin{parag}
    鳳姐兒笑道:“到底是你們有年紀的人經歷的多。我這大姐兒時常肯病,也不知是個什麼原故。”劉姥姥道“這也有的事。富貴人家養的孩子多太嬌嫩,自然禁不得一些兒委曲;再他小人兒家,過於尊貴了,也禁不起。以後姑奶奶少疼他些就好了。”鳳姐兒道:“這也有理。我想起來,他還沒個名字,你就給他起個名字。一則藉藉你的壽;二則你們是莊家人,不怕你惱,到底貧苦些,你貧苦人起個名字,只怕壓的住他。”\begin{note}庚雙夾:一篇愚婦無理之談,實是世間必有之事。\end{note}劉姥姥聽說,便想了一想,笑道:“不知他幾時生的?”鳳姐兒道:“正是生日的日子不好呢,可巧是七月初七日。”劉姥姥忙笑道:“這個正好,就叫他是巧哥兒。這叫作‘以毒攻毒,以火攻火’的法子。姑奶奶定要依我這名字,他必長命百歲。日後大了,各人成家立業,或一時有不遂心的事,必然是遇難成祥,逢凶化吉,卻從這‘巧’字上來。”\begin{note}蒙側:作讖語以影射後文。\end{note}
\end{parag}


\begin{parag}
    鳳姐兒聽了,自是歡喜,忙道謝,又笑道:“只保佑他應了你的話就好了。”\begin{note} 該批:“應了這話就好”,批書人焉能不心傷?獄廟相逢之日始知“遇難成祥,逢凶化吉”實伏線於千里,哀哉傷哉!此後文字不忍卒讀。辛卯冬日。\end{note}說著叫平兒來吩咐道:“明兒咱們有事,恐怕不得閒兒。你這空兒把送姥姥的東西打點了,他明兒一早就好走的便宜了。”劉姥姥忙說:“不敢多破費了。已經遭擾了幾日,又拿著走,越發心裏不安起來。”\begin{note}蒙側:世俗常態,逼真。\end{note}鳳姐兒道:“也沒有什麼,不過隨常的東西。好也罷,歹也罷,帶了去,你們街坊鄰舍看著也熱鬧些,也是上城一次。”只見平兒走來說:“姥姥過這邊瞧瞧。”
\end{parag}


\begin{parag}
    劉姥姥忙趕了平兒到那邊屋裏,只見堆著半炕東西。平兒一一的拿與他瞧著,說道:“這是昨日你要的青紗一匹,奶奶另外送你一個實地子月白紗做裏子。這是兩個繭綢,作襖兒裙子都好。這包袱裏是兩匹綢子,年下做件衣裳穿。這是一盒子各樣內造點心,也有你喫過的,也有你沒喫過的,拿去擺碟子請客,比你們買的強些。這兩條口袋是你昨日裝瓜果子來的,如今這一個裏頭裝了兩鬥御田粳米,熬粥是難得的;這一條裏頭是園子裏果子和各樣乾果子。這一包是八兩銀子。這都是我們奶奶的。這兩包每包裏頭五十兩,共是一百兩,是太太給的,叫你拿去或者作個小本買賣,或者置幾畝地,以後再別求親靠友的。”說著又悄悄笑道:“這兩件襖兒和兩條裙子,還有四塊包頭,一包絨線,可是我送姥姥的。衣裳雖是舊的,我也沒大狠穿,你要棄嫌,我就不敢說了。”平兒說一樣劉姥姥就唸一句佛,已經唸了幾千聲佛了,又見平兒也送他這些東西,又如此謙遜,忙唸佛道:“姑娘說那裏話?這樣好東西我還棄嫌!我便有銀子也沒處去買這樣的呢。只是我怪臊的,收了又不好,不收又辜負了姑娘的心。”平兒笑道:“休說外話,咱們都是自己,我才這樣。你放心收了罷,我還和你要東西呢。到年下,你只把你們曬的那個灰條菜乾子和豇豆、扁豆、茄子、葫蘆條兒各樣乾菜帶些來,我們這裏上上下下都愛喫。這個就算了,別的一概不要,別罔費了心。”劉姥姥千恩萬謝答應了。平兒道:“你只管睡你的去。我替你收拾妥當了就放在這裏,明兒一早打發小廝們僱輛車裝上,不用你費一點心的。”
\end{parag}


\begin{parag}
    劉姥姥越發感激不盡,過來又千恩萬謝的辭了鳳姐兒,過賈母這一邊睡了一夜,次早梳洗了就要告辭。因賈母欠安,衆人都過來請安,出去傳請大夫。一時婆子回大夫來了,老媽媽請賈母進幔子去坐。賈母道:“我也老了,那裏養不出那阿物兒來,還怕他不成!不要放幔子,就這樣瞧罷。”衆婆子聽了,便拿過一張小桌來,放下一個小枕頭,便命人請。
\end{parag}


\begin{parag}
    一時只見賈珍、賈璉、賈蓉三個人將王太醫領來。王太醫不敢走甬路,只走旁階,跟著賈珍到了階磯上。早有兩個婆子在兩邊打起簾子,兩個婆子在前導引進去,又見寶玉迎了出來。只見賈母穿著青皺綢一斗珠的羊皮褂子,端坐在榻上,兩邊四個未留頭的小丫鬟都拿著蠅帚漱盂等物;又有五六個老嬤嬤雁翅擺在兩旁,碧紗櫥後隱隱約約有許多穿紅著綠戴寶簪珠的人。王太醫便不敢抬頭,忙上來請了安。賈母見他穿著六品服色,便知御醫了,也便含笑問:“供奉好?”因問賈珍: “這位供奉貴姓?”賈珍等忙回:“姓王。”賈母道:“當日太醫院正堂王君效,好脈息。”王太醫忙躬身低頭,含笑回說:“那是晚晚生家叔祖。”賈母聽了,笑道:“原來這樣,也是世交了。”一面說,一面慢慢的伸手放在小枕頭上。老嬤嬤端著一張小杌,連忙放在小桌前,略偏些。王太醫便屈一膝坐下,歪著頭診了半日,又診了那隻手,忙欠身低頭退出。賈母笑說:“勞動了。珍兒讓出去好生看茶。”
\end{parag}


\begin{parag}
    賈珍賈璉等忙答了幾個“是”,復領王太醫出到外書房中。王太醫說:“太夫人並無別症,偶感一點風涼,究竟不用吃藥,不過略清淡些,暖著一點兒,就好了。如今寫個方子在這裏,若老人家愛喫,便按方煎一劑喫,若懶待喫,也就罷了。” 說著喫過茶寫了方子。剛要告辭,只見奶子抱了大姐兒出來,笑說:“王老爺也瞧瞧我們。”王太醫聽說忙起身,就奶子懷中,左手託著大姐兒的手,右手診了一診,又摸了一摸頭,又叫伸出舌頭來瞧瞧,笑道:“我說姐兒又罵我了,只是要清清淨淨的餓兩頓就好了,不必喫煎藥,我送丸藥來,臨睡時用薑湯研開,喫下去就是了。”說畢作辭而去。
\end{parag}


\begin{parag}
    賈珍等拿了藥方來,回明賈母原故,將藥方放在桌上出去,不在話下。這裏王夫人和李紈、鳳姐兒、寶釵姊妹等見大夫出去,方從櫥後出來。王夫人略坐一坐,也回房去了。
\end{parag}


\begin{parag}
    劉姥姥見無事,方上來和賈母告辭。賈母說:“閒了再來。”又命鴛鴦來,“好生打發劉姥姥出去。我身上不好,不能送你。”劉姥姥道了謝,又作辭,方同鴛鴦出來。到了下房,鴛鴦指炕上一個包袱說道:“這是老太太的幾件衣服,都是往年間生日節下衆人孝敬的,老太太從不穿人家做的,收著也可惜,卻是一次也沒穿過的。\begin{note}蒙側:寫富貴常態,一筆作三五筆用,妙文。\end{note}昨日叫我拿出兩套兒送你帶去,或是送人,或是自己家裏穿罷,別見笑。這盒子裏是你要的面果子。這包子裏是你前兒說的藥:梅花點舌丹也有,紫金錠也有,活絡丹也有,催生保命丹也有,每一樣是一張方子包著,總包在裏頭了。這是兩個荷包,帶著頑罷。”說著便抽系子,掏出兩個筆錠如意的錁子來給他瞧,又笑道:“荷包拿去,這個留下給我罷。”劉姥姥已喜出望外,早又唸了幾千聲佛,聽鴛鴦如此說,便說道:“姑娘只管留下罷。”鴛鴦見他信以爲真,仍與他裝上,笑道:“哄你頑呢,我有好些呢。留著年下給小孩子們罷。”\begin{note}蒙側:逼真。\end{note}說著,只見一個小丫頭拿了個成窯鍾子來遞與劉姥姥,“這是寶二爺給你的。”劉姥姥道:“這是那裏說起。我那一世修了來的,今兒這樣。”說著便接了過來。鴛鴦道:“前兒我叫你洗澡,換的衣裳是我的,你不棄嫌,我還有幾件,也送你罷。”劉姥姥又忙道謝。鴛鴦果然又拿出兩件來與他包好。劉姥姥又要到園中辭謝寶玉和衆姊妹王夫人等去。鴛鴦道: “不用去了。他們這會子也不見人,回來我替你說罷。閒了再來。”又命了一個老婆子,吩咐他:“二門上叫兩個小廝來,幫著姥姥拿了東西送出去。”婆子答應了,又和劉姥姥到了鳳姐兒那邊一併拿了東西,在角門上命小廝們搬了出去,直送劉姥姥上車去了。不在話下。
\end{parag}


\begin{parag}
    且說寶釵等喫過早飯,又往賈母處問過安,回園至分路之處,寶釵便叫黛玉道:“顰兒跟我來,有一句話問你。”黛玉便同了寶釵,來至蘅蕪院中。進了房,寶釵便坐了笑道:“你跪下,我要審你。”\begin{note}蒙側:嚴整。\end{note}黛玉不解何故,因笑道:“你瞧寶丫頭瘋了!審問我什麼?”寶釵冷笑道:“好個千金小姐!好個不出閨門的女孩兒!滿嘴說的是什麼?你只實說便罷。”黛玉不解,只管發笑,心裏也不免疑惑起來,口裏只說:“我何曾說什麼?你不過要捏我的錯兒罷了。你倒說出來我聽聽。”寶釵笑道:“你還裝憨兒。昨兒行酒令你說的是什麼?我竟不知那裏來的。”\begin{note}蒙側:何等愛惜。\end{note}黛玉一想,方想起來昨兒失於檢點,那《牡丹亭》、《西廂記》說了兩句,不覺紅了臉,便上來摟著寶釵,笑道:“好姐姐,原是我不知道隨口說的。你教給我,再不說了。”\begin{note}蒙側:真能受教尊敬之態嬌憨之態,令人愛煞。\end{note}寶釵笑道:“我也不知道,聽你說的怪生的,所以請教你。”黛玉道:“好姐姐,你別說與別人,我以後再不說了。”寶釵見他羞得滿臉飛紅,滿口央告,便不肯再往下追問,因拉他坐下喫茶,\begin{note}蒙側:若無下文,自己何由而知?筆下一絲不露痕跡中,補足存小姐身分,顰兒不得反問。\end{note}款款的告訴他道:“你當我是誰,我也是個淘氣的。從小七八歲上也夠個人纏的。我們家也算\begin{note}該批:“也算”二字太謙。\end{note}是個讀書人家,祖父手裏也愛藏書。先時人口多,姊妹弟兄都在一處,都怕看正經書。弟兄們也有愛詩的,也有愛詞的,諸如這些《西廂》《琵琶》以及‘元人百種’,無所不有。\begin{note}蒙側:藏書家當,留意。\end{note}他們是偷背著我們看,我們卻也偷背著他們看。後來大人知道了,打的打,罵的罵,燒的燒,才丟開了。所以咱們女孩兒家不認得字的倒好。男人們讀書不明理,尚且不如不讀書的好,何況你我。就連作詩寫字等事,原不是你我分內之事,究竟也不是男人分內之事。\begin{note}該批:男人分內究是何事?\end{note}男人們讀書明理,輔國治民,這便好了。\begin{note}蒙側:作者一片苦心,代佛說法,代聖講道,看書者不可輕忽。\end{note}\begin{note}該批:讀書明理治民輔國者能有幾人?\end{note}只是如今並不聽見有這樣的人,讀了書倒更壞了。這是書誤了他,可惜他也把書遭塌了,所以竟不如耕種買賣,倒沒有什麼大害處。你我只該做些針黹紡織的事纔是,偏又認得了字,既認得了字,不過揀那正經的看也罷了,最怕見了些雜書,移了性情,就不可救了。”一席話,說的黛玉垂頭喫茶,心下暗伏,只有答應“是”的一字。\begin{note}蒙側:結得妙。\end{note}忽見素雲進來說:“我們奶奶請二位姑娘商議要緊的事呢。二姑娘、三姑娘、四姑娘、史姑娘、寶二爺都在那裏等著呢。”寶釵道:“又是什麼事?”黛玉道:“咱們到了那裏就知道了。”說著便和寶釵往稻香村來,果見衆人都在那裏。
\end{parag}


\begin{parag}
    李紈見了他兩個,笑道:“社還沒起,就有脫滑的了,四丫頭要告一年的假呢。”黛玉笑道:“都是老太太昨兒一句話,又叫他畫什麼園子圖兒,惹得他樂得告假了。”探春笑道:“也別要怪老太太,都是劉姥姥一句話。”林黛玉忙笑道:“可是呢,都是他一句話。他是那一門子的姥姥,直叫他是個‘母蝗蟲’就是了。” 說著大家都笑起來。寶釵笑道:“世上的話,到了鳳丫頭嘴裏也就盡了。幸而鳳丫頭不認得字,不大通,不過一概是市俗取笑。更有顰兒這促狹嘴,他用‘春秋’的法子,將市俗的粗話,撮其要,刪其繁,再加潤色比方出來,一句是一句。\begin{note}蒙側:觸目驚心,請自思量。\end{note}這‘母蝗蟲’三字,把昨兒那些形景都現出來了。虧他想的倒也快。”衆人聽了,都笑道:“你這一註解,也就不在他兩個以下。”李紈道:“我請你們大家商議,給他多少日子的假。我給了他一個月他嫌少,你們怎麼說?”黛玉道:“論理一年也不多。這園子蓋才蓋了一年,如今要畫自然得二年工夫呢。又要研墨,又要蘸筆,又要鋪紙,又要著顏色,又要……”剛說到這裏,衆人知道他是取笑惜春,便都笑問說:“還要怎樣?”黛玉也自己掌不住笑道:“又要照著這樣兒慢慢的畫,可不得二年的工夫!”衆人聽了,都拍手笑個不住。寶釵笑道:“‘又要照著這個慢慢的畫’,這落後一句最妙。所以昨兒那些笑話兒雖然可笑,回想是沒味的。你們細想顰兒這幾句話雖是淡的,回想卻有滋味。我倒笑的動不得了。”\begin{note}庚雙夾:看他劉姥姥笑後復一笑,亦想不到之文也。聽寶卿之評亦千古定論。\end{note}惜春道:“都是寶姐姐讚的他越發逞強,這會子拿我也取笑兒。”黛玉忙拉他笑道:“我且問你,還是單畫這園子呢,還是連我們衆人都畫在上頭呢?”惜春道:“原說只畫這園子的,昨兒老太太又說,單畫了園子成個房樣子了,叫連人都畫上,就象‘行樂’似的纔好。我又不會這工細樓臺,又不會畫人物,又不好駁回,正爲這個爲難呢。”黛玉道:“人物還容易,你草蟲上不能。” 李紈道:“你又說不通的話了,這個上頭那裏又用的著草蟲?或者翎毛倒要點綴一兩樣。”黛玉笑道:“別的草蟲不畫罷了,昨兒‘母蝗蟲’不畫上,豈不缺了典!”衆人聽了,又都笑起來。黛玉一面笑的兩手捧著胸口,一面說道:“你快畫罷,我連題跋都有了,起個名字,就叫作《攜蝗大嚼圖》。”\begin{note}蒙側:愈出愈奇\end{note}衆人聽了,越發鬨然大笑,前仰後合。只聽 “咕咚”一聲響,不知什麼倒了,急忙看時,原來是湘雲伏在椅子背兒上,那椅子原不曾放穩,被他全身伏著背子大笑,他又不提防,兩下里錯了勁,向東一歪,連人帶椅都歪倒了,幸有板壁擋住,不曾落地。衆人一見,越發笑個不住。寶玉忙趕上去扶了起來,方漸漸止了笑。寶玉和黛玉使個眼色兒,黛玉會意,\begin{note}蒙側:何等妙文心故意唐突\end{note}便走至裏間將鏡袱揭起,照了一照,只見兩鬢略鬆了些,忙開了李紈的妝奩,拿出抿子來,對鏡抿了兩抿,仍舊收拾好了,方出來,指著李紈道:“這是叫你帶著我們作針線教道理呢,你反招我們來大頑大笑的。”李紈笑道:“你們聽他這刁話。他領著頭兒鬧,引著人笑了,倒賴我的不是。真真恨的我只保佑明兒你得一個利害婆婆,再得幾個千刁萬惡的大姑子小姑子,試試你那會子還這麼刁不刁了。”
\end{parag}


\begin{parag}
    林黛玉早紅了臉,拉著寶釵說:“咱們放他一年的假罷。”寶釵道:“我有一句公道話,你們聽聽。藕丫頭雖會畫,不過是幾筆寫意。如今畫這園子,非離了肚子裏頭有幾幅丘壑的才能成畫。這園子卻是象畫兒一般,山石樹木,樓閣房屋,遠近疏密,也不多,也不少,恰恰的是這樣。你就照樣兒往紙上一畫,是必不能討好的。這要看紙的地步遠近,該多該少,分主分賓,該添的要添,該減的要減,該藏的要藏,該露的要露。這一起了稿子,再端詳斟酌,方成一幅圖樣。第二件,這些樓臺房舍,是必要用界劃的。一點不留神,欄杆也歪了,柱子也塌了,門窗也倒豎過來,階磯也離了縫,甚至於桌子擠到牆裏去,花盆放在簾子上來,豈不倒成了一張笑‘話’兒了。第三,要插人物,也要有疏密,有高低。衣摺裙帶,手指足步,最是要緊;一筆不細,不是腫了手就是跏了腿,染臉撕發倒是小事。依我看來竟難的很。如今一年的假也太多,一月的假也太少,竟給他半年的假,再派了寶兄弟幫著他。並不是爲寶兄弟知道教著他畫,那就更誤了事;爲的是有不知道的,或難安插的,寶兄弟好拿出去問問那會畫的相公,就容易了。”
\end{parag}


\begin{parag}
    寶玉聽了,先喜的說:“這話極是。詹子亮的工細樓臺就極好,程日興的美人是絕技,如今就問他們去。”寶釵道:“我說你是無事忙,說了一聲你就問去。等著商議定了再去。如今且拿什麼畫?”寶玉道:“家裏有雪浪紙,又大又托墨。”寶釵冷笑道:“我說你不中用!那雪浪紙寫字畫寫意畫兒,或是會山水的畫南宗山水,托墨,禁得皴搜。拿了畫這個,又不託色,又難滃,畫也不好,紙也可惜。我教你一個法子。
    原先蓋這園子,就有一張細緻圖樣,雖是匠人描的,那地步方向是不錯的。
    你和太太要了出來,也比著那紙大小,和鳳丫頭要一塊重絹,叫相公礬了,叫他照著這圖樣刪補著立了稿子,添了人物就是了。
    就是配這些青綠顏色並泥金泥銀,也得他們配去。你們也得另爖上風爐子,預備化膠、出膠、洗筆。
    還得一張粉油大案,鋪上氈子。你們那些碟子也不全,筆也不全,都得從新再置一分兒纔好。”惜春道:“我何曾有這些畫器?不過隨手寫字的筆畫畫罷了。
    就是顏色,只有赭石、廣花、藤黃、胭脂這四樣。再有,不過是兩支著色筆就完了。”寶釵道: “你不該早說。這些東西我卻還有,只是你也用不著,給你也白放著。如今我且替你收著,等你用著這個時候我送你些,也只可留著畫扇子,若畫這大幅的也就可惜了的。今兒替你開個單子,照著單子和老太太要去。你們也未必知道的全,我說著,寶兄弟寫。”寶玉早已預備下筆硯了,原怕記不清白,要寫了記著,聽寶釵如此說,喜的提起筆來靜聽。寶釵說道:“頭號排筆四支,二號排筆四支,三號排筆四支,大染四支,中染四支,小染四支,大南蟹爪十支,小蟹爪十支,鬚眉十支,大著色二十支,小著色二十支,開面十支,柳條二十支,箭頭朱四兩,南赭四兩,石黃四兩,石青四兩,石綠四兩,管黃四兩,廣花八兩,蛤粉四匣,胭脂十片,大赤飛金二百帖,青金二百帖,廣勻膠四兩,淨礬四兩。礬絹的膠礬在外,別管他們,你只把絹交出去叫他們礬去。
    這些顏色,咱們淘澄飛跌著,又頑了,又使了,包你一輩子都夠使了。再要頂 絹籮四個,志籮四個,擔筆四支,大小乳鉢四個,大粗碗二十個,五寸粗碟十個,三寸粗白碟二十個,風爐兩個,沙鍋大小四個,新瓷罐二口,新水桶四隻,一尺長白布口袋四條,浮炭二十斤,柳木炭一斤,三屜木箱一個,實地紗一丈,生薑二兩,醬半斤。”黛玉忙道:“鐵鍋一口,鍋鏟一個。”寶釵道:“這作什麼?”黛玉笑道:“你要生薑和醬這些作料,我替你要鐵鍋來,好炒顏色喫的。”衆人都笑起來。寶釵笑道:“你那裏知道。那粗色碟子保不住不上火烤,不拿薑汁子和醬預先抹在底子上烤過了,一經了火是要炸的。”衆人聽說,都道:“原來如此。”
\end{parag}


\begin{parag}
    黛玉又看了一回單子,笑著拉探春悄悄的道:“你瞧瞧,畫個畫兒又要這些水缸箱子來了。想必他糊塗了,把他的嫁妝單子也寫上了。”探春“噯”了一聲,笑個不住,說道:“寶姐姐,你還不擰他的嘴?你問問他編排你的話。”寶釵笑道:“不用問,狗嘴裏還有象牙不成!”一面說,一面走上來,把黛玉按在炕上,便要擰他的臉。黛玉笑著忙央告:“好姐姐,饒了我罷!顰兒年紀小,只知說,不知道輕重,作姐姐的教導我。姐姐不饒我,還求誰去?”衆人不知話內有因,都笑道: “說的好可憐見的,連我們也軟了,饒了他罷。”寶釵原是和他頑,忽聽他又拉扯前番說他胡看雜書的話,便不好再和他廝鬧,放起他來。黛玉笑道:“到底是姐姐,要是我,再不饒人的。”寶釵笑指他道:“怪不得老太太疼你,衆人愛你伶俐,今兒我也怪疼你的了。過來,我替你把頭髮攏一攏。”黛玉果然轉過身來,寶釵用手攏上去。寶玉在旁看著,只覺更好,不覺後悔不該令他抿上鬢去,也該留著,此時叫他替他抿去。\begin{note}蒙側:又一點。作者可稱無漏子。\end{note}正自胡思,只見寶釵說道:“寫完了,明兒回老太太去。若家裏有的就罷,若沒有的,就拿些錢去買了來,我幫著你們配。”寶玉忙收了單子。
\end{parag}


\begin{parag}
    大家又說了一回閒話。至晚飯後又往賈母處來請安。賈母原沒有大病,不過是勞乏了,兼著了些涼,溫存了一日,又吃了一劑藥疏散一疏散,至晚也就好了。不知次日又有何話,且聽下回分解。
\end{parag}


\begin{parag}
    \begin{note}蒙回末總:描寫富貴至於家人女子,無不妝顏論詩書講畫法,皆書其妙,而其中隱語警人教人,不一而足,作者之用心,誠佛菩薩之用心,讀者不可因其淺近而渺忽之。\end{note}
\end{parag}

