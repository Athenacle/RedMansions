\chap{四十二}{蘅芜君兰言解疑语 潇湘子雅谬补余香}


\begin{parag}
    \begin{note}庚:钗玉名虽两个,人却一身,此幻笔也。今书至三十八回时已过三分之一有余,故写是回使二人合而为一。请看黛玉逝后宝钗之文字便知余言不谬矣。\end{note}
\end{parag}


\begin{parag}
    \begin{note}蒙回前总:谁谓诗书鲜误人,豪华相尚失天真。见得古人原立意,不正心身总莫论。\end{note}
\end{parag}


\begin{parag}
    话说他姊妹复进园来,吃过饭,大家散出,都无别话。
\end{parag}


\begin{parag}
    且说刘姥姥带著板儿,先来见凤姐儿,说:“明日一早定要家去了。虽住了两三天,日子却不多,把古往今来没见过的,没吃过的,没听见过的,都经验了。难得老太太和姑奶奶并那些小姐们,连各房里的姑娘们,都这样怜贫惜老照看我。我这一回去后没别的报答,惟有请些高香天天给你们念佛,保佑你们长命百岁的,就算我的心了。”凤姐儿笑道:“你别喜欢。都是为你,老太太也被风吹病了,睡著说不好过;我们大姐儿也著了凉,在那里发热呢。”刘姥姥听了,忙叹道:“老太太有年纪的人,不惯十分劳乏的。”凤姐儿道:“从来没象昨儿高兴。往常也进园子逛去,不过到一二处坐坐就回来了。昨儿因为你在这里,要叫你逛逛,一个园子倒走了多半个。大姐儿因为找我去,太太递了一块糕给他,谁知风地里吃了,就发起热来。”刘姥姥道:“小姐儿只怕不大进园子,生地方儿,小人儿家原不该去。比不得我们的孩子,会走了,那个坟圈子里不跑去。一则风扑了也是有的;二则只怕他身上干净,眼睛又净,或是遇见什么神了。依我说,给他瞧瞧祟书本子,仔细撞客著了。”一语提醒了凤姐儿,便叫平儿拿出《玉匣记》著彩明来念。彩明翻了一回念道:“八月二十五日,病者在东南方得遇花神。用五色纸钱四十张,向东南方四十步送之,大吉。”凤姐儿笑道:“果然不错,园子里头可不是花神!只怕老太太也是遇见了。”一面命人请两分纸钱来,著两个人来,一个与贾母送祟,一个与大姐儿送祟。果见大姐儿安稳睡了。\begin{note}庚双夹:岂真送了就安稳哉?盖妇人之心意皆如此,即不送岂有一夜不睡之理?作者正描愚人之见耳。\end{note}
\end{parag}


\begin{parag}
    凤姐儿笑道:“到底是你们有年纪的人经历的多。我这大姐儿时常肯病,也不知是个什么原故。”刘姥姥道“这也有的事。富贵人家养的孩子多太娇嫩,自然禁不得一些儿委曲;再他小人儿家,过于尊贵了,也禁不起。以后姑奶奶少疼他些就好了。”凤姐儿道:“这也有理。我想起来,他还没个名字,你就给他起个名字。一则藉藉你的寿;二则你们是庄家人,不怕你恼,到底贫苦些,你贫苦人起个名字,只怕压的住他。”\begin{note}庚双夹:一篇愚妇无理之谈,实是世间必有之事。\end{note}刘姥姥听说,便想了一想,笑道:“不知他几时生的?”凤姐儿道:“正是生日的日子不好呢,可巧是七月初七日。”刘姥姥忙笑道:“这个正好,就叫他是巧哥儿。这叫作‘以毒攻毒,以火攻火’的法子。姑奶奶定要依我这名字,他必长命百岁。日后大了,各人成家立业,或一时有不遂心的事,必然是遇难成祥,逢凶化吉,却从这‘巧’字上来。”\begin{note}蒙侧:作谶语以影射后文。\end{note}
\end{parag}


\begin{parag}
    凤姐儿听了,自是欢喜,忙道谢,又笑道:“只保佑他应了你的话就好了。”\begin{note} 该批:“应了这话就好”,批书人焉能不心伤?狱庙相逢之日始知“遇难成祥,逢凶化吉”实伏线于千里,哀哉伤哉!此后文字不忍卒读。辛卯冬日。\end{note}说著叫平儿来吩咐道:“明儿咱们有事,恐怕不得闲儿。你这空儿把送姥姥的东西打点了,他明儿一早就好走的便宜了。”刘姥姥忙说:“不敢多破费了。已经遭扰了几日,又拿著走,越发心里不安起来。”\begin{note}蒙侧:世俗常态,逼真。\end{note}凤姐儿道:“也没有什么,不过随常的东西。好也罢,歹也罢,带了去,你们街坊邻舍看著也热闹些,也是上城一次。”只见平儿走来说:“姥姥过这边瞧瞧。”
\end{parag}


\begin{parag}
    刘姥姥忙赶了平儿到那边屋里,只见堆著半炕东西。平儿一一的拿与他瞧著,说道:“这是昨日你要的青纱一匹,奶奶另外送你一个实地子月白纱做里子。这是两个茧绸,作袄儿裙子都好。这包袱里是两匹绸子,年下做件衣裳穿。这是一盒子各样内造点心,也有你吃过的,也有你没吃过的,拿去摆碟子请客,比你们买的强些。这两条口袋是你昨日装瓜果子来的,如今这一个里头装了两斗御田粳米,熬粥是难得的;这一条里头是园子里果子和各样干果子。这一包是八两银子。这都是我们奶奶的。这两包每包里头五十两,共是一百两,是太太给的,叫你拿去或者作个小本买卖,或者置几亩地,以后再别求亲靠友的。”说著又悄悄笑道:“这两件袄儿和两条裙子,还有四块包头,一包绒线,可是我送姥姥的。衣裳虽是旧的,我也没大狠穿,你要弃嫌,我就不敢说了。”平儿说一样刘姥姥就念一句佛,已经念了几千声佛了,又见平儿也送他这些东西,又如此谦逊,忙念佛道:“姑娘说那里话?这样好东西我还弃嫌!我便有银子也没处去买这样的呢。只是我怪臊的,收了又不好,不收又辜负了姑娘的心。”平儿笑道:“休说外话,咱们都是自己,我才这样。你放心收了罢,我还和你要东西呢。到年下,你只把你们晒的那个灰条菜干子和豇豆、扁豆、茄子、葫芦条儿各样干菜带些来,我们这里上上下下都爱吃。这个就算了,别的一概不要,别罔费了心。”刘姥姥千恩万谢答应了。平儿道:“你只管睡你的去。我替你收拾妥当了就放在这里,明儿一早打发小厮们雇辆车装上,不用你费一点心的。”
\end{parag}


\begin{parag}
    刘姥姥越发感激不尽,过来又千恩万谢的辞了凤姐儿,过贾母这一边睡了一夜,次早梳洗了就要告辞。因贾母欠安,众人都过来请安,出去传请大夫。一时婆子回大夫来了,老妈妈请贾母进幔子去坐。贾母道:“我也老了,那里养不出那阿物儿来,还怕他不成!不要放幔子,就这样瞧罢。”众婆子听了,便拿过一张小桌来,放下一个小枕头,便命人请。
\end{parag}


\begin{parag}
    一时只见贾珍、贾琏、贾蓉三个人将王太医领来。王太医不敢走甬路,只走旁阶,跟著贾珍到了阶矶上。早有两个婆子在两边打起帘子,两个婆子在前导引进去,又见宝玉迎了出来。只见贾母穿著青皱绸一斗珠的羊皮褂子,端坐在榻上,两边四个未留头的小丫鬟都拿著蝇帚漱盂等物;又有五六个老嬷嬷雁翅摆在两旁,碧纱橱后隐隐约约有许多穿红著绿戴宝簪珠的人。王太医便不敢抬头,忙上来请了安。贾母见他穿著六品服色,便知御医了,也便含笑问:“供奉好?”因问贾珍: “这位供奉贵姓?”贾珍等忙回:“姓王。”贾母道:“当日太医院正堂王君效,好脉息。”王太医忙躬身低头,含笑回说:“那是晚晚生家叔祖。”贾母听了,笑道:“原来这样,也是世交了。”一面说,一面慢慢的伸手放在小枕头上。老嬷嬷端著一张小杌,连忙放在小桌前,略偏些。王太医便屈一膝坐下,歪著头诊了半日,又诊了那只手,忙欠身低头退出。贾母笑说:“劳动了。珍儿让出去好生看茶。”
\end{parag}


\begin{parag}
    贾珍贾琏等忙答了几个“是”,复领王太医出到外书房中。王太医说:“太夫人并无别症,偶感一点风凉,究竟不用吃药,不过略清淡些,暖著一点儿,就好了。如今写个方子在这里,若老人家爱吃,便按方煎一剂吃,若懒待吃,也就罢了。” 说著吃过茶写了方子。刚要告辞,只见奶子抱了大姐儿出来,笑说:“王老爷也瞧瞧我们。”王太医听说忙起身,就奶子怀中,左手托著大姐儿的手,右手诊了一诊,又摸了一摸头,又叫伸出舌头来瞧瞧,笑道:“我说姐儿又骂我了,只是要清清净净的饿两顿就好了,不必吃煎药,我送丸药来,临睡时用姜汤研开,吃下去就是了。”说毕作辞而去。
\end{parag}


\begin{parag}
    贾珍等拿了药方来,回明贾母原故,将药方放在桌上出去,不在话下。这里王夫人和李纨、凤姐儿、宝钗姊妹等见大夫出去,方从橱后出来。王夫人略坐一坐,也回房去了。
\end{parag}


\begin{parag}
    刘姥姥见无事,方上来和贾母告辞。贾母说:“闲了再来。”又命鸳鸯来,“好生打发刘姥姥出去。我身上不好,不能送你。”刘姥姥道了谢,又作辞,方同鸳鸯出来。到了下房,鸳鸯指炕上一个包袱说道:“这是老太太的几件衣服,都是往年间生日节下众人孝敬的,老太太从不穿人家做的,收著也可惜,却是一次也没穿过的。\begin{note}蒙侧:写富贵常态,一笔作三五笔用,妙文。\end{note}昨日叫我拿出两套儿送你带去,或是送人,或是自己家里穿罢,别见笑。这盒子里是你要的面果子。这包子里是你前儿说的药:梅花点舌丹也有,紫金锭也有,活络丹也有,催生保命丹也有,每一样是一张方子包著,总包在里头了。这是两个荷包,带著顽罢。”说著便抽系子,掏出两个笔锭如意的锞子来给他瞧,又笑道:“荷包拿去,这个留下给我罢。”刘姥姥已喜出望外,早又念了几千声佛,听鸳鸯如此说,便说道:“姑娘只管留下罢。”鸳鸯见他信以为真,仍与他装上,笑道:“哄你顽呢,我有好些呢。留著年下给小孩子们罢。”\begin{note}蒙侧:逼真。\end{note}说著,只见一个小丫头拿了个成窑钟子来递与刘姥姥,“这是宝二爷给你的。”刘姥姥道:“这是那里说起。我那一世修了来的,今儿这样。”说著便接了过来。鸳鸯道:“前儿我叫你洗澡,换的衣裳是我的,你不弃嫌,我还有几件,也送你罢。”刘姥姥又忙道谢。鸳鸯果然又拿出两件来与他包好。刘姥姥又要到园中辞谢宝玉和众姊妹王夫人等去。鸳鸯道: “不用去了。他们这会子也不见人,回来我替你说罢。闲了再来。”又命了一个老婆子,吩咐他:“二门上叫两个小厮来,帮著姥姥拿了东西送出去。”婆子答应了,又和刘姥姥到了凤姐儿那边一并拿了东西,在角门上命小厮们搬了出去,直送刘姥姥上车去了。不在话下。
\end{parag}


\begin{parag}
    且说宝钗等吃过早饭,又往贾母处问过安,回园至分路之处,宝钗便叫黛玉道:“颦儿跟我来,有一句话问你。”黛玉便同了宝钗,来至蘅芜院中。进了房,宝钗便坐了笑道:“你跪下,我要审你。”\begin{note}蒙侧:严整。\end{note}黛玉不解何故,因笑道:“你瞧宝丫头疯了!审问我什么?”宝钗冷笑道:“好个千金小姐!好个不出闺门的女孩儿!满嘴说的是什么?你只实说便罢。”黛玉不解,只管发笑,心里也不免疑惑起来,口里只说:“我何曾说什么?你不过要捏我的错儿罢了。你倒说出来我听听。”宝钗笑道:“你还装憨儿。昨儿行酒令你说的是什么?我竟不知那里来的。”\begin{note}蒙侧:何等爱惜。\end{note}黛玉一想,方想起来昨儿失于检点,那《牡丹亭》、《西厢记》说了两句,不觉红了脸,便上来搂著宝钗,笑道:“好姐姐,原是我不知道随口说的。你教给我,再不说了。”\begin{note}蒙侧:真能受教尊敬之态娇憨之态,令人爱煞。\end{note}宝钗笑道:“我也不知道,听你说的怪生的,所以请教你。”黛玉道:“好姐姐,你别说与别人,我以后再不说了。”宝钗见他羞得满脸飞红,满口央告,便不肯再往下追问,因拉他坐下吃茶,\begin{note}蒙侧:若无下文,自己何由而知?笔下一丝不露痕迹中,补足存小姐身分,颦儿不得反问。\end{note}款款的告诉他道:“你当我是谁,我也是个淘气的。从小七八岁上也够个人缠的。我们家也算\begin{note}该批:“也算”二字太谦。\end{note}是个读书人家,祖父手里也爱藏书。先时人口多,姊妹弟兄都在一处,都怕看正经书。弟兄们也有爱诗的,也有爱词的,诸如这些《西厢》《琵琶》以及‘元人百种’,无所不有。\begin{note}蒙侧:藏书家当,留意。\end{note}他们是偷背著我们看,我们却也偷背著他们看。后来大人知道了,打的打,骂的骂,烧的烧,才丢开了。所以咱们女孩儿家不认得字的倒好。男人们读书不明理,尚且不如不读书的好,何况你我。就连作诗写字等事,原不是你我分内之事,究竟也不是男人分内之事。\begin{note}该批:男人分内究是何事?\end{note}男人们读书明理,辅国治民,这便好了。\begin{note}蒙侧:作者一片苦心,代佛说法,代圣讲道,看书者不可轻忽。\end{note}\begin{note}该批:读书明理治民辅国者能有几人?\end{note}只是如今并不听见有这样的人,读了书倒更坏了。这是书误了他,可惜他也把书遭塌了,所以竟不如耕种买卖,倒没有什么大害处。你我只该做些针黹纺织的事才是,偏又认得了字,既认得了字,不过拣那正经的看也罢了,最怕见了些杂书,移了性情,就不可救了。”一席话,说的黛玉垂头吃茶,心下暗伏,只有答应“是”的一字。\begin{note}蒙侧:结得妙。\end{note}忽见素云进来说:“我们奶奶请二位姑娘商议要紧的事呢。二姑娘、三姑娘、四姑娘、史姑娘、宝二爷都在那里等著呢。”宝钗道:“又是什么事?”黛玉道:“咱们到了那里就知道了。”说著便和宝钗往稻香村来,果见众人都在那里。
\end{parag}


\begin{parag}
    李纨见了他两个,笑道:“社还没起,就有脱滑的了,四丫头要告一年的假呢。”黛玉笑道:“都是老太太昨儿一句话,又叫他画什么园子图儿,惹得他乐得告假了。”探春笑道:“也别要怪老太太,都是刘姥姥一句话。”林黛玉忙笑道:“可是呢,都是他一句话。他是那一门子的姥姥,直叫他是个‘母蝗虫’就是了。” 说著大家都笑起来。宝钗笑道:“世上的话,到了凤丫头嘴里也就尽了。幸而凤丫头不认得字,不大通,不过一概是市俗取笑。更有颦儿这促狭嘴,他用‘春秋’的法子,将市俗的粗话,撮其要,删其繁,再加润色比方出来,一句是一句。\begin{note}蒙侧:触目惊心,请自思量。\end{note}这‘母蝗虫’三字,把昨儿那些形景都现出来了。亏他想的倒也快。”众人听了,都笑道:“你这一注解,也就不在他两个以下。”李纨道:“我请你们大家商议,给他多少日子的假。我给了他一个月他嫌少,你们怎么说?”黛玉道:“论理一年也不多。这园子盖才盖了一年,如今要画自然得二年工夫呢。又要研墨,又要蘸笔,又要铺纸,又要著颜色,又要……”刚说到这里,众人知道他是取笑惜春,便都笑问说:“还要怎样?”黛玉也自己掌不住笑道:“又要照著这样儿慢慢的画,可不得二年的工夫!”众人听了,都拍手笑个不住。宝钗笑道:“‘又要照著这个慢慢的画’,这落后一句最妙。所以昨儿那些笑话儿虽然可笑,回想是没味的。你们细想颦儿这几句话虽是淡的,回想却有滋味。我倒笑的动不得了。”\begin{note}庚双夹:看他刘姥姥笑后复一笑,亦想不到之文也。听宝卿之评亦千古定论。\end{note}惜春道:“都是宝姐姐赞的他越发逞强,这会子拿我也取笑儿。”黛玉忙拉他笑道:“我且问你,还是单画这园子呢,还是连我们众人都画在上头呢?”惜春道:“原说只画这园子的,昨儿老太太又说,单画了园子成个房样子了,叫连人都画上,就象‘行乐’似的才好。我又不会这工细楼台,又不会画人物,又不好驳回,正为这个为难呢。”黛玉道:“人物还容易,你草虫上不能。” 李纨道:“你又说不通的话了,这个上头那里又用的著草虫?或者翎毛倒要点缀一两样。”黛玉笑道:“别的草虫不画罢了,昨儿‘母蝗虫’不画上,岂不缺了典!”众人听了,又都笑起来。黛玉一面笑的两手捧著胸口,一面说道:“你快画罢,我连题跋都有了,起个名字,就叫作《携蝗大嚼图》。”\begin{note}蒙侧:愈出愈奇\end{note}众人听了,越发哄然大笑,前仰后合。只听 “咕咚”一声响,不知什么倒了,急忙看时,原来是湘云伏在椅子背儿上,那椅子原不曾放稳,被他全身伏著背子大笑,他又不提防,两下里错了劲,向东一歪,连人带椅都歪倒了,幸有板壁挡住,不曾落地。众人一见,越发笑个不住。宝玉忙赶上去扶了起来,方渐渐止了笑。宝玉和黛玉使个眼色儿,黛玉会意,\begin{note}蒙侧:何等妙文心故意唐突\end{note}便走至里间将镜袱揭起,照了一照,只见两鬓略松了些,忙开了李纨的妆奁,拿出抿子来,对镜抿了两抿,仍旧收拾好了,方出来,指著李纨道:“这是叫你带著我们作针线教道理呢,你反招我们来大顽大笑的。”李纨笑道:“你们听他这刁话。他领著头儿闹,引著人笑了,倒赖我的不是。真真恨的我只保佑明儿你得一个利害婆婆,再得几个千刁万恶的大姑子小姑子,试试你那会子还这么刁不刁了。”
\end{parag}


\begin{parag}
    林黛玉早红了脸,拉著宝钗说:“咱们放他一年的假罢。”宝钗道:“我有一句公道话,你们听听。藕丫头虽会画,不过是几笔写意。如今画这园子,非离了肚子里头有几幅丘壑的才能成画。这园子却是象画儿一般,山石树木,楼阁房屋,远近疏密,也不多,也不少,恰恰的是这样。你就照样儿往纸上一画,是必不能讨好的。这要看纸的地步远近,该多该少,分主分宾,该添的要添,该减的要减,该藏的要藏,该露的要露。这一起了稿子,再端详斟酌,方成一幅图样。第二件,这些楼台房舍,是必要用界划的。一点不留神,栏杆也歪了,柱子也塌了,门窗也倒竖过来,阶矶也离了缝,甚至于桌子挤到墙里去,花盆放在帘子上来,岂不倒成了一张笑‘话’儿了。第三,要插人物,也要有疏密,有高低。衣折裙带,手指足步,最是要紧;一笔不细,不是肿了手就是跏了腿,染脸撕发倒是小事。依我看来竟难的很。如今一年的假也太多,一月的假也太少,竟给他半年的假,再派了宝兄弟帮著他。并不是为宝兄弟知道教著他画,那就更误了事;为的是有不知道的,或难安插的,宝兄弟好拿出去问问那会画的相公,就容易了。”
\end{parag}


\begin{parag}
    宝玉听了,先喜的说:“这话极是。詹子亮的工细楼台就极好,程日兴的美人是绝技,如今就问他们去。”宝钗道:“我说你是无事忙,说了一声你就问去。等著商议定了再去。如今且拿什么画?”宝玉道:“家里有雪浪纸,又大又托墨。”宝钗冷笑道:“我说你不中用!那雪浪纸写字画写意画儿,或是会山水的画南宗山水,托墨,禁得皴搜。拿了画这个,又不托色,又难滃,画也不好,纸也可惜。我教你一个法子。
    原先盖这园子,就有一张细致图样,虽是匠人描的,那地步方向是不错的。
    你和太太要了出来,也比著那纸大小,和凤丫头要一块重绢,叫相公矾了,叫他照著这图样删补著立了稿子,添了人物就是了。
    就是配这些青绿颜色并泥金泥银,也得他们配去。你们也得另爖上风炉子,预备化胶、出胶、洗笔。
    还得一张粉油大案,铺上毡子。你们那些碟子也不全,笔也不全,都得从新再置一分儿才好。”惜春道:“我何曾有这些画器?不过随手写字的笔画画罢了。
    就是颜色,只有赭石、广花、藤黄、胭脂这四样。再有,不过是两支著色笔就完了。”宝钗道: “你不该早说。这些东西我却还有,只是你也用不著,给你也白放著。如今我且替你收著,等你用著这个时候我送你些,也只可留著画扇子,若画这大幅的也就可惜了的。今儿替你开个单子,照著单子和老太太要去。你们也未必知道的全,我说著,宝兄弟写。”宝玉早已预备下笔砚了,原怕记不清白,要写了记著,听宝钗如此说,喜的提起笔来静听。宝钗说道:“头号排笔四支,二号排笔四支,三号排笔四支,大染四支,中染四支,小染四支,大南蟹爪十支,小蟹爪十支,须眉十支,大著色二十支,小著色二十支,开面十支,柳条二十支,箭头朱四两,南赭四两,石黄四两,石青四两,石绿四两,管黄四两,广花八两,蛤粉四匣,胭脂十片,大赤飞金二百帖,青金二百帖,广匀胶四两,净矾四两。矾绢的胶矾在外,别管他们,你只把绢交出去叫他们矾去。
    这些颜色,咱们淘澄飞跌著,又顽了,又使了,包你一辈子都够使了。再要顶 绢箩四个,志箩四个,担笔四支,大小乳钵四个,大粗碗二十个,五寸粗碟十个,三寸粗白碟二十个,风炉两个,沙锅大小四个,新瓷罐二口,新水桶四只,一尺长白布口袋四条,浮炭二十斤,柳木炭一斤,三屉木箱一个,实地纱一丈,生姜二两,酱半斤。”黛玉忙道:“铁锅一口,锅铲一个。”宝钗道:“这作什么?”黛玉笑道:“你要生姜和酱这些作料,我替你要铁锅来,好炒颜色吃的。”众人都笑起来。宝钗笑道:“你那里知道。那粗色碟子保不住不上火烤,不拿姜汁子和酱预先抹在底子上烤过了,一经了火是要炸的。”众人听说,都道:“原来如此。”
\end{parag}


\begin{parag}
    黛玉又看了一回单子,笑著拉探春悄悄的道:“你瞧瞧,画个画儿又要这些水缸箱子来了。想必他糊涂了,把他的嫁妆单子也写上了。”探春“嗳”了一声,笑个不住,说道:“宝姐姐,你还不拧他的嘴?你问问他编排你的话。”宝钗笑道:“不用问,狗嘴里还有象牙不成!”一面说,一面走上来,把黛玉按在炕上,便要拧他的脸。黛玉笑著忙央告:“好姐姐,饶了我罢!颦儿年纪小,只知说,不知道轻重,作姐姐的教导我。姐姐不饶我,还求谁去?”众人不知话内有因,都笑道: “说的好可怜见的,连我们也软了,饶了他罢。”宝钗原是和他顽,忽听他又拉扯前番说他胡看杂书的话,便不好再和他厮闹,放起他来。黛玉笑道:“到底是姐姐,要是我,再不饶人的。”宝钗笑指他道:“怪不得老太太疼你,众人爱你伶俐,今儿我也怪疼你的了。过来,我替你把头发拢一拢。”黛玉果然转过身来,宝钗用手拢上去。宝玉在旁看著,只觉更好,不觉后悔不该令他抿上鬓去,也该留著,此时叫他替他抿去。\begin{note}蒙侧:又一点。作者可称无漏子。\end{note}正自胡思,只见宝钗说道:“写完了,明儿回老太太去。若家里有的就罢,若没有的,就拿些钱去买了来,我帮著你们配。”宝玉忙收了单子。
\end{parag}


\begin{parag}
    大家又说了一回闲话。至晚饭后又往贾母处来请安。贾母原没有大病,不过是劳乏了,兼著了些凉,温存了一日,又吃了一剂药疏散一疏散,至晚也就好了。不知次日又有何话,且听下回分解。
\end{parag}


\begin{parag}
    \begin{note}蒙回末总:描写富贵至于家人女子,无不妆颜论诗书讲画法,皆书其妙,而其中隐语警人教人,不一而足,作者之用心,诚佛菩萨之用心,读者不可因其浅近而渺忽之。\end{note}
\end{parag}

