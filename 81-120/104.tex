\chap{一百零四}{醉金剛小鰍生大浪 癡公子餘痛觸前情}



\begin{parag}
    話說賈雨村剛欲過渡,見有人飛奔而來,跑到跟前,口稱:“老爺,方纔進的那廟火起了!”雨村回首看時,只見烈炎燒天,飛灰蔽目。雨村心想,“這也奇怪,我纔出來,走不多遠,這火從何而來?莫非士隱遭劫於此?”欲待回去,又恐誤了過河,若不回去,心下又不安。想了一想,便問道:“你方纔見這老道士出來了沒有?”那人道:“小的原隨老爺出來,因腹內疼痛,略走了一走。回頭看見一片火光,原來就是那廟中火起,特趕來稟知老爺。並沒有見有人出來。”雨村雖則心裏狐疑,究竟是名利關心的人,那肯回去看視,便叫那人:“你在這裏等火滅了進去瞧那老道在與不在,即來回稟。”那人只得答應了伺候。
\end{parag}


\begin{parag}
    雨村過河,仍自去查看,查了幾處,遇公館便自歇下。明日又行一程,進了都門,衆衙役接着,前呼後擁的走着。雨村坐在轎內,聽見轎前開路的人吵嚷。雨村問是何事。那開路的拉了一個人過來跪在轎前稟道:“那人酒醉不知迴避,反衝突過來。小的吆喝他,他倒恃酒撒賴,躺在街心,說小的打了他了。”雨村便道:“我是管理這裏地方的。你們都是我的子民,知道本府經過,喝了酒不知退避,還敢撒賴!”那人道:“我喝酒是自己的錢,醉了躺的是皇上的地,便是大人老爺也管不得。”雨村怒道:“這人目無法紀,問他叫什麼名字。”那人回道:“我叫醉金剛倪二。”雨村聽了生氣,叫人:“打這金剛,瞧他是金剛不是!”手下把倪二按倒,着實的打了幾鞭。倪二負痛,酒醒求饒。雨村在轎內笑道:“原來是這麼個金剛麼。我且不打你,叫人帶進衙門慢慢的問你。”衆衙役答應,拴了倪二,拉着便走。倪二哀求,也不中用。雨村進內復旨回曹,那裏把這件事放在心上。
\end{parag}


\begin{parag}
    那街上看熱鬧的三三兩兩傳說:“倪二仗着有些力氣,恃酒訛人,今兒碰在賈大人手裏,只怕不輕饒的。”這話已傳到他妻女耳邊。那夜果等倪二不見回家,他女兒便到各處賭場尋覓,那賭博的都是這麼說,他女兒急得哭了。衆人都道:“你不用着急。那賈大人是榮府的一家。榮府裏的一個什麼二爺和你父親相好,你同你母親去找他說個情,就放出來了。”倪二的女兒聽了,想了一想,”果然我父親常說間壁賈二爺和他好,爲什麼不找他去。”趕着回來,即和母親說了。孃兒兩個去找賈芸。
\end{parag}


\begin{parag}
    那日賈芸恰在家,見他母女兩個過來,便讓坐。賈芸的母親便倒茶。倪家母女即將倪二被賈大人拿去的話說了一遍,”求二爺說情放出來”。賈芸一口應承,說:“這算不得什麼,我到西府裏說一聲就放了。那賈大人全仗我家的西府裏才得做了這麼大官,只要打發個人去一說就完了。”倪家母女歡喜,回來便到府裏告訴了倪二,叫他不用忙,已經求了賈二爺,他滿口應承,討個情便放出來的。倪二聽了也喜歡。
\end{parag}


\begin{parag}
    不料賈芸自從那日給鳳姐送禮不收,不好意思進來,也不常到榮府。那榮府的門上原看着主子的行事,叫誰走動纔有些體面,一時來了他便進去通報,若主子不大理了,不論本家親戚,他一概不回,支了去就完事。那日賈芸到府上說“給璉二爺請安”。門上的說:“二爺不在家,等回來我們替回罷。”賈芸欲要說“請二奶奶的安”,生恐門上厭煩,只得回家。又被倪家母女催逼着說:“二爺常說府上是不論那個衙門,說一聲誰敢不依。如今還是府裏的一家,又不爲什麼大事,這個情還討不來,白是我們二爺了。”賈芸臉上下不來,嘴裏還說硬話:“昨兒我們家裏有事,沒打發人說去,少不得今兒說了就放。什麼大不了的事!”倪家母女只得聽信。
\end{parag}


\begin{parag}
    豈知賈芸近日大門竟不得進去,繞到後頭要進園內找寶玉,不料園門鎖着,只得垂頭喪氣的回來。想起“那年倪二借銀與我,買了香料送給他,纔派我種樹。如今我沒有錢去打點,就把我拒絕。他也不是什麼好的,拿着太爺留下的公中銀錢在外放加一錢,我們窮本家要借一兩也不能。他打諒保得住一輩子不窮的了,那知外頭的聲名很不好。我不說罷了,若說起來,人命官司不知有多少呢。”一面想着,來到家中,只見倪家母女都等着。賈芸無言可支,便說道:“西府裏已經打發人說了,只言賈大人不依。你還求我們家的奴才周瑞的親戚冷子興去才中用。”倪家母女聽了說:“二爺這樣體面爺們還不中用,若是奴才,是更不中用了。”賈芸不好意思,心裏發急道:“你不知道,如今的奴才比主子強多着呢。”倪家母女聽來無法,只得冷笑幾聲說:“這倒難爲二爺白跑了這幾天,等我們那一個出來再道乏罷。”說畢出來,另託人將倪二弄了出來,只打了幾板,也沒有什麼罪。
\end{parag}


\begin{parag}
    倪二回家,他妻女將賈家不肯說情的話說了一遍。倪二正喝着酒,便生氣要找賈芸,說:“這小雜種,沒良心的東西!頭裏他沒有飯喫要到府內鑽謀事辦,虧我倪二爺幫了他。如今我有了事他不管。好罷咧,若是我倪二鬧出來,連兩府裏都不乾淨!”他妻女忙勸道:“噯,你又喝了黃湯便是這樣有天沒日頭的,前兒可不是醉了鬧的亂子,捱了打還沒好呢,你又鬧了。”倪二道:“捱了打便怕他不成,只怕拿不着由頭!我在監裏的時候,倒認得了好幾個有義氣的朋友,聽見他們說起來,不獨是城內姓賈的多,外省姓賈的也不少。前兒監裏收下了好幾個賈家的家人。我倒說,這裏的賈家小一輩子並奴才們雖不好,他們老一輩的還好,怎麼犯了事。我打聽打聽,說是和這裏賈家是一家,都住在外省,審明白瞭解進來問罪的,我才放心。若說賈二這小子他忘恩負義,我便和幾個朋友說他家怎樣倚勢欺人,怎樣盤剝小民,怎樣強娶有男婦女,叫他們吵嚷出來,有了風聲到了都老爺耳朵裏,這一鬧起來,叫你們才認得倪二金剛呢!”他女人道:“你喝了酒睡去罷!他又強佔誰家的女人來了,沒有的事你不用混說了。”倪二道:“你們在家裏那裏知道外頭的事。前年我在賭場裏碰見了小張,說他女人被賈家佔了,他還和我商量。我倒勸他才了事的。但不知這小張如今那裏去了,這兩年沒見。若碰着了他,我倪二出個主意叫賈老二死,給我好好的孝敬孝敬我倪二太爺才罷了。你倒不理我了!”說着,倒身躺下,嘴裏還是咕咕嘟嘟的說了一回,便睡去了。他妻女只當是醉話,也不理他。明日早起,倪二又往賭場中去了。不題。
\end{parag}


\begin{parag}
    且說雨村回到家中,歇息了一夜,將道上遇見甄士隱的事告訴了他夫人一遍。他夫人便埋怨他:“爲什麼不回去瞧一瞧,倘或燒死了,可不是咱們沒良心!”說着,掉下淚來。雨村道:“他是方外的人了,不肯和咱們在一處的。”正說着,外頭傳進話來,稟說:“前日老爺吩咐瞧火燒廟去的回來了回話。”雨村踱了出來。那衙役打千請了安,回說:“小的奉老爺的命回去,也不等火滅,便冒火進去瞧那個道士,豈知他坐的地方多燒了。小的想着那道士必定燒死了。那燒的牆屋往後塌去,道士的影兒都沒有,只有一個蒲團,一個瓢兒還是好好的。小的各處找尋他的屍首,連骨頭都沒有一點兒。小的恐老爺不信,想要拿這蒲團瓢兒回來做個證見,小的這麼一拿,豈知都成了灰了。”雨村聽畢,心下明白,知士隱仙去,便把那衙役打發了出去。回到房中,並沒提吉士隱火化之言,恐他婦女不知,反生悲感,只說並無形跡,必是他先走了。
\end{parag}


\begin{parag}
    雨村出來,獨坐書房,正要細想士隱的話,忽有家人傳報說:“內廷傳旨,交看事件。”雨村疾忙上轎進內,只聽見人說:“今日賈存周江西糧道被參回來,在朝內謝罪。”雨村忙到了內閣,見了各大人,將海疆辦理不善的旨意看了,出來即忙找着賈政,先說了些爲他抱屈的話,後又道喜,問:“一路可好?”賈政也將違別以後的話細細的說了一遍。雨村道:“謝罪的本上了去沒有?”賈政道:“已上去了,等膳後下來看旨意罷。”正說着,只聽裏頭傳出旨來叫賈政,賈政即忙進去。各大人有與賈政關切的,都在裏頭等着。等了好一回方見賈政出來,看見他帶着滿頭的汗。衆人迎上去接着,問:“有什麼旨意。”賈政吐舌道:“嚇死人,嚇死人!倒蒙各位大人關切,幸喜沒有什麼事。”衆人道:“旨意問了些什麼?”賈政道:“旨意問的是雲南私帶神槍一案。本上奏明是原任太師賈化的家人。主上一直記着我們先祖的名字,便問起來。我忙着磕頭奏明先祖的名字是代化,主上便笑了,還降旨意說:‘前放兵部,後降府尹的,不是也叫賈化麼?’”那時雨村也在傍邊,倒嚇了一跳,便問賈政道:“老先生怎麼奏的?”賈政道:“我便慢慢奏道:’原任太師賈化是雲南人;現任府尹賈某是浙江人。”主上又問:‘蘇州刺史奏的賈範,是你一家子麼?’我又磕頭奏道:‘是。’主上便變色道:‘縱使家奴強佔良民妻女,還成事麼?’我一句不敢奏。主上又問道:‘賈範是你什麼人?’我忙奏道:‘是遠族。’主上哼了一聲,降旨叫了出來。可不是詫事!”衆人道:“本來也巧。怎麼一連有這兩件事?”賈政道:“事倒不奇,倒是都姓賈的不好。算來我們寒族人多,年代久了,各族都有。現在雖沒有事,究竟主上記着一個“賈”字就不好。”衆人說:“真是真,假是假,怕什麼?”賈政道:“我心裏巴不得不做官,只是不敢告老,現在我們家裏兩個世襲,這也無可奈何的。” 雨村道:“如今老先生仍是工部,想來京官是沒有事的。”賈政道:“京官雖然沒事,我究竟做過兩次外任,也就說不齊了。”衆人道:“二老爺的人品行事,我們都佩服的。就是令兄大老爺,也是個好人。只要在令侄輩上嚴緊些就是了。”賈政道:“我因在家的日子少,舍侄的事情不大查考,我心裏也不甚放心。諸位今日提起,都是至相好,或者聽見東宅的侄兒家有什麼不奉規矩的事麼?”衆人道:“沒聽見別的,只是幾位侍郎心裏不大和睦,內監裏頭也有些。想來不怕什麼,只要囑咐那邊令侄,諸事留神就是了。”衆人說畢,舉手而散。
\end{parag}


\begin{parag}
    賈政然後回家。衆子侄等都迎接上來。賈政迎着請賈母的安,然後衆子侄俱請了賈政的安,一同進府。王夫人等已到了榮禧堂迎接。賈政先到了賈母那裏拜見了,陳述些違別的話。賈母問探春消息,賈政將許配探春的事都稟明瞭,還說:“兒子起身急促,難過重陽,雖沒有親見,聽見那邊親家的人來說的極好。親家老爺太太都說請老爺太太的安。還說今冬明春,大約還可調進京來。這便好了。如今聞得海疆有事,只怕那時還不能調。”賈母始則因賈政降調回來,知探春遠在他鄉,一無親故,心下傷感;後聽賈政將官事說明,探春安好,也便轉悲爲喜,便笑着叫賈政出去。然後弟兄相見,衆子侄拜見,定了明日清晨拜祠堂。
\end{parag}


\begin{parag}
    賈政回到自己屋內,王夫人等見過,寶玉,賈璉替另拜見。賈政見了寶玉果然比起身之時臉面豐滿,倒覺安靜,獨不知他心裏糊塗,所以心甚喜歡,不以降調爲念,心想幸虧老太太辦理的好。又見寶釵沉厚更勝老時,蘭兒文雅俊秀,便喜形於色。獨見環兒仍是先前,究不甚鍾愛。歇息了半天,忽然想起:“爲何今日短了一人?”王夫人知是想着黛玉,前因家書未報:今日又剛到家,正是喜歡,不必直告,只說是病着。豈知寶玉的心裏已如刀攪,因父親到家只得把持心性伺候。王夫人設筵接風,子孫敬酒。風姐雖是侄媳,現辦家事,也隨了寶釵等敬酒。賈政便叫遞了一巡酒,“都歇息去吧。”命衆家人不必伺候,待明早拜過宗祠,然後進見。分派已定,賈政與王夫人說些別後的話,餘者王夫人都不敢言。倒是賈政先提起王子騰的事來,王夫人也不敢悲慼。賈政又說蟠兒的事,王夫人只說他是自作自受;趁便也將黛玉已死的話告訴。賈政反嚇了一驚,不覺掉下淚來連聲嘆息。王夫人也掌不住,也哭了。傍邊彩雲等即忙拉衣,王夫人止住,重又說些喜歡的話,便安寢了。
\end{parag}


\begin{parag}
    次日一早,至宗祠行禮,衆子侄都隨往。賈政便在祠旁廂房坐下,叫了賈珍,賈璉過來,問起家中事務。賈珍揀可說的說了。賈政又道:“我初回家,也不便來細細查問,只是聽見外頭說起你家裏更不比從前,諸事要謹慎纔好。你年紀也不小了,孩子們該管教管教,別叫他們在外頭得罪人。璉兒也該聽着。不是纔回家就說你們,因我有所聞所以才說的。你們更該小心些。”賈珍等臉漲通紅的,也只答應個“是”字,不敢說什麼。賈政也就罷了。迴歸西府,衆家人磕頭畢,仍復進內,衆女僕行禮,不必多贅。
\end{parag}


\begin{parag}
    只說寶玉因昨日賈政問起黛玉,王夫人答以有病,他便暗裏傷心,直待賈政命他回去,一路上,已滴了好些眼淚。回到房中,見寶釵和襲人等說話,他便獨坐外間納悶。寶釵叫襲人送過茶去,知他必是怕老爺查問功課,所以如此,只得過來安慰。寶玉便藉此走去向寶釵說:“你今晚先睡,我要定定神。這時更不如從前了三言倒忘兩語,老爺瞧着不好。你先睡,叫襲人陪我略坐坐。”寶釵不便強他,點頭應允。
\end{parag}


\begin{parag}
    寶玉出來便輕輕和襲人說,央他:“把紫鵑叫來,有話問他。但紫鵑見了我,臉上總是有氣,組須得你去解勸開了再來纔好。”襲人道:“你說要定神,我倒喜歡,怎麼又定到這上頭去了?有話你明兒問不得?”寶玉道:“我就是今晚得閒,明日倘或老爺叫幹什麼,便沒空了。好姐姐,你快去叫他來。”襲人道:“他不是二奶奶叫是不來的。”寶玉道:“所以你得去說明了纔好。”襲人道:“叫我說什麼?”寶玉道:“你還不知道我的心和他的心麼?都爲的是林姑娘。你說我並不是負心,我如今叫你們弄成了一個負心的人了!”說着這話,他瞧瞧裏間屋子,用手指着說:“他是我本不願意的,都是老太太他們捉弄的。好端端把個林姑娘弄死了。就是他死,也該叫我見見,說個明白,他死了也不抱怨我嘎。你到底聽見三姑娘他們說過的,臨死恨怨我。那紫鵑爲他們姑娘,也是恨的我了不得。你想我是無情的人麼?晴雯到底是個丫頭,也沒有什麼大好處,他死了,我實告訴你罷,我還做個祭文祭他呢。這是林姑娘親眼見的。如今林姑娘死了,難道倒不及晴雯麼?我連祭都不能祭一祭,況且林姑娘死了還有靈聖的,他想起來不是更抱怨我麼?”襲人道:“你要祭就祭去,誰攔着你呢。”寶玉道:“我自從好了起來,就想要做一篇祭文,不知道如今怎麼一點靈機都沒有了。要祭別人呢,胡亂還使得,祭他是斷斷粗糙不得一點的。所以叫紫鵑來問他姑娘的心,他打那裏看出來的。我沒病的頭裏還想得出來,病後都記不得了。你倒說林姑娘已經好了,怎麼忽然死的?他好的時候我不去,他怎麼說來着?我病的時候,他不來,他又怎麼說來着?所有他的東西,我誆過來,你二奶奶總不叫動,不知什麼意思。”襲人道:“二奶奶惟恐你傷心罷了,還有什麼呢。”寶玉道:“我不信。林姑娘既是念我爲什麼臨死把詩稿燒了,不留給我做個紀念?又聽見說天上有音樂響,必是他成了神,或是登了仙去。我雖見過了棺材,到底不知道棺材裏有他沒有。”襲人道:“你這話越發糊塗了,怎麼一個人沒死就擱在棺材裏當死了的呢!”寶玉道:“不是嘎!大凡成仙的人,或是肉身去的,或是脫胎去的。好姐姐,你到底叫了紫鵑來。”襲人道:“如今等我細細的說明了你的心,他要肯來還好,要不肯來,還得費多少話;就是來了,見你也不肯細說。據我的主意:明日等二奶奶上去了,我慢慢的問他,或是倒可仔細。遇着閒空兒,我再慢慢的告訴你。寶玉道:“你說得也是,你不知道我心裏的着急。”
\end{parag}


\begin{parag}
    正說着,麝月出來說:“二奶奶說,天已四更了,請二爺進去睡罷,襲人姐姐必是說高了興了,忘了時候。”襲人聽了,道:“可不是該睡了,有話明兒再說罷。”寶玉無奈,只得進去,又向襲人耳語道:“明兒好歹別忘了。”襲人笑道:“知道了。”麝月抹着臉笑道:“你們兩個又鬧鬼兒了。爲什麼不和二奶奶說明了,就到襲人那邊睡去?由着你們說一夜,我們也不管。”寶玉擺手道:“不用言語。”襲人恨道:“小蹄子兒,你又嚼舌根,看我明兒撕你的嘴!”回頭對寶玉道:“這不是你鬧的?說了四更天的話。”一面說,一面送寶玉進屋,各人散去。
\end{parag}


\begin{parag}
    那夜寶玉無眠,到了次日,還想這事。只聽得外頭傳進話來,說:“衆親朋因老爺回家,都要送戲接風。老爺再三推辭,說不必唱戲,竟在家裏備了水酒,倒請親朋過來大家談談。於是定了後兒擺席請人,所以進來告訴。”不知所請何人,下回分解。
\end{parag}