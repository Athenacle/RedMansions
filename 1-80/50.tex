\chap{五十}{蘆雪廣爭聯即景詩 暖香塢創制春燈謎}


\begin{parag}
    \begin{note}蒙回前總評:此回著重在寶琴,卻出色寫湘雲。寫湘雲聯句極敏捷聰慧,而寶琴之聯句不少於湘雲,可知出色寫湘雲,正所以出色寫寶琴。出色寫寶琴者,全爲與寶玉提親作引也。金針暗渡,不可不知。\end{note}
\end{parag}


\begin{parag}
    話說薛寶釵道:“到底分個次序,讓我寫出來。”說著,便令衆人拈鬮爲序。\begin{note}庚雙夾:起首恰是李氏。一定要按次序,恰又不按次序,似脫落處而不脫落,文章歧路如此。然後按次各各開出。\end{note}(按:此段批語混入正文。)鳳姐兒說道:“既是這樣說,我也說一句在上頭。”衆人都笑說道: “更妙了!”寶釵便將稻香老農之上補了一個“鳳”字,李紈又將題目講與他聽。鳳姐兒想了半日,笑道:“你們別笑話我。我只有一句粗話,下剩的我就不知道了。”衆人都笑道:“越是粗話越好,你說了只管幹正事去罷。”鳳姐兒笑道:“我想下雪必刮北風。昨夜聽見了一夜的北風,我有了一句,就是‘一夜北風緊’,可使得?”衆人聽了,都相視笑道:“這句雖粗,不見底下的,這正是會作詩的起法。不但好,而且留了多少地步與後人。就是這句爲首,稻香老農快寫上續下去。”鳳姐和李嬸平兒又吃了兩杯酒,自去了。這裏李紈便寫了:
\end{parag}


\begin{poem}
    \begin{pl} 一夜北風緊,\end{pl}
\end{poem}


\begin{parag}
    自己聯道:
\end{parag}


\begin{poem}
    \begin{pl} 開門雪尚飄。入泥憐潔白,\end{pl}
\end{poem}


\begin{parag}
    香菱道:
\end{parag}


\begin{poem}
    \begin{pl} 匝地惜瓊瑤。有意榮枯草,\end{pl}
\end{poem}


\begin{parag}
    探春道:
\end{parag}


\begin{poem}
    \begin{pl} 無心飾萎苕。價高村釀熟,\end{pl}
\end{poem}


\begin{parag}
    李綺道:
\end{parag}


\begin{poem}
    \begin{pl} 年稔府粱饒。葭動灰飛管,\end{pl}
\end{poem}


\begin{parag}
    李紋道:
\end{parag}


\begin{poem}
    \begin{pl} 陽回斗轉杓。寒山已失翠,\end{pl}
\end{poem}


\begin{parag}
    岫煙道:
\end{parag}


\begin{poem}
    \begin{pl} 凍浦不聞潮。易掛疏枝柳,\end{pl}
\end{poem}


\begin{parag}
    湘雲道:
\end{parag}


\begin{poem}
    \begin{pl} 難堆破葉蕉。麝煤融寶鼎,\end{pl}
\end{poem}


\begin{parag}
    寶琴道:
\end{parag}


\begin{poem}
    \begin{pl} 綺袖籠金貂。光奪窗前鏡,\end{pl}
\end{poem}


\begin{parag}
    黛玉道:
\end{parag}


\begin{poem}
    \begin{pl} 香粘壁上椒。斜風仍故故,\end{pl}
\end{poem}


\begin{parag}
    寶玉道:
\end{parag}


\begin{poem}
    \begin{pl} 清夢轉聊聊。何處梅花笛?\end{pl}
\end{poem}


\begin{parag}
    寶釵道:
\end{parag}


\begin{poem}
    \begin{pl} 誰家碧玉簫?鰲愁坤軸陷,\end{pl}
\end{poem}


\begin{parag}
    李紈笑道:“我替你們看熱酒去罷。”
\end{parag}


\begin{parag}
    寶釵命寶琴續聯,只見湘雲站起來道:
\end{parag}


\begin{poem}
    \begin{pl} 龍鬥陣雲銷。野岸回孤棹,\end{pl}
\end{poem}


\begin{parag}
    寶琴也站起道:
\end{parag}


\begin{poem}
    \begin{pl} 吟鞭指灞橋。賜裘憐撫戍,\end{pl}
\end{poem}


\begin{parag}
    湘雲那裏肯讓人,且別人也不如他敏捷,都看他揚眉挺身的說道:
\end{parag}


\begin{poem}
    \begin{pl} 加絮念徵徭。拗垤審夷險,\end{pl}
\end{poem}


\begin{parag}
    寶釵連聲贊好,也便聯道:
\end{parag}


\begin{poem}
    \begin{pl} 枝柯怕動搖。皚皚輕趁步,\end{pl}
\end{poem}


\begin{parag}
    黛玉忙聯道:
\end{parag}


\begin{poem}
    \begin{pl} 剪剪舞隨腰。煮芋成新賞,\end{pl}
\end{poem}


\begin{parag}
    一面說,一面推寶玉,命他聯。寶玉正看寶釵、寶琴、黛玉三人共戰湘雲,十分有趣,那裏還顧得聯詩,今見黛玉推他,方聯道:
\end{parag}


\begin{poem}
    \begin{pl} 撒鹽是舊謠。葦蓑猶泊釣,\end{pl}
\end{poem}


\begin{parag}
    湘雲笑道:“你快下去,你不中用,倒耽擱了我。”一面只聽寶琴聯道:
\end{parag}


\begin{poem}
    \begin{pl} 林斧不聞樵。伏象千峯凸,\end{pl}
\end{poem}


\begin{parag}
    湘雲忙聯道:
\end{parag}


\begin{poem}
    \begin{pl} 盤蛇一徑遙。花緣經冷結,\end{pl}
\end{poem}


\begin{parag}
    寶釵與衆人又忙贊好。探春又聯道:
\end{parag}


\begin{poem}
    \begin{pl} 色豈畏霜凋。深院驚寒雀,\end{pl}
\end{poem}


\begin{parag}
    湘雲正渴了,忙忙的喫茶,已被岫煙道:
\end{parag}


\begin{poem}
    \begin{pl} 空山泣老鴞。階墀隨上下,\end{pl}
\end{poem}


\begin{parag}
    湘雲忙丟了茶杯,忙聯道:
\end{parag}


\begin{poem}
    \begin{pl} 池水任浮漂。照耀臨清曉,\end{pl}
\end{poem}


\begin{parag}
    黛玉聯道:
\end{parag}


\begin{poem}
    \begin{pl} 繽紛入永宵。誠忘三尺冷,\end{pl}
\end{poem}


\begin{parag}
    湘雲忙笑聯道:
\end{parag}


\begin{poem}
    \begin{pl} 瑞釋九重焦。僵臥誰相問,\end{pl}
\end{poem}


\begin{parag}
    寶琴也忙笑聯道:
\end{parag}


\begin{poem}
    \begin{pl} 狂遊客喜招。天機斷縞帶,\end{pl}
\end{poem}


\begin{parag}
    湘雲又忙道:
\end{parag}


\begin{poem}
    \begin{pl} 海市失鮫綃。\end{pl}
\end{poem}


\begin{parag}
    林黛玉不容他出,接著便道:
\end{parag}


\begin{poem}
    \begin{pl} 寂寞對臺榭,\end{pl}
\end{poem}


\begin{parag}
    湘雲忙聯道:
\end{parag}


\begin{poem}
    \begin{pl} 清貧懷簞瓢。\end{pl}
\end{poem}


\begin{parag}
    寶琴也不容情,也忙道:
\end{parag}


\begin{poem}
    \begin{pl} 烹茶冰漸沸,\end{pl}
\end{poem}


\begin{parag}
    湘雲見這般,自爲得趣,又是笑,又忙聯道:
\end{parag}


\begin{poem}
    \begin{pl} 煮酒葉難燒。\end{pl}
\end{poem}


\begin{parag}
    黛玉也笑道:
\end{parag}


\begin{poem}
    \begin{pl} 沒帚山僧掃,\end{pl}
\end{poem}


\begin{parag}
    寶琴也笑道:
\end{parag}


\begin{poem}
    \begin{pl} 埋琴稚子挑。\end{pl}
\end{poem}


\begin{parag}
    湘雲笑的彎了腰,忙唸了一句,衆人問:“到底說的什麼?”湘雲喊道:
\end{parag}


\begin{poem}
    \begin{pl} 石樓閒睡鶴,\end{pl}
\end{poem}


\begin{parag}
    黛玉笑的握著胸口,高聲嚷道:
\end{parag}


\begin{poem}
    \begin{pl} 錦罽暖親貓。\end{pl}
\end{poem}


\begin{parag}
    寶琴也忙笑道:
\end{parag}


\begin{poem}
    \begin{pl} 月窟翻銀浪,\end{pl}
\end{poem}


\begin{parag}
    湘雲忙聯道:
\end{parag}


\begin{poem}
    \begin{pl} 霞城隱赤標。\end{pl}
\end{poem}


\begin{parag}
    黛玉忙笑道:
\end{parag}


\begin{poem}
    \begin{pl} 沁梅香可嚼,\end{pl}
\end{poem}


\begin{parag}
    寶釵笑稱好,也忙聯道:
\end{parag}


\begin{poem}
    \begin{pl} 淋竹醉堪調。\end{pl}
\end{poem}


\begin{parag}
    寶琴也忙道:
\end{parag}


\begin{poem}
    \begin{pl} 或溼鴛鴦帶,\end{pl}
\end{poem}


\begin{parag}
    湘雲忙聯道:
\end{parag}


\begin{poem}
    \begin{pl} 時凝翡翠翹。\end{pl}
\end{poem}


\begin{parag}
    黛玉又忙道:
\end{parag}


\begin{poem}
    \begin{pl} 無風仍脈脈,\end{pl}
\end{poem}


\begin{parag}
    寶琴又忙笑聯道:
\end{parag}


\begin{poem}
    \begin{pl} 不雨亦瀟瀟。\end{pl}
\end{poem}


\begin{parag}
    湘雲伏著已笑軟了。衆人看他三人對搶,也都不顧作詩,看著也只是笑。黛玉還推他往下聯,又道:“你也有才盡之時。我聽聽還有什麼舌根嚼了!”湘雲只伏在寶釵懷裏,笑個不住。寶釵推他起來道:“你有本事,把‘二蕭’的韻全用完了,我才伏你。”湘雲起身笑道:“我也不是作詩,竟是搶命呢。”\begin{note}庚:的是湘雲。寫海棠是一樣筆墨,如今聯句又是一樣寫法。\end{note}衆人笑道:“倒是你說罷。”探春早已料定沒有自己聯的了,便早寫出來,因說:“還沒收住呢。”李紈聽了,接過來便聯了一句道:
\end{parag}


\begin{poem}
    \begin{pl} 欲志今朝樂,\end{pl}
\end{poem}


\begin{parag}
    李綺收了一句道:
\end{parag}


\begin{poem}
    \begin{pl} 憑詩祝舜堯。\end{pl}
\end{poem}


\begin{parag}
    李紈道:“夠了,夠了。雖沒作完了韻,剩的字若生扭用了,倒不好了。”說著,大家來細細評論一回,獨湘雲的多,都笑道:“這都是那塊鹿肉的功勞。”
\end{parag}


\begin{parag}
    李紈笑道:“逐句評去都還一氣,只是寶玉又落了第了。”寶玉笑道:“我原不會聯句,只好擔待我罷。”李紈笑道:“也沒有社社擔待你的。又說韻險了,又整誤了,又不會聯句了,今日必罰你。我纔看見櫳翠庵的紅梅有趣,我要折一枝來插瓶。可厭妙玉爲人,我不理他。如今罰你去取一枝來。”衆人都道這罰的又雅又有趣。寶玉也樂爲,答應著就要走。湘雲黛玉一齊說道:“外頭冷得很,你且喫杯熱酒再去。”湘雲早執起壺來,黛玉遞了一個大杯,滿斟了一杯。湘雲笑道:“你吃了我們的酒,你要取不來,加倍罰你。”寶玉忙喫一杯,冒雪而去。李紈命人好好跟著。黛玉忙攔說:“不必,有了人反不得了。”李紈點頭說:“是。”一面命丫鬟將一個美女聳肩瓶拿來,貯了水準備插梅,因又笑道:“回來該詠紅梅了。”湘雲忙道:“我先作一首。”寶釵忙道:“今日斷乎不容你再作了。你都搶了去,別人都閒著,也沒趣。回來還罰寶玉,他說不會聯句,如今就叫他自己作去。”\begin{note}庚雙夾:想此刻寶玉已到庵中矣。\end{note}黛玉笑道:“這話很是。我還有個主意,方纔聯句不夠,莫若揀著聯的少的人作紅梅。”寶釵笑道:“這話是極。方纔邢李三位屈才,且又是客。琴兒和顰兒雲兒三個人也搶了許多,我們一概都別作,只讓他三個作纔是。”李紈因說:“綺兒也不大會作,還是讓琴妹妹作罷。”寶釵只得依允,\begin{note}庚雙夾:想此刻二玉已會,不知肯見賜否。\end{note}又道:“就用 ‘紅梅花’三個字作韻,每人一首七律。邢大妹妹作‘紅’字,你們李大妹妹作‘梅’字,琴兒作‘花’字。”李紈道:“饒過寶玉去,我不服。”湘雲忙道:“有個好題目命他作。”衆人問何題目?湘雲道:“命他就作‘訪妙玉乞紅梅’,豈不有趣?”衆人聽了,都說有趣。
\end{parag}


\begin{parag}
    一語未了,只見寶玉笑欣欣掮了一枝紅梅進來。衆丫鬟忙已接過,插入瓶內。衆人都笑稱謝。寶玉笑道:“你們如今賞罷,也不知費了我多少精神呢。”說著,探春早又遞過一鍾暖酒來,衆丫鬟走上來接了蓑笠撣雪。各人房中丫鬟都添送衣服來,\begin{note}庚雙夾:冬日午後景況。\end{note}襲人也遣人送了半舊的狐腋褂來。李紈命人將那蒸的大芋頭盛了一盤,又將朱橘、黃橙、橄欖等物盛了兩盤,命人帶與襲人去。湘雲且告訴寶玉方纔的詩題,又催寶玉快作。寶玉道:“姐姐妹妹們,讓我自己用韻罷,別限韻了。”衆人都說:“隨你作去罷。”
\end{parag}


\begin{parag}
    一面說一面大家看梅花。原來這枝梅花只有二尺來高,旁有一橫枝縱橫而出,約有五六尺長,其間小枝分歧,或如蟠螭,或如僵蚓,或孤削如筆,或密聚如林,花 碼 脂,香欺蘭蕙,\begin{note}庚雙夾:一篇《紅 犯場貳\end{note}各各稱賞。誰知邢岫煙、李紋、薛寶琴三人都已吟成,各自寫了出來。衆人便依“紅梅花”三字之序看去,寫道是:
\end{parag}


\begin{poem}
    \begin{pl}詠紅梅花\authorr{得“紅”字 邢岫煙}\end{pl}

    \begin{pl}桃未芳菲杏未紅,衝寒先已笑東風。\end{pl}

    \begin{pl}魂飛庾嶺春難辨,霞隔羅浮夢未通。\end{pl}

    \begin{pl}綠萼添妝融寶炬,縞仙扶醉跨殘虹。\end{pl}

    \begin{pl}看來豈是尋常色,濃淡由他冰雪中。\end{pl}
    \emptypl

    \begin{pl}詠紅梅花\authorr{得“梅”字 李紋}\end{pl}

    \begin{pl}白梅懶賦賦紅梅,逞豔先迎醉眼開。\end{pl}

    \begin{pl}凍臉有痕皆是血,酸心無恨亦成灰。\end{pl}

    \begin{pl}誤吞丹藥移真骨,偷下瑤池脫舊胎。\end{pl}

    \begin{pl}江北江南春燦爛,寄言蜂蝶漫疑猜。\end{pl}
    \emptypl

    \begin{pl}詠紅梅花\authorr{得“花”字 薛寶琴}\end{pl}

    \begin{pl}疏是枝條豔是花,春妝兒女競奢華。\end{pl}

    \begin{pl}閒庭曲檻無餘雪,流水空山有落霞。\end{pl}

    \begin{pl}幽夢冷隨紅袖笛,遊仙香泛絳河槎。\end{pl}

    \begin{pl}前身定是瑤臺種,無復相疑色相差。\end{pl}
\end{poem}


\begin{parag}
    衆人看了,都笑稱讚了一番,又指末一首說更好。寶玉見寶琴年紀最小,才又敏捷,深爲奇異。黛玉湘雲二人斟了一小杯酒,齊賀寶琴。寶釵笑道:“三首各有各好。你們兩個天天捉弄厭了我,如今捉弄他來了。”李紈又問寶玉:“你可有了?”寶玉忙道:“我倒有了,才一看見那三首,又嚇忘了,等我再想。”湘雲聽了,便拿了一支銅火箸擊著手爐,笑道:“我擊鼓了,若鼓絕不成,又要罰的。”寶玉笑道:“我已有了。”黛玉提起筆來,說道:“你念,我寫。”湘雲便擊了一下笑道:“一鼓絕。”寶玉笑道:“有了,你寫吧。”衆人聽他念道:
\end{parag}

\begin{poem}
    \begin{pl}
        “酒未開樽句未裁”,
    \end{pl}
\end{poem}


\begin{parag}
    黛玉寫了,搖頭笑道:“起的平平。”湘雲又道“快著!”寶玉笑道:
\end{parag}


\begin{poem}
    \begin{pl} 尋春問臘到蓬萊。\end{pl}
\end{poem}

\begin{parag}
    黛玉湘雲都點頭笑道:“有些意思了。”寶玉又道:
\end{parag}


\begin{poem}
    \begin{pl}
        不求大士瓶中露,爲乞嫦娥檻外梅。
    \end{pl}
\end{poem}


\begin{parag}
    黛玉寫了,又搖頭道:“湊巧而已。”湘雲忙催二鼓,寶玉又笑道:
\end{parag}


\begin{poem}
    \begin{pl}入世冷挑紅雪去,離塵香割紫雲來。槎枒誰惜詩肩瘦,衣上猶沾佛院苔。\end{pl}
\end{poem}


\begin{parag}
    黛玉寫畢,湘雲大家才評論時,又見幾個丫鬟跑進來道:“老太太來了。”衆人忙迎出來。大家又笑道:“怎麼這等高興!”說著,遠遠見賈母圍了大斗篷,帶著灰鼠暖兜,坐著小竹轎,打著青綢油傘,鴛鴦琥珀等五六個丫鬟,每人都是打著傘,擁轎而來。李紈等忙往上迎,賈母命人止住說:“只在那裏就是了。”來至跟前,賈母笑道:“我瞞著你太太和鳳丫頭來了。大雪地下坐著這個無妨,沒的叫他們來跴雪。”衆人忙一面上前接斗篷,攙扶著,一面答應著。賈母來至室中,先笑道:“好俊梅花!你們也會樂,我來著了。”說著,李紈早命拿了一個大狼皮褥來鋪在當中。賈母坐了,因笑道:“你們只管頑笑喫喝。我因爲天短了,不敢睡中覺,抹了一回牌,想起你們來了,我也來湊個趣兒。”李紈早又捧過手爐來,探春另拿了一副杯箸來,親自斟了暖酒,奉與賈母。賈母便飲了一口,問那個盤子裏是什麼東西。衆人忙捧了過來,回說是糟鵪鶉。賈母道:“這倒罷了,撕一兩點腿子來。”李紈忙答應了,要水洗手,親自來撕。賈母又道:“你們仍舊坐下說笑我聽。”又命李紈:“你也坐下,就如同我沒來的一樣纔好,不然我就去了。”衆人聽了,方依次坐下,這李紈便挪到盡下邊。賈母因問作何事了,衆人便說作詩。賈母道:“有作詩的,不如作些燈謎,大家正月裏好頑的。”衆人答應了。說笑了一回,賈母便說:“這裏潮溼,你們別久坐,仔細受了潮溼。”因說:“你四妹妹那裏暖和,我們到那裏瞧瞧他的畫兒,趕年可有了。”衆人笑道:“那裏能年下就有了?只怕明年端陽有了。”賈母道:“這還了得!他竟比蓋這園子還費工夫了。”
\end{parag}


\begin{parag}
    說著,仍坐了竹轎,大家圍隨,過了藕香榭,穿入一條夾道,東西兩邊皆有過街門,門樓上裏外皆嵌著石頭匾,如今進的是西門,向外的匾上鑿著“穿雲”二字,向裏的鑿著“度月”兩字。來至當中,進了向南的正門,賈母下了轎,惜春已接了出來。從裏邊遊廊過去,便是惜春臥房,門斗上有“暖香塢”三個字。\begin{note}庚雙夾:看他又寫出一處,從起至末一筆一部之文也有,千萬筆成一部之文也有,一二筆成一部之文也有。如“試才”一回起若都說完,以後則索然無味,故留此幾處以爲後文之點染也。此方活潑不板,耳目屢新。\end{note}早有幾個人打起猩紅氈簾,已覺溫香拂臉。\begin{note}庚雙夾:各處皆如此,非獨因“暖香”二字方有此景。戲注於此,以博一笑耳。\end{note}大家進入房中,賈母並不歸坐,只問畫在那裏。惜春因笑回:“天氣寒冷了,膠性皆凝澀不潤,畫了恐不好看,故此收起來。”賈母笑道: “我年下就要的。你別託懶兒,快拿出來給我快畫。”一語未了,忽見鳳姐兒披著紫羯褂,笑嘻嘻的來了,口內說道:“老祖宗今兒也不告訴人,私自就來了,要我好找。”賈母見他來了,心中自是喜悅,便道:“我怕你們冷著了,所以不許人告訴你們去。你真是個鬼靈精兒,到底找了我來。以理,孝敬也不在這上頭。”鳳姐兒笑道:“我那裏是孝敬的心找了來?我因爲到了老祖宗那裏,鴉沒雀靜的,\begin{note}庚雙夾:這四個字俗語中常聞,但不能落紙筆耳。便欲寫時,究竟不知系何四字,今如此寫來,真是不可移易。\end{note}問小丫頭子們,他又不肯說,叫我找到園裏來。我正疑惑,忽然來了兩三個姑子,我心裏才明白。我想姑子必是來送年疏,或要年例香例銀子,老祖宗年下的事也多,一定是躲債來了。我趕忙問了那姑子,果然不錯。我連忙把年例給了他們去了。如今來回老祖宗,債主已去,不用躲著了。已預備下希嫩的野雞,請用晚飯去,再遲一回就老了。”他一行說,衆人一行笑。
\end{parag}


\begin{parag}
    鳳姐兒也不等賈母說話,便命人抬過轎子來。賈母笑著,攙了鳳姐的手,仍舊上轎,帶著衆人,說笑出了夾道東門。一看四面粉妝銀砌,忽見寶琴披著鳧靨裘站在山坡上遙等,身後一個丫鬟抱著一瓶紅梅。衆人都笑道:“少了兩個人,他卻在這裏等著,也弄梅花去了。”賈母喜的忙笑道:“你們瞧,這山坡上配上他的這個人品,又是這件衣裳,後頭又是這梅花,象個什麼?”衆人都笑道:“就象老太太屋裏掛的仇十洲畫的《雙豔圖》。”賈母搖頭笑道:“那畫的那裏有這件衣裳?人也不能這樣好!”一語未了,只見寶琴背後轉出一個披大紅猩氈的人來。賈母道:“那又是那個女孩兒?”衆人笑道:“我們都在這裏,那是寶玉。”賈母笑道: “我的眼越發花了。”說話之間,來至跟前,可不是寶玉和寶琴。寶玉笑向寶釵黛玉等道:“我才又到了櫳翠庵。妙玉每人送你們一枝梅花,我已經打發人送去了。”衆人都笑說:“多謝你費心。”
\end{parag}


\begin{parag}
    說話之間,已出了園門,來至賈母房中。喫畢飯大家又說笑了一回。忽見薛姨媽也來了,說:“好大雪,一日也沒過來望候老太太。今日老太太倒不高興?正該賞雪纔是。”賈母笑道:“何曾不高興!我找了他們姊妹們去頑了一會子。”薛姨媽笑道:“昨日晚上,我原想著今日要和我們姨太太借一日園子,擺階來志,請老太太賞雪的,又見老太太安息的早。我聞得女兒說,老太太心下不大爽,因此今日也沒敢驚動。早知如此,我正該請。”賈母笑道:“這纔是十月裏頭場雪,往後下雪的日子多呢,再破費不遲。”薛姨媽笑道:“果然如此,算我的孝心虔了。”鳳姐兒笑道:“姨媽仔細忘了,如今先秤五十兩銀子來,交給我收著,一下雪,我就預備下酒,姨媽也不用操心,也不得忘了。”賈母笑道:“既這麼說,姨太太給他五十兩銀子收著,我和他每人分二十五兩,到下雪的日子,我裝心裏不快,混過去了,姨太太更不用操心,我和鳳丫頭倒得了實惠。”鳳姐將手一拍,笑道:“妙極了,這和我的主意一樣。”衆人都笑了。賈母笑道:“呸!沒臉的,就順著竿子爬上來了!你不該說姨太太是客,在咱們家受屈,我們該請姨太太纔是,那裏有破費姨太太的理!不這樣說呢,還有臉先要五十兩銀子,真不害臊!”鳳姐兒笑道:“我們老祖宗最是有眼色的,試一試,姨媽若松呢,拿出五十兩來,就和我分。這會子估量著不中用了,翻過來拿我做法子,說出這些大方話來。如今我也不和姨媽要銀子,竟替姨媽出銀子治了酒,請老祖宗吃了,我另外再封五十兩銀子孝敬老祖宗,算是罰我個包攬閒事。這可好不好?”話未說完,衆人已笑倒在炕上。
\end{parag}


\begin{parag}
    賈母因又說及寶琴雪下折梅比畫兒上還好,因又細問他的年庚八字並家內景況。薛姨媽度其意思,大約是要與寶玉求配。薛姨媽心中固也遂意,只是已許過梅家了,因賈母尚未明說,自己也不好擬定,遂半吐半露告訴賈母道:“可惜這孩子沒福,前年他父親就沒了。他從小兒見的世面倒多,跟他父母四山五嶽都走遍了。他父親是好樂的,各處因有買賣,帶著家眷,這一省逛一年,明年又往那一省逛半年,所以天下十停走了有五六停了。那年在這裏,把他許了梅翰林的兒子,偏第二年他父親就辭世了,他母親又是痰症。”鳳姐也不等說完,便嗐聲跺腳的說:“偏不巧,我正要作個媒呢,又已經許了人家。”賈母笑道:“你要給誰說媒?”鳳姐兒說道:“老祖宗別管,我心裏看準了他們兩個是一對。如今已許了人,說也無益,不如不說罷了。”賈母也知鳳姐兒之意,聽見已有了人家,也就不提了。大家又閒話了一會方散。一宿無話。
\end{parag}


\begin{parag}
    次日雪晴。飯後,賈母又親囑惜春:“不管冷暖,你只畫去,趕到年下,十分不能便罷了。第一要緊把昨日琴兒和丫頭梅花,照模照樣,一筆別錯,快快添上。”惜春聽了雖是爲難,只得應了。一時衆人都來看他如何畫,惜春只是出神。李紈因笑向衆人道:“讓他自己想去,咱們且說話兒。昨兒老太太只叫作燈謎,回家和綺兒紋兒睡不著,我就編了兩個‘四書’的。他兩個每人也編了兩個。”衆人聽了,都笑道:“這倒該作的。先說了,我們猜猜。”李紈笑道:“‘觀音未有世家傳’,打《四書》一句。”湘雲接著就說“在止於至善。”寶釵笑道:“你也想一想‘世家傳’三個字的意思再猜。”李紈笑道:“再想。”黛玉笑道:“哦,是了。是‘雖善無徵’。”衆人都笑道:“這句是了。”李紈又道:“一池青草草何名。”湘雲忙道:“這一定是‘蒲蘆也’。再不是不成?”李紈笑道:“這難爲你猜。紋兒的是‘水向石邊流出冷’,打一古人名。”探春笑問道:“可是山濤?”李紋笑道:“是。”李紈又道:“綺兒的是個 ‘螢’字,打一個字。”衆人猜了半日,寶琴笑道:“這個意思卻深,不知可是花草的‘花’字?”李綺笑道:“恰是了。”衆人道:“螢與花何干?”黛玉笑道: “妙得很!螢可不是草化的?”衆人會意,都笑了說;“好!”寶釵道:“這些雖好,不合老太太的意思,不如作些淺近的物兒,大家雅俗共賞纔好。”衆人都道: “也要作些淺近的俗物纔是。”湘雲笑道:“我編了一支《點絳脣》,恰是俗物,你們猜猜。”說著便念道:
\end{parag}


\begin{poem}
    \begin{pl}溪壑分離,紅塵遊戲,真何趣?名利猶虛,後事終難繼。\end{pl}
\end{poem}


\begin{parag}
    衆人不解,想了半日,也有猜是和尚的,也有猜是道士的,也有猜是偶戲人的。寶玉笑了半日,道:“都不是,我猜著了,一定是耍的猴兒。”湘雲笑道: “正是這個了。”衆人道:“前頭都好,末後一句怎麼解?”湘雲道:“那一個耍的猴子不是剁了尾巴去的?”衆人聽了,都笑起來,說:“他編個謎兒也是刁鑽古怪的。”李紈道:“昨日姨媽說,琴妹妹見的世面多,走的道路也多,你正該編謎兒,正用著了。你的詩且又好,何不編幾個我們猜一猜?”寶琴聽了,點頭含笑,自去尋思。寶釵也有了一個,念道:
\end{parag}


\begin{poem}
    \begin{pl}鏤檀鍥梓一層層,豈系良工堆砌成?\end{pl}

    \begin{pl}雖是半天風雨過,何曾聞得梵鈴聲!\end{pl}

\end{poem}


\begin{parag}
    打一物。
\end{parag}


\begin{parag}
    衆人猜時,寶玉也有了一個,念道:
\end{parag}


\begin{poem}
    \begin{pl}天上人間兩渺茫,琅玕節過謹隄防。\end{pl}

    \begin{pl}鸞音鶴信須凝睇,好把唏噓答上蒼。\end{pl}

\end{poem}


\begin{parag}
    黛玉也有了一個,念道是:
\end{parag}


\begin{poem}
    \begin{pl}騄駬何勞縛紫繩?馳城逐塹勢猙獰。\end{pl}

    \begin{pl}主人指示風雷動,鰲背三山獨立名。\end{pl}

\end{poem}


\begin{parag}
    探春也有了一個,方慾念時,寶琴走過來笑道:“我從小兒所走的地方的古蹟不少,我今揀了十個地方的古蹟,作了十首懷古的詩。詩雖粗鄙,卻懷往事,又暗隱俗物十件,姐姐們請猜一猜。”衆人聽了,都說:“這倒巧,何不寫出來大家一看?”要知端的
\end{parag}


\begin{parag}
    \begin{note}蒙回末總:詩詞之峭麗、燈謎之隱秀不待言,須看他極整齊、極參差,愈忙迫愈安閒,一波一折路轉峯迴,一落一起山斷雲連,各人居度各人情性都現。至李紈主壇,而起句卻在鳳姐,李紈主壇,而結句卻在最少之李綺,另是一樣弄奇。\end{note}
\end{parag}


\begin{parag}
    \begin{note}蒙回末總:最愛他中幅惜春作畫一段,似與本文無涉,而前後文之景色人物莫不筋動脈搖,而前後文之起伏照應莫不穿插映帶。文字之奇難以言狀。\end{note}
\end{parag}

