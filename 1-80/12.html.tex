\chap{一十二}{王熙凤毒设相思局 贾天祥正照风月鉴}

\begin{parag}
    \begin{note}蒙:反正从来总一心,镜光至意两相寻。有朝敲破蒙头瓮,绿水青山任好春。\end{note}
\end{parag}


\begin{parag}
    话说凤姐正与平儿说话,只见有人回说:“瑞大爷来了。”凤姐急命:\begin{note}庚侧:立意追命。\end{note}“快请进来。”贾瑞见往里让,心中喜出望外,急忙进来,见了凤姐,满面陪笑,\begin{note}庚侧:如蛇。\end{note}连连问好。凤姐儿也假意殷勤,让坐让茶。
\end{parag}


\begin{parag}
    贾瑞见凤姐如此打扮,益发酥倒,因饧了眼问道:“二哥哥怎么还不回来?”凤姐道:“不知什么原故。”贾瑞笑道:“别是路上有人绊住了脚了,\begin{note}蒙侧:旁敲远引。\end{note}舍不得回来也未可知?”凤姐道:“也未可知。男人家见一个爱一个也是有的。”\begin{note}蒙侧:这是钩。\end{note}贾瑞笑道:\begin{note}庚双夹:如闻其声。\end{note}“嫂子这话错了,我就不这样。”\begin{note}庚双夹:渐渐入港。\end{note}凤姐笑道:“象你这样的人能有几个呢,十个里也挑不出一个来。”\begin{note}庚眉:勿作正面看为幸。畸笏。蒙侧:游鱼虽有入瓮之志,无钩不能上岸;一上钩来,欲去亦不可得。\end{note}贾瑞听了,喜的抓耳挠腮,又道:“嫂子天天也闷的很?”凤姐道:“正是呢,只盼个人来说话解解闷儿。”贾瑞笑道:“我倒天天闲著,天天过来替嫂子解解闲闷可好不好?”凤姐笑道:“你哄我呢,你那里肯往我这里来?”贾瑞道:“我嫂子跟前,若有一点谎话,天打雷劈!只因素人闻得人说,嫂子是个利害人,在你跟前一点也错不得,所以唬住了我。如今见嫂子最是个有说有笑极疼人的,\begin{note}庚双夹:奇妙!\end{note}我怎么不来,——死了也愿意!”\begin{note}庚侧:这倒不假。\end{note}凤姐笑道:“果然你是个明白人,比贾蓉两个强远了。我看他那样清秀,只当他们心里明白,谁知竟是两个糊涂虫,\begin{note}庚侧:反文著眼。\end{note}一点不知人心。”
\end{parag}


\begin{parag}
    贾瑞听这话,越发撞在心坎儿上,由不得又往前凑了一凑,\begin{note}写呆人痴性活现。\end{note}觑著眼看凤姐带的荷包,然后又问戴著什么戒指。凤姐悄悄道 :“放尊重著,别叫丫头们看了笑话。”贾瑞如听纶音佛语一般,忙往后退。凤姐笑道:“你该走了。”\begin{note}庚双夹:叫去正是叫来也。\end{note}贾瑞道:“我再坐一坐儿。”“好狠心的嫂子!”凤姐又悄悄的道:“大天白白,人来人往,你就在这里也不方便。你且去,等著晚上起了更你来,悄悄的在西边穿堂儿等我。”\begin{note}庚眉:先写穿堂,只知房舍之大,岂料有许多用处。\end{note}\begin{note}蒙侧:凡人在平静时,物来言至,无不照见。若迷于一事一物,虽风雷交作,有所不闻。即“穿堂尔等”之一语,府第非比凡常,关门户,必要查看,且更夫仆妇,势必往来,岂容人藏过于其间?只因色迷,闻声联诺,不能有回思之暇,信可悲夫!\end{note}贾瑞听了,如得珍宝,忙问道:“你别哄我。但只那里人过的多,怎么好躲的?”凤姐道:“你只放心。我把上夜的小厮们都放了假,两边门一关,再没别人了。”贾瑞听了,喜之不尽,忙忙的告辞而去,心内以为得手。\begin{note}庚侧:未必。\end{note}
\end{parag}


\begin{parag}
    盼到晚上,果然黑地里摸入荣府,趁掩门时,钻入穿堂。果见漆黑无人,往贾母那边去的门户已锁倒,只有向东的门未关。贾瑞侧耳听著,半日不见人来,忽听咯登一声,东边的门也倒关了。\begin{note}庚侧:平平略施小计。\end{note}贾瑞急的也不敢则声,只得悄悄的出来,将门撼了撼,关得铁桶一般。此时要求出去,亦不能够。\begin{note}蒙侧:此大抵是凤姐调遣。不先为点明者,可以少许多事故,又可以藏拙。\end{note}南北皆是大房墙,要跳亦无攀援。这屋内又是过门风,空落落;现是腊月天气,夜又长,朔风凛凛,侵肌裂骨,一夜几乎不曾冻死。\begin{note}庚眉:可为偷情一戒。蒙侧:教导之法、慈悲之心尽矣,无奈迷径不悟何!\end{note}好容易盼到早晨,只见一个老婆子先将东门开了,进去叫西门。贾瑞瞅他背著脸,一溜烟抱著肩跑了出来,幸而天气尚早,人都未起,从后门一径跑回家去。
\end{parag}


\begin{parag}
    原来贾瑞父母早亡,只有他祖父代儒教养。那代儒素日教训最严,\begin{note}庚眉:教训最严,奈其心何!一叹。\end{note}不许贾瑞多走一步,生怕他在外吃酒赌钱,有误学业。今忽见他一夜不归,只料定他在外非饮即赌,嫖娼宿妓,\begin{note}庚侧:辗转灵活,一人不放,一笔不肖。\end{note}那里想到这段公案,\begin{note}庚侧:世人万万想不到,况老学究乎!\end{note}因此气了一夜。贾瑞也捻著一把汗,少不得回来撒慌,只说:“往舅舅家去了,天黑了,留我住了一夜。”代儒道:“自来出门,非禀我不敢擅出,如何昨日私自去了?据此亦该打,何况是撒谎!”\begin{note}庚眉:处处点父母痴心、子孙不肖。此书系自愧而成。\end{note}因此,发狠到底打了三四十板,不许吃饭,令他跪在院内读文章,定要补出十天工课来方罢。贾瑞直冻了一夜,今又遭了苦打,且饿著肚子跪在风地里念文章,\begin{note}教令何尝不好,孽种故此不同。\end{note}其苦万状。\begin{note}庚双夹:祸福无门,唯人自招。\end{note}
\end{parag}


\begin{parag}
    此时贾瑞前心犹是未改,\begin{note}庚侧:四字是寻死之根。庚眉:苦海无边,回头是岸。若个能回头也?叹叹!壬午春。畸笏。\end{note}再想不到是凤姐捉弄他。过后两日,得了空,便仍来找凤姐。凤姐故意抱怨他失信,贾瑞急的赌身发誓。凤姐因见他自投罗网,\begin{note}庚侧:可谓因人而使。\end{note}少不得再寻别计令他知改,\begin{note}庚侧:四字是作者明阿凤身份,勿得轻轻看过。\end{note}故又约他道:“今日晚上,你别在那里了。你在我这房后小过道子里那间空屋里等我,可别冒撞了。”\begin{note}庚双夹:伏得妙!\end{note}贾瑞道:“果真?”凤姐道:“谁可哄你,你不信就别来。”\begin{note}庚侧:紧一句。\end{note}\begin{note}蒙侧:大士心肠。\end{note}贾瑞道:“来,来,来。死也要来!”\begin{note}庚双夹:不差。\end{note}凤姐道:“这会子你先去罢。”贾瑞料定晚间必妥,\begin{note}庚侧:未必。\end{note}此时先去了。凤姐在这里便点兵派将,\begin{note}庚侧:四字用得新,必有新文字好看。\end{note}\begin{note}蒙侧:新文,最妙!\end{note}设下圈套。
\end{parag}


\begin{parag}
    那贾瑞只盼不到夜上,偏生家里有亲戚又来了,\begin{note}庚双夹:专能忙中写闲,狡猾之甚!\end{note}直等吃了晚饭才去,那天已有掌灯时候。又等他祖父安歇了,方溜进荣府,直往那夹道中屋子里来等著,热锅上的蚂蚁一般,\begin{note}蒙侧:有心人记著,其实苦恼。\end{note}只是干转。左等不见人影,右听也没声音,心下自思:“别是又不来了,又冻我一夜不成?”\begin{note}蒙侧:似醒非醒语。\end{note}正自胡猜,只见黑魆魆的来了一个人,\begin{note}庚侧:真到了。\end{note}贾瑞便意定是凤姐,不管皂白,饿虎一般,等那人刚至门前,便如猫儿捕鼠的一般,抱住叫道:“亲嫂子,等死我了。”说著,抱到屋里炕上就亲嘴扯裤子,满口里“亲娘”“亲爹”的乱叫起来。\begin{note}蒙侧:丑态可笑。\end{note}那人只不做声,\begin{note}庚侧:好极!\end{note}贾瑞拉了自己裤子,硬帮帮的就想顶入。\begin{note}庚侧:将到矣。\end{note}忽然灯光一闪,只见贾蔷举著个捻子照道:“谁在屋里?”只见炕上那人笑道:“瑞大叔要臊我呢。”贾瑞一见,却是贾蓉,\begin{note}庚双夹:奇绝!\end{note}真臊的无地可入,\begin{note}庚侧:亦未必真。\end{note}不知要怎么样才好,回身就要跑,被贾蔷一把揪住道:“别走!如今琏二婶已经告到太太跟前,\begin{note}庚侧:好题目。\end{note}说你无故调戏他。\begin{note}庚眉:调戏还要有故?一笑!\end{note}他暂用了个脱身计,哄你在这边等著,太太气死过去,\begin{note}庚侧:好大题目。\end{note}因此叫我来拿你。刚才你又拦住他,没的说,跟我去见太太!”
\end{parag}


\begin{parag}
    贾瑞听了,魂不附体,只说:“好侄儿,只说没有见我,明日我重重的谢你。”贾蔷道:“你若谢我,放你不值什么,只不知你谢我多少?况且口说无凭,写一文契来。”贾瑞道:“这如何落纸呢?”\begin{note}庚侧:也知写不得。一叹!\end{note}贾蔷道:“这也不妨,写一个赌钱输了外人账目,借头家银若干两便罢。”贾瑞道:“这也容易。只是此时无纸笔。”贾蔷道:“这也容易。”说罢,翻身出来,纸笔现成,\begin{note}庚侧:二字妙!\end{note}拿来命贾瑞写。他两作好作歹,只写了五十两银,然后画了押,贾蔷收起来。然后撕罗贾蓉。\begin{note}蒙侧:可怜至此!好事者当自度。\end{note}贾蓉先咬定牙不依,只说:“明日告诉族中的人评评理。”贾瑞急的至于叩头。贾蔷做好做歹的,\begin{note}蒙侧:此是加一倍法。\end{note}也写了一张五十两欠契才罢。贾蔷又道:“如今要放你,我就担著不是。\begin{note}庚双夹:又生波澜。\end{note}老太太那边的门早已关了,老爷正在厅上看南京的东西,那一条路定难过去,如今只好走后门。若这一走,倘或遇见了人,连我也完了。等我们先去哨探哨探,再来领你。这屋你还藏不得,少时就来堆东西。等我寻个地方。”说毕,拉著贾瑞,仍熄了灯,\begin{note}庚双夹:细。\end{note}出至院外,摸著大台矶底下,说道:“这窝儿里好,你只蹲著,别哼一声,等我们来再动。”\begin{note}庚侧:未必如此收场。\end{note}说毕,二人去了。
\end{parag}


\begin{parag}
    贾瑞此时身不由己,只得蹲在那里。心下正盘算,只听头顶上一声响,哗拉拉一净桶尿粪从上面直泼下来,可巧浇了他一头一身,贾瑞掌不住嗳哟了一声,忙又掩住口,\begin{note}庚双夹:更奇。\end{note}不敢声张,满头满脸浑身皆是尿屎,冰冷打战。\begin{note}庚侧:余料必有新奇解恨文字收场,方是《石头记》笔力。\end{note}\begin{note}庚眉:瑞奴实当如是报之。此一节可入《西厢记》批评内十大快中。畸笏。\end{note}\begin{note}蒙侧:这也未必不是预为埋伏者。总是慈悲设教,遇难教者,不得不现三头六臂,并吃人心、喝人血之相,以警戒之耳。\end{note}只见贾蔷跑来叫:“快走,快走!”贾瑞如得了命,三步两步从后门跑到家里,天已三更,只得叫门。开门人见他这般光景,问是怎的。少不得撒谎说:“黑了,失脚掉在茅厕里了。”一面到自己房中更衣洗濯,心下方想到是凤姐顽他,因此发一回恨;再想想凤姐的模样儿,\begin{note}庚侧:欲根未断。\end{note}又恨不得一时搂在怀,一夜竟不曾合眼。
\end{parag}


\begin{parag}
    自此满心想凤姐,\begin{note}庚眉:此刻还不回头,真自寻死路矣。\end{note}\begin{note}蒙侧:孙行者非有紧箍儿,虽老君之炉、五行之山,何尝屈其一二?\end{note}只不敢往荣府去了。贾蓉两个常常的来索银子,他又怕祖父知道,正是相思尚且难禁,更又添了债务;日间工课又紧,他二十来岁之人,尚未娶亲,迩来想著凤姐,未免有那指头告了消乏等事;更兼两回冻恼奔波,\begin{note}庚双夹:写得历历病源,如何不死?\end{note}因此三五下里夹攻,\begin{note}庚侧:所谓步步紧。\end{note}不觉就得了一病:心内发膨胀,口内无滋味,脚下如绵,眼中似醋,黑夜作烧,白昼常倦,下溺连精,嗽痰带血。诸如此症,不上一年,都添全了。\begin{note}庚侧:简洁之至!\end{note}于是不能支持,一头睡倒,合上眼还只梦魂颠倒,满口乱说胡话,惊怖异常。百般请医治疗,诸如肉桂、附子、鳖甲、麦冬、玉竹等药,吃了有几十斤下去,也不见个动静。\begin{note}庚双夹:说得有趣。\end{note}
\end{parag}


\begin{parag}
    倏又腊尽春回,这病更又沉重。代儒也著了忙,各处请医疗治,皆不见效。因后来吃“独参汤”,代儒如何有这力量,只得往荣府来寻。王夫人命凤姐秤二两给他,\begin{note}庚双夹:王夫人之慈若是。\end{note}凤姐回说:“前儿新近都替老太太配了药,那整的太太又说留著送杨提督的太太配药,偏生昨儿我已送了去了。”王夫人道:“就是咱们这边没了,你打发个人往你婆婆那边问问,或是你珍大哥哥那府里再寻些来,凑著给人家。吃好了,救人一命,也是你的好处。”\begin{note}庚双夹:夹写王夫人。\end{note}凤姐听了,也不遣人去寻,只得将些渣末泡须凑了几钱,命人送去,只说:\begin{note}蒙侧:“只说”。\end{note}“太太送来的,再也没了。”然后回王夫人说:“都寻了来,共凑了有二两多送去。”\begin{note}庚双夹:然便有二两独参汤,贾瑞固亦不能微好,又岂能望好,但凤姐之毒何如是?终是瑞之自失也。\end{note}
\end{parag}


\begin{parag}
    那贾瑞此时要命心胜,无药不吃,只是白花钱,不见效。忽然这日有个跛足道人\begin{note}庚双夹:自甄士隐随君一去,别来无恙否?\end{note}来化斋,口称专治冤业之症。贾瑞偏生在内就听见了,直著声叫喊\begin{note}庚双夹:如闻其声,吾不忍听也。\end{note}说:“快请进那位菩萨来救我!”一面叫,一面在枕上叩首。\begin{note}庚双夹:如见其形,吾不忍看也。\end{note}众人只得带了那道士进来。贾瑞一把拉住,连叫:“菩萨救我!”\begin{note}庚双夹:人之将死,其言也哀,作者如何下笔?\end{note}那道士叹道:“你这病非药可医!我有个宝贝与你,你天天看时,此命可保矣。”说毕,从褡裢中\begin{note}庚双夹:妙极!此褡裢犹是士隐所抢背者乎?\end{note}取出一面镜子来\begin{note}庚双夹:凡看书人从此细心体贴,方许你看,否则此书哭矣。\end{note}——两面皆可照人,\begin{note}庚双夹:此书表里皆有喻也。\end{note}镜把上面錾著“风月宝鉴”四字\begin{note}庚双夹:明点。\end{note}——递与贾瑞道:“这物出自太虚幻境空灵殿上,警幻仙子所制,\begin{note}庚双夹:言此书原系空虚幻设。\end{note}\begin{note}庚眉:与“红楼梦”呼应。\end{note}专治邪思妄动之症,\begin{note}庚双夹:毕真。\end{note}有济世保生之功。\begin{note}庚双夹:毕真。\end{note}所以带他到世上,单与那些聪明俊杰、风雅王孙等看照。\begin{note}庚双夹:所谓无能纨绔是也。\end{note}千万不可照正面,\begin{note}庚侧:谁人识得此句!\end{note}\begin{note}庚双夹:观者记之,不要看这书正面,方是会看。\end{note}只照他的背面,\begin{note}庚双夹:记之。\end{note}要紧,要紧!三日后吾来收取,管叫你好了。”说毕,佯常而去,众人苦留不住。
\end{parag}


\begin{parag}
    贾瑞收了镜子,想道:“这道士倒有些意思,我何不照一照试试。”想毕,拿起“风月鉴”来,向反面一照,只见一个骷髅立在里面,\begin{note}庚双夹:所谓“好知青冢骷髅骨,就是红楼掩面人”是也。作者好苦心思。\end{note}唬得贾瑞连忙掩了,骂:“道士混账,如何吓我!”“我倒再照照正面是什么。”想著,又将正面一照,只见凤姐站在里面招手\begin{note}庚侧:可怕是“招手”二字。\end{note}叫他。\begin{note}庚双夹:奇绝!\end{note}贾瑞心中一喜,荡悠悠的觉得进了镜子,\begin{note}庚双夹:写得奇峭,真好笔墨。\end{note}与凤姐云雨一番,凤姐仍送他出来。到了床上,“嗳哟”了一声,一睁眼,镜子从手里掉过来,仍是反面立著一个骷髅。贾瑞自觉汗津津的,底下已遗了一滩精。\begin{note}蒙侧:此一句力如龙象,意谓:正面你方才已自领略了,你也当思想反面才是。\end{note}心中到底不足,又翻过正面来,只见凤姐还招手叫他,他又进去。如此三四次。到了这次,刚要出镜子来,只见两个人走来,拿铁锁把他套住,拉了就走。\begin{note}庚双夹:所谓醉生梦死也。\end{note}贾瑞叫道:“让我拿了镜子再走!”\begin{note}庚双夹:可怜!大众齐来看此。\end{note}\begin{note}蒙侧:这是作书者之立意,要写情种,故于此试一深写之。在贾瑞则是求仁而得仁,未尝不含笑九泉,虽死亦不解脱者,悲矣!\end{note}——只说了这句,就再不能说话了。
\end{parag}


\begin{parag}
    旁边伏侍的贾瑞的众人,只见他先还拿著镜子照,落下来,仍睁开眼拾在手内,末后镜子落下来便不动了。众人上来看看,已没了气,身子底下冰凉渍湿一大滩精,这才忙著穿衣抬床。代儒夫妇哭的死去活来,大骂道士,“是何妖镜!\begin{note}庚双夹:此书不免腐儒一谤。\end{note}若不早毁此物,\begin{note}庚双夹:凡野史俱可毁,独此书不可毁。\end{note}遗害于世不小。”\begin{note}庚双夹:腐儒。\end{note}遂命架火来烧,只听镜内哭道:“谁叫你们瞧正面了!你们自己以假为真,何苦来烧我?”\begin{note}庚双夹:观者记之。\end{note}正哭著,只见那跛足道人从外跑来,喊道:“谁毁‘风月鉴’,吾来救也!”说著,直入中堂,抢入手内,飘然去了。
\end{parag}


\begin{parag}
    当下,代儒料理丧事,各处去报丧。三日起经,七日发引,寄灵于铁槛寺,\begin{note}庚双夹:所谓“铁门限”事业。先安一开路道之人,以备秦氏仙柩有方也。\end{note}日后带回原籍。当下贾家众人齐来吊问,荣府贾赦赠银二十两,贾政亦是二十两,宁国府贾珍亦有二十两,别者族中人贫富不等,或三两五两,不可胜数。另有各同窗家分资,也凑了二三十两。代儒家道虽然淡薄,倒也丰丰富富完了此事。
\end{parag}


\begin{parag}
    谁知这年冬底,林如海的书信寄来,却为身染重疾,写书特来接林黛玉回去。\begin{note}蒙侧:须要林黛玉长住,偏要暂离。\end{note}贾母听了,未免又加忧闷,只得忙忙的打点黛玉起身。宝玉大不自在,争奈父女之情,也不好拦劝。于是贾母定要贾琏送他去,仍叫带回来。一应土仪盘缠,不消烦说,自然要妥贴。作速择了日期,贾琏与林黛玉辞别了贾母等,带领仆从,登舟往扬州去了。要知端的,且听下回分解。
\end{parag}


\begin{parag}
    \begin{note}庚:此回忽遣黛玉去者,正为下回可儿之文也。若不遣去,只写可儿、阿凤等人,却置黛玉于荣府,成何文哉?故必遣去,方好放笔写秦,方不脱发。况黛玉乃书中正人,秦为陪客,岂因陪而失正耶?后大观园方是宝玉、宝钗、黛玉等正经文字,前皆系陪衬之文也。\end{note}
\end{parag}


\begin{parag}
    \begin{note}蒙回末总评:儒家正心,道者炼心,释辈戒心。可见此心无有不到,无不能入者,独畏其入于邪而不反,故用正炼戒以缚之。请看贾瑞一起念,及至于死,专诚不二,虽经两次警教,毫无反悔,可谓痴子,可谓愚情。相乃可思,不能相而独欲思,岂逃倾颓?作者以此作一新样情理,以助解者生笑,以为痴者设以棒喝耳!\end{note}
\end{parag}