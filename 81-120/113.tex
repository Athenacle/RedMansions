\chap{一百一十三}{忏宿冤凤姐托村妪 释旧憾情婢感痴郎}



\begin{parag}
    话说赵姨娘在寺内得了暴病,见人少了,更加混说起来,唬得众人都恨,就有两个女人搀着。赵姨娘双膝跪在地下,说一回,哭一回,有时爬在地下叫饶,说:“打杀我了!红胡子的老爷,我再不敢了。”有一时双手合着,也是叫疼。眼睛突出,嘴里鲜血直流,头发披散,人人害怕,不敢近前。那时又将天晚,赵姨娘的声音只管喑哑起来了,居然鬼嚎一般。无人敢在他跟前,只得叫了几个有胆量的男人进来坐着,赵姨娘一时死去,隔了些时又回过来,整整的闹了一夜。到了第二天,也不言语,只装鬼脸,自己拿手撕开衣服,露出胸膛,好象有人剥他的样子。可怜赵姨娘虽说不出来,其痛苦之状实在难堪。正在危急,大夫来了,也不敢诊,只嘱咐“办理后事罢”,说了起身就走。那送大夫的家人再三央告说:“请老爷看看脉,小的好回禀家主。”那大夫用手一摸,已无脉息。贾环听了,然后大哭起来。众人只顾贾环,谁料理赵姨娘。只有周姨娘心里苦楚,想到:“做偏房侧室的下场头不过如此!况他还有儿子的,我将来死起来还不知怎样呢!”于是反哭的悲切。且说那人赶回家去回禀了。贾政即派家人去照例料理,陪着环儿住了三天,一同回来。
\end{parag}


\begin{parag}
    那人去了,这里一人传十,十人传百,都知道赵姨娘使了毒心害人被阴司里拷打死了。又说是“琏二奶奶只怕也好不了,怎么说琏二奶奶告的呢。”这些话传到平儿耳内,甚是着急,看着凤姐的样子实在是不能好的了,看着贾琏近日并不似先前的恩爱,本来事也多,竟象不与他相干的。平儿在凤姐跟前只管劝慰,又想着邢王二夫人回家几日,只打发人来问问,并不亲身来看。凤姐心里更加悲苦。贾琏回来也没有一句贴心的话。凤姐此时只求速死,心里一想,邪魔悉至。只见尤二姐从房后走来,渐近床前说:“姐姐,许久的不见了。做妹妹的想念的很,要见不能,如今好容易进来见见姐姐。姐姐的心机也用尽了,咱们的二爷糊涂,也不领姐姐的情,反倒怨姐姐作事过于苛刻,把他的前程去了,叫他如今见不得人。我替姐姐气不平。”凤姐恍惚说道:“我如今也后悔我的心忒窄了,妹妹不念旧恶,还来瞧我。”平儿在旁听见,说道:“奶奶说什么?”凤姐一时苏醒,想起尤二姐已死,必是他来索命。被平儿叫醒,心里害怕,又不肯说出,只得勉强说道:“我神魂不定,想是说梦话。给我捶捶。”平儿上去捶着,见个小丫头子进来,说是“刘姥姥来了,婆子们带着来请奶奶的安。”平儿急忙下来说:“在那里呢?”小丫头子说:“他不敢就进来,还听奶奶的示下。”平儿听了点头,想凤姐病里必是懒待见人,便说道:“奶奶现在养神呢,暂且叫他等着。你问他来有什么事么?”小丫头子说道:“他们问过了,没有事。说知道老太太去世了,因没有报才来迟了。”小丫头子说着,凤姐听见,便叫“平儿,你来,人家好心来瞧,不要冷淡人家。你去请了刘姥姥进来,我和他说说话儿。”平儿只得出来请刘姥姥这里坐。
\end{parag}


\begin{parag}
    凤姐刚要合眼,又见一个男人一个女人走向炕前,就象要上炕似的。凤姐着忙,便叫平儿说:“那里来了一个男人跑到这里来了!”连叫两声,只见丰儿小红赶来说:“奶奶要什么?”凤姐睁眼一瞧,不见有人,心里明白,不肯说出来,便问丰儿道:“平儿这东西那里去了?”丰儿道:“不是奶奶叫去请刘姥姥去了么。”凤姐定了一会神,也不言语。
\end{parag}


\begin{parag}
    只见平儿同刘姥姥带了一个小女孩儿进来,说:“我们姑奶奶在那里?”平儿引到炕边,刘姥姥便说:“请姑奶奶安。”凤姐睁眼一看,不觉一阵伤心,说:“姥姥你好?怎么这时候才来?你瞧你外孙女儿也长的这么大了。”刘姥姥看着凤姐骨瘦如柴,神情恍惚,心里也就悲惨起来,说:“我的奶奶,怎么这几个月不见,就病到这个分儿。我糊涂的要死,怎么不早来请姑奶奶的安!”便叫青儿给姑奶奶请安。青儿只是笑,凤姐看了倒也十分喜欢,便叫小红招呼着。刘姥姥道:“我们屯乡里的人不会病的,若一病了就要求神许愿,从不知道吃药的。我想姑奶奶的病不要撞着什么了罢?”平儿听着那话不在理,便在背地里扯他。刘姥姥会意,便不言语。那里知道这句话倒合了凤姐的意,扎挣着说:“姥姥你是有年纪的人,说的不错。你见过的赵姨娘也死了,你知道么?”刘姥姥诧异道:“阿弥陀佛!好端端一个人怎么就死了?我记得他也有一个小哥儿,这便怎么样呢?”平儿道:“这怕什么,他还有老爷太太呢。”刘姥姥道:“姑娘,你那里知道,不好死了是亲生的,隔了肚皮子是不中用的。”这句话又招起凤姐的愁肠,呜呜咽咽的哭起来了。众人都来劝解。
\end{parag}


\begin{parag}
    巧姐儿听见他母亲悲哭,便走到炕前用手拉着凤姐的手,也哭起来。凤姐一面哭着道:“你见过了姥姥了没有?”巧姐儿道:“没有。”凤姐道:“你的名字还是他起的呢,就和干娘一样,你给他请个安。”巧姐儿便走到跟前,刘姥姥忙着拉着道:“阿弥陀佛,不要折杀我了!巧姑娘,我一年多不来,你还认得我么?”巧姐儿道:“怎么不认得。那年在园里见的时候我还小,前年你来,我还合你要来年的蝈蝈儿,你也没有给我,必是忘了。”刘姥姥道:“好姑娘,我是老糊涂了。若说蝈蝈儿,我们屯里多得很,只是不到我们那里去,若去了,要一车也容易。”凤姐道:“不然你带了他去罢。”刘姥姥笑道:“姑娘这样千金贵体,绫罗裹大了的,吃的是好东西,到了我们那里,我拿什么哄他顽,拿什么给他吃呢?这倒不是坑杀我了么。”说着,自己还笑,他说:“那么着,我给姑娘做个媒罢。我们那里虽说是屯乡里,也有大财主人家,几千顷地,几百牲口,银子钱亦不少,只是不象这里有金的,有玉的。姑奶奶是瞧不起这种人家,我们庄家人瞧着这样大财主,也算是天上的人了。”凤姐道:“你说去,我愿意就给。”刘姥姥道:“这是顽话儿罢咧。放着姑奶奶这样,大官大府的人家只怕还不肯给,那里肯给庄家人。就是姑奶奶肯了,上头太太们也不给。”巧姐因他这话不好听,便走了去和青儿说话。两个女孩儿倒说得上,渐渐的就熟起来了。
\end{parag}


\begin{parag}
    这里平儿恐刘姥姥话多,搅烦了凤姐,便拉了刘姥姥说:“你提起太太来,你还没有过去呢。我出去叫人带了你去见见,也不枉来这一趟。”刘姥姥便要走。凤姐道:“忙什么,你坐下,我问你近来的日子还过的么?”刘姥姥千恩万谢的说道:“我们若不仗着姑奶奶”,说着,指着青儿说:“他的老子娘都要饿死了。如今虽说是庄家人苦,家里也挣了好几亩地,又打了一眼井,种些菜蔬瓜果,一年卖的钱也不少,尽够他们嚼吃的了。这两年姑奶奶还时常给些衣服布匹,在我们村里算过得的了。阿弥陀佛,前日他老子进城,听见姑奶奶这里动了家,我就几乎唬杀了。亏得又有人说不是这里,我才放心。后来又听见说这里老爷升了,我又喜欢,就要来道喜,为的是满地的庄家来不得。昨日又听说老太太没有了,我在地里打豆子,听见了这话,唬得连豆子都拿不起来了,就在地里狠狠的哭了一大场。我和女婿说,我也顾不得你们了,不管真话谎话,我是要进城瞧瞧去的。我女儿女婿也不是没良心的,听见了也哭了一回子,今儿天没亮就赶着我进城来了。我也不认得一个人,没有地方打听,一径来到后门,见是门神都糊了,我这一唬又不小。进了门找周嫂子,再找不着,撞见一个小姑娘,说周嫂子他得了不是了,撵了。我又等了好半天,遇见了熟人,才得进来。不打谅姑奶奶也是那么病。”说着,又掉下泪来。平儿等着急,也不等他说完拉着就走,说:“你老人家说了半天,口干了,咱们喝碗茶去罢。”拉着刘姥姥到下房坐着,青儿在巧姐儿那边。刘姥姥道:“茶倒不要。好姑娘,叫人带了我去请太太的安,哭哭老太太去罢。”平儿道:“你不用忙,今儿也赶不出城的了。方才我是怕你说话不防头招的我们奶奶哭,所以催你出来的。别思量。”刘姥姥道:“阿弥陀佛,姑娘是你多心,我知道。倒是奶奶的病怎么好呢?”平儿道:“你瞧去妨碍不妨碍?”刘姥姥道:“说是罪过,我瞧着不好。”正说着,又听凤姐叫呢。平儿及到床前,凤姐又不言语了。平儿正问丰儿,贾琏进来,向炕上一瞧,也不言语,走到里间气哼哼的坐下。只有秋桐跟了进去,倒了茶,殷勤一回,不知嘁嘁喳喳的说些什么。回来贾琏叫平儿来问道:“奶奶不吃药么?”平儿道:“不吃药。怎么样呢?”贾琏道:“我知道么!你拿柜子上的钥匙来罢。”平儿见贾琏有气,又不敢问,只得出来凤姐耳边说了一声。凤姐不言语,平儿便将一个匣子搁在贾琏那里就走。贾琏道:“有鬼叫你吗!你搁着叫谁拿呢?”平儿忍气打开,取了钥匙开了柜子,便问道:“拿什么?”贾琏道:“咱们有什么吗?”平儿气得哭道:“有话明白说,人死了也愿意!”贾琏道:“还要说么!头里的事是你们闹的。如今老太太的还短了四五千银子,老爷叫我拿公中的地帐弄银子,你说有么?外头拉的帐不开发使得么?谁叫我应这个名儿!只好把老太太给我的东西折变去罢了。你不依么?”平儿听了,一句不言语,将柜里东西搬出。只见小红过来说:“平姐姐快走,奶奶不好呢。”平儿也顾不得贾琏,急忙过来,见凤姐用手空抓,平儿用手攥着哭叫。贾琏也过来一瞧,把脚一跺道:“若是这样,是要我的命了。”说着,掉下泪来。丰儿进来说:“外头找二爷呢。”贾琏只得出去。
\end{parag}


\begin{parag}
    这里凤姐愈加不好,丰儿等不免哭起来。巧姐听见赶来。刘姥姥也急忙走到炕前,嘴里念佛,捣了些鬼,果然凤姐好些。一时王夫人听了丫头的信,也过来了,先见凤姐安静些,心下略放心,见了刘姥姥,便说:“刘姥姥你好?什么时候来的?”刘姥姥便说:“请太太安。”不及细说,只言凤姐的病。讲究了半天,彩云进来说:“老爷请太太呢。”王夫人叮咛了平儿几句话,便过去了。凤姐闹了一回,此时又觉清楚些,见刘姥姥在这里,心里信他求神祷告,便把丰儿等支开,叫刘姥姥坐在头边,告诉他心神不宁如见鬼怪的样。刘姥姥便说我们屯里什么菩萨灵,什么庙有感应。凤姐道:“求你替我祷告,要用供献的银钱我有。”便在手腕上褪下一支金镯子来交给他。刘姥姥道:“姑奶奶,不用那个。我们村庄人家许了愿,好了,花上几百钱就是了,那用这些。就是我替姑奶奶求去,也是许愿。等姑奶奶好了,要花什么自己去花罢。”凤姐明知刘姥姥一片好心,不好勉强,只得留下,说:“姥姥,我的命交给你了。我的巧姐儿也是千灾百病的,也交给你了。”刘姥姥顺口答应,便说:“这么着,我看天气尚早,还赶得出城去,我就去了。明儿姑奶奶好了,再请还愿去。”凤姐因被众冤魂缠绕害怕,巴不得他就去,便说:“你若肯替我用心,我能安稳睡一觉,我就感激你了。你外孙女儿叫他在这里住下罢。”刘姥姥道:“庄家孩子没有见过世面,没的在这里打嘴。我带他去的好。”凤姐道:“这就是多心了。既是咱们一家,这怕什么。虽说我们穷了,这一个人吃饭也不碍什么。”刘姥姥见凤姐真情,落得叫青儿住几天,又省了家里的嚼吃。只怕青儿不肯,不如叫他来问问,若是他肯,就留下。于是和青儿说了几句。青儿因与巧姐儿顽得熟了,巧姐又不愿他去,青儿又愿意在这里。刘姥姥便吩咐了几句,辞了平儿,忙忙的赶出城去。不题。
\end{parag}


\begin{parag}
    且说栊翠庵原是贾府的地址,因盖省亲园子,将那庵圈在里头,向来食用香火并不动贾府的钱粮。今日妙玉被劫,那女尼呈报到官,一则候官府缉盗的下落,二则是妙玉基业不便离散,依旧住下。不过回明了贾府。那时贾府的人虽都知道,只为贾政新丧,且又心事不宁,也不敢将这些没要紧的事回禀。只有惜春知道此事,日夜不安。渐渐传到宝玉耳边,说妙玉被贼劫去,又有的说妙玉凡心动了跟人而走。宝玉听得十分纳闷,想来必是被强徒抢去,这个人必不肯受,一定不屈而死。但是一无下落,心下甚不放心,每日长嘘短叹。还说:“这样一个人自称为‘槛外人’,怎么遭此结局!”又想到:“当日园中何等热闹,自从二姐姐出阁以来,死的死,嫁的嫁,我想他一尘不染是保得住的了,岂知风波顿起,比林妹妹死的更奇!”由是一而二,二而三,追思起来,想到《庄子》上的话,虚无缥缈,人生在世,难免风流云散,不禁的大哭起来。袭人等又道是他的疯病发作,百般的温柔解劝。宝钗初时不知何故,也用话箴规。怎奈宝玉抑郁不解,又觉精神恍惚。宝钗想不出道理,再三打听,方知妙玉被劫不知去向,也是伤感,只为宝玉愁烦,便用正言解释。因提起“兰儿自送殡回来,虽不上学,闻得日夜攻苦。他是老太太的重孙,老太太素来望你成人,老爷为你日夜焦心,你为闲情痴意糟蹋自己,我们守着你如何是个结果!”说得宝玉无言可答,过了一回才说道:“我那管人家的闲事,只可叹咱们家的运气衰颓。”宝钗道:“可又来,老爷太太原为是要你成人,接续祖宗遗绪。你只是执迷不悟,如何是好。”宝玉听来,话不投机,便靠在桌上睡去。宝钗也不理他,叫麝月等伺候着,自己却去睡了。
\end{parag}


\begin{parag}
    宝玉见屋里人少,想起:“紫鹃到了这里,我从没合他说句知心的话儿,冷冷清清撂着他,我心里甚不过意。他呢,又比不得麝月秋纹,我可以安放得的。想起从前我病的时候,他在我这里伴了好些时,如今他的那一面小镜子还在我这里,他的情义却也不薄了。如今不知为什么,见我就是冷冷的。若说为我们这一个呢,他是和林妹妹最好的,我看他待紫鹃也不错。我有不在家的日子,紫鹃原与他有说有讲的,到我来了,紫鹃便走开了。想来自然是为林妹妹死了我便成了家的原故。嗳,紫鹃,紫鹃,你这样一个聪明女孩儿,难道连我这点子苦处都看不出来么!”因又一想:“今晚他们睡的睡,做活的做活,不如趁着这个空儿我找他去,看他有什么话。倘或我还有得罪之处,便陪个不是也使得。”想定主意,轻轻的走出了房门,来找紫鹃。
\end{parag}


\begin{parag}
    那紫鹃的下房也就在西厢里间。宝玉悄悄的走到窗下,只见里面尚有灯光,便用舌头舔破窗纸往里一瞧,见紫鹃独自挑灯,又不是做什么,呆呆的坐着。宝玉便轻轻的叫道:“紫鹃姐姐还没有睡么?”紫鹃听了唬了一跳,怔怔的半日才说:“是谁?”宝玉道:“是我。”紫鹃听着,似乎是宝玉的声音,便问:“是宝二爷么?”宝玉在外轻轻的答应了一声。紫鹃问道:“你来做什么?”宝玉道:“我有一句心里的话要和你说说,你开了门,我到你屋里坐坐。”紫鹃停了一会儿说道:“二爷有什么话,天晚了,请回罢,明日再说罢。”宝玉听了,寒了半截。自己还要进去,恐紫鹃未必开门,欲要回去,这一肚子的隐情,越发被紫鹃这一句话勾起。无奈,说道:“我也没有多余的话,只问你一句。”紫鹃道:“既是一句,就请说。”宝玉半日反不言语。紫鹃在屋里不见宝玉言语,知他素有痴病,恐怕一时实在抢白了他,勾起他的旧病倒也不好了,因站起来细听了一听,又问道:“是走了,还是傻站着呢?有什么又不说,尽着在这里怄人。已经怄死了一个,难道还要怄死一个么!这是何苦来呢!”说着,也从宝玉舔破之处往外一张,见宝玉在那里呆听。紫鹃不便再说,回身剪了剪烛花。忽听宝玉叹了一声道:“紫鹃姐姐,你从来不是这样铁心石肠,怎么近来连一句好好儿的话都不和我说了?我固然是个浊物,不配你们理我,但只我有什么不是,只望姐姐说明了,那怕姐姐一辈子不理我,我死了倒作个明白鬼呀!”紫鹃听了,冷笑道:“二爷就是这个话呀,还有什么?若就是这个话呢,我们姑娘在时我也跟着听俗了!若是我们有什么不好处呢,我是太太派来的,二爷倒是回太太去,左右我们丫头们更算不得什么了。”说到这里,那声儿便哽咽起来,说着又醒鼻涕,宝玉在外知他伤心哭了,便急的跺脚道:“这是怎么说,我的事情你在这里几个月还有什么不知道的。就便别人不肯替我告诉你,难道你还不叫我说,叫我憋死了不成!”说着,也呜咽起来了。
\end{parag}


\begin{parag}
    宝玉正在这里伤心,忽听背后一个人接言道:“你叫谁替你说呢?谁是谁的什么?自己得罪了人自己央及呀,人家赏脸不赏在人家,何苦来拿我们这些没要紧的垫喘儿呢。”这一句话把里外两个人都吓了一跳。你道是谁,原来却是麝月。宝玉自觉脸上没趣。只见麝月又说道:“到底是怎么着?一个陪不是,一个人又不理。你倒是快快的央及呀。嗳,我们紫鹃姐姐也就太狠心了,外头这么怪冷的,人家央及了这半天,总连个活动气儿也没有。”又向宝玉道:“刚才二奶奶说了,多早晚了,打量你在那里呢,你却一个人站在这房檐底下做什么!”紫鹃里面接着说道:“这可是什么意思呢?早就请二爷进去,有话明日说罢。这是何苦来!”宝玉还要说话,因见麝月在那里,不好再说别的,只得一面同麝月走回,一面说道:“罢了,罢了!我今生今世也难剖白这个心了!惟有老天知道罢了!”说到这里,那眼泪也不知从何处来的,滔滔不断了。麝月道:“二爷,依我劝你死了心罢,白陪眼泪也可惜了儿的。”宝玉也不答言,遂进了屋子。只见宝钗睡了,宝玉也知宝钗装睡。却是袭人说了一句道:“有什么话明日说不得,巴巴儿的跑那里去闹,闹出——”说到这里也就不肯说,迟了一迟才接着道:“身上不觉怎么样?”宝玉也不言语,只摇摇头儿,袭人一面才打发睡下。一夜无眠,自不必说。
\end{parag}


\begin{parag}
    这里紫鹃被宝玉一招,越发心里难受,直直的哭了一夜。思前想后,“宝玉的事,明知他病中不能明白,所以众人弄鬼弄神的办成了。后来宝玉明白了,旧病复发,常时哭想,并非忘情负义之徒。今日这种柔情,一发叫人难受,只可怜我们林姑娘真真是无福消受他。如此看来,人生缘分都有一定,在那未到头时,大家都是痴心妄想。乃至无可如何,那糊涂的也就不理会了,那情深义重的也不过临风对月,洒泪悲啼。可怜那死的倒未必知道,这活的真真是苦恼伤心,无休无了。算来竟不如草木石头,无知无觉,倒也心中干净!”想到此处,倒把一片酸热之心一时冰冷了。才要收拾睡时,只听东院里吵嚷起来。未知何事,下回分解。
\end{parag}