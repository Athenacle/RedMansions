\chap{九十四}{宴海棠賈母賞花妖 失寶玉通靈知奇禍}



\begin{parag}
    話說賴大帶了賈芹出來,一宿無話,靜候賈政回來。單是那些女尼女道重進園來,都喜歡的了不得,欲要到各處逛逛,明日預備進宮。不料賴大便吩咐了看院的婆子並小廝看守,惟給了些飲食,卻是一步不準走開。那些女孩子摸不着頭腦,只得坐着等到天亮。園裏各處的丫頭雖都知道拉進女尼們來預備宮裏使喚,卻也不能深知原委。
\end{parag}


\begin{parag}
    到了明日早起,賈政正要下班,因堂上發下兩省城工估銷冊子立刻要查覈,一時不能回家,便叫人告訴賈璉說:“賴大回來,你務必查問明白。該如何辦就如何辦了,不必等我。”賈璉奉命,先替芹兒喜歡,又想道:若是辦得一點影兒都沒有,又恐賈政生疑,”不如回明二太太討個主意辦去,便是不合老爺的心,我也不至甚擔干係。”主意定了,進內去見王夫人,陳說:“昨日老爺見了揭帖生氣,把芹兒和女尼女道等都叫進府來查辦。今日老爺沒空問這種不成體統的事,叫我來回太太,該怎麼便怎麼樣。我所以來請示太太,這件事如何辦理?”王夫人聽了,詫異道:“這是怎麼說!若是芹兒這麼樣起來,這還成咱們家的人了麼!但只這個貼帖兒的也可惡,這些話可是混嚼說得的麼。你到底問了芹兒有這件事沒有呢?”賈璉道:“剛纔也問過了。太太想,別說他幹了沒有,就是幹了,一個人幹了混賬事也肯應承麼?但只我想芹兒也不敢行此事,知道那些女孩子都是娘娘一時要叫的,倘或鬧出事來,怎麼樣呢?依侄兒的主見,要問也不難,若問出來,太太怎麼個辦法呢?”王夫人道:“如今那些女孩子在那裏?”賈璉道:“都在園裏鎖着呢。”王夫人道:“姑娘們知道不知道?”賈璉道:“大約姑娘們也都知道是預備宮裏頭的話,外頭並沒提起別的來。”王夫人道:“很是。這些東西一刻也是留不得的。頭裏我原要打發他們去來着,都是你們說留着好,如今不是弄出事來了麼。你竟叫賴大那些人帶去,細細的問他的本家有人沒有,將文書查出,花上幾十兩銀子,僱只船,派個妥當人送到本地,一概連文書發還了,也落得無事。若是爲着一兩個不好,個個都押着他們還俗,那又太造孽了。若在這裏發給官媒,雖然我們不要身價,他們弄去賣錢,那裏顧人的死活呢。芹兒呢,你便狠狠的說他一頓。除了祭祀喜慶,無事叫他不用到這裏來,看仔細碰在老爺氣頭兒上,那可就吃不了兜着走了。並說與賬房兒裏,把這一項錢糧檔子銷了。還打發個人到水月庵,說老爺的諭:除了上墳燒紙,若有本家爺們到他那裏去,不許接待。若再有一點不好風聲,連老姑子一併攆出去。”
\end{parag}


\begin{parag}
    賈璉一一答應了,出去將王夫人的話告訴賴大,說:“是太太主意,叫你這麼辦去。辦完了,告訴我去回太太。你快辦去罷。回來老爺來,你也按着太太的話回去。”賴大聽說,便道:“我們太太真正是個佛心。這班東西着人送回去。既是太太好心,不得不挑個好人。芹哥兒竟交給二爺開發了罷。那個貼帖兒的,奴才想法兒查出來,重重的收拾他纔好。”賈璉點頭說:“是了。”即刻將賈芹發落。賴大也趕着把女尼等領出,按着主意辦去了。晚上賈政回家,賈璉賴大回明賈政。賈政本是省事的人,聽了也便撂開手了。獨有那些無賴之徒,聽得賈府發出二十四個女孩子出來,那個不想。究竟那些人能夠回家不能,未知着落,亦難虛擬。
\end{parag}


\begin{parag}
    且說紫鵑因黛玉漸好,園中無事,聽見女尼等預備宮內使喚,不知何事,便到賈母那邊打聽打聽,恰遇着鴛鴦下來,閒着坐下說閒話兒,提起女尼的事。鴛鴦詫異道:“我並沒有聽見,回來問問二奶奶就知道了。”正說着,只見傅試家兩個女人過來請賈母的安,鴛鴦要陪了上去。那兩個女人因賈母正睡晌覺,就與鴛鴦說了一聲兒回去了。紫鵑問:“這是誰家差來的?”鴛鴦道:“好討人嫌。家裏有了一個女孩兒生得好些,便獻寶的似的,常常在老太太面前誇他家姑娘長得怎麼好,心地怎麼好,禮貌上又能,說話兒又簡絕,做活計兒手兒又巧,會寫會算,尊長上頭最孝敬的,就是待下人也是極和平的。來了就編這麼一大套,常常說給老太太聽。我聽着很煩。這幾個老婆子真討人嫌。我們老太太偏愛聽那些個話。老太太也罷了,還有寶玉,素常見了老婆子便很厭煩的,偏見了他們家的老婆子便不厭煩。你說奇不奇!前兒還來說,他們姑娘現有多少人家兒來求親,他們老爺總不肯應,心裏只要和咱們這種人家作親才肯。一回誇獎,一回奉承,把老太太的心都說活了。”紫鵑聽了一呆,便假意道:“若老太太喜歡,爲什麼不就給寶玉定了呢?”鴛鴦正要說出原故,聽見上頭說:“老太太醒了。”鴛鴦趕着上去。
\end{parag}


\begin{parag}
    紫鵑只得起身出來,回到園裏。一頭走,一頭想道:“天下莫非只有一個寶玉,你也想他,我也想他。我們家的那一位越發癡心起來了,看他的那個神情兒,是一定在寶玉身上的了。三番五次的病,可不是爲着這個是什麼!這家裏金的銀的還鬧不清,若添了一個什麼傅姑娘,更了不得了。我看寶玉的心也在我們那一位的身上,聽着鴛鴦的說話竟是見一個愛一個的。這不是我們姑娘白操了心了嗎?”紫鵑本是想着黛玉,往下一想,連自己也不得主意了,不免掉下淚來。要想叫黛玉不用瞎操心呢,又恐怕他煩惱,若是看着他這樣,又可憐見兒的。左思右想,一時煩躁起來,自己啐自己道:“你替人耽什麼憂!就是林姑娘真配了寶玉,他的那性情兒也是難伏侍的。寶玉性情雖好,又是貪多嚼不爛的。我倒勸人不必瞎操心,我自己纔是瞎操心呢。從今以後,我盡我的心伏侍姑娘,其餘的事全不管!”這麼一想,心裏倒覺清淨。回到瀟湘館來,見黛玉獨自一人坐在炕上,理從前做過的詩文詞稿。抬頭見紫鵑來,便問:“你到那裏去了?”紫鵑道:“我今兒瞧了瞧姐妹們去。”黛玉道:“敢是找襲人姐姐去麼?”紫鵑道:“我找他做什麼。”黛玉一想這話,怎麼順嘴說了出來,反覺不好意思,便啐道:“你找誰與我什麼相干!倒茶去罷。”
\end{parag}


\begin{parag}
    紫鵑也心裏暗笑,出來倒茶。只聽見園裏的一迭聲亂嚷,不知何故,一面倒茶,一面叫人去打聽。回來說道:“怡紅院裏的海棠本來萎了幾棵,也沒人去澆灌他。昨日寶玉走去,瞧見枝頭上好象有了骨朵兒似的。人都不信,沒有理他。忽然今日開得很好的海棠花,衆人詫異,都爭着去看。連老太太,太太都鬨動了來瞧花兒呢,所以大奶奶叫人收拾園裏敗葉枯枝,這些人在那裏傳喚。”黛玉也聽見了,知道老太太來,便更了衣,叫雪雁去打聽,”若是老太太來了,即來告訴我。”雪雁去不多時,便跑來說:“老太太,太太好些人都來了,請姑娘就去罷。”黛玉略自照了一照鏡子,掠了一掠鬢髮,便扶着紫鵑到怡紅院來。已見老太太坐在寶玉常臥的榻上,黛玉便說道:“請老太太安。”退後,便見了邢王二夫人,回來與李紈,探春,惜春,邢岫煙彼此問了好。只有鳳姐因病未來,史湘雲因他叔叔調任回京,接了家去,薛寶琴跟他姐姐家去住了,李家姐妹因見園內多事,李嬸孃帶了在外居住:所以黛玉今日見的只有數人。大家說笑了一回,講究這花開得古怪。賈母道:“這花兒應在三月裏開的,如今雖是十一月,因節氣遲,還算十月,應着小陽春的天氣,這花開因爲和暖是有的。”王夫人道:“老太太見的多,說得是。也不爲奇。”邢夫人道:“我聽見這花已經萎了一年,怎麼這回不應時候兒開了,必有個原故。”李紈笑道:“老太太與太太說得都是。據我的糊塗想頭,必是寶玉有喜事來了,此花先來報信。”探春雖不言語,心內想:“此花必非好兆。大凡順者昌,逆者亡。草木知運,不時而發,必是妖孽。”只不好說出來。獨有黛玉聽說是喜事,心裏觸動,便高興說道:“當初田家有荊樹一棵,三個弟兄因分了家,那荊樹便枯了。後來感動了他弟兄們仍舊在一處,那荊樹也就榮了。可知草木也隨人的。如今二哥哥認真唸書,舅舅喜歡,那棵樹也就發了。”賈母王夫人聽了喜歡,便說:“林姑娘比方得有理,很有意思。”正說着,賈赦,賈政,賈環,賈蘭都進來看花。賈赦便說:“據我的主意,把他砍去,必是花妖作怪。”賈政道:“見怪不怪,其怪自敗。不用砍他,隨他去就是了。”賈母聽見,便說:“誰在這裏混說!人家有喜事好處,什麼怪不怪的。若有好事,你們享去,若是不好,我一個人當去。你們不許混說。”賈政聽了,不敢言語,訕訕的同賈赦等走了出來。
\end{parag}


\begin{parag}
    那賈母高興,叫人傳話到廚房裏,快快預備酒席,大家賞花。叫:“寶玉,環兒,蘭兒各人做一首詩誌喜。林姑娘的病纔好,不要他費心,若高興,給你們改改。”對着李紈道:“你們都陪我喝酒。”李紈答應了”是”,便笑對探春笑道:“都是你鬧的。”探春道:“饒不叫我們做詩,怎麼我們鬧的。”李紈道:“海棠社不是你起的麼,如今那棵海棠也要來入社了。”大家聽着都笑了。一時擺上酒菜,一面喝着,彼此都要討老太太的歡喜,大家說些興頭話。寶玉上來,斟了酒,便立成了四句詩,寫出來念與賈母聽道:
\end{parag}


\begin{poem}
    \begin{pl}
        海棠何事忽摧隤,今日繁花爲底開?
    \end{pl}


    \begin{pl}
        應是北堂增壽考,一陽旋復佔先梅。
    \end{pl}

\end{poem}


\begin{parag}
    賈環也寫了來唸道:
\end{parag}

\begin{poem}
    \begin{pl}
        草木逢春當茁芽,海棠未發候偏差。
    \end{pl}


    \begin{pl}
        人間奇事知多少,冬月開花獨我家。
    \end{pl}
\end{poem}


\begin{parag}
    賈蘭恭楷謄正,呈與賈母,賈母命李紈念道:
\end{parag}


\begin{poem}


    \begin{pl}
        煙凝媚色春前萎,霜浥微紅雪後開。
    \end{pl}


    \begin{pl}
        莫道此花知識淺,欣榮預佐合歡杯。
    \end{pl}

\end{poem}

\begin{parag}
    賈母聽畢,便說:“我不大懂詩,聽去倒是蘭兒的好,環兒做得不好。都上來喫飯罷。”寶玉看見賈母喜歡,更是興頭。因想起:“晴雯死的那年海棠死的,今日海棠復榮,我們院內這些人自然都好。但是晴雯不能象花的死而復生了。”頓覺轉喜爲悲。忽又想起前日巧姐提鳳姐要把五兒補入,或此花爲他而開,也未可知,卻又轉悲爲喜,依舊說笑。
\end{parag}


\begin{parag}
    賈母還坐了半天,然後扶了珍珠回去了。王夫人等跟着過來。只見平兒笑嘻嘻的迎上來說:“我們奶奶知道老太太在這裏賞花,自己不得來,叫奴才來伏侍老太太,太太們,還有兩匹紅送給寶二爺包裹這花,當作賀禮。”襲人過來接了,呈與賈母看。賈母笑道:“偏是鳳丫頭行出點事兒來,叫人看着又體面,又新鮮,很有趣兒。”襲人笑着向平兒道:“回去替寶二爺給二奶奶道謝。要有喜大家喜。”賈母聽了笑道:“噯喲,我還忘了呢,鳳丫頭雖病着,還是他想得到,送得也巧。”一面說着,衆人就隨着去了。平兒私與襲人道:“奶奶說,這花開得奇怪,叫你鉸塊紅綢子掛掛,便應在喜事上去了。以後也不必只管當作奇事混說。”襲人點頭答應,送了平兒出去。不題。
\end{parag}


\begin{parag}
    且說那日寶玉本來穿着一裹圓的皮襖在家歇息,因見花開,只管出來看一回,賞一回,嘆一回,愛一回的,心中無數悲喜離合,都弄到這株花上去了。忽然聽說賈母要來,便去換了一件狐腋箭袖,罩一件元狐腿外褂,出來迎接賈母。匆匆穿換,未將通靈寶玉掛上。及至後來賈母去了,仍舊換衣。襲人見寶玉脖子上沒有掛着,便問:“那塊玉呢?”寶玉道:“纔剛忙亂換衣,摘下來放在炕桌上,我沒有帶。”襲人回看桌上並沒有玉,便向各處找尋,蹤影全無,嚇得襲人滿身冷汗。寶玉道:“不用着急,少不得在屋裏的。問他們就知道了。”襲人當作麝月等藏起嚇他頑,便向麝月等笑着說道:“小蹄子們,頑呢到底有個頑法。把這件東西藏在那裏了?別真弄丟了,那可就大家活不成了。”麝月等都正色道:“這是那裏的話!頑是頑笑是笑,這個事非同兒戲,你可別混說。你自己昏了心了,想想罷,想想擱在那裏了。這會子又混賴人了。”襲人見他這般光景,不象是頑話,便着急道:“皇天菩薩小祖宗,到底你擺在那裏去了?”寶玉道:“我記得明明放在炕桌上的,你們到底找啊。”襲人,麝月,秋紋等也不敢叫人知道,大家偷偷兒的各處搜尋。鬧了大半天,毫無影響,甚至翻箱倒籠,實在沒處去找,便疑到方纔這些人進來,不知誰撿了去了。襲人說道:“進來的誰不知道這玉是性命似的東西呢,誰敢撿了去呢。你們好歹先別聲張,快到各處問去。若有姐妹們撿着嚇我們頑呢,你們給他磕頭要了回來,若是小丫頭偷了去,問出來也不回上頭,不論把什麼送給他換了出來都使得的。這可不是小事,真要丟了這個,比丟了寶二爺的還利害呢。”麝月秋紋剛要往外走,襲人又趕出來囑咐道:“頭裏在這裏喫飯的倒先別問去,找不成再惹出些風波來,更不好了。”麝月等依言分頭各處追問,人人不曉,個個驚疑。麝月等回來,俱目瞪口呆,面面相窺。寶玉也嚇怔了。襲人急的只是乾哭。找是沒處找,回又不敢回,怡紅院裏的人嚇得個個象木雕泥塑一般。
\end{parag}


\begin{parag}
    大家正在發呆,只見各處知道的都來了。探春叫把園門關上,先命個老婆子帶着兩個丫頭,再往各處去尋去,一面又叫告訴衆人:若誰找出來,重重的賞銀。大家頭宗要脫干係,二宗聽見重賞,不顧命的混找了一遍,甚至於茅廝裏都找到。誰知那塊玉竟象繡花針兒一般,找了一天,總無影響。李紈急了,說:“這件事不是頑的,我要說句無禮的話了。”衆人道:“什麼呢?”李紈道:“事情到了這裏,也顧不得了。現在園裏除了寶玉,都是女人,要求各位姐姐,妹妹,姑娘都要叫跟來的丫頭脫了衣服,大家搜一搜。若沒有,再叫丫頭們去搜那些老婆子並粗使的丫頭。”大家說道:“這話也說的有理。現在人多手亂,魚龍混雜,倒是這麼一來,你們也洗洗清。”探春獨不言語。那些丫頭們也都願意洗淨自己。先是平兒起,平兒說道:“打我先搜起。”於是各人自己解懷,李紈一氣兒混搜。探春嗔着李紈道:“大嫂子,你也學那起不成材料的樣子來了。那個人既偷了去,還肯藏在身上?況且這件東西在家裏是寶,到了外頭,不知道的是廢物,偷他做什麼?我想來必是有人使促狹。”衆人聽說,又見環兒不在這裏,昨兒是他滿屋裏亂跑,都疑到他身上,只是不肯說出來。探春又道:“使促狹的只有環兒。你們叫個人去悄悄的叫了他來,背地裏哄着他,叫他拿出來,然後嚇着他,叫他不要聲張。這就完了。”大家點頭稱是。
\end{parag}


\begin{parag}
    李紈便向平兒道:“這件事還是得你去才弄得明白。”平兒答應,就趕着去了。不多時同了環兒來了。衆人假意裝出沒事的樣子,叫人沏了碗茶擱在裏間屋裏,衆人故意搭訕走開。原叫平兒哄他,平兒便笑着向環兒道:“你二哥哥的玉丟了,你瞧見了沒有?”賈環便急得紫漲了臉,瞪着眼說道:“人家丟了東西,你怎麼又叫我來查問,疑我。我是犯過案的賊麼!”平兒見這樣子,倒不敢再問,便又陪笑道:“不是這麼說,怕三爺要拿了去嚇他們,所以白問問瞧見了沒有,好叫他們找。”賈環道:“他的玉在他身上,看見不看見該問他,怎麼問我。捧着他的人多着咧!得了什麼不來問我,丟了東西就來問我!”說着,起身就走。衆人不好攔他。這裏寶玉倒急了,說道:“都是這勞什子鬧事,我也不要他了。你們也不用鬧了。環兒一去,必是嚷得滿院裏都知道了,這可不是鬧事了麼。”襲人等急得又哭道:“小祖宗,你看這玉丟了沒要緊,若是上頭知道了,我們這些人就要粉身碎骨了!”說着,便嚎啕大哭起來。
\end{parag}


\begin{parag}
    衆人更加傷感,明知此事掩飾不來,只得要商議定了話,回來好回賈母諸人。寶玉道:“你們竟也不用商議,硬說我砸了就完了。”平兒道:“我的爺,好輕巧話兒!上頭要問爲什麼砸的呢,他們也是個死啊。倘或要起砸破的碴兒來,那又怎麼樣呢?”寶玉道:“不然便說我前日出門丟了。”衆人一想,這句話倒還混得過去,但是這兩天又沒上學,又沒往別處去。寶玉道:“怎麼沒有,大前兒還到南安王府裏聽戲去了呢,便說那日丟的。”探春道:“那也不妥。既是前兒丟的,爲什麼當日不來回。”衆人正在胡思亂想,要裝點撒謊,只聽得趙姨娘的聲兒哭着喊着走來說:“你們丟了東西自己不找,怎麼叫人背地裏拷問環兒。我把環兒帶了來,索性交給你們這一起洑上水的,該殺該剮,隨你們罷。”說着,將環兒一推說:“你是個賊,快快的招罷!”氣得環兒也哭喊起來。
\end{parag}


\begin{parag}
    李紈正要勸解,丫頭來說:“太太來了。”襲人等此時無地可容,寶玉等趕忙出來迎接。趙姨娘暫且也不敢作聲,跟了出來。王夫人見衆人都有驚惶之色,纔信方纔聽見的話,便道:“那塊玉真丟了麼?”衆人都不敢作聲,王夫人走進屋裏坐下,便叫襲人。慌得襲人連忙跪下,含淚要稟。王夫人道:“你起來,快快叫人細細找去,一忙亂倒不好了。”襲人哽咽難言。寶玉生恐襲人真告訴出來,便說道:“太太,這事不與襲人相干。是我前日到南安王府那裏聽戲,在路上丟了。”王夫人道:“爲什麼那日不找?”寶玉道:“我怕他們知道,沒有告訴他們。我叫焙茗等在外頭各處找過的。”王夫人道:“胡說!如今脫換衣服不是襲人他們伏侍的麼。大凡哥兒出門回來,手巾荷包短了,還要問個明白,何況這塊玉不見了,便不問的麼!”寶玉無言可答。趙姨娘聽見,便得意了,忙接過口道:“外頭丟了東西,也賴環兒!”話未說完,被王夫人喝道:“這裏說這個,你且說那些沒要緊的話!”趙姨娘便不敢言語了。還是李紈探春從實的告訴了王夫人一遍,王夫人也急得淚如雨下,索性要回明賈母,去問邢夫人那邊跟來的這些人去。
\end{parag}


\begin{parag}
    鳳姐病中也聽見寶玉失玉,知道王夫人過來,料躲不住,便扶了豐兒來到園裏。正值王夫人起身要走,鳳姐嬌怯怯的說:“請太太安。”寶玉等過來問了鳳姐好。王夫人因說道:“你也聽見了麼,這可不是奇事嗎?剛纔眼錯不見就丟了,再找不着。你去想想,打從老太太那邊丫頭起至你們平兒,誰的手不穩,誰的心促狹。我要回了老太太,認真的查出來纔好。不然是斷了寶玉的命根子了。”鳳姐回道:“咱們家人多手雜,自古說的,‘知人知面不知心’,那裏保得住誰是好的。但是一吵嚷已經都知道了,偷玉的人若叫太太查出來,明知是死無葬身之地,他着了急,反要毀壞了滅口,那時可怎麼處呢。據我的糊塗想頭,只說寶玉本不愛他,撂丟了,也沒有什麼要緊。只要大家嚴密些,別叫老太太老爺知道。這麼說了,暗暗的派人去各處察訪,哄騙出來,那時玉也可得,罪名也好定。不知太太心裏怎麼樣?”王夫人遲了半日,才說道:“你這話雖也有理,但只是老爺跟前怎麼瞞的過呢。”便叫環兒過來道:“你二哥哥的玉丟了,白問了你一句,怎麼你就亂嚷。若是嚷破了,人家把那個毀壞了,我看你活得活不得!”賈環嚇得哭道:“我再不敢嚷了。”趙姨娘聽了,那裏還敢言語。王夫人便吩咐衆人道:“想來自然有沒找到的地方兒,好端端的在家裏的,還怕他飛到那裏去不成。只是不許聲張。限襲人三天內給我找出來,要是三天找不着,只怕也瞞不住,大家那就不用過安靜日子了。”說着,便叫鳳姐兒跟到邢夫人那邊商議踩緝。不題。
\end{parag}


\begin{parag}
    這裏李紈等紛紛議論,便傳喚看園子的一干人來,叫把園門鎖上,快傳林之孝家的來,悄悄兒的告訴了他,叫他吩咐前後門上,三天之內,不論男女下人從裏頭可以走動,要出時一概不許放出,只說裏頭丟了東西,待這件東西有了着落,然後放人出來。林之孝家的答應了”是”,因說:“前兒奴才家裏也丟了一件不要緊的東西,林之孝必要明白,上街去找了一個測字的,那人叫做什麼劉鐵嘴,測了一個字,說的很明白,回來依舊一找便找着了。”襲人聽見,便央及林家的道:“好林奶奶,出去快求林大爺替我們問問去。”那林之孝家的答應着出去了。邢岫煙道:“若說那外頭測字打卦的,是不中用的。我在南邊聞妙玉能扶乩,何不煩他問一問。況且我聽見說這塊玉原有仙機,想來問得出來。”衆人都詫異道:“咱們常見的,從沒有聽他說起。”麝月便忙問岫煙道:“想來別人求他是不肯的,好姑娘,我給姑娘磕個頭,求姑娘就去,若問出來了,我一輩子總不忘你的恩。”說着,趕忙就要磕下頭去,岫煙連忙攔住。黛玉等也都慫恿着岫煙速往櫳翠庵去。一面林之孝家的進來說道:“姑娘們大喜。林之孝測了字回來說,這玉是丟不了的,將來橫豎有人送還來的。”衆人聽了,也都半信半疑,惟有襲人麝月喜歡的了不得。探春便問:“測的是什麼字?”林之孝家的道:“他的話多,奴才也學不上來,記得是拈了個賞人東西的‘賞’字。那劉鐵嘴也不問,便說:‘丟了東西不是?’”李紈道:“這就算好。”林之孝家的道:“他還說,‘賞’字上頭一個‘小’字,底下一個‘口’字,這件東西很可嘴裏放得,必是個珠子寶石。”衆人聽了,誇讚道:“真是神仙。往下怎麼說?”林之孝家的道:“他說底下‘貝’字,拆開不成一個‘見’字,可不是‘不見’了?因上頭拆了‘當’字,叫快到當鋪裏找去。‘賞’字加一‘人’字,可不是‘償’字?只要找着當鋪就有人,有了人便贖了來,可不是償還了嗎。”衆人道:“既這麼着,就先往左近找起,橫豎幾個當鋪都找遍了,少不得就有了。咱們有了東西,再問人就容易了。”李紈道:“只要東西,那怕不問人都使得。林嫂子,煩你就把測字的話快去告訴二奶奶,回了太太,先叫太太放心。就叫二奶奶快派人查去。”林家的答應了便走。
\end{parag}


\begin{parag}
    衆人略安了一點兒神,呆呆的等岫煙回來。正呆等,只見跟寶玉的焙茗在門外招手兒,叫小丫頭子快出來。那小丫頭趕忙的出去了。焙茗便說道:“你快進去告訴我們二爺和裏頭太太奶奶姑娘們天大喜事。”那小丫頭子道:“你快說罷,怎麼這麼累贅。”焙茗笑着拍手道:“我告訴姑娘,姑娘進去回了,咱們兩個人都得賞錢呢。你打量什麼,寶二爺的那塊玉呀,我得了準信來了。”未知如何,下回分解。
\end{parag}