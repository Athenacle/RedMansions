\chap{二十七}{滴翠亭杨妃戏彩蝶 埋香冢飞燕泣残红}
\begin{parag}
    \begin{note}庚辰:《葬花吟》是大观园诸艳之归源小引,故用在践花日诸艳毕集之期。践花日不论其典与不典,只取其韵耳。\end{note}
\end{parag}


\begin{parag}
    话说林黛玉正自悲泣,忽听院门响处,只见宝钗出来了,宝玉袭人一群人送了出来。待要上去问著宝玉,又恐当著众人问羞了宝玉不便,因而闪过一旁,让宝钗去了,宝玉等进去关了门,方转过来,犹望著门洒了几点泪。\begin{note}庚辰侧批:四字闪煞颦儿也。\end{note}自觉无味,方转身回来,无精打彩的卸了残妆。
\end{parag}


\begin{parag}
    紫鹃雪雁素日知道林黛玉的情性:无事闷坐,不是愁眉,\begin{note}庚辰侧批:画美人之秘诀。\end{note}便是长叹,且好端端的不知为了什么,常常的便自泪道不干的。\begin{note}庚辰侧批:补写,却是避繁文法。\end{note}先时还有人解劝,怕他思父母,想家乡,受了委曲,只得用话宽慰解劝。谁知后来一年一月的竟常常的如此,\begin{note}甲戌侧批:补潇湘馆常文也。\end{note}把这个样儿看惯,也都不理论了。所以也没人理,由他去闷坐,\begin{note}庚辰侧批:所谓“久病床前少孝子”是也。\end{note}只管睡觉去了。那林黛玉倚著床栏杆,两手抱著膝,\begin{note}甲戌侧批:画美人秘诀。\end{note}眼睛含著泪,\begin{note}庚辰侧批:前批的画美人秘诀,今竟画出《金闺夜坐图》来了。\end{note}好似木雕泥塑\begin{note}甲戌侧批:木是旃檀,泥是金沙方可。\end{note}的一般,直坐到二更多天方才睡了。一宿无话。
\end{parag}


\begin{parag}
    至次日乃是四月二十六日,原来这日未时交芒种节。尚古风俗:凡交芒种节的这日,都要设摆各色礼物,祭饯花神,言芒种一过,便是夏日了,众花皆卸,花神退位,\begin{note}庚辰侧批:无论事之有无,看去有理。\end{note}须要饯行。然闺中更兴这件风俗,所以大观园中之人都早起来了。那些女孩子们,或用花瓣柳枝编成轿马的,或用绫锦纱罗叠成干旄旌幢的,都用彩线系了。每一颗树上,每一枝花上,都系了这些物事。满园里绣带飘飖,花枝招展,\begin{note}甲戌侧批:数句大观园景倍胜省亲一回,在一园人俱得闲闲寻乐上看,彼时只有元春一人闲耳。\end{note}\begin{note}庚辰侧批:数句抵省亲一回文字,反觉闲闲有趣有味的领略。\end{note}更兼这些人打扮得桃羞杏让,燕妒莺惭,\begin{note}甲戌侧批:桃、杏、燕、莺是这样用法。\end{note}一时也道不尽。
\end{parag}


\begin{parag}
    且说宝钗、迎春、探春、惜春、李纨、凤姐\begin{note}庚辰眉批:不写凤姐随大众一笔,见红玉一段则认为泛文矣。何一丝不漏若此。畸笏。\end{note}等并巧姐、大姐、香菱与众丫鬟们在园内玩耍,独不见林黛玉。迎春因说道:“林妹妹怎么不见?好个懒丫头!这会子还睡觉不成?”宝钗道:“你们等著,我去闹了他来。”说著便丢下了众人,一直往潇湘馆来。正走著,只见文官等十二个女孩子也来了,\begin{note}庚辰侧批:一人不漏。\end{note}上来问了好,说了一回闲话。宝钗回身指道:“他们都在那里呢,你们找他们去罢。我叫林姑娘去就来。”说著便逶迤往潇湘馆来。\begin{note}甲戌侧批:安插一处,好写一处,正一张口难说两家话也。\end{note}忽然抬头见宝玉进去了,宝钗便站住低头想了想:宝玉和林黛玉是从小儿一处长大,他兄妹间多有不避嫌疑之处,嘲笑喜怒无常;\begin{note}庚辰侧批:道尽二玉连日事。\end{note}况且林黛玉素习猜忌,好弄小性儿的。此刻自己也跟了进去,一则宝玉不便,二则黛玉嫌疑。\begin{note}甲戌侧批:道尽黛玉每每小性,全不在宝钗身上。\end{note}罢了,倒是回来的妙。想毕抽身回来。
\end{parag}


\begin{parag}
    刚要寻别的姊妹去,忽见前面一双玉色蝴蝶,大如团扇,一上一下迎风翩跹,十分有趣。宝钗意欲扑了来玩耍,遂向袖中取出扇子来,向草地下来扑。\begin{note}甲戌侧批:可是一味知书识礼女夫子行止?写宝钗无不相宜。\end{note}只见那一双蝴蝶忽起忽落,来来往往,穿花度柳,将欲过河去了。倒引的宝钗蹑手蹑脚的,一直跟到池中滴翠亭上,香汗淋漓,娇喘细细。\begin{note}庚辰侧批:若玉兄在,必有许多张罗。\end{note}宝钗也无心扑了,\begin{note}庚辰侧批:原是无可无不可。\end{note}刚欲回来,只听滴翠亭里边嘁嘁喳喳有人说话。\begin{note}甲戌侧批:无闲纸闲笔之文如此。\end{note}原来这亭子四面俱是游廊曲桥,盖造在池中水上,四面雕镂槅子糊著纸。
\end{parag}


\begin{parag}
    宝钗在亭外听见说话,便煞住脚往里细听,\begin{note}庚辰眉批:这桩风流案,又一体写法,甚当。己卯冬夜。\end{note}只听说道:“你瞧瞧这手帕子,果然是你丢的那块,你就拿著;要不是,就还芸二爷去。”又有一人说话:“可不是我那块!拿来给我罢。”又听道:“你拿什么谢我呢?难道白寻了来不成。”又答道:“我既许了谢你,自然不哄你。”又听说道:“我寻了来给你,自然谢我;但只是拣的人,你就不拿什么谢他?”又回道:“你别胡说。他是个爷们家,拣了我的东西,自然该还的。我拿什么谢他呢?”又听说道:“你不谢他,我怎么回他呢?况且他再三再四的和我说了,若没谢的,不许我给你呢。”半晌,又听答道:“也罢,拿我这个给他,算谢他的罢。──你要告诉别人呢?须说个誓来。”又听说道:“我要告诉一个人,就长一个疔,日后不得好死!”又听说道:“嗳呀!咱们只顾说话,看有人来悄悄在外头听见。\begin{note}庚辰侧批:岂敢。\end{note}\begin{note}庚辰眉批:这是自难自法,好极好极!惯用险笔如此。壬午夏,雨窗。\end{note}不如把这槅子都推开了,\begin{note}庚辰侧批:贼起飞志,不假。\end{note}便是有人见咱们在这里,他们只当我们说顽话呢。若走到跟前,咱们也看的见,就别说了。”
\end{parag}


\begin{parag}
    宝钗在外面听见这话,心中吃惊,\begin{note}甲戌侧批:四字写宝钗守身如此。\end{note}想道:“怪道从古至今那些奸淫狗盗的人,心机都不错。\begin{note}庚辰侧批:道尽矣。\end{note}这一开了,见我在这里,他们岂不臊了。况才说话的语音,大似宝玉房里的红儿的言语。他素昔眼空心大,是个头等刁钻古怪东西。今儿我听了他的短儿,一时人急造反,狗急跳墙,不但生事,而且我还没趣。如今便赶著躲了,料也躲不及,少不得要使个‘金蝉脱壳’的法子。”犹未想完,只听“咯吱”一声,宝钗便故意放重了脚步,\begin{note}庚辰侧批:闺中弱女机变,如此之便,如此之急。\end{note}笑著叫道:“颦儿,我看你往那里藏!”一面说,一面故意往前赶。那亭内的红玉坠儿刚一推窗,只听宝钗如此说著往前赶,\begin{note}庚辰眉批:此句实借红玉反写宝钗也,勿得认错作者章法。\end{note}两个人都唬怔了。宝钗反向他二人笑道:“你们把林姑娘藏在那里了?”\begin{note}庚辰侧批:像极!好煞,妙煞!焉的不拍案叫绝!\end{note}坠儿道:“何曾见林姑娘了。”宝钗道:“我才在河那边看著林姑娘在这里蹲著弄水儿的。我要悄悄的唬他一跳,还没有走到跟前,他倒看见我了,朝东一绕就不见了。别是藏在这里头了。”\begin{note}庚辰侧批:像极!是极!\end{note}一面说,一面故意进去寻了一寻,抽身就走,口内说道: “一定是又钻在山子洞里去了。遇见蛇,咬一口也罢了。”一面说一面走,心中又好笑:\begin{note}甲戌侧批:真弄婴儿,轻便如此,即余至此亦要发笑。\end{note}这件事算遮过去了,不知他二人是怎样。
\end{parag}


\begin{parag}
    谁知红玉听了宝钗的话,便信以为真,\begin{note}甲戌侧批:宝钗身份。\end{note}\begin{note}庚辰侧批:实有这一句的。\end{note}让宝钗去远,便拉坠儿道:“了不得了!林姑娘蹲在这里,一定听了话去了!”\begin{note}庚辰侧批:移东挪西,任意写去,却是真有的。\end{note}坠儿听说,也半日不言语。红玉又道:“这可怎么样呢?”\begin{note}甲戌侧批:二句系黛玉身份。\end{note}坠儿道:“便是听了,管谁筋疼,各人干各人的就完了。”\begin{note}庚辰侧批:勉强话。\end{note}红玉道:“若是宝姑娘听见,还倒罢了。林姑娘嘴里又爱刻薄人,心里又细,他一听见了,倘或走露了风声,怎么样呢?”二人正说著,只见文官、香菱、司棋、侍书等上亭子来了。二人只得掩住这话,且和他们顽笑。
\end{parag}


\begin{parag}
    只见凤姐儿站在山坡上招手叫,红玉连忙弃了众人,跑至凤姐前,笑问:“奶奶使唤作什么?”凤姐打谅了一打谅,见他生的干净俏丽,说话知趣,因笑道: “我的丫头今儿没跟进来。我这会子想起一件事来,使唤个人出去,可不知你能干不能干,说的齐全不齐全?”红玉笑道:“奶奶有什么话,只管吩咐我说去。若说不齐全,误了奶奶的事,凭奶奶责罚罢了。”\begin{note}甲戌侧批:操必胜之券。红儿机括志量,自知能应阿凤使令意。\end{note}凤姐笑道:“你是那位小姐房里的?\begin{note}庚辰侧批:反如此问。\end{note}我使出去,他回来找你,我好替你答应。”\begin{note}庚辰侧批:问那小姐为此。\end{note}红玉道:“我是宝二爷房里的。”凤姐听了笑道:“嗳哟!你原来是宝玉房里的,怪道呢,\begin{note}甲戌侧批:“哎哟”“怪道”四字,一是玉兄手下无能为者。前文打量生的“干净俏丽”四字,合而观之,小红则活现于纸上矣。\end{note}\begin{note}庚辰侧批:夸赞语也。\end{note}也罢了。你到我们家,告诉你平姐姐:外头屋里桌子上汝窑盘子架儿底下放著一卷银子,那是一百六十两,给绣匠的工价,等张材家的来要,当面称给他瞧了,再给他拿去。\begin{note}庚辰侧批:一件。\end{note}再里头床头间有一个小荷包拿了来。”\begin{note}庚辰侧批:二件。\end{note}
\end{parag}


\begin{parag}
    红玉听说撤身去了,回来只见凤姐不在这山坡子上了。因见司棋从山洞里出来,站著系裙子,\begin{note}庚辰侧批:小点缀。一笑。\end{note}便赶上来问道:“姐姐,不知道二奶奶往那里去了?”司棋道:“没理论。”\begin{note}庚辰侧批:妙极!\end{note}红玉听了,抽身又往四下里一看,只见那边探春宝钗在池边看鱼。红玉上来陪笑问道:“姑娘们可知道二奶奶那去了?”探春道:“往你大奶奶院里找去。”红玉听了,才往稻香村来,顶头只见\begin{note}庚辰侧批:又一折。\end{note} 晴雯、绮霰、碧痕、紫绡、麝月、侍书、入画、莺儿等一群人来了。晴雯一见了红玉,便说道:“你只是疯罢!院子里花儿也不浇,雀儿也不喂,茶炉子也不爖,就在外头逛。”\begin{note}庚辰侧批:必有此数句,方引出称心得意之语来。再不用本院人见小红,此差只几分遂心。\end{note} 红玉道:“昨儿二爷说了,今儿不用浇花,过一日浇一回罢。我喂雀儿的时侯,姐姐还睡觉呢。” 碧痕道:“茶炉子呢?”\begin{note}甲戌侧批:岔一人问,俱是不受用意。\end{note}红玉道:“今儿不该我爖的班儿,有茶没茶别问我。”绮霰道:“你听听他的嘴!你们别说了,让他逛去罢。”红玉道:“你们再问问我逛了没有。二奶奶使唤我说话取东西的。”\begin{note}甲戌侧批:非小红夸耀,系尔等逼出来的,离怡红意已定矣。\end{note}说著将荷包举给他们看,\begin{note}庚辰侧批:得意!称心如意,在此一举荷包。\end{note}方没言语了, \begin{note}甲戌侧批:众女儿何苦自讨之。\end{note}大家分路走开。晴雯冷笑道:“怪道呢!原来爬上高枝儿去了,把我们不放在眼里。不知说了一句话半句话,名儿姓儿知道了不曾呢,就把他兴的这样!这一遭半遭儿的算不得什么,过了后儿还得听呵!有本事从今儿出了这园子,长长远远的在高枝儿上才算得。”\begin{note}庚辰侧批:虽是醋语,却与下无痕。\end{note}一面说著去了。
\end{parag}


\begin{parag}
    这里红玉听说,不便分证,只得忍著气来找凤姐儿。到了李氏房中,果见凤姐儿在这里和李氏说话儿呢。红玉上来回道:“平姐姐说,奶奶刚出来了,他就把银子收了起来,\begin{note}甲戌侧批:交代不在盘架下了。\end{note}才张材家的来讨,当面称了给他拿去了。”说著将荷包递了上去,\begin{note}庚辰侧批:两件完了。\end{note}又道:“平姐姐教我回奶奶:才旺儿进来讨奶奶的示下,好往那家子去。平姐姐就把那话按著奶奶的主意打发他去了。”凤姐笑道:“他怎么按我的主意打发去了?”\begin{note}甲戌侧批:可知前红玉云“就把那按奶奶的主意”是欲俭,但恐累赘耳,故阿凤有是问,彼能细答。\end{note}红玉道:“平姐姐说:我们奶奶问这里奶奶好。原是我们二爷不在家,虽然迟了两天,只管请奶奶放心。等五奶奶\begin{note}甲戌侧批:又一门。\end{note}好些,我们奶奶还会了五奶奶来瞧奶奶呢。五奶奶前儿打发了人来说,舅奶奶\begin{note}甲戌侧批:又一门。\end{note}带了信来了,问奶奶好,还要和这里的姑奶奶寻两丸延年神验万全丹。若有了,奶奶\begin{note}甲戌侧批:又一门。\end{note}打发人来,只管送在我们奶奶这里。明儿有人去,就顺路给那边舅奶奶带去的。”
\end{parag}


\begin{parag}
    话未说完,\begin{note}庚辰侧批:又一润色。\end{note}李氏道:“嗳哟!\begin{note}甲戌侧批:红玉今日方遂心如意,却为宝玉后伏线。\end{note}这些话我就不懂了。什么‘奶奶’‘爷爷’的一大堆。”凤姐笑道:“怨不得你不懂,这是四五门子的话呢。”说著又向红玉笑道:“好孩子,难为你说的齐全。别像他们扭扭捏捏的蚊子似的。\begin{note}庚辰侧批:写死假斯文。\end{note}嫂子不知道,如今除了我随手使的几个人之外,我就怕和人说话。他们必定把一句话拉长了作两三截儿,咬文咬字,拿著腔儿,哼哼唧唧的,急的我冒火,他们那里知道!先时我们平儿也是这么著,我就问著他:难道必定装蚊子哼哼就是美人了?\begin{note}庚辰侧批:贬杀,骂杀。\end{note}说了几遭才好些儿了。”李宫裁笑道: “都像你泼皮破落户才好。”凤姐又道:“这一个丫头就好。\begin{note}甲戌侧批:红玉听见了吗?\end{note}方才两遭,说话虽不多,听那口声就简断。”\begin{note}甲戌侧批:红玉此刻心内想:可惜晴雯等不在傍。\end{note}说著又向红玉笑道:“你明儿伏侍我去罢。我认你作女儿,我一调理你就出息了。”\begin{note}庚辰侧批:不假。\end{note}
\end{parag}


\begin{parag}
    红玉听了,扑哧一笑。凤姐道:“你怎么笑?你说我年轻,比你能大几岁,就作你的妈了?你别作春梦呢!你打听打听,这些人头比你大的大的,赶著我叫妈,我还不理。今儿抬举了你呢!”红玉笑道:“我不是笑这个,我笑奶奶认错了辈数了。我妈是奶奶的女儿,\begin{note}庚辰侧批:所以说“比你大的大的”。\end{note}这会子又认我作女儿。”凤姐道:“谁是你妈?”\begin{note}庚辰侧批:晴雯说过。\end{note}李宫裁笑道:“你原来不认得他?他是林之孝之女。”\begin{note}甲戌侧批:管家之女,而晴卿辈挤之,招祸之媒也。\end{note}凤姐听了十分诧异,说道:“哦!原来是他的丫头。”\begin{note}甲戌侧批:传神。\end{note}又笑道:“林之孝两口子都是锥子扎不出一声儿来的。我成日家说,他们倒是配就了的一对夫妻,一对天聋地哑。\begin{note}甲戌侧批:用的是阿凤口角。\end{note}那里承望养出这么个伶俐丫头来!你十几岁了?”红玉道:“十七岁了。”又问名字,\begin{note}甲戌侧批:真真不知名,可叹!\end{note}红玉道:“原叫红玉的,因为重了宝二爷,如今只叫红儿了。”
\end{parag}


\begin{parag}
    凤姐听说将眉一皱,把头一回,说道:“讨人嫌的很!\begin{note}庚辰侧批:又一下针。\end{note}得了玉的益似的,你也玉,我也玉。”因说道:“既这么著肯跟,我还和他妈说,‘赖大家的如今事多,也不知这府里谁是谁,你替我好好的挑两个丫头我使’,他一般答应著。他饶不挑,倒把这女孩子送了别处去。难道跟我必定不好?”李氏笑道:“你可是又多心了。他进来在先,你说话在后,怎么怨的他妈!”凤姐道:“既这么著,明儿我和宝玉说,叫他再要人,\begin{note}甲戌侧批:有悌弟之心。\end{note}叫这丫头跟我去。可不知本人愿意不愿意?”\begin{note}甲戌侧批:总是追写红玉十分心事。\end{note}红玉笑道:“愿意不愿意,我们也不敢说。\begin{note}甲戌侧批:好答!可知两处俱是主见。\end{note}只是跟著奶奶,我们也学些眉眼高低,\begin{note}庚辰侧批:千愿意万愿意之言。\end{note}出入上下,大小的事也得见识见识。”\begin{note}甲戌侧批:且系本心本意,“狱神庙”回内方见。\end{note}\begin{note}庚辰眉批:奸邪婢岂是怡红应答者,故即逐之。前良儿,后篆儿,便是确证。作者又不得有也。己卯冬夜。\end{note}\begin{note}庚辰眉批:此系未见“抄没”、“狱神庙”诸事,故有是批。丁亥夏。畸笏。\end{note}刚说著,只见王夫人的丫头来请,\begin{note}庚辰侧批:截得真好。\end{note}凤姐便辞了李宫裁去了。红玉回怡红院去,\begin{note}庚辰侧批:好,接得更好。\end{note}不在话下。
\end{parag}


\begin{parag}
    如今且说林黛玉因夜间失寐,次日起来迟了,闻得众姊妹都在园中作饯花会,恐人笑他痴懒,连忙梳洗了出来。刚到了院中,只见宝玉进门来了,笑道:“好妹妹,你昨儿可告我了不曾?\begin{note}甲戌侧批:明知无是事,不得不作开谈。\end{note}教我悬了一夜心。”\begin{note}庚辰侧批:并不为告悬心。\end{note}林黛玉便回头叫紫鹃道:\begin{note}甲戌侧批:不见宝玉,阿颦断无此一段闲言,总在欲言不言难禁之意,了却“情情”之正文也。\end{note}\begin{note}庚辰侧批:倒像不曾听见的。\end{note}“把屋子收拾了,撂下一扇纱屉;看那大燕子回来,把帘子放下来,拿狮子倚住;烧了香就把炉罩上。”一面说一面又往外走。宝玉见他这样,还认作是昨日中晌的事,\begin{note}甲戌侧批:毕真不错。\end{note}那知晚间的这段公案,还打恭作揖的。林黛玉正眼也不看,各自出了院门,一直找别的姊妹去了。宝玉心中纳闷,自己猜疑:看起这个光景来,不象是为昨日的事;但只昨日我回来的晚了,又没见他,再没有冲撞了他的去处。\begin{note}庚辰侧批:毕真不错。\end{note}一面想,一面由不得随后追了来。
\end{parag}


\begin{parag}
    只见宝钗探春正在那边看仙鹤,\begin{note}庚辰侧批:二玉文字岂是容易写的,故有此截。\end{note}\begin{note}庚辰眉批:《石头记》用截法、岔法、突然法、伏线法、由近渐远法、将繁改简法、重作轻抹法、虚敲实应法种种诸法,总在人意料之外,且不曾见一丝牵强,所谓“信手拈来无不是”是也。\end{note}见黛玉来了,三个一同站著说话儿。又见宝玉来了,探春便笑道:“宝哥哥,身上好?我整整三天没见了。”\begin{note}甲戌侧批:横云截岭,好极,妙极!二玉文原不易写,《石头记》得力处在兹。\end{note}宝玉笑道: “妹妹身上好?我前儿还在大嫂子跟前问你呢。”探春道:“哥哥往这里来,我和你说话。”\begin{note}庚辰侧批:是移一处语。\end{note}宝玉听说,便跟了他来到一棵石榴树下。探春因说道:“这几天老爷可叫你没有?”\begin{note}甲戌侧批:老爷叫宝玉再无喜事,故园中合宅皆知。\end{note}宝玉笑道:“没有叫。”探春说:“昨儿我恍惚听见说老爷叫你出去的。”宝玉笑道:“那想是别人听错了,并没叫的。”\begin{note}甲戌侧批:非谎也,避繁也。\end{note}\begin{note}庚批:怕文繁。\end{note}探春又笑道:“这几个月,我又攒下有十来吊钱了。你还拿了去,明儿出门逛去的时候,或是好字画,好轻巧顽意儿,替我带些来。”\begin{note}庚辰眉批:若无此一岔,二玉和合则成嚼蜡文字。《石头记》得力处正此。丁亥夏。畸笏叟。\end{note}宝玉道:“我这么城里城外、大廊小庙的逛,也没见个新奇精致东西,左不过是那些金玉铜磁没处撂的古董,再就是绸缎吃食衣服了。”探春道:“谁要这些。怎么像你上回买的那柳枝儿编的小篮子,整竹子根抠的香盒儿,泥垛的风炉儿,这就好了。我喜欢的什么似的,谁知他们都爱上了,都当宝贝似的抢了去了。”宝玉笑道:“原来要这个。这不值什么,拿五百钱出去给小子们,管拉一车来。”\begin{note}庚批:不知物理艰难,公子口气也。\end{note}探春道:“小厮们知道什么。你拣那朴而不俗、直而不拙者,\begin{note}甲戌侧批:是论物?是论人?看官著眼。\end{note}这些东西,你多多的替我带了来。我还象上回的鞋作一双你穿,比那一双还加工夫,如何呢?”
\end{parag}


\begin{parag}
    宝玉笑道:“你提起鞋来,我想起个故事:那一回我穿著,可巧遇见了老爷,\begin{note}庚辰侧批:补遗法。\end{note}老爷就不受用,问是谁作的。我那里敢提‘三妹妹’三个字,我就回说是前儿我生日,是舅母给的。老爷听了是舅母给的,才不好说什么,半日还说:‘何苦来!虚耗人力,作践绫罗,作这样的东西。’我回来告诉了袭人,袭人说这还罢了,赵姨娘气的抱怨的了不得:‘正经兄弟,\begin{note}庚辰侧批:指环哥。\end{note}鞋搭拉袜搭拉的\begin{note}甲戌侧批:何至如此,写妒妇信口逗。\end{note}没人看的见,且作这些东西!’”探春听说,登时沉下脸来,道:“这话糊涂到什么田地!怎么我是该作鞋的人么?环儿难道没有分例的,没有人的?一般的衣裳是衣裳,鞋袜是鞋袜,丫头老婆一屋子,怎么抱怨这些话!给谁听呢!我不过是闲著没事儿,作一双半双,爱给那个哥哥兄弟,随我的心。谁敢管我不成!这也是白气。”宝玉听了,点头笑道:“你不知道,他心里自然又有个想头了。”探春听说,益发动了气,将头一扭,说道:“连你也糊涂了!他那想头自然是有的,不过是那阴微鄙贱的见识。他只管这么想,我只管认得老爷、太太两个人,别人我一概不管。就是姊妹弟兄跟前,谁和我好,我就和谁好,什么偏的庶的,我也不知道。论理我不该说他,但忒昏愦的不象了!还有笑话呢:\begin{note}甲戌侧批:开一步,妙妙!\end{note}就是上回我给你那钱,替我带那顽的东西。过了两天,他见了我,也是说没钱使,怎么难,我也不理论。谁知后来丫头们出去了,他就抱怨起来,说我攒的钱为什么给你使,倒不给环儿使呢。我听见这话,又好笑又好气,我就出来往太太跟前去了。”\begin{note}庚辰眉批:这一节特为“兴利除弊”一回伏线。\end{note}正说著,只见宝钗那边笑道:\begin{note}庚辰侧批:截得好。\end{note}“说完了,来罢。显见的是哥哥妹妹了,丢下别人,且说梯己去。我们听一句儿就使不得了!”说著,探春宝玉二人方笑著来了。
\end{parag}


\begin{parag}
    宝玉因不见了林黛玉,\begin{note}甲戌侧批:兄妹话虽久长,心事总未少歇,接得好。\end{note}便知他躲了别处去了,想了一想,索性迟两日,\begin{note}甲戌侧批:作书人调侃耶?\end{note}等他的气消一消再去也罢了。因低头看见许多凤仙石榴等各色落花,锦重重的落了一地,\begin{note}庚辰眉批:不因见落花,宝玉如何突至埋香冢?不至埋香冢,如何写《葬花吟》?《石头记》无闲文闲字正此。丁亥夏。畸笏叟。\end{note}因叹道:“这是他心里生了气,也不收拾这花儿来了。待我送了去,明儿再问著他。”\begin{note}甲戌侧批:至埋香冢方不牵强,好情理。\end{note}说著,只见宝钗约著他们往外头去。\begin{note}甲戌侧批:收拾的干净。\end{note}宝玉道:“我就来。”说毕,等他二人去远了,\begin{note}甲戌侧批:怕人笑说。\end{note}便把那花兜了起来,登山渡水,过树穿花,一直奔了那日同林黛玉葬桃花的去处来。将已到了花冢,\begin{note}庚辰侧批:新鲜。\end{note}犹未转过山坡,只听山坡那边有呜咽之声,一行数落著,哭的好不伤感。\begin{note}甲戌侧批:奇文异文,俱出《石头记》上,且愈出愈奇文。\end{note}宝玉心下想道:“这不知是那房里的丫头,受了委曲,\begin{note}甲戌侧批:岔开线络,活泼之至!\end{note}跑到这个地方来哭。”一面想,一面煞住脚步,听他哭道是:\begin{note}甲戌侧批:诗词歌赋,如此章法写于书上者乎?\end{note}\begin{note}庚辰侧批:诗词文章,试问有如此行笔者乎?\end{note}\begin{note}甲戌眉批:开生面,立新场,是书多多矣,惟此回处[更]生更新。非颦儿断无是佳吟,非石兄断无是情聆,难为了作者了,故留数字以慰之。\end{note}\begin{note}庚辰眉批:开生面,立新场,是书不止“红楼梦”一回,惟是回更生更新,且读去非阿颦无是且(佳)吟,非石兄断无是章法行文,愧杀古今小说家也。畸笏。\end{note}
\end{parag}


\begin{poem}
    \begin{pl} 花谢花飞花满天,红消香断有谁怜?游丝软系飘春榭,落絮轻沾扑绣帘。 \end{pl}
    \begin{pl} 帘中女儿惜春暮,愁绪满怀无处诉,手把花锄出绣帘,忍踏落花来复去? \end{pl}
    \begin{pl} 柳丝榆荚自芳菲,不管桃飘与李飞。桃李明年能再发,明岁闺中知有谁? \end{pl}
    \begin{pl} 三月香巢已垒成,梁间燕子太无情!明年花发虽可啄,却不道人去梁空巢也倾。 \end{pl}
    \begin{pl} 一年三百六十日,风刀霜剑严相逼,明媚鲜妍能几时?一朝飘泊难寻觅。 \end{pl}
    \begin{pl} 花开易见落难寻,阶前闷杀葬花人,独倚花锄泪暗洒,洒上花枝见血痕。 \end{pl}
    \begin{pl} 杜鹃无语正黄昏,荷锄归去掩重门。青灯照壁人初睡,冷雨敲窗被未温。 \end{pl}
    \begin{pl} 怪奴底事倍伤神,半为怜春半恼春。怜春忽至恼忽去,至又无言去不闻。 \end{pl}
    \begin{pl} 昨宵庭外悲歌发,知是花魂与鸟魂。花魂鸟魂总难留,鸟自无言花自羞。 \end{pl}
    \begin{pl} 愿奴胁下生双翼,随花飞到天尽头。天尽头,何处有香丘? \end{pl}
    \begin{pl} 未若锦囊收艳骨,一抔净土掩风流。质本洁来还洁去,强于污淖陷渠沟。 \end{pl}
    \begin{pl} 尔今死去侬收葬,未卜侬身何日丧?侬今葬花人笑痴,他年葬侬知是谁? \end{pl}
    \begin{pl} 试看春残花渐落,便是红颜老死时。一朝春尽红颜老,花落人亡两不知! \end{pl}
\end{poem}


\begin{parag}
    宝玉不觉痴倒。要知端底,再看下回。
\end{parag}


\begin{parag}
    \begin{note}甲戌回尾:余读《葬花吟》,至再至三四,其凄楚感慨,令人身世两忘,举笔再四,不能下批。有客曰:“先生身非宝玉,何能下笔?即字字双圈,批词通仙,料难遂颦儿之意。俟看玉兄之后文再批。”噫唏!阻余者想亦《石头记》来的?故掷笔以待。\end{note}
\end{parag}


\begin{parag}
    \begin{note}甲戌:饯花辰不论典与不典,只取其韵致生趣耳。\end{note}
\end{parag}


\begin{parag}
    \begin{note}甲戌:池边戏蝶,偶尔适兴;亭外急智脱壳。明写宝钗非拘拘然一女夫子。\end{note}
\end{parag}


\begin{parag}
    \begin{note}甲戌:凤姐用小红,可知晴雯等埋没其人久矣,无怪有私心私情。且红玉后有宝玉大得力处,此于千里外伏线也。\end{note}
\end{parag}


\begin{parag}
    \begin{note}甲戌:《石头记》用截法、岔法、突然法、伏线法、由近渐远法、将繁改简法、重作轻抹法、虚敲实应法种种诸法,总在人意料之外,且不曾见一丝牵强,所谓“信手拈来无不是”是也。\end{note}
\end{parag}


\begin{parag}
    \begin{note}甲戌:不因见落花,宝玉如何突至埋香冢;不至埋香冢又如何写《葬花吟》。\end{note}
\end{parag}


\begin{parag}
    \begin{note}甲戌:埋香冢葬花乃诸艳归源,《葬花吟》又系诸艳一偈也。\end{note}
\end{parag}


\begin{parag}
    \begin{note}蒙回后总评:幸逢知己无回避,审语歌窗怕有人。总是关心浑不了,叮咛嘱咐为轻春。\end{note}
\end{parag}


\begin{parag}
    \begin{note}蒙回后总评:心事将谁告,花飞动我悲。埋香吟哭后,日日敛双眉。\end{note}
\end{parag}

