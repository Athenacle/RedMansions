\chap{八十四}{试文字宝玉始提亲 探惊风贾环重结怨}



\begin{parag}
    却说薛姨妈一时因被金桂这场气怄得肝气上逆,左肋作痛。宝钗明知是这个原故,也等不及医生来看,先叫人去买了几钱钩藤来,浓浓的煎了一碗,给他母亲吃了。又和秋菱给薛姨妈捶腿揉胸,停了一会儿,略觉安顿。这薛姨妈只是又悲又气,气的是金桂撒泼,悲的是宝钗有涵养,倒觉可怜。宝钗又劝了一回,不知不觉的睡了一觉,肝气也渐渐平复了。宝钗便说道:“妈妈,你这种闲气不要放在心上才好。过几天走的动了,乐得往那边老太太姨妈处去说说话儿散散闷也好。家里横竖有我和秋菱照看着,谅他也不敢怎么样。”薛姨妈点点头道:“过两日看罢了。”
\end{parag}


\begin{parag}
    且说元妃疾愈之后,家中俱各喜欢。过了几日,有几个老公走来,带着东西银两,宣贵妃娘娘之命,因家中省问勤劳,俱有赏赐。把对象银两一一交代清楚。贾赦贾政等禀明了贾母,一齐谢恩毕,太监吃了茶去了。大家回到贾母房中,说笑了一回。外面老婆子传进来说:“小厮们来回道,那边有人请大老爷说要紧的话呢。”贾母便向贾赦道:“你去罢。”贾赦答应着,退出来自去了。
\end{parag}


\begin{parag}
    这里贾母忽然想起,和贾政笑道:“娘娘心里却甚实惦记着宝玉,前儿还特特的问他来着呢。”贾政陪笑道:“只是宝玉不大肯念书,辜负了娘娘的美意。”贾母道:“我倒给他上了个好儿,说他近日文章都做上来了。”贾政笑道:“那里能象老太太的话呢。”贾母道:“你们时常叫他出去作诗作文,难道他都没作上来么。小孩子家慢慢的教导他,可是人家说的,‘胖子也不是一口儿吃的’。”贾政听了这话,忙陪笑道:“老太太说的是。”贾母又道:“提起宝玉,我还有一件事和你商量。如今他也大了,你们也该留神看一个好孩子给他定下。这也是他终身的大事。也别论远近亲戚,什么穷啊富的,只要深知那姑娘的脾性儿好模样儿周正的就好。”贾政道:“老太太吩咐的很是。但只一件,姑娘也要好,第一要他自己学好才好,不然不稂不莠的,反倒耽误了人家的女孩儿,岂不可惜。”贾母听了这话,心里却有些不喜欢,便说道:“论起来,现放着你们作父母的,那里用我去张心。但只我想宝玉这孩子从小儿跟着我,未免多疼他一点儿,耽误了他成人的正事也是有的。只是我看他那生来的模样儿也还齐整,心性儿也还实在,未必一定是那种没出息的,必至遭踏了人家的女孩儿。也不知是我偏心,我看着横竖比环儿略好些,不知你们看着怎么样。”几句话说得贾政心中甚实不安,连忙陪笑道:“老太太看的人也多了,既说他好有造化的,想来是不错的。只是儿子望他成人性儿太急了一点,或者竟和古人的话相反,倒是‘莫知其子之美’了。”一句话把贾母也怄笑了,众人也都陪着笑了。贾母因说道:“你这会子也有了几岁年纪,又居着官,自然越历练越老成。”说到这里,回头瞅着邢夫人和王夫人笑道:“想他那年轻的时侯,那一种古怪脾气,比宝玉还加一倍呢。直等娶了媳妇,才略略的懂了些人事儿。如今只抱怨宝玉,这会子我看宝玉比他还略体些人情儿呢。”说的邢夫人王夫人都笑了。因说道:“老太太又说起逗笑儿的话儿来了。”说着,小丫头子们进来告诉鸳鸯:“请示老太太,晚饭伺侯下了。”贾母便问:“你们又咕咕唧唧的说什么?”鸳鸯笑着回明了。贾母道:“那么着,你们也都吃饭去罢,单留凤姐儿和珍哥媳妇跟着我吃罢。”贾政及邢王二夫人都答应着,伺侯摆上饭来,贾母又催了一遍,才都退出各散。
\end{parag}


\begin{parag}
    却说邢夫人自去了。贾政同王夫人进入房中。贾政因提起贾母方才的话来,说道:“老太太这样疼宝玉,毕竟要他有些实学,日后可以混得功名,才好不枉老太太疼他一场,也不至糟踏了人家的女儿。”王夫人道:“老爷这话自然是该当的。”贾政因着个屋里的丫头传出去告诉李贵:“宝玉放学回来,索性吃饭后再叫他过来,说我还要问他话呢。”李贵答应了“是”。至宝玉放了学刚要过来请安,只见李贵道:“二爷先不用过去。老爷吩咐了,今日叫二爷吃了饭再过去呢,听见还有话问二爷呢。”宝玉听了这话,又是一个闷雷。只得见过贾母,便回园吃饭。三口两口吃完,忙漱了口,便往贾政这边来。
\end{parag}


\begin{parag}
    贾政此时在内书房坐着,宝玉进来请了安,一旁侍立。贾政问道:“这几日我心上有事,也忘了问你。那一日你说你师父叫你讲一个月的书就要给你开笔,如今算来将两个月了,你到底开了笔了没有?”宝玉道:“才做过三次。师父说且不必回老爷知道,等好些再回老爷知道罢。因此这两天总没敢回。”贾政道:“是什么题目?”宝玉道:“一个是《吾十有五而志于学》,一个是《人不知而不愠》,一个是《则归墨》三字。”贾政道:“都有稿儿么?”宝玉道:“都是做了抄出来师父又改的。”贾政道:“你带了家来了还是在学房里呢?”宝玉道:“在学房里呢。”贾政道:“叫人取了来我瞧。”宝玉连忙叫人传话与焙茗:“叫他往学房中去,我书桌子抽屉里有一本薄薄儿竹纸本子,上面写着‘窗课’两字的就是,快拿来。”一回儿焙茗拿了来递给宝玉。宝玉呈与贾政。贾政翻开看时,见头一篇写着题目是《吾十有五而志于学》。他原本破的是“圣人有志于学,幼而已然矣。”代儒却将幼字抹去,明用“十五”。贾政道:“你原本‘幼’字便扣不清题目了。‘幼’字是从小起至十六以前都是‘幼’。这章书是圣人自言学问工夫与年俱进的话,所以十五,三十,四十,五十,六十,七十俱要明点出来,才见得到了几时有这么个光景,到了几时又有那么个光景。师父把你‘幼’字改了‘十五’,便明白了好些。”看到承题,那抹去的原本云:“夫不志于学,人之常也。”贾政摇头道:“不但是孩子气,可见你本性不是个学者的志气。”又看后句”圣人十五而志之,不亦难乎”,说道:“这更不成话了。”然后看代儒的改本云:“夫人孰不学,而志于学者卒鲜。此圣人所为自信于十五时欤。”便问“改的懂得么?”宝玉答应道:“懂得。”又看第二艺,题目是《人不知而不愠》,便先看代儒的改本云:“不以不知而愠者,终无改其说乐矣。”方觑着眼看那抹去的底本,说道:“你是什么?——‘能无愠人之心,纯乎学者也。’上一句似单做了‘而不愠’三个字的题目,下一句又犯了下文君子的分界。必如改笔才合题位呢。且下句找清上文,方是书理。须要细心领略。”宝玉答应着。贾政又往下看,“夫不知,未有不愠者也,而竟不然。是非由说而乐者,曷克臻此。”原本末句“非纯学者乎。”贾政道:“这也与破题同病的。这改的也罢了,不过清楚,还说得去。”第三艺是《则归墨》,贾政看了题目,自己扬着头想了一想,因问宝玉道:“你的书讲到这里了么?”宝玉道:“师父说,《孟子》好懂些,所以倒先讲《孟子》,大前日才讲完了。如今讲‘上论语’呢。”贾政因看这个破承倒没大改。破题云:“言于舍杨之外,若别无所归者焉。”贾政道:“第二句倒难为你。”’夫墨,非欲归者也,而墨之言已半天下矣,则舍杨之外,欲不归于墨,得乎?”贾政道:“这是你做的么?”宝玉答应道:“是。”贾政点点头儿,因说道:“这也并没有什么出色处,但初试笔能如此,还算不离。前年我在任上时,还出过《惟士为能》这个题目。那些童生都读过前人这篇,不能自出心裁,每多抄袭。你念过没有?”宝玉道:“也念过。”贾政道:“我要你另换个主意,不许雷同了前人,只做个破题也使得。”宝玉只得答应着,低头搜索枯肠。贾政背着手,也在门口站著作想。只见一个小小厮往外飞走,看见贾政,连忙侧身垂手站住。贾政便问道:“作什么?”小厮回道:“老太太那边姨太太来了,二奶奶传出话来,叫预备饭呢。”贾政听了,也没言语。那小厮自去了。
\end{parag}


\begin{parag}
    谁知宝玉自从宝钗搬回家去,十分想念,听见薛姨妈来了,只当宝钗同来,心中早已忙了,便乍着胆子回道:“破题倒作了一个,但不知是不是。”贾政道:“你念来我听。”宝玉念道:“天下不皆士也,能无产者亦仅矣。”贾政听了,点着头道:“也还使得。以后作文,总要把界限分清,把神理想明白了再去动笔。你来的时侯老太太知道不知道?”宝玉道:“知道的。”贾政道:“既如此,你还到老太太处去罢。”宝玉答应了个“是”,只得拿捏着慢慢的退出,刚过穿廊月洞门的影屏,便一溜烟跑到老太太院门口。急得焙茗在后头赶着叫:“看跌倒了!老爷来了。”宝玉那里听得见。刚进得门来,便听见王夫人,凤姐,探春等笑语之声。
\end{parag}


\begin{parag}
    丫鬟们见宝玉来了,连忙打起帘子,悄悄告诉道:“姨太太在这里呢。”宝玉赶忙进来给薛姨妈请安,过来才给贾母请了晚安。贾母便问:“你今儿怎么这早晚才散学?”宝玉悉把贾政看文章并命作破题的话述了一遍。贾母笑容满面。宝玉因问众人道:“宝姐姐在那里坐着呢?”薛姨妈笑道:“你宝姐姐没过来,家里和香菱作活呢。”宝玉听了,心中索然,又不好就走。只见说着话儿已摆上饭来,自然是贾母薛姨妈上坐,探春等陪坐。薛姨妈道:“宝哥儿呢?”贾母忙笑说道:“宝玉跟着我这边坐罢。”宝玉连忙回道:“头里散学时李贵传老爷的话,叫吃了饭过去。我赶着要了一碟菜,泡茶吃了一碗饭,就过去了。老太太和姨妈姐姐们用罢。”贾母道:“既这么着,凤丫头就过来跟着我。你太太才说他今儿吃斋,叫他们自己吃去罢。”王夫人也道:“你跟着老太太姨太太吃罢,不用等我,我吃斋呢。”于是凤姐告了坐,丫头安了杯箸,凤姐执壶斟了一巡,才归坐。
\end{parag}


\begin{parag}
    大家吃着酒。贾母便问道:“可是才姨太太提香菱,我听见前儿丫头们说‘秋菱’,不知是谁,问起来才知道是他。怎么那孩子好好的又改了名字呢?”薛姨妈满脸飞红,叹了一口气道:“老太太再别提起。自从蟠儿娶了这个不知好歹的媳妇,成日家咕咕唧唧,如今闹的也不成个人家了。我也说过他几次,他牛心不听说,我也没那么大精神和他们尽着吵去,只好由他们去。可不是他嫌这丫头的名儿不好改的。”贾母道:“名儿什么要紧的事呢?”薛姨妈道:“说起来我也怪臊的,其实老太太这边有什么不知道的。他那里是为这名儿不好,听见说他因为是宝丫头起的,他才有心要改。”贾母道:“这又是什么原故呢?”薛姨妈把手绢子不住的擦眼泪,未曾说,又叹了一口气,道:“老太太还不知道呢,这如今媳妇子专和宝丫头怄气。前日老太太打发人看我去,我们家里正闹呢。”贾母连忙接着问道:“可是前儿听见姨太太肝气疼,要打发人看去,后来听见说好了,所以没着人去。依我,劝姨太太竟把他们别放在心上。再者,他们也是新过门的小夫妻,过些时自然就好了。我看宝丫头性格儿温厚和平,虽然年轻,比大人还强几倍。前日那小丫头子回来说,我们这边还都赞叹了他一会子。都象宝丫头那样心胸儿脾气儿,真是百里挑一的。不是我说句冒失话,那给人家做了媳妇儿,怎么叫公婆不疼,家里上上下下的不宾服呢。”宝玉头里已经听烦了,推故要走,及听见这话,又坐了呆呆的往下听。薛姨妈道:“不中用。他虽好,到底是女孩儿家。养了蟠儿这个糊涂孩子,真真叫我不放心,只怕在外头喝点子酒,闹出事来。幸亏老太太这里的大爷二爷常和他在一块儿,我还放点儿心。”宝玉听到这里,便接口道:“姨妈更不用悬心。薛大哥相好的都是些正经买卖大客人,都是有体面的,那里就闹出事来。”薛姨妈笑道:“依你这样说,我敢只不用操心了。”说话间,饭已吃完。宝玉先告辞了,晚间还要看书,便各自去了。
\end{parag}


\begin{parag}
    这里丫头们刚捧上茶来,只见琥珀走过来向贾母耳朵旁边说了几句,贾母便向凤姐儿道:“你快去罢,瞧瞧巧姐儿去罢。”凤姐听了,还不知何故,大家也怔了。琥珀遂过来向凤姐道:“刚才平儿打发小丫头子来回二奶奶,说巧姐身上不大好,请二奶奶忙着些过来才好呢。”贾母因说道:“你快去罢,姨太太也不是外人。”凤姐连忙答应,在薛姨妈跟前告了辞。又见王夫人说道:“你先过去,我就去。小孩子家魂儿还不全呢,别叫丫头们大惊小怪的,屋里的猫儿狗儿,也叫他们留点神儿。尽着孩子贵气,偏有这些琐碎。”凤姐答应了,然后带了小丫头回房去了。
\end{parag}


\begin{parag}
    这里薛姨妈又问了一回黛玉的病。贾母道:“林丫头那孩子倒罢了,只是心重些,所以身子就不大很结实了。要赌灵性儿,也和宝丫头不差什么,要赌宽厚待人里头,却不济他宝姐姐有耽待,有尽让了。”薛姨妈又说了两句闲话儿,便道:“老太太歇着罢。我也要到家里去看看,只剩下宝丫头和香菱了。打那么同着姨太太看看巧姐儿。”贾母道:“正是。姨太太上年纪的人看看是怎么不好,说给他们,也得点主意儿。”薛姨妈便告辞,同着王夫人出来,往凤姐院里去了。
\end{parag}


\begin{parag}
    却说贾政试了宝玉一番,心里却也喜欢,走向外面和那些门客闲谈。说起方才的话来,便有新近到来最善大棋的一个王尔调名作梅的说道:“据我们看来,宝二爷的学问已是大进了。”贾政道:“那有进益,不过略懂得些罢咧,‘学问’两个字早得很呢。”詹光道:“这是老世翁过谦的话。不但王大兄这般说,就是我们看,宝二爷必定要高发的。”贾政笑道:“这也是诸位过爱的意思。”那王尔调又道:“晚生还有一句话,不揣冒昧,和老世翁商议。”贾政道:“什么事?”王尔调陪笑道:“也是晚生的相与,做过南韶道的张大老爷家有一位小姐,说是生得德容功貌俱全,此时尚未受聘。他又没有儿子,家资巨万。但是要富贵双全的人家,女婿又要出众,才肯作亲。晚生来了两个月,瞧着宝二爷的人品学业,都是必要大成的。老世翁这样门楣,还有何说。若晚生过去,包管一说就成。”贾政道:“宝玉说亲却也是年纪了,并且老太太常说起。但只张大老爷素来尚未深悉。”詹光道:“王兄所提张家,晚生却也知道。况和大老爷那边是旧亲,老世翁一问便知。”贾政想了一回,道:“大老爷那边不曾听得这门亲戚。”詹光道:“老世翁原来不知,这张府上原和邢舅太爷那边有亲的。”贾政听了,方知是邢夫人的亲戚。坐了一回,进来了,便要同王夫人说知,转问邢夫人去。谁知王夫人陪了薛姨妈到凤姐那边看巧姐儿去了。那天已经掌灯时候,薛姨妈去了,王夫人才过来了。贾政告诉了王尔调和詹光的话,又问巧姐儿怎么了。王夫人道:“怕是惊风的光景。”贾政道:“不甚利害呀?”王夫人道:“看着是搐风的来头,只还没搐出来呢。”贾政听了,便不言语,各自安歇,一宿晚景不提。
\end{parag}


\begin{parag}
    却说次日邢夫人过贾母这边来请安,王夫人便提起张家的事,一面回贾母,一面问邢夫人。邢夫人道:“张家虽系老亲,但近年来久已不通音信,不知他家的姑娘是怎么样的。倒是前日孙亲家太太打发老婆子来问安,却说起张家的事,说他家有个姑娘,托孙亲家那边有对劲的提一提。听见说只这一个女孩儿,十分娇养,也识得几个字,见不得大阵仗儿,常在房中不出来的。张大老爷又说,只有这一个女孩儿,不肯嫁出去,怕人家公婆严,姑娘受不得委屈,必要女婿过门赘在他家,给他料理些家事。”贾母听到这里,不等说完便道:“这断使不得。我们宝玉别人伏侍他还不够呢,倒给人家当家去。”邢夫人道:“正是老太太这个话。”贾母因向王夫人道:“你回来告诉你老爷,就说我的话,这张家的亲事是作不得的。”王夫人答应了。贾母便问:“你们昨日看巧姐儿怎么样?头里平儿来回我说很不大好,我也要过去看看呢。”邢王二夫人道:“老太太虽疼他,他那里耽的住。”贾母道:“却也不止为他,我也要走动走动,活活筋骨儿。”说着,便吩咐:“你们吃饭去罢,回来同我过去。”邢王二夫人答应着出来,各自去了。
\end{parag}


\begin{parag}
    一时吃了饭,都来陪贾母到凤姐房中。凤姐连忙出来接了进去。贾母便问巧姐儿到底怎么样。凤姐儿道:“只怕是搐风的来头。”贾母道:“这么着还不请人赶着瞧!”凤姐道:“已经请去了。”贾母因同邢王二夫人进房来看,只见奶子抱着,用桃红绫子小绵被儿裹着,脸皮趣青,眉梢鼻翅微有动意。贾母同邢王二夫人看了看,便出外间坐下。正说间,只见一个小丫头回凤姐道:“老爷打发人问姐儿怎么样。”凤姐道:“替我回老爷,就说请大夫去了。一会儿开了方子,就过去回老爷。”贾母忽然想起张家的事来,向王夫人道:“你该就去告诉你老爷,省得人家去说了回来又驳回。”又问邢夫人道:“你们和张家如今为什么不走了?”邢夫人因又说:“论起那张家行事,也难和咱们作亲,太啬克,没的玷辱了宝玉。”凤姐听了这话,已知八九,便问道:“太太不是说宝兄弟的亲事?”邢夫人道:“可不是么。”贾母接着因把刚才的话告诉凤姐。凤姐笑道:“不是我当着老祖宗太太们跟前说句大胆的话,现放着天配的姻缘,何用别处去找。”贾母笑问道:“在那里?”凤姐道:“一个‘宝玉’,一个‘金锁’,老太太怎么忘了?”贾母笑了一笑,因说:“昨日你姑妈在这里,你为什么不提?”凤姐道:“老祖宗和太太们在前头,那里有我们小孩子家说话的地方儿。况且姨妈过来瞧老祖宗,怎么提这些个,这也得太太们过去求亲才是。”贾母笑了,邢王二夫人也都笑了。贾母因道:“可是我背晦了。”
\end{parag}


\begin{parag}
    说着人回:“大夫来了。”贾母便坐在外间,邢王二夫人略避。那大夫同贾琏进来,给贾母请了安,方进房中。看了出来,站在地下躬身回贾母道:“妞儿一半是内热,一半是惊风。须先用一剂发散风痰药,还要用四神散才好,因病势来得不轻。如今的牛黄都是假的,要找真牛黄方用得。”贾母道了乏,那大夫同贾琏出去开了方子,去了。凤姐道:“人参家里常有,这牛黄倒怕未必有,外头买去,只是要真的才好。”王夫人道:“等我打发人到姨太太那边去找找。他家蟠儿是向与那些西客们做买卖,或者有真的也未可知。我叫人去问问。”正说话间,众姊妹都来瞧来了,坐了一回,也都跟着贾母等去了。
\end{parag}


\begin{parag}
    这里煎了药给巧姐儿灌了下去,只听喀的一声,连药带痰都吐出来,凤姐才略放了一点儿心。只见王夫人那边的小丫头拿着一点儿的小红纸包儿说道:“二奶奶,牛黄有了。太太说了,叫二奶奶亲自把分两对准了呢。”凤姐答应着接过来,便叫平儿配齐了真珠,冰片,朱砂,快熬起来。自己用戥子按方称了,搀在里面,等巧姐儿醒了好给他吃。只见贾环掀帘进来说:“二姐姐,你们巧姐儿怎么了?妈叫我来瞧瞧他。”凤姐见了他母子便嫌,说:“好些了。你回去说,叫你们姨娘想着。”那贾环口里答应,只管各处瞧看。看了一回,便问凤姐儿道:“你这里听的说有牛黄,不知牛黄是怎么个样儿,给我瞧瞧呢。”凤姐道:“你别在这里闹了,妞儿才好些。那牛黄都煎上了。”贾环听了,便去伸手拿那吊子瞧时,岂知措手不及,沸的一声,吊子倒了,火已泼灭了一半。贾环见不是事,自觉没趣,连忙跑了。凤姐急的火星直爆,骂道:“真真那一世的对头冤家!你何苦来还来使促狭!从前你妈要想害我,如今又来害妞儿。我和你几辈子的仇呢!”一面骂平儿不照应。正骂着,只见丫头来找贾环。凤姐道:“你去告诉赵姨娘,说他操心也太苦了。巧姐儿死定了,不用他惦着了!”平儿急忙在那里配药再熬,那丫头摸不着头脑,便悄悄问平儿道:“二奶奶为什么生气?”平儿将环哥弄倒药吊子说了一遍。丫头道:“怪不得他不敢回来,躲了别处去了。这环哥儿明日还不知怎么样呢。平姐姐,我替你收拾罢。”平儿说:“这倒不消。幸亏牛黄还有一点,如今配好了,你去罢。”丫头道:“我一准回去告诉赵姨奶奶,也省得他天天说嘴。”
\end{parag}


\begin{parag}
    丫头回去果然告诉了赵姨娘。赵姨娘气的叫:“快找环儿!”环儿在外间屋子里躲着,被丫头找了来。赵姨娘便骂道:“你这个下作种子!你为什么弄洒了人家的药,招的人家咒骂。我原叫你去问一声,不用进去,你偏进去,又不就走,还要虎头上捉虱子。你看我回了老爷,打你不打!”这里赵姨娘正说着,只听贾环在外间屋子里更说出些惊心动魄的话来。未知何言,下回分解。
\end{parag}