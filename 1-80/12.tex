\chap{一十二}{王熙鳳毒設相思局 賈天祥正照風月鑑}

\begin{parag}
    \begin{note}蒙:反正從來總一心,鏡光至意兩相尋。有朝敲破矇頭甕,綠水青山任好春。\end{note}
\end{parag}


\begin{parag}
    話說鳳姐正與平兒說話,只見有人回說:“瑞大爺來了。”鳳姐急命:\begin{note}庚側:立意追命。\end{note}“快請進來。”賈瑞見往裏讓,心中喜出望外,急忙進來,見了鳳姐,滿面陪笑,\begin{note}庚側:如蛇。\end{note}連連問好。鳳姐兒也假意殷勤,讓坐讓茶。
\end{parag}


\begin{parag}
    賈瑞見鳳姐如此打扮,益發酥倒,因餳了眼問道:“二哥哥怎麼還不回來?”鳳姐道:“不知什麼原故。”賈瑞笑道:“別是路上有人絆住了腳了,\begin{note}蒙側:旁敲遠引。\end{note}捨不得回來也未可知?”鳳姐道:“也未可知。男人家見一個愛一個也是有的。”\begin{note}蒙側:這是鉤。\end{note}賈瑞笑道:\begin{note}庚雙夾:如聞其聲。\end{note}“嫂子這話錯了,我就不這樣。”\begin{note}庚雙夾:漸漸入港。\end{note}鳳姐笑道:“象你這樣的人能有幾個呢,十個裏也挑不出一個來。”\begin{note}庚眉:勿作正面看爲幸。畸笏。蒙側:游魚雖有入甕之志,無鉤不能上岸;一上鉤來,欲去亦不可得。\end{note}賈瑞聽了,喜的抓耳撓腮,又道:“嫂子天天也悶的很?”鳳姐道:“正是呢,只盼個人來說話解解悶兒。”賈瑞笑道:“我倒天天閒著,天天過來替嫂子解解閒悶可好不好?”鳳姐笑道:“你哄我呢,你那裏肯往我這裏來?”賈瑞道:“我嫂子跟前,若有一點謊話,天打雷劈!只因素人聞得人說,嫂子是個利害人,在你跟前一點也錯不得,所以唬住了我。如今見嫂子最是個有說有笑極疼人的,\begin{note}庚雙夾:奇妙!\end{note}我怎麼不來,——死了也願意!”\begin{note}庚側:這倒不假。\end{note}鳳姐笑道:“果然你是個明白人,比賈蓉兩個強遠了。我看他那樣清秀,只當他們心裏明白,誰知竟是兩個糊塗蟲,\begin{note}庚側:反文著眼。\end{note}一點不知人心。”
\end{parag}


\begin{parag}
    賈瑞聽這話,越發撞在心坎兒上,由不得又往前湊了一湊,\begin{note}寫呆人癡性活現。\end{note}覷著眼看鳳姐帶的荷包,然後又問戴著什麼戒指。鳳姐悄悄道 :“放尊重著,別叫丫頭們看了笑話。”賈瑞如聽綸音佛語一般,忙往後退。鳳姐笑道:“你該走了。”\begin{note}庚雙夾:叫去正是叫來也。\end{note}賈瑞道:“我再坐一坐兒。”“好狠心的嫂子!”鳳姐又悄悄的道:“大天白白,人來人往,你就在這裏也不方便。你且去,等著晚上起了更你來,悄悄的在西邊穿堂兒等我。”\begin{note}庚眉:先寫穿堂,只知房舍之大,豈料有許多用處。\end{note}\begin{note}蒙側:凡人在平靜時,物來言至,無不照見。若迷於一事一物,雖風雷交作,有所不聞。即“穿堂爾等”之一語,府第非比凡常,關門戶,必要查看,且更夫僕婦,勢必往來,豈容人藏過於其間?只因色迷,聞聲聯諾,不能有回思之暇,信可悲夫!\end{note}賈瑞聽了,如得珍寶,忙問道:“你別哄我。但只那裏人過的多,怎麼好躲的?”鳳姐道:“你只放心。我把上夜的小廝們都放了假,兩邊門一關,再沒別人了。”賈瑞聽了,喜之不盡,忙忙的告辭而去,心內以爲得手。\begin{note}庚側:未必。\end{note}
\end{parag}


\begin{parag}
    盼到晚上,果然黑地裏摸入榮府,趁掩門時,鑽入穿堂。果見漆黑無人,往賈母那邊去的門戶已鎖倒,只有向東的門未關。賈瑞側耳聽著,半日不見人來,忽聽咯登一聲,東邊的門也倒關了。\begin{note}庚側:平平略施小計。\end{note}賈瑞急的也不敢則聲,只得悄悄的出來,將門撼了撼,關得鐵桶一般。此時要求出去,亦不能夠。\begin{note}蒙側:此大抵是鳳姐調遣。不先爲點明者,可以少許多事故,又可以藏拙。\end{note}南北皆是大房牆,要跳亦無攀援。這屋內又是過門風,空落落;現是臘月天氣,夜又長,朔風凜凜,侵肌裂骨,一夜幾乎不曾凍死。\begin{note}庚眉:可爲偷情一戒。蒙側:教導之法、慈悲之心盡矣,無奈迷徑不悟何!\end{note}好容易盼到早晨,只見一個老婆子先將東門開了,進去叫西門。賈瑞瞅他背著臉,一溜煙抱著肩跑了出來,幸而天氣尚早,人都未起,從後門一徑跑回家去。
\end{parag}


\begin{parag}
    原來賈瑞父母早亡,只有他祖父代儒教養。那代儒素日教訓最嚴,\begin{note}庚眉:教訓最嚴,奈其心何!一嘆。\end{note}不許賈瑞多走一步,生怕他在外喫酒賭錢,有誤學業。今忽見他一夜不歸,只料定他在外非飲即賭,嫖娼宿妓,\begin{note}庚側:輾轉靈活,一人不放,一筆不肖。\end{note}那裏想到這段公案,\begin{note}庚側:世人萬萬想不到,況老學究乎!\end{note}因此氣了一夜。賈瑞也捻著一把汗,少不得回來撒慌,只說:“往舅舅家去了,天黑了,留我住了一夜。”代儒道:“自來出門,非稟我不敢擅出,如何昨日私自去了?據此亦該打,何況是撒謊!”\begin{note}庚眉:處處點父母癡心、子孫不肖。此書系自愧而成。\end{note}因此,發狠到底打了三四十板,不許喫飯,令他跪在院內讀文章,定要補出十天工課來方罷。賈瑞直凍了一夜,今又遭了苦打,且餓著肚子跪在風地裏念文章,\begin{note}教令何嘗不好,孽種故此不同。\end{note}其苦萬狀。\begin{note}庚雙夾:禍福無門,唯人自招。\end{note}
\end{parag}


\begin{parag}
    此時賈瑞前心猶是未改,\begin{note}庚側:四字是尋死之根。庚眉:苦海無邊,回頭是岸。若個能回頭也?嘆嘆!壬午春。畸笏。\end{note}再想不到是鳳姐捉弄他。過後兩日,得了空,便仍來找鳳姐。鳳姐故意抱怨他失信,賈瑞急的賭身發誓。鳳姐因見他自投羅網,\begin{note}庚側:可謂因人而使。\end{note}少不得再尋別計令他知改,\begin{note}庚側:四字是作者明阿鳳身份,勿得輕輕看過。\end{note}故又約他道:“今日晚上,你別在那裏了。你在我這房後小過道子裏那間空屋裏等我,可別冒撞了。”\begin{note}庚雙夾:伏得妙!\end{note}賈瑞道:“果真?”鳳姐道:“誰可哄你,你不信就別來。”\begin{note}庚側:緊一句。\end{note}\begin{note}蒙側:大士心腸。\end{note}賈瑞道:“來,來,來。死也要來!”\begin{note}庚雙夾:不差。\end{note}鳳姐道:“這會子你先去罷。”賈瑞料定晚間必妥,\begin{note}庚側:未必。\end{note}此時先去了。鳳姐在這裏便點兵派將,\begin{note}庚側:四字用得新,必有新文字好看。\end{note}\begin{note}蒙側:新文,最妙!\end{note}設下圈套。
\end{parag}


\begin{parag}
    那賈瑞只盼不到夜上,偏生家裏有親戚又來了,\begin{note}庚雙夾:專能忙中寫閒,狡猾之甚!\end{note}直等吃了晚飯纔去,那天已有掌燈時候。又等他祖父安歇了,方溜進榮府,直往那夾道中屋子裏來等著,熱鍋上的螞蟻一般,\begin{note}蒙側:有心人記著,其實苦惱。\end{note}只是幹轉。左等不見人影,右聽也沒聲音,心下自思:“別是又不來了,又凍我一夜不成?”\begin{note}蒙側:似醒非醒語。\end{note}正自胡猜,只見黑魆魆的來了一個人,\begin{note}庚側:真到了。\end{note}賈瑞便意定是鳳姐,不管皁白,餓虎一般,等那人剛至門前,便如貓兒捕鼠的一般,抱住叫道:“親嫂子,等死我了。”說著,抱到屋裏炕上就親嘴扯褲子,滿口裏“親孃”“親爹”的亂叫起來。\begin{note}蒙側:醜態可笑。\end{note}那人只不做聲,\begin{note}庚側:好極!\end{note}賈瑞拉了自己褲子,硬幫幫的就想頂入。\begin{note}庚側:將到矣。\end{note}忽然燈光一閃,只見賈薔舉著個捻子照道:“誰在屋裏?”只見炕上那人笑道:“瑞大叔要臊我呢。”賈瑞一見,卻是賈蓉,\begin{note}庚雙夾:奇絕!\end{note}真臊的無地可入,\begin{note}庚側:亦未必真。\end{note}不知要怎麼樣纔好,回身就要跑,被賈薔一把揪住道:“別走!如今璉二嬸已經告到太太跟前,\begin{note}庚側:好題目。\end{note}說你無故調戲他。\begin{note}庚眉:調戲還要有故?一笑!\end{note}他暫用了個脫身計,哄你在這邊等著,太太氣死過去,\begin{note}庚側:好大題目。\end{note}因此叫我來拿你。剛纔你又攔住他,沒的說,跟我去見太太!”
\end{parag}


\begin{parag}
    賈瑞聽了,魂不附體,只說:“好侄兒,只說沒有見我,明日我重重的謝你。”賈薔道:“你若謝我,放你不值什麼,只不知你謝我多少?況且口說無憑,寫一文契來。”賈瑞道:“這如何落紙呢?”\begin{note}庚側:也知寫不得。一嘆!\end{note}賈薔道:“這也不妨,寫一個賭錢輸了外人賬目,借頭家銀若干兩便罷。”賈瑞道:“這也容易。只是此時無紙筆。”賈薔道:“這也容易。”說罷,翻身出來,紙筆現成,\begin{note}庚側:二字妙!\end{note}拿來命賈瑞寫。他兩作好作歹,只寫了五十兩銀,然後畫了押,賈薔收起來。然後撕羅賈蓉。\begin{note}蒙側:可憐至此!好事者當自度。\end{note}賈蓉先咬定牙不依,只說:“明日告訴族中的人評評理。”賈瑞急的至於叩頭。賈薔做好做歹的,\begin{note}蒙側:此是加一倍法。\end{note}也寫了一張五十兩欠契才罷。賈薔又道:“如今要放你,我就擔著不是。\begin{note}庚雙夾:又生波瀾。\end{note}老太太那邊的門早已關了,老爺正在廳上看南京的東西,那一條路定難過去,如今只好走後門。若這一走,倘或遇見了人,連我也完了。等我們先去哨探哨探,再來領你。這屋你還藏不得,少時就來堆東西。等我尋個地方。”說畢,拉著賈瑞,仍熄了燈,\begin{note}庚雙夾:細。\end{note}出至院外,摸著大臺磯底下,說道:“這窩兒裏好,你只蹲著,別哼一聲,等我們來再動。”\begin{note}庚側:未必如此收場。\end{note}說畢,二人去了。
\end{parag}


\begin{parag}
    賈瑞此時身不由己,只得蹲在那裏。心下正盤算,只聽頭頂上一聲響,譁拉拉一淨桶尿糞從上面直潑下來,可巧澆了他一頭一身,賈瑞掌不住噯喲了一聲,忙又掩住口,\begin{note}庚雙夾:更奇。\end{note}不敢聲張,滿頭滿臉渾身皆是尿屎,冰冷打戰。\begin{note}庚側:餘料必有新奇解恨文字收場,方是《石頭記》筆力。\end{note}\begin{note}庚眉:瑞奴實當如是報之。此一節可入《西廂記》批評內十大快中。畸笏。\end{note}\begin{note}蒙側:這也未必不是預爲埋伏者。總是慈悲設教,遇難教者,不得不現三頭六臂,並喫人心、喝人血之相,以警戒之耳。\end{note}只見賈薔跑來叫:“快走,快走!”賈瑞如得了命,三步兩步從後門跑到家裏,天已三更,只得叫門。開門人見他這般光景,問是怎的。少不得撒謊說:“黑了,失腳掉在茅廁裏了。”一面到自己房中更衣洗濯,心下方想到是鳳姐頑他,因此發一回恨;再想想鳳姐的模樣兒,\begin{note}庚側:欲根未斷。\end{note}又恨不得一時摟在懷,一夜竟不曾閤眼。
\end{parag}


\begin{parag}
    自此滿心想鳳姐,\begin{note}庚眉:此刻還不回頭,真自尋死路矣。\end{note}\begin{note}蒙側:孫行者非有緊箍兒,雖老君之爐、五行之山,何嘗屈其一二?\end{note}只不敢往榮府去了。賈蓉兩個常常的來索銀子,他又怕祖父知道,正是相思尚且難禁,更又添了債務;日間工課又緊,他二十來歲之人,尚未娶親,邇來想著鳳姐,未免有那指頭告了消乏等事;更兼兩回凍惱奔波,\begin{note}庚雙夾:寫得歷歷病源,如何不死?\end{note}因此三五下里夾攻,\begin{note}庚側:所謂步步緊。\end{note}不覺就得了一病:心內發膨脹,口內無滋味,腳下如綿,眼中似醋,黑夜作燒,白晝常倦,下溺連精,嗽痰帶血。諸如此症,不上一年,都添全了。\begin{note}庚側:簡潔之至!\end{note}於是不能支持,一頭睡倒,合上眼還只夢魂顛倒,滿口亂說胡話,驚怖異常。百般請醫治療,諸如肉桂、附子、鱉甲、麥冬、玉竹等藥,吃了有幾十斤下去,也不見個動靜。\begin{note}庚雙夾:說得有趣。\end{note}
\end{parag}


\begin{parag}
    倏又臘盡春回,這病更又沉重。代儒也著了忙,各處請醫療治,皆不見效。因後來喫“獨蔘湯”,代儒如何有這力量,只得往榮府來尋。王夫人命鳳姐秤二兩給他,\begin{note}庚雙夾:王夫人之慈若是。\end{note}鳳姐回說:“前兒新近都替老太太配了藥,那整的太太又說留著送楊提督的太太配藥,偏生昨兒我已送了去了。”王夫人道:“就是咱們這邊沒了,你打發個人往你婆婆那邊問問,或是你珍大哥哥那府裏再尋些來,湊著給人家。喫好了,救人一命,也是你的好處。”\begin{note}庚雙夾:夾寫王夫人。\end{note}鳳姐聽了,也不遣人去尋,只得將些渣末泡須湊了幾錢,命人送去,只說:\begin{note}蒙側:“只說”。\end{note}“太太送來的,再也沒了。”然後回王夫人說:“都尋了來,共湊了有二兩多送去。”\begin{note}庚雙夾:然便有二兩獨蔘湯,賈瑞固亦不能微好,又豈能望好,但鳳姐之毒何如是?終是瑞之自失也。\end{note}
\end{parag}


\begin{parag}
    那賈瑞此時要命心勝,無藥不喫,只是白花錢,不見效。忽然這日有個跛足道人\begin{note}庚雙夾:自甄士隱隨君一去,別來無恙否?\end{note}來化齋,口稱專治冤業之症。賈瑞偏生在內就聽見了,直著聲叫喊\begin{note}庚雙夾:如聞其聲,吾不忍聽也。\end{note}說:“快請進那位菩薩來救我!”一面叫,一面在枕上叩首。\begin{note}庚雙夾:如見其形,吾不忍看也。\end{note}衆人只得帶了那道士進來。賈瑞一把拉住,連叫:“菩薩救我!”\begin{note}庚雙夾:人之將死,其言也哀,作者如何下筆?\end{note}那道士嘆道:“你這病非藥可醫!我有個寶貝與你,你天天看時,此命可保矣。”說畢,從褡褳中\begin{note}庚雙夾:妙極!此褡褳猶是士隱所搶背者乎?\end{note}取出一面鏡子來\begin{note}庚雙夾:凡看書人從此細心體貼,方許你看,否則此書哭矣。\end{note}——兩面皆可照人,\begin{note}庚雙夾:此書表裏皆有喻也。\end{note}鏡把上面鏨著“風月寶鑑”四字\begin{note}庚雙夾:明點。\end{note}——遞與賈瑞道:“這物出自太虛幻境空靈殿上,警幻仙子所制,\begin{note}庚雙夾:言此書原系空虛幻設。\end{note}\begin{note}庚眉:與“紅樓夢”呼應。\end{note}專治邪思妄動之症,\begin{note}庚雙夾:畢真。\end{note}有濟世保生之功。\begin{note}庚雙夾:畢真。\end{note}所以帶他到世上,單與那些聰明俊傑、風雅王孫等看照。\begin{note}庚雙夾:所謂無能紈絝是也。\end{note}千萬不可照正面,\begin{note}庚側:誰人識得此句!\end{note}\begin{note}庚雙夾:觀者記之,不要看這書正面,方是會看。\end{note}只照他的背面,\begin{note}庚雙夾:記之。\end{note}要緊,要緊!三日後吾來收取,管叫你好了。”說畢,佯常而去,衆人苦留不住。
\end{parag}


\begin{parag}
    賈瑞收了鏡子,想道:“這道士倒有些意思,我何不照一照試試。”想畢,拿起“風月鑑”來,向反面一照,只見一個骷髏立在裏面,\begin{note}庚雙夾:所謂“好知青冢骷髏骨,就是紅樓掩面人”是也。作者好苦心思。\end{note}唬得賈瑞連忙掩了,罵:“道士混賬,如何嚇我!”“我倒再照照正面是什麼。”想著,又將正面一照,只見鳳姐站在裏面招手\begin{note}庚側:可怕是“招手”二字。\end{note}叫他。\begin{note}庚雙夾:奇絕!\end{note}賈瑞心中一喜,盪悠悠的覺得進了鏡子,\begin{note}庚雙夾:寫得奇峭,真好筆墨。\end{note}與鳳姐雲雨一番,鳳姐仍送他出來。到了牀上,“噯喲”了一聲,一睜眼,鏡子從手裏掉過來,仍是反面立著一個骷髏。賈瑞自覺汗津津的,底下已遺了一灘精。\begin{note}蒙側:此一句力如龍象,意謂:正面你方纔已自領略了,你也當思想反面纔是。\end{note}心中到底不足,又翻過正面來,只見鳳姐還招手叫他,他又進去。如此三四次。到了這次,剛要出鏡子來,只見兩個人走來,拿鐵鎖把他套住,拉了就走。\begin{note}庚雙夾:所謂醉生夢死也。\end{note}賈瑞叫道:“讓我拿了鏡子再走!”\begin{note}庚雙夾:可憐!大衆齊來看此。\end{note}\begin{note}蒙側:這是作書者之立意,要寫情種,故於此試一深寫之。在賈瑞則是求仁而得仁,未嘗不含笑九泉,雖死亦不解脫者,悲矣!\end{note}——只說了這句,就再不能說話了。
\end{parag}


\begin{parag}
    旁邊伏侍的賈瑞的衆人,只見他先還拿著鏡子照,落下來,仍睜開眼拾在手內,末後鏡子落下來便不動了。衆人上來看看,已沒了氣,身子底下冰涼漬溼一大灘精,這才忙著穿衣抬牀。代儒夫婦哭的死去活來,大罵道士,“是何妖鏡!\begin{note}庚雙夾:此書不免腐儒一謗。\end{note}若不早毀此物,\begin{note}庚雙夾:凡野史俱可毀,獨此書不可毀。\end{note}遺害於世不小。”\begin{note}庚雙夾:腐儒。\end{note}遂命架火來燒,只聽鏡內哭道:“誰叫你們瞧正面了!你們自己以假爲真,何苦來燒我?”\begin{note}庚雙夾:觀者記之。\end{note}正哭著,只見那跛足道人從外跑來,喊道:“誰毀‘風月鑑’,吾來救也!”說著,直入中堂,搶入手內,飄然去了。
\end{parag}


\begin{parag}
    當下,代儒料理喪事,各處去報喪。三日起經,七日發引,寄靈於鐵檻寺,\begin{note}庚雙夾:所謂“鐵門限”事業。先安一開路道之人,以備秦氏仙柩有方也。\end{note}日後帶回原籍。當下賈家衆人齊來弔問,榮府賈赦贈銀二十兩,賈政亦是二十兩,寧國府賈珍亦有二十兩,別者族中人貧富不等,或三兩五兩,不可勝數。另有各同窗家分資,也湊了二三十兩。代儒家道雖然淡薄,倒也豐豐富富完了此事。
\end{parag}


\begin{parag}
    誰知這年冬底,林如海的書信寄來,卻爲身染重疾,寫書特來接林黛玉回去。\begin{note}蒙側:須要林黛玉長住,偏要暫離。\end{note}賈母聽了,未免又加憂悶,只得忙忙的打點黛玉起身。寶玉大不自在,爭奈父女之情,也不好攔勸。於是賈母定要賈璉送他去,仍叫帶回來。一應土儀盤纏,不消煩說,自然要妥貼。作速擇了日期,賈璉與林黛玉辭別了賈母等,帶領僕從,登舟往揚州去了。要知端的,且聽下回分解。
\end{parag}


\begin{parag}
    \begin{note}庚:此回忽遣黛玉去者,正爲下回可兒之文也。若不遣去,只寫可兒、阿鳳等人,卻置黛玉於榮府,成何文哉?故必遣去,方好放筆寫秦,方不脫髮。況黛玉乃書中正人,秦爲陪客,豈因陪而失正耶?後大觀園方是寶玉、寶釵、黛玉等正經文字,前皆系陪襯之文也。\end{note}
\end{parag}


\begin{parag}
    \begin{note}蒙回末總評:儒家正心,道者煉心,釋輩戒心。可見此心無有不到,無不能入者,獨畏其入於邪而不反,故用正煉戒以縛之。請看賈瑞一起念,及至於死,專誠不二,雖經兩次警教,毫無反悔,可謂癡子,可謂愚情。相乃可思,不能相而獨欲思,豈逃傾頹?作者以此作一新樣情理,以助解者生笑,以爲癡者設以棒喝耳!\end{note}
\end{parag}