\chap{八十七}{感深秋撫琴悲往事 坐禪寂走火入邪魔}



\begin{parag}
    卻說黛玉叫進寶釵家的女人來,問了好,呈上書子。黛玉叫他去喝茶,便將寶釵來書打開看時,只見上面寫着:
\end{parag}


\begin{qute2sp}
    妹生辰不偶,家運多艱,姊妹伶仃,萱親衰邁。兼之猇聲狺語,旦暮無休。更遭慘禍飛災,不啻驚風密雨。夜深輾側,愁緒何堪。屬在同心,能不爲之愍惻乎?回憶海棠結社,序屬清秋,對菊持螯,同盟歡洽。猶記“孤標傲世偕誰隱,一樣花開爲底遲”之句,未嘗不嘆冷節遺芳,如吾兩人也。感懷觸緒,聊賦四章,匪曰無故呻吟,亦長歌當哭之意耳。悲時序之遞嬗兮,又屬清秋。感遭家之不造兮,獨處離愁。北堂有萱兮,何以忘憂?無以解憂兮,我心咻咻。一解。雲憑憑兮秋風酸,步中庭兮霜葉幹。何去何從兮,失我故歡。靜言思之兮惻肺肝!二解。惟鮪有潭兮,惟鶴有梁。鱗甲潛伏兮,羽毛何長!搔首問兮茫茫,高天厚地兮,誰知餘之永傷。三解。銀河耿耿兮寒氣侵,月色橫斜兮玉漏沉。憂心炳炳兮發我哀吟,吟復吟兮寄我知音。四解。
\end{qute2sp}


\begin{parag}
    黛玉看了,不勝傷感。又想:“寶姐姐不寄與別人,單寄與我,也是惺惺惜惺惺的意思。”正在沉吟,只聽見外面有人說道:“林姐姐在家裏呢麼?”黛玉一面把寶釵的書迭起,口內便答應道:“是誰?”正問着,早見幾個人進來,卻是探春,湘雲,李紋,李綺。彼此問了好,雪雁倒上茶來,大家喝了,說些閒話。因想起前年的菊花詩來,黛玉便道:“寶姐姐自從挪出去,來了兩遭,如今索性有事也不來了,真真奇怪。我看他終久還來我們這裏不來。”探春微笑道:“怎麼不來,橫豎要來的。如今是他們尊嫂有些脾氣,姨媽上了年紀的人,又兼有薛大哥的事,自然得寶姐姐照料一切,那裏還比得先前有工夫呢。”正說着,忽聽得忽喇喇一片風聲,吹了好些落葉,打在窗紙上。停了一回兒,又透過一陣清香來。衆人聞着,都說道:“這是何處來的香風?這象什麼香?”黛玉道:“好象木樨香。”探春笑道:“林姐姐終不脫南邊人的話,這大九月裏的,那裏還有桂花呢。”黛玉笑道:“原是啊,不然怎麼不竟說是桂花香只說似乎象呢。”湘雲道:“三姐姐,你也別說。你可記得‘十里荷花,三秋桂子’?在南邊,正是晚桂開的時候了。你只沒有見過罷了,等你明日到南邊去的時候,你自然也就知道了。”探春笑道:“我有什麼事到南邊去?況且這個也是我早知道的,不用你們說嘴。”李紋李綺只抿着嘴兒笑。黛玉道:“妹妹,這可說不齊。俗語說,‘人是地行仙’,今日在這裏,明日就不知在那裏。譬如我,原是南邊人,怎麼到了這裏呢?”湘雲拍着手笑道:“今兒三姐姐可叫林姐姐問住了。不但林姐姐是南邊人到這裏,就是我們這幾個人就不同。也有本來是北邊的,也有根子是南邊,生長在北邊的,也有生長在南邊,到這北邊的,今兒大家都湊在一處。可見人總有一個定數,大凡地和人總是各自有緣分的。”衆人聽了都點頭,探春也只是笑。又說了一會子閒話兒,大家散出。黛玉送到門口,大家都說:“你身上纔好些,別出來了,看着了風。”於是黛玉一面說着話兒,一面站在門口又與四人殷勤了幾句,便看着他們出院去了。進來坐着,看看已是林鳥歸山,夕陽西墜。因史湘雲說起南邊的話,便想着“父母若在,南邊的景緻,春花秋月,水秀山明,二十四橋,六朝遺蹟。不少下人伏侍,諸事可以任意,言語亦可不避。香車畫舫,紅杏青帘,惟我獨尊。今日寄人籬下,縱有許多照應,自己無處不要留心。不知前生作了什麼罪孽,今生這樣孤悽。真是李後主說的‘此間日中只以眼淚洗面’矣!”一面思想,不知不覺神往那裏去了。
\end{parag}


\begin{parag}
    紫鵑走來,看見這樣光景,想着必是因剛纔說起南邊北邊的話來,一時觸着黛玉的心事了,便問道:“姑娘們來說了半天話,想來姑娘又勞了神了。剛纔我叫雪雁告訴廚房裏給姑娘作了一碗火肉白菜湯,加了一點兒蝦米兒,配了點青筍紫菜。姑娘想着好麼?”黛玉道:“也罷了。”紫鵑道:“還熬了一點江米粥。”黛玉點點頭兒,又說道:“那粥該你們兩個自己熬了,不用他們廚房裏熬纔是。”紫鵑道:“我也怕廚房裏弄的不乾淨,我們各自熬呢。就是那湯,我也告訴雪雁和柳嫂兒說了,要弄乾淨着。柳嫂兒說了,他打點妥當,拿到他屋裏叫他們五兒瞅着燉呢。”黛玉道:“我倒不是嫌人家骯贓,只是病了好些日子,不周不備,都是人家。這會子又湯兒粥兒的調度,未免惹人厭煩。”說着,眼圈兒又紅了。紫鵑道:“姑娘這話也是多想。姑娘是老太太的外孫女兒,又是老太太心坎兒上的。別人求其在姑娘跟前討好兒還不能呢,那裏有抱怨的。”黛玉點點頭兒,因又問道:“你才說的五兒,不是那日和寶二爺那邊的芳官在一處的那個女孩兒?”紫鵑道:“就是他。”黛玉道:“不聽見說要進來麼?”紫鵑道:“可不是,因爲病了一場,後來好了纔要進來,正是晴雯他們鬧出事來的時候,也就耽擱住了。”黛玉道:“我看那丫頭倒也還頭臉兒乾淨。”說着,外頭婆子送了湯來。雪雁出來接時,那婆子說道:“柳嫂兒叫回姑娘,這是他們五兒作的,沒敢在大廚房裏作,怕姑娘嫌骯贓。”雪雁答應着接了進來。黛玉在屋裏已聽見了,吩咐雪雁告訴那老婆子回去說,叫他費心。雪雁出來說了,老婆子自去。這裏雪雁將黛玉的碗箸安放在小几兒上,因問黛玉道:“還有咱們南來的五香大頭菜,拌些麻油醋可好麼?”黛玉道:“也使得,只不必累贅了。”一面盛上粥來,黛玉吃了半碗,用羹匙舀了兩口湯喝,就擱下了。兩個丫鬟撤了下來,拭淨了小几端下去,又換上一張常放的小几。黛玉漱了口,盥了手,便道:“紫鵑,添了香了沒有?”紫鵑道:“就添去。”黛玉道:“你們就把那湯和粥吃了罷,味兒還好,且是乾淨。待我自己添香罷。”兩個人答應了,在外間自喫去了。
\end{parag}


\begin{parag}
    這裏黛玉添了香,自己坐着。纔要拿本書看,只聽得園內的風自西邊直透到東邊,穿過樹枝,都在那裏唏哩譁喇不住的響。一回兒,檐下的鐵馬也只管叮叮噹噹的亂敲起來。一時雪雁先喫完了,進來伺候。黛玉便問道:“天氣冷了,我前日叫你們把那些小毛兒衣服晾晾,可曾晾過沒有?”雪雁道:“都晾過了。”黛玉道:“你拿一件來我披披。”雪雁走去將一包小毛衣服抱來,打開氈包,給黛玉自揀。只見內中夾着個絹包兒,黛玉伸手拿起打開看時,卻是寶玉病時送來的舊手帕,自己題的詩,上面淚痕猶在,裏頭卻包着那剪破了的香囊扇袋並寶玉通靈玉上的穗子。原來晾衣服時從箱中撿出,紫鵑恐怕遺失了,遂夾在這氈包裏的。這黛玉不看則已,看了時也不說穿那一件衣服,手裏只拿着那兩方手帕,呆呆的看那舊詩。看了一回,不覺的簌簌淚下。紫鵑剛從外間進來,只見雪雁正捧着一氈包衣裳在旁邊呆立,小几上卻擱着剪破的香囊,兩三截兒扇袋和那鉸折了的穗子,黛玉手中自拿着兩方舊帕,上邊寫着字跡,在那裏對着滴淚。正是:
\end{parag}


\begin{poem}
    \begin{pl}
        失意人逢失意事,新啼痕間舊啼痕。
    \end{pl}
\end{poem}


\begin{parag}
    紫鵑見了這樣,知是他觸物傷情,感懷舊事,料道勸也無益,只得笑着道:“姑娘還看那些東西作什麼,那都是那幾年寶二爺和姑娘小時一時好了,一時惱了,鬧出來的笑話兒。要象如今這樣斯抬斯敬,那裏能把這些東西白遭塌了呢。”紫鵑這話原給黛玉開心,不料這幾句話更提起黛玉初來時和寶玉的舊事來,一發珠淚連綿起來。紫鵑又勸道:“雪雁這裏等着呢,姑娘披上一件罷。”那黛玉才把手帕撂下。紫鵑連忙拾起,將香袋等物包起拿開。這黛玉方披了一件皮衣,自己悶悶的走到外間來坐下。回頭看見案上寶釵的詩啓尚未收好,又拿出來瞧了兩遍,嘆道:“境遇不同,傷心則一。不免也賦四章,翻入琴譜,可彈可歌,明日寫出來寄去,以當和作。”便叫雪雁將外邊桌上筆硯拿來,濡墨揮毫,賦成四迭。又將琴譜翻出,借他《猗蘭》《思賢》兩操,合成音韻,與自己做的配齊了,然後寫出,以備送與寶釵。又即叫雪雁向箱中將自己帶來的短琴拿出,調上弦,又操演了指法。黛玉本是個絕頂聰明人,又在南邊學過幾時,雖是手生,到底一理就熟。撫了一番,夜已深了,便叫紫鵑收拾睡覺。不題。
\end{parag}


\begin{parag}
    卻說寶玉這日起來梳洗了,帶着焙茗正往書房中來,只見墨雨笑嘻嘻的跑來迎頭說道:“二爺今日便宜了,太爺不在書房裏,都放了學了。”寶玉道:“當真的麼?”墨雨道:“二爺不信,那不是三爺和蘭哥兒來了。”寶玉看時,只見賈環賈蘭跟着小廝們,兩個笑嘻的嘴裏咭咭呱呱不知說些什麼,迎頭來了。見了寶玉,都垂手站住。寶玉問道:“你們兩個怎麼就回來了?”賈環道:“今日太爺有事,說是放一天學,明兒再去呢。”寶玉聽了,方回身到賈母賈政處去稟明瞭,然後回到怡紅院中。襲人問道:“怎麼又回來了?”寶玉告訴了他,只坐了一坐兒,便往外走。襲人道:“往那裏去,這樣忙法?就放了學,依我說也該養養神兒了。”寶玉站住腳,低了頭,說道:“你的話也是。但是好容易放一天學,還不散散去,你也該可憐我些兒了。”襲人見說的可憐,笑道:“由爺去罷。”正說着,端了飯來。寶玉也沒法兒,只得且喫飯,三口兩口忙忙的喫完,漱了口,一溜煙往黛玉房中去了。
\end{parag}


\begin{parag}
    走到門口,只見雪雁在院中晾絹子呢。寶玉因問:“姑娘吃了飯了麼?”雪雁道:“早起喝了半碗粥,懶待喫飯。這時候打盹兒呢。二爺且到別處走走,回來再來罷。”寶玉只得回來。
\end{parag}


\begin{parag}
    無處可去,忽然想起惜春有好幾天沒見,便信步走到蓼風軒來。剛到窗下,只見靜悄悄一無人聲。寶玉打諒他也睡午覺,不便進去。纔要走時,只聽屋裏微微一響,不知何聲。寶玉站住再聽,半日又拍的一響。寶玉還未聽出,只見一個人道:“你在這裏下了一個子兒,那裏你不應麼?”寶玉方知是下大棋,但只急切聽不出這個人的語音是誰。底下方聽見惜春道:“怕什麼,你這麼一喫我,我這麼一應,你又這麼喫,我又這麼應。還緩着一着兒呢,終久連得上。”那一個又道:“我要這麼一喫呢?”惜春道:“阿嗄,還有一着‘反撲’在裏頭呢!我倒沒防備。”寶玉聽了,聽那一個聲音很熟,卻不是他們姊妹。料着惜春屋裏也沒外人,輕輕的掀簾進去。看時不是別人,卻是那櫳翠庵的檻外人妙玉。這寶玉見是妙玉,不敢驚動。妙玉和惜春正在凝思之際,也沒理會。寶玉卻站在旁邊看他兩個的手段。只見妙玉低着頭問惜春道:“你這個‘畸角兒’不要了麼?”惜春道:“怎麼不要。你那裏頭都是死子兒,我怕什麼。”妙玉道:“且別說滿話,試試看。”惜春道:“我便打了起來,看你怎麼樣。”妙玉卻微微笑着,把邊上子一接,卻搭轉一喫,把惜春的一個角兒都打起來了,笑着說道:“這叫做‘倒脫靴勢’。”
\end{parag}


\begin{parag}
    惜春尚未答言,寶玉在旁情不自禁,哈哈一笑,把兩個人都唬了一大跳。惜春道:“你這是怎麼說,進來也不言語,這麼使促狹唬人。你多早晚進來的?”寶玉道:“我頭裏就進來了,看着你們兩個爭這個‘畸角兒’。”說着,一面與妙玉施禮,一面又笑問道:“妙公輕易不出禪關,今日何緣下凡一走?”妙玉聽了,忽然把臉一紅,也不答言,低了頭自看那棋。寶玉自覺造次,連忙陪笑道:“倒是出家人比不得我們在家的俗人,頭一件心是靜的。靜則靈,靈則慧。”寶玉尚未說完,只見妙玉微微的把眼一抬,看了寶玉一眼,復又低下頭去,那臉上的顏色漸漸的紅暈起來。寶玉見他不理,只得訕訕的旁邊坐了。惜春還要下子,妙玉半日說道:“再下罷。”便起身理理衣裳,重新坐下,癡癡的問着寶玉道:“你從何處來?”寶玉巴不得這一聲,好解釋前頭的話,忽又想道:“或是妙玉的機鋒。”轉紅了臉答應不出來。妙玉微微一笑,自和惜春說話。惜春也笑道:“二哥哥,這什麼難答的,你沒的聽見人家常說的‘從來處來’麼。這也值得把臉紅了,見了生人的似的。”妙玉聽了這話,想起自家,心上一動,臉上一熱,必然也是紅的,倒覺不好意思起來。因站起來說道:“我來得久了,要回庵裏去了。”惜春知妙玉爲人,也不深留,送出門口。妙玉笑道:“久已不來這裏,彎彎曲曲的,回去的路頭都要迷住了。”寶玉道:“這倒要我來指引指引何如?”妙玉道:“不敢,二爺前請。”於是二人別了惜春,離了蓼風軒,彎彎曲曲,走近瀟湘館,忽聽得叮咚之聲。妙玉道:“那裏的琴聲?”寶玉道:“想必是林妹妹那裏撫琴呢。”妙玉道:“原來他也會這個,怎麼素日不聽見提起?”寶玉悉把黛玉的事述了一遍,因說:“咱們去看他。”妙玉道:“從古只有聽琴,再沒有‘看琴’的。”寶玉笑道:“我原說我是個俗人。”說着,二人走至瀟湘館外,在山子石坐着靜聽,甚覺音調清切。只聽得低吟道:
\end{parag}


\begin{poem}
    \begin{pl}
        風蕭蕭兮秋氣深,美人千里兮獨沉吟。望故鄉兮何處,倚欄杆兮涕沾襟。
    \end{pl}
\end{poem}


\begin{parag}
    歇了一回,聽得又吟道:
\end{parag}


\begin{poem}
    \begin{pl}
        山迢迢兮水長,照軒窗兮明月光。耿耿不寐兮銀河渺茫,羅衫怯怯兮風露涼。
    \end{pl}
\end{poem}


\begin{parag}
    又歇了一歇。妙玉道:“剛纔‘侵’字韻是第一迭,如今‘陽’字韻是第二迭了。咱們再聽。”裏邊又吟道:
\end{parag}

\begin{poem}
    \begin{pl}
        子之遭兮不自由,予之遇兮多煩憂。之子與我兮心焉相投,思古人兮俾無尤。
    \end{pl}
\end{poem}


\begin{parag}
    妙玉道:“這又是一拍。何憂思之深也!”寶玉道:“我雖不懂得,但聽他音調,也覺得過悲了。”裏頭又調了一回弦。妙玉道:“君弦太高了,與無射律只怕不配呢。”裏邊又吟道:
\end{parag}

\begin{poem}
    \begin{pl}
        人生斯世兮如輕塵,天上人間兮感夙因。感夙因兮不可惙,素心如何天上月。
    \end{pl}
\end{poem}


\begin{parag}
    妙玉聽了,呀然失色道:“如何忽作變徵之聲?音韻可裂金石矣。只是太過。”寶玉道:“太過便怎麼?”妙玉道:“恐不能持久。”正議論時,聽得君弦蹦的一聲斷了。妙玉站起來連忙就走。寶玉道:“怎麼樣?”妙玉道:“日後自知,你也不必多說。”竟自走了。弄得寶玉滿肚疑團,沒精打彩的歸至怡紅院中,不表。單說妙玉歸去,早有道婆接着,掩了庵門,坐了一回,把“禪門日誦”唸了一遍。吃了晚飯,點上香拜了菩薩,命道婆自去歇着,自己的禪牀靠背俱已整齊,屏息垂簾,跏趺坐下,斷除妄想,趨向真如。坐到三更過後,聽得屋上骨碌碌一片瓦響,妙玉恐有賊來,下了禪牀,出到前軒,但見雲影橫空,月華如水。那時天氣尚不很涼,獨自一個憑欄站了一回,忽聽房上兩個貓兒一遞一聲廝叫。那妙玉忽想起日間寶玉之言,不覺一陣心跳耳熱。自己連忙收懾心神,走進禪房,仍到禪牀上坐了。怎奈神不守舍,一時如萬馬奔馳,覺得禪牀便恍蕩起來,身子已不在庵中。便有許多王孫公子要求娶他,又有些媒婆扯扯拽拽扶他上車,自己不肯去。一回兒又有盜賊劫他,持刀執棍的逼勒,只得哭喊求救。早驚醒了庵中女尼道婆等衆,都拿火來照看。只見妙玉兩手撒開,口中流沫。急叫醒時,只見眼睛直豎,兩顴鮮紅,罵道:“我是有菩薩保佑,你們這些強徒敢要怎麼樣!”衆人都唬的沒了主意,都說道:“我們在這裏呢,快醒轉來罷。”妙玉道:“我要回家去,你們有什麼好人送我回去罷。”道婆道:“這裏就是你住的房子。”說着,又叫別的女尼忙向觀音前禱告,求了籤,翻開籤書看時,是觸犯了西南角上的陰人。就有一個說:“是了。大觀園中西南角上本來沒有人住,陰氣是有的。”一面弄湯弄水的在那裏忙亂。那女尼原是自南邊帶來的,伏侍妙玉自然比別人盡心,圍着妙玉,坐在禪牀上。妙玉回頭道:“你是誰?”女尼道:“是我。”妙玉仔細瞧了一瞧,道:“原來是你。”便抱住那女尼嗚嗚咽咽的哭起來,說道:“你是我的媽呀,你不救我,我不得活了。”那女尼一面喚醒他,一面給他揉着。道婆倒上茶來喝了,直到天明才睡了。
\end{parag}


\begin{parag}
    女尼便打發人去請大夫來看脈,也有說是思慮傷脾的,也有說是熱入血室的,也有說是邪祟觸犯的,也有說是內外感冒的,終無定論。後請得一個大夫來看了,問:“曾打坐過沒有?”道婆說道:“向來打坐的。”大夫道:“這病可是昨夜忽然來的麼?”道婆道:“是。”大夫道:“這是走魔入火的原故。”衆人問:“有礙沒有?”大夫道:“幸虧打坐不久,魔還入得淺,可以有救。”寫了降伏心火的藥,吃了一劑,稍稍平復些。外面那些遊頭浪子聽見了,便造作許多謠言說:“這樣年紀,那裏忍得住。況且又是很風流的人品,很乖覺的性靈,以後不知飛在誰手裏,便宜誰去呢。”過了幾日,妙玉病雖略好,神思未復,終有些恍惚。
\end{parag}


\begin{parag}
    一日惜春正坐着,彩屏忽然進來回道:“姑娘知道妙玉師父的事嗎?”惜春道:“他有什麼事?”彩屏道:“我昨日聽見邢姑娘和大奶奶那裏說呢。他自從那日和姑娘下棋回去,夜間忽然中了邪,嘴裏亂嚷說強盜來搶他來了,到如今還沒好。姑娘你說這不是奇事嗎。”惜春聽了,默默無語,因想:“妙玉雖然潔淨,畢竟塵緣未斷。可惜我生在這種人家不便出家。我若出了家時,那有邪魔纏擾,一念不生,萬緣俱寂。”想到這裏,驀與神會,若有所得,便口占一偈雲:
\end{parag}

\begin{poem}
    \begin{pl}
        大造本無方,云何是應住。
    \end{pl}


    \begin{pl}
        既從空中來,應向空中去。
    \end{pl}
\end{poem}


\begin{parag}
    佔畢,即命丫頭焚香。自己靜坐了一回,又翻開那棋譜來,把孔融王積薪等所著看了幾篇。內中“荷葉包蟹勢”、“黃鶯搏兔勢”都不出奇,“三十六局殺角勢”一時也難會難記,獨看到“八龍走馬”,覺得甚有意思。正在那裏作想,只聽見外面一個人走進院來,連叫彩屏。未知是誰,下回分解。
\end{parag}