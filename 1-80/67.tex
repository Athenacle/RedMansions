\chap{六十七}{見土儀顰卿思故里 聞祕事鳳姐訊家童}

\begin{parag}
    話說尤三姐自戕之後,尤老孃以及尤二姐、賈珍、尤氏並賈蓉、賈璉等俱不勝悲慟傷感,忙着買棺盛殮,送往城外埋葬。柳湘蓮見尤三姐身亡,迷性不悟,尚有癡情眷戀,卻被道人數句偈言打破迷關,竟自削髮出家,隨一瘋道人飄然而去,不知何往。
\end{parag}


\begin{parag}
    薛姨媽聞知湘蓮已說定了尤三姐,正打算替他買房置器,擇日迎娶過門,以報他救命之恩。忽有家中小廝吿知尤三姐自戕與柳湘蓮出家之事,心甚嘆息。時值寶釵從園中過來,聽了這些話,並不在意,乃勸道:“俗語說的好,‘天有不測風雲,人有旦夕禍福’。這也是他們前生命定。前日媽媽爲他救了哥哥,商量著替他料理,如今已經死的死了,走的走了,依我說,也只好由他罷了。媽媽也不必爲他們傷感了。倒是自從哥哥打江南迴來了一二十日,販了來的貨物,想來也該發完了,媽媽和哥哥商議商議,酬謝酬謝那同去的那張德輝纔是。夥計們辛辛苦苦的,回來幾個月了,也該請一請,別叫人家看著無禮似的。”
\end{parag}


\begin{parag}
    母女正說話間,見薛蟠自外而入,眼中尚有淚痕,一進門來,便向他母親拍手說道:“媽媽可知道柳二哥尤三姐的事麼?”薛姨媽說:“我才聽見說,正在這裏合你妹妹說這件公案呢。”薛蟠道:“媽媽可聽見說湘蓮跟著一個道士出了家了麼?”薛姨媽道:“這越發奇了。怎麼柳相公那樣一個年輕的聰明人,一時胡塗了,就跟著道士去了呢?我想你們好了一場,他又無父母兄弟,隻身一人在此,你該各處找找他纔是。靠那道士,能往那裏遠去?左不過是在這方近左右的廟裏寺裏罷了。”薛蟠說:“何嘗不是呢?我一聽見這個信兒,就連忙帶了小廝們在各處尋找,連一個影兒也沒有。又去問人,都說沒看見。”
\end{parag}


\begin{parag}
    薛姨媽說:“你既找尋過,沒有,也算把你做朋友的心盡了。再者,你妹妹才說你也回家半個多月了,想貨物也該發完了,也該擺桌酒,給張德輝和夥計們,道道乏纔是。”薛蟠聽說,便道:“媽媽說的很是。倒是妹妹想的周到。因這些日子,爲各處發貨,又爲柳二哥的事忙了這幾日,把正經事都誤了。要不然,定了明兒後兒,下帖兒請罷。”薛姨媽道:“由你辦去罷。”
\end{parag}


\begin{parag}
    話猶未了,外面小廝在門外回說:“張管總着人送了兩個箱子來。”薛蟠聽了,便命小廝央門外幾個夥計搬進了兩個夾板夾的大棕箱。薛蟠一見說:“特給媽和妹妹帶來的東西,不是夥計送家裏來,我都忘了。”
\end{parag}


\begin{parag}
    薛姨媽同寶釵問:“是什麼好東西,這樣捆着夾着的?”便命人挑了繩子,去了夾板,開了鎖看時,卻是些綢緞、綾錦、洋貨等家常應用之物。獨有寶釵他的那個箱子裏,除了筆、墨、硯、各色箋紙、香袋、香珠、扇子、扇墜、花粉、胭脂、頭油等物外,還有虎丘帶來的自行人、酒令兒、水銀灌的打筋斗的小小子,沙子燈,一出一出的泥人兒的戲,用青紗罩的匣子裝着,又有在虎丘山上作的薛蟠的小像,泥捏成的與薛蟠毫無相差,以及許多碎小頑意兒的東西。寶釵一見,拿着薛蟠的小像細細看了,又看看他哥哥捂着嘴微笑,再和母兄說了一回閒話。便吩咐鶯兒:“你帶幾個老婆子,將我的這個箱子,拿到園子裏去,我好就近從那邊送人。”說着,便起身辭了母兄往園子裏去了。這裏薛姨媽將自己這個箱子裏的東西取出,一份一份的打點清楚,着鶯兒送往賈母並王夫人等處。
\end{parag}


\begin{parag}
    寶釵隨着箱子到了自己房中,將東西逐件過了目,除將自己留用之外,遂一一配妥當:也有送筆、墨、紙、硯的,也有送香袋、扇子、香墜的,也有送脂粉、頭油的,有單送頑意兒的。一一打點完畢,使鶯兒同一個老婆子,送往各處。
\end{parag}


\begin{parag}
    寶釵送東西的ㄚ頭回來,說:“也有道謝的,也有賞錢的,獨有給巧姐兒的那一份,仍舊拿回來了。”寶釵一見,不知何意,便問:“爲什麼這一份沒送去,還是送了去沒收呢?”鶯兒說:“我方纔給環哥兒送東西的時候,見璉二奶奶往老太太房裏去了。”寶釵說:“二奶奶不在家,你只管交給丫頭們收下,等二奶奶回來,自有他們告訴就是了。”鶯兒聽了,又與老婆子出了園子,到了鳳姐這邊,送了東西,回來見寶釵。
\end{parag}


\begin{parag}
    寶釵問道:“你見了璉二奶奶沒有?”鶯兒說:“我沒見。”寶釵說:“二奶奶還沒有回來?”鶯兒說:“回來是回來了。因豐兒對我說:‘二奶奶自老太太屋裏回來,一臉怒氣,叫了平兒去,唧唧咕咕的說話,也不叫人聽見。你不必見,等我替你回一聲兒就是了。’因此豐兒拿進去,回了二奶奶。我們就回來了。”寶釵聽了,自己納悶,想不出鳳姐是爲什麼生氣。
\end{parag}


\begin{parag}
    衆人不過收了東西,皆說些見面再謝等語而已。惟有林黛玉見是江南家鄉之物,便對着揮淚自嘆。紫鵑深知黛玉心腸,在一旁勸道:“寶姑娘送來這些東西,姑娘看着該喜歡纔是。”
\end{parag}


\begin{parag}
    話猶未畢,只見寶玉已進來。寶玉見黛玉淚痕滿面,便問:“妹妹,又是爲的什麼?”黛玉不答。旁邊紫鵑將嘴向牀後桌上一努,寶玉會意,便往牀上一看,見堆着許多東西,就知道是寶釵送來的。寶玉深知黛玉是因見了江南來的故鄉之物,勾起傷感落淚。便道:“妹妹,你放心!等我明年往江南去,與你帶兩船來。”黛玉聽了這話,說道:“你那裏知道我的緣故。”說着眼淚又流了下來。寶玉忙走到牀前,挨著黛玉坐下,將那些東西一件一件拿起來,擺弄著細瞧,故意問:“這是什麼,叫什麼名字?那是什麼做的,這樣齊整?這是什麼,要它做什麼使用?妹妹,你瞧,這一件可以擺在書閣兒上作陳設,那件放在條案上當古董兒倒好呢!”一味的將些沒要緊的話來支吾,搭訕。黛玉見寶玉可笑的樣子,稍將煩惱丟開。寶玉便說道:“寶姐姐送東西來給咱們,我想著,咱們也該到她那裏道個謝去纔是,不知妹妹可去不去?”黛玉道:“自家姐妹,這倒不必。只是到他那邊,薛大哥回來了,必然告訴他些南邊的新聞故事兒,我去聽聽,只當回了家鄉一趟的。”說著,眼圈兒又紅了。寶玉便站著等他。黛玉只得和他出來,往寶釵那裏去了。
\end{parag}


\begin{parag}
    二人到寶釵處,道了謝,寶玉又口口稱讚泥人兒等物有趣。寶釵笑道:“原不是什麼好東西,不過是遠路帶來的土物兒,大家看著新鮮些就是了。”黛玉道:“這些東西我們小時候倒不理會,如今看見,真是新鮮物兒了。”寶釵因笑道:“妹妹知道,這就是俗語說的‘物離鄉貴’,其實可算什麼呢。”寶玉聽了這話正觸著黛玉方纔的心事,連忙拿話岔開:“明年大哥哥還去江南嗎?——”話沒說完,黛玉早接口道:“——姐姐,你瞧,寶哥哥不是給姐姐來道謝,竟又要定下明年的東西來了。”說的寶釵寶玉都笑了。
\end{parag}


\begin{parag}
    三個人又閒話了一回,因提起黛玉的病來,寶釵勸了一回,因說道:“妹妹若覺著身上不爽快,倒要自己勉強扎掙著出來,各處走走逛逛,散散心,比在屋裏悶坐著到底好些。我那兩日,不是覺著發懶,渾身發熱,只是要歪著?也因爲時氣不好,怕病,因此尋些事情,自己混著。這兩日才覺得好些了。”黛玉道:“姐姐說的何嘗不是?我也是這麼想著呢。”大家又坐了一會子方散。寶玉仍把黛玉送至瀟湘館門首,才各自回去了。
\end{parag}


\begin{parag}
    且說那趙姨娘因見寶釵送環哥兒物件,心中甚喜,滿嘴誇獎:“人人都說寶姑娘會行事,很大方,今日看來,果然不錯。他哥哥能帶了多少東西來,他挨家送到,並不遺漏一處,也不露出誰薄誰厚,連我們他都想到了,若是林姑娘,即或有人帶了東西來,那裏輪得到我們孃兒倆身上呢!可見人會行事,真真露着各別另樣的好。”趙姨娘因環哥兒得了東西,深爲得意,不住的託在掌上擺弄瞧看一會。想寶釵乃系王夫人之表侄女,特要在王夫人跟前賣好兒。自己蠍蠍螫螫的拿着那東西,走至王夫人房中,站在一旁說道:“這是寶姑娘纔給環哥的,他年輕輕的人想得周到,我還給了送東西的小ㄚ頭二百錢。聽說姨太太也給太太送來了,不知是什麼東西?你們瞧瞧這一個門裏頭,就是兩份兒,能有多少呢?怪不得老太太同太太都誇他疼他,果然招人疼。”說着,將手裏的東西遞過去與王夫人瞧,誰知王夫人頭也沒擡,手也沒伸,只口內說了聲“好,給環哥兒頑去罷”,並無正眼看一看。趙姨娘因招了一鼻子灰,滿肚氣惱,無精打彩的回房,將東西丟在一邊,也無人問他,他卻自己咕嘟着嘴,一邊子坐着。
\end{parag}


\begin{parag}
    且說薛蟠聽了母親之言,次日請了張德輝與四位夥計,俱已到齊,不免說些販賣賬目發貨之事。不一時,上席讓坐,薛蟠挨次斟了酒,薛姨媽又使人出來致謝,大家喝著酒說閒話兒。內中一個道:“今兒這席上短了柳二爺。”薛蟠聞言,把眉一皺,嘆口氣道:“什麼是柳二爺,如今不知那裏作‘柳道爺’去了。”衆人都詫異道:“這是怎麼說?”薛蟠便把湘蓮前後事體說了一遍。衆人聽了,越發駭異,因說道:“怪不的。前兒我們在店裏,髣髣髴髴也聽見人吵嚷,說:‘有一個道士,三言兩語,把一個人度了去了。’又說“‘一陣風颳了去了。’只不知是誰。我們正發貨,那裏有閒工夫打聽這個事去?到如今還是似信不信的,誰知就是柳二爺呢?張德輝道:“柳二爺那樣個伶俐人,未必是真跟了道士去罷。他原會些武藝,又有力量,或看破那道士的妖術邪法,特意跟他去,在背地擺佈他,也未可知。”薛蟠道:“果然如此,倒也罷了。”衆夥計隨便喝了幾杯酒,吃了飯,大家散了。
\end{parag}


\begin{parag}
    話說寶玉回來,想着黛玉的孤苦,不免替他傷感起來。襲人見寶玉從外面進來坐在那發呆,便問:“就回來了?是不是同林姑娘一塊去了寶姑娘那兒?”寶玉說:“我會林姑娘同去的——送林姑娘的東西比送我們的多一兩倍呢。”說着話兒,便叫取了枕來,要在牀上歪着。襲人說:“璉二奶奶自從病了一場之後,我早就想着要到他那裏去看看,你同晴雯麝月待著,我去看看就來。”寶玉說:“你只管去罷。”言畢,襲人遂換了兩件新鮮衣服。囑咐了晴雯、麝月幾句,便出了怡紅院。
\end{parag}


\begin{parag}
    至沁芳橋上立住,往四下裏觀看那園中景緻。那時正是夏末秋初,園內蟬鬧蟲鳴;只是花也開敗了,芙蓉池中荷葉新殘相間,也將殘上來了。倒是近着池邊,都發了紅鋪鋪的咕嘟子,襯着碧綠的葉兒,着實可愛。於是一壁裏瞧着,一壁裏下了橋。走了不遠,迎見李紈房裏的丫頭素雲捧着個洋漆盒兒走來。襲人便問:“往那裏去送東西?”素雲說:“這是我們奶奶給三姑娘送去的菱角兒、雞頭米。”襲人說:“這個東西,是咱們園子裏河內採的,還是外頭買來的呢?”素雲說:“是我們那邊劉媽媽的女兒從鄉下帶來孝敬我們奶奶的。因三姑娘在我們那裏坐,奶奶叫人剝了讓他喫。他說:‘才吃了熱茶了,一會子再喫罷。’所以命我給三姑娘送過去。”言畢,各自散了。
\end{parag}


\begin{parag}
    襲人走著,沿堤看頑了一回。猛抬頭看見那邊葡萄架底下有人拿著撣子在那裏撣什麼呢,走到跟前,卻是老祝媽。那老婆子見了襲人,便笑嘻嘻的迎上來,說道:“姑娘怎麼今日得工夫出來逛逛?”襲人道:“可不是。我要到璉二奶奶家去。你在這裏做什麼呢?”那婆子道:“我在這裏趕蜜蜂兒。今年三伏裏雨水少,這果子樹上都有蟲子,把果子喫的疤瘌流星的掉了好些下來。姑娘還不知道呢,這馬蜂最可惡的,一嘟嚕上只咬破三兩個兒,那破的水滴到好的上頭,連這一嘟嚕都是要爛的。姑娘你瞧,咱們說話的空兒沒趕,就落上許多了。”襲人道:“你就是不住手的趕,也趕不了許多。你倒是告訴買辦,叫他多多做些小冷布口袋兒,一嘟嚕套上一個,又透風,又不遭塌。”婆子笑道:“倒是姑娘說的是。我今年才管上,那裏知道這個巧法兒呢。”
\end{parag}


\begin{parag}
    襲人說:“如今這園子裏這些果品有好些種,到是那樣先熟的快
    些?”老祝婆子說:“如今才入七月的門,果子都是才紅上來,要是好
    喫,想來還得月盡頭兒才熟透了呢。姑娘不信,我摘一個給姑娘嚐嚐。”
    襲人正色說道:“這那裏使得?不但沒熟喫不得,就是熟了,一則沒有
    供鮮,二則主子們尚然沒喫,我如何先喫得呢?”老婆子忙笑道:“姑娘說得有理。我因
    爲姑娘問我,我白這樣說。”襲人說:“我方纔告訴你要口袋的話,你就回一回二奶奶,叫管事的作去罷。”言畢,遂一直的出了園子的門,就到鳳姐這裏來了。
\end{parag}


\begin{parag}
    一到院裏,只聽鳳姐說道:“天理良心,我在這屋裏熬的越發成了賊了。”襲人聽見這話,知道有原故了,又不好回來,又不好進去,遂把腳步放重些,隔著窗子問道:“平姐姐在家裏呢麼?”平兒忙答應著迎出來。襲人便問:“二奶奶也在家裏呢麼,身上可大安了?”說著,已走進來。鳳姐裝著在牀上歪著呢,見襲人進來,也笑著站起來,說:“好些了,叫你惦著。怎麼這幾日不過我們這邊坐坐?”襲人道:“奶奶身上欠安,本該天天過來請安纔是。但只怕奶奶身上不爽快,倒要靜靜兒的歇歇兒,我們來了,倒吵的奶奶煩。”鳳姐笑道:“常聽見平兒說你背地裏還惦著我,常常問我。這就是你盡心了。”一面說著,叫平兒挪了張杌子放在牀旁邊,讓襲人坐下。豐兒端進茶來,襲人欠身道:“妹妹坐著罷。”一面說閒話兒。只見一個小丫頭子在外間屋裏悄悄的和平兒說:“旺兒來了。在二門上伺候著呢。”襲人知他們有事,又說了兩句話,便起身要走。鳳姐道:“閒來坐坐,說說話兒,我倒開心。”因命平兒:“送送你妹妹。”平兒答應著送出來。只見兩三個小丫頭子,都在那裏屏聲息氣齊齊的伺候著。襲人不知何事,便自去了。
\end{parag}


\begin{parag}
    卻說平兒送出襲人,進來回道:“旺兒纔來了,因襲人在這裏我叫他先到外頭等等兒,這會子還是立刻叫他呢,還是等著?請奶奶的示下。”鳳姐道:“叫他來。”平兒忙叫小丫頭去傳旺兒進來。這裏鳳姐又問平兒:“你到底是怎麼聽見說的?”平兒道:“就是頭裏那小丫頭子的話。他說他在二門裏頭聽見外頭兩個小廝說:‘這個新二奶奶比咱們舊二奶奶還俊呢,脾氣兒也好。’不知是旺兒還是誰,吆喝了兩個一頓,說:‘什麼新奶奶舊奶奶的,還不快悄悄兒的呢,叫裏頭知道了,把你的舌頭還割了呢。’”平兒正說著,只見一個小丫頭進來回說:“旺兒在外頭伺候著呢。”鳳姐聽了,冷笑了一聲說:“叫他進來。”那小丫頭出來說:“奶奶叫呢。”旺兒連忙答應著進來。旺兒請了安,在外間門口垂手侍立。鳳姐兒道:“你過來,我問你話。”旺兒才走到裏間門旁站著。鳳姐兒道:“你二爺在外頭弄了人,你知道不知道?”旺兒又打著千兒回道:“奴才天天在二門上聽差事,如何能知道二爺外頭的事呢。”鳳姐冷笑道:“你自然不知道。你要知道,你怎麼攔人呢。”旺兒見這話,知道剛纔的話已經走了風了,料著瞞不過,便又跪回道:“奴才實在不知。就是頭裏興兒和喜兒兩個人在那裏混說,奴才吆喝了他們兩句。內中深情底裏奴才不知道,不敢妄回。求奶奶問興兒,他是長跟二爺出門的。”鳳姐聽了,下死勁啐了一口,罵道:“你們這一起沒良心的混帳忘八崽子!都是一條藤兒,打量我不知道呢。先去給我把興兒那個忘八崽子叫了來,你也不許走。問明白了他,回來再問你。好,好,好,這纔是我使出來的好人呢!”那旺兒只得連聲答應幾個是,磕了個頭爬起來出去,去叫興兒。
\end{parag}


\begin{parag}
    卻說興兒正在帳房兒裏和小廝們玩呢,聽見說二奶奶叫,先唬了一跳,卻也想不到是這件事發作了,連忙跟著旺兒進來。旺兒先進去,回說:“興兒來了。”鳳姐兒厲聲道:“叫他!”那興兒聽見這個聲音兒,早已沒了主意了,只得乍著膽子進來。鳳姐兒一見,便說:“好小子啊!你和你爺辦的好事啊!你只實說罷!”興兒一聞此言,又看見鳳姐氣色,早唬軟了,不覺跪下,只是磕頭。鳳姐兒道:“論起這事來,我也聽見說不與你相干。但只你不早來回我知道,這就是你的不是了。你要實說了,我還饒你;再有一字虛言,你先摸摸你腔子上幾個腦袋瓜子!”興兒戰戰兢兢的朝上磕頭道:“奶奶問的是什麼事,奴才同爺辦壞了?”鳳姐聽了,一腔火都發作起來,喝命:“打嘴巴!”旺兒過來纔要打時,鳳姐兒罵道:“什麼糊塗忘八崽子!叫他自己打,用你打嗎!一會子你再各人打你那嘴巴子還不遲呢。”那興兒真個自己左右開弓打了自己十幾個嘴巴。鳳姐兒喝聲“站住”,問道:“你二爺外頭娶了什麼新奶奶舊奶奶的事,你大概不知道啊。”興兒見說出這件事來,越發著了慌,連忙把帽子抓下來在磚地上咕咚咕咚碰的頭山響,口裏說道:“只求奶奶超生,奴才再不敢撒一個字兒的謊。”鳳姐道:“快說!”興兒直蹶蹶的跪起來回道:“這事頭裏奴才也不知道。就是這一天,東府裏大老爺送了殯,俞祿往珍大爺廟裏去領銀子。二爺同著蓉哥兒到了東府裏,道兒上爺兒兩個說起珍大奶奶那邊的二位姨奶奶來。二爺誇他好,蓉哥兒哄著二爺,說把二姨奶奶說給二爺。”鳳姐聽到這裏,使勁啐道:“呸,沒臉的忘八蛋!他是你那一門子的姨奶奶!”興兒忙又磕頭說:“奴才該死!”往上啾著,不敢言語。鳳姐兒道:“完了嗎?怎麼不說了?”興兒方纔又回道:“奶奶恕奴才,奴才纔敢回。”鳳姐啐道:“放你媽的屁,這還什麼恕不恕了。你好生給我往下說,好多著呢。” 興兒又回道:“二爺聽見這個話就喜歡了。後來奴才也不知道怎麼就弄真了。”鳳姐微微冷笑道:“這個自然麼,你可那裏知道呢!你知道的只怕都煩了呢。是了,說底下的罷!”興兒回道:“後來就是蓉哥兒給二爺找了房子。”鳳姐忙問道:“如今房子在那裏?”興兒道:“就在府後頭。”鳳姐兒道:“哦。”回頭瞅著平兒道:“咱們都是死人哪。你聽聽!”平兒也不敢作聲。興兒又回道:“珍大爺那邊給了張家不知多少銀子,那張家就不問了。”鳳姐道:“這裏頭怎麼又扯拉上什麼張家李家咧呢?”興兒回道:“奶奶不知道,這二奶奶……”剛說到這裏,又自己打了個嘴巴,想了想,說道: “那珍大奶奶的妹子……”鳳姐兒接著道:“怎麼樣?快說呀。”興兒道:“那珍大奶奶的妹子原來從小兒有人家的,姓張,叫什麼張華,如今窮的待好討飯。珍大爺許了他銀子,他就退了親了。”鳳姐兒聽到這裏,點了點頭兒,回頭便望平兒說道:“你都聽見了?小忘八崽子,頭裏他還說他不知道呢!”興兒又回道: “後來二爺才叫人裱糊了房子,娶過來了。”鳳姐道:“打那裏娶過來的?”興兒回道:“就在他老孃家抬過來的。”鳳姐又問:“沒人送親麼?”興兒道:“就是蓉哥兒。還有幾個丫頭老婆子們,沒別人。”鳳姐道:“你大奶奶沒來嗎?”興兒道:“過了兩天,大奶奶纔拿了些東西來瞧的。”鳳姐兒回頭向平兒道:“怪道那兩天二爺稱讚大奶奶不離嘴呢。”掉過臉來又問興兒,“誰伏侍呢?自然是你了。”興兒趕著碰頭不言語。鳳姐又問:“前頭那些日子說給那府裏辦事,想來辦的就是這個了。”興兒回道:“也有辦事的時候,也有往新房子裏去的時候。”
\end{parag}


\begin{parag}
    鳳姐聽了這一篇言詞,只氣得癡呆了半天,面如金紙,兩隻吊稍丹鳳眼越發直豎起來了,渾身亂戰。半晌,連話也說不上來,只是發怔。猛低頭,見興兒還在地下跪着,便說道:“你這個猴兒崽子就該打死。這有什麼瞞著我的?你想著瞞了我,就在你那糊塗爺跟前討了好兒了,你新奶奶好疼你。”興兒道:“未能早回奶奶,是奴才該死!”便叩頭有聲。
\end{parag}


\begin{parag}
    鳳姐又問道:“誰和他住著呢。”興兒道:“先是和他娘和妹子在一處。就在十幾天前,他妹子自己抹了脖子。他娘得病,昨兒也死了。”鳳姐道:“這又爲什麼?”興兒隨將柳湘蓮的事說了一遍。鳳姐道:“這個人還算造化高,省了當那出名兒的忘八。”因又問道:“沒了別的事了麼?”興兒道:“別的事奴才不知道。奴才剛纔說的字字是實話,一字虛假,奶奶問出來只管打死奴才,奴才也無怨的。”鳳姐低了一回頭,便又指著興兒說道:“我不看你剛纔還有點怕懼兒,不敢撒謊,我把你的腿不給你砸折了。”說著喝聲“出去!”興兒瞌了個頭,才爬起來,退到外間門口,不敢就走。鳳姐道:“過來,我還有話呢。”興兒趕忙垂手敬聽。鳳姐道:“你忙什麼,新奶奶等著賞你什麼呢?”興兒也不敢抬頭。鳳姐道:“我什麼時候叫你,你什麼時候到。遲一步兒,你試試!出去罷。”興兒忙答應幾個“是”,退出門來。鳳姐又叫道:“興兒!”興兒趕忙站住。鳳姐道:“快出去告訴你二爺去,是不是啊?”興兒回道:“奴才不敢。”鳳姐道:“你出去提一個字兒,隄防你的皮!”興兒連忙答應著纔出去了。鳳姐又叫:“旺兒呢?”旺兒連忙答應著過來。鳳姐把眼直瞪瞪的瞅了兩三句話的工夫,才說道:“好旺兒,很好,去罷!外頭有人提一個字兒,全在你身上。”旺兒答應著也出去了。
\end{parag}


\begin{parag}
    且說鳳姐見興兒出去,回頭向平兒說:“方纔興兒說的話,你都聽見了沒有?天下那有這樣沒臉的男人!喫着碗裏,看着鍋裏,見一個,愛一個,真成了喂不飽的狗,實在是個棄舊迎新的壞貨。只可惜這五六品的頂帶給他!他別想着俗語說的‘家花那有野花香’的話,他要信了這個話,可就大錯了。多早晚在外面鬧一個沒臉、親戚朋友見不得的事出來,他才罷手呢!”平兒一旁勸道:“奶奶身子纔好了,也不可過於氣惱。看二爺自從鮑二的女人那一件事之後,倒收了心,好了呢,如今爲什麼又幹起這樣事來?這都是珍大爺他的不是。”鳳姐說:“珍大爺固有不是,也總因咱們那位下作不堪的爺他眼饞,人家才引誘他的。俗語說‘牛兒不喫水,也強按頭麼?’珍大爺幹這樣的事,珍大奶奶也該攔着不依纔是。珍大奶奶也不想一想,把一個妹子要許幾家子弟纔好,先許了姓張的,今又嫁了姓賈的;天下的男人都死絕了,都嫁到賈家來!難道賈家的衣食這樣好不成?那妹子本來也不是他親的,而且聽見說原是個混賬爛桃。難道珍大奶奶現做着命婦,家中有這樣一個打嘴現世的妹子,也不知道羞臊,躲避着些,反倒大面上揚鈴打鼓的,在這門裏丟醜,也不怕笑話?珍大爺也是做官的人,別的律例不知道也罷了,連個服中娶親,停妻再娶,使不得的規矩,他也不知道不成?你替他細想一想,他乾的這件事,是疼兄弟,還是害兄弟呢?”平兒說:“珍大爺只顧眼前,叫兄弟喜歡,也不管日後的輕重干係了。”鳳姐兒冷笑道:“這是什麼‘叫兄弟喜歡’,這是給他毒藥喫!若論親叔伯兄弟中,他年紀又最大,又居長,不知教導兄弟學好,反引誘兄弟學不長進,擔罪名兒,日後鬧出事來,他在一邊缸沿兒上站着看熱鬧,真真我要罵也罵不出口來。他在那邊府裏的醜事壞名聲,已經叫人聽不上了,必定也叫兄弟學他一樣,纔好顯不出他的醜來。這是什麼作哥哥的道理?倒不如撒泡尿浸死了,替大老爺死了也罷,活着作什麼。”
\end{parag}


\begin{parag}
    平兒看鳳姐越說越氣,便跪在地下,再三苦勸安慰一會子,鳳姐才略消了些氣惱。喝了口茶,喘息一回,便要了拐枕,歪在牀上,閉眼養神。平兒只得悄悄的退出去了。鳳姐將前事從頭至尾細細的盤算多時,才得了主意,也不吿訴平兒,卻作出個嘻笑自若、毫無惱恨妒嫉的樣子來。心下早已算定,只待賈璉起程去平安州,再作道理。要知端的,且聽下回分解。
\end{parag}

