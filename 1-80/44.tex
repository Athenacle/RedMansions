\chap{四十四}{變生不測鳳姐潑醋 喜出望外平兒理妝}


\begin{parag}
    \begin{note}蒙回前總:雲雨誰家院,飄來花自奇。鶯鶯燕燕闘芳菲,枝枝因風滴玉露,正春時。\end{note}
\end{parag}


\begin{parag}
    說衆人看演《荊釵記》,寶玉和姐妹一處坐著。林黛玉因看到《男祭》這一出上,便和寶釵說道:“這王十朋也不通的很,不管在那裏祭一祭罷了,必定跑到江邊子上來作什麼!俗語說‘睹物思人’,天下的水總歸一源,不拘那裏的水舀一碗看著哭去,也就盡情了。”寶釵不答。寶玉回頭要熱酒敬鳳姐兒。
\end{parag}


\begin{parag}
    原來賈母說今日不比往日,定要叫鳳姐痛樂一日。本來自己懶待坐席,只在裏間屋裏榻上歪著和薛姨媽看戲,隨心愛喫的揀幾樣放在小几上,隨意喫著說話兒;將自己兩桌席面賞那沒有席面的大小丫頭並那應差聽差的婦人等,命他們在窗外廊檐下也只管坐著隨意喫喝,不必拘禮。王夫人和邢夫人在地下高桌上坐著,外面几席是他姊妹們坐。賈母不時吩咐尤氏等:“讓鳳丫頭坐在上面,你們好生替我待東,難爲他一年到頭辛苦。”尤氏答應了,又笑回說道:“他坐不慣首席,坐在上頭橫不是豎不是的,酒也不肯喫。”賈母聽了,笑道:“你不會,等我親自讓他去。”鳳姐兒忙也進來笑說:“老祖宗別信他們的話,我吃了好幾鍾了。”賈母笑著,命尤氏:“快拉他出去,按在椅子上,你們都輪流敬他。他再不喫,我當真的就親自去了。”尤氏聽說,忙笑著又拉他出來坐下,命人拿了臺盞斟了酒,笑道:“一年到頭難爲你孝順老太太、太太和我。我今兒沒什麼疼你的,親自斟杯酒,乖乖兒的在我手裏喝一口。”鳳姐兒笑道:“你要安心孝敬我,跪下我就喝。”尤氏笑道:“說的你不知是誰!我告訴你說,好容易今兒這一遭,過了後兒,知道還得象今兒這樣不得了?趁著盡力灌喪兩鍾罷。”\begin{note}庚雙夾:閒閒一語伏下後文,令人可傷,所謂“盛筵難再”。\end{note}鳳姐兒見推不過,只得喝了兩鍾。接著衆姊妹也來,鳳姐也只得每人的喝一口。賴大媽媽見賈母尚這等高興,也少不得來湊趣兒,領著些嬤嬤們也來敬酒。鳳姐兒也難推脫,只得喝了兩口。鴛鴦等也來敬,鳳姐兒真不能了,忙央告道:“好姐姐們,饒了我罷,我明兒再喝罷。”鴛鴦笑道:“真個的,我們是沒臉的了?就是我們在太太跟前,太太還賞個臉兒呢。往常倒有些體面,今兒當著這些人,倒拿起主子的款兒來了。我原不該來。不喝,我們就走。” 說著真個回去了。鳳姐兒忙趕上拉住,笑道:“好姐姐,我喝就是了。”說著拿過酒來,滿滿的斟了一杯喝乾。鴛鴦方笑了散去,然後又入席。
\end{parag}


\begin{parag}
    鳳姐兒自覺酒沉了,心裏突突的似往上撞,要往家去歇歇,只見那耍百戲的上來,便和尤氏說:“預備賞錢,我要洗洗臉去。”尤氏點頭。鳳姐兒瞅人不防,便出了席,往房門後檐下走來。平兒留心,也忙跟了來,鳳姐兒便扶著他。才至穿廊下,只見他房裏的一個小丫頭正在那裏站著,見他兩個來了,回身就跑。鳳姐兒便疑心忙叫。那丫頭先只裝聽不見,無奈後面連平兒也叫,只得回來。鳳姐兒越發起了疑心,忙和平兒進了穿堂,叫那小丫頭子也進來,把槅扇關了,鳳姐兒坐在小院子的臺階上,命那丫頭子跪了,喝命平兒:“叫兩個二門上的小廝來,拿繩子鞭子,把那眼睛裏沒主子的小蹄子打爛了!”那小丫頭子已經唬的魂飛魄散,哭著只管碰頭求饒。鳳姐兒問道:“我又不是鬼,你見了我,不說規規矩矩站住,怎麼倒往前跑?”小丫頭子哭道:“我原沒看見奶奶來。我又記掛著房裏無人,所以跑了。”鳳姐兒道:“房裏既沒人,誰叫你來的?你便沒看見我,我和平兒在後頭扯著脖子叫了你十來聲,越叫越跑。離的又不遠,你聾了不成?你還和我強嘴!”說著便揚手一掌打在臉上,打的那小丫頭一栽;這邊臉上又一下,登時小丫頭子兩腮紫脹起來。平兒忙勸:“奶奶仔細手疼。”鳳姐便說:“你再打著問他跑什麼。他再不說,把嘴撕爛了他的!”那小丫頭子先還強嘴,後來聽見鳳姐兒要燒了紅烙鐵來烙嘴,方哭道:“二爺在家裏,打發我來這裏瞧著奶奶的,若見奶奶散了,先叫我送信兒去的。不承望奶奶這會子就來了。”鳳姐兒見話中有文章,“叫你瞧著我作什麼?難道怕我家去不成?必有別的原故,快告訴我,我從此以後疼你。你若不細說,立刻拿刀子來割你的肉。”說著,回頭向頭上拔下一根簪子來,向那丫頭嘴上亂戳,唬的那丫頭一行躲,一行哭求道:“我告訴奶奶,可別說我說的。”平兒一旁勸,一面催他,叫他快說。丫頭便說道:“二爺也是纔來房裏的,睡了一會醒了,打發人來瞧瞧奶奶,說才坐席,還得好一會纔來呢。二爺就開了箱子,拿了兩塊銀子,還有兩根簪子,兩匹緞子,叫我悄悄的送與鮑二的老婆去,叫他進來。他收了東西就往咱們屋裏來了。二爺叫我來瞧著奶奶,底下的事我就不知道了。”
\end{parag}


\begin{parag}
    鳳姐聽了,已氣的渾身發軟,忙立起來一徑來家。剛至院門,只見又有一個小丫頭在門前探頭兒,一見了鳳姐,也縮頭就跑。\begin{note}庚雙夾:如見其形。\end{note}鳳姐兒提著名字喝住。那丫頭本來伶俐,見躲不過了,越性跑了出來,笑道:“我正要告訴奶奶去呢,可巧奶奶來了。”鳳姐兒道:“告訴我什麼?”那小丫頭便說二爺在家這般如此如此,將方纔的話也說了一遍。鳳姐啐道:“你早作什麼了?這會子我看見你了,你來推乾淨兒!”說著也揚手一下打的那丫頭一個趔趄,便攝手攝腳的走至窗前,往裏聽時,只聽裏頭說笑。那婦人笑道:“多早晚你那閻王老婆死了就好了。”賈璉道:“他死了,再娶一個也是這樣,又怎麼樣呢?”那婦人道:“他死了,你倒是把平兒扶了正,只怕還好些。”賈璉道:“如今連平兒他也不叫我沾一沾了。平兒也是一肚子委曲不敢說。我命裏怎麼就該犯了‘夜叉星’。”
\end{parag}


\begin{parag}
    鳳姐聽了,氣的渾身亂戰,又聽他倆都贊平兒,便疑平兒素日背地裏自然也有憤怨語了,那酒越發湧了上來,也並不忖奪,回身把平兒先打了兩下,\begin{note}庚雙夾:奇極!先打平兒可是世人想得著的?\end{note}一腳踢開門進去,也不容分說,抓著鮑二家的撕打一頓。又怕賈璉走出去,便堵著門站著罵道:“好淫婦!你偷主子漢子,還要治死主子老婆!平兒過來!你們淫婦忘八一條藤兒,多嫌著我,外面兒你哄我!”說著又把平兒打幾下,打的平兒有冤無處訴,只氣得乾哭,罵道:“你們做這些沒臉的事,好好的又拉上我做什麼!”說著也把鮑二家的撕打起來。賈璉也因喫多了酒,進來高興,未曾作的機密,一見鳳姐來了,已沒了主意,又見平兒也鬧起來,把酒也氣上來了。鳳姐兒打鮑二家的,他已又氣又愧,只不好說的,今見平兒也打,便上來踢罵道:“好娼婦!你也動手打人!”平兒氣怯,忙住了手,哭道:“你們背地裏說話,爲什麼拉我呢?”鳳姐見平兒怕賈璉,越發氣了,又趕上來打著平兒,偏叫打鮑二家的。平兒急了,便跑出來找刀子要尋死。外面衆婆子丫頭忙攔住解勸。這裏鳳姐見平兒尋死去,便一頭撞在賈璉懷裏,叫道:“你們一條藤兒害我,被我聽見了,倒都唬起我來。你也勒死我!”賈璉氣的牆上拔出劍來,說道:“不用尋死,我也急了,一齊殺了,我償了命,大家乾淨。”正鬧的不開交,只見尤氏等一羣人來了,說:“這是怎麼說,纔好好的,就鬧起來。”賈璉見了人,越發“倚酒三分醉”,逞起威風來,\begin{note}庚雙夾:天下小人大都如是。\end{note}故意要殺鳳姐兒。鳳姐兒見人來了,便不似先前那般潑了,\begin{note}庚雙夾:天下奸雄妒婦惡婦大都如是,只是恨無阿鳳之才耳。\end{note}丟下衆人,便哭著往賈母那邊跑。
\end{parag}


\begin{parag}
    此時戲已散出,鳳姐跑到賈母跟前,爬在賈母懷裏,只說:“老祖宗救我!璉二爺要殺我呢!”\begin{note}庚雙夾:瞧他稱呼。\end{note}賈母、邢夫人、王夫人等忙問怎麼了。鳳姐兒哭道:“我才家去換衣裳,不防璉二爺在家和人說話,我只當是有客來了,唬得我不敢進去。在窗戶外頭聽了一聽,原來是和鮑二家的媳婦商議,說我利害,要拿毒藥給我吃了治死我,把平兒扶了正。我原氣了,又不敢和他吵,原打了平兒兩下,問他爲什麼要害我。他臊了,就要殺我。”賈母等聽了,都信以爲真,說:“這還了得!快拿了那下流種子來!”一語未完,只見賈璉拿著劍趕來,後面許多人跟著。賈璉明仗著賈母素昔疼他們,連母親嬸母也無礙,故逞強鬧了來。邢夫人王夫人見了,氣的忙攔住罵道:“這下流種子!你越發反了,老太太在這裏呢!”賈璉乜斜著眼,道:“都是老太太慣的他,他才這樣,連我也罵起來了!”邢夫人氣的奪下劍來,只管喝他“快出去!”那賈璉撒嬌撒癡,涎言涎語的還只亂說。賈母氣的說道:“我知道你也不把我們放在眼睛裏,叫人把他老子叫來!”賈璉聽見這話,方趔趄著腳兒出去了,賭氣也不往家去,便往外書房來。
\end{parag}


\begin{parag}
    這裏邢夫人王夫人也說鳳姐兒。賈母笑道:“什麼要緊的事!小孩子們年輕,饞嘴貓兒似的,那裏保得住不這麼著。從小兒世人都打這麼過的。都是我的不是,他多吃了兩口酒,又喫起醋來。”說的衆人都笑了。賈母又道:“你放心,等明兒我叫他來替你賠不是。你今兒別要過去臊著他。”因又罵:“平兒那蹄子,素日我倒看他好,怎麼暗地裏這麼壞。”尤氏等笑道:“平兒沒有不是,是鳳丫頭拿著人家出氣。兩口子不好對打,都拿著平兒煞性子。平兒委曲的什麼似的呢,老太太還罵人家。”賈母道:“原來這樣,我說那孩子倒不象那媚魘道的。既這麼著,可憐見的,白受他們的氣。”因叫琥珀來:“你出去告訴平兒,就說我的話:我知道他受了委曲,明兒我叫鳳姐兒替他賠不是。今兒是他主子的好日子,不許他胡鬧。”
\end{parag}


\begin{parag}
    原來平兒早被李紈拉入大觀園去了。\begin{note}庚雙夾:可知喫蟹一回非閒文也。\end{note}平兒哭得哽咽難抬。寶釵勸道:“你是個明白人,\begin{note}庚雙夾:必用寶釵評出方是身份。\end{note}素日鳳丫頭何等待你,今兒不過他多喫一口酒。他可不拿你出氣,難道倒拿別人出氣不成?別人又笑話他喫醉了。你只管這會子委曲,素日你的好處,豈不都是假的了?”正說著,只見琥珀走來,說了賈母的話。平兒自覺面上有了光輝,方纔漸漸的好了,也不往前頭來。寶釵等歇息了一回,方來看賈母鳳姐。
\end{parag}


\begin{parag}
    寶玉便讓平兒到怡紅院中來。襲人忙接著,笑道:“我先原要讓你的,只因大奶奶和姑娘們都讓你,我就不好讓的了。”平兒也陪笑說:“多謝。”因又說道: “好好兒的從那裏說起,無緣無故白受了一場氣。”襲人笑道:“二奶奶素日待你好,這不過是一時氣急了。”平兒道:“二奶奶倒沒說的,只是那淫婦治的我,他又偏拿我湊趣,況還有我們那糊塗爺倒打我。”說著便又委曲,禁不住落淚。寶玉忙勸道:“好姐姐,別傷心,我替他兩個賠不是罷。”平兒笑道:“與你什麼相干?”寶玉笑道:“我們弟兄姊妹都一樣。他們得罪了人,我替他賠個不是也是應該的。”又道:“可惜這新衣裳也沾了,這裏有你花妹妹的衣裳,何不換了下來,拿些燒酒噴了熨一熨。把頭也另梳一梳,洗洗臉。”一面說,一面便吩咐了小丫頭子們舀洗臉水,燒熨斗來。平兒素習只聞人說寶玉專能和女孩兒們接交;寶玉素日因平兒是賈璉的愛妾,又是鳳姐兒的心腹,故不肯和他廝近,因不能盡心,也常爲恨事。平兒今見他這般,心中也暗暗的敁敠:果然話不虛傳,色色想的周到。又見襲人特特的開了箱子,拿出兩件不大穿的衣裳來與他換,便趕忙的脫下自己的衣服,忙去洗了臉。寶玉一旁笑勸道:“姐姐還該擦上些脂粉,不然倒象是和鳳姐姐賭氣了似的。況且又是他的好日子,而且老太太又打發了人來安慰你。”平兒聽了有理,便去找粉,只不見粉。寶玉忙走至妝臺前,將一個宣窯瓷盒揭開,裏面盛著一排十根玉簪花棒,拈了一根遞與平兒。又笑向他道:“這不是鉛粉,這是紫茉莉花種,研碎了兌上香料制的。”平兒倒在掌上看時,果見輕白紅香,四樣俱美,攤在面上也容易勻淨,且能潤澤肌膚,不似別的粉青重澀滯。然後看見胭脂也不是成張的,卻是一個小小的白玉盒子,裏面盛著一盒,如如玫瑰膏子一樣。寶玉笑道: “那市賣的胭脂都不乾淨,顏色也薄。這是上好的胭脂擰出汁子來,淘澄淨了渣滓,配了花露蒸疊成的。只用細簪子挑一點兒抹在手心裏,用一點水化開抹在脣上;手心裏就夠打頰腮了。”平兒依言妝飾,果見鮮豔異常,且又甜香滿頰。寶玉又將盆內的一枝並蒂秋蕙用竹剪刀擷了下來,與他簪在鬢上。忽見李紈打發丫頭來喚他,方忙忙的去了。\begin{note}庚雙夾:忽使平兒在絳芸軒中梳妝,非世人想不到,寶玉亦想不到者也。作者費盡心機了。寫寶玉最善閨閣中事,諸如脂粉等類,不寫成別緻文章,則寶玉不成寶玉矣。然要寫又不便特爲此費一番筆墨,故思及借人發端。然借人又無人,若襲人輩則逐日皆如此,又何必揀一日細寫?似覺無味。若寶釵等又系姊妹,更不便來細搜襲人之妝奩,況也是自幼知道的了。因左想右想須得一個又甚親、又甚疏、又可唐突、又不可唐突、又和襲人等極親、又和襲人等不大常處、又得襲人輩之美、又不得襲人輩之修飾一人來方可發端。故思及平兒一人方如此,故放手細寫絳芸閨中之什物也。\end{note}
\end{parag}


\begin{parag}
    寶玉因自來從未在平兒前盡過心,──且平兒又是個極聰明極清俊的上等女孩兒,比不得那起俗蠢拙物──深爲恨怨。今日是金釧兒的生日,故一日不樂。\begin{note}庚雙夾:原來爲此!寶玉之私祭,玉釧之潛哀俱針對矣。然於此刻補明,又一法也。真十變萬化之文,萬法具備,毫無脫漏,真好書也。\end{note}不想落後鬧出這件事來,竟得在平兒前稍盡片心,亦今生意中不想之樂也。因歪在牀上,心內怡然自得。忽又思及賈璉惟知以淫樂悅己,並不知作養脂粉。又思平兒並無父母兄弟姊妹,獨自一人,供應賈璉夫婦二人。賈璉之俗,鳳姐之威,他竟能周全妥貼,今兒還遭荼毒,想來此人薄命,比黛玉猶甚。想到此間,便又傷感起來,不覺灑然淚下。因見襲人等不在房內,盡力落了幾點痛淚。復起身,又見方纔的衣裳上噴的酒已半乾,便拿熨斗熨了疊好;見他的手帕子忘去,上面猶有淚漬,又拿至臉盆中洗了晾上。又喜又悲,悶了一回,也往稻香村來,說一回閒話,掌燈後方散。
\end{parag}


\begin{parag}
    平兒就在李紈處歇了一夜,鳳姐兒只跟著賈母。賈璉晚間歸房,冷清清的,又不好去叫,只得胡亂睡了一夜。次日醒了,想昨日之事,大沒意思,後悔不來。邢夫人記掛著昨日賈璉醉了,忙一早過來,叫了賈璉過賈母這邊來。賈璉只得忍愧前來,在賈母面前跪下。賈母問他:“怎麼了?”賈璉忙陪笑說:“昨兒原是吃了酒,驚了老太太的駕了,今兒來領罪。”賈母啐道:“下流東西,灌了黃湯,不說安分守己的挺屍去,倒打起老婆來了!鳳丫頭成日家說嘴,霸王似的一個人,昨兒唬得可憐。要不是我,你要傷了他的命,這會子怎麼樣?”賈璉一肚子的委屈,不敢分辯,只認不是。賈母又道:“那鳳丫頭和平兒還不是個美人胎子?你還不足!成日家偷雞摸狗,髒的臭的,都拉了你屋裏去。爲這起淫婦打老婆,又打屋裏的人,你還虧是大家子的公子出身,活打了嘴了。若你眼睛裏有我,你起來,我饒了你,乖乖的替你媳婦賠個不是,拉了他家去,我就喜歡了。要不然,你只管出去,我也不敢受你的跪。”賈璉聽如此說,又見鳳姐兒站在那邊,也不盛妝,哭的眼睛腫著,也不施脂粉,黃黃臉兒,\begin{note}庚雙夾:大妙大奇之文,此一句便伏下病根了,草草看去便可惜了作者行文苦心。\end{note}比往常更覺可憐可愛。想著:“不如賠了不是,彼此也好了,又討老太太的喜歡了。”想畢,便笑道:“老太太的話,我不敢不依,只是越發縱了他了。”賈母笑道:“胡說!我知道他最有禮的,再不會衝撞人。他日後得罪了你,我自然也作主,叫你降伏就是了。”
\end{parag}


\begin{parag}
    賈璉聽說,爬起來,便與鳳姐兒作了一個揖,笑道:“原來是我的不是,二奶奶饒過我罷。”滿屋裏的人都笑了。賈母笑道:“鳳丫頭,不許惱了,再惱我就惱了。”說著,又命人去叫了平兒來,命鳳姐兒和賈璉兩個安慰平兒。賈璉見了平兒,越發顧不得了,所謂“妻不如妾,妾不如偷”,聽賈母一說,便趕上來說道: “姑娘昨日受了屈了,都是我的不是。奶奶得罪了你,也是因我而起。我賠了不是不算外,還替你奶奶賠個不是。”說著,也作了一個揖,引的賈母笑了,鳳姐兒也笑了。賈母又命鳳姐兒來安慰他。平兒忙走上來給鳳姐兒磕頭,說:“奶奶的千秋,我惹了奶奶生氣,是我該死。”鳳姐兒正自愧悔昨日酒喫多了,不念素日之情,浮躁起來,爲聽了旁人的話,無故給平兒沒臉。今反見他如此,又是慚愧,又是心酸,忙一把拉起來,落下淚來。平兒道:“我伏侍了奶奶這麼幾年,也沒彈我一指甲。就是昨兒打我,我也不怨奶奶,都是那淫婦治的,怨不得奶奶生氣。”說著,也滴下淚來了。\begin{note}庚雙夾:婦人女子之情畢肖,但世之大英雄羽翼偶摧尚按劍生悲,況阿鳳與平兒哉?所謂此書真是哭成的。\end{note}賈母便命人將他三人送回房去,“有一個再提此事,即刻來回我,我不管是誰,拿柺棍子給他一頓。”
\end{parag}


\begin{parag}
    三人從新給賈母、邢王二位夫人磕了頭。老嬤嬤答應了,送他三人回去。至房中,鳳姐兒見無人,方說道:“我怎麼象個閻王,又象夜叉?那淫婦咒我死,你也幫著咒我。千日不好,也有一日好。可憐我熬的連個淫婦也不如了,我還有什麼臉來過這日子?”說著又哭了。\begin{note}庚雙夾:轄治丈夫此是首計,懦夫來看此句。\end{note}賈璉道:“你還不足?你細想想,昨兒誰的不是多?\begin{note}庚雙夾:妙!不敢自說沒不是,只論多少,懦夫來看。\end{note}今兒當著人還是我跪了一跪,又賠不是,你也爭足了光了。這會子還叨叨,難道還叫我替你跪下才罷?太要足了強也不是好事。”說的鳳姐兒無言可對,平兒嗤的一聲又笑了。賈璉也笑道:“又好了!真真我也沒法了。”
\end{parag}


\begin{parag}
    正說著,只見一個媳婦來回說:“鮑二媳婦吊死了。”賈璉鳳姐兒都吃了一驚。鳳姐忙收了怯色,反喝道:“死了罷了,有什麼大驚小怪的!”\begin{note}庚雙夾:寫阿鳳如此。\end{note}一時,只見林之孝家的進來悄回鳳姐道:“鮑二媳婦吊死了,\begin{note}庚雙夾:倒也有氣性,只是又是情累一個,可憐!\end{note}他孃家的親戚要告呢。”鳳姐兒笑道:\begin{note}庚雙夾:偏於此處寫阿鳳笑。壞哉阿鳳!\end{note}“這倒好了,我正想要打官司呢!”林之孝家的道:“我才和衆人勸了他們,又威嚇了一陣,又許了他幾個錢,也就依了。”鳳姐兒道:“我沒一個錢!有錢也不給,只管叫他告去。也不許勸他,也不用震嚇他,只管讓他告去。告不成倒問他個‘以屍訛詐’!”\begin{note}庚雙夾:寫阿鳳如此。\end{note}林之孝家的正在爲難,見賈璉和他使眼色兒,心下明白,便出來等著。賈璉道:“我出去瞧瞧,看是怎麼樣。”鳳姐兒道:“不許給他錢。”賈璉一徑出來,和林之孝來商議,著人去作好作歹,許了二百兩發送才罷。賈璉生恐有變,又命人去和王子騰說,將番役仵作人等叫了幾名來,幫著辦喪事。那些人見了如此,縱要復辨亦不敢辨,只得忍氣吞聲罷了。賈璉又命林之孝將那二百銀子入在流年帳上,分別添補開銷過去。\begin{note}庚雙夾:大弊小弊,無一不到。\end{note}又梯己給鮑二些銀兩,安慰他說:“另日再挑個好媳婦給你。”鮑二又有體面,又有銀子,有何不依,便仍然奉承賈璉,\begin{note}庚雙夾:爲天下夫妻一哭。\end{note}不在話下。
\end{parag}


\begin{parag}
    裏面鳳姐心中雖不安,面上只管佯不理論,因房中無人,便拉平兒笑道:“我昨兒灌喪了酒了,你別憤怨,打了那裏,讓我瞧瞧。”平兒道:“也沒打重。”只聽得說,奶奶姑娘都進來了。要知端的,下回分解。
\end{parag}


\begin{parag}
    \begin{note}蒙回末總:富貴少年多好色,哪如寶玉會風流。閻王夜叉誰曾說,死到臨頭身不由。\end{note}
\end{parag}
