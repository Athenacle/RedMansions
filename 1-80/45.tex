\chap{四十五}{金蘭契互剖金蘭語 風雨夕悶制風雨詞}


\begin{parag}
    \begin{note}蒙回前總:富貴榮華春暖,夢破皇糧愁晚,金玉坐樓基,也是戲場妝點。莫緩,莫緩,遺卻靈光無限。\end{note}
\end{parag}


\begin{parag}
    話說鳳姐兒正撫卹平兒,忽見衆姊妹進來,忙讓坐了,平兒斟上茶來。鳳姐兒笑道:“今兒來的這麼齊,倒象下帖子請了來的。”探春笑道:“我們有兩件事:一件是我的,一件是四妹妹的,還夾著老太太的話。”鳳姐兒笑道:“有什麼事,這麼要緊?”探春笑道:“我們起了個詩社,頭一社就不齊全,衆人臉軟,所以就亂了。我想必得你去作個監社御史,鐵面無私纔好。再四妹妹爲畫園子,用的東西這般那般不全,回了老太太,老太太說:‘只怕後頭樓底下還有當年剩下的,找一找,若有呢拿出來,若沒有,叫人買去。’”鳳姐笑道:“我又不會作什麼溼的乾的,要我喫東西去不成?”探春道:“你雖不會作,也不要你作。你只監察著我們裏頭有偷安怠惰的,該怎麼樣罰他就是了。”鳳姐兒笑道:“你們別哄我,我猜著了,那裏是請我作監社御史!分明是叫我作個進錢的銅商。你們弄什麼社,必是要輪流作東道的。你們的月錢不夠花了,想出這個法子來拗了我去,好和我要錢。可是這個主意?”一席話說的衆人都笑起來了。李紈笑道:“真真你是個水晶心肝玻璃人。”鳳姐兒笑道:“虧你是個大嫂子呢!把姑娘們原交給你帶著唸書學規矩針線的,他們不好,你要勸。這會子他們起詩社,能用幾個錢,你就不管了?老太太、太太罷了,原是老封君。你一個月十兩銀子的月錢,比我們多兩倍銀子。老太太、太太還說你寡婦失業的,可憐,不夠用,又有個小子,足的又添了十兩,和老太太、太太平等。又給你園子地,各人取租子。年終分年例,你又是上上分兒。你娘兒們,主子奴才共總沒十個人,喫的穿的仍舊是官中的。一年通共算起來,也有四五百銀子。這會子你就每年拿出一二百兩銀子來陪他們頑頑,能幾年的限?他們各人出了閣,難道還要你賠不成?這會子你怕花錢,調唆他們來鬧我,我樂得去喫一個河涸海乾,我還通不知道呢!”
\end{parag}


\begin{parag}
    李紈笑道:“你們聽聽,我說了一句,他就瘋了,說了兩車的無賴泥腿市俗專會打細算盤分斤撥兩的話出來。\begin{note}庚雙夾:心直口拙之人急了恨不得將萬句話來併成一句說死那人,畢肖!\end{note}這東西虧他託生在詩書大宦名門之家做小姐,出了嫁又是這樣,他還是這麼著;若是生在貧寒小戶人家,作個小子,還不知怎麼下作貧嘴惡舌的呢!天下人都被你算計了去!昨兒還打平兒呢,虧你伸的出手來!那黃湯難道灌喪了狗肚子裏去了?氣的我只要給平兒打報不平兒。忖奪了半日,好容易‘狗長尾巴尖兒’的好日子,又怕老太太心裏不受用,因此沒來,究竟氣還未平。你今兒又招我來了。給平兒拾鞋也不要,你們兩個只該換一個過子纔是。”說的衆人都笑了。鳳姐兒忙笑道:“竟不是爲詩爲畫來找我,這臉子竟是爲平兒來報仇的。竟不承望平兒有你這一位仗腰子的人。早知道,便有鬼拉著我的手打他,我也不打了。平姑娘,過來!我當著大奶奶姑娘們替你賠個不是,擔待我酒後無德罷。”說著,衆人又都笑起來了。李紈笑問平兒道:“如何?我說必定要給你爭爭氣才罷。”平兒笑道:“雖如此,奶奶們取笑,我禁不起。”李紈道:“什麼禁不起,有我呢。快拿了鑰匙叫你主子開了樓房找東西去。”
\end{parag}


\begin{parag}
    鳳姐兒笑道:“好嫂子,你且同他們回園子裏去。纔要把這米帳合算一算,那邊大太太又打發人來叫,又不知有什麼話說,須得過去走一趟。還有年下你們添補的衣服,還沒打點給他們做去。”李紈笑道:“這些事情我都不管,你只把我的事完了我好歇著去,省得這些姑娘小姐鬧我。”鳳姐忙笑道:“好嫂子,賞我一點空兒。你是最疼我的,怎麼今兒爲平兒就不疼我了?往常你還勸我說,事情雖多,也該保養身子,撿點著偷空兒歇歇,你今兒反到逼我的命了。況且誤了別人的年下衣裳無礙,他姊妹們的若誤了,卻是你的責任,老太太豈不怪你不管閒事,這一句現成的話也不說?我寧可自己落不是,豈敢帶累你呢。”李紈笑道:“你們聽聽,說的好不好?把他會說話的!我且問你:這詩社你到底管不管?”鳳姐兒笑道:“這是什麼話,我不入社花幾個錢,不成了大觀園的反叛了,還想在這裏喫飯不成?明兒一早就到任,下馬拜了印,先放下五十兩銀子給你們慢慢作會社東道。過後幾天,我又不作詩作文,只不過是個俗人罷了。‘監察’也罷,不‘監察’也罷,有了錢了,你們還攆出我來!”說的衆人又都笑起來。鳳姐兒道:“過會子我開了樓房,凡有這些東西都叫人搬出來你們看,若使得,留著使,若少什麼,照你們單子,我叫人替你們買去就是了。畫絹我就裁出來。那圖樣沒有在太太跟前,還在那邊珍大爺那裏呢。說給你們,別碰釘子去。我打發人取了來,一併叫人連絹交給相公們礬去。如何?”李紈點首笑道:“這難爲你,果然這樣還罷了。既如此,咱們家去罷,等著他不送了去再來鬧他。”說著,便帶了他姊妹就走。鳳姐兒道:“這些事再沒兩個人,都是寶玉生出來的。”李紈聽了,忙回身笑道:“正是爲寶玉來,反忘了他。頭一社是他誤了。我們臉軟,你說該怎麼罰他?”鳳姐想了一想,說道: “沒有別的法子,只叫他把你們各人屋子裏的地罰他掃一遍纔好。”衆人都笑道:“這話不差。”
\end{parag}


\begin{parag}
    說著纔要回去,只見一個小丫頭扶了賴嬤嬤進來。鳳姐兒等忙站起來,笑道:“大娘坐。”又都向他道喜。賴嬤嬤向炕沿上坐了,笑道:“我也喜,主子們也喜。若不是主子們的恩典,我們這喜從何來?昨兒奶奶又打發彩哥兒賞東西,我孫子在門上朝上磕了頭了。”李紈笑道:“多早晚上任去?”賴嬤嬤嘆道:“我那裏管他們,由他們去罷!前兒在家裏給我磕頭,我沒好話,我說:‘哥哥兒,你別說你是官兒了,橫行霸道的!你今年活了三十歲,雖然是人家的奴才,一落孃胎胞,主子恩典,放你出來,上託著主子的洪福,下託著你老子娘,也是公子哥兒似的讀書認字,也是丫頭、老婆、奶子捧鳳凰似的,長了這麼大。你那裏知道那“奴才” 兩字是怎麼寫的!只知道享福,也不知道你爺爺和你老子受的那苦惱,熬了兩三輩子,好容易掙出你這麼個東西來。從小兒三災八難,花的銀子也照樣打出你這麼個銀人兒來了。到二十歲上,又蒙主子的恩典,許你捐個前程在身上。你看那正根正苗的忍飢挨餓的要多少?你一個奴才秧子,仔細折了福!如今樂了十年,不知怎麼弄神弄鬼的,求了主子,又選了出來。州縣官兒雖小,事情卻大,爲那一州的州官,就是那一方的父母。你不安分守己,盡忠報國,孝敬主子,只怕天也不容你。 ’”李紈鳳姐兒都笑道:“你也多慮。我們看他也就好了。先那幾年還進來了兩次,這有好幾年沒來了,年下生日,只見他的名字就罷了。前兒給老太太、太太磕頭來,在老太太那院裏,見他又穿著新官的服色,倒發的威武了,比先時也胖了。他這一得了官,正該你樂呢,反倒愁起這些來!他不好,還有他父親呢,你只受用你的就完了。閒了坐個轎子進來,和老太太鬥一日牌,說一天話兒,誰好意思的委屈了你。家去一般也是樓房廈廳,誰不敬你,自然也是老封君似的了。”
\end{parag}


\begin{parag}
    平兒斟上茶來,賴嬤嬤忙站起來接了,笑道:“姑娘不管叫那個孩子倒來罷了,又折受我。”說著,一面喫茶,一面又道:“奶奶不知道。這些小孩子們全要管的嚴。饒這麼嚴,他們還偷空兒鬧個亂子來叫大人操心。知道的說小孩子們淘氣;不知道的,人家就說仗著財勢欺人,連主子名聲也不好。恨的我沒法兒,常把他老子叫來罵一頓,纔好些。”因又指寶玉道:“不怕你嫌我,如今老爺不過這麼管你一管,老太太護在頭裏。當日老爺小時挨你爺爺的打,誰沒看見的。老爺小時,何曾象你這麼天不怕地不怕的了。還有那大老爺,雖然淘氣,也沒象你這扎窩子的樣兒,也是天天打。還有東府裏你珍哥兒的爺爺,那纔是火上澆油的性子,說聲惱了,什麼兒子,竟是審賊!如今我眼裏看著,耳朵裏聽著,那珍大爺管兒子倒也象當日老祖宗的規矩,只是管的到三不著兩的。他自己也不管一管自己,這些兄弟侄兒怎麼怨的不怕他?你心裏明白,喜歡我說,不明白,嘴裏不好意思,心裏不知怎麼罵我呢!”
\end{parag}


\begin{parag}
    正說著,只見賴大家的來了,接著周瑞家的張材家的都進來回事情。鳳姐兒笑道:“媳婦來接婆婆來了。”賴大家的笑道:“不是接他老人家,倒是打聽打聽奶奶姑娘們賞臉不賞臉?”賴嬤嬤聽了,笑道:“可是我糊塗了,正經說的話且不說,且說陳穀子爛芝麻的混搗熟。因爲我們小子選了出來,衆親友要給他賀喜,少不得家裏擺個酒。我想,擺一日酒,請這個也不是,請那個也不是。又想了一想,託主子洪福,想不到的這樣榮耀,就傾了家,我也是願意的。因此吩咐他老子連擺三日酒:頭一日,在我們破花園子裏擺几席酒,一臺戲,請老太太、太太們、奶奶姑娘們去散一日悶;外頭大廳上一臺戲,擺几席酒,請老爺們、爺們去增增光;第二日再請親友;第三日再把我們兩府裏的伴兒請一請。熱鬧三天,也是託著主子的洪福一場,光輝光輝。”李紈鳳姐兒都笑道:“多早晚的日子?我們必去,只怕老太太高興要去也定不得。”賴大家的忙道:“擇了十四的日子,只看我們奶奶的老臉罷了。”鳳姐笑道:“別人我不知道,我是一定去的。先說下,我是沒有賀禮的,也不知道放賞,喫完了一走,可別笑話。”賴大家的笑道:“奶奶說那裏話?奶奶要賞,賞我們三二萬銀子就有了。” 嬤嬤笑道:“我纔去請老太太,老太太也說去,可算我這臉還好。”說畢又叮嚀了一回,方起身要走,因看見周瑞家的,便想起一事來,因說道:“可是還有一句話問奶奶,這周嫂子的兒子犯了什麼不是,攆了他不用?”鳳姐兒聽了,笑道:“正是我要告訴你媳婦,事情多也忘了。賴嫂子回去說給你老頭子,兩府裏不許收留他小子,叫他各人去罷。”
\end{parag}


\begin{parag}
    賴大家的只得答應著。周瑞家的忙跪下央求。嬤嬤忙道:“什麼事說給我評評。”鳳姐兒道:“前日我生日,裏頭還沒喫酒,他小子先醉了。老孃那邊送了禮來,他不說在外頭張羅,他倒坐著罵人,禮也不送進來。兩個女人進來了,他才帶著小幺們往裏抬。小幺們倒好,他拿的一盒子倒失了手,撒了一院子饅頭。人去了,打發彩明去說他,他倒罵了彩明一頓。這樣無法無天的忘八羔子,不攆了作什麼!”賴嬤嬤笑道:“我當什麼事情,原來爲這個。奶奶聽我說:他有不是,打他罵他,使他改過,攆了去斷乎使不得。他又比不得是咱們家的家生子兒,他現是太太的陪房。奶奶只顧攆了他,太太臉上不好看。依我說,奶奶教導他幾板子,以戒下次,仍舊留著纔是。不看他娘,也看太太。”鳳姐兒聽說,便向賴大家的說道:“既這樣,打他四十棍,以後不許他喫酒。”賴大家的答應了。周瑞家的磕頭起來,又要與賴嬤嬤磕頭,賴大家的拉著方罷。然後他三人去了,李紈等也就回園中來。
\end{parag}


\begin{parag}
    至晚,果然鳳姐命人找了許多舊收的畫具出來,送至園中。寶釵等選了一回,各色東西可用的只有一半,將那一半又開了單子,與鳳姐兒去照樣置買,不必細說。
\end{parag}


\begin{parag}
    一日,外面礬了絹,起了稿子進來。寶玉每日便在惜春這裏幫忙。\begin{note}庚雙夾:自忙不暇,又加上一“幫”字,可笑可笑,所謂《春秋》筆法。\end{note}探春、李紈、迎春、寶釵等也多往那裏閒坐,一則觀畫,二則便於會面。寶釵因見天氣涼爽,夜復漸長,\begin{note}庚雙夾:“復”字妙,補出寶釵每年夜長之事,皆《春秋》字法也。\end{note}遂至母親房中商議打點些針線來。日間至賈母處王夫人處省候兩次,不免又承色陪坐半時,園中姊妹處也要度時閒話一回,故日間不大得閒,每夜燈下女工必至三更方寢。\begin{note}庚雙夾:代下收夕,寫針線下“商議”二字,直將寡母訓女多少溫存活現在紙上。不寫阿呆兄已見阿呆兄終日飽醉優遊,怒則吼、喜則躍,家務一概無聞之形景畢露矣。《春秋》筆法。\end{note}黛玉每歲至春分秋分之後,必犯嗽疾;今秋又遇賈母高興,多遊玩了兩次,未免過勞了神,近日又復嗽起來,覺得比往常又重,所以總不出門,只在自己房中將養。有時悶了,又盼個姊妹來說些閒話排遣;及至寶釵等來望候他,說不得三五句話又厭煩了。衆人都體諒他病中,且素日形體嬌弱,禁不得一些委屈,所以他接待不周,禮數粗忽,也都不苛責。
\end{parag}


\begin{parag}
    這日寶釵來望他,因說起這病症來。寶釵道:“這裏走的幾個太醫雖都還好,只是你喫他們的藥總不見效,不如再請一個高明的人來瞧一瞧,治好了豈不好?每年間鬧一春一夏,又不老又不小,成什麼?不是個常法。”黛玉道:“不中用。我知道我這樣病是不能好的了。且別說病,只論好的日子我是怎麼形景,就可知了。”寶釵點頭道:“可正是這話。古人說:‘食谷者生。’你素日喫的竟不能添養精神氣血,也不是好事。”黛玉嘆道:“‘死生有命,富貴在天’,也不是人力可強的。今年比往年反覺又重了些似的。”說話之間,已咳嗽了兩三次。寶釵道:“昨兒我看你那藥方上,人蔘肉桂覺得太多了。雖說益氣補神,也不宜太熱。依我說,先以平肝健胃爲要,肝火一平,不能克土,胃氣無病,飲食就可以養人了。每日早起拿上等燕窩一兩,冰糖五錢,用銀銚子熬出粥來,若喫慣了,比藥還強,最是滋陰補氣的。”
\end{parag}


\begin{parag}
    黛玉嘆道:“你素日待人,固然是極好的,然我最是個多心的人,只當你心裏藏奸。從前日你說看雜書不好,又勸我那些好話,竟大感激你。往日竟是我錯了,實在誤到如今。細細算來,我母親去世的早,又無姊妹兄弟,我長了今年十五歲,\begin{note}庚雙夾:黛玉才十五歲,記清。\end{note}竟沒一個人象你前日的話教導我。怨不得雲丫頭說你好,我往日見他贊你,我還不受用,昨兒我親自經過,才知道了。比如若是你說了那個,我再不輕放過你的;你竟不介意,反勸我那些話,可知我竟自誤了。若不是從前日看出來,今日這話,再不對你說。你方纔說叫我喫燕窩粥的話,雖然燕窩易得,但只我因身上不好了,每年犯這個病,也沒什麼要緊的去處。請大夫,熬藥,人蔘肉桂,已經鬧了個天翻地覆,這會子我又興出新文來熬什麼燕窩粥,老太太、太太、鳳姐姐這三個人便沒話說,那些底下的婆子丫頭們,未免不嫌我太多事了。你看這裏這些人,因見老太太多疼了寶玉和鳳丫頭兩個,他們尚虎視眈眈,背地裏言三語四的,何況於我?況我又不是他們這裏正經主子,原是無依無靠投奔了來的,他們已經多嫌著我了。如今我還不知進退,何苦叫他們咒我?”寶釵道:“這樣說,我也是和你一樣。”黛玉道:“你如何比我?你又有母親,又有哥哥,這裏又有買賣地土,家裏又仍舊有房有地。你不過是親戚的情分,白住了這裏,一應大小事情,又不沾他們一文半個,要走就走了。我是一無所有,喫穿用度,一草一紙,皆是和他們家的姑娘一樣,那起小人豈有不多嫌的。”寶釵笑道:“將來也不過多費得一副嫁妝罷了,如今也愁不到這裏。”\begin{note}庚雙夾:寶釵此一戲直抵通部黛玉之戲寶釵矣,又懇切、又真情、又平和、又雅緻、又不穿鑿、又不牽強,黛玉因識得寶釵後方吐真情,寶釵亦識得黛玉後方肯戲也,此是大關節大章法,非細心看不出。二人此時好看之極,真是兒女小窗中喁喁也。\end{note}黛玉聽了,不覺紅了臉,笑道:“人家纔拿你當個正經人,把心裏的煩難告訴你聽,你反拿我取笑兒。”寶釵笑道:“雖是取笑兒,卻也是真話。你放心,我在這裏一日,我與你消遣一日。你有什麼委屈煩難,只管告訴我,我能解的,自然替你解一日。我雖有個哥哥,你也是知道的,只有個母親比你略強些。咱們也算同病相憐。你也是個明白人,何必作‘司馬牛之嘆’?\begin{note}庚雙夾:通部衆人必從寶釵之評方定,然寶釵亦必從顰兒之評始可,何妙之至!\end{note}你才說的也是,多一事不如省一事。我明日家去和媽媽說了,只怕我們家裏還有,與你送幾兩,每日叫丫頭們就熬了,又便宜,又不驚師動衆的。”黛玉忙笑道:“東西事小,難得你多情如此。”寶釵道:“這有什麼放在口裏的!只愁我人人跟前失於應候罷了。只怕你煩了,我且去了。”黛玉道:“晚上再來和我說句話兒。”寶釵答應著便去了,不在話下。
\end{parag}


\begin{parag}
    這裏黛玉喝了兩口稀粥,仍歪在牀上,不想日未落時天就變了,淅淅瀝瀝下起雨來。秋霖脈脈,陰晴不定,那天漸漸的黃昏,且陰的沉黑,兼著那雨滴竹梢,更覺淒涼。知寶釵不能來,便在燈下隨便拿了一本書,卻是《樂府雜稿》,有《秋閨怨》《別離怨》等詞。黛玉不覺心有所感,亦不禁發於章句,遂成《代別離》一首,擬《春江花月夜》之格,乃名其詞曰《秋窗風雨夕》。其詞曰:
\end{parag}


\begin{poem}
    \begin{pl}秋花慘淡秋草黃,耿耿秋燈秋夜長。\end{pl}

    \begin{pl}已覺秋窗秋不盡,那堪風雨助淒涼!\end{pl}

    \begin{pl}助秋風雨來何速!驚破秋窗秋夢綠。\end{pl}

    \begin{pl}抱得秋情不忍眠,自向秋屏移淚燭。\end{pl}

    \begin{pl}淚燭搖搖蓺短檠,牽愁照恨動離情。\end{pl}

    \begin{pl}誰家秋院無風入?何處秋窗無雨聲?\end{pl}

    \begin{pl}羅衾不奈秋風力,殘漏聲催秋雨急。\end{pl}

    \begin{pl}連宵脈脈復颼颼,燈前似伴離人泣。\end{pl}

    \begin{pl}寒煙小院轉蕭條,疏竹虛窗時滴瀝。\end{pl}

    \begin{pl}不知風雨幾時休,已教淚灑紗窗溼。\end{pl}

\end{poem}


\begin{parag}
    吟罷擱筆,方要安寢,丫鬟報說:“寶二爺來了。”一語未完,只見寶玉頭上帶著大箬笠,身上披著蓑衣。黛玉不覺笑了:“那裏來的漁翁!”寶玉忙問:“今兒好些?\begin{note}庚雙夾:一句。\end{note}吃了藥沒有?\begin{note}庚雙夾:兩句。\end{note}今兒一日吃了多少飯?”\begin{note}庚雙夾:三句。\end{note}一面說,一面摘了笠,脫了蓑衣,忙一手舉起燈來,一手遮住燈光,向黛玉臉上照了一照,覷著眼細瞧了一瞧,笑道:“今兒氣色好了些。”
\end{parag}


\begin{parag}
    黛玉看脫了蓑衣,裏面只穿半舊紅綾短襖,系著綠汗巾子,膝下露出油綠綢撒花褲子,底下是掐金滿繡的綿紗襪子,靸著蝴蝶落花鞋。黛玉問道:“上頭怕雨,底下這鞋襪子是不怕雨的?也倒乾淨。”寶玉笑道:“我這一套是全的。有一雙棠木屐,才穿了來,脫在廊檐上了。”黛玉又看那蓑衣斗笠不是尋常市賣的,十分細緻輕巧,因說道:“是什麼草編的?怪道穿上不象那刺蝟似的。”寶玉道:“這三樣都是北靜王送的。他閒了下雨時在家裏也是這樣。你喜歡這個,我也弄一套來送你。別的都罷了,惟有這斗笠有趣,竟是活的。上頭的這頂兒是活的,冬天下雪,帶上帽子,就把竹信子抽了,去下頂子來,只剩了這圈子。下雪時男女都戴得,我送你一頂,冬天下雪戴。”黛玉笑道:“我不要他。戴上那個,成個畫兒上畫的和戲上扮的漁婆了。”及說了出來,方想起話未忖奪,與方纔說寶玉的話相連,後悔不及,羞的臉飛紅,便伏在桌上嗽個不住。\begin{note}庚雙夾:妙極之文。使黛玉自己直說出夫妻來,卻又云“畫的”“扮的”,本是閒談,卻是暗隱不吉之兆。所謂 “畫兒中愛寵”是也,誰曰不然?\end{note}
\end{parag}


\begin{parag}
    寶玉卻不留心,\begin{note}庚雙夾:必雲“不留心”方好,方是寶玉,若著心則又有何文字?且直是一時時獵色一賊矣。\end{note}因見案上有詩,遂拿起來看了一遍,又不禁叫好。黛玉聽了,忙起來奪在手內,向燈上燒了。寶玉笑道:“我已背熟了,燒也無礙。”黛玉道:“我也好了許多,謝你一天來幾次瞧我,下雨還來。這會子夜深了,我也要歇著,你且請回去,明兒再來。”寶玉聽說,回手向懷中掏出一個核桃大小的一個金錶來,瞧了一瞧,那針已指到戌末亥初之間,忙又揣了,說道: “原該歇了,又擾的你勞了半日神。”說著,披蓑戴笠出去了,又翻身進來問道:“你想什麼喫,告訴我,我明兒一早回老太太,豈不比老婆子們說的明白?”\begin{note}庚雙夾:直與後部寶釵之文遙遙針對。想彼姊妹房中婆子丫鬟皆有,隨便皆可遣使,今寶玉獨雲“婆子”而不雲“丫鬟”者,心內已度定丫鬟之爲人,一言一事無論大小,是方無錯謬者也,一何可笑。\end{note}黛玉笑道:“等我夜裏想著了,明兒早起告訴你。你聽雨越發緊了,快去罷。可有人跟著沒有?”有兩個婆子答應:“有人,外面拿著傘點著燈籠呢。”黛玉笑道:“這個天點燈籠?”寶玉道:“不相干,是明瓦的,不怕雨。”黛玉聽了,回手向書架上把個玻璃繡球燈拿了下來,命點一支小蠟來,遞與寶玉,道:“這個又比那個亮,正是雨裏點的。”寶玉道:“我也有這麼一個,怕他們失腳滑倒了打破了,所以沒點來。”黛玉道:“跌了燈值錢,跌了人值錢?你又穿不慣木屐子。那燈籠命他們前頭點著。這個又輕巧又亮,原是雨裏自己拿著的,你自己手裏拿著這個,豈不好?明兒再送來。就失了手也有限的,怎麼忽然又變出這‘剖腹藏珠’的脾氣來!”寶玉聽說,連忙接了過來,前頭兩個婆子打著傘提著明瓦燈,後頭還有兩個小丫鬟打著傘。寶玉便將這個燈遞與一個小丫頭捧著,寶玉扶著他的肩,一徑去了。
\end{parag}


\begin{parag}
    就有蘅蕪苑的一個婆子,也打著傘提著燈,送了一大包上等燕窩來,還有一包子潔粉梅片雪花洋糖。說:“這比買的強。姑娘說了:姑娘先喫著,完了再送來。”黛玉道:“回去說‘費心’。”命他外頭坐了喫茶。婆子笑道:“不喫茶了,我還有事呢。”黛玉笑道:“我也知道你們忙。如今天又涼,夜又長,越發該會個夜局,痛賭兩場了。”婆子笑道:“不瞞姑娘說,今年我大沾光兒了。橫豎每夜各處有幾個上夜的人,誤了更也不好,不如會個夜局,又坐了更,又解悶兒。今兒又是我的頭家,如今園門關了,就該上場了。”\begin{note}庚雙夾:幾句閒話將潭潭大宅夜間所有之事描寫一盡。雖諾大一園,且值秋冬之夜,豈不寥落哉?今用老嫗數語,更寫得每夜深人定之後,各處燈光燦爛、人煙簇集,柳陌之上、花巷之中,或提燈同酒,或寒月烹茶者,竟仍有絡繹人跡不絕,不但不見寥落,且覺更勝於日間繁華矣。此是大宅妙景,不可不寫出,又伏下後文,且又襯出後文之冷落。此閒話中寫出,正是不寫之寫也。脂硯齋評。\end{note}黛玉聽說笑道:“難爲你。誤了你發財,冒雨送來。”命人給他幾百錢,打些酒喫,避避雨氣。那婆子笑道:“又破費姑娘賞酒喫。”說著,磕了一個頭,外面接了錢,打傘去了。
\end{parag}


\begin{parag}
    紫鵑收起燕窩,然後移燈下簾,伏侍黛玉睡下。黛玉自在枕上感念寶釵,一時又羨他有母兄;一面又想寶玉雖素習和睦,終有嫌疑。又聽見窗外竹梢焦葉之上,雨聲淅瀝,清寒透幕,不覺又滴下淚來。直到四更將闌,方漸漸的睡了。暫且無話。要知端的——
\end{parag}


\begin{parag}
    \begin{note}蒙回末總:請看賴大,則知貴家奴婢身份,而本主毫不以爲過分,習慣自然故是有之。見者當自度是否可也。\end{note}
\end{parag}

