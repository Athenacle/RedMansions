\chap{六十三}{壽怡紅羣芳開夜宴 死金丹獨豔理親喪}


\begin{parag}
    \begin{note}蒙回前總:此書寫世人之富貴子弟易流邪鄙,其作長上者,有不能稽查之處,如寶玉之夜宴,始見之,文雅韻極,細思之,何事生端不基於此?更能寫賈芸之惡賴無恥亦世家之必有者,讀者當以“三人行必有我師”之說爲念,方能領會作者之用意也。戒之!\end{note}
\end{parag}


\begin{parag}
    話說寶玉回至房中洗手,因與襲人商議:“晚間喫酒,大家取樂,不可拘泥。如今喫什麼,好早說給他們備辦去。”襲人笑道:“你放心,我和晴雯、麝月、秋紋四個人,每人五錢銀子,共是二兩。芳官、碧痕、小燕、四兒四個人,每人三錢銀子,他們有假的不算,共是三兩二錢銀子,早已交給了柳嫂子,預備四十碟果子。我和平兒說了,已經抬了一罈好紹興酒藏在那邊了。我們八個人單替你過生日。”寶玉聽了,喜的忙說:“他們是那裏的錢,不該叫他們出纔是。”晴雯道: “他們沒錢,難道我們是有錢的!這原是各人的心。那怕他偷的呢,只管領他們的情就是。”寶玉聽了,笑說:“你說的是。”襲人笑道:“你一天不挨他兩句硬話村你,你再過不去。”晴雯笑道:“你如今也學壞了,專會架橋撥火兒。”說著,大家都笑了。寶玉說:“關院門罷。”襲人笑道:“怪不得人說你是‘無事忙’,這會子關了門,人倒疑惑,越性再等一等。”寶玉點頭,因說:“我出去走走,四兒舀水去,小燕一個跟我來罷。”說著,走至外邊,因見無人,便問五兒之事。小燕道:“我才告訴了柳嫂子,他倒喜歡的很。只是五兒那夜受了委屈煩惱,回家去又氣病了,那裏來得。只等好了罷。”寶玉聽了,不免後悔長嘆,因又問:“這事襲人知道不知道?”小燕道:“我沒告訴,不知芳官可說了不曾。”寶玉道:“我卻沒告訴過他,也罷,等我告訴他就是了。”說畢,復走進來,故意洗手。
\end{parag}


\begin{parag}
    已是掌燈時分,聽得院門前有一羣人進來。大家隔窗悄視,果見林之孝家的和幾個管事的女人走來,前頭一人提著大燈籠。晴雯悄笑道:“他們查上夜的人來了。這一出去,咱們好關門了。”只見怡紅院凡上夜的人都迎了出去,林之孝家的看了不少。林之孝家的吩咐:“別耍錢喫酒,放倒頭睡到大天亮。我聽見是不依的。”衆人都笑說:“那裏有那樣大膽子的人。”林之孝家的又問:“寶二爺睡下了沒有?”衆人都回不知道。襲人忙推寶玉。寶玉靸了鞋,便迎出來,笑道:“我還沒睡呢。媽媽進來歇歇。”又叫:“襲人倒茶來。”林之孝家的忙進來,笑說:“還沒睡?如今天長夜短了,該早些睡,明兒起的方早。不然到了明日起遲了,人笑話說不是個讀書上學的公子了,倒像那起挑腳漢了。”說畢,又笑。寶玉忙笑道:“媽媽說的是。我每日都睡的早,媽媽每日進來可都是我不知道的,已經睡了。今兒因吃了面怕停住食,所以多頑一會子。”林之孝家的又向襲人等笑說:“該沏些個普洱茶喫。”襲人晴雯二人忙笑說:“沏了一盄子女兒茶,已經喫過兩碗了。大娘也嘗一碗,都是現成的。”說著,晴雯便倒了一碗來。林之孝家的又笑道:“這些時我聽見二爺嘴裏都換了字眼,趕著這幾位大姑娘們竟叫起名字來。雖然在這屋裏,到底是老太太、太太的人,還該嘴裏尊重些纔是。若一時半刻偶然叫一聲使得,若只管叫起來,怕以後兄弟侄兒照樣,便惹人笑話,說這家子的人眼裏沒有長輩。” 寶玉笑道:“媽媽說的是。我原不過是一時半刻的。”襲人晴雯都笑說:“這可別委屈了他。直到如今,他可姐姐沒離了口。不過頑的時候叫一聲半聲名字,若當著人卻是和先一樣。”林之孝家的笑道:“這纔好呢,這纔是讀書知禮的。越自己謙越尊重,別說是三五代的陳人,現從老太太、太太屋裏撥過來的,便是老太太、太太屋裏撥過來的,便是老太太、太太屋裏的貓兒狗兒,輕易也傷他不的。這纔是受過調教的公子行事。”說畢,吃了茶,便說:“請安歇罷,我們走了。”寶玉還說:“再歇歇。”那林之孝家的已帶了衆人,又查別處去了。
\end{parag}


\begin{parag}
    這裏晴雯等忙命關了門,進來笑說:“這位奶奶那裏吃了一杯來了,嘮三叨四的,又排場了我們一頓去了。”麝月笑道:“他也不是好意的,少不得也要常提著些兒。也隄防著怕走了大褶兒的意思。”說著,一面擺上酒果。襲人道:“不用圍桌,咱們把那張花梨圓炕桌子放在炕上坐,又寬綽,又便宜。”說著,大家果然抬來。麝月和四兒那邊去搬果子,用兩個大茶盤做四五次方搬運了來。兩個老婆子蹲在外面火盆上篩酒。寶玉說:“天熱,咱們都脫了大衣裳纔好。”衆人笑道:“你要脫你脫,我們還要輪流安席呢。”寶玉笑道:“這一安就安到五更天了。知道我最怕這些俗套子,在外人跟前不得已的,這會子還慪我就不好了。”衆人聽了,都說:“依你。”於是先不上坐,且忙著卸妝寬衣。\begin{note}庚雙夾:凡喫酒從未先如此者,此獨怡紅風俗。故王夫人云“他行事總是與世人兩樣的”,知子莫過母也。\end{note}
\end{parag}


\begin{parag}
    一時將正裝卸去,頭上只隨便挽著□兒,身上皆是長裙短襖。寶玉只穿著大紅棉紗小襖子,下面綠綾彈墨袷褲,散著褲腳,倚著一個各色玫瑰芍藥花瓣裝的玉色夾紗新枕頭,和芳官兩個先划拳。當時芳官滿口嚷熱,\begin{note}庚雙夾:餘此時亦太熱了,恨不得一冷。既冷時思此熱,果然一夢矣。\end{note}只穿著一件玉色紅青酡絨三色緞子斗的水田小夾襖,束著一條柳綠汗巾,底下是水紅撒花夾褲,也散著褲腿。頭上眉額編著一圈小辮,總歸至頂心,結一根鵝卵粗細的總辮,拖在腦後。右耳眼內只塞著米粒大小的一個小玉塞子,左耳上單帶著一個白果大小的硬紅鑲金大墜子,越顯的面如滿月猶白,眼如秋水還清。引的衆人笑說:“他兩個倒像是雙生的弟兄兩個。”襲人等一一的斟了酒來,說:“且等等再划拳,雖不安席,每人在手裏喫我們一口罷了。”於是襲人爲先,端在脣上吃了一口,餘依次下去,一一喫過,大家方團圓坐定。小燕四兒因炕沿坐不下,便端了兩張椅子,近炕放下。那四十個碟子,皆是一色白粉定窯的,不過只有小茶碟大,裏面不過是山南海北,中原外國,或幹或鮮,或水或陸,天下所有的酒饌果菜。寶玉因說:“咱們也該行個令纔好。”襲人道:“斯文些的纔好,別大呼小叫,惹人聽見。二則我們不識字,可不要那些文的。”麝月笑道:“拿骰子咱們搶紅罷。”寶玉道:“沒趣,不好。咱們佔花名兒好。”晴雯笑道:“正是早已想弄這個頑意兒。” 襲人道:“這個頑意雖好,人少了沒趣。”小燕笑道:“依我說,咱們竟悄悄的把寶姑娘林姑娘請了來頑一回子,到二更天再睡不遲。”襲人道:“又開門喝戶的鬧,倘或遇見巡夜的問呢?”寶玉道:“怕什麼,咱們三姑娘也喫酒,再請他一聲纔好。還有琴姑娘。”衆人都道:“琴姑娘罷了,他在大奶奶屋裏,叨登的大發了。”寶玉道:“怕什麼,你們就快請去。”小燕四兒都得不了一聲,二人忙命開了門,分頭去請。
\end{parag}


\begin{parag}
    晴雯、麝月、襲人三人又說:“他兩個去請,只怕寶林兩個不肯來,須得我們請去,死活拉他來。”於是襲人晴雯忙又命老婆子打個燈籠,二人又去。果然寶釵說夜深了,黛玉說身上不好,他二人再三央求說:“好歹給我們一點體面,略坐坐再來。”探春聽了卻也歡喜。因想:“不請李紈,倘或被他知道了倒不好。”便命翠墨同了小燕也再三的請了李紈和寶琴二人,會齊,先後都到了怡紅院中。襲人又死活拉了香菱來。炕上又並了一張桌子,方坐開了。
\end{parag}


\begin{parag}
    寶玉忙說:“林妹妹怕冷,過這邊靠板壁坐。”又拿個靠背墊著些。襲人等都端了椅子在炕沿下一陪。黛玉卻離桌遠遠的靠著靠背,因笑向寶釵、李紈、探春等道:“你們日日說人夜聚飲博,今兒我們自己也如此,以後怎麼說人。”李紈笑道:“這有何妨。一年之中不過生日節間如此,並無夜夜如此,這倒也不怕。”說著,晴雯拿了一個竹雕的籤筒來,裏面裝著象牙花名籤子,搖了一搖,放在當中。又取過骰子來,盛在盒內,搖了一搖,揭開一看,裏面是五點,數至寶釵。寶釵便笑道:“我先抓,不知抓出個什麼來。”說著,將筒搖了一搖,伸手掣出一根,大家一看,只見簽上畫著一支牡丹,題著“豔冠羣芳”四字,下面又有鐫的小字一句唐詩,道是:
\end{parag}


\begin{poem}
    \begin{pl}任是無情也動人。\end{pl}
\end{poem}


\begin{parag}
    又注著:“在席共賀一杯,此爲羣芳之冠,隨意命人,不拘詩詞雅謔,道一則以侑酒。”衆人看了,都笑說:“巧的很,你也原配牡丹花。”說著,大家共賀了一杯。寶釵喫過,便笑說:“芳官唱一支我們聽罷。”芳官道:“既這樣,大家喫門杯好聽的。”於是大家喫酒。芳官便唱:“壽筵開處風光好。”衆人都道:“快打回去。這會子很不用你來上壽,揀你極好的唱來。”芳官只得細細的唱了一支《賞花時》:
\end{parag}


\begin{qute2sp}
    翠鳳毛翎扎帚叉,閒踏天門掃落花。您看那風起玉塵沙。猛可的那一層雲下,抵多少門外即天涯。您再休要劍斬黃龍一線兒差,再休向東老貧窮賣酒家。您與俺眼向雲霞。洞賓呵,您得了人可便早些兒回話;若遲呵,錯教人留恨碧桃花。
\end{qute2sp}


\begin{parag}
    才罷。寶玉卻只管拿著那籤,口內顛來倒去唸“任是無情也動人”,聽了這曲子,眼看著芳官不語。湘雲忙一手奪了,擲與寶釵。寶釵又擲了一個十六點,數到探春。探春笑道:“我還不知得個什麼呢。”伸手掣了一根出來,自己一瞧,便擲在地下,紅了臉,笑道:“這東西不好,不該行這令。這原是外頭男人們行的令,許多混話在上頭。”衆人不解,襲人等忙拾了起來,衆人看上面是一枝杏花,那紅字寫著“瑤池仙品”四字,詩云:
\end{parag}


\begin{poem}
    \begin{pl}日邊紅杏倚雲栽。\end{pl}

\end{poem}


\begin{parag}
    注云:“得此籤者,必得貴婿,大家恭賀一杯,共同飲一杯。”衆人笑道:“我說是什麼呢。這籤原是閨閣中取戲的,除了這兩三根有這話的,並無雜話,這有何妨。我們家已有了個王妃,難道你也是王妃不成。大喜,大喜。”說著,大家來敬。探春那裏肯飲,卻被史湘雲、香菱、李紈等三四個人強死強活灌了下去。探春只命蠲了這個,再行別的,衆人斷不肯依。湘雲拿著他的手強擲了個十九點出來,便該李氏掣。李氏搖了一搖,掣出一根來一看,笑道:“好極。你們瞧瞧,這勞什子竟有些意思。”衆人瞧那簽上,畫著一枝老梅,是寫著“霜曉寒姿”四字,那一面舊詩是:
\end{parag}


\begin{poem}
    \begin{pl}竹籬茅舍自甘心。\end{pl}

\end{poem}


\begin{parag}
    注云:“自飲一杯,下家擲骰。”李紈笑道:“真有趣,你們擲去罷。我只自喫一杯,不問你們的廢與興。”說著,便喫酒,將骰過與黛玉。黛玉一擲,是個十八點,便該湘雲掣。湘雲笑著,揎拳擄袖的伸手掣了一根出來。大家看時,一面畫著一枝海棠,題著“香夢沉酣”四字,那面詩道是:
\end{parag}


\begin{poem}
    \begin{pl}只恐夜深花睡去。\end{pl}
\end{poem}


\begin{parag}
    黛玉笑道:“‘夜深’兩個字,改‘石涼’兩個字。”衆人便知他趣白日間湘雲醉臥的事,都笑了。湘雲笑指那自行船與黛玉看,又說:“快坐上那船家去罷,別多話了。”衆人都笑了。因看注云:“既雲‘香夢沉酣’,掣此籤者不便飲酒,只令上下二家各飲一杯。”湘雲拍手笑道:“阿彌陀佛,真真好籤!”恰好黛玉是上家,寶玉是下家。二人斟了兩杯只得要飲。寶玉先飲了半杯,瞅人不見,遞與芳官,端起來便一揚脖。黛玉只管和人說話,將酒全折在漱盂內了。湘雲便綽起骰子來一擲個九點,數去該麝月。麝月便掣了一根出來。大家看時,這面上一枝荼縻花,題著“韶華勝極”四字,那邊寫著一句舊詩,道是:
\end{parag}


\begin{poem}
    \begin{pl}開到荼縻花事了。\end{pl}
\end{poem}


\begin{parag}
    注云:“在席各飲三杯送春。”麝月問怎麼講,寶玉愁眉忙將籤藏了說:“咱們且喝酒。”說著,大家吃了三口,以充三杯之數。麝月一擲個十九點,該香菱。香菱便掣了一根並蒂花,題著“聯春繞瑞”,那面寫著一句詩,道是:
\end{parag}


\begin{poem}
    \begin{pl}連理枝頭花正開。\end{pl}
\end{poem}


\begin{parag}
    注云:“共賀掣者三杯,大家陪飲一杯。”香菱便又擲了個六點,該黛玉掣。黛玉默默的想道:“不知還有什麼好的被我掣著方好。”一面伸手取了一根,只見上面畫著一枝芙蓉,題著“風露清愁”四字,那面一句舊詩,道是:
\end{parag}


\begin{poem}
    \begin{pl}莫怨東風當自嗟。\end{pl}
\end{poem}


\begin{parag}
    注云:“自飲一杯,牡丹陪飲一杯。”衆人笑說:“這個好極。除了他,別人不配作芙蓉。”黛玉也自笑了。於是飲了酒,便擲了個二十點,該著襲人。襲人便伸手取了一支出來,卻是一枝桃花,題著“武陵別景”四字,那一面舊詩寫著道是:
\end{parag}


\begin{poem}
    \begin{pl}桃紅又是一年春。\end{pl}
\end{poem}


\begin{parag}
    注云:“杏花陪一盞,坐中同庚者陪一盞,同辰者陪一盞,同姓者陪一盞。”衆人笑道:“這一回熱鬧有趣。”大家算來,香菱、晴雯、寶釵三人皆與他同庚,黛玉與他同辰,只無同姓者。芳官忙道:“我也姓花,我也陪他一鍾。”於是大家斟了酒,黛玉因向探春笑道:“命中該著招貴婿的,你是杏花,快喝了,我們好喝。”探春笑道:“這是個什麼,大嫂子順手給他一下子。”李紈笑道:“人家不得貴婿反捱打,我也不忍的。”說的衆人都笑了。
\end{parag}


\begin{parag}
    襲人才要擲,只聽有人叫門。老婆子忙出去問時,原來是薛姨媽打發人來了接黛玉的。衆人因問幾更了,人回:“二更以後了,鍾打過十一下了。”寶玉猶不信,要過表來瞧了一瞧,已是子初初刻十分了。黛玉便起身說:“我可撐不住了,回去還要吃藥呢。”衆人說:“也都該散了。”襲人寶玉等還要留著衆人。李紈寶釵等都說:“夜太深了不像,這已是破格了。”襲人道:“既如此,每位再喫一杯再走。”說著,晴雯等已都斟滿了酒,每人吃了,都命點燈。襲人等直送過沁芳亭河那邊方回來。
\end{parag}


\begin{parag}
    關了門,大家復又行起令來。襲人等又用大鐘斟了幾鍾,用盤攢了各樣果菜與地下的老嬤嬤們喫。彼此有了三分酒,便猜拳贏唱小曲兒。那天已四更時分,老嬤嬤們一面明喫,一面暗偷,酒罈已罄,衆人聽了納罕,方收拾盥漱睡覺。芳官喫的兩腮胭脂一般,眉梢眼角越添了許多丰韻,身子圖不得,便睡在襲人身上,“好姐姐,心跳的很。”襲人笑道:“誰許你盡力灌起來。”小燕四兒也圖不得,早睡了。晴雯還只管叫。寶玉道:“不用叫了,咱們且胡亂歇一歇罷。”自己便枕了那紅香枕,身子一歪,便也睡著了。襲人見芳官醉的很,恐鬧他唾酒,只得輕輕起來,就將芳官扶在寶玉之側,由他睡了。自己卻在對面榻上倒下。
\end{parag}


\begin{parag}
    大家黑甜一覺,不知所之。及至天明,襲人睜眼一看,只見天色晶明,忙說:“可遲了。”向對面牀上瞧了一瞧,只見芳官頭枕著炕沿上,睡猶未醒,連忙起來叫他。寶玉已翻身醒了,笑道:“可遲了!”因又推芳官起身。那芳官坐起來,猶發怔揉眼睛。襲人笑道:“不害羞,你喫醉了,怎麼也不揀地方兒亂挺下了。”芳官聽了,瞧了一瞧,方知道和寶玉同榻,忙笑的下地來,說:“我怎麼喫的不知道了。”寶玉笑道:“我竟也不知道了。若知道,給你臉上抹些黑墨。”說著,丫頭進來伺候梳洗。寶玉笑道:“昨兒有擾,今兒晚上我還席。”襲主笑道:“罷罷罷,今兒可別鬧了,再鬧就有人說話了。”寶玉道:“怕什麼,不過才兩次罷了。咱們也算是會喫酒了,那一罈子酒,怎麼就喫光了。正是有趣,偏又沒了。”襲人笑道:“原要這樣纔有趣。必至興盡了,反無後味了。昨兒都好上來了,晴雯連臊也忘了,我記得他還唱了一個。”四兒笑道:“姐姐忘了,連姐姐還唱了一個呢。在席的誰沒唱過!”衆人聽了,俱紅了臉,用兩手握著笑個不住。
\end{parag}


\begin{parag}
    忽見平兒笑嘻嘻的走來,說親自來請昨日在席的人:“今兒我還東,短一個也使不得。”衆人忙讓坐喫茶。晴雯笑道:“可惜昨夜沒他。”平兒忙問:“你們夜裏做什麼來?”襲人便說:“告訴不得你。昨兒夜裏熱鬧非常,連往日老太太、太太帶著衆人頑也不及昨兒這一頑。一罈酒我們都鼓搗光了,一個個喫的把臊都丟了,三不知的又都唱起來。四更多天才橫三豎四的打了一個盹兒。”平兒笑道:“好,白和我要了酒來,也不請我,還說著給我聽,氣我。”晴雯道:“今兒他還席,必來請你的,等著罷。”平兒笑問道:“他是誰,誰是他?”晴雯聽了,趕著笑打,說道:“偏你這耳朵尖,聽得真。”平兒笑道:“這會子有事不和你說,我幹事去了。一回再打發人來請,一個不到,我是打上門來的。”寶玉等忙留,他已經去了。
\end{parag}


\begin{parag}
    這裏寶玉梳洗了正喫茶,忽然一眼看見硯臺底下壓著一張紙,因說道:“你們這隨便混壓東西也不好。”襲人晴雯等忙問:“又怎麼了,誰又有了不是了?”寶玉指道:“硯臺下是什麼?一定又是那位的樣子忘記了收的。”晴雯忙啓硯拿了出來,卻是一張字帖兒,遞與寶玉看時,原來是一張粉箋子,上面寫著“檻外人妙玉恭肅遙叩芳辰”。\begin{note}庚雙夾:帖文亦蹈俗套之□。\end{note}寶玉看畢,直跳了起來,忙問:“這是誰接了來的?也不告訴。”襲人晴雯等見了這般,不知當是那個要緊的人來的帖子,忙一齊問:“昨兒誰接下了一個帖子?”四兒忙飛跑進來,笑說:“昨兒妙玉並沒親來,只打發個媽媽送來。我就擱在那裏,誰知一頓酒就忘了。”衆人聽了,道:“我當誰的,這樣大驚小怪。這也不值的。”寶玉忙命:“快拿紙來。”當時拿了紙,研了墨,看他下著“檻外人”三字,自己竟不知回帖上回個什麼字樣才相敵。只管提筆出神,半天仍沒主意。因又想:“若問寶釵去,他必又批評怪誕,不如問黛玉去。”
\end{parag}


\begin{parag}
    想罷,袖了帖兒,徑來尋黛玉。剛過了沁芳亭,忽見岫煙顫顫巍巍的迎面走來。寶玉忙問:“姐姐那裏去?”岫煙笑道:“我找妙玉說話。”寶玉聽了詫異,說道:“他爲人孤癖,不合時宜,萬人不入他目。原來他推重姐姐,竟知姐姐不是我們一流的俗人。”岫煙笑道:“他也未必真心重我,但我和他做過十年的鄰居,只一牆之隔。他在蟠香寺修煉,我家原寒素,賃的是他廟裏的房子,住了十年,無事到他廟裏去作伴。我所認的字都是承他所授。我和他又是貧賤之交,又有半師之分。因我們投親去了,聞得他因不合時宜,權勢不容,竟投到這裏來。如今又天緣湊合,我們得遇,舊情竟未易。承他青目,更勝當日。”寶玉聽了,恍如聽了焦雷一般,喜的笑道:“怪道姐姐舉止言談,超然如野鶴閒雲,原來有本而來。正因他的一件事我爲難,要請教別人去。如今遇見姐姐,真是天緣巧合,求姐姐指教。” 說著,便將拜帖取與岫煙看。岫煙笑道:“他這脾氣竟不能改,竟是生成這等放誕詭僻了。從來沒見拜帖上下別號的,這可是俗語說的‘僧不僧,俗不俗,女不女,男不男’,成個什麼道理。”寶玉聽說,忙笑道:“姐姐不知道,他原不在這些人中算,他原是世人意外之人。因取我是個些微有知識的,方給我這帖子。我因不知回什麼字樣纔好,竟沒了主意,正要去問林妹妹,可巧遇見了姐姐。”岫煙聽了寶玉這話,且只顧用眼上下細細打量了半日,方笑道:“怪道俗語說的‘聞名不如見面’,又怪不得妙玉竟下這帖子給你,又怪不得上年竟給你那些梅花。既連他這樣,少不得我告訴你原故。他常說:‘古人中自漢晉五代唐宋以來皆無好詩,只有兩句好,說道:”縱有千年鐵門檻,終須一個土饅頭。“所以他自稱‘檻外之人’。又常贊文是莊子的好,故又或稱爲‘畸人’。他若帖子上是自稱‘畸人’的,你就還他個‘世人’。畸人者,他自稱是畸零之人;你謙自己乃世中擾擾之人,他便喜了。如今他自稱‘檻外之人’,是自謂蹈於鐵檻之外了;故你如今只下‘檻內人 ’,便合了他的心了。”寶玉聽了,如醍醐灌頂,噯喲了一聲,方笑道:“怪道我們家廟說是 ‘鐵檻寺’呢,原來有這一說。姐姐就請,讓我去寫回帖。”岫煙聽了,便自往櫳翠庵來。寶玉回房寫了帖子,上面只寫“檻內人寶玉薰沐謹拜”幾字,親自拿了到櫳翠庵,只隔門縫兒投進去便回來了。
\end{parag}


\begin{parag}
    因又見芳官梳了頭,挽起䰖來,帶了些花翠,忙命他改妝,又命將周圍的短髮剃了去,露出碧青頭皮來,當中分大頂,又說:“冬天作大貂鼠臥兔兒帶,腳上穿虎頭盤雲五彩小戰靴,或散著褲腿,只用淨襪厚底鑲鞋。”又說:“芳官之名不好,竟改了男名才別緻。”因又改作 “雄奴”。芳官十分稱心,又說:“既如此,你出門也帶我出去。有人問,只說我和茗煙一樣的小廝就是了。”寶玉笑道:“到底人看的出來。”芳官笑道:“我說你是無才的。\begin{note}庚雙夾:用芳官一罵,有趣。\end{note}咱家現有幾家土番,你就說我是個小土番兒。況且人人說我打聯垂好看,你想這話可妙?”寶玉聽了,喜出意外,忙笑道:“這卻很好。我亦常見官員人等多有跟從外國獻俘之種,圖其不畏風霜,鞍馬便捷。既這等,再起個番名,叫作‘耶律雄奴’。‘雄奴’二音,又與匈奴相通,都是犬戎名姓。況且這兩種人自堯舜時便爲中華之患,晉唐諸朝,深受其害。幸得咱們有福,生在當今之世,大舜之正裔,聖虞之功德仁孝,赫赫格天,同天地日月億兆不朽,所以凡歷朝中跳猖獗之小丑,到了如今竟不用一干一戈,皆天使其拱手俛頭緣遠來降。我們正該作踐他們,爲君父生色。”芳官笑道:“既這樣著,你該去操習弓馬,學些武藝,挺身出去拿幾個反叛來,豈不進忠效力了。何必借我們,你鼓脣搖舌的,自己開心作戲,卻說是稱功頌德呢。”寶玉笑道:“所以你不明白。如今四海賓服,八方寧靜,千載百載不用武備。咱們雖一戲一笑,也該稱頌,方不負坐享昇平了。”芳官聽了有理,二人自爲妥貼甚宜。寶玉便叫他“耶律雄奴”。
\end{parag}


\begin{parag}
    究竟賈府二宅皆有先人當年所獲之囚賜爲奴隸,只不過令其飼養馬匹,皆不堪大用。湘雲素習憨戲異常,他也最喜武扮的,每每自己束鑾帶,穿折袖。近見寶玉將芳官扮成男子,他便將葵官也扮了個小子。那葵官本是常刮剔短髮,好便於面上粉墨油彩,手腳又伶便,打扮了又省一層手。李紈探春見了也愛,便將寶琴的荳官也就命他打扮了一個小童,頭上兩個丫髻,短襖紅鞋,只差了塗臉,便儼是戲上的一個琴童。湘雲將葵官改了,換作“大英”。因他姓韋,便叫他作韋大英,方合自己的意思,暗有“惟大英雄能本色”之語,何必塗朱抹粉,纔是男子。荳官身量年紀皆極小,又極鬼靈,故曰荳官。園中人也有喚他作“阿荳”的,也有喚作“炒豆子”的。寶琴反說琴童書童等名太熟了,竟是荳字別緻,便換作“荳童”。
\end{parag}


\begin{parag}
    因飯後平兒還席,說紅香圃太熱,便在榆蔭堂中擺了几席新酒佳餚。可喜尤氏又帶了佩鳳偕鴛二妾過來遊頑。這二妾亦是青年姣憨女子,不常過來的,今既入了這園,再遇見湘雲、香菱、芳、蕊一干女子,所謂“方以類聚,物以羣分”二語不錯,只見他們說笑不了,也不管尤氏在那裏,只憑丫鬟們去伏侍,且同衆人一一的遊頑。一時到了怡紅院,忽聽寶玉叫“耶律雄奴”,把佩鳳、偕鴛、香菱三個人笑在一處,問是什麼話,大家也學著叫這名字,又叫錯了音韻,或忘了字眼,甚至於叫出“野驢子”來,引的合園中人凡聽見無不笑倒。寶玉又見人人取笑,恐作踐了他,忙又說:“海西福朗思牙,聞有金星玻璃寶石,他本國番語以金星玻璃名爲 ‘溫都里納’。如今將你比作他,就改名喚叫‘溫都里納’可好?”芳官聽了更喜,說:“就是這樣罷。”因此又喚了這名。衆人嫌拗口,仍翻漢名,就喚“玻璃”。
\end{parag}


\begin{parag}
    閒言少述,且說當下衆人都在榆蔭堂中以酒爲名,大家頑笑,命女先兒擊鼓。平兒採了一枝芍藥,大家約二十來人傳花爲令,熱鬧了一回。因人回說:“甄家有兩個女人送東西來了。”探春和李紈尤氏三人出去議事廳相見,這裏衆人且出來散一散。佩鳳偕鴛兩個去打鞦韆頑耍,\begin{note}庚雙夾:大家千金不令作此戲,故寫不及探春等人也。\end{note}寶玉便說:“你兩個上去,讓我送。”慌的佩鳳說:“罷了,別替我們鬧亂子,倒是叫‘野驢子’來送送使得。”寶玉忙笑說:“好姐姐們別頑了,沒的叫人跟著你們學著罵他。”偕鴛又說:“笑軟了,怎麼打呢。掉下來栽出你的黃子來。”佩鳳便趕著他打。
\end{parag}


\begin{parag}
    正頑笑不絕,忽見東府中幾個人慌慌張張跑來說:“老爺賓天了。”衆人聽了,唬了一大跳,忙都說:“好好的並無疾病,怎麼就沒了?”家下人說:“老爺天天修煉,定是功行圓滿,昇仙去了。”尤氏一聞此言,又見賈珍父子並賈璉等皆不在家,一時竟沒個著已的男子來,未免忙了。只得忙卸了妝飾,命人先到玄真觀將所有的道士都鎖了起來,等大爺來家審問。一面忙忙坐車帶了賴升一干家人媳婦出城。又請太醫看視到底系何病。大夫們見人已死,何處診脈來,素知賈敬導氣之術總屬虛誕,更至參星禮斗,守庚申,服靈砂,妄作虛爲,過於勞神費力,反因此傷了性命的。如今雖死,肚中堅硬似鐵,麪皮嘴脣燒的紫絳皺裂。便向媳婦回說: “系玄教中吞金服砂,燒脹而歿。”衆道士慌的回說:“原是老爺祕法新制的丹砂喫壞事,小道們也曾勸說‘功行未到且服不得’,不承望老爺於今夜守庚申時悄悄的服了下去,便昇仙了。這恐是虔心得道,已出苦海,脫去皮囊,自了去也。”尤氏也不聽,只命鎖著,等賈珍來發放,且命人去飛馬報信。一面看視這裏窄狹,不能停放,橫豎也不能進城的,忙裝裹好了,用軟轎抬至鐵檻寺來停放,掐指算來,至早也得半月的工夫,賈珍方能來到。目今天氣炎熱,實不得相待,遂自行主持,命天文生擇了日期入殮。壽木已係早年備下寄在此廟的,甚是便宜。三日後便開喪破孝。一面且做起道場來等賈珍。
\end{parag}


\begin{parag}
    榮府中鳳姐兒出不來,李紈又照顧姊妹,寶玉不識事體,只得將外頭之事暫託了幾個家中二等管事人。賈㻞、賈珖、賈珩、賈瓔、賈菖、賈菱等各有執事。尤氏不能回家,便將他繼母接來在寧府看家。他這繼母只得將兩個未出嫁的小女帶來,一併起居才放心。\begin{note}庚雙夾:原爲放心而來,終是放心而去,妙甚!\end{note}
\end{parag}


\begin{parag}
    且說賈珍聞了此信,即忙告假,並賈蓉是有職之人。禮部見當今隆敦孝弟,不敢自專,具本請旨。原來天子極是仁孝過天的,且更隆重功臣之裔,一見此本,便詔問賈敬何職。禮部代奏:“系進士出身,祖職已蔭其子賈珍。賈敬因年邁多疾,常養靜于都城之外玄真觀。今因疾歿於寺中,其子珍,其孫蓉,現因國喪隨駕在此,故乞假歸殮。”天子聽了,忙下額外恩旨曰:“賈敬雖白衣無功於國,念彼祖父之功,追賜五品之職。令其子孫扶柩由北下之門進都,入彼私第殯殮。任子孫盡喪禮畢扶柩回籍外,著光祿寺按上例賜祭。朝中由王公以下準其祭弔。欽此。”此旨一下,不但賈府中人謝恩,連朝中所有大 皆嵩呼稱頌不絕。
\end{parag}


\begin{parag}
    賈珍父子星夜馳回,半路中又見賈㻞賈珖二人領家丁飛騎而來,看見賈珍,一齊滾鞍下馬請安。賈珍忙問:“作什麼?”賈㻞回說:“嫂子恐哥哥和侄兒來了,老太太路上無人,叫我們兩個來護送老太太的。”賈珍聽了,贊稱不絕,又問家中如何料理。賈㻞等便將如何拿了道士,如何挪至家廟,怕家內無人接了親家母和兩個姨娘在上房住著。賈蓉當下也下了馬,聽見兩個姨娘來了,便和賈珍一笑。賈珍忙說了幾聲“妥當”,加鞭便走,店也不投,連夜換馬飛馳。一日到了都門,先奔入鐵檻寺。那天已是四更天氣,坐更的聞知,忙喝起衆人來。賈珍下了馬,和賈蓉放聲大哭,從大門外便跪爬進來,至棺前稽顙泣血,直哭到天亮喉嚨都啞了方住。尤氏等都一齊見過。賈珍父子忙按禮換了凶服,在棺前俯伏,無奈自要理事,竟不能目不視物,耳不聞聲,少不得減些悲慼,好指揮衆人。因將恩旨備述與衆親友聽了。一面先打發賈蓉家中料理停靈之事。
\end{parag}


\begin{parag}
    賈蓉得不得一聲兒,先騎馬飛來至家,忙命前廳收桌椅,下槅扇,掛孝幔子,門前起鼓手棚牌樓等事。又忙著進來看外祖母兩個姨娘。原來尤老安人年高喜睡,常歪著,他二姨娘三姨娘都和丫頭們作活計,他來了都道煩惱。賈蓉且嘻嘻的望他二姨娘笑說:“二姨娘,你又來了,我們父親正想你呢。”尤二姐便紅了臉,罵道:“蓉小子,我過兩日不罵你幾句,你就過不得了。越發連個體統都沒了。還虧你是大家公子哥兒,每日唸書學禮的,越發連那小家子瓢坎的也跟不上。”說著順手拿起一個熨斗來,摟頭就打,嚇的賈蓉抱著頭滾到懷裏告饒。尤三姐便上來撕嘴,又說:“等姐姐來家,咱們告訴他。”賈蓉忙笑著跪在炕上求饒,他兩個又笑了。賈蓉又和二姨搶砂仁喫,尤二姐嚼了一嘴渣子,吐了他一臉。賈蓉用舌頭都舔著吃了。衆丫頭看不過,都笑說:“熱孝在身上,老孃才睡了覺,他兩個雖小,到底是姨娘家,你太眼裏沒有奶奶了。回來告訴爺,你吃不了兜著走。”賈蓉撇下他姨娘,便抱著丫頭們親嘴:“我的心肝,你說的是,咱們饞他兩個。”丫頭們忙推他,恨的罵:“短命鬼兒,你一般有老婆丫頭,只和我們鬧。知道的說是頑;\begin{note}庚雙夾:妙極之“頑”,天下有是之頑亦有趣甚,此語餘亦親聞者,非編有也。\end{note}不知道的人,再遇見那髒心爛肺的愛多管閒事嚼舌頭的人,吵嚷的那府裏誰不知道,誰不背地裏嚼舌說咱們這邊亂帳。”賈蓉笑道:“各門另戶,誰管誰的事。都夠使的了。從古至今,連漢朝和唐朝,人還說髒唐臭漢,何況咱們這宗人家。誰家沒風流事,別討我說出來。連那邊大老爺這麼利害,璉叔還和那小姨娘不乾淨呢。鳳姑娘那樣剛強,瑞叔還想他的帳。那一件瞞了我!”
\end{parag}


\begin{parag}
    賈蓉只管信口開合胡言亂道之間,只見他老孃醒了,請安問好,又說:“難爲老祖宗勞心,又難爲兩位姨娘受委屈,我們爺兒們感戴不盡。惟有等事完了,我們合家大小,登門去磕頭。”尤老人點頭道:“我的兒,倒是你們會說話。親戚們原是該的。”又問:“你父親好?幾時得了信趕到的?”賈蓉笑道:“纔剛趕到的,先打發我瞧你老人家來了。好歹求你老人家事完了再去。”說著,又和他二姨擠眼,那尤二姐便悄悄咬牙含笑罵:“很會嚼舌頭的猴兒崽子,留下我們給你爹作娘不成!”賈蓉又戲他老孃道:“放心罷,我父親每日爲兩位姨娘操心,要尋兩個又有根基又富貴又年青又俏皮的兩位姨爹,好聘嫁這二位姨娘的。這幾年總沒揀得,可巧前日路上才相準了一個。”尤老只當真話,忙問是誰家的,二姊妹丟了活計,一頭笑,一頭趕著打。說:“媽別信這雷打的。”連丫頭們都說:“天老爺有眼,仔細雷要緊!”又值人來回話:“事已完了,請哥兒出去看了,回爺的話去。”那賈蓉方笑嘻嘻的去了。不知如何,且聽下回分解。
\end{parag}


\begin{parag}
    \begin{note}蒙回末總:寶玉品高性雅,其終日花圍翠繞,用力維持其間,淫蕩之至,而能使旁人不覺,彼人不厭。賈蓉不分長幼微賤,縱意馳騁於中,惡習可恨。二人之行景天淵而終邪,其一濫也,所謂五十步之間而。持家有意於子弟者,揣此而照察之。可也!\end{note}
\end{parag}

