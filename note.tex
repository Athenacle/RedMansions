\chapter*{凡例}
\addcontentsline{toc}{chapter}{凡例}

\begin{qute2sp}
    \large
    \begin{parag}
        《紅樓夢》旨意。是書題名極多,一曰《紅樓夢》,是總其全部之名也。又曰《風月寶鑑》,是戒妄動風月之情。又曰《石頭記》,是自譬石頭所記之事也。此三名則書中曾已點睛矣。如寶玉做夢,夢中有曲,名曰《紅樓夢》十二支,此則《紅樓夢》之點睛。又如賈瑞病,跛道人持一鏡來,上面即鏨風月寶鑑四字,此則《風月寶鑑》之點睛。又如道人親見石上大書一篇故事,則系石頭所記之往來,此則《石頭記》之點睛處。然此書又名曰《金陵十二釵》,審其名,則必系金陵十二女子也。然通部細搜檢去,上中下女子豈止十二人哉?若雲其中自有十二個,則又未嘗指明白系某某,極至“紅樓夢”一回中,亦曾翻出金陵十二釵之簿籍,又有十二支曲可考。
    \end{parag}

    \begin{parag}
        書中凡寫長安,在文人筆墨之間,則從古之稱;凡愚夫婦兒女子家常口角,則曰中京,是不欲著跡於方向也。蓋天子之邦,亦當以中爲尊,特避其東南西北四字樣也。
    \end{parag}

    \begin{parag}
        此書只是著意於閨中,故敘閨中之事切,略涉於外事者則簡,不得謂其不均也。
    \end{parag}

    \begin{parag}
        此書不敢幹涉朝廷,凡有不得不用朝政者只略用一筆帶出,蓋實不敢以寫兒女之筆墨唐突朝廷之上也。又不得謂其不備。
    \end{parag}
\end{qute2sp}

\cleardoublepage
\clearpage