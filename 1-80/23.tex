\chap{二十三}{西廂記妙詞通戲語 牡丹亭豔曲警芳心}

\begin{parag}
    \begin{note}蒙回前詩:羣豔大觀中,柳弱絮春風。惜花與度曲,笑看利名空。\end{note}
\end{parag}


\begin{parag}
    話說賈元春自那日幸大觀園回宮去後,便命將那日所有的題詠,命探春依次抄錄妥協,自己編次,敘其優劣,又命在大觀園勒石,爲千古風流雅事。因此,賈政命人各處選拔精工名匠,在大觀園磨石鐫字,賈珍率領蓉、萍等監工。因賈薔又管理著文官等十二個女戲並行頭等事,不大得便,因此賈珍又將賈菖、賈菱喚來監工。一日,湯蠟釘朱,動起手來。這也不在話下。
\end{parag}


\begin{parag}
    且說那個玉皇廟並達摩庵兩處,一班的十二個小沙彌並十二個小道士,如今挪出大觀園來,賈政正想發到各廟去分住。不想后街上住的賈芹之母周氏,正盤算著也要到賈政這邊謀一個大小事務與兒子管管,也好弄些銀錢使用,可巧聽見這件事出來,便坐轎子來求鳳姐。鳳姐因見他素日不大拿班作勢的,便依允了,想了幾句話\begin{note}庚側:一派心機。\end{note}便回王夫人說:“這些小和尚道士萬不可打發到別處去,一時娘娘出來就要承應。倘或散了,若再用時,可是又費事。依我的主意,不如將他們竟送到咱們家廟裏鐵檻寺去,月間不過派一個人拿幾兩銀子去買柴米就完了。說聲用,走去叫來,一點兒不費事呢。”王夫人聽了,便商之於賈政。賈政聽了笑道:“倒是提醒了我,就是這樣。” 即時喚賈璉來。
\end{parag}


\begin{parag}
    當下賈璉正同鳳姐喫飯,一聞呼喚,不知何事,放下飯便走。鳳姐一把拉住,笑道:“你且站住,聽我說話。若是別的事我不管,若是爲小和尚們的事,好歹依我這麼著。”如此這般教了一套話。賈璉笑道:“我不知道,你有本事你說去。”風姐聽了,把頭一梗,把筷子一放,\begin{note}蒙側:活跳。\end{note}腮上似笑不笑的瞅著賈璉道:“你當真的,是玩話?”賈璉笑道:“西廊下五嫂子的兒子芸兒來求了我兩三遭,\begin{note}蒙側:發人一笑。\end{note}要個事情管管。我依了,叫他等著。好容易出來這件事,你又奪了去。”鳳姐兒笑道:“你放心。園子東北角子上,娘娘說了,還叫多多的種松柏樹,樓底下還叫種些花草。等這件事出來,我管保叫芸兒管這件工程。”賈璉道:“果這樣也罷了。只是昨兒晚上,我不過是要改個樣兒,你就扭手扭腳的。”\begin{note}蒙側:粗蠢惜景可笑。\end{note}\begin{note}蒙側:後將有大觀園中一段奇情韻,不得不先爲此等醜語一造(?),以作未火先煙之象。\end{note}\begin{note}庚側:寫鳳姐風月之文如此,總不脫漏。\end{note}鳳姐兒聽了,嗤的一聲笑了,\begin{note}庚側:好章法!\end{note}向賈璉啐了一口,低下頭便喫飯。
\end{parag}


\begin{parag}
    賈璉已經笑著去了,到了前面見了賈政,果然是小和尚一事。賈璉便依了鳳姐主意,說道:“如今看來,芹兒倒大大的出息了,這件事竟交予他去管辦。橫豎照在裏頭的規例,每月叫芹兒支領就是了。”賈政原不大理論這些事,聽賈璉如此說,便如此依了。賈璉回到房中告訴鳳姐兒,鳳姐即命人去告訴了周氏。賈芹便來見賈璉夫妻兩個,感謝不盡。風姐又作情央賈璉先支三個月的,叫他寫了領字,賈璉批票畫了押,登時發了對牌出去。銀庫上按數發出三個月的供給來,白花花二三百兩。賈芹隨手拈一塊,撂予掌平的人,叫他們喫茶罷。於是命小廝拿回家,與母親商議。登時僱了大叫驢,自己騎上,又僱了幾輛車,至榮國府角門,喚出二十四個人來,坐上車,一徑往城外鐵檻寺去了。當下無話。
\end{parag}


\begin{parag}
    如今且說賈元春,因在宮中自編大觀園題詠之後,忽想起那大觀園中景緻,自己幸過之後,賈政必定敬謹封鎖,不敢使人進去騷擾,豈不寥落。況家中現有幾個能詩會賦的姊妹,何不命他們進去居住,也不使佳人落魄,花柳無顏。\begin{note}庚側:韻人行韻事。\end{note}卻又想到寶玉自幼在姊妹叢中長大,\begin{note}蒙側:何等精細!\end{note}不比別的兄弟,若不命他進去,只怕他冷清了,一時不大暢快,未免賈母王夫人愁慮,須得也命他進園居住方妙。\begin{note}庚眉:大觀園原系十二釵棲止之所,然工程浩大,故借元春之名而起,再用元春之命以安諸豔,不見一絲扭捻。己冬夜。\end{note}想畢,遂命太監夏守忠到榮國府來下一道諭,命寶釵等只管在園中居住,不可禁約封錮,命寶玉仍隨進去讀書。
\end{parag}


\begin{parag}
    賈政,王夫人接了這諭,待夏守忠去後,便來回明賈母,遣人進去各處收拾打掃,安設簾幔牀帳。別人聽了還自猶可,惟寶玉聽了這諭,喜的無可不可。正和賈母盤算,要這個,弄那個,忽見丫鬟來說:“老爺叫寶玉。”\begin{note}庚側:多大力量寫此句。餘亦驚駭,況寶玉乎!回思十二三時,亦曾有是病來。想時不再至,不禁淚下。\end{note}寶玉聽了,\begin{note}蒙側:大家風範!\end{note}好似打了個焦雷,登時掃去興頭,臉上轉了顏色,便拉著賈母扭的好似扭股兒糖,殺死不敢去。賈母只得安慰他道: “好寶貝,你只管去,有我呢,他不敢委屈了你。\begin{note}蒙側:寫盡祖母溺愛,作後文之本!\end{note}況且你又作了那篇好文章。想是娘娘叫你進去住,他吩咐你幾句,不過不教你在裏頭淘氣。他說什麼,你只好生答應著就是了。”一面安慰,一面喚了兩個老嬤嬤來,吩咐:“好生帶了寶玉去,別叫他老子唬著他。”老嬤嬤答應了。
\end{parag}


\begin{parag}
    寶玉只得前去,一步挪不了三寸,蹭到這邊來。可巧賈政在王夫人房中商議事情,金釧兒、彩雲、彩霞、繡鸞、繡鳳等衆丫鬟都在廊檐底下站著呢,一見寶玉來,都抿著嘴笑。金釧一把拉住寶玉,\begin{note}庚側:有是事,有是人。\end{note}悄悄的笑道:“我這嘴上是才擦的香浸胭脂,\begin{note}庚側:活像活現。\end{note}你這會子可喫不吃了?”彩雲一把推開金釧,笑道:“人家正心裏不自在,你還奚落他。趁這會子喜歡,快進去罷。”寶玉只得挨進門去。原來賈政和王夫人都在裏間呢。趙姨娘打起簾子,寶玉躬身進去。只見賈政和王夫人對面坐在炕上說話,地下一溜椅子,迎春、探春、惜春、賈環四個人都坐在那裏。一見他進來,惟有探春和惜春、賈環站了起來。
\end{parag}


\begin{parag}
    賈政一舉目,見寶玉站在跟前,神彩飄逸,秀色奪人,\begin{note}庚側:“消氣散”用的好。\end{note}看看賈環,人物委瑣,舉止荒疏,忽又想起賈珠來,\begin{note}庚側:批至此,幾乎失聲哭出。\end{note}再看看王夫人只有這一個親生的兒子,素愛如珍,自己的鬍鬚將已蒼白:因這幾件上,把素日嫌惡處分寶玉之心不覺減了八九。\begin{note}蒙側:爲天下年老父母一哭!\end{note}半晌說道:“娘娘吩咐說,你日日外頭嬉遊,漸次疏懶,如今叫禁管,\begin{note}庚眉:寫寶玉可入園,用“禁管”二字,得體理之至。壬午九月。\end{note}同你姊妹在園裏讀書寫字。你可好生用心習學,再如不守分安常,你可仔細!”寶玉連連的答應了幾個“是”。王夫人便拉他在身旁坐下。\begin{note}蒙側:活現!\end{note}他姊弟三人依舊坐下。
\end{parag}


\begin{parag}
    王夫人摸挲著寶玉的脖項說道:“前兒的丸藥都喫完了?”寶玉答道:“還有一丸。”王夫人道:“明兒再取十丸來,天天臨睡的時候,叫襲人伏侍你吃了再睡。”寶玉道:“只從太太吩咐了,襲人天天晚上想著,打發我喫。”\begin{note}庚側:大家細細聽去,活似小兒口氣。\end{note}賈政問道:“襲人是何人?”王夫人道:“是個丫頭。”賈政道:“丫頭不管叫個什麼罷了,是誰這樣刁鑽,起這樣的名字?”王夫人見賈政不自在了,便替寶玉掩飾道:“是老太太起的。”賈政道:“老太太如何知道這話,一定是寶玉。”寶玉見瞞不過,只得起身回道:“因素日讀詩,曾記古人有一句詩云:‘花氣襲人知晝暖 ’。因這個丫頭姓花,便隨口起了這個名字。”王夫人忙又道:“寶玉,你回去改了罷。老爺也不用爲這小事動氣。”賈政道:“究竟也無礙,又何用改。\begin{note}庚側:幾乎改去好名。\end{note}只是可見寶玉不務正,專在這些濃詞豔賦上作工夫。”說畢,斷喝一聲:\begin{note}庚側:好收拾。\end{note}\begin{note}蒙側:嚴父慈母,其事異,其行則一。\end{note}“作業的畜生,還不出去!”王夫人也忙道:“去罷,只怕老太太等你喫飯呢。”寶玉答應了,慢慢的退出去,向金釧兒笑著伸伸舌頭,帶著兩個嬤嬤一溜煙去了。
\end{parag}


\begin{parag}
    剛至穿堂門前,\begin{note}庚雙夾:妙!這便是鳳姐掃雪拾玉之處,一絲不亂。\end{note}只見襲人倚門立在那裏\begin{note}蒙側:何等牽連!\end{note},一見寶玉平安回來,堆下笑來問\begin{note}庚側:等壞了,愁壞了。所以有“堆下笑來問”之話。\end{note}道:“叫你作什麼?”寶玉告訴他:“沒有什麼,不過怕我進園去淘氣,吩咐吩咐。”\begin{note}就說大話,畢肖之至!\end{note}一面說,一面回至賈母跟前,回明原委。只見林黛玉正在那裏,寶玉便問他:“你住那一處好?”林黛玉正心裏盤算這事,\begin{note}庚側:顰兒亦有盤算事,揀擇清幽處耳,未知擇鄰否?一笑。\end{note}忽見寶玉問他,便笑道:“我心裏想著瀟湘館好,愛那幾竿竹子隱著一道曲欄,比別處更覺幽靜。”寶玉聽了拍手笑道:“正和我的主意一樣,我也要叫你住這裏呢。我就住怡紅院,咱們兩個又近,又都清幽。”\begin{note}庚側:擇鄰出於玉兄,所謂真知己。\end{note}\begin{note}蒙側:作後文無限章本。\end{note}
\end{parag}


\begin{parag}
    兩人正計較,就有賈政遣人來回賈母說:“二月二十二曰子好,哥兒姐兒們好搬進去的。這幾日內遣人進去分派收拾。”薛寶釵住了蘅蕪苑,林黛玉住了瀟湘館,賈迎春住了綴錦樓,探春住了秋爽齋,惜春住了蓼風軒,李氏住了稻香村,寶玉住了怡紅院。每一處添兩個老嬤嬤,四個丫頭,除各人奶孃親隨丫鬟不算外,另有專管收拾打掃的。至二十二日,一齊進去,登時園內花招繡帶,柳拂香風,\begin{note}庚雙夾:八字寫得滿園之內處處有人,無一處不到。\end{note}不似前番那等寂寞了。
\end{parag}


\begin{parag}
    閒言少敘。且說寶玉自進花園以來,心滿意足,再無別項可生貪求之心。每日只和姊妹丫頭們一處,或讀書,\begin{note}庚側:末必。\end{note}或寫字,或彈琴下棋,作畫吟詩,以至描鸞刺鳳,\begin{note}庚側:有之。\end{note}鬥草簪花,低吟悄唱,拆字猜枚,無所不至,倒也十分快樂。他曾有幾首即事詩,雖不算好,卻倒是真情真景,略記幾首雲:
\end{parag}


\begin{poem}
    \begin{pl}春夜即事\end{pl}

    \begin{pl}霞綃雲幄任鋪陳,隔巷蟆更聽未真。\end{pl}

    \begin{pl}枕上輕寒窗外雨,眼前春色夢中人。\end{pl}

    \begin{pl}盈盈燭淚因誰泣,點點花愁爲我嗔。\end{pl}

    \begin{pl}自是小鬟嬌懶慣,擁衾不耐笑言頻。\end{pl}
\end{poem}


\begin{poem}
    \begin{pl}夏夜即事\end{pl}

    \begin{pl}倦繡佳人幽夢長,金籠鸚鵡喚茶湯。\end{pl}

    \begin{pl}窗明麝月開宮鏡,室靄檀雲品御香。\end{pl}

    \begin{pl}琥珀杯傾荷露滑,玻璃檻納柳風涼。\end{pl}

    \begin{pl}水亭處處齊紈動,簾卷朱樓罷晚妝。\end{pl}

\end{poem}


\begin{poem}
    \begin{pl}秋夜即事\end{pl}

    \begin{pl}絳芸軒裏絕喧譁,桂魄流光浸茜紗。\end{pl}

    \begin{pl}苔鎖石紋容睡鶴,井飄桐露溼棲鴉。\end{pl}

    \begin{pl}抱衾婢至舒金鳳,倚檻人歸落翠花。\end{pl}

    \begin{pl}靜夜不眠因酒渴,沉煙重撥索烹茶。\end{pl}

\end{poem}


\begin{poem}
    \begin{pl}冬夜即事\end{pl}

    \begin{pl}梅魂竹夢已三更,錦罽鸘衾睡未成。\end{pl}

    \begin{pl}松影一庭惟見鶴,梨花滿地不聞鶯。\end{pl}

    \begin{pl}女兒翠袖詩懷冷,公子金貂酒力輕。\end{pl}

    \begin{pl}卻喜侍兒知試茗,掃將新雪及時烹。\end{pl}\begin{note}庚眉:四詩作盡安福尊榮之貴介公子也。壬午孟夏。\end{note}
\end{poem}


\begin{parag}
    因這幾首詩,當時有一等勢利人,見是榮國府十二三歲的公子作的,抄錄出來各處稱頌,再有一等輕浮子弟,愛上那風騷妖豔之句,也寫在扇頭壁上,不時吟哦賞讚。因此竟有人來尋詩覓字,倩畫求題的。寶玉亦發得了意,鎮日家作這些外務。
\end{parag}


\begin{parag}
    誰想靜中生煩惱,忽一日不自在起來,這也不好,那也不好,出來進去只是悶悶的。園中那些人多半是女孩兒,正在混沌世界,天真爛漫之時,坐臥不避,嘻笑無心,那裏知寶玉此時的心事。那寶玉心內不自在,便懶在園內,只在外頭鬼混,卻又癡癡的。\begin{note}庚雙夾:不進園去,真不知何心事。\end{note}
\end{parag}


\begin{parag}
    茗煙見他這樣,因想與他開心,左思右想,皆是寶玉頑煩了的,不能開心,惟有這件,寶玉不曾看見過。\begin{note}庚側:書房伴讀累累如是,餘至今痛恨。\end{note}想畢,便走去到書坊內,把那古今小說並那飛燕、合德、武則天、楊貴妃的外傳與那傳奇角本買了許多來,引寶玉看。寶玉何曾見過這些書,一看見了便如得了珍寶。茗煙囑咐他不可拿進園去,“若叫人知道了,我就吃不了兜著走呢。”寶玉那裏舍的不拿進園去,踟躕再三,單把那文理細密的揀了幾套進去,放在牀頂上,無人時自己密看。那粗俗過露的,都藏在外面書房裏。
\end{parag}


\begin{parag}
    那一日正當三月中浣,早飯後,寶玉攜了一套《會真記》,走到沁芳閘橋邊桃花底下一塊石上坐著,展開《會真記》,從頭細玩。正看到“落紅成陣”,只見一陣風過,把樹頭上桃花吹下一大半來,\begin{note}庚側:好一陣湊趣風。\end{note}落的滿身滿書滿地皆是。寶玉要抖將下來,恐怕腳步踐踏了,\begin{note}庚雙夾:情不情。\end{note}只得兜了那花瓣,來至池邊,抖在池內。那花瓣浮在水面,飄飄蕩蕩,竟流出沁芳閘去了。
\end{parag}


\begin{parag}
    回來只見地下還有許多,寶玉正踟躕間,只聽背後有人說道:“你在這裏作什麼?”寶玉一回頭,卻是林黛玉來了,肩上擔著花鋤,\begin{note}庚側:一幅採芝圖,非葬花圖也。\end{note}鋤上掛著花囊,\begin{note}蒙側:真是韻人韻事!\end{note}手內拿著花帚。\begin{note}庚眉:此圖欲畫之心久矣,誓不過仙筆不寫,恐褻我顰卿故也。己冬。\end{note}\begin{note}庚眉:丁亥春間,偶識一浙省新發,其白描美人,真神品物,甚合餘意。奈彼因宦緣所纏無暇,且不能久留都下,未幾南行矣。餘至今耿耿,悵然之至。恨與阿顰結一筆墨之難若此!嘆嘆!丁亥夏。笏叟。\end{note}寶玉笑道:“好,好,來把這個花掃起來,\begin{note}庚側:如見如聞。\end{note}撂在那水裏。我才撂了好些在那裏呢。”林黛玉道:“撂在水裏不好。你看這裏的水乾淨,只一流出去,有人家的地方髒的臭的混倒,仍舊把花遭塌了。那畸角上我有一個花冢,\begin{note}庚側:好名色!新奇!葬花亭裏埋花人。\end{note}如今把他掃了,裝在這絹袋裏,拿土埋上,日久不過隨土化了,\begin{note}庚側:寧使香魂隨土化。\end{note}豈不乾淨。”\begin{note}庚雙夾:寫黛玉又勝寶玉十倍癡情。\end{note}寶玉聽了喜不自禁,笑道:“待我放下書,幫你來收拾。”\begin{note}庚側:顧了這頭,忘卻那頭。\end{note}黛玉道:“什麼書?”寶玉見問,慌的藏之不迭,便說道:“不過是《中庸》《大學》。”黛玉笑道:“你又在我跟前弄鬼。趁早兒給我瞧,好多著呢。”寶玉道:“好妹妹,若論你,我是不怕的。你看了,好歹別告訴別人去。真真這是好書!你要看了,連飯也不想喫呢。”一面說,一面遞了過去。林黛玉把花具且都放下,接書來瞧,從頭看去,越看越愛看,不到一頓飯工夫,將十六出俱已看完,自覺詞藻警人,餘香滿口。雖看完了書,卻只管出神,心內還默默記誦。
\end{parag}


\begin{parag}
    寶玉笑道:“妹妹,你說好不好?”林黛玉笑道:“果然有趣。”寶玉笑道:“我就是個‘多愁多病身’,你就是那‘傾國傾城貌’。”\begin{note}庚側:看官說寶玉忘情有之,若認作有心取笑,則看不得《石頭記》。\end{note}林黛玉聽了,不覺帶腮連耳通紅,登時直豎起兩道似蹙非蹙的眉,瞪了兩隻似睜非睜的眼,微腮帶怒,薄面含嗔,指寶玉道:“你這該死的胡說!好好的把這淫詞豔曲弄了來,還學了這些混話來欺負我。我告訴舅舅舅母去。”說到“欺負”兩個字上,早又把眼睛圈兒紅了,轉身就走。\begin{note}庚側:唬殺!急殺!\end{note}寶玉著了急,向前攔住說道:“好妹妹,千萬饒我這一遭,原是我說錯了。若有心欺負你,明兒我掉在池子裏,教個癩頭黿吞了去,變個大忘八,等你明兒做了‘一品夫人’病老歸西的時候,我往你墳上替你馱一輩子的碑去。”\begin{note}庚側:雖是混話一串,卻成了最新最奇的妙文。[此誓新鮮。]\end{note}說的林黛玉嗤的一聲笑了,\begin{note}庚側:看官想用何等話令黛玉一笑收科?\end{note}揉著眼睛,一面笑道:“一般也唬的這個調兒,還只管胡說。呸,原來是‘苗而不秀,是個銀樣鑞槍頭’。”\begin{note}庚側:[更借得妙!]\end{note}寶玉聽了,笑道:“你這個呢?我也告訴去。”林黛玉笑道:“你說你會過目成誦,難道我就不能一目十行麼?”\begin{note}蒙側:兒女情,絲毫無淫念,韻雅直至!\end{note}
\end{parag}


\begin{parag}
    寶玉一面收書,一面笑道:“正經快把花埋了罷,別提那個了。”二人便收拾落花,正才掩埋妥協,只見襲人走來,說道:“那裏沒找到,摸在這裏來。那邊大老爺身上不好,姑娘們都過去請安,老太太叫打發你去呢。快回去換衣裳去罷。”寶玉聽了,忙拿了書,別了黛玉,同襲人回房換衣不提。\begin{note}庚雙夾:一語度下。\end{note}
\end{parag}


\begin{parag}
    這裏林黛玉見寶玉去了,又聽見衆姊妹也不在房,自己悶悶的。\begin{note}庚雙夾:有原故。\end{note}正欲回房,剛走到梨香院牆角上,只聽牆內笛韻悠揚,歌聲婉轉。\begin{note}庚側:入正文方不牽強。\end{note}林黛玉便知是那十二個女孩子演習戲文呢。只是林黛玉素習不大喜看戲文,\begin{note}庚雙夾:妙法!必雲“不大喜看”。\end{note}便不留心,只管往前走。偶然兩句吹到耳內,明明白白,一字不落,唱\begin{note}庚雙夾:卻一喜便總不忘,方見楔得緊。\end{note}道是:“原來奼紫嫣紅開遍,似這般都付與斷井頹垣。”\begin{note}庚眉:情小姐故以情小姐詞曲警之,恰極當極!己冬。\end{note}林黛玉聽了,倒也十分感慨纏綿,便止住步側耳細聽,又聽唱道是:“良辰美景奈何天,賞心樂事誰家院。”聽了這兩句,不覺點頭自嘆,心下自思道:“原來戲上也有好文章。\begin{note}庚側:非不及釵,系不曾於雜學上用意也。\end{note}可惜世人只知看戲,未必能領略這其中的趣味。”\begin{note}庚側:將進門便是知音。\end{note}想畢,又後悔不該胡想,耽誤了聽曲子。又側耳時,只聽唱道:“則爲你如花美眷,似水流年……”林黛玉聽了這兩句,不覺心動神搖。又聽道:“你在幽閨自憐”等句,亦發如醉如癡,站立不住,便一蹲身坐在一塊山子石上,細嚼“如花美眷,似水流年”八個字的滋味。忽又想起前日見古人詩中有“水流花謝兩無情”之句,再又有詞中有“流水落花春去也,天上人間”之句,又兼方纔所見《西廂記》中“花落水流紅,閒愁萬種”之句,都一時想起來,湊聚在一處。仔細忖度,不覺心痛神癡,眼中落淚。正沒個開交,忽覺背上擊了一下,及回頭看時,原來是……且聽下回分解。正是:
\end{parag}


\begin{poem}
    \begin{pl}妝晨繡夜心無矣,對月臨風恨有之。\end{pl}
\end{poem}


\begin{parag}
    \begin{note}庚:前以《會真記》文,後以《牡丹亭》曲,加以有情有景消魂落魄詩詞,總是急於令顰兒種病根也。看其一路不跡不離,曲曲折折寫來,令觀者亦自難持,況怯怯之弱女乎!\end{note}
\end{parag}


\begin{parag}
    \begin{note}蒙回末總評:詩童才女,添大觀園之顏色;埋花聽曲,寫靈慧之悠嫺。妒婦主謀,愚夫聽命,惡僕殷勤,淫詞胎邪。開楞嚴之密語,閉法戒之真宗,以撞心之言,與石頭講道,悲夫!\end{note}
\end{parag}