\chap{七十六}{ 凸碧堂品笛感悽清 凹晶館聯詩悲寂寞}

\begin{parag}
    \begin{note}蒙回前總:此回著筆最難,不敘中元夜宴則漏,敘夜宴則與上元相犯,不敘諸人酬和則俗,敘酬和又與起社相犯,諸人在賈政面前吟詩,諸人各自爲一席,又非禮,既敘夜宴再敘酬和,不漏不俗更不相犯。雲行月移,水流花放,別有機括,深宜玩恭。\end{note}
\end{parag}


\begin{parag}
    話說賈赦賈政帶領賈珍等散去不提。且說賈母這裏命將圍屏撤去,兩席並而爲一。衆媳婦另行擦桌整果,更杯洗箸,陳設一番。賈母等都添了衣,盥漱喫茶,方又入坐,團團圍繞。賈母看時,寶釵姊妹二人不在坐內,知他們家去圓月去了,且李紈鳳姐二人又病著,少了四個人,便覺冷清了好些。\begin{note}庚雙夾:不想這次中秋反寫得十分悽楚。\end{note}賈母因笑道:“往年你老爺們不在家,咱們越性請過姨太太來,大家賞月,卻十分鬧熱。忽一時想起你老爺來,又不免想到母子夫妻兒女不能一處,也都沒興。及至今年你老爺來了,正該大家團圓取樂,又不便請他們娘兒們來說說笑笑。況且他們今年又添了兩口人,也難丟了他們跑到這裏來。偏又把鳳丫頭病了,有他一人來說說笑笑,還抵得十個人的空兒。可見天下事總難十全。”說畢,不覺長嘆一聲,遂命拿大杯來斟熱酒。王夫人笑道:“今日得母子團圓,自比往年有趣。往年娘兒們雖多,終不似今年自己骨肉齊全的好。”賈母笑道:“正是爲此,所以才高興拿大杯來喫酒。你們也換大杯纔是。”邢夫人等只得換上大杯來。因夜深體乏,且不能勝酒,未免都有些倦意,無奈賈母興猶未闌,只得陪飲。
\end{parag}


\begin{parag}
    賈母又命將罽氈鋪於階上,命將月餅西瓜果品等類都叫搬下去,令丫頭媳婦們也都團團圍坐賞月。賈母因見月至中天,比先越發精彩可愛,因說:“如此好月,不可不聞笛。”因命人將十番上女孩子傳來。賈母道:“音樂多了,反失雅緻,只用吹笛的遠遠的吹起來就夠了。”說畢,剛纔去吹時,只見跟邢夫人的媳婦走來向邢夫人前說了兩句話。賈母便問:“說什麼事?”那媳婦便回說:“方纔大老爺出去,被石頭絆了一下,崴了腿。”賈母聽說,忙命兩個婆子快看去,又命邢夫人快去。邢夫人遂告辭起身。賈母便又說:“珍哥媳婦也趁著便就家去罷,我也就睡了。”尤氏笑道:“我今日不回去了,定要和老祖宗喫一夜。” 賈母笑道:“使不得,使不得。你們小夫妻家,今夜不要團圓團圓,如何爲我耽擱了。”尤氏紅了臉,笑道:“老祖宗說的我們太不堪了。我們雖然年輕,已經是十來年的夫妻,也奔四十歲的人了。況且孝服未滿,陪著老太太頑一夜還罷了,豈有自去團圓的理。”賈母聽說,笑道:“這話很是,我倒也忘了孝未滿。可憐你公公已是二年多了,\begin{note}庚雙夾:不是算賈敬,卻是算赦死期也。\end{note}可是我倒忘了,該罰我一大杯。既這樣,你就越性別送,陪著我罷了。你叫蓉兒媳婦送去,就順便回去罷。”尤氏說了。蓉妻答應著,送出邢夫人,一同至大門,各自上車回去。不在話下。
\end{parag}


\begin{parag}
    這裏賈母仍帶衆人賞了一回桂花,又入席換暖酒來。正說著閒話,猛不防只聽那壁廂桂花樹下,嗚嗚咽咽,悠悠揚揚,吹出笛聲來。趁著這明月清風,天空地淨,真令人煩心頓解,萬慮齊除,都肅然危坐,默默相賞。聽約兩盞茶時,方纔止住,大家稱讚不已。於是遂又斟上暖酒來。賈母笑道:“果然可聽麼?”衆人笑道:“實在可聽。我們也想不到這樣,須得老太太帶領著,我們也得開些心胸。”賈母道:“這還不大好,須得揀那曲譜越慢的吹來越好。”說著,便將自己喫的一個內造瓜仁油松穰月餅,又命斟一大杯熱酒,送給譜笛之人,慢慢的吃了再細細的吹一套來。媳婦們答應了,方送去,只見方纔瞧賈赦的兩個婆子回來了,說:“右腳面上白腫了些,如今調服了藥,疼的好些了,也不甚大關係。”賈母點頭嘆道:“我也太操心。打緊說我偏心,我反這樣。”因就將方纔賈赦的笑話說與王夫人尤氏等聽。王夫人等因笑勸道:“這原是酒後大家說笑,不留心也是有的,豈有敢說老太太之理。老太太自當解釋纔是。”只見鴛鴦拿了軟巾兜與大斗篷來,說:“夜深了,恐露水下來,風吹了頭,須要添了這個。坐坐也該歇了。”賈母道:“偏今兒高興,你又來催。難道我醉了不成,偏到天亮!”因命再斟酒來。一面戴上兜巾,披了斗篷,大家陪著又飲,說些笑話。只聽桂花陰裏,嗚嗚咽咽,嫋嫋悠悠,又發出一縷笛音來,果真比先越發淒涼。大家都寂然而坐。夜靜月明,且笛聲悲怨,賈母年老帶酒之人,聽此聲音,不免有觸於心,禁不住墮下淚來。衆人彼此都不禁有淒涼寂寞之意,半日,方知賈母傷感,才忙轉身陪笑,發語解釋。\begin{note}庚雙夾:“轉身”妙!畫出對月聽笛如癡如呆、不覺尊長在上之形景來。\end{note}又命暖酒,且住了笛。尤氏笑道:“我也就學一個笑話,說與老太太解解悶。”賈母勉強笑道:“這樣更好,快說來我聽。”尤氏說道:“一家子養了四個兒子:大兒子只一個眼睛,二兒子只一個耳朵,三兒子只一個鼻子眼,四兒子倒都齊全,偏又是個啞叭。”正說到這裏,只見賈母已朦朧雙眼,似有睡去之態。\begin{note}庚雙夾:總寫出淒涼無興景況來。\end{note}尤氏方住了,忙和王夫人輕輕的請醒。賈母睜眼笑道: “我不困,白閉閉眼養神。你們只管說,我聽著呢。”王夫人等笑道:“夜已四更了,風露也大,請老太太安歇罷。明日再賞十六,也不辜負這月色。”賈母道: “那裏就四更了?”王夫人笑道:“實已四更,他們姊妹們熬不過,都去睡了。”賈母聽說,細看了一看,果然都散了,只有探春在此。賈母笑道:“也罷。你們也熬不慣,況且弱的弱,病的病,去了倒省心。只是三丫頭可憐見的,尚還等著。你也去罷,我們散了。”說著,便起身,吃了一口清茶,便有預備下的竹椅小轎,便圍著斗篷坐上,兩個婆子搭起,衆人圍隨出園去了。不在話下。
\end{parag}


\begin{parag}
    這裏衆媳婦收拾杯盤碗盞時,卻少了個細茶杯,各處尋覓不見,又問衆人:“必是誰失手打了。撂在那裏,告訴我拿了磁瓦去交收是證見,不然又說偷起來。” 衆人都說:“沒有打了,只怕跟姑娘的人打了,也未可知。你細想想,或問問他們去。”一語提醒了這管傢伙的媳婦,因笑道:“是了,那一會兒記得是翠縷拿著的。我去問他。”說著便去找時,剛下了甬道,就遇見了紫鵑和翠縷來了。\begin{note}庚雙夾:妙!又出一個。\end{note}翠縷便問道:“老太太散了,可知我們姑娘那去了?”\begin{note}庚雙夾:更妙!\end{note}這媳婦道:“我來問那一個茶鍾往那裏去了,你們倒問我要姑娘。”翠縷笑道:“我因倒茶給姑娘喫的,展眼回頭,就連姑娘也沒了。”那媳婦道:“太太才說都睡覺去了。你不知那裏頑去了,還不知道呢。”翠縷向紫鵑道:“斷乎沒有悄悄的睡去之理,只怕在那裏走了一走。如今見老太太散了,趕過前邊送去,也未可知。我們且往前邊找找去。有了姑娘,自然你的茶鍾也有了。你明日一早再找,有什麼忙的。”媳婦笑道:“有了下落就不必忙了,明兒就和你要罷。”說畢回去,仍查收傢伙。這裏紫鵑和翠縷便往賈母處來。不在話下。
\end{parag}


\begin{parag}
    原來黛玉和湘雲二人並未去睡覺。只因黛玉見賈府中許多人賞月,賈母猶嘆人少,不似當年熱鬧,又提寶釵姊妹家去母女弟兄自去賞月等語,不覺對景感懷,自去俯欄垂淚。寶玉近因晴雯病勢甚重,諸務無心,\begin{note}庚雙夾:帶一筆,妙!更覺謹密不漏。\end{note}王夫人再四遣他去睡,他也便去了。探春又因近日家事著惱,無暇遊玩。雖有迎春惜春二人,偏又素日不大甚合。所以只剩了湘雲一人寬慰他,因說:“你是個明白人,何必作此形像自苦。我也和你一樣,我就不似你這樣心窄。何況你又多病,還不自己保養。可恨寶姐姐,姊妹天天說親道熱,早已說今年中秋要大家一處賞月,必要起社,大家聯句,到今日便棄了咱們,自己賞月去了。社也散了,詩也不作了。倒是他們父子叔侄縱橫起來。你可知宋太祖說的好:‘臥榻之側,豈許他人酣睡。’他們不作,咱們兩個竟聯起句來,明日羞他們一羞。”黛玉見他這般勸慰,不肯負他的豪興,因笑道:“你看這裏這等人聲嘈雜,有何詩興。”湘雲笑道:“這山上賞月雖好,終不及近水賞月更妙。你知道這山坡底下就是池沿,山坳裏近水一個所在就是凹晶館。可知當日蓋這園子時就有學問。這山之高處,就叫凸碧;山之低窪近水處,就叫作凹晶。這‘凸’‘凹’二字,歷來用的人最少。如今直用作軒館之名,更覺新鮮,不落窠臼。可知這兩處一上一下,一明一暗,一高一矮,一山一水,竟是特因玩月而設此處。有愛那山高月小的,便往這裏來;有愛那皓月清波的,便往那裏去。只是這兩個字俗念作‘窪’‘拱’二音,便說俗了,不大見用,只陸放翁用了一個‘凹’字,說‘古硯微凹聚墨多’,還有人批他俗,豈不可笑。”林黛玉道:“也不只放翁才用,古人中用者太多。如江淹《青苔賦》,東方朔《神異經》,以至《畫記》上雲張僧繇畫一乘寺的故事,不可勝舉。只是今人不知,誤作俗字用了。實和你說罷,這兩個字還是我擬的呢。因那年試寶玉,因他擬了幾處,也有存的,也有刪改的,也有尚未擬的。這是後來我們大家把這沒有名色的也都擬出來了,注了出處,寫了這房屋的坐落,一併帶進去與大姐姐瞧了。他又帶出來,命給舅舅瞧過。誰知舅舅倒喜歡起來,又說:‘早知這樣,那日該就叫他姊妹一併擬了,豈不有趣。’所以凡我擬的,一字不改都用了。如今就往凹晶館去看看。”
\end{parag}


\begin{parag}
    說著,二人便同下了山坡。只一轉彎,就是池沿,沿上一帶竹欄相接,直通著那邊藕香榭的路徑。\begin{note}庚雙夾:點明妙!不然此園竟有多大地畝了。\end{note}因這幾間就在此山懷抱之中,乃凸碧山莊之退居,因窪而近水,故顏其額曰“凹晶溪館”。因此處房宇不多,且又矮小,故只有兩個老婆子上夜。今日打聽得凸碧山莊的人應差,與他們無干,這兩個老婆子關了月餅果品並犒賞的酒食來,二人喫得既醉且飽,早已息燈睡了。\begin{note}庚雙夾:妙極!此處又進一步寫法。如王夫人云“他姊妹可憐,那裏像當日林姑媽那樣”,又如賈母雲“如今人少,當日人多”等數語,此爲進一步法也。也有退一步法,如寶釵之對邢岫煙雲“此一時也,彼一時也,如今比不得先的話了,只好隨事適分”,又如鳳姐之對平兒雲“如今我也明白了,我如今也要作好好先生罷”等類,此爲退一步法也。今有方收拾,故賈母高樂卻又寫出二婆子高樂,此進一步之實事也。如前文海棠詩四首已足,忽又用湘雲獨成二律反壓卷,此又進一步之實事也。所謂“法法皆全,絲絲不爽”也。\end{note}
\end{parag}


\begin{parag}
    黛玉湘雲見息了燈,湘雲笑道:“倒是他們睡了好。咱們就在這捲棚底下近水賞月如何?”二人遂在兩個湘妃竹墩上坐下。只見天上一輪皓月,池中一輪水月,上下爭輝,如置身於晶宮鮫室之內。微風一過,粼粼然池面皺碧鋪紋,真令人神清氣淨。湘雲笑道:“怎得這會子坐上船喫酒倒好。這要是我家裏這樣,我就立刻坐船了。”黛玉笑道:“正是古人常說的好,‘事若求全何所樂’。據我說,這也罷了,偏要坐船起來。”湘雲笑道:“得隴望蜀,人之常情。可知那些老人家說的不錯。說貧窮之家自爲富貴之家事事趁心,告訴他說竟不能遂心,他們不肯信的;必得親歷其境,他方知覺了。就如咱們兩個,雖父母不在,然卻也忝在富貴之鄉,只你我竟有許多不遂心的事。”黛玉笑道:“不但你我不能趁心,就連老太太,太太以至寶玉探丫頭等人,無論事大事小,有理無理,其不能各遂其心者,同一理也,何況你我旅居客寄之人哉!”\begin{note}庚雙夾:以立未不怡然得享自然之樂者矣。書中若干女子從生及婢未必有,各有所覺、各有所恃、各有所長者皆未如寶寶無可關切籌劃,可嘆。\end{note}湘雲聽說,恐怕黛玉又傷感起來,忙道:“休說這些閒話,咱們且聯詩。”
\end{parag}


\begin{parag}
    正說間,只聽笛韻悠揚起來。黛玉笑道:“今日老太太、太太高興了,這笛子吹的有趣,到是助咱們的興趣了。\begin{note}庚雙夾:妙!正是吹笛之時分,認作一處之笛也。\end{note}咱兩個都愛五言,就還是五言排律罷。”湘雲道:“限何韻?”黛玉笑道:“咱們數這個欄杆的直棍,這頭到那頭爲止。他是第幾根就用第幾韻。若十六根,便是‘一先’起。這可新鮮?”湘雲笑道:“這倒別緻。”於是二人起身,便從頭數至盡頭,止得十三根。湘雲道:“偏又是‘十三元’了。這韻少,作排律只怕牽強不能押韻呢。少不得你先起一句罷了。”黛玉笑道:“倒要試試咱們誰強誰弱,只是沒有紙筆記。”湘雲道:“不妨,明兒再寫。只怕這一點聰明還有。” 黛玉道:“我先起一句現成的俗語罷。”因念道:
\end{parag}


\begin{poem}
    \begin{pl}三五中秋夕,\end{pl}
\end{poem}


\begin{parag}
    湘雲想了一想,道:
\end{parag}


\begin{poem}
    \begin{pl} 清遊擬上元。撒天箕斗燦,\end{pl}
\end{poem}


\begin{parag}
    林黛玉笑道:
\end{parag}


\begin{poem}
    \begin{pl}匝地管絃繁。幾處狂飛盞,\end{pl}
\end{poem}


\begin{parag}
    湘雲笑道:“這一句‘幾處狂飛盞’有些意思。這倒要對的好呢。”想了一想,笑道:
\end{parag}


\begin{poem}
    \begin{pl}誰家不啓軒。輕寒風剪剪,\end{pl}
\end{poem}


\begin{parag}
    黛玉道:“對的比我的卻好。只是底下這句又說熟話了,就該加勁說了去纔是。”湘雲道:“詩多韻險,也要鋪陳些纔是。縱有好的,且留在後頭。”黛玉笑道:“到後頭沒有好的,我看你羞不羞。”因聯道:
\end{parag}


\begin{poem}
    \begin{pl}良夜景暄暄。爭餅嘲黃髮,\end{pl}
\end{poem}


\begin{parag}
    湘雲笑道:“這句不好,是你杜撰,用俗事來難我了。”黛玉笑道:“我說你不曾見過書呢。喫餅是舊典,唐書唐志你看了來再說。”湘雲笑道:“這也難不倒我,我也有了。”因聯道:
\end{parag}


\begin{poem}
    \begin{pl}分瓜笑綠嬡。香新榮玉桂,\end{pl}
\end{poem}


\begin{parag}
    黛玉笑道:“分瓜可是實實的你杜撰了。”湘雲笑道:“明日咱們對查了出來大家看看,這會子別耽誤工夫。”黛玉笑道:“雖如此,下句也不好,不犯著又用‘玉桂’‘金蘭’等字樣來塞責。”因聯道:
\end{parag}


\begin{poem}
    \begin{pl}色健茂金萱。蠟燭輝瓊宴,\end{pl}
\end{poem}


\begin{parag}
    湘雲笑道:“‘金萱’二字便宜了你,省了多少力。這樣現成的韻被你得了,只是不犯著替他們頌聖去。況且下句你也是塞責了。”黛玉笑道:“你不說‘玉桂’,我難道強對個‘金萱’麼?再也要鋪陳些富麗,方纔是即景之實事。”湘雲只得又聯道:
\end{parag}


\begin{poem}
    \begin{pl}觥籌亂綺園。分曹尊一令,\end{pl}
\end{poem}


\begin{parag}
    黛玉笑道:“下句好,只是難對些。”因想了一想,聯道:
\end{parag}


\begin{poem}
    \begin{pl}射覆聽三宣。骰彩紅成點,\end{pl}
\end{poem}


\begin{parag}
    湘雲笑道:“‘三宣’有趣,竟化俗成雅了。只是下句又說上骰子。”少不得聯道:
\end{parag}


\begin{poem}
    \begin{pl}傳花鼓濫喧。晴光搖院宇,\end{pl}
\end{poem}


\begin{parag}
    黛玉笑道:“對的卻好。下句又溜了,只管拿些風月來塞責。”湘雲道:“究竟沒說到月上,也要點綴點綴,方不落題。”黛玉道:“且姑存之,明日再斟酌。”因聯道:
\end{parag}


\begin{poem}
    \begin{pl}素彩接乾坤。賞罰無賓主,\end{pl}
\end{poem}


\begin{parag}
    湘雲道:“又說他們作什麼,不如說咱們。”只得聯道:
\end{parag}


\begin{poem}
    \begin{pl}吟詩序仲昆。構思時倚檻,\end{pl}
\end{poem}


\begin{parag}
    黛玉道:“這可以入上你我了。”因聯道:
\end{parag}


\begin{poem}
    \begin{pl}擬景或依門。酒盡情猶在,\end{pl}
\end{poem}


\begin{parag}
    湘雲說道:“是時侯了。”乃聯道:
\end{parag}


\begin{poem}
    \begin{pl}更殘樂已諼。漸聞語笑寂,\end{pl}
\end{poem}


\begin{parag}
    黛玉說道:“這時侯可知一步難似一步了。”因聯道:
\end{parag}


\begin{poem}
    \begin{pl}空剩雪霜痕。階露團朝菌,\end{pl}
\end{poem}


\begin{parag}
    湘雲笑道:“這一句怎麼押韻,讓我想想。”因起身負手,想了一想,笑道:“夠了,幸而想出一個字來,幾乎敗了。”因聯道:
\end{parag}


\begin{poem}
    \begin{pl}庭煙斂夕棔。秋湍瀉石髓,\end{pl}
\end{poem}


\begin{parag}
    黛玉聽了,不禁也起身叫妙,說:“這促狹鬼,果然留下好的。這會子才說‘棔’字,虧你想得出。”湘雲道:“幸而昨日看歷朝文選見了這個字,我不知是何樹,因要查一查。寶姐姐說不用查,這就是如今俗叫作明開夜合的。我信不及,到底查了一查,果然不錯。看來寶姐姐知道的竟多。”黛玉笑道:“‘棔’字用在此時更恰,也還罷了。只是‘秋湍’一句虧你好想。只這一句,別的都要抹倒。我少不得打起精神來對一句,只是再不能似這一句了。”因想了一想,道:
\end{parag}


\begin{poem}
    \begin{pl}風葉聚雲根。寶婺情孤潔,\end{pl}
\end{poem}


\begin{parag}
    湘雲道:“這對的也還好。只是下一句你也溜了,幸而是景中情,不單用‘寶婺’來塞責。”因聯道:
\end{parag}


\begin{poem}
    \begin{pl}銀蟾氣吐吞。藥經靈兔搗,\end{pl}
\end{poem}


\begin{parag}
    黛玉不語點頭,半日隨念道:
\end{parag}


\begin{poem}
    \begin{pl}人向廣寒奔。犯鬥邀牛女,\end{pl}
\end{poem}


\begin{parag}
    湘雲也望月點首,聯道:
\end{parag}


\begin{poem}
    \begin{pl}乘槎待帝孫。虛盈輪莫定,\end{pl}
\end{poem}


\begin{parag}
    黛玉笑道:“又用比興了。”因聯道:
\end{parag}


\begin{poem}
    \begin{pl}晦朔魄空存。壺漏聲將涸,\end{pl}
\end{poem}


\begin{parag}
    湘雲方欲聯時,黛玉指池中黑影與湘雲看道:“你看那河裏怎麼象個人在黑影裏去了,敢是個鬼罷?”湘雲笑道:“可是又見鬼了。我是不怕鬼的,等我打他一下。”因彎腰拾了一塊小石片向那池中打去,只聽打得水響,一個大圓圈將月影蕩散復聚者幾次。\begin{note}庚雙夾:寫得出。試思若非親歷其境者如何摹寫得如此。\end{note}只聽那黑影裏嘎然一聲,卻飛起一個大白鶴來,\begin{note}庚雙夾:寫得出。\end{note}直往藕香榭去了。黛玉笑道:“原來是他,猛然想不到,反嚇了一跳。”湘雲笑道:“這個鶴有趣,倒助了我了。”因聯道:
\end{parag}


\begin{poem}
    \begin{pl}窗燈焰已昏。寒塘渡鶴影,\end{pl}
\end{poem}


\begin{parag}
    林黛玉聽了,又叫好,又跺足,說:“了不得,這鶴真是助他的了!這一句更比‘秋湍’不同,叫我對什麼纔好?‘影’字只有一個‘魂’字可對,況且‘寒塘渡鶴’何等自然,何等現成,何等有景且又新鮮,我竟要擱筆了。”湘雲笑道:“大家細想就有了,不然就放著明日再聯也可。”黛玉只看天,不理他,半日,猛然笑道:“你不必說嘴,我也有了,你聽聽。”因對道:
\end{parag}


\begin{poem}
    \begin{pl}冷月葬花魂。\end{pl}
\end{poem}


\begin{parag}
    湘雲拍手讚道:“果然好極!非此不能對。好個‘葬花魂’!”因又嘆道:“詩固新奇,只是太頹喪了些。你現病著,不該作此過於清奇詭譎之語。”黛玉笑道:“不如此如何壓倒你。下句竟還未得,只爲用工在這一句了。”
\end{parag}


\begin{parag}
    一語未了,只見欄外山石後轉出一個人來,笑道:“好詩,好詩,果然太悲涼了。不必再往下聯,若底下只這樣去,反不顯這兩句了,倒覺得堆砌牽強。”二人不防,倒唬了一跳。細看時,不是別人,卻是妙玉。二人皆詫異,\begin{note}庚雙夾:原可詫異,餘亦詫異。\end{note}因問:“你如何到了這裏?”妙玉笑道:“我聽見你們大家賞月,又吹的好笛,我也出來玩賞這清池皓月。順腳走到這裏,忽聽見你兩個聯詩,更覺清雅異常,故此聽住了。只是方纔我聽見這一首中,有幾句雖好,只是過於頹敗悽楚。此亦關人之氣數而有,所以我出來止住。如今老太太都已早散了,滿園的人想俱已睡熟了,你兩個的丫頭還不知在那裏找你們呢。你們也不怕冷了?快同我來,到我那裏去喫杯茶,只怕就天亮了。”黛玉笑道:“誰知道就這個時侯了。”
\end{parag}


\begin{parag}
    三人遂一同來至櫳翠庵中。只見龕焰猶青,爐香未燼。幾個老嬤嬤也都睡了,只有小丫鬟在蒲團上垂頭打盹。妙玉喚他起來,現去烹茶。忽聽叩門之聲,小丫鬟忙去開門看時,卻是紫鵑翠縷與幾個老嬤嬤來找他姊妹兩個。進來見他們正喫茶,因都笑道:“要我們好找,一個園裏走遍了,連姨太太那裏都找到了。纔到了那山坡底下小亭裏找時,可巧那裏上夜的正睡醒了。我們問他們,他們說,方纔亭外頭棚下兩個人說話,後來又添了一個,聽見說大家往庵裏去。我們就知是這裏了。” 妙玉忙命小丫鬟引他們到那邊去坐著歇息喫茶。自取了筆硯紙墨出來,將方纔的詩命他二人念著,遂從頭寫出來。黛玉見他今日十分高興,便笑道:“從來沒見你這樣高興。我也不敢唐突請教,這還可以見教否?若不堪時,便就燒了;若或可政,即請改正改正。”妙玉笑道:“也不敢妄加評贊。只是這纔有了二十二韻。我意思想著你二位警句已出,再若續時,恐後力不加。我竟要續貂,又恐有玷。”黛玉從沒見妙玉作過詩,今見他高興如此,忙說:“果然如此,我們的雖不好,亦可以帶好了。”妙玉道:“如今收結,到底還該歸到本來面目上去。若只管丟了真情真事且去搜奇撿怪,一則失了咱們的閨閣面目,二則也與題目無涉了。”二人皆道極是。妙玉遂提筆一揮而就,遞與他二人道:“休要見笑。依我必須如此,方翻轉過來,雖前頭有悽楚之句,亦無甚礙了。”二人接了看時,只見他續道:
\end{parag}


\begin{poem}
    \begin{pl}香篆銷金鼎,脂冰膩玉盆。\end{pl}

    \begin{pl}簫增嫠婦泣,衾倩侍兒溫。\end{pl}

    \begin{pl}空帳懸文鳳,閒屏掩彩鴛。\end{pl}

    \begin{pl}露濃苔更滑,霜重竹難捫。\end{pl}

    \begin{pl}猶步縈紆沼,還登寂歷原。\end{pl}

    \begin{pl}石奇神鬼搏,木怪虎狼蹲。\end{pl}

    \begin{pl}贔屓朝光透,罘罳曉露屯。\end{pl}

    \begin{pl}振林千樹鳥,啼谷一聲猿。\end{pl}

    \begin{pl}歧熟焉忘徑,泉知不問源。\end{pl}

    \begin{pl}鐘鳴櫳翠寺,雞唱稻香村。\end{pl}

    \begin{pl}有興悲何繼,無愁意豈煩。\end{pl}

    \begin{pl}芳情只自遣,雅趣向誰言。\end{pl}

    \begin{pl}徹旦休雲倦,烹茶更細論。\end{pl}

\end{poem}


\begin{parag}
    後書:《右中秋夜大觀園即景聯句三十五韻》。
\end{parag}


\begin{parag}
    黛玉湘雲二人皆讚賞不已,說:“可見我們天天是舍近而求遠。現有這樣詩仙在此,卻天天去紙上談兵。”妙玉笑道:“明日再潤色。此時想也快天亮了,到底要歇息歇息纔是。”林史二人聽說,便起身告辭,帶領丫鬟出來。妙玉送至門外,看他們去遠,方掩門進來。不在話下。
\end{parag}


\begin{parag}
    這裏翠縷向湘雲道:“大奶奶那裏還有人等著咱們睡去呢。如今還是那裏去好?”湘雲笑道:“你順路告訴他們,叫他們睡罷。我這一去未免驚動病人,不如鬧林姑娘半夜去罷。”說著,大家走至瀟湘館中,有一半人已睡去。二人進去,方纔卸妝寬衣,盥漱已畢,方上牀安歇。紫鵑放下綃帳,移燈掩門出去。誰知湘雲有擇席之病,雖在枕上,只是睡不著。黛玉又是個心血不足常常失眠的,今日又錯過困頭,自然也是睡不著。二人在枕上翻來覆去。黛玉因問道:“怎麼你還沒睡著?” 湘雲微笑道:“我有擇席的病,況且走了困,只好躺躺罷。你怎麼也睡不著?”黛玉嘆道:\begin{note}庚雙夾:一“笑”一“嘆”,只二字便寫出平日之形景。\end{note}“我這睡不著也並非今日,大約一年之中,通共也只好睡十夜滿足的。”湘雲道:“都是你病的原故,所以……”不知下文什麼──
\end{parag}


\begin{parag}
    \begin{note}蒙回末總評:詩詞清遠閒曠,自是慧業才人,何須贅評?須看他衆人聯句填詞時,個人性情,個人意見,敘來恰肖其人;二人聯詩時,一番譏評,一番賞嘆,敘來更得其神。再看漏永吟殘,忽開一洞天福地,字字出人意表。\end{note}
\end{parag}


\begin{parag}
    \begin{note}蒙回末總評:只一品笛,疑有疑無,若近若遠,有無限逸緻。\end{note}
\end{parag}