\chap{七十九}{薛文龍悔娶河東獅 賈迎春誤嫁中山狼}

\begin{parag}
    \begin{note}蒙回前總:靜含天地自寬,動盪吉凶難定,一喙一飲系生成,何必夢中說醒。\end{note}
\end{parag}


\begin{parag}
    話說寶玉祭完了晴雯,只聽花影中有人聲,倒唬了一跳。走出來細看,不是別人,卻是林黛玉,滿面含笑,口內說道:“好新奇的祭文!可與曹娥碑並傳的了。” 寶玉聽了,不覺紅了臉,笑答道:“我想著世上這些祭文都蹈於熟濫了,所以改個新樣,原不過是我一時的頑意,誰知又被你聽見了。有什麼大使不得的,何不改削改削。”黛玉道:“原稿在那裏?倒要細細一讀。長篇大論,不知說的是什麼,只聽見中間兩句,什麼‘紅綃帳裏,公子多情,黃土壟中,女兒薄命。’這一聯意思卻好,只是‘紅綃帳裏’未免熟濫些。放著現成真事,爲什麼不用?”寶玉忙問:“什麼現成的真事?”黛玉笑道:“咱們如今都系霞影紗糊的窗槅,何不說‘茜紗窗下,公子多情’呢?”寶玉聽了,不禁跌足笑道:“好極,是極!到底是你想的出,說的出。可知天下古今現成的好景妙事盡多,只是愚人蠢子說不出想不出罷了。但只一件:雖然這一改新妙之極,但你居此則可,在我實不敢當。”說著,又接連說了一二十句“不敢”。黛玉笑道:“何妨。我的窗即可爲你之窗,何必分晰得如此生疏。古人異姓陌路,尚然同肥馬,衣輕裘,敝之而無憾,何況咱們。”寶玉笑道:“論交之道,不在肥馬輕裘,即黃金白璧,亦不當錙銖較量。倒是這唐突閨閣,萬萬使不得的。如今我越性將‘公子’‘女兒’改去,竟算是你誄他的倒妙。況且素日你又待他甚厚,故今寧可棄此一篇大文,萬不可棄此‘茜紗’新句。竟莫若改作‘茜紗窗下,小姐多情,黃土壟中,丫鬟薄命。’如此一改,雖於我無涉,我也是愜懷的。”黛玉笑道:“他又不是我的丫頭,何用作此語。況且小姐丫鬟亦不典雅,等我的紫鵑死了,我再如此說,還不算遲。”\begin{note}庚雙夾:明是爲與阿顰作讖,卻先偏說紫鵑,總用此狡猾之法。\end{note}寶玉聽了,忙笑道:“這是何苦又咒他。”\begin{note}庚雙夾:又畫出寶玉來,究竟不知是咒誰,使人一笑一嘆。\end{note}黛玉笑道:“是你要咒的,並不是我說的。”寶玉道:“我又有了,這一改可妥當了。莫若說:‘茜紗窗下,我本無緣;\begin{note}庚雙夾:雙關句,意妥極。\end{note}黃土壟中,卿何薄命。’”\begin{note}庚雙夾:如此我亦爲妥極。但試問當面用“爾” “我”字樣究竟不知是爲誰之讖,一笑一嘆。一篇誄文總因此二句而有,又當知雖晴雯而又實誄黛玉也。奇幻至此!若雲必因晴雯誄,則呆之至矣。\end{note}黛玉聽了,忡然變色,\begin{note}庚雙夾:慧心人可爲一哭。觀此句便知誄文實不爲晴雯而作也。\end{note}心中雖有無限的狐疑亂擬,\begin{note}庚雙夾:用此事更妙,蓋又欲瞞觀者。\end{note}外面卻不肯露出,反連忙含笑點頭稱妙,說:“果然改的好。再不必亂改了,快去幹正經事罷。纔剛太太打發人叫你明兒一早快過大舅母那邊去。你二姐姐已有人家求準了,想是明兒那家人來拜允,所以叫你們過去呢。”寶玉拍手道:“何必如此忙?我身上也不大好,明兒還未必能去呢。”黛玉道:“又來了,我勸你把脾氣改改罷。一年大二年小,……”一面說話,一面咳嗽起來。\begin{note}庚雙夾:總爲後文伏筆。阿顰之文可見不是一筆兩筆所寫。\end{note}寶玉忙道:“這裏風冷,咱們只顧呆站在這裏,快回去罷。”黛玉道:“我也家去歇息了,明兒再見罷。”說著,便自取路去了。寶玉只得悶悶的轉步,又忽想起來黛玉無人隨伴,忙命小丫頭子跟了送回去。自己到了怡紅院中,果有王夫人打發老嬤嬤來,吩咐他明日一早過賈赦那邊去,與方纔黛玉之言相對。
\end{parag}


\begin{parag}
    原來賈赦已將迎春許與孫家了。這孫家乃是大同府人氏,\begin{note}庚雙夾:設雲“大概相同”也,若必雲真大同府則呆。\end{note}祖上系軍官出身,乃當日寧榮府中之門生,算來亦繫世交。如今孫家只有一人在京,現襲指揮之職,此人名喚孫紹祖,生得相貌魁梧,體格健壯,弓馬嫺熟,應酬權變,\begin{note}庚雙夾:畫出一個俗物來。\end{note}年紀未滿三十,且又家資饒富,\begin{note}庚雙夾:此句斷不可少。\end{note}現在兵部候缺題升。因未有室,賈赦見是世交之孫,且人品家當都相稱合,遂青目擇爲東牀嬌婿。亦曾回明賈母。賈母心中卻不十分稱意,想來攔阻亦恐不聽,兒女之事自有天意前因,況且他是親父主張,何必出頭多事,爲此只說“知道了”三字,餘不多及。賈政又深惡孫家,雖是世交,當年不過是彼祖希慕榮寧之勢,有不能了結之事才拜在門下的,並非詩禮名族之裔,因此倒勸諫過兩次,無奈賈赦不聽,也只得罷了。
\end{parag}


\begin{parag}
    寶玉卻從未會過這孫紹祖一面的,次日只得過去聊以塞責。只聽見說娶親的日子甚急,不過今年就要過門的,又見邢夫人等回了賈母將迎春接出大觀園去等事,越發掃去了興頭,每日癡癡呆呆的,不知作何消遣。又聽得說陪四個丫頭過去,更又跌足自嘆道:“從今後這世上又少了五個清潔人了。”因此天天到紫菱洲一帶地方徘徊瞻顧,見其軒窗寂寞,屏帳翛然,不過有幾個該班上夜的老嫗。\begin{note}庚雙夾:先爲對 暗顰兒作引。\end{note}再看那岸上的蓼花葦葉,池內的翠荇香菱,也都覺搖搖落落,似有追憶故人之態,迥非素常逞妍鬥色之可比。既領略得如此寥落悽慘之景,是以情不自禁,乃信口吟成一歌曰:\begin{note}庚雙夾:此回題上半截是“悔娶河東獅”,今卻偏連“中山狼”倒裝業下情工細下賦寫來。\end{note}\begin{note}可見迎春是書中正傳,阿呆夫妻是副,賓主次序嚴肅之至。其婚娶俗禮一概不及,只用寶玉一人過去,正是書中之大節。\end{note}
\end{parag}


\begin{poem}
    \begin{pl}池塘一夜秋風冷,吹散芰荷紅玉影。\end{pl}

    \begin{pl}蓼花菱葉不勝愁,重露繁霜壓纖梗。\end{pl}

    \begin{pl}不聞永晝敲棋聲,燕泥點點污棋枰。\end{pl}

    \begin{pl}古人惜別憐朋友,況我今當手足情!\end{pl}

\end{poem}


\begin{parag}
    寶玉方纔吟罷,忽聞背後有人笑道:“你又發什麼呆呢?”寶玉回頭忙看是誰,原來是香菱。寶玉便轉身笑問道:“我的姐姐,你這會子跑到這裏來做什麼?許多日子也不進來逛逛。”香菱拍手笑嘻嘻的說道:“我何曾不來。如今你哥哥回來了,那裏比先時自由自在的了。纔剛我們奶奶使人找你鳳姐姐的,竟沒找著,說往園子裏來了。我聽見了這信,我就討了這件差進來找他。遇見他的丫頭,說在稻香村呢。如今我往稻香村去,誰知又遇見了你。我且問你,襲人姐姐這幾日可好?怎麼忽然把個晴雯姐姐也沒了,到底是什麼病?二姑娘搬出去的好快,你瞧瞧這地方好空落落的。”寶玉應之不迭,又讓他同到怡紅院去喫茶。\begin{note}庚雙夾:斷不可少。\end{note}香菱道:“此刻竟不能,等找著璉二奶奶,說完了正經事再來。”寶玉道:“什麼正經事這麼忙?”香菱道:“爲你哥哥娶嫂子的事,所以要緊。”\begin{note}庚雙夾:出題卻閒閒引出。\end{note}寶玉道:“正是。說的到底是那一家的?只聽見吵嚷了這半年,今兒又說張家的好,明兒又要李家的,後兒又議論王家的。這些人家的女兒他也不知道造了什麼罪了,叫人家好端端議論。”香菱道:“這如今定了,可以不用搬扯別家了。”寶玉忙問:“定了誰家的?”香菱道:“因你哥哥上次出門貿易時,在順路到了個親戚家去。這門親原是老親,且又和我們是同在戶部掛名行商,也是數一數二的大門戶。前日說起來,你們兩府都也知道的。合長安城中,上至王侯,下至買賣人,都稱他家是‘桂花夏家’。”\begin{note}庚雙夾:夏日何得有桂?又桂花時節焉得又有雪?三事原系風馬牛,全若強湊合,故終不相符。運敗之事大都如此,當事者自不解耳。\end{note}寶玉笑問道:\begin{note}庚雙夾:聽得“桂花”字號原覺新雅,故不覺一笑,餘亦欲笑。\end{note}“如何又稱爲‘桂花夏家’?”香菱道:“他家本姓夏,非常的富貴。其餘田地不用說,單有幾十頃地獨種桂花,凡這長安城裏城外桂花局俱是他家的,連宮裏一應陳設盆景亦是他家貢奉,因此纔有這個渾號。如今太爺也沒了,只有老奶奶帶著一個親生的姑娘過活,也並沒有哥兒兄弟,可惜他竟一門盡絕了。”寶玉忙道:“咱們也別管他絕後不絕後,只是這姑娘可好?你們大爺怎麼就中意了?”\begin{note}庚雙夾:補出阿呆素日難得中意來。\end{note}香菱笑道:“一則是天緣,二則是‘情人眼裏出西施’。當年又是通家來往,從小兒都一處廝混過。敘起親是姑舅兄妹,又沒嫌疑。雖離開了這幾年,前兒一到他家,夏奶奶又是沒兒子的,一見了你哥哥出落的這樣,又是哭,又是笑,竟比見了兒子的還勝。又令他兄妹相見,誰知這姑娘出落得花朵似的了,在家裏也讀書寫字,所以你哥哥當時就一心看準了。連當鋪里老朝奉夥計們一羣人蹧擾了人家三四日,他們還留多住幾日,好容易苦辭才放回家。你哥哥一進門,就咕咕唧唧求我們奶奶去求親。我們奶奶原也是見過這姑娘的,且又門當戶對,也就依了。和這裏姨太太鳳姑娘商議了,打發人去一說就成了。只是娶的日子太急,所以我們忙亂的很。\begin{note}庚雙夾:阿呆求婦一段文字卻從香菱口中補明,省卻多少閒文累筆。\end{note}我也巴不得早些過來,又添一個作詩的人了。”\begin{note}庚雙夾:妙極!香菱口聲,斷不可少。看他下作死語,便知其心中略無忌諱疑慮等意,直是渾然天真之人,餘爲一哭。\end{note}寶玉冷笑道:\begin{note}庚雙夾:忽曰“冷笑”,二字便有文章。\end{note}“雖如此說,但只我聽這話不知怎麼倒替你耽心慮後呢。”\begin{note}庚雙夾:又爲香菱之讖,偏是此等事體等到。\end{note}香菱聽了,不覺紅了臉,正色道:“這是什麼話!素日咱們都是廝抬廝敬的,今日忽然提起這些事來,是什麼意思!怪不得人人都說你是個親近不得的人。”一面說,一面轉身走了。
\end{parag}


\begin{parag}
    寶玉見他這樣,便悵然如有所失,呆呆的站了半天,思前想後,不覺滴下淚來,只得沒精打彩,還入怡紅院來。一夜不曾安穩,睡夢之中猶喚晴雯,或魘魔驚怖,種種不寧。次日便懶進飲食,身體作熱。此皆近日抄檢大觀園、逐司棋、別迎春、悲晴雯等羞辱驚恐悲悽之所致,兼以風寒外感,故釀成一疾,臥牀不起。賈母聽得如此,天天親來看視。王夫人心中自悔不合因晴雯過於逼責了他。心中雖如此,臉上卻不露出。只吩咐衆奶孃等好生伏侍看守,一日兩次帶進醫生來診脈下藥。一月之後,方纔漸漸的痊癒。賈母命好生保養,過百日方許動葷腥油麪等物,方可出門行走。這一百日內,連院門前皆不許到,只在房中頑笑。四五十日後,就把他拘約的火星亂迸,那裏忍耐得住。雖百般設法,無奈賈母王夫人執意不從,也只得罷了。因此和那些丫鬟們無所不至,恣意耍笑作戲。又聽得薛蟠擺酒唱戲,熱鬧非常,已娶親入門,聞得這夏家小姐十分俊俏,也略通文翰,寶玉恨不得就過去一見纔好。再過些時,又聞得迎春出了閣。寶玉思及當時姊妹們一處,耳鬢廝磨,從今一別,縱得相逢,也必不似先前那等親密了。眼前又不能去一望,真令人悽惶迫切之至。少不得潛心忍耐,暫同這些丫鬟們廝鬧釋悶,倖免賈政責備逼迫讀書之難。這百日內,只不曾拆毀了怡紅院,和這些丫頭們無法無天,凡世上所無之事,都頑耍出來。如今且不消細說。
\end{parag}


\begin{parag}
    且說香菱自那日搶白了寶玉之後,心中自爲寶玉有意唐突他,“怨不得我們寶姑娘不敢親近,可見我不如寶姑娘遠矣;怨不得林姑娘時常和他角口氣的痛哭,自然唐突他也是有的了。從此倒要遠避他纔好。” 因此,以後連大觀園也不輕易進來。日日忙亂著,薛蟠娶過親,自爲得了護身符,自己身上分去責任,到底比這樣安寧些;二則又聞得是個有才有貌的佳人,自然是典雅和平的:因此他心中盼過門的日子比薛蟠還急十倍。好容易盼得一日娶過了門,他便十分殷勤小心伏侍。
\end{parag}


\begin{parag}
    原來這夏家小姐今年方十七歲,生得亦頗有姿色,亦頗識得幾個字。若論心中的邱壑經緯,頗步熙鳳之後塵。只吃虧了一件,從小時父親去世的早,又無同胞弟兄,寡母獨守此女,嬌養溺愛,不啻珍寶,凡女兒一舉一動,彼母皆百依百隨,因此未免嬌養太過,竟釀成個盜蹠的性氣。愛自己尊若菩薩,窺他人穢如糞土;外具花柳之姿,內秉風雷之性。在家中時常就和丫鬟們使性弄氣,輕罵重打的。今日出了閣,自爲要作當家的奶奶,比不得作女兒時靦腆溫柔,須要拿出這威風來,才鈐壓得住人;況且見薛蟠氣質剛硬,舉止驕奢,若不趁熱竈一氣炮製熟爛,將來必不能自豎旗幟矣;又見有香菱這等一個才貌俱全的愛妾在室,越發添了“宋太祖滅南唐”之意,“臥榻之側豈容他人酣睡”之心。因他家多桂花,他小名就喚做金桂。他在家時不許人口中帶出金桂二字來,凡有不留心誤道一字者,他便定要苦打重罰才罷。他因想桂花二字是禁止不住的,須另換一名,因想桂花曾有廣寒嫦娥之說,便將桂花改爲嫦娥花,又寓自己身分如此。
\end{parag}


\begin{parag}
    薛蟠本是個憐新棄舊的人,且是有酒膽無飯力的,如今得了這樣一個妻子,正在新鮮興頭上,凡事未免儘讓他些。那夏金桂見了這般形景,便也試著一步緊似一步。一月之中,二人氣概還都相平;至兩月之後,便覺薛蟠的氣概漸次低矮了下去。一日薛蟠酒後,不知要行何事,先與金桂商議,金桂執意不從。薛蟠忍不住便發了幾句話,賭氣自行了,這金桂便氣的哭如醉人一般,茶湯不進,裝起病來。請醫療治,醫生又說“氣血相逆,當進寬胸順氣之劑。”薛姨娘恨的罵了薛蟠一頓,說:“如今娶了親,眼前抱兒子了,還是這樣胡鬧。人家鳳凰蛋似的,好容易養了一個女兒,比花朵兒還輕巧,原看的你是個人物,纔給你作老婆。你不說收了心安分守己,一心一計和和氣氣的過日子,還是這樣胡鬧,撞嗓了黃湯,折磨人家。這會子花錢吃藥白遭心。”一席話說的薛蟠後悔不迭,反來安慰金桂。金桂見婆婆如此說丈夫,越發得了意,便裝出些張致來,總不理薛蟠。薛蟠沒了主意,惟自怨而已,好容易十天半月之後,才漸漸的哄轉過金桂的心來,自此便加一倍小心,不免氣概又矮了半截下來。那金桂見丈夫旗纛漸倒,婆婆良善,也就漸漸的持戈試馬起來。先時不過挾制薛蟠,後來倚嬌作媚,將及薛姨媽,又將至薛寶釵。寶釵久察其不軌之心,每隨機應變,暗以言語彈壓其志。金桂知其不可犯,每欲尋隙,又無隙可乘,只得曲意附就。一日金桂無事,因和香菱閒談,問香菱家鄉父母。香菱皆答忘記,金桂便不悅,說有意欺瞞了他。回問他“香菱”二字是誰起的名字,香菱便答:“姑娘起的。”金桂冷笑道:“人人都說姑娘通,只這一個名字就不通。”香菱忙笑道:“噯喲,奶奶不知道,我們姑娘的學問連我們姨老爺時常還誇呢。”欲明後事,且見下回。
\end{parag}


\begin{parag}
    \begin{note}蒙回末總:作誄後,黛玉飄然而至,增一番感慨,及說至迎春事,遂飄然而去。作詞後,香菱飄然而至,增一番感慨,及說至薛蟠事,遂飄然而去。一點一逗,爲下文引線。且二段俱以“正經事”三字作眼,而正經裏更有大不正經者在,文家固無一呆字死句。\end{note}
\end{parag}


\begin{parag}
    \begin{note}蒙回末總:從起名上設色,別有可玩。\end{note}
\end{parag}

