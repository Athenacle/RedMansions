\chap{八十}{懦弱迎春腸回九曲 姣怯香菱病入膏肓}

\begin{parag}
    \begin{note}蒙回前總:敘桂花妒用實筆,敘孫家惡用虛筆,敘寶玉病是省筆,敘寶玉燒香是停筆。\end{note}
\end{parag}


\begin{parag}
    話說金桂聽了,將脖項一扭,嘴脣一撇,\begin{note}庚雙夾:畫出一個悍婦來。\end{note}鼻孔裏哧了兩聲,\begin{note}庚雙夾:真真追魂攝魄之筆。\end{note}拍著掌冷笑道:“菱角花誰聞見香來著?若說菱角香了,正經那些香花放在那裏?可是不通之極!”香菱道:“不獨菱角花,就連荷葉蓮蓬,都是有一股清香的。但他那原不是花香可比,若靜日靜夜或清早半夜細領略了去,那一股香比是花兒都好聞呢。就連菱角、雞頭、葦葉、蘆根得了風露,那一股清香,就令人心神爽快的。”\begin{note}庚雙夾:說得出便是慧心人,何況菱卿哉?\end{note}金桂道:“依你說,那蘭花桂花倒香的不好了?”\begin{note}庚雙夾:又虛陪一個蘭花,一則是自高身價,二則是誘人犯法。\end{note}香菱說到熱鬧頭上,忘了忌諱,便接口道:“蘭花桂花的香,又非別花之香可比。”一句未完,金桂的丫鬟名喚寶蟾者,忙指著香菱的臉兒說道:“要死,要死!你怎麼真叫起姑娘的名字來!”香菱猛省了,反不好意思,忙陪笑賠罪說:“一時說順了嘴,奶奶別計較。”金桂笑道:“這有什麼,你也太小心了。但只是我想這個 ‘香’字到底不妥,意思要換一個字,不知你服不服?”香菱忙笑道:“奶奶說那裏話,此刻連我一身一體俱屬奶奶,何得換一名字反問我服不服,叫我如何當得起。奶奶說那一個字好,就用那一個。”金桂笑道:“你雖說的是,只怕姑娘多心,說:‘我起的名字,反不如你?你能來了幾日,就駁我的回了。’”香菱笑道: “奶奶有所不知,當日買了我來時,原是老奶奶使喚的,故此姑娘起得名字。後來我自伏侍了爺,就與姑娘無涉了。如今又有了奶奶,益發不與姑娘相干。況且姑娘又是極明白的人,如何惱得這些呢。”金桂道:“既這樣說,‘香’字竟不如‘秋’字妥當。菱角菱花皆盛於秋,豈不比‘香’字有來歷些。”香菱道:“就依奶奶這樣罷了。”自此後遂改了秋字,寶釵亦不在意。
\end{parag}


\begin{parag}
    只因薛蟠天性是“得隴望蜀”的,如今得娶了金桂,又見金桂的丫鬟寶蟾有三分姿色,舉止輕浮可愛,便時常要茶要水的故意撩逗他。寶蟾雖亦解事,只是怕著金桂,不敢造次,且看金桂的眼色。金桂亦頗覺察其意,想著:"正要擺佈香菱,無處尋隙,如今他既看上了寶蟾,如今且捨出寶蟾去與他,他一定就和香菱疏遠了,我且乘他疏遠之時,便擺佈了香菱。那時寶蟾原是我的人,也就好處了。"打定了主意,伺機而發。
\end{parag}


\begin{parag}
    這日薛蟠晚間微醺,又命寶蟾倒茶來喫。薛蟠接碗時,故意捏他的手。寶蟾又喬裝躲閃,連忙縮手。兩下失誤,豁啷一聲,茶碗落地,潑了一身一地的茶。薛蟠不好意思,佯說寶蟾不好生拿著。寶蟾說:“姑爺不好生接。”金桂冷笑道:“兩個人的腔調兒都夠使了。別打諒誰是傻子。”薛蟠低頭微笑不語,寶蟾紅了臉出去。一時安歇之時,金桂便故意的攆薛蟠別處去睡,“省得你饞癆餓眼。”薛蟠只是笑。金桂道:“要作什麼和我說,別偷偷摸摸的不中用。”薛蟠聽了,仗著酒蓋臉,便趁勢跪在被上拉著金桂笑道:“好姐姐,你若要把寶蟾賞了我,你要怎樣就怎樣。你要人腦子也弄來給你。”金桂笑道:“這話好不通。你愛誰,說明了,就收在房裏,省得別人看著不雅。我可要什麼呢。”薛蟠得了這話,喜的稱謝不盡,是夜曲盡丈夫之道,\begin{note}庚雙夾:“曲盡丈夫之道”,奇文奇語。\end{note}奉承金桂。次日也不出門,只在家中廝奈,越發放大了膽。
\end{parag}


\begin{parag}
    至午後,金桂故意出去,讓個空兒與他二人。薛蟠便拉拉扯扯的起來。寶蟾心裏也知八九,也就半推半就,正要入港。誰知金桂是有心等候的,料必在難分之際,便叫丫頭小舍兒過來。原來這小丫頭也是金桂從小兒在家使喚的,因他自幼父母雙亡,無人看管,便大家叫他作小舍兒,專作些粗笨的生活。\begin{note}庚雙夾:鋪敘小舍兒首尾,忙中又點“薄命”二字,與癡丫頭遙遙作對。\end{note}金桂如今有意獨喚他來吩咐道:“你去告訴秋菱,到我屋裏將手帕取來,不必說我說的。”\begin{note}庚雙夾:金桂壞極!所以獨使小舍爲此。\end{note}小舍兒聽了,一徑尋著香菱說:“菱姑娘,奶奶的手帕子忘記在屋裏了。你去取來送上去豈不好?”香菱正因金桂近日每每的折挫他,不知何意,百般竭力挽回不暇。\begin{note}庚雙夾:總爲癡心人一嘆。\end{note}聽了這話,忙往房裏來取。不防正遇見他二人推就之際,一頭撞了進去,自己倒羞的耳面飛紅,忙轉身迴避不迭。那薛蟠自爲是過了明路的,除了金桂,無人可怕,所以連門也不掩,今見香菱撞來,故也略有些慚愧,還不十分在意。無奈寶蟾素日最是說嘴要強的,今遇見了香菱,便恨無地縫兒可入,忙推開薛蟠,一徑跑了,口內還恨怨不迭,說他強姦力逼等語。薛蟠好容易圈哄的要上手,卻被香菱打散,不免一腔興頭變作了一腔惡怒,都在香菱身上,不容分說,趕出來啐了兩口,罵道:“死娼婦,你這會子作什麼來撞屍遊魂!”香菱料事不好,三步兩步早已跑了。薛蟠再來找寶蟾,已無蹤跡了,於是恨的只罵香菱。至晚飯後,已喫得醺醺然,洗澡時不防水略熱了些,燙了腳,便說香菱有意害他,赤條精光趕著香菱踢打了兩下。香菱雖未受過這氣苦,既到此時,也說不得了,只好自悲自怨,各自走開。
\end{parag}


\begin{parag}
    彼時金桂已暗和寶蟾說明,今夜令薛蟠和寶蟾在香菱房中去成親,命香菱過來陪自己先睡。先是香菱不肯,金桂說他嫌髒了,再必是圖安逸,怕夜裏勞動伏侍,又罵說:“你那沒見世面的主子,見一個,愛一個,把我的人霸佔了去,又不叫你來。到底是什麼主意,想必是逼我死罷了。”薛蟠聽了這話,又怕鬧黃了寶蟾之事,忙又趕來罵香菱:“不識抬舉!再不去便要打了!”香菱無奈,只得抱了鋪蓋來。金桂命他在地下鋪睡。香菱無奈,只得依命。剛睡下,便叫倒茶,一時又叫捶腿,如是一夜七八次,總不使其安逸穩臥片時。那薛蟠得了寶蟾,如獲珍寶,一概都置之不顧。恨的金桂暗暗的發恨道:“且叫你樂這幾天,等我慢慢的擺佈了來,那時可別怨我!”一面隱忍,一面設計擺佈香菱。
\end{parag}


\begin{parag}
    半月光景,忽又裝起病來,只說心疼難忍,四肢不能轉動。\begin{note}庚雙夾:半月工夫,諸計安矣。\end{note}請醫療治不效,衆人都說是香菱氣的。鬧了兩日,忽又從金桂的枕頭內抖出紙人來,上面寫著金桂的年庚八字,有五根針釘在心窩並四肢骨節等處。於是衆人反亂起來,當作新聞,先報與薛姨媽。薛姨媽先忙手忙腳的,薛蟠自然更亂起來,立刻要拷打衆人。金桂笑道:“何必冤枉衆人,大約是寶蟾的鎮魘法兒。”\begin{note}庚雙夾:惡極!壞極!\end{note}薛蟠道:“他這些時並沒多空兒在你房裏,何苦賴好人。”\begin{note}庚雙夾:正要老兄此句。\end{note}金桂冷笑道:“除了他還有誰,莫不是我自己不成!雖有別人,誰可敢進我的房呢。”薛蟠道:“香菱如今是天天跟著你,他自然知道,先拷問他就知道了。”金桂冷笑道:“拷問誰,誰肯認?依我說竟裝個不知道,大家丟開手罷了。橫豎治死我也沒什麼要緊,樂得再娶好的。若據良心上說,左不過你三個多嫌我一個。”說著,一面痛哭起來。薛蟠更被這一席話激怒,順手抓起一根門閂來,\begin{note}庚雙夾:與前要打死寶玉遙遙一對。\end{note}一徑搶步找著香菱,不容分說便劈頭劈面打起來,一口咬定是香菱所施。香菱叫屈,薛姨媽跑來禁喝說:“不問明白,你就打起人來了。這丫頭伏侍了你這幾年,那一點不周到,不盡心?他豈肯如今作這沒良心的事!你且問個清渾皁白,再動粗鹵。”金桂聽見他婆婆如此說著,怕薛蟠耳軟心活,便益發嚎啕大哭起來,一面又哭喊說:“這半個多月把我的寶蟾霸佔了去,不容他進我的房,唯有秋菱跟著我睡。我要拷問寶蟾,你又護到頭裏。你這會子又賭氣打他去。治死我,再揀富貴的標緻的娶來就是了,何苦作出這些把戲來!”薛蟠聽了這些話,越發著了急。薛姨媽聽見金桂句句挾制著兒子,百般惡賴的樣子,十分可恨。無奈兒子偏不硬氣,已是被他挾制軟慣了。如今又勾搭上丫頭,被他說霸佔了去,他自己反要佔溫柔讓夫之禮。這魘魔法究竟不知誰作的,實是俗語說的“清官難斷家務事”,此事正是公婆難斷牀幃事了。因此無法,只得賭氣喝罵薛蟠說:“不爭氣的孽障!騷狗也比你體面些!誰知你三不知的把陪房丫頭也摸索上了,叫老婆說嘴霸佔了丫頭,什麼臉出去見人!也不知誰使的法子,也不問青紅皁白,好歹就打人。我知道你是個得新棄舊的東西,白辜負了我當日的心。他既不好,你也不許打,我立即叫人牙子來賣了他,你就心淨了。”說著,命香菱“收拾了東西跟我來”,一面叫人去,“快叫個人牙子來,多少賣幾兩銀子,拔去肉中刺,眼中釘,大家過太平日子。” 薛蟠見母親動了氣,早也低下頭了。金桂聽了這話,便隔著窗子往外哭道:“你老人家只管賣人,不必說著一個扯著一個的。我們很是那喫醋拈酸容不下人的不成,怎麼‘拔出肉中刺,眼中釘’?是誰的釘,誰的刺?但凡多嫌著他,也不肯把我的丫頭也收在房裏了。”薛姨媽聽說,氣的身戰氣咽道:“這是誰家的規矩?婆婆這裏說話,媳婦隔著窗子拌嘴。虧你是舊家人家的女兒!滿嘴裏大呼小喊,說的是些什麼!”薛蟠急的跺腳說:“罷喲,罷喲!看人聽見笑話。"金桂意謂一不作,二不休,越發發潑喊起來了,說:"我不怕人笑話!你的小老婆治我害我,我倒怕人笑話了!再不然,留下他,就賣了我。誰還不知道你薛家有錢,行動拿錢墊人,又有好親戚挾制著別人。你不趁早施爲,還等什麼?嫌我不好,誰叫你們瞎了眼,三求四告的跑了我們家作什麼去了!這會子人也來了,金的銀的也賠了,略有個眼睛鼻子的也霸佔去了,該擠發我了!"一面哭喊,一面滾揉,自己拍打。薛蟠急的說又不好,勸又不好,打又不好,央告又不好,只是出入咳聲嘆氣,抱怨說運氣不好。\begin{note}庚雙夾:果然不差。\end{note}當下薛姨媽早被薛寶釵勸進去了,只命人來賣香菱。寶釵笑道:“咱們家從來只知買人,並不知賣人之說。媽可是氣的胡塗了,倘或叫人聽見,豈不笑話。哥哥嫂子嫌他不好,留下我使喚,我正也沒人使呢。”薛姨媽道:“留著他還是淘氣,不如打發了他倒乾淨。”寶釵笑道:“他跟著我也是一樣,橫豎不叫他到前頭去。從此斷絕了他那裏,也如賣了一般。”香菱早已跑到薛姨媽跟前痛哭哀求,只不願出去,情願跟著姑娘,薛姨媽也只得罷了。
\end{parag}


\begin{parag}
    自此以後,香菱果跟隨寶釵去了,把前面路徑竟一心斷絕。雖然如此,終不免對月傷悲,挑燈自嘆。本來怯弱,雖在薛蟠房中幾年,皆由血分中有病,是以並無胎孕。今復加以氣怒傷感,內外折挫不堪,竟釀成幹血之症,日漸羸瘦作燒,飲食懶進,請醫診視服藥亦不效驗。那時金桂又吵鬧了數次,氣的薛姨媽母女惟暗自垂淚,怨命而已。薛蟠雖曾仗著酒膽挺撞過兩三次,持棍欲打,那金桂便遞與他身子隨意叫打;這裏持刀欲殺時,便伸與他脖項。薛蟠也實不能下手,只得亂鬧了一陣罷了。如今習慣成自然,反使金桂越發長了威風,薛蟠越發軟了氣骨。雖是香菱猶在,卻亦如不在的一般,雖不能十分暢快,就不覺的礙眼了,且姑置不究。如此又漸次尋趁寶蟾。寶蟾卻不比香菱的情性,最是個烈火乾柴,既和薛蟠情投意合,便把金桂忘在腦後。近見金桂又作踐他,他便不肯服低容讓半點。先是一衝一撞的拌嘴,後來金桂氣急了,甚至於罵,再至於打。他雖不敢還言還手,便大撒潑性,拾頭打滾,尋死覓活,晝則刀剪,夜則繩索,無所不鬧。薛蟠此時一身難以兩顧,惟徘徊觀望於二者之間,十分鬧的無法,便出門躲在外廂。金桂不發作性氣,有時歡喜,便糾聚人來鬥紙牌、擲骰子作樂。又生平最喜啃骨頭,每日務要殺雞鴨,將肉賞人喫,只單以油炸焦骨頭下酒。喫的不奈煩或動了氣,便肆行海罵,說:“有別的忘八粉頭樂的,我爲什麼不樂!”薛家母女總不去理他。薛蟠亦無別法,惟日夜悔恨不該娶這攪家星罷了,都是一時沒了主意。\begin{note}庚雙夾:補足本題。\end{note}於是寧榮二宅之人,上上下下,無有不知,無有不嘆者。
\end{parag}


\begin{parag}
    此時寶玉已過了百日,出門行走。亦曾過來見過金桂,“舉止形容也不怪厲,一般是鮮花嫩柳,與衆姊妹不差上下的人,焉得這等樣情性,可爲奇之至極”。\begin{note}庚雙夾:別書中形容妒婦必曰“黃髮黧面”,豈不可笑。\end{note}因此心下納悶。這日與王夫人請安去,又正遇見迎春奶孃來家請安,說起孫紹祖甚屬不端,“姑娘惟有背地裏淌眼抹淚的,只要接了來家散誕兩日”。王夫人因說:“我正要這兩日接他去,只因七事八事的都不遂心,\begin{note}庚雙夾:草蛇灰線,後文方不見突然。\end{note}所以就忘了。前兒寶玉去了,回來也曾說過的。\begin{note}庚雙夾:補明。\end{note}明日是個好日子,就接去。”正說著,賈母打發人來找寶玉,說:“明兒一早往天齊廟還願。”寶玉如今巴不得各處去逛逛,聽見如此,喜的一夜不曾閤眼,盼明不明的。
\end{parag}


\begin{parag}
    次日一早,梳洗穿帶已畢,隨了兩三個老嬤嬤坐車出西城門外天齊廟來燒香還願。這廟裏已是昨日預備停妥的。寶玉天生性怯,不敢近猙獰神鬼之像。這天齊廟本系前朝所修,極其宏壯。如今年深歲久,又極其荒涼。裏面泥胎塑像皆極其兇惡,是以忙忙的焚過紙馬錢糧,便退至道院歇息。一時喫過飯,衆嬤嬤和李貴等人圍隨寶玉到處散誕頑耍了一回。寶玉睏倦,復回至靜室安歇。衆嬤嬤生恐他睡著了,便請當家的老王道士來陪他說話兒。這老王道士專意在江湖上賣藥,弄些海上方治人射利,這廟外現掛著招牌,丸散膏丹,色色俱備,亦長在寧榮兩宅走動熟慣,都與他起了個渾號,喚他作“王一貼”,言他的膏藥靈驗,只一貼百病皆除之意。當下王一貼進來,寶玉正歪在炕上想睡,李貴等正說“哥兒別睡著了”,廝混著。看見王一貼進來,都笑道:“來的好,來的好。王師父,你極會說古記的,說一個與我們小爺聽聽。”王一貼笑道:“正是呢。哥兒別睡,仔細肚裏麪筋作怪。”說著,滿屋裏人都笑了。\begin{note}庚雙夾:王一貼又與張道士遙遙一對,特犯不犯。\end{note}寶玉也笑著起身整衣。王一貼喝命徒弟們快泡好釅茶來。茗煙道:“我們爺不喫你的茶,連這屋裏坐著還嫌膏藥氣息呢。”王一貼笑道:“沒當家花花的,膏藥從不拿進這屋裏來的。知道哥兒今日必來,頭三五天就拿香薰了又燻的。”寶玉道:“可是呢,天天只聽見你的膏藥好,到底治什麼病?”王一貼道:“哥兒若問我的膏藥,說來話長,其中細理,一言難盡。共藥一百二十味,君臣相際,賓客得宜,溫涼兼用,貴賤殊方。內則調元補氣,開胃口,養榮衛,寧神安志,去寒去暑,化食化痰;外則和血脈,舒筋絡,出死肌,生新肉,去風散毒。其效如神,貼過的便知。”寶玉道:“我不信一張膏藥就治這些病。我且問你,倒有一種病可也貼的好麼?”王一貼道:“百病千災,無不立效。若不見效,哥兒只管揪著鬍子打我這老臉,拆我這廟何如?只說出病源來。”寶玉笑道:“你猜,若你猜的著,便貼的好了。”王一貼聽了,尋思一會笑道:“這倒難猜,只怕膏藥有些不靈了。”寶玉命李貴等:“你們且出去散散。這屋裏人多,越發蒸臭了。”李貴等聽說,且都出去自便,只留下茗煙一人。這茗煙手內點著一枝夢甜香,\begin{note}庚雙夾:於前文一出。\end{note}寶玉命他坐在身旁,卻倚在他身上。王一貼心有所動,\begin{note}庚雙夾:四字好。萬端生於心,心邪則意在於邪。\end{note}便笑嘻嘻走近前來,悄悄的說道:“我可猜著了。想是哥兒如今有了房中的事情,要滋助的藥,可是不是?”話猶未完,茗煙先喝道:“該死,打嘴!”寶玉猶未解,\begin{note}庚雙夾:“未解”妙!若解則不成文矣。\end{note}忙問:“他說什麼?”茗煙道:“信他胡說。”唬的王一貼不敢再問,只說:“哥兒明說了罷。”寶玉道:“我問你,可有貼女人的妒病方子沒有?”王一貼聽說,拍手笑道:“這可罷了。不但說沒有方子,就是聽也沒有聽見過。”寶玉笑道:“這樣還算不得什麼。”王一貼又忙道:“這貼妒的膏藥倒沒經過,倒有一種湯藥或者可醫,只是慢些兒,不能立竿見影的效驗。”寶玉道:“什麼湯藥,怎麼喫法?”王一貼道:“這叫做‘療妒湯’:用極好的秋梨一個,二錢冰糖,一錢陳皮,水三碗,梨熟爲度,每日清早喫這麼一個梨,喫來喫去就好了。”寶玉道:“這也不值什麼,只怕未必見效。”王一貼道:“一劑不效喫十劑,今日不效明日再喫,今年不效喫到明年。橫豎這三味藥都是潤肺開胃不傷人的,甜絲絲的,又止咳嗽,又好喫。喫過一百歲,人橫豎是要死的,死了還妒什麼!那時就見效了。”\begin{note}庚雙夾:此科諢一收,方爲奇趣之至。\end{note}說著,寶玉茗煙都大笑不止,罵“油嘴的牛頭”。王一貼笑道:“不過是閒著解午盹罷了,有什麼關係。說笑了你們就值錢。實告你們說,連膏藥也是假的。我有真藥,我還吃了作神仙呢。有真的,跑到這裏來混?”\begin{note}庚雙夾:寓意深遠,在此數語。\end{note}正說著,吉時已到,請寶玉出去焚化錢糧散福。功課完畢,方進城回家。
\end{parag}


\begin{parag}
    那時迎春已來家好半日,孫家的婆娘媳婦等人已待過晚飯,打發回家去了。迎春方哭哭啼啼的在王夫人房中訴委曲,說孫紹祖“一味好色,好賭酗酒,家中所有的媳婦丫頭將及淫遍。略勸過兩三次,便罵我是‘醋汁子老婆擰出來的’。\begin{note}庚雙夾:奇文奇罵。爲迎春一哭。恨薛蟠何等剛霸,偏不能以此語金桂,使人忿忿。是書中全是不平,有全是意外之料。\end{note}又說老爺曾收著他五千銀子,不該使了他的。如今他來要了兩三次不得,他便指著我的臉說道:‘你別和我充夫人娘子,你老子使了我五千銀子,把你准折買給我的。好不好,打一頓攆在下房裏睡去。當日有你爺爺在時,希圖上我們的富貴,趕著相與的。論理我和你父親是一輩,如今強壓我的頭,賣了一輩。又不該作了這門親,倒沒的叫人看著趕勢利似的。’”\begin{note}庚雙夾:不通,可笑。遁詞如聞。\end{note}一行說,一行哭的嗚嗚咽咽,連王夫人並衆姊妹無不落淚。王夫人只得用言語解勸說:“已是遇見了這不曉事的人,可怎麼樣呢。想當日你叔叔也曾勸過大老爺,不叫作這門親的。大老爺執意不聽,一心情願,到底作不好了。我的兒,這也是你的命。”迎春哭道:“我不信我的命就這麼不好!從小兒沒了娘,幸而過嬸子這邊過了幾年心淨日子,如今偏又是這麼個結果!”王夫人一面解勸,一面問他隨意要在那裏安歇。迎春道:“乍乍的離了姊妹們,只是眠思夢想。二則還記掛著我的屋子,還得在園裏舊房子裏住得三五天,死也甘心了。不知下次還可能得住不得住了呢!”王夫人忙勸道:“快休亂說。不過年輕的夫妻們,閒牙鬥齒,亦是萬萬人之常事,何必說這喪話。”仍命人忙忙的收拾紫菱洲房屋,命姊妹們陪伴著解釋,又吩咐寶玉:“不許在老太太跟前走漏一些風聲,倘或老太太知道了這些事,都是你說的。”寶玉唯唯的聽命。迎春是夕仍在舊館安歇。衆姊妹等更加親熱異常。一連住了三日,才往邢夫人那邊去。先辭過賈母及王夫人,然後與衆姊妹分別,更皆悲傷不捨。還是王夫人薛姨媽等安慰勸釋,方止住了過那邊去。\begin{note}庚雙夾:凡迎春之文皆從寶玉眼中寫出。前“悔娶河東獅”是實寫,“誤嫁中山狼”出迎春口中可爲虛寫,以虛虛實實變幻體格,各盡其法。\end{note}又在邢夫人處住了兩日,就有孫紹祖的人來接去。迎春雖不願去,無奈懼孫紹祖之惡,只得勉強忍情作辭了。邢夫人本不在意,也不問其夫妻和睦,家務煩難,只面情塞責而已。終不知端的,且聽下回分解。
\end{parag}


\begin{parag}
    \begin{note}蒙回末總:此文一爲擇婿者說法,一爲擇妻者說法,擇婿者必以得人物軒昂、家道豐厚、陰襲公子爲快,擇妻者必以得容貌豔麗、妝奩富厚、子女盈門爲快,殊不知以貌取人失之子羽。試者桂花夏家指擇孫家,何等可羨可樂。卒至迎春含悲,薛蟠遺恨,可慨矣夫!\end{note}
\end{parag}
