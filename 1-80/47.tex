\chap{四十七}{呆霸王調情遭苦打 冷郎君懼禍走他鄉}


\begin{parag}
    \begin{note}蒙回前總:不是同人,且莫浪作知心語。似假如真,事事應難許。著緊溫存,白雪陽和曲。誰堪比?船上要離,未解奸俠起。\end{note}
\end{parag}


\begin{parag}
    話說王夫人聽見邢夫人來了,連忙迎了出去。邢夫人猶不知賈母已知鴛鴦之事,正還要來打聽信息,進了院門,早有幾個婆子悄悄的回了他,他方知道。待要回去,裏面已知,又見王夫人接了出來,少不得進來,先與賈母請安,賈母一聲兒不言語,自己也覺得愧悔。鳳姐兒早指一事迴避了。鴛鴦也自回房去生氣。薛姨媽王夫人等恐礙著邢夫人的臉面,也都漸漸的退了。邢夫人且不敢出去。
\end{parag}


\begin{parag}
    賈母見無人,方說道:“我聽見你替你老爺說媒來了。你倒也三從四德,只是這賢慧也太過了!你們如今也是孫子兒子滿眼了,你還怕他,勸兩句都使不得,還由著你老爺性兒鬧。”邢夫人滿面通紅,回道:“我勸過幾次不依。老太太還有什麼不知道呢,我也是不得已兒。”賈母道:“他逼著你殺人,你也殺去?如今你也想想,你兄弟媳婦本來老實,又生得多病多痛,上上下下那不是他操心?你一個媳婦雖然幫著,也是天天丟下笆兒弄掃帚。凡百事情,我如今都自己減了。他們兩個就有一些不到的去處,有鴛鴦,那孩子還心細些,我的事情他還想著一點子,該要去的,他就要了來,該添什麼,他就度空兒告訴他們添了。鴛鴦再不這樣,他孃兒兩個,裏頭外頭,大的小的,那裏不忽略一件半件,我如今反倒自己操心去不成?還是天天盤算和你們要東西去?我這屋裏有的沒的,剩了他一個,年紀也大些,我凡百的脾氣性格兒他還知道些。二則他還投主子們的緣法,也並不指著我和這位太太要衣裳去,又和那位奶奶要銀子去。所以這幾年一應事情,他說什麼,從你小嬸和你媳婦起,以至家下大大小小,沒有不信的。所以不單我得靠,連你小嬸媳婦也都省心。我有了這麼個人,便是媳婦和孫子媳婦有想不到的,我也不得缺了,也沒氣可生了。這會子他去了,你們弄個什麼人來我使?你們就弄他那麼一個真珠的人來,不會說話也無用。我正要打發人和你老爺說去,他要什麼人,我這裏有錢,叫他只管一萬八千的買,就只這個丫頭不能。留下他伏侍我幾年,就比他日夜伏侍我盡了孝的一般。你來的也巧,你就去說,更妥當了。”
\end{parag}


\begin{parag}
    說畢,命人來:“請了姨太太你姑娘們來說個話兒。才高興,怎麼又都散了!”丫頭們忙答應著去了。衆人忙趕的又來。只有薛姨媽向丫鬟道:“我纔來了,又作什麼去?你就說我睡了覺了。”那丫頭道:“好親親的姨太太,姨祖宗!我們老太太生氣呢,你老人家不去,沒個開交了,只當疼我們罷。你老人家嫌乏,我背了你老人家去。”薛姨媽道:“小鬼頭兒,你怕些什麼?不過罵幾句完了。”說著,只得和這小丫頭子走來。賈母忙讓坐,又笑道:“咱們鬥牌罷。姨太太的牌也生,咱們一處坐著,別叫鳳姐兒混了我們去。”薛姨媽笑道:“正是呢,老太太替我看著些兒。就是咱們孃兒四個鬥呢,還是再添個呢?”王夫人笑道:“可不只四個。”\begin{note}庚雙夾:老實人言語。\end{note}鳳姐兒道:“再添一個人熱鬧些。”賈母道:“叫鴛鴦來,叫他在這下手裏坐著。姨太太眼花了,咱們兩個的牌都叫他瞧著些兒。”鳳姐兒嘆了一聲,向探春道:“你們知書識字的,倒不學算命!”探春道:“這又奇了。這會子你倒不打點精神贏老太太幾個錢,又想算命。”鳳姐兒道: “我正要算算命今兒該輸多少呢,我還想贏呢!你瞧瞧,場子沒上,左右都埋伏下了。”說的賈母薛姨媽都笑起來。
\end{parag}


\begin{parag}
    一時鴛鴦來了,便坐在賈母下手,鴛鴦之下便是鳳姐兒。鋪下紅氈,洗牌告幺,五人起牌。鬥了一回,鴛鴦見賈母的牌已十嚴,只等一張二餅,便遞了暗號與鳳姐兒。鳳姐兒正該發牌,便故意躊躇了半晌,笑道:“我這一張牌定在姨媽手裏扣著呢。我若不發這一張,再頂不下來的。”薛姨媽道:“我手裏並沒有你的牌。” 鳳姐兒道:“我回來是要查的。”薛姨媽道:“你只管查。你且發下來,我瞧瞧是張什麼。”鳳姐兒便送在薛姨媽跟前。薛姨媽一看是個二餅,便笑道:“我倒不稀罕他,只怕老太太滿了。”鳳姐兒聽了,忙笑道:“我發錯了。”賈母笑的已擲下牌來,說:“你敢拿回去!誰叫你錯的不成?”鳳姐兒道:“可是我要算一算命呢。這是自己發的,也怨埋伏!”賈母笑道:“可是呢,你自己該打著你那嘴,問著你自己纔是。”又向薛姨媽笑道:“我不是小器愛贏錢,原是個彩頭兒。”薛姨媽笑道:“可不是這樣,那裏有那樣糊塗人說老太太愛錢呢?”鳳姐兒正數著錢,聽了這話,忙又把錢穿上了,向衆人笑道;“夠了我的了。竟不爲贏錢,單爲贏彩頭兒。我到底小器,輸了就數錢,快收起來罷。”賈母規矩是鴛鴦代洗牌,因和薛姨媽說笑,不見鴛鴦動手,賈母道:“你怎麼惱了,連牌也不替我洗。”鴛鴦拿起牌來,笑道:“二奶奶不給錢。”賈母道:“他不給錢,那是他交運了。”便命小丫頭子:“把他那一吊錢都拿過來。”小丫頭子真就拿了,擱在賈母旁邊。鳳姐兒笑道:“賞我罷,我照數兒給就是了。”薛姨媽笑道:“果然是鳳丫頭小器,不過是頑兒罷了。”鳳姐聽說,便站起來,拉著薛姨媽,回頭指著賈母素日放錢的一個木匣子笑道:“姨媽瞧瞧,那個裏頭不知頑了我多少去了。這一吊錢頑不了半個時辰,那裏頭的錢就招手兒叫他了。只等把這一吊也叫進去了,牌也不用鬥了,老祖宗的氣也平了,又有正經事差我辦去了。”話說未完,引的賈母衆人笑個不住。偏有平兒怕錢不夠,又送了一吊來。鳳姐兒道:“不用放在我跟前,也放在老太太的那一處罷。一齊叫進去倒省事,不用做兩次,叫箱子裏的錢費事。”賈母笑的手裏的牌撒了一桌子,推著鴛鴦,叫:“快撕他的嘴!”
\end{parag}


\begin{parag}
    平兒依言放下錢,也笑了一回,方回來。至院門前遇見賈璉,問他:“太太在那裏呢?老爺叫我請過去呢。”平兒忙笑道:“在老太太跟前呢,站了這半日還沒動呢。趁早兒丟開手罷。老太太生了半日氣,這會子虧二奶奶湊了半日趣兒,才略好了些。”賈璉道:“我過去只說討老太太的示下,十四往賴大家去不去,好預備轎子的。又請了太太,又湊了趣兒,豈不好?”平兒笑道:“依我說,你竟不去罷。閤家子連太太寶玉都有了不是,這會子你又填限去了。”賈璉道:“已經完了,難道還找補不成?況且與我又無干。二則老爺親自吩咐我請太太的,這會子我打發了人去,倘或知道了,正沒好氣呢,指著這個拿我出氣罷。”說著就走。平兒見他說得有理,也便跟了過來。
\end{parag}


\begin{parag}
    賈璉到了堂屋裏,便把腳步放輕了,往裏間探頭,只見邢夫人站在那裏。鳳姐兒眼尖,先瞧見了,使眼色兒不命他進來,又使眼色與邢夫人。邢夫人不便就走,只得倒了一碗茶來,放在賈母跟前。賈母一回身,賈璉不防,便沒躲伶俐。賈母便問:“外頭是誰?倒象個小子一伸頭。”鳳姐兒忙起身說:“我也恍惚看見一個人影兒,讓我瞧瞧去。”一面說,一面起身出來。賈璉忙進去,陪笑道:“打聽老太太十四可出門?好預備轎子。”賈母道:“既這麼樣,怎麼不進來?又作鬼作神的。”賈璉陪笑道:“見老太太玩牌,不敢驚動,不過叫媳婦出來問問。”賈母道:“就忙到這一時,等他家去,你問多少問不得?那一遭兒你這麼小心來著!又不知是來作耳報神的,也不知是來作探子的,鬼鬼祟祟的,倒唬了我一跳。什麼好下流種子!你媳婦和我頑牌呢,還有半日的空兒,你家去再和那趙二家的商量治你媳婦去罷!”說著,衆人都笑了。鴛鴦笑道:“鮑二家的,老祖宗又拉上趙二家的。”賈母也笑道:“可是,我那裏記得什麼抱著背著的,提起這些事來,不由我不生氣!我進了這門子作重孫子媳婦起,到如今我也有了重孫子媳婦了,連頭帶尾五十四年,憑著大驚大險千奇百怪的事,也經了些,從沒經過這些事。還不離了我這裏呢!”
\end{parag}


\begin{parag}
    賈璉一聲兒不敢說,忙退了出來。平兒站在窗外悄悄的笑道:“我說著你不聽,到底碰在網裏了。”正說著,只見邢夫人也出來,賈璉道:“都是老爺鬧的,如今都搬在我和太太身上。”邢夫人道:“我把你沒孝心雷打的下流種子!人家還替老子死呢,白說了幾句,你就抱怨了。你還不好好的呢,這幾日生氣,仔細他捶你。”賈璉道:“太太快過去罷,叫我來請了好半日了。”說著,送他母親出來過那邊去。
\end{parag}


\begin{parag}
    邢夫人將方纔的話只略說了幾 ,賈赦無法,又含愧,自此便告病, 不敢見賈母,只打發邢夫人及賈璉每日過去請安。只得又各處遣人購求尋覓,終究費了八百兩銀子買了一個十七歲的女孩子來,名喚嫣紅,收在屋內。不在話下。
\end{parag}


\begin{parag}
    這裏鬥了半日牌,喫晚飯才罷。此一二日間無話。
\end{parag}


\begin{parag}
    展眼到了十四日,黑早,賴大的媳婦又進來請。賈母高興,便帶了王夫人薛姨媽及寶玉姊妹等,到賴大花園中坐了半日。那花園雖不及大觀園,卻也十分齊整寬闊,泉石林木,樓閣亭軒,也有好幾處驚人駭目的。外面廳上,薛蟠、賈珍、賈璉、賈蓉並幾個近族的,很遠的也沒來,賈赦也沒來。賴大家內也請了幾個現任的官長並幾個世家子弟作陪。因其中有柳湘蓮,薛蟠自上次會過一次,已念念不忘。又打聽他最喜串戲,且串的都是生旦風月戲文,不免錯會了意,誤認他作了風月子弟,正要與他相交,恨沒有個引進,這日可巧遇見,竟覺無可不可。且賈珍等也慕他的名,酒蓋住了臉,就求他串了兩齣戲。下來,移席和他一處坐著,問長問短,說此說彼。
\end{parag}


\begin{parag}
    那柳湘蓮原是世家子弟,讀書不成,父母早喪,素性爽俠,不拘細事,酷好耍槍舞劍,賭博喫酒,以至眠花臥柳,吹笛彈箏,無所不爲。因他年紀又輕,生得又美,不知他身分的人,卻誤認作優伶一類。那賴大之子賴尚榮與他素習交好,故他今日請來作陪。不想酒後別人猶可,獨薛蟠又犯了舊病。他心中早已不快,得便意欲走開完事,無奈賴尚榮死也不放。賴尚榮又說:“方纔寶二爺又囑咐我,才一進門雖見了,只是人多不好說話,叫我囑咐你散的時候別走,他還有話說呢。你既一定要去,等我叫出他來,你兩個見了再走,與我無干。”說著,便命小廝們到裏頭找一個老婆子,悄悄告訴“請出寶二爺來。”那小廝去了沒一盞茶時,果見寶玉出來了。賴尚榮向寶玉笑道:“好叔叔,把他交給你,我張羅人去了。”說著,一徑去了。
\end{parag}


\begin{parag}
    寶玉便拉了柳湘蓮到廳側小書房中坐下,問他這幾日可到秦鐘的墳上去了。\begin{note}庚雙夾:忽提此人使我墮淚。近幾回不見提此人,自謂不表矣。乃忽於此處柳湘蓮提及,所謂“方以類聚,物以羣分”也。\end{note}湘蓮道:“怎麼不去?前日我們幾個人放鷹去,離他墳上還有二里,我想今年夏天的雨水勤,恐怕他的墳站不住。我背著衆人,走去瞧了一瞧,果然又動了一點子。回家來就便弄了幾百錢,第三日一早出去,僱了兩個人收拾好了。”寶玉道:“怪道呢,上月我們大觀園的池子裏頭結了蓮蓬,我摘了十個,叫茗煙出去到墳上供他去,回來我也問他可被雨沖壞了沒有。他說不但不衝,且比上回又新了些。我想著,不過是這幾個朋友新築了。我只恨我天天圈在家裏,一點兒做不得主,行動就有人知道,不是這個攔就是那個勸的,能說不能行。雖然有錢,又不由我使。”湘蓮道:“這個事也用不著你操心,外頭有我,你只心裏有了就是。眼前十月初一,我已經打點下上墳的花消。你知道我一貧如洗,家裏是沒的積聚,縱有幾個錢來,隨手就光的,不如趁空兒留下這一分,省得到了跟前扎煞手。”寶玉道:“我也正爲這個要打發茗煙找你,你又不大在家,知道你天天萍蹤浪跡,沒個一定的去處。”湘蓮道:“這也不用找我。這個事不過各盡其道。眼前我還要出門去走走,外頭逛個三年五載再回來。”寶玉聽了,忙問道:“這是爲何?”柳湘蓮冷笑道:“你不知道我的心事,等到跟前你自然知道。我如今要別過了。”寶玉道:“好容易會著,晚上同散豈不好?”湘蓮道:“你那令姨表兄還是那樣,再坐著未免有事,不如我回避了倒好。”寶玉想了一想,道:“既是這樣,倒是迴避他爲是。只是你要果真遠行,必須先告訴我一聲,千萬別悄悄的去了。”說著便滴下淚來。柳湘蓮道:“自然要辭的。你只別和別人說就是。”說著便站起來要走,又道:“你們進去,不必送我。”
\end{parag}


\begin{parag}
    一面說,一面出了書房。剛至大門前,早遇見薛蟠在那裏亂嚷亂叫說:“誰放了小柳兒走了!”柳湘蓮聽了,火星亂迸,恨不得一拳打死,復思酒後揮拳,又礙著賴尚榮的臉面,只得忍了又忍。薛蟠忽見他走出來,如得了珍寶,忙趔趄著上來一把拉住,笑道:“我的兄弟,你往那裏去了?”湘蓮道:“走走就來。”薛蟠笑道:“好兄弟,你一去都沒興了,好歹坐一坐,你就疼我了。憑你有什麼要緊的事,交給哥,你只別忙,有你這個哥,你要做官發財都容易。”湘蓮見他如此不堪,心中又恨又愧,早生一計,便拉他到避人之處,笑道:“你真心和我好,假心和我好呢?”薛蟠聽這話,喜的心癢難撓,乜斜著眼忙笑道:“好兄弟,你怎麼問起我這話來?我要是假心,立刻死在眼前!”湘蓮道:“既如此,這裏不便。等坐一坐,我先走,你隨後出來,跟到我下處,咱們替另喝一夜酒。我那裏還有兩個絕好的孩子,從沒出門。你可連一個跟的人也不用帶,到了那裏,伏侍的人都是現成的。”薛蟠聽如此說,喜得酒醒了一半,說:“果然如此?”湘蓮道:“如何!人拿真心待你,你倒不信了!”薛蟠忙笑道:“我又不是呆子,怎麼有個不信的呢!既如此,我又不認得,你先去了,我在那裏找你?”湘蓮道:“我這下處在北門外頭,你可捨得家,城外住一夜去?”薛蟠笑道:“有了你,我還要家做什麼!”湘蓮道:“既如此,我在北門外頭橋上等你。咱們席上且喫酒去。你看我走了之後你再走,他們就不留心了。”薛蟠聽了,連忙答應。於是二人復又入席,飲了一回。那薛蟠難熬,只拿眼看湘蓮,心內越想越樂,左一壺右一壺,並不用人讓,自己便吃了又喫,不覺酒已八九分了。
\end{parag}


\begin{parag}
    湘蓮便起身出來,瞅人不防去了,至門外,命小廝杏奴:“先家去罷,我到城外就來。”說畢,已跨馬直出北門,橋上等候薛蟠。沒頓飯時工夫,只見薛蟠騎著一匹大馬,遠遠的趕了來,張著嘴,瞪著眼,頭似撥浪鼓一般不住左右亂瞧。及至從湘蓮馬前過去,只顧望遠處瞧,不曾留心近處,反踩過去了。湘蓮又是笑,又是恨,便也撒馬隨後趕來。薛蟠往前看時,漸漸人煙稀少,便又圈馬回來再找,不想一回頭見了湘蓮,如獲奇珍,忙笑道:“我說你是個再不失信的。”湘蓮笑道: “快往前走,仔細人看見跟了來,就不便了。”說著,先就撒馬前去,薛蟠也緊緊跟來。
\end{parag}


\begin{parag}
    湘蓮見前面人跡已稀,且有一帶葦塘,便下馬,將馬拴在樹上,向薛蟠笑道:“你下來,咱們先設個誓,日後要變了心,告訴人去的,便應了誓。”薛蟠笑道: “這話有理。”連忙下了馬,也拴在樹上,便跪下說道:“我要日久變心,告訴人去的,天誅地滅!”一語未了,只聽“嘡”的一聲,頸後好似鐵錘砸下來,只覺得一陣黑,滿眼金星亂迸,身不由己,便倒下來。湘蓮走上來瞧瞧,知道他是個笨家,不慣捱打,只使了三分氣力,向他臉上拍了幾下,登時便開了果子鋪。薛蟠先還要掙挫起來,又被湘蓮用腳尖點了兩點,仍舊跌倒,口內說道:“原是兩家情願,你不依,只好說,爲什麼哄出我來打我?”一面說,一面亂罵。湘蓮道:“我把你瞎了眼的,你認認柳大爺是誰!你不說哀求,你還傷我!我打死你也無益,只給你個利害罷。”說著,便取了馬鞭過來,從背至脛,打了三四十下。薛蟠酒已醒了大半,覺得疼痛難禁,不禁有“噯喲”之聲。湘蓮冷笑道:“也只如此!我只當你是不怕打的。”一面說,一面又把薛蟠的左腿拉起來,朝葦中濘泥處拉了幾步,滾的滿身泥水,又問道:“你可認得我了?”薛蟠不應,只伏著哼哼。湘蓮又擲下鞭子,用拳頭向他身上擂了幾下。薛蟠便亂滾亂叫,說:“肋條折了。我知道你是正經人,因爲我錯聽了旁人的話了。”湘蓮道:“不用拉別人,你只說現在的。”薛蟠道:“現在沒什麼說的。不過你是個正經人,我錯了。”湘蓮道:“還要說軟些才饒你。”薛蟠哼哼著道:“好兄弟。”湘蓮便又一拳。薛蟠“噯喲”了一聲道:“好哥哥。”湘蓮又連兩拳。薛蟠忙“噯喲”叫道:“好老爺,饒了我這沒眼睛的瞎子罷!從今以後我敬你怕你了。”湘漣道:“你把那水喝兩口!”薛蟠一面聽了,一面皺眉道:“那水髒得很,怎麼喝得下去!”湘蓮舉拳就打。薛蟠忙道:“我喝,喝。”說著,只得俯頭向葦根下喝了一口,猶未嚥下去,只聽“哇”的一聲,把方纔喫的東西都吐了出來。湘蓮道:“好髒東西,你快吃盡了饒你。”薛蟠聽了,叩頭不迭道:“好歹積陰功饒我罷!這至死不能喫的。”湘蓮道:“這樣氣息,倒薰壞了我。”說著丟了薛蟠,便牽馬認鐙去了。這裏薛蟠見他已去,心內方放下心來,後悔自己不該誤認了人。待要掙挫起來,無奈遍身疼痛難禁。
\end{parag}


\begin{parag}
    誰知賈珍等席上忽然不見了他兩個,各處尋找不見。有人說:“恍惚出北門去了。”薛蟠的小廝們素日是懼他的,他吩咐不許跟去,誰還敢找去?\begin{note}庚雙夾:亦如秦法自誤。\end{note}後來還是賈珍不放心,命賈蓉帶著小廝們尋蹤問跡的直找出北門,下橋二里多路,忽見葦坑邊薛蟠的馬拴在那裏。衆人都道:“可好了!有馬必有人。”一齊來至馬前,只聽葦中有人呻吟。大家忙走來一看,只見薛蟠衣衫零碎,面目腫破,沒頭沒臉,遍身內外,滾的似個泥豬一般。賈蓉心內已猜著九分了,忙下馬令人攙了出來,笑道:“薛大叔天天調情,今兒調到葦子坑裏來了。必定是龍王爺也愛上你風流,要你招駙馬去,你就碰到龍犄角上了。”薛蟠羞的恨沒地縫兒鑽不進去,那裏爬的上馬去?賈蓉只得命人趕到關廂裏僱了一乘小轎子,薛蟠坐了,一齊進城。賈蓉還要抬往賴家去赴席,薛蟠百般央告,又命他不要告訴人,賈蓉方依允了,讓他各自回家。賈蓉仍往賴家回覆賈珍,並說方纔形景。賈珍也知爲湘蓮所打,也笑道:“他須得喫個虧纔好。”至晚散了,便來問候。薛蟠自在臥房將養,推病不見。
\end{parag}


\begin{parag}
    賈母等回來各自歸家時,薛姨媽與寶釵見香菱哭得眼睛腫了。問其原故,忙趕來瞧薛蟠時,臉上身上雖有傷痕,並未傷筋動骨。薛姨媽又是心疼,又是發恨,罵一回薛蟠,又罵一回柳湘蓮,意欲告訴王夫人,遣人尋拿柳湘蓮。寶釵忙勸道:“這不是什麼大事,不過他們一處喫酒,酒後反臉常情。誰醉了,多挨幾下子打,也是有的。況且咱們家無法無天,也是人所共知的。媽不過是心疼的緣故。要出氣也容易,等三五天哥哥養好了出的去時,那邊珍大爺璉二爺這幹人也未必白丟開了,自然備個東道,叫了那個人來,當著衆人替哥哥賠不是認罪就是了。如今媽先當件大事告訴衆人,倒顯得媽偏心溺愛,縱容他生事招人,今兒偶然吃了一次虧,媽就這樣興師動衆,倚著親戚之勢欺壓常人。”薛姨媽聽了道:“我的兒,到底是你想的到,我一時氣糊塗了。”寶釵笑道:“這纔好呢。他又不怕媽,又不聽人勸,一天縱似一天,喫過兩三個虧,他倒罷了。”薛蟠睡在炕上痛罵柳湘蓮,又命小廝們去拆他的房子,打死他,和他打官司。薛姨媽禁住小廝們,只說柳湘蓮一時酒後放肆,如今酒醒,後悔不及,懼罪逃走了。薛蟠聽見如此說了,要知端的
\end{parag}


\begin{parag}
    \begin{note}蒙回末總:自開牌一節,寫貴家長上之尊重,卑幼之侍奉,寫薛蟠之醜,湘蓮之豪,薛母寶釵之言無不逼真。\end{note}
\end{parag}

