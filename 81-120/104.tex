\chap{一百零四}{醉金刚小鳅生大浪 痴公子余痛触前情}



\begin{parag}
    话说贾雨村刚欲过渡,见有人飞奔而来,跑到跟前,口称:“老爷,方才进的那庙火起了!”雨村回首看时,只见烈炎烧天,飞灰蔽目。雨村心想,“这也奇怪,我才出来,走不多远,这火从何而来?莫非士隐遭劫于此?”欲待回去,又恐误了过河,若不回去,心下又不安。想了一想,便问道:“你方才见这老道士出来了没有?”那人道:“小的原随老爷出来,因腹内疼痛,略走了一走。回头看见一片火光,原来就是那庙中火起,特赶来禀知老爷。并没有见有人出来。”雨村虽则心里狐疑,究竟是名利关心的人,那肯回去看视,便叫那人:“你在这里等火灭了进去瞧那老道在与不在,即来回禀。”那人只得答应了伺候。
\end{parag}


\begin{parag}
    雨村过河,仍自去查看,查了几处,遇公馆便自歇下。明日又行一程,进了都门,众衙役接着,前呼后拥的走着。雨村坐在轿内,听见轿前开路的人吵嚷。雨村问是何事。那开路的拉了一个人过来跪在轿前禀道:“那人酒醉不知回避,反冲突过来。小的吆喝他,他倒恃酒撒赖,躺在街心,说小的打了他了。”雨村便道:“我是管理这里地方的。你们都是我的子民,知道本府经过,喝了酒不知退避,还敢撒赖!”那人道:“我喝酒是自己的钱,醉了躺的是皇上的地,便是大人老爷也管不得。”雨村怒道:“这人目无法纪,问他叫什么名字。”那人回道:“我叫醉金刚倪二。”雨村听了生气,叫人:“打这金刚,瞧他是金刚不是!”手下把倪二按倒,着实的打了几鞭。倪二负痛,酒醒求饶。雨村在轿内笑道:“原来是这么个金刚么。我且不打你,叫人带进衙门慢慢的问你。”众衙役答应,拴了倪二,拉着便走。倪二哀求,也不中用。雨村进内复旨回曹,那里把这件事放在心上。
\end{parag}


\begin{parag}
    那街上看热闹的三三两两传说:“倪二仗着有些力气,恃酒讹人,今儿碰在贾大人手里,只怕不轻饶的。”这话已传到他妻女耳边。那夜果等倪二不见回家,他女儿便到各处赌场寻觅,那赌博的都是这么说,他女儿急得哭了。众人都道:“你不用着急。那贾大人是荣府的一家。荣府里的一个什么二爷和你父亲相好,你同你母亲去找他说个情,就放出来了。”倪二的女儿听了,想了一想,”果然我父亲常说间壁贾二爷和他好,为什么不找他去。”赶着回来,即和母亲说了。娘儿两个去找贾芸。
\end{parag}


\begin{parag}
    那日贾芸恰在家,见他母女两个过来,便让坐。贾芸的母亲便倒茶。倪家母女即将倪二被贾大人拿去的话说了一遍,”求二爷说情放出来”。贾芸一口应承,说:“这算不得什么,我到西府里说一声就放了。那贾大人全仗我家的西府里才得做了这么大官,只要打发个人去一说就完了。”倪家母女欢喜,回来便到府里告诉了倪二,叫他不用忙,已经求了贾二爷,他满口应承,讨个情便放出来的。倪二听了也喜欢。
\end{parag}


\begin{parag}
    不料贾芸自从那日给凤姐送礼不收,不好意思进来,也不常到荣府。那荣府的门上原看着主子的行事,叫谁走动才有些体面,一时来了他便进去通报,若主子不大理了,不论本家亲戚,他一概不回,支了去就完事。那日贾芸到府上说“给琏二爷请安”。门上的说:“二爷不在家,等回来我们替回罢。”贾芸欲要说“请二奶奶的安”,生恐门上厌烦,只得回家。又被倪家母女催逼着说:“二爷常说府上是不论那个衙门,说一声谁敢不依。如今还是府里的一家,又不为什么大事,这个情还讨不来,白是我们二爷了。”贾芸脸上下不来,嘴里还说硬话:“昨儿我们家里有事,没打发人说去,少不得今儿说了就放。什么大不了的事!”倪家母女只得听信。
\end{parag}


\begin{parag}
    岂知贾芸近日大门竟不得进去,绕到后头要进园内找宝玉,不料园门锁着,只得垂头丧气的回来。想起“那年倪二借银与我,买了香料送给他,才派我种树。如今我没有钱去打点,就把我拒绝。他也不是什么好的,拿着太爷留下的公中银钱在外放加一钱,我们穷本家要借一两也不能。他打谅保得住一辈子不穷的了,那知外头的声名很不好。我不说罢了,若说起来,人命官司不知有多少呢。”一面想着,来到家中,只见倪家母女都等着。贾芸无言可支,便说道:“西府里已经打发人说了,只言贾大人不依。你还求我们家的奴才周瑞的亲戚冷子兴去才中用。”倪家母女听了说:“二爷这样体面爷们还不中用,若是奴才,是更不中用了。”贾芸不好意思,心里发急道:“你不知道,如今的奴才比主子强多着呢。”倪家母女听来无法,只得冷笑几声说:“这倒难为二爷白跑了这几天,等我们那一个出来再道乏罢。”说毕出来,另托人将倪二弄了出来,只打了几板,也没有什么罪。
\end{parag}


\begin{parag}
    倪二回家,他妻女将贾家不肯说情的话说了一遍。倪二正喝着酒,便生气要找贾芸,说:“这小杂种,没良心的东西!头里他没有饭吃要到府内钻谋事办,亏我倪二爷帮了他。如今我有了事他不管。好罢咧,若是我倪二闹出来,连两府里都不干净!”他妻女忙劝道:“嗳,你又喝了黄汤便是这样有天没日头的,前儿可不是醉了闹的乱子,捱了打还没好呢,你又闹了。”倪二道:“捱了打便怕他不成,只怕拿不着由头!我在监里的时候,倒认得了好几个有义气的朋友,听见他们说起来,不独是城内姓贾的多,外省姓贾的也不少。前儿监里收下了好几个贾家的家人。我倒说,这里的贾家小一辈子并奴才们虽不好,他们老一辈的还好,怎么犯了事。我打听打听,说是和这里贾家是一家,都住在外省,审明白了解进来问罪的,我才放心。若说贾二这小子他忘恩负义,我便和几个朋友说他家怎样倚势欺人,怎样盘剥小民,怎样强娶有男妇女,叫他们吵嚷出来,有了风声到了都老爷耳朵里,这一闹起来,叫你们才认得倪二金刚呢!”他女人道:“你喝了酒睡去罢!他又强占谁家的女人来了,没有的事你不用混说了。”倪二道:“你们在家里那里知道外头的事。前年我在赌场里碰见了小张,说他女人被贾家占了,他还和我商量。我倒劝他才了事的。但不知这小张如今那里去了,这两年没见。若碰着了他,我倪二出个主意叫贾老二死,给我好好的孝敬孝敬我倪二太爷才罢了。你倒不理我了!”说着,倒身躺下,嘴里还是咕咕嘟嘟的说了一回,便睡去了。他妻女只当是醉话,也不理他。明日早起,倪二又往赌场中去了。不题。
\end{parag}


\begin{parag}
    且说雨村回到家中,歇息了一夜,将道上遇见甄士隐的事告诉了他夫人一遍。他夫人便埋怨他:“为什么不回去瞧一瞧,倘或烧死了,可不是咱们没良心!”说着,掉下泪来。雨村道:“他是方外的人了,不肯和咱们在一处的。”正说着,外头传进话来,禀说:“前日老爷吩咐瞧火烧庙去的回来了回话。”雨村踱了出来。那衙役打千请了安,回说:“小的奉老爷的命回去,也不等火灭,便冒火进去瞧那个道士,岂知他坐的地方多烧了。小的想着那道士必定烧死了。那烧的墙屋往后塌去,道士的影儿都没有,只有一个蒲团,一个瓢儿还是好好的。小的各处找寻他的尸首,连骨头都没有一点儿。小的恐老爷不信,想要拿这蒲团瓢儿回来做个证见,小的这么一拿,岂知都成了灰了。”雨村听毕,心下明白,知士隐仙去,便把那衙役打发了出去。回到房中,并没提吉士隐火化之言,恐他妇女不知,反生悲感,只说并无形迹,必是他先走了。
\end{parag}


\begin{parag}
    雨村出来,独坐书房,正要细想士隐的话,忽有家人传报说:“内廷传旨,交看事件。”雨村疾忙上轿进内,只听见人说:“今日贾存周江西粮道被参回来,在朝内谢罪。”雨村忙到了内阁,见了各大人,将海疆办理不善的旨意看了,出来即忙找着贾政,先说了些为他抱屈的话,后又道喜,问:“一路可好?”贾政也将违别以后的话细细的说了一遍。雨村道:“谢罪的本上了去没有?”贾政道:“已上去了,等膳后下来看旨意罢。”正说着,只听里头传出旨来叫贾政,贾政即忙进去。各大人有与贾政关切的,都在里头等着。等了好一回方见贾政出来,看见他带着满头的汗。众人迎上去接着,问:“有什么旨意。”贾政吐舌道:“吓死人,吓死人!倒蒙各位大人关切,幸喜没有什么事。”众人道:“旨意问了些什么?”贾政道:“旨意问的是云南私带神枪一案。本上奏明是原任太师贾化的家人。主上一直记着我们先祖的名字,便问起来。我忙着磕头奏明先祖的名字是代化,主上便笑了,还降旨意说:‘前放兵部,后降府尹的,不是也叫贾化么?’”那时雨村也在傍边,倒吓了一跳,便问贾政道:“老先生怎么奏的?”贾政道:“我便慢慢奏道:’原任太师贾化是云南人;现任府尹贾某是浙江人。”主上又问:‘苏州刺史奏的贾范,是你一家子么?’我又磕头奏道:‘是。’主上便变色道:‘纵使家奴强占良民妻女,还成事么?’我一句不敢奏。主上又问道:‘贾范是你什么人?’我忙奏道:‘是远族。’主上哼了一声,降旨叫了出来。可不是诧事!”众人道:“本来也巧。怎么一连有这两件事?”贾政道:“事倒不奇,倒是都姓贾的不好。算来我们寒族人多,年代久了,各族都有。现在虽没有事,究竟主上记着一个“贾”字就不好。”众人说:“真是真,假是假,怕什么?”贾政道:“我心里巴不得不做官,只是不敢告老,现在我们家里两个世袭,这也无可奈何的。” 雨村道:“如今老先生仍是工部,想来京官是没有事的。”贾政道:“京官虽然没事,我究竟做过两次外任,也就说不齐了。”众人道:“二老爷的人品行事,我们都佩服的。就是令兄大老爷,也是个好人。只要在令侄辈上严紧些就是了。”贾政道:“我因在家的日子少,舍侄的事情不大查考,我心里也不甚放心。诸位今日提起,都是至相好,或者听见东宅的侄儿家有什么不奉规矩的事么?”众人道:“没听见别的,只是几位侍郎心里不大和睦,内监里头也有些。想来不怕什么,只要嘱咐那边令侄,诸事留神就是了。”众人说毕,举手而散。
\end{parag}


\begin{parag}
    贾政然后回家。众子侄等都迎接上来。贾政迎着请贾母的安,然后众子侄俱请了贾政的安,一同进府。王夫人等已到了荣禧堂迎接。贾政先到了贾母那里拜见了,陈述些违别的话。贾母问探春消息,贾政将许配探春的事都禀明了,还说:“儿子起身急促,难过重阳,虽没有亲见,听见那边亲家的人来说的极好。亲家老爷太太都说请老爷太太的安。还说今冬明春,大约还可调进京来。这便好了。如今闻得海疆有事,只怕那时还不能调。”贾母始则因贾政降调回来,知探春远在他乡,一无亲故,心下伤感;后听贾政将官事说明,探春安好,也便转悲为喜,便笑着叫贾政出去。然后弟兄相见,众子侄拜见,定了明日清晨拜祠堂。
\end{parag}


\begin{parag}
    贾政回到自己屋内,王夫人等见过,宝玉,贾琏替另拜见。贾政见了宝玉果然比起身之时脸面丰满,倒觉安静,独不知他心里糊涂,所以心甚喜欢,不以降调为念,心想幸亏老太太办理的好。又见宝钗沉厚更胜老时,兰儿文雅俊秀,便喜形于色。独见环儿仍是先前,究不甚钟爱。歇息了半天,忽然想起:“为何今日短了一人?”王夫人知是想着黛玉,前因家书未报:今日又刚到家,正是喜欢,不必直告,只说是病着。岂知宝玉的心里已如刀搅,因父亲到家只得把持心性伺候。王夫人设筵接风,子孙敬酒。风姐虽是侄媳,现办家事,也随了宝钗等敬酒。贾政便叫递了一巡酒,“都歇息去吧。”命众家人不必伺候,待明早拜过宗祠,然后进见。分派已定,贾政与王夫人说些别后的话,余者王夫人都不敢言。倒是贾政先提起王子腾的事来,王夫人也不敢悲戚。贾政又说蟠儿的事,王夫人只说他是自作自受;趁便也将黛玉已死的话告诉。贾政反吓了一惊,不觉掉下泪来连声叹息。王夫人也掌不住,也哭了。傍边彩云等即忙拉衣,王夫人止住,重又说些喜欢的话,便安寝了。
\end{parag}


\begin{parag}
    次日一早,至宗祠行礼,众子侄都随往。贾政便在祠旁厢房坐下,叫了贾珍,贾琏过来,问起家中事务。贾珍拣可说的说了。贾政又道:“我初回家,也不便来细细查问,只是听见外头说起你家里更不比从前,诸事要谨慎才好。你年纪也不小了,孩子们该管教管教,别叫他们在外头得罪人。琏儿也该听着。不是才回家就说你们,因我有所闻所以才说的。你们更该小心些。”贾珍等脸涨通红的,也只答应个“是”字,不敢说什么。贾政也就罢了。回归西府,众家人磕头毕,仍复进内,众女仆行礼,不必多赘。
\end{parag}


\begin{parag}
    只说宝玉因昨日贾政问起黛玉,王夫人答以有病,他便暗里伤心,直待贾政命他回去,一路上,已滴了好些眼泪。回到房中,见宝钗和袭人等说话,他便独坐外间纳闷。宝钗叫袭人送过茶去,知他必是怕老爷查问功课,所以如此,只得过来安慰。宝玉便借此走去向宝钗说:“你今晚先睡,我要定定神。这时更不如从前了三言倒忘两语,老爷瞧着不好。你先睡,叫袭人陪我略坐坐。”宝钗不便强他,点头应允。
\end{parag}


\begin{parag}
    宝玉出来便轻轻和袭人说,央他:“把紫鹃叫来,有话问他。但紫鹃见了我,脸上总是有气,组须得你去解劝开了再来才好。”袭人道:“你说要定神,我倒喜欢,怎么又定到这上头去了?有话你明儿问不得?”宝玉道:“我就是今晚得闲,明日倘或老爷叫干什么,便没空了。好姐姐,你快去叫他来。”袭人道:“他不是二奶奶叫是不来的。”宝玉道:“所以你得去说明了才好。”袭人道:“叫我说什么?”宝玉道:“你还不知道我的心和他的心么?都为的是林姑娘。你说我并不是负心,我如今叫你们弄成了一个负心的人了!”说着这话,他瞧瞧里间屋子,用手指着说:“他是我本不愿意的,都是老太太他们捉弄的。好端端把个林姑娘弄死了。就是他死,也该叫我见见,说个明白,他死了也不抱怨我嘎。你到底听见三姑娘他们说过的,临死恨怨我。那紫鹃为他们姑娘,也是恨的我了不得。你想我是无情的人么?晴雯到底是个丫头,也没有什么大好处,他死了,我实告诉你罢,我还做个祭文祭他呢。这是林姑娘亲眼见的。如今林姑娘死了,难道倒不及晴雯么?我连祭都不能祭一祭,况且林姑娘死了还有灵圣的,他想起来不是更抱怨我么?”袭人道:“你要祭就祭去,谁拦着你呢。”宝玉道:“我自从好了起来,就想要做一篇祭文,不知道如今怎么一点灵机都没有了。要祭别人呢,胡乱还使得,祭他是断断粗糙不得一点的。所以叫紫鹃来问他姑娘的心,他打那里看出来的。我没病的头里还想得出来,病后都记不得了。你倒说林姑娘已经好了,怎么忽然死的?他好的时候我不去,他怎么说来着?我病的时候,他不来,他又怎么说来着?所有他的东西,我诓过来,你二奶奶总不叫动,不知什么意思。”袭人道:“二奶奶惟恐你伤心罢了,还有什么呢。”宝玉道:“我不信。林姑娘既是念我为什么临死把诗稿烧了,不留给我做个纪念?又听见说天上有音乐响,必是他成了神,或是登了仙去。我虽见过了棺材,到底不知道棺材里有他没有。”袭人道:“你这话越发糊涂了,怎么一个人没死就搁在棺材里当死了的呢!”宝玉道:“不是嘎!大凡成仙的人,或是肉身去的,或是脱胎去的。好姐姐,你到底叫了紫鹃来。”袭人道:“如今等我细细的说明了你的心,他要肯来还好,要不肯来,还得费多少话;就是来了,见你也不肯细说。据我的主意:明日等二奶奶上去了,我慢慢的问他,或是倒可仔细。遇着闲空儿,我再慢慢的告诉你。宝玉道:“你说得也是,你不知道我心里的着急。”
\end{parag}


\begin{parag}
    正说着,麝月出来说:“二奶奶说,天已四更了,请二爷进去睡罢,袭人姐姐必是说高了兴了,忘了时候。”袭人听了,道:“可不是该睡了,有话明儿再说罢。”宝玉无奈,只得进去,又向袭人耳语道:“明儿好歹别忘了。”袭人笑道:“知道了。”麝月抹着脸笑道:“你们两个又闹鬼儿了。为什么不和二奶奶说明了,就到袭人那边睡去?由着你们说一夜,我们也不管。”宝玉摆手道:“不用言语。”袭人恨道:“小蹄子儿,你又嚼舌根,看我明儿撕你的嘴!”回头对宝玉道:“这不是你闹的?说了四更天的话。”一面说,一面送宝玉进屋,各人散去。
\end{parag}


\begin{parag}
    那夜宝玉无眠,到了次日,还想这事。只听得外头传进话来,说:“众亲朋因老爷回家,都要送戏接风。老爷再三推辞,说不必唱戏,竟在家里备了水酒,倒请亲朋过来大家谈谈。于是定了后儿摆席请人,所以进来告诉。”不知所请何人,下回分解。
\end{parag}