\chap{七十五}{開夜宴異兆發悲音 賞中秋新詞得佳讖}


\begin{parag}
    \begin{note}蒙回前總:賈珍居長,不能承先啓後丕震家風,兄弟問柳尋花,父子呼幺喝六,賈氏宗風,其墜地矣。安得不發先靈一嘆!\end{note}
\end{parag}


\begin{parag}
    \begin{note}庚:乾隆二十一年五月初七日對清。\end{note}
\end{parag}


\begin{parag}
    \begin{note}庚:缺中秋詩俟雪芹。\end{note}
\end{parag}


\begin{parag}
    話說尤氏從惜春處賭氣出來,正欲往王夫人處去。跟從的老嬤嬤們因悄悄的回道:“奶奶且別往上房去。纔有甄家的幾個人來,還有些東西,不知是作什麼機密事。奶奶這一去恐不便。”尤氏聽了道:“昨日聽見你爺說,看邸報甄家犯了罪,現今抄沒傢俬,調取進京治罪。怎麼又有人來?”老嬤嬤道:“正是呢。纔來了幾個女人,氣色不成氣色,慌慌張張的,想必有什麼瞞人的事情也是有的。”
\end{parag}


\begin{parag}
    尤氏聽了,便不往前去,仍往李氏這邊來了。\begin{note}庚雙夾:前文有探春一語,過至此回又用尤氏陪點,且輕輕淡染出甄家事故,此畫家未落墨之法也。\end{note}恰好太醫才診了脈去。李紈近日也略覺精爽了些,擁衾倚枕,坐在牀上,正欲一二人來說些閒話。因見尤氏進來不似往日和藹可親,只呆呆的坐著。李紈因問道:“你過來了這半日,可在別屋裏喫些東西沒有?只怕餓了。”命素雲瞧有什麼新鮮點心揀了來。尤氏忙止道:“不必,不必。你這一向病著,那裏有什麼新鮮東西。況且我也不餓。”李紈道:“昨日他姨娘家送來的好茶麪子,倒是對碗來你喝罷。”說畢,便吩咐人去對茶。尤氏出神無語。跟來的丫頭媳婦們因問:“奶奶今日中晌尚未洗臉,這會子趁便可淨一淨好?”尤氏點頭。李紈忙命素雲來取自己的妝奩。素雲一面取來,一面將自己的胭粉拿來,笑道:“我們奶奶就少這個。奶奶不嫌髒,這是我的,能著用些。”李紈道:“我雖沒有,你就該往姑娘們那裏取去。怎麼公然拿出你的來。幸而是他,若是別人,豈不惱呢。”尤氏笑道:“這又何妨。自來我凡過來,誰的沒使過,今日忽然又嫌髒了?”一面說,一面盤膝坐在炕沿上。銀蝶上來忙代爲卸去腕鐲戒指,又將一大袱手巾蓋在下截,將衣裳護嚴。小丫鬟炒豆兒捧了一大盆溫水走至尤氏跟前,只彎腰捧著。李紈道:“怎麼這樣沒規矩。”銀蝶笑道:“說一個個沒機變的,說一個葫蘆就是一個瓢。奶奶不過待咱們寬些,在家裏不管怎樣罷了,你就得了意,不管在家出外,當著親戚也只隨著便了。”尤氏道:“你隨他去罷,橫豎洗了就完事了。”炒豆兒忙趕著跪下。尤氏笑道:“我們家下大小的人只會講外面假禮假體面,究竟作出來的事都夠使的了。”\begin{note}庚雙夾:按尤氏犯七出之條不過只是“過於從夫”四字,此世間婦人之常情耳。其心術慈厚寬順竟可出於阿鳳之上,特用之明犯七出之人從公一論,可知賈宅中暗犯七出之人亦不少。似明犯者猶 懾端。其飾己非而揚人惡者,陰昧僻譎之流,實不能容於世者也。此爲打草驚蛇法,實寫邢夫人也。\end{note}李紈聽如此說,便知他已知道昨夜的事,因笑道:“你這話有因,誰作事究竟夠使了?”尤氏道:“你倒問我!你敢是病著死過去了!”
\end{parag}


\begin{parag}
    一語未了,只見人報:“寶姑娘來了。”忙說快請時,寶釵已走進來。尤氏忙擦臉起身讓坐,因問:“怎麼一個人忽然走來,別的姊妹都怎麼不見?”寶釵道: “正是我也沒有見他們。只因今日我們奶奶身上不自在,家裏兩個女人也都因時症未起炕,別的靠不得,我今兒要出去伴著老人家夜裏作伴兒。要去回老太太,太太,我想又不是什麼大事,且不用提,等好了我橫豎進來的,所以來告訴大嫂子一聲。”李紈聽說,只看著尤氏笑。尤氏也只看著李紈笑。一時尤氏盥沐已畢,大家吃麪茶。李紈因笑道:“既這樣,且打發人去請姨娘的安,問是何病。我也病著,不能親自來的。好妹妹,你去只管去,我自打發人去到你那裏去看屋子。你好歹住一兩天還進來,別叫我落不是。”寶釵笑道:“落什麼不是呢,這也是通共常情,你又不曾賣放了賊。依我的主意,也不必添人過去,竟把雲丫頭請了來,你和他住一兩日,豈不省事。”尤氏道:“可是史大妹妹往那裏去了?”寶釵道:“我纔打發他們找你們探丫頭去了,叫他同到這裏來,我也明白告訴他。”
\end{parag}


\begin{parag}
    正說著,果然報:“雲姑娘和三姑娘來了。”大家讓坐已畢,寶釵便說要出去一事,探春道:“很好。不但姨媽好了還來的,就便好了不來也使得。”尤氏笑道:“這話奇怪,怎麼攆起親戚來了?”探春冷笑道:“正是呢,有叫人攆的,不如我先攆。親戚們好,也不在必要死住著纔好。咱們倒是一家子親骨肉呢,一個個不象烏眼雞,恨不得你吃了我,我吃了你!”尤氏忙笑道:“我今兒是那裏來的晦氣,偏都碰著你姊妹們的氣頭兒上了。”探春道:“誰叫你趕熱竈來了!”因問: “誰又得罪了你呢?”因又尋思道:“四丫頭不犯羅唣你,卻是誰呢?”尤氏只含糊答應。探春知他畏事不肯多言,因笑道:“你別裝老實了。除了朝廷治罪,沒有砍頭的,你不必畏頭畏尾。實告訴你罷,我昨日把王善保家那老婆子打了,我還頂著個罪呢。不過背地裏說我些閒話,難道他還打我一頓不成!”寶釵忙問因何又打他,探春悉把昨夜怎的抄檢,怎的打他,一一說了出來。尤氏見探春已經說了出來,便把惜春方纔之事也說了出來。探春道:“這是他的僻性,孤介太過,我們再傲不過他的。”又告訴他們說:“今日一早不見動靜,打聽鳳辣子又病了。我就打發我媽媽出去打聽王善保家的是怎樣。回來告訴我說,王善保家的捱了一頓打,大太太嗔著他多事。”尤氏李紈道:“這倒也是正理。”探春冷笑道:“這種掩飾誰不會作,且再瞧就是了。”尤氏李紈皆默無所答。一時估著前頭用飯,湘雲和寶釵回房打點衣衫,不在話下。
\end{parag}


\begin{parag}
    尤氏等遂辭了李紈,往賈母這邊來。賈母歪在榻上,王夫人說甄家因何獲罪,如今抄沒了家產,回京治罪等語。賈母聽了正不自在,恰好見他姊妹來了,因問: “從那裏來的?可知鳳姐妯娌兩個的病今日怎樣?”尤氏等忙回道:“今日都好些。”賈母點頭嘆道:“咱們別管人家的事,且商量咱們八月十五日賞月是正經。”\begin{note}庚雙夾:賈母已看破狐悲兔死,故不改正,聊來自遣耳。\end{note}王夫人笑道:“都已預備下了。不知老太太揀那裏好,只是園裏空,夜晚風冷。”賈母笑道: “多穿兩件衣服何妨,那裏正是賞月的地方,豈可倒不去的。”說話之間,早有媳婦丫鬟們抬過飯桌來,王夫人尤氏等忙上來放箸捧飯。賈母見自己的幾色菜已擺完,另有兩大捧盒內捧了幾色菜來,便知是各房另外孝敬的舊規矩。賈母因問:“都是些什麼?上幾次我就吩咐,如今可以把這些蠲了罷,你們還不聽。如今比不得在先輻輳的時光了。”鴛鴦忙道:“我說過幾次,都不聽,也只罷了。” 王夫人笑道:“不過都是家常東西。今日我喫齋沒有別的。那些麪筋豆腐老太太又不大甚愛喫,只揀了一樣椒油蓴齏醬來。”賈母笑道:“這樣正好,正想這個喫。” 鴛鴦聽說,便將碟子挪在跟前。寶琴一一的讓了,方歸坐。賈母便命探春來同吃。探春也都讓過了,便和寶琴對面坐下。待書忙去取了碗來。鴛鴦又指那幾樣菜道: “這兩樣看不出是什麼東西來,大老爺送來的。這一碗是雞髓筍,是外頭老爺送上來的。”一面說,一面就只將這碗筍送至桌上。賈母略嚐了兩點,便命:“將那兩樣著人送回去,就說我吃了。以後不必天天送,我想喫自然來要。”媳婦們答應著,仍送過去,不在話下。
\end{parag}


\begin{parag}
    賈母因問:“有稀飯喫些罷了。”尤氏早捧過一碗來,說是紅稻米粥。賈母接來吃了半碗,便吩咐:“將這粥送給鳳哥兒喫去,”又指著“這一碗筍和這一盤風醃果子狸給顰兒寶玉兩個喫去,那一碗肉給蘭小子喫去。”又向尤氏道:“我吃了,你就來吃了罷。”尤氏答應,待賈母漱口洗手畢,賈母便下地和王夫人說閒話行食。尤氏告坐。探春寶琴二人也起來了,笑道:“失陪,失陪。”尤氏笑道:“剩我一個人,大排桌的喫不慣。”賈母笑道:“鴛鴦琥珀來趁勢也喫些,又作了陪客。”尤氏笑道:“好,好,好,我正要說呢。”賈母笑道:“看著多多的人喫飯,最有趣的。”又指銀蝶道:“這孩子也好,也來同你主子一塊來喫,等你們離了我,再立規矩去。”尤氏道:“快過來,不必裝假。”賈母負手看著取樂。因見伺候添飯的人手內捧著一碗下人的米飯,尤氏喫的仍是白粳米飯,賈母問道:“你怎麼昏了,盛這個飯來給你奶奶。”那人道:“老太太的飯喫完了。今日添了一位姑娘,所以短了些。”鴛鴦道:“如今都是可著頭做帽子了,要一點兒富餘也不能的。”王夫人忙回道:“這一二年旱澇不定,田上的米都不能按數交的。這幾樣細米更艱難了,所以都可著喫的多少關去,生恐一時短了,買的不順口。”賈母笑道:“這正是‘巧媳婦做不出沒米的粥’來。”衆人都笑起來。鴛鴦道:“既這然,就去把三姑娘的飯拿來添也是一樣,就這樣笨。”尤氏笑道:“我這個就夠了,也不用取去。”鴛鴦道:“你夠了,我不會喫的。”地下的媳婦們聽說,方忙著取去了。\begin{note}庚雙夾:總伏下文。\end{note}一時王夫人也去用飯,這裏尤氏直陪賈母說話取笑。
\end{parag}


\begin{parag}
    到起更的時候,賈母說:“黑了,過去罷。”尤氏方告辭出來。走至大門前上了車,銀蝶坐在車沿上。衆媳婦放下簾子來,便帶著小丫頭們先直走過那邊大門口等著去了。因二府之門相隔沒有一箭之路,每日家常來往不必定要周備,況天黑夜晚之間回來的遭數更多,所以老嬤嬤帶著小丫頭,只幾步便走了過來。兩邊大門上的人都到東西街口,早把行人斷住。尤氏大車上也不用牲口,只用七八個小廝挽環拽輪,輕輕的便推拽過這邊階磯上來。於是衆小廝退過獅子以外,衆嬤嬤打起簾子,銀蝶先下來,然後攙下尤氏來。大小七八個燈籠照的十分真切。尤氏因見兩邊獅子下放著四五輛大車,便知系來赴賭之人所乘,遂向銀蝶衆人道:“你看,坐車的是這樣,騎馬的還不知有幾個呢。馬自然在圈裏拴著,咱們看不見。也不知道他孃老子掙下多少錢與他們,這麼開心兒。”一面說,一面已到了廳上。賈蓉之妻帶領家下媳婦丫頭們,也都秉燭接了出來。尤氏笑道:“成日家我要偷著瞧瞧他們,也沒得便。今兒倒巧,就順便打他們窗戶跟前走過去。”衆媳婦答應著,提燈引路,又有一個先去悄悄的知會伏侍的小廝們不要失驚打怪。於是尤氏一行人悄悄的來至窗下,只聽裏面稱三贊四,耍笑之音雖多,\begin{note}庚雙夾:妙!先畫贏家。\end{note}又兼有恨五罵六,忿怨之聲亦不少。\begin{note}庚雙夾:妙!又畫輸家。\end{note}
\end{parag}


\begin{parag}
    原來賈珍近因居喪,每不得遊頑曠蕩,又不得觀優聞樂作遣。無聊之極,便生了個破悶之法。日間以習射爲由,請了各世家弟兄及諸富貴親友來較射。因說: “白白的只管亂射,終無裨益,不但不能長進,而且壞了式樣,必須立個罰約,賭個利物,大家纔有勉力之心。”因此在天香樓下箭道內立了鵠子,皆約定每日早飯後來射鵠子。賈珍不肯出名,便命賈蓉作局家。這些來的皆繫世襲公子,人人家道豐富,且都在少年,正是鬥雞走狗,問柳評花的一干遊蕩紈褲。因此大家議定,每日輪流作晚飯之主,──每日來射,不便獨擾賈蓉一人之意。於是天天宰豬割羊,屠鵝戮鴨,好似臨潼鬥寶一般,都要賣弄自己的好廚役好烹炮。不到半月工夫,賈赦賈政聽見這般,不知就裏,反說這纔是正理,文既誤矣,武事當亦該習,況在武蔭之屬。兩處遂也命賈環、賈琮、寶玉、賈蘭等四人於飯後過來,跟著賈珍習射一回,方許回去。
\end{parag}


\begin{parag}
    賈珍之志不在此,再過一二日便漸次以歇臂養力爲由,晚間或抹抹骨牌,賭個酒東而已,至後漸次至錢。如今三四月的光景,竟一日一日賭勝於射了,公然鬥葉擲骰,放頭開局,夜賭起來。家下人藉此各有些進益,巴不得的如此,所以竟成了勢了。外人皆不知一字。近日邢夫人之胞弟邢德全也酷好如此,故也在其中。又有薛蟠,頭一個慣喜送錢與人的,見此豈不快樂。邢德全雖系邢夫人之胞弟,卻居心行事大不相同。這個邢德全只知喫酒賭錢,眠花宿柳爲樂,手中濫漫使錢,待人無二心,好酒者喜之,不飲者則不去親近,無論上下主僕皆出自一意,並無貴賤之分,因此都喚他“傻大舅”。薛蟠早已出名的呆大爺。今日二人皆湊在一處,都愛 “搶新快”爽利,便又會了兩家,在外間炕上“搶新快”。別的又有幾家在當地下大桌上打公番。裏間又一起斯文些的,抹骨牌打天九。此間伏侍的小廝都是十五歲以下的孩子,若成丁的男子到不了這裏,故仁戲角至窗外偷看。其中有兩個十六七歲孌童以備奉酒的,都打扮的粉妝玉琢。今日薛蟠又輸了一張,正沒好氣,幸而擲第二張完了,算來除翻過來倒反贏了,心中只是興頭起來。賈珍道:“且打住,吃了東西再來。”因問那兩處怎樣。裏頭打天九的,也作了帳等喫飯。打公番的未清,且不肯喫。於是各不能催,先擺下一大桌,賈珍陪著喫,命賈蓉落後陪那一起。薛蟠興頭了,便摟著一個孌童喫酒,又命將酒去敬邢傻舅。傻舅輸家,沒心緒,吃了兩碗,便有些醉意,嗔著兩個孌童只趕著贏家不理輸家了,因罵道:“你們這起兔子,就是這樣專洑上水。天天在一處,誰的恩你們不沾,只不過我這一會子輸了幾兩銀子,你們就三六九等了。難道從此以後再沒有求著我們的事了!”衆人見他帶酒,忙說:“很是,很是。果然他們風俗不好。”因喝命:“快敬酒賠罪。”兩個孌童都是演就的局套,忙都跪下奉酒,說:“我們這行人,師父教的不論遠近厚薄,只看一時有錢有勢就親敬,便是活佛神仙,一時沒了錢勢了,也不許去理他。況且我們又年輕,又居這個行次,求舅太爺體恕些我們就過去了。”\begin{note}庚雙夾:調侃,罵死世人。庚眉:此一段孌童語句太真,反不得其爲錢爲勢之神,當改作委曲認罪語方妥。\end{note}說著,便舉著酒俯膝跪下。邢大舅心內雖軟了,只還故作怒意不理。衆人又勸道:“這孩子是實情話。老舅是久慣憐香惜玉的,如何今日反這樣起來?若不喫這酒,他兩個怎樣起來。”邢大舅已撐不住了,便說道:“若不是衆位說,我再不理。”說著,方接過來一氣喝乾了。又斟一碗來。這邢大舅便酒勾往事,醉露真情起來,乃拍案對賈珍嘆道:“怨不的他們視錢如命。多少世宦大家出身的,若提起‘錢勢’二字,連骨肉都不認了。老賢甥,昨日我和你那邊的令伯母賭氣,你可知道否?” 賈珍道:“不曾聽見。”邢大舅嘆道:“就爲錢這件混帳東西。利害,利害!”賈珍深知他與邢夫人不睦,每遭邢夫人棄惡,扳出怨言,因勸道:“老舅,你也太散漫些。若只管花去,有多少給老舅花的。”邢大舅道:“老賢甥,你不知我邢家底裏。我母親去世時我尚小,世事不知。他姊妹三個人,只有你令伯母年長出閣,一分傢俬都是他把持帶來。如今二家姐雖也出閣,他家也甚艱窘,三家姐尚在家裏,一應用度都是這裏陪房王善保家的掌管。我便來要錢,也非要的是你賈府的,我邢家傢俬也就夠我花了。無奈竟不得到手,所以有冤無處訴。”\begin{note}庚雙夾:衆惡之必察也。今邢夫人一人,賈母先惡之,恐賈母心偏,亦可解之。若賈璉阿鳳之怨,恐兒女之私,亦可解之。若探春之怒,恐女子不識大而知小,亦可解之。今又忽用乃弟一怨,吾不知將又何如矣。\end{note}賈珍見他酒後叨叨,恐人聽見不雅,連忙用話解勸。
\end{parag}


\begin{parag}
    外面尤氏聽得十分真切,乃悄向銀蝶笑道:“你聽見了?這是北院裏大太太的兄弟抱怨他呢。可憐他親兄弟還是這樣說,這就怨不得這些人了。”因還要聽時,正值打公番者也歇住了,要喫酒。因有一個問道:“方纔是誰得罪了老舅,我們竟不曾聽明白,且告訴我們評評理。”邢德全見問,便把兩個孌童不理輸的只趕贏的話說了一遍。這一個年少的紈褲道:“這樣說,原可惱的,怨不得舅太爺生氣。我且問你兩個:舅太爺雖然輸了,輸的不過是銀子錢,並沒有輸丟了,怎就不理他了?”說著,衆人大笑起來,連邢德全也噴了一地飯。尤氏在外面那牡啐了一口,罵道:“你聽聽,這一起子沒廉恥的小挨刀的,才丟了腦袋骨子,就胡唚嚼毛了。再肏攮下黃湯去,還不知唚出些什麼來呢。”一面說,一面便進去卸妝安歇。至四更時,賈珍方散,往佩鳳房裏去了。
\end{parag}


\begin{parag}
    次日起來,就有人回西瓜月餅都全了,只待分派送人。賈珍吩咐佩鳳道:“你請你奶奶看著送罷,我還有別的事呢。”佩鳳答應去了,回了尤氏,尤氏只得一一分派遣人送去。一時佩鳳又來說:“爺問奶奶,今兒出門不出?說咱們是孝家,明兒十五過不得節,今兒晚上倒好,可以大家應個景兒,喫些瓜餅酒。”尤氏道: “我倒不願出門呢。那邊珠大奶奶又病了,鳳丫頭又睡倒了,我再不過去,越發沒個人了。況且又不得閒,應什麼景兒。”佩鳳道:“爺說了,今兒已辭了衆人,直等十六纔來呢,好歹定要請奶奶喫酒的。”尤氏笑道:“請我,我沒的還席。”佩鳳笑著去了,一時又來笑道:“爺說,連晚飯也請奶奶喫,好歹早些回來,叫我跟了奶奶去呢。”尤氏道:“這樣,早飯喫什麼?快些吃了,我好走。”佩鳳道:“爺說早飯在外頭喫,請奶奶自己喫罷。”尤氏問道:“今日外頭有誰?”佩鳳道: “聽見說外頭有兩個南京新來的,倒不知是誰。”說話之間,賈蓉之妻也梳妝了來見過。少時擺上飯來,尤氏在上,賈蓉之妻在下相陪,婆媳二人喫畢飯。尤氏便換了衣服,仍過榮府來,至晚方回去。
\end{parag}


\begin{parag}
    果然賈珍煮了一口豬,燒了一腔羊,餘者桌菜及果品之類,不可勝記,就在會芳園叢綠堂中,屏開孔雀,褥設芙蓉,帶領妻子姬妾。先飯後酒,開懷賞月作樂。將一更時分,真是風清月朗,上下如銀。賈珍因要行令,尤氏便叫佩鳳等四個人也都入席,下面一溜坐下,猜枚划拳,飲了一回。賈珍有了幾分酒,益發高興,便命取了一竿紫竹簫來,命佩鳳吹簫,文花唱曲,喉清嗓嫩,真令人魄醉魂飛。唱罷復又行令。那天將有三更時分,賈珍酒已八分。大家正添衣飲茶,換盞更酌之際,忽聽那邊牆下有人長嘆之聲。大家明明聽見,都悚然疑畏起來。\begin{note}庚雙夾:餘亦悚然疑畏。\end{note}賈珍忙厲聲叱吒,問:“誰在那裏?”連問幾聲,沒有人答應。尤氏道:“必是牆外邊家裏人也未可知。”賈珍道:“胡說。這牆四面皆無下人的房子,況且那邊又緊靠著祠堂,\begin{note}庚雙夾:奇絕神想,餘更爲之懼矣。\end{note}焉得有人。”一語未了,只聽得一陣風聲,竟過牆去了。恍惚聞得祠堂內槅扇開闔之聲。只覺得風氣森森,比先更覺涼颯起來,月色慘淡,也不似先明朗。衆人都覺毛髮倒豎。賈珍酒已醒了一半,只比別人撐持得住些,心下也十分疑畏,便大沒興頭起來。勉強又坐了一會子,就歸房安歇去了。次日一早起來,乃是十五日,帶領衆子侄開祠堂行朔望之禮,細查祠內,都仍是照舊好好的,並無怪異之跡。賈珍自爲醉後自怪,也不提此事。禮畢,仍閉上門,看著鎖禁起來。\begin{note}庚雙夾:未寫榮府慶中秋,卻先寫寧府開夜宴,未寫榮府數盡,先寫寧府異道。蓋寧乃家宅,凡有關於吉凶者,故必先示之。且列祖祠在此,豈無得而警乎?凡人先人雖遠,然氣運相關,必有之理也。非寧府之祖獨有感應也。\end{note}
\end{parag}


\begin{parag}
    賈珍夫妻至晚飯後方過榮府來。只見賈赦賈政都在賈母房內坐著說閒話,與賈母取笑。賈璉,寶玉,賈環,賈蘭皆在地下侍立。賈珍來了,都一一見過。說了兩句話後,賈母命坐,賈珍方在近門小杌子上告了坐,警身側坐。賈母笑問道:“這兩日你寶兄弟的箭如何了?”賈珍忙起身笑道:“大長進了,不但樣式好,而且弓也長了一個力氣。”賈母道:“這也夠了,且別貪力,仔細努傷。”賈珍忙答應幾個“是”。賈母又道:“你昨日送來的月餅好,西瓜看著好,打開卻也罷了。”賈珍笑道:“月餅是新來的一個專做點心的廚子,我試了試果然好,纔敢做了孝敬。西瓜往年都還可以,不知今年怎麼就不好了。”賈政道:“大約今年雨水太勤之故。”賈母笑道:“此時月已上了,咱們且去上香。”說著,便起身扶著寶玉的肩,帶領衆人齊往園中來。
\end{parag}


\begin{parag}
    當下園之正門俱已大開,吊著羊角大燈。嘉蔭堂前月臺上,焚著斗香,秉著風燭,陳獻著瓜餅及各色果品。邢夫人等一干女客皆在裏面久候。真是月明燈綵,人氣香菸,晶豔氤氳,不可形狀。地下鋪著拜毯錦褥。賈母盥手上香拜畢,於是大家皆拜過。賈母便說:“賞月在山上最好。”因命在那山脊上的大廳上去。衆人聽說,就忙著在那裏去鋪設。賈母且在嘉蔭堂中喫茶少歇,說些閒話。一時,人回:“都齊備了。”賈母方扶著人上山來。王夫人等因說:“恐石上苔滑,還是坐竹椅上去。”賈母道:“天天有人打掃,況且極平穩的寬路,何必不疏散疏散筋骨。”於是賈赦賈政等在前導引,又是兩個老婆子秉著兩把羊角手罩,鴛鴦、琥珀、尤氏等貼身攙扶,邢夫人等在後圍隨,從下逶迤而上,不過百餘步,至山之峯脊上,便是這座敞廳。因在山之高脊,故名曰凸碧山莊。於廳前平臺上列下桌椅,又用一架大圍屏隔作兩間。凡桌椅形式皆是圓的,特取團圓之意。上面居中賈母坐下,左垂首賈赦、賈珍、賈璉、賈蓉,右垂首賈政、寶玉、賈環、賈蘭,團團圍坐。只坐了半壁,下面還有半壁餘空。賈母笑道:“常日倒還不覺人少,今日看來,還是咱們的人也甚少,算不得甚麼。\begin{note}庚雙夾:未飲先感人丁,總是將散之兆。\end{note}想當年過的日子,到今夜男女三四十個,何等熱鬧。今日就這樣,太少了。待要再叫幾個來,他們都是有父母的,家裏去應景,不好來的。如今叫女孩們來坐那邊罷。”於是令人向圍屏後邢夫人等席上將迎春,探春,惜春三個請出來。賈璉寶玉等一齊出坐,先盡他姊妹坐了,然後在下方依次坐定。賈母便命折一枝桂花來,命一媳婦在屏後擊鼓傳花。若花到誰手中,飲酒一杯,罰說笑話一個。\begin{note}庚雙夾:不犯前幾次飲酒。\end{note}於是先從賈母起,次賈赦,一一接過。鼓聲兩轉,恰恰在賈政手中住了,\begin{note}庚雙夾:奇妙!偏在政老手中,竟能使政老一謔,真大文章矣。\end{note}只得飲了酒。衆姊妹弟兄皆你悄悄的扯我一下,我暗暗的又捏你一把,都含笑倒要聽是何笑話。\begin{note}庚雙夾:餘也要細聽。\end{note}賈政見賈母喜悅,只得承歡。方欲說時,賈母又笑道:“若說的不笑了,還要罰。”賈政笑道:“只得一個,說來不笑,也只好受罰了。”因笑道:“一家子一個人最怕老婆的。”才說了一句,大家都笑了。因從不曾見賈政說過笑話,所以才笑。\begin{note}庚雙夾:是極,摹神之至。\end{note}賈母笑道:“這必是好的。”賈政笑道:“若好,老太太多喫一杯。”賈母笑道:“自然。”賈政又說道:“這個怕老婆的人從不敢多走一步。偏是那日是八月十五,到街上買東西,便遇見了幾個朋友,死活拉到家裏去喫酒。不想喫醉了,便在朋友家睡著了,第二日才醒,後悔不及,只得來家賠罪。他老婆正洗腳,說:‘既是這樣,你替我舔舔就饒你。’這男人只得給他舔,未免噁心要吐。他老婆便惱了,要打,說:‘你這樣輕狂!’唬得他男人忙跪下求說:‘並不是奶奶的腳髒。只因昨晚喫多了黃酒,又吃了幾塊月餅餡子,所以今日有些作酸呢。’”說的賈母與衆人都笑了。\begin{note}庚雙夾:這方是賈政之謔\end{note}賈政忙斟了一杯,送與賈母。賈母笑道:“既這樣,快叫人取燒酒來,別叫你們受累。”衆人又都笑起來。
\end{parag}


\begin{parag}
    於是又擊鼓,便從賈政傳起,可巧傳至寶玉鼓止。寶玉因賈政在坐,自是踧踖不安,花偏又在他手內,因想:“說笑話倘或不發笑,又說沒口才,連一笑話不能說,何況是別的,這有不是。若說好了,又說正經的不會,只慣油嘴貧舌,更有不是。不如不說的好。”\begin{note}庚雙夾:實寫舊日往事。\end{note}乃起身辭道:“我不能說笑話,求再限別的罷了。”賈政道:“既這樣,限一個‘秋’字,就即景作一首詩。若好,便賞你,若不好,明日仔細。”賈母忙道:“好好的行令,如何又要作詩?”賈政道:“他能的。”賈母聽說,“既這樣就作。”命人取了紙筆來,賈政道:“只不許用那些冰玉晶銀彩光明素等樣堆砌字眼,要另出己見,試試你這幾年的情思。”寶玉聽了,碰在心坎上,遂立想了四句,向紙上寫了,呈與賈政看,道是……(按:此處有缺文。)賈政看了,點頭不語。賈母見這般,知無甚大不好,便問:“怎麼樣?”賈政因欲賈母喜悅,便說:“難爲他。只是不肯唸書,到底詞句不雅。”賈母道:“這就罷了。他能多大,定要他做才子不成!這就該獎勵他,以後越發上心了。”賈政道:“正是。”因回頭命個老嬤嬤出去吩咐書房內的小廝,“把我海南帶來的扇子取兩把給他。”寶玉忙拜謝,仍復歸座行令。當下賈蘭見獎勵寶玉,他便出席也做一首遞與賈政看時,寫道是……(按:此處有缺文。)賈政看了喜不自勝,遂並講與賈母聽時,賈母也十分歡喜,也忙令賈政賞他。於是大家歸坐,復行起令來。
\end{parag}


\begin{parag}
    這次在賈赦手內住了,只得吃了酒,說笑話。因說道:“一家子一個兒子最孝順。偏生母親病了,各處求醫不得,便請了一個鍼灸的婆子來。婆子原不知道脈理,只說是心火,如今用鍼灸之法,鍼灸鍼灸就好了。這兒子慌了,便問:‘心見鐵即死,如何針得?’婆子道:‘不用針心,只針肋條就是了。’兒子道,‘肋條離心甚遠,怎麼就好?’婆子道:‘不妨事。你不知天下父母心偏的多呢。’”衆人聽說,都笑起來。賈母也只得喫半杯酒,半日笑道:“我也得這個婆子針一針就好了。”賈赦聽說,便知自己出言冒撞,賈母疑心,忙起身笑與賈母把盞,以別言解釋。賈母亦不好再提,且行起令來。
\end{parag}


\begin{parag}
    不料這次花卻在賈環手裏。賈環近日讀書稍進,其脾味中不好務正也與寶玉一樣,故每常也好看些詩詞,專好奇詭仙鬼一格。今見寶玉作詩受獎,他便技癢,只當著賈政不敢造次。如今可巧花在手中,便也索紙筆來立揮一絕與賈政。\begin{note}庚雙夾:前文賈政戲謔已是異文,而賈環作詩更奇中又奇之奇文也,總在人意料之外。竟有人曰:“賈環如何又有好詩,似前文不搭後語矣。”蓋不可向說問。賈環亦榮府公子正脈,雖少年頑劣,現今小兒之常情耳。讀書豈無長進之理哉?況賈政之教是弟子目已大覺疏忽矣。若是賈環連一平仄也不知,豈榮府是尋常膏粱不知詩書之家哉?然後之寶玉之一種情思,正非有益子總明不得謂比諸人皆妙者也。\end{note}賈政看了,亦覺罕異,只是詞句終帶著不樂讀書之意,遂不悅道:“可見是弟兄了。發言吐氣總屬邪派,將來都是不由規矩準繩,一起下流貨。妙在古人中有‘二難 ’,你兩個也可以稱‘二難’了。只是你兩個的‘難’字,卻是作難以教訓之‘難’字講纔好。哥哥是公然以溫飛卿自居,如今兄弟又自爲曹唐再世了。”說的賈赦等都笑了。賈赦乃要詩瞧了一遍,連聲贊好,道:“這詩據我看甚是有骨氣。想來咱們這樣人家,原不比那起寒酸,定要‘雪窗熒火’,一日蟾宮折桂,方得揚眉吐氣。咱們的子弟都原該讀些書,不過比別人略明白些,可以做得官時就跑不了一個官的。何必多費了工夫,反弄出書呆子來。所以我愛他這詩,竟不失咱們侯門的氣概。”因回頭吩咐人去取了自己的許多玩物來賞賜與他。因又拍著賈環的頭,笑道:“以後就這麼做去,方是咱們的口氣,將來這世襲的前程定跑不了你襲呢。”賈政聽說,忙勸說:“不過他胡謅如此,那裏就論到後事了。” 說著便斟上酒,又行了一回令。\begin{note}庚雙夾:便又輕輕抹去也。\end{note}賈母便說:“你們去罷。自然外頭還有相公們候著,也不可輕忽了他們。況且二更多了,你們散了,再讓我和姑娘們多樂一回,好歇著了。”賈赦等聽了,方止了令,又大家公進了一杯酒,方帶著子侄們出去了。要知端詳,再聽下回。
\end{parag}


\begin{parag}
    \begin{note}蒙回末總:下回有一篇極清雅文字,下幅有半篇極整齊文字,故先敘搶快摸牌,沉湎酒色爲反振,有駿馬下坡勢、鳥將翔勢。\end{note}
\end{parag}


\begin{parag}
    \begin{note}蒙回末總:看聚賭一段,宛然“宵小羣居衆日圖”,看賞月一段,又宛然“望族序齒燕毛錄”,說火則熱,而說水則寒,文心故,無所不可。\end{note}
\end{parag}
