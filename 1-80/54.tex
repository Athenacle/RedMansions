\chap{五十四}{史太君破陳腐舊套 王熙鳳效戲彩斑衣}


\begin{parag}
    \begin{note}庚:首回楔子內雲“古今小說千部共成一套”云云,猶未泄真。今借老太君一寫,是勸後來胸中無機軸之諸君子不可動筆作書。\end{note}
\end{parag}


\begin{parag}
    \begin{note}庚:鳳姐乃太君之要緊陪堂,今題“斑衣戲彩”是作者酬我阿鳳之勞,特貶賈珍璉輩之無能耳。\end{note}
\end{parag}


\begin{parag}
    \begin{note}蒙回前總:積德於今到子孫,都中旺族首吾門。可憐立業英雄輩,遺脈誰知祖父恩。\end{note}
\end{parag}


\begin{parag}
    卻說賈珍賈璉暗暗預備下大簸籮的錢,聽見賈母說“賞”,他們也忙命小廝們快撒錢。只聽滿臺錢響,賈母大悅。
\end{parag}


\begin{parag}
    二人遂起身,小廝們忙將一把新暖銀壺捧在賈璉手內,隨了賈珍趨至裏面。賈珍先至李嬸席上,躬身取下杯來,回身,賈璉忙斟了一盞;然後便至薛姨媽席上,也斟了。二人忙起身笑說:“二位爺請坐著罷了,何必多禮。”於是除邢王二夫人,滿席都離了席,俱垂手旁侍。賈珍等至賈母榻前,因榻矮,二人便屈膝跪了。賈珍在先捧杯,賈璉在後捧壺。雖止二人奉酒,那賈環弟兄等,卻也是排班按序,一溜隨著他二人進來,見他二人跪下,也都一溜跪下。寶玉也忙跪下了。史湘雲悄推他笑道:“你這會又幫著跪下作什麼?有這樣,你也去斟一巡酒豈不好?”寶玉悄笑道:“再等一會子再斟去。”說著,等他二人斟完起來,方起來。又與邢夫人王夫人斟過來。賈珍笑道:“妹妹們怎麼樣呢?”賈母等都說:“你們去罷,他們倒便宜些。”說了,賈珍等方退出。
\end{parag}


\begin{parag}
    當下天未二鼓,戲演的是《八義》中《觀燈》八出。正在熱鬧之際,寶玉因下席往外走。賈母因說:“你往那裏去!外頭爆竹利害,仔細天上吊下火紙來燒了。”寶玉回說:“不往遠去,只出去就來。”賈母命婆子們好生跟著。於是寶玉出來,只有麝月秋紋並幾個小丫頭隨著。賈母因說:“襲人怎麼不見?他如今也有些拿大了,單支使小女孩子出來。”王夫人忙起身笑回道:“他媽前日沒了,因有熱孝,不便前頭來。”賈母聽了點頭,又笑道:“跟主子卻講不起這孝與不孝。若是他還跟我,難道這會子也不在這裏不成?皆因我們太寬了,有人使,不查這些,竟成了例了。”鳳姐兒忙過來笑回道:“今兒晚上他便沒孝,那園子裏也須得他看著,燈燭花炮最是耽險的。這裏一唱戲,園子裏的人誰不偷來瞧瞧。他還細心,各處照看照看。況且這一散後寶兄弟回去睡覺,各色都是齊全的。若他再來了,衆人又不經心,散了回去,鋪蓋也是冷的,茶水也不齊備,各色都不便宜,所以我叫他不用來,只看屋子。散了又齊備,我們這裏也不耽心,又可以全他的禮,豈不三處有益。老祖宗要叫他,我叫他來就是了。”賈母聽了這話,忙說:“你這話很是,比我想的周到,快別叫他了。但只他媽幾時沒了,我怎麼不知道。”鳳姐笑道: “前兒襲人去親自回老太太的,怎麼倒忘了。”賈母想了一想笑說:“想起來了。我的記性竟平常了。”衆人都笑說:“老太太那裏記得這些事。”賈母因又嘆道: “我想著,他從小兒伏侍了我一場,又伏侍了雲兒一場,末後給了一個魔王寶玉,虧他魔了這幾年。他又不是咱們家的根生土長的奴才,沒受過咱們什麼大恩典。他媽沒了,我想著要給他幾兩銀子發送,也就忘了。”鳳姐兒道:“前兒太太賞了他四十兩銀子,也就是了。”賈母聽說,點頭道:“這還罷了。正好鴛鴦的娘前兒也死了,我想他老子娘都在南邊,我也沒叫他家去走走守孝,如今叫他兩個一處作伴兒去。”又命婆子將些果子菜饌點心之類與他兩個喫去。琥珀笑說:“還等這會子呢,他早就去了。”說著,大家又喫酒看戲。
\end{parag}


\begin{parag}
    且說寶玉一徑來至園中,衆婆子見他回房,便不跟去,只坐在園門裏茶房裏烤火,和管茶的女人偷空飲酒鬥牌。寶玉至院中,雖是燈光燦爛,卻無人聲。麝月道:“他們都睡了不成?咱們悄悄的進去唬他們一跳。”於是大家躡足潛蹤的進了鏡壁一看,只見襲人和一人二人對面都歪在地炕上,那一頭有兩三個老嬤嬤打盹。寶玉只當他兩個睡著了,纔要進去,忽聽鴛鴦嘆了一聲,說道:“可知天下事難定。論理你單身在這裏,父母在外頭,每年他們東去西來,沒個定準,想來你是不能送終的了,偏生今年就死在這裏,你倒出去送了終。”襲人道:“正是。我也想不到能夠看父母回首。太太又賞了四十兩銀子,這倒也算養我一場,我也不敢妄想了。”寶玉聽了,忙轉身悄向麝月等道:“誰知他也來了。我這一進去,他又賭氣走了,不如咱們回去罷,讓他兩個清清靜靜的說一回。襲人正一個悶著,他幸而來的好。”說著,仍悄悄的出來。
\end{parag}


\begin{parag}
    寶玉便走過山石之後去站著撩衣,麝月秋紋皆站住背過臉去,口內笑說:“蹲下再解小衣,仔細風吹了肚子。”後面兩個小丫頭子知是小解,忙先出去茶房預備去了。這裏寶玉剛轉過來,只見兩個媳婦子迎面來了,問是誰,秋紋道:“寶玉在這裏,你大呼小叫,仔細唬著罷。”那媳婦們忙笑道:“我們不知道,大節下來惹禍了。姑娘們可連日辛苦了。”說著,已到了跟前。麝月等問:“手裏拿的是什麼?”媳婦們道:“是老太太賞金、花二位姑娘喫的。”秋紋笑道:“外頭唱的是《八義》,沒唱《混元盒》,那裏又跑出‘金花娘娘’來了。”寶玉笑命:“揭起來我瞧瞧。”秋紋麝月忙上去將兩個盒子揭開。兩個媳婦忙蹲下身子,\begin{note}庚雙夾:細膩之極!一部大觀園之文皆若食肥蟹,至此一句則又三月於鎮江江上啖出網之鮮鰣矣。\end{note}寶玉看了兩盒內都是席上所有的上等果品菜饌,點了一點頭,邁步就走。麝月二人忙胡亂擲了盒蓋,跟上來。寶玉笑道:“這兩個女人倒和氣,會說話,他們天天乏了,倒說你們連日辛苦,倒不是那矜功自伐的。”麝月道:“這好的也很好,那不知禮的也太不知禮。”寶玉笑道:“你們是明白人,耽待他們是粗笨可憐的人就完了。”一面說,一面來至園門。那幾個婆子雖喫酒鬥牌,卻不住出來打探,見寶玉來了,也都跟上了。來至花廳後廊上,只見那兩個小丫頭一個捧著小沐盆,一個搭著手巾,又拿著漚子壺在那裏久等。秋紋先忙伸手向盆內試了一試,說道:“你越大越粗心了,那裏弄的這冷水。”小丫頭笑道:“姑娘瞧瞧這個天,我怕水冷,巴巴的倒的是滾水,這還冷了。”正說著,可巧見一個老婆子提著一壺滾水走來。小丫頭便說:“好奶奶,過來給我倒上些。”那婆子道:“哥哥兒,這是老太太泡茶的,勸你走了舀去罷,那裏就走大了腳。”秋紋道:“憑你是誰的,你不給?我管把老太太茶吊子倒了洗手。”那婆子回頭見是秋紋,忙提起壺來就倒。秋紋道:“夠了。你這麼大年紀也沒個見識,誰不知是老太太的水!要不著的人就敢要了。”婆子笑道:“我眼花了,沒認出這姑娘來。”寶玉洗了手,那小丫頭子拿小壺倒了些漚子在他手內,寶玉漚了。秋紋麝月也趁熱水洗了一回,漚了,跟進寶玉來。
\end{parag}


\begin{parag}
    寶玉便要了一壺暖酒,也從李嬸薛姨媽斟起,二人也讓坐。賈母便說:“他小,讓他斟去,大家倒要幹過這杯。”說著,便自己幹了。邢王二夫人也忙幹了,讓他二人。薛李也只得幹了。賈母又命寶玉道:“連你姐姐妹妹一齊斟上,不許亂斟,都要叫他幹了。”寶玉聽說,答應著,一一按次斟了。至黛玉前,偏他不飲,拿起杯來,放在寶玉脣上邊,寶玉一氣飲幹。黛玉笑說:“多謝。”寶玉替他斟上一杯。鳳姐兒便笑道:“寶玉,別喝冷酒,仔細手顫,明兒寫不得字,拉不得弓。” 寶玉忙道:“沒有喫冷酒。”鳳姐兒笑道:“我知道沒有,不過白囑咐你。”然後寶玉將裏面斟完,只除賈蓉之妻是丫頭們斟的。復出至廊上,又與賈珍等斟了。坐了一回,方進來仍歸舊坐。
\end{parag}


\begin{parag}
    一時上湯後,又接獻元宵來。賈母便命將戲暫歇歇:“小孩子們可憐見的,也給他們些滾湯滾菜的吃了再唱。”又命將各色果子元宵等物拿些與他們喫去。一時歇了戲,便有婆子帶了兩個門下常走的女先生兒進來,放兩張杌子在那一邊命他坐了,將弦子琵琶遞過去。賈母便問李薛聽何書,他二人都回說:“不拘什麼都好。”賈母便問:“近來可有添些什麼新書?”那兩個女先兒回說道:“倒有一段新書,是殘唐五代的故事。”賈母問是何名,女先兒道:“叫做《鳳求鸞》。”賈母道:“這一個名字倒好,不知因什麼起的,先大概說說原故,若好再說。”女先兒道:“這書上乃說殘唐之時,有一位鄉紳,本是金陵人氏,名喚王忠,曾做過兩朝宰輔,如今告老還家,膝下只有一位公子,名喚王熙鳳。”衆人聽了,笑將起來。賈母笑道:“這重了我們鳳丫頭了。”媳婦忙上去推他,“這是二奶奶的名字,少混說。”賈母笑道:“你說,你說。”女先生忙笑著站起來,說:“我們該死了,不知是奶奶的諱。”鳳姐兒笑道:“怕什麼,你們只管說罷,重名重姓的多呢。”女先生又說道:“這年王老爺打發了王公子上京趕考,那日遇見大雨,進到一個莊上避雨。誰知這莊上也有個鄉紳,姓李,與王老爺是世交,便留下這公子住在書房裏。這李鄉紳膝下無兒,只有一位千金小姐。這小姐芳名叫作雛鸞,琴棋書畫,無所不通。”賈母忙道:“怪道叫作《鳳求鸞》。不用說,我猜著了,自然是這王熙鳳要求這雛鸞小姐爲妻。”女先兒笑道:“老祖宗原來聽過這一回書。”衆人都道:“老太太什麼沒聽過!便沒聽過,也猜著了。”賈母笑道:“這些書都是一個套子,左不過是些佳人才子,最沒趣兒。把人家女兒說的那樣壞,還說是佳人,編的連影兒也沒有了。開口都是書香門第,父親不是尚書就是宰相,生一個小姐必是愛如珍寶。這小姐必是通文知禮,無所不曉,竟是個絕代佳人。只一見了一個清俊的男人,不管是親是友,便想起終身大事來,父母也忘了,書禮也忘了,鬼不成鬼,賊不成賊,那一點兒是佳人?便是滿腹文章,做出這些事來,也算不得是佳人了。比如男人滿腹文章去作賊,\begin{note}批:“滿腹文章去作賊”,餘謂多事。\end{note}難道那王法就說他是才子,就不入賊情一案不成?可知那編書的是自己塞了自己的嘴。再者,既說是世宦書香大家小姐都知禮讀書,連夫人都知書識禮,便是告老還家,自然這樣大家人口不少,奶母丫鬟伏侍小姐的人也不少,怎麼這些書上,凡有這樣的事,就只小姐和緊跟的一個丫鬟?你們白想想,那些人都是管什麼的,可是前言不答後語?”衆人聽了,都笑說:“老太太這一說,是謊都批出來了。”賈母笑道:“這有個原故:編這樣書的,有一等妒人家富貴,或有求不遂心,所以編出來污穢人家。再一等,他自己看了這些書看魔了,他也想一個佳人,所以編了出來取樂。何嘗他知道那世宦讀書家的道理!別說他那書上那些世宦書禮大家,如今眼下真的,拿我們這中等人家說起,也沒有這樣的事,別說是那些大家子。可知是謅掉了下巴的話。所以我們從不許說這些書,丫頭們也不懂這些話。這幾年我老了,他們姊妹們住的遠,我偶然悶了,說幾句聽聽,他們一來,就忙歇了。”李薛二人都笑說:“這正是大家的規矩,連我們家也沒這些雜話給孩子們聽見。”
\end{parag}


\begin{parag}
    鳳姐兒走上來斟酒,笑道:“罷,罷,酒冷了,老祖宗喝一口潤潤嗓子再掰謊。這一回就叫作《掰謊記》,就出在本朝本地本年本月本日本時,老祖宗一張口難說兩家話,花開兩朵,各表一枝,是真是謊且不表,再整那觀燈看戲的人。老祖宗且讓這二位親戚喫一杯酒看兩齣戲之後,再從昨朝話言掰起如何?”他一面斟酒,一面笑說,未曾說完,衆人俱已笑倒。兩個女先生也笑個不住,都說:“奶奶好剛口。奶奶要一說書,真連我們喫飯的地方也沒了。”薛姨媽笑道:“你少興頭些,外頭有人,比不得往常。”鳳姐兒笑道:“外頭的只有一位珍大爺。我們還是論哥哥妹妹,從小兒一處淘氣了這麼大。這幾年因做了親,我如今立了多少規矩了。便不是從小兒的兄妹,便以伯叔論,那《二十四孝》上‘斑衣戲彩’,他們不能來‘戲彩’引老祖宗笑一笑,我這裏好容易引的老祖宗笑了一笑,多吃了一點兒東西,大家喜歡,都該謝我纔是,難道反笑話我不成?”賈母笑道:“可是這兩日我竟沒有痛痛的笑一場,倒是虧他才一路笑的我心裏痛快了些,我再喫一鍾酒。”喫著酒,又命寶玉:“也敬你姐姐一杯。”鳳姐兒笑道:“不用他敬,我討老祖宗的壽罷。”說著,便將賈母的杯拿起來,將半杯剩酒吃了,將杯遞與丫鬟,另將溫水浸的杯換了一個上來。於是各席上的杯都撤去,另將溫水浸著待換的杯斟了新酒上來,然後歸坐。
\end{parag}


\begin{parag}
    女先生回說:“老祖宗不聽這書,或者彈一套曲子聽聽罷。”賈母便說道:“你們兩個對一套《將軍令》罷。”二人聽說,忙和絃按調撥弄起來。賈母因問: “天有幾更了。”衆婆子忙回:“三更了。”賈母道:“怪道寒浸浸的起來。”早有衆丫鬟拿了添換的衣裳送來。王夫人起身笑說道:“老太太不如挪進暖閣裏地炕上倒也罷了。這二位親戚也不是外人,我們陪著就是了。”賈母聽說,笑道:“既這樣說,不如大家都挪進去,豈不暖和?”王夫人道:“恐裏間坐不下。”賈母笑道:“我有道理。如今也不用這些桌子,只用兩三張並起來,大家坐在一處擠著,又親香,又暖和。”衆人都道:“這纔有趣。”說著,便起了席。衆媳婦忙撤去殘席,裏面直順並了三張大桌,另又添換了果饌擺好。賈母便說:“這都不要拘禮,只聽我分派你們就坐纔好。”說著便讓薛李正面上坐,自己西向坐了,叫寶琴、黛玉、湘雲三人皆緊依左右坐下,向寶玉說:“你挨著你太太。”於是邢夫人王夫人之中夾著寶玉,寶釵等姊妹在西邊,挨次下去便是婁氏帶著賈菌,尤氏李紈夾著賈蘭,下面橫頭便是賈蓉之妻。賈母便說:“珍哥兒帶著你兄弟們去罷,我也就睡了。”
\end{parag}


\begin{parag}
    賈珍忙答應,又都進來。賈母道:“快去罷!不用進來,才坐好了,又都起來。你快歇著,明日還有大事呢。”賈珍忙答應了,又笑說:“留下蓉兒斟酒纔是。”賈母笑道:“正是忘了他。”賈珍答應了一個“是”,便轉身帶領賈璉等出來。二人自是歡喜,便命人將賈琮賈璜各自送回家去,便邀了賈璉去追歡買笑,不在話下。
\end{parag}


\begin{parag}
    這裏賈母笑道:“我正想著雖然這些人取樂,竟沒一對雙全的,就忘了蓉兒。這可全了,蓉兒就合你媳婦坐在一處,倒也團圓了。”因有媳婦回說開戲,賈母笑道:“我們娘兒們正說的興頭,又要吵起來。況且那孩子們熬夜怪冷的,也罷,叫他們且歇歇,把咱們的女孩子們叫了來,就在這臺上唱兩出給他們瞧瞧。”媳婦聽了,答應了出來,忙的一面著人往大觀園去傳人,一面二門口去傳小廝們伺候。小廝們忙至戲房將班中所有的大人一概帶出,只留下小孩子們。
\end{parag}


\begin{parag}
    一時,梨香院的教習帶了文官等十二個人,從遊廊角門出來。婆子們抱著幾個軟包,因不及抬箱,估料著賈母愛聽的三五齣戲的綵衣包了來。婆子們帶了文官等進去見過,只垂手站著。賈母笑道:“大正月裏,你師父也不放你們出來逛逛。你等唱什麼?剛纔八出《八義》鬧得我頭疼,咱們清淡些好。你瞧瞧,薛姨太太這李親家太太都是有戲的人家,不知聽過多少好戲的。這些姑娘們都比咱們家姑娘見過好戲,聽過好曲子。如今這小戲子又是那有名玩戲家的班子,雖是小孩子們,卻比大班還強。咱們好歹別落了褒貶,少不得弄個新樣兒的。叫芳官唱一出《尋夢》,只提琴至管簫合,笙笛一概不用。”文官笑道:“這也是的,我們的戲自然不能入姨太太和親家太太姑娘們的眼,不過聽我們一個發脫口齒,再聽一個喉嚨罷了。”賈母笑道:“正是這話了。”李嬸薛姨媽喜的都笑道:“好個靈透孩子,他也跟著老太太打趣我們。”賈母笑道:“我們這原是隨便的頑意兒,又不出去做買賣,所以竟不大合時。”說著又道:“叫葵官唱一出《惠明下書》,也不用抹臉。只用這兩出叫他們聽個疏異罷了。若省一點力,我可不依。”文官等聽了出來,忙去扮演上臺,先是《尋夢》,次是《下書》。衆人都鴉雀無聞,薛姨媽因笑道:“實在虧他,戲也看過幾百班,從沒見用簫管的。”賈母道:“也有,只是象方纔《西樓•楚江晴》一支,多有小生吹簫和的。這大套的實在少,這也在主人講究不講究罷了。這算什麼出奇?”指湘雲道:“我象他這麼大的時節,他爺爺有一班小戲,偏有一個彈琴的湊了來,即如《西廂記》的《聽琴》,《玉簪記》的《琴挑》,《續琵琶》的《胡笳十八拍》,竟成了真的了,比這個更如何?”衆人都道:“這更難得了。”賈母便命個媳婦來,吩咐文官等叫他們吹一套《燈月圓》。媳婦領命而去。
\end{parag}


\begin{parag}
    當下賈蓉夫妻二人捧酒一巡,鳳姐兒因見賈母十分高興,便笑道:“趁著女先兒們在這裏,不如叫他們擊鼓,咱們傳梅,行一個‘春喜上眉梢’的令如何?”賈母笑道:“這是個好令,正對時對景。”忙命人取了一面黑漆銅釘花腔令鼓來,與女先兒們擊著,席上取了一枝紅梅。賈母笑道:“若到誰手裏住了,喫一杯,也要說個什麼纔好。”鳳姐兒笑道:“依我說,誰象老祖宗要什麼有什麼呢。我們這不會的,豈不沒意思。依我說也要雅俗共賞,不如誰輸了誰說個笑話罷。”衆人聽了,都知道他素日善說笑話,最是他肚內有無限的新鮮趣談。今兒如此說,不但在席的諸人喜歡,連地下伏侍的老小人等無不歡喜。那小丫頭子們都忙出去,找姐喚妹的告訴他們:“快來聽,二奶奶又說笑話兒了。”衆丫頭子們便擠了一屋子。於是戲完樂罷。賈母命將些湯點果菜與文官等喫去,便命響鼓。那女先兒們皆是慣的,或緊或慢,或如殘漏之滴,或如迸豆之疾,或如驚馬之亂馳,或如疾電之光而忽暗。其鼓聲慢,傳梅亦慢;鼓聲疾,傳梅亦疾。恰恰至賈母手中,鼓聲忽住。大家呵呵一笑,賈蓉忙上來斟了一杯。衆人都笑道:“自然老太太先喜了,我們才托賴些喜。”賈母笑道:“這酒也罷了,只是這笑話倒有些個難說。”衆人都說: “老太太的比鳳姐兒的還好還多,賞一個我們也笑一笑兒。”賈母笑道:“並沒什麼新鮮發笑的,少不得老臉皮子厚的說一個罷了。”因說道:“一家子養了十個兒子,娶了十房媳婦。惟有第十個媳婦最聰明伶俐,心巧嘴乖,公婆最疼,成日家說那九個不孝順。這九個媳婦委屈,便商議說:‘咱們九個心裏孝順,只是不象那小蹄子嘴巧,所以公公婆婆老了,只說他好,這委屈向誰訴去?’大媳婦有主意,便說道:‘咱們明兒到閻王廟去燒香,和閻王爺說去,問他一問,叫我們託生人,爲什麼單單的給那小蹄子一張乖嘴,我們都是笨的。’衆人聽了都喜歡,說這主意不錯。第二日便都到閻王廟裏來燒了香,九個人都在供桌底下睡著了。九個魂專等閻王駕到,左等不來,右等也不到。正著急,只見孫行者駕著筋斗雲來了,看見九個魂便要拿金箍棒打,唬得九個魂忙跪下央求。孫行者問原故,九個人忙細細的告訴了他。孫行者聽了,把腳一跺,嘆了一口氣道:‘這原故幸虧遇見我,等著閻王來了,他也不得知道的。’九個人聽了,就求說:‘大聖發個慈悲,我們就好了。’ 孫行者笑道:‘這卻不難。那日你們妯娌十個託生時,可巧我到閻王那裏去的,因爲撒了泡尿在地下,你那小嬸子便吃了。你們如今要伶俐嘴乖,有的是尿,再撒泡你們吃了就是了。’”說畢,大家都笑起來。
\end{parag}


\begin{parag}
    鳳姐兒笑道:“好的,幸而我們都笨嘴笨腮的,不然也就吃了猴兒尿了。”尤氏婁氏都笑向李紈道:“咱們這裏誰是喫過猴兒尿的,別裝沒事人兒。”薛姨媽笑道:“笑話兒不在好歹,只要對景就發笑。”說著又擊起鼓來。小丫頭子們只要聽鳳姐兒的笑話,便俏俏的和女先兒說明,以咳嗽爲記。須臾傳至兩遍,剛到了鳳姐兒手裏,小丫頭子們故意咳嗽,女先兒便住了。衆人齊笑道:“這可拿住他了。快吃了酒說一個好的,別太逗的人笑的腸子疼。”鳳姐兒想了一想,笑道:“一家子也是過正月半,閤家賞燈喫酒,真真的熱鬧非常,祖婆婆、太婆婆、婆婆、媳婦、孫子媳婦、重孫子媳婦、親孫子、侄孫子、重孫子、灰孫子、滴滴搭搭的孫子、孫女兒、外孫女兒、姨表孫女兒、姑表孫女兒,……噯喲喲,真好熱鬧!”衆人聽他說著,已經笑了,都說:“聽數貧嘴,又不知編派那一個呢?”尤氏笑道:“你要招我,我可撕你的嘴。”鳳姐兒起身拍手笑道:“人家費力說,你們混,我就不說了。”賈母笑道:“你說你說,底下怎麼樣?”鳳姐兒想了一想,笑道:“底下就團團的坐了一屋子,吃了一夜酒就散了。”衆人見他正言厲色的說了,別無他話,都怔怔的還等下話,只覺冰涼無味。史湘雲看了他半日,鳳姐兒笑道:“再說一個過正月半的。幾個人抬著個房子大的炮仗往城外放去,引了上萬的人跟著瞧去。有一個性急的人等不得,便偷著拿香點著了。只聽‘噗哧’一聲,衆人鬨然一笑都散了。這抬炮仗的人抱怨賣炮仗的捍的不結實,沒等放就散了。”湘雲道:“難道他本人沒聽見響?”鳳姐兒道:“這本人原是聾子。”衆人聽說,一回想,不覺一齊失聲都大笑起來。又想著先前那一個沒完的,問他:“先一個怎麼樣?也該說完。”鳳姐兒將桌子一拍,說道:“好羅唆,到了第二日是十六日,年也完了,節也完了,我看著人忙著收東西還鬧不清,那裏還知道底下的事了。”衆人聽說,復又笑將起來。鳳姐兒笑道:“外頭已經四更,依我說,老祖宗也乏了,咱們也該‘聾子放炮仗──散了’罷。”尤氏等用手帕子握著嘴,笑的前仰後合,指他說道:“這個東西真會數貧嘴。”賈母笑道:“真真這鳳丫頭越發貧嘴了。”一面說,一面吩咐道:“他提炮仗來,咱們也把煙火放了解解酒。” 賈蓉聽了,忙出去帶著小廝們就在院內安下屏架,將煙火設吊齊備。這煙火皆系各處進貢之物,雖不甚大,卻極精巧,各色故事俱全,夾著各色花炮。林黛玉稟氣柔弱,不禁畢駁之聲,賈母便摟他在懷中。薛姨媽摟著湘雲。湘雲笑道:“我不怕。”寶釵等笑道:“他專愛自己放大炮仗,還怕這個呢。”王夫人便將寶玉摟入懷內。鳳姐兒笑道:“我們是沒有人疼的了。”尤氏笑道:“有我呢,我摟著你。也不怕臊,你這孩子又撒嬌了,聽見放炮仗,吃了蜜蜂兒屎的,今兒又輕狂起來。” 鳳姐兒笑道:“等散了,咱們園子裏放去。我比小廝們還放的好呢。”說話之間,外面一色一色的放了又放,又有許多的滿天星、九龍入雲、一聲雷、飛天十響之類的零碎小爆竹。放罷,然後又命小戲子打了一回“蓮花落”,撒了滿臺錢,命那孩子們滿臺搶錢取樂。又上湯時,賈母說道:“夜長,覺的有些餓了。”鳳姐兒忙回說:“有預備的鴨子肉粥。”賈母道:“我喫些清淡的罷。”鳳姐兒忙道:“也有棗兒熬的粳米粥,預備太太們喫齋的。”賈母笑道:“不是油膩膩的就是甜的。” 鳳姐兒又忙道:“還有杏仁茶,只怕也甜。”賈母道:“倒是這個還罷了。”說著,已經撤去殘席,外面另設上各種精緻小菜。大家隨便隨意吃了些,用過漱口茶,方散。
\end{parag}


\begin{parag}
    十七日一早,又過寧府行禮,伺候掩了宗祠,收過影像,方回來。此日便是薛姨媽家請喫年酒。十八日便是賴大家,十九日便是寧府賴升家,二十日便是林之孝家,二十一日便是單大良家,二十二日便是吳新登家。這幾家,賈母也有去的,也有不去的,也有高興直待衆人散了方回的,也有興盡半日一時就來的。凡諸親友來請或來赴席的,賈母一概怕拘束不會,自有邢夫人、王夫人、鳳姐兒三人料理。連寶玉只除王子騰家去了,餘者亦皆不會,只說賈母留下解悶。所以倒是家下人家來請,賈母可以自便之處,方高興去逛逛。閒言不提,且說當下元宵已過──
\end{parag}


\begin{parag}
    \begin{note}蒙回末總:讀此回者凡三變。不善讀者徒贊其如何演戲、如何行令、如何掛花燈、如何放爆竹,目眩耳聾,應接不暇。少解讀者,贊其座次有倫、巡酒有度,從演戲渡至女先,從女先渡至鳳姐,從鳳姐渡至行令,從行令渡至放花爆:脫卸下來,井然秩然,一絲不亂。會讀者須另具卓識,單著眼史太君一席話,將普天下不近理之 “奇文”、不近情之“妙作”一起抹倒。是作者借他人酒杯,消自己傀儡,畫一幅行樂圖,鑄一面菱花鏡,爲全部總評。噫!作者已逝,聖嘆雲亡,愚不自量,輒擬數語,知我罪我,其聽之矣。\end{note}
\end{parag}
