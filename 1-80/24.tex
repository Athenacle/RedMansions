\chap{二十四}{醉金剛輕財尚義俠 癡女兒遺帕惹相思}


\begin{parag}
    \begin{note}庚:夾寫“醉金剛”一回是書中之大淨場,聊醒看官倦眼耳。然亦書中必不可少之文,必不可少之人。今寫在市井俗人身上,又加一“俠”字,則大有深意存焉。\end{note}
\end{parag}


\begin{parag}
    \begin{note}蒙回前總:夾寫醉金剛一回,是處中之大文字,聊醒看官倦眠而,然亦書中之必不可少之文字,必不可少之人,今寫在市井俗人身上,加一“俠”字,則有大深意存焉。\end{note}
\end{parag}


\begin{parag}
    \begin{note}靖:“醉金剛”一回文字,伏芸哥仗義探庵。餘三十年來得遇金剛之樣人不少,不及金剛者亦不少。惜不便一一註明耳。壬午孟夏。\end{note}
\end{parag}


\begin{parag}
    話說林黛玉正自情思縈逗,纏綿固結之時,忽有人從背後擊了一掌,說道:“你作什麼一個人在這裏?”林黛玉倒唬了一跳,回頭看時,不是別人,卻是香菱。林黛玉道:“你這個傻\begin{note}庚側:此“傻”字加於香菱,則有多少丰神跳於紙上,其嬌憨之態可想而知。\end{note}丫頭,唬我這麼一跳好的。你這會子打那裏來?”香菱嘻嘻的笑道:“我來尋我們的姑娘的,找他總找不著。你們紫鵑也找你呢,\begin{note}庚側:一絲不漏。\end{note}說璉二奶奶送了什麼茶葉來給你的。走罷,回家去坐著。”\begin{note}庚側:“回家去坐著”之言,是恐石上冷意。\end{note}一面說著,一面拉著黛玉的手回瀟湘館來了。果然鳳姐兒送了兩小瓶上用新茶來。林黛玉和香菱坐了。況他們有甚正事談講。\begin{note}庚側:爲學詩伏線。\end{note}不過說些這一個繡的好,那一個刺的精,又下一回棋,看兩句書,\begin{note}庚雙夾:棋不論盤,書不論章,皆是嬌憨女兒神理,寫得不即不離,似有似無,妙極!\end{note}香菱便走了。不在話下。\begin{note}庚眉:是書最好看如此等處,系畫家山水樹 非褊志惚 ,末用濃淡墨點苔法也。亥夏。畸笏叟。\end{note}
\end{parag}


\begin{parag}
    如今且說寶玉因被襲人找回房去,果見鴛鴦歪在牀上看襲人的針線呢,見寶玉來了,便說道:“你往那裏去了?老太太等著你呢,叫你過那邊請大老爺的安去。還不快換了衣服走呢。”襲人便進房去取衣服。寶玉坐在牀沿上,褪了鞋等靴子穿的工夫,回頭見鴛鴦穿著水紅綾子襖兒,青緞子背心,束著白縐綢汗巾兒,臉向那邊低著頭看針線,脖子上戴著花領子。寶玉便把臉湊在他脖項上,聞那香油氣,不住用手摩挲,其白膩不在襲人之下,便猴上身去涎皮笑道:“好姐姐,把你嘴上的胭脂賞我吃了罷。”\begin{note}庚側:胭脂是這樣喫法。看官可經過否?\end{note}一面說著,一面扭股糖似的粘在身上。
\end{parag}


\begin{parag}
    鴛鴦便叫道:“襲人,你出來瞧瞧。\begin{note}庚側:不向寶玉說話,又叫襲人,鴛鴦亦是幻情洞天也。\end{note}你跟他一輩子,也不勸勸,還是這麼著。”襲人抱了衣服出來,向寶玉道:“左勸也不改,右勸也不改,你到底是怎麼樣?你再這麼著,\begin{note}庚側:此五字內有深意深心。\end{note}這個地方可就難住了。”一邊說,一邊催他穿了衣服,同鴛鴦往前面來見賈母。見過賈母,出至外面,人馬俱已齊備。剛欲上馬,只見賈璉請安回來了,\begin{note}庚側:一絲不漏。\end{note}正下馬,二人對面,彼此問了兩句話。只見旁邊轉出一個人來,\begin{note}庚側:芸哥此處一現,後文不見突然。\end{note}“請寶叔安”。寶玉看時,只見這人容長臉,長挑身材,年紀只好十八九歲,生得著實斯文清秀,倒也十分面善,只是想不起是那一房的,\begin{note}庚側:大族人衆,畢真,有是理。\end{note}叫什麼名字。賈璉笑道:“你怎麼發呆,連他也不認得?他是後廊上住的五嫂子的兒子芸兒。”寶玉笑道:“是了,是了,我怎麼就忘了。”因問他母親好,這會子什麼勾當。賈芸指賈璉道:“找二叔說句話。”寶玉笑道:“你倒比先越發出挑了,\begin{note}庚側:何嘗是十二三歲小孩語。\end{note}倒象我的兒子。”賈璉笑道:“好不害臊!人家比你大四五歲呢,就替你作兒子了?”寶玉笑道:“你今年十幾歲了?”賈芸道:“十八歲。”
\end{parag}


\begin{parag}
    原來這賈芸最伶俐乖覺,聽寶玉這樣說,便笑道:“俗語說的,‘搖車裏的爺爺,拄拐的孫孫’。雖然歲數大,山高高不過太陽。只從我父親沒了,這幾年也無人照管教導。\begin{note}庚側:雖是隨機而應,伶俐人之語,餘卻傷心。\end{note}如若寶叔不嫌侄兒蠢笨,認作兒子,就是我的造化了。”賈璉笑道:“你聽見了?認兒子不是好開交的呢。”\begin{note}庚側:是兄湊弟趣,可嘆!\end{note}說著就進去了。寶玉笑道:“明兒你閒了,只管來找我,別和他們鬼鬼祟祟的。\begin{note}庚側:何其堂皇正大之語。\end{note}這會子我不得閒兒。明兒你到書房裏來,和你說天話兒,我帶你園裏頑耍去。”說著扳鞍上馬,衆小廝圍隨往賈赦這邊來。
\end{parag}


\begin{parag}
    見了賈赦,不過是偶感些風寒,先述了賈母問的話,然後自己請了安。賈赦先站起來回了賈母話,\begin{note}庚側:一絲不亂。\end{note}次後便喚人來:“帶哥兒進去太太屋裏坐著。”寶玉退出,來至後面,進入上房。邢夫人見了他來,先倒站了起來請過賈母安,\begin{note}庚側:一絲不亂。\end{note}寶玉方請安。\begin{note}[好規矩。]\end{note}邢夫人拉他上炕坐了,方問別人好,又命人倒茶來。\begin{note}庚側:好層次,好禮法,誰家故事?\end{note}一鍾茶未喫完,只見那賈琮來問寶玉好。邢夫人道:“那裏找活猴兒去!你那奶媽子死絕了,也不收拾收拾你,弄的黑眉烏嘴的,那裏象大家子唸書的孩子!”
\end{parag}


\begin{parag}
    正說著,只見賈環、賈蘭小叔侄兩個也來了,請過安,邢夫人便叫他兩個椅子上坐了。賈環見寶玉同邢夫人坐在一個坐褥上,邢夫人又百般摩挲撫弄他,早已心中不自在了,\begin{note}庚側:千里伏線。\end{note}坐不多時,便和賈蘭使眼色兒要走。賈蘭只得依他,一同起身告辭。寶玉見他們要走,自己也就起身,要一同回去。邢夫人笑道:“你且坐著,我還和你說話呢。”寶玉只得坐了。邢夫人向他兩個道:“你們回去,各人替我問你們各人母親好。你們姑娘、姐姐妹妹都在這裏呢,鬧的我頭暈,今兒不留你們喫飯了。”\begin{note}庚側:明顯薄情之至。\end{note}賈環等答應著,便出來回家去了。
\end{parag}


\begin{parag}
    寶玉笑道:“可是姐姐們都過來了,怎麼不見?”邢夫人道:“他們坐了一會子,都往後頭不知那屋裏去了。”寶玉道:“大娘方纔說有話說,不知是什麼話?” 邢夫人笑道:“那裏有什麼話,不過是叫你等著,同你姊妹們吃了飯去。還有一個好玩的東西給你帶回去玩。”孃兒兩個說話,不覺早又晚飯時節。調開桌椅,羅列杯盤,母女姊妹們喫畢了飯。寶玉去辭賈赦,同姊妹們一同回家,見過賈母,王夫人等,各自回房安息。不在話下。\begin{note}庚雙夾:逐步一段爲五鬼魘魔法作引。脂硯。\end{note}
\end{parag}


\begin{parag}
    且說賈芸進去見了賈璉,因打聽可有什麼事情。賈璉告訴他:“前兒倒有一件事情出來,偏生你嬸子再三求了我,\begin{note}庚側:反說體面話,懼內人累累如是。\end{note}給了賈芹了。他許了我,說明兒園裏還有幾處要栽花木的地方,等這個工程出來,一定給你就是了。”賈芸聽了,半晌說道:“既是這樣,我就等著罷。叔叔也不必先在嬸子跟前提我今兒來打聽的話,\begin{note}庚側:已得了主意了。\end{note}到跟前再說也不遲。”賈璉道:“提他作什麼,\begin{note}庚側:已被芸哥瞞過了。\end{note}我那裏有這些工夫說閒話兒呢。明兒一個五更,還要到興邑去走一趟,須得當日趕回來纔好。你先去等著,後日起更以後你來討信兒,來早了我不得閒。”說著便回後面換衣服去了。
\end{parag}


\begin{parag}
    賈芸出了榮國府回家,一路思量,想出一個主意來,便一徑往他母舅卜世仁家來。\begin{note}庚側:既雲“不是人”,如何肯共事?想芸哥此來空了。\end{note}原來卜世仁現開香料鋪,方纔從鋪子裏來,忽見賈芸進來,彼此見過了,因問他這早晚什麼事跑了來。賈芸道:“有件事求舅舅幫襯幫襯。我有一件事,用些冰片麝香使用,好舅舅每樣賒四兩給我,八月裏按數送了銀子來 ”\begin{note}庚雙夾:甥舅之談如此,嘆嘆!\end{note}卜世仁冷笑道:“再休提賒欠一事。\begin{note}庚側:何如,何如?餘言不謬。\end{note}前兒也是我們鋪子裏一個夥計,替他的親戚賒了幾兩銀子的貨,至今總未還上。因此我們大家賠上,立了合同,再不許替親友賒欠。誰要賒欠,就要罰他二十兩銀子的東道。況且如今這個貨也短,你就拿現銀子到我們這不三不四的鋪子裏來買,\begin{note}庚側:推脫之辭。\end{note}也還沒有這些,只好倒扁兒去。這是一。二則你那裏有正經事,不過賒了去又是胡鬧。你只說舅舅見你一遭兒就派你一遭兒不是。你小人兒家很不知好歹,也到底立個主見,賺幾個錢,弄得穿是穿喫是喫的,我看著也喜歡。”
\end{parag}


\begin{parag}
    賈芸笑道:“舅舅說的倒乾淨。我父親沒的時候,我年紀又小,不知事。後來聽見我母親說,都還虧舅舅們在我們家出主意,料理的喪事。難道舅舅就不知道的,還是有一畝地兩間房子,如今在我手裏花了不成?巧媳婦做不出沒米的粥來,叫我怎麼樣呢?還虧是我呢,要是別個,死皮賴臉三日兩頭兒來纏著舅舅,\begin{note}庚側:芸哥亦善談,井井有理。\end{note}要三升米二升豆子的,\begin{note}庚側:餘二人亦不曾有是氣?\end{note}舅舅也就沒有法呢。”
\end{parag}


\begin{parag}
    卜世仁道:“我的兒,舅舅要有,還不是該的。我天天和你舅母說,只愁你沒算計兒。你但凡立的起來,到你大房裏,就是他們爺兒們見不著,便下個氣,和他們的管家或者管事的人們嬉和嬉和,\begin{note}庚側:可憐可嘆,餘竟爲之一哭。\end{note}也弄個事兒管管。前日我出城去,撞見了你們三房裏的老四,騎著大叫驢,帶著五輛車,有四五十和尚道士,\begin{note}庚雙夾:妙極!寫小人口角,羨慕之言加一倍,畢肖。卻又是背面傅粉法。\end{note}往家廟去了。他那不虧能幹,這事就到他了!”賈芸聽他韶刀的不堪,便起身告辭。\begin{note}庚側:有志氣,有果斷。\end{note}卜世仁道:“怎麼急的這樣,吃了飯再去罷。”一句未完,只見他娘子說道:“你又糊塗了。\begin{note}庚側:雖寫小人家澀細,一吹一唱,酷肖之至,卻是一氣逼出,後文方不突然。《石頭記》筆仗全在如此樣者。\end{note}說著沒有米,這裏買了半斤面來下給你喫,這會子還裝胖呢。留下外甥捱餓不成?”卜世仁說:“再買半斤來添上就是了。”他娘子便叫女孩兒:“銀姐,往對門王奶奶家去問,有錢借二三十個,明兒就送過來。”夫妻兩個說話,那賈芸早說了幾個“不用費事”,去的無影無蹤了。\begin{note}庚側:有知識有果斷人,自是不同。\end{note}
\end{parag}


\begin{parag}
    不言卜家夫婦,且說賈芸賭氣離了母舅家門,一徑迴歸舊路,心下正自煩惱,一邊想,一邊低頭只管走,不想一頭就碰在一個醉漢身上,把賈芸唬了一跳。\begin{note}庚批:自上看來,可是一口氣否?\end{note}聽醉漢罵道:“臊你孃的!瞎了眼睛,碰起我來了。”賈芸忙要躲身,早被那醉漢一把抓住,對面一看,不是別人,卻是緊鄰倪二。原來這倪二是個潑皮,專放重利債,在賭博場喫閒錢,專管打降喫酒。如今正從欠錢人家索了利錢,喫醉回來,不想被賈芸碰了一頭,正沒好氣,掄拳就要打。\begin{note}庚眉:這一節對《水滸》楊志賣大刀遇沒毛大蟲一回看,覺好看多矣。己冬夜。脂硯。\end{note}只聽那人叫道:“老二住手!是我衝撞了你。”倪二聽見是熟人的語音,將醉眼睜開看時,見是賈芸,忙把手鬆了,趔趄著笑道:\begin{note}庚側:寫生之筆。\end{note}“原來是賈二爺,\begin{note}庚側:如此稱呼,可知芸哥素日行止,是“金盆雖破分量在”也。\end{note}我該死,我該死。這會子往那裏去?”賈芸道:“告訴不得你,平白的又討了個沒趣兒。”\begin{note}庚側:本無心之談也。\end{note}倪二道: “不妨不妨,\begin{note}庚側:如聞。\end{note}有什麼不平的事,告訴我,替你出氣。\begin{note}庚側:寫得酷肖,總是漸次逼出,不見一絲勉強。\end{note}這三街六巷,憑他是誰,有人得罪了我醉金剛倪二的街坊,管叫他人離家散!”賈芸道:“老二,你且彆氣,聽我告訴你這原故。”\begin{note}庚側:可是一順而來?\end{note}說著,便把卜世仁一段事告訴了倪二。倪二聽了大怒,“要不是令舅,我便罵不出好話來,\begin{note}庚側:仗義人豈有不知禮者乎?何嘗是破落戶?冤殺金剛了。\end{note}真真氣死我倪二。也罷,你也不用愁煩,我這裏現有幾兩銀子,你若用什麼,只管拿去買辦。但只一件,你我作了這些年的街坊,我在外頭有名放帳,你卻從沒有和我張過口。也不知你厭惡我是個潑皮,\begin{note}庚側:知己知彼之話。\end{note}怕低了你的身分,也不知是你怕我難纏,利錢重?若說怕利錢重,這銀子我是不要利錢的,也不用寫文約,若說怕低了你的身分,\begin{note}庚側:知己知彼之話。\end{note}我就不敢借給你了,各自走開。”一面說,一面果然從搭包裏掏出一卷銀子來。
\end{parag}


\begin{parag}
    賈芸心下自思:“素日倪二雖然是潑皮無賴,卻因人而使,\begin{note}庚側:四字是評,難得難得,非豪傑不可當。\end{note}頗頗的有義俠之名。若今日不領他這情,怕他臊了,倒恐生事。不如借了他的,改日加倍還他也倒罷了。”想畢笑道:“老二,你果然是個好漢,我何曾不想著你,和你張口。但只是我見你所相與交結的,都是些有膽量的有作爲的人,似我們這等無能無力的你倒不理。\begin{note}庚側:芸哥亦善談,好口齒。\end{note}我若和你張口,你豈肯借給我。今日既蒙高情,我怎敢不領,回家按例寫了文約過來便是了。”倪二大笑道:“好會說話的人。我卻聽不上這話。\begin{note}庚側:“光棍眼內揉不下沙子”是也。\end{note}既說‘相與交結’四個字,如何放帳給他,使他的利錢!\begin{note}庚側:如今不單是親友言利,不但親友,即閨閣中亦然,不但生意新發戶,即大戶舊族頗頗有之。\end{note}既把銀子借與他,圖他的利錢,便不是相與交結了。閒話也不必講。既肯青目,這是十五兩三錢有零的銀子,便拿去治買東西。你要寫什麼文契,趁早把銀子還我,讓我放給那些有指望的人使去。”\begin{note}庚側:爽快人,爽快語。\end{note}賈芸聽了,一面接了銀子,一面笑道:“我便不寫罷了,有何著急的。”倪二笑道:“這不是話。天氣黑了,也不讓茶讓酒,我還到那邊有點事情去,你竟請回去。我還求你帶個信兒與舍下,叫他們早些關門睡罷,我不回家去了,倘或有要緊事兒,叫我們女兒明兒一早到馬販子王短腿家\begin{note}庚側:常起坐處人,畢真。\end{note}來找我。”一面說,一面趔趄著腳兒去了,\begin{note}庚側:仍應前。\end{note}不在話下。\begin{note}庚眉:讀閱“醉金剛”一回,務喫劉鉉丹家山楂丸一付,一笑。餘卅年來得遇金剛之樣人不少,不及金剛者亦不少,惜書上不便歷歷註上芳諱,是餘不是心事也。壬午孟夏。\end{note}
\end{parag}


\begin{parag}
    且說賈芸偶然碰了這件事,心中也十分罕希,想那倪二倒果然有些意思,只是還怕他一時醉中慷慨,到明日加倍的要起來,便怎處,心內猶豫不決。\begin{note}庚側:芸哥實怕倪二,並非以小人之心度君子也。\end{note}忽又想道:“不妨,等那件事成了,也可加倍還他。”想畢,一直走到個錢鋪裏,將那銀子稱一稱,十五兩三錢四分二釐。賈芸見倪二不撒謊,心下越發歡喜,收了銀子,來至家門,先到隔壁將倪二的信捎了與他娘子知道,方回家來。見他母親自在炕上拈線,見他進來,便問那去了一日。賈芸恐他母親生氣,便不說起卜世仁的事來,\begin{note}庚側:孝子可敬。此人後來榮府事敗,必有一番作爲。\end{note}\begin{note}該批:果然。\end{note}只說在西府裏等璉二叔的,問他母親吃了飯不曾。他母親已喫過了,說留的飯在那裏。小丫頭子拿過來與他喫。
\end{parag}


\begin{parag}
    那天已是掌燈時候,賈芸吃了飯收拾歇息,一宿無話。次日一早起來,洗了臉,便出南門,大香鋪裏買了冰麝,便往榮國府來。打聽賈璉出了門,賈芸便往後面來。
\end{parag}


\begin{parag}
    到賈璉院門前,只見幾個小廝拿著大高笤帚在那裏掃院子呢。忽見周瑞家的從門裏出來叫小廝們:“先別掃,奶奶出來了。”賈芸忙上前笑問:“二嬸嬸那去?”周瑞家的道:“老太太叫,想必是裁什麼尺頭。”正說著,只見一羣人簇著鳳姐出來了。\begin{note}庚側:當家人有是派頭。\end{note}賈芸深知鳳姐是喜奉承尚排場的,\begin{note}庚側:那一個不喜奉承。\end{note}忙把手逼著,恭恭敬敬搶上來請安。鳳姐連正眼也不看,仍往前走著,只問他母親好,“怎麼不來我們這裏逛逛?”賈芸道:“只是身上不大好,倒時常記掛著嬸子,要來瞧瞧,又不能來。”鳳姐笑道:“可是會撒謊,不是我提起他來,你就不說他想我了。”賈芸笑道:“侄兒不怕雷打了,就敢在長輩前撒謊。昨兒晚上還提起嬸子來,說嬸子身子生的單弱,事情又多,虧嬸子好大精神,竟料理的週週全全,要是差一點兒的,早累的不知怎麼樣呢。”\begin{note}庚眉:自往卜世仁處去已安排下的。芸哥可用。己冬夜。\end{note}
\end{parag}


\begin{parag}
    鳳姐聽了滿臉是笑,不由的便止了步,問道:“怎麼好好的你娘兒們在背地裏嚼起我來?”\begin{note}庚側:過下無痕,天然而來文字。\end{note}賈芸道:“有個原故,\begin{note}庚側:接得如何?\end{note}只因我有個朋友,家裏有幾個錢,現開香鋪。只因他身上捐著個通判,前兒選了雲南不知那一處,\begin{note}庚側:隨口語,極妙!\end{note}連家眷一齊去,把這香鋪也不在這裏開了。便把帳物攢了一攢,該給人的給人,該賤發的賤發了,\begin{note}蒙側:世法人情,隨便招來,皆是奇妙文章。\end{note}象這細貴的貨,都分著送與親朋。他就一共送了我些冰片,麝香。我就和我母親商量,\begin{note}庚側:像得緊,何嘗撒謊?\end{note}若要轉買,不但賣不出原價來,而且誰家拿這些銀子買這個作什麼,便是很有錢的大家子,也不過使個幾分幾錢就挺折腰了,若說送人,也沒個人配使這些,\begin{note}蒙側:作者是何神聖,具此等大光明眼,無微不照?\end{note}倒叫他一文不值半文轉賣了。因此我就想起嬸子來。往年間我還見嬸子大包的銀子買這些東西呢,別說今年貴妃宮中,就是這個端陽節下,不用說這些香料自然是比往常加上十倍去的。因此想來想去,只孝順嬸子一個人才合式,方不算遭塌這東西。”一邊說,一邊將一個錦匣舉起來。
\end{parag}


\begin{parag}
    鳳姐正是要辦端陽的節禮,採買香料藥餌的時節,忽見賈芸如此一來,聽這一篇話,心下又是得意又是歡喜,便命豐兒:“接過芸哥兒的來,\begin{note}庚側:像個嬸子口氣,好看殺!\end{note}送了家去,交給平兒。”因又說道:“看著你這樣知好歹,怪道你叔叔常提你,說你說話兒也明白,心裏有見識。”\begin{note}庚雙夾:看官須記,鳳姐所喜是奉承之言,打動了心,不是見物而歡喜,若說是見物而喜,便不是阿鳳矣。\end{note}賈芸聽這話入了港,便打進一步來,故意問道:“原來叔叔也曾提我的?”鳳姐見問,纔要告訴他與他管事情的那話,便忙又止住,心下想道:\begin{note}庚側:的是阿鳳行事心機筆意。\end{note}“我如今要告訴他那話,倒叫他看著我見不得東西似的,爲得了這點子香,就混許他管事了。今兒先別提起這事。”想畢,便把派他監種花木工程的事都隱瞞的一字不提,隨口說了兩句淡話,便往賈母那裏去了。賈芸也不好提的,只得回來。
\end{parag}


\begin{parag}
    因昨日見了寶玉,叫他到外書房等著,賈芸吃了飯便又進來,到賈母那邊儀門外綺霰齋書房裏來。只見焙茗,鋤藥兩個小廝下象棋,爲奪“車”正拌嘴,還有引泉、掃花、\begin{note}庚側:好名色。\end{note}挑雲、伴鶴四五個,又在房檐上掏小雀兒玩。
\end{parag}


\begin{parag}
    賈芸進入院內,把腳一跺,說道:“猴頭們淘氣,我來了。”衆小廝看見賈芸進來,都才散了。賈芸進入房內,便坐在椅子上問:“寶二爺沒下來?”焙茗道: “今兒總沒下來。二爺說什麼,我替你哨探哨探去。”\begin{note}庚側:五遁之外,名曰“哨探遁”法。\end{note}說著,便出去了。這裏賈芸便看字畫古玩,有一頓飯工夫還不見來,再看看別的小廝,都頑去了。正是煩悶,只聽門前嬌聲嫩語的叫了一聲“哥哥”。
\end{parag}


\begin{parag}
    賈芸往外瞧時,看是一個十六七歲的丫頭,生的倒也細巧幹淨。那丫頭見了賈芸,便抽身躲了過去。恰當很走來,見那丫頭在門前,便說道:“好,好,\begin{note}庚側:二“好”字是遮飾半句來不到語。\end{note}正抓不著個信兒。”賈芸見了焙茗,也就趕了出來,問怎麼樣。焙茗道:“等了這一日,也沒個人兒過來。這就是寶二爺房裏的。好姑娘,\begin{note}庚側:口氣極像。\end{note}你進去帶個信兒,就說廊上的二爺來了。”那丫頭聽說,方知是本家的爺們,便不似先前那等迴避,\begin{note}庚側:一句,禮當。\end{note}下死眼把賈芸釘了兩眼。\begin{note}庚側:這句是情孽上生。\end{note}聽那賈芸說道:“什麼是廊上廊下的,你只說是芸兒就是了。”半晌,那丫頭冷笑了一笑:\begin{note}庚側:神情是深知房中事的。\end{note}“依我說,二爺竟請回家去,有什麼話明兒再來。今兒晚上得空兒我回了他。”焙茗道:“這是怎麼說?”那丫頭道:“他\begin{note}庚側:一連兩個“他”字,怡紅院中使得,否則有假矣。\end{note}今兒也沒睡中覺,自然喫的晚飯早。晚上他又不下來。難道只是耍的二爺在這裏等著捱餓不成!不如家去,明兒來是正經。便是回來有人帶信,那都是不中用的。他不過口裏應著,他倒給帶呢!”賈芸聽這丫頭說話簡便俏麗,待要問他的名字,因是寶玉房裏的,又不便問,只得說道:“這話倒是,我明兒再來。”說著便往外走。焙茗道:“我倒茶去,\begin{note}庚側:滑賊。\end{note}二爺吃了茶再去。”賈芸一面走,一面回頭說: “不喫茶,我還有事呢。”口裏說話,眼睛瞧那丫頭還站在那裏呢。
\end{parag}


\begin{parag}
    那賈芸一徑回家。至次日來至大門前,可巧遇見鳳姐往那邊去請安,才上了車,見賈芸來,便命人喚住,隔窗子笑道:“芸兒,你竟有膽子在我的跟前弄鬼。\begin{note}庚側:也作得不像撒謊,用心機人可怕是此等處。\end{note}道你送東西給我,原來你有事求我。昨兒你叔叔才告訴我說你求他。”賈芸笑道:“求叔叔這事,嬸子休提,我昨兒正後悔呢。早知這樣,我竟一起頭求嬸子,這會子也早完了。誰承望叔叔竟不能的。”鳳姐笑道:“怪道你那裏沒成兒,昨兒又來尋我。”賈芸道:“嬸子辜負了我的孝心,我並沒有這個意思。若有這個意思,昨兒還不求嬸子。如今嬸子既知道了,我倒要把叔叔丟下,少不得求嬸子好歹疼我一點兒。”鳳姐冷笑道:“你們要揀遠路兒走,叫我也難說。\begin{note}庚側:曹操語。\end{note}早告訴我一聲兒,有什麼不成的,多大點子事,耽誤到這會子。那園子裏還要種花,我只想不出一個人來,你早來不早完了。”賈芸笑道:“既這樣,嬸子明兒就派我罷。”鳳姐半晌道:“這個我看著不大好。\begin{note}庚側:又一折。\end{note}等明年正月裏煙火燈燭那個大宗兒下來,再派你罷。”賈芸道:“好嬸子,先把這個派了我罷。果然這個辦的好,再派我那個。”鳳姐笑道:“你倒會拉長線兒。罷了,要不是你叔叔說,我不管你的事。\begin{note}庚側:總不認受冰麝賄。\end{note}我也不過吃了飯就過來,你到午錯的時候來領銀子,後兒就進去種樹。”說畢,令人駕起香車,一徑去了。
\end{parag}


\begin{parag}
    賈芸喜不自禁,來至綺霰齋打聽寶玉,誰知寶玉一早便往北靜王府裏去了。賈芸便呆呆的坐到晌午,打聽鳳姐回來,便寫個領票來領對牌。至院外,命人通報了,彩明走了出來,單要了領票進去,批了銀數年月,一併連對牌交與了賈芸。賈芸接了,看那批上銀數批了二百兩,心中喜不自禁,翻身走到銀庫上,交與收牌票的,領了銀子。回家告訴母親,自是母子俱各歡喜。次日一個五鼓,賈芸先找了倪二,將前銀按數還他。那倪二見賈芸有了銀子,他便按數收回,不在話下。這裏賈芸又拿了五十兩,出西門找到花兒匠方椿家裏去買樹,不在話下。\begin{note}庚雙夾:至此便完種樹工程。一者見得趲趕工程原非正文,不過虛描盛時光景,藉此以出情文。二者又爲避難法。若不如此了,必曰其樹其價怎麼,買定必株,豈不煩絮矣?\end{note}
\end{parag}


\begin{parag}
    如今且說寶玉,自那日見了賈芸,曾說明日著他進來說話兒。如此說了之後,他原是富貴公子的口角,那裏還把這個放在心上,因而便忘懷了。\begin{note}庚側:若是一個女孩子,可保不忘的。\end{note}這日晚上,從北靜王府裏回來,見過賈母,王夫人等,回至園內,換了衣服,正要洗澡。襲人因被薛寶釵煩了去打結子,秋紋,碧痕兩個去催水,檀雲又因他母親的生日接了出去,麝月又現在家中養病,雖還有幾個作粗活聽喚的丫頭,估著叫不著他們,都出去尋夥覓伴的玩去了。不想這一刻的工夫,\begin{note}庚雙夾:妙!必用“一刻”二字方是寶玉的房中,見得時時原有人的,又有今一刻無人,所謂湊巧其一也。\end{note}只剩了寶玉在房內。偏生的\begin{note}庚雙夾:三字不可少。\end{note}寶玉要喫茶,一連叫了兩三聲,方見兩三個老嬤嬤走進來。\begin{note}庚雙夾:妙!文字細密,一絲不落,非批得出者。\end{note}寶玉見了他們,連忙搖手兒說:“罷,罷,不用你們了。”\begin{note}庚雙夾:是寶玉口氣。\end{note}老婆子們只得退出。
\end{parag}


\begin{parag}
    寶玉見沒丫頭們,只得自己下來,拿了碗向茶壺去倒茶。只聽背後說道:“二爺仔細燙了手,讓我們來倒。”\begin{note}庚側:神龍變化之文,人豈能測?\end{note}一面說,一面走上來,早接了碗過去。寶玉倒唬了一跳,問:“你在那裏的?忽然來了,唬我一跳。”那丫頭一面遞茶,一面回說:“我在後院子裏,才從裏間的後門進來,難道二爺就沒聽見腳步響?”寶玉一面喫茶,一面\begin{note}庚雙夾:六個“一面”,是神情,並不覺厭。\end{note}仔細打量那丫頭:穿著幾件半新不舊的衣裳,倒是一頭黑鬒鬒的頭髮,挽著個髻,容長臉面,細巧身材,卻十分俏麗乾淨。\begin{note}庚雙夾:與賈芸目中所見不差。\end{note}寶玉看了,便笑問道:\begin{note}庚雙夾:神情寫得出。\end{note}“你也是我這屋裏的人麼?”\begin{note}庚雙夾:妙問。必如此問方是籠絡前文。\end{note}那丫頭道:“是的。”寶玉道:“既是這屋裏的,我怎麼不認得?”那丫頭聽說,便冷笑了一聲道:\begin{note}庚雙夾:神情如畫。\end{note}“認不得的也多,豈只我一個。從來我又不遞茶遞水,拿東拿西,眼見的事一點兒不作,那裏認得呢。”寶玉道:“你爲什麼不作那眼見的事?”\begin{note}庚側:這是下情不能上達意語也。\end{note}那丫頭道:
\end{parag}


\begin{parag}
    “這話我也難說。\begin{note}庚側:不伏氣語,況非爾可完,故云“難說”。\end{note}只是有一句話回二爺:昨兒有個什麼芸兒來找二爺。我想二爺不得空兒,便叫焙茗回他,叫他今日早起來,不想二爺又往北府裏去了。”剛說到這句話,只見秋紋,碧痕嘻嘻哈哈的說笑著進來,兩個人共提著一桶水,一手撩著衣裳,趔趔趄趄,潑潑撒撒的。那丫頭便忙迎去接。\begin{note}庚側:好!有眼色。\end{note}那秋紋碧痕正對著抱怨,“你溼了我的裙子”,那個又說“你踹了我的鞋”。忽見走出一個人來接水,二人看時,不是別人,原來是小紅。二人便都詫異,將水放下,忙進房來東瞧西望,\begin{note}庚側:四字漸露大丫頭素日怡紅細事也。\end{note}\begin{note}庚眉:怡紅細事俱用帶筆白描,是大章法也。丁亥夏。畸笏叟。\end{note}並沒個別人,只有寶玉,便心中大不自在。只得預備下洗澡之物,待寶玉脫了衣裳,二人便帶上門出來,\begin{note}庚側:清楚之至。\end{note}
\end{parag}


\begin{parag}
    走到那邊房內便找小紅,問他方纔在屋裏說什麼。小紅道:“我何曾在屋裏的?只因我的手帕子不見了,往後頭找手帕子去。不想二爺要茶喫,叫姐姐們一個沒有,是我進去了,才倒了茶,姐姐們便來了。”秋紋聽了,兜臉啐了一口,罵道:“沒臉的下流東西!正經叫你去催水去,你說有事故,倒叫我們去,你可等著做這個巧宗兒。\begin{note}庚側:難說小紅無心,白描。\end{note}一里一里的,這不上來了。難道我們倒跟不上你了?你也拿鏡子照照,配遞茶遞水不配!”\begin{note}庚側:“難說” 二字全在此句來。\end{note}碧痕道:“明兒我說給他們,凡要茶要水送東送西的事,咱們都別動,只叫他去便是了。”秋紋道:“這麼說,不如我們散了,單讓他在這屋裏呢。”二人你一句我一句,正鬧著,只見有個老嬤嬤進來傳鳳姐的話說:“明日有人帶花兒匠來種樹,叫你們嚴禁些,衣服裙子別混曬混晾的。那土山上一溜都都攔著幃幙呢,可別混跑。”秋紋便問:\begin{note}庚側:用秋紋問,是暗透之法。\end{note}“明兒不知是誰帶進匠人來監工?”那婆子道:“說什麼後廊上的芸哥兒。”秋紋,碧痕聽了都不知道,只管混問別的話。那小紅聽見了,\begin{note}庚側:可是暗透法。\end{note}心內卻明白,就知是昨兒外書房所見那人了。
\end{parag}


\begin{parag}
    原來這小紅本姓林,\begin{note}庚雙夾:又是個林。\end{note}小名紅玉,\begin{note}庚雙夾:“紅”字切“絳珠”,“玉”字則直通矣。\end{note}只因“玉”字犯了林黛玉、寶玉,\begin{note}庚雙夾:妙文。\end{note}便都把這個字隱起來,便都叫他“小紅”。原是榮國府中世代的舊僕,他父母現在收管各處房田事務。這紅玉年方十六歲,因分人在大觀園的時節,把他便分在怡紅院中,倒也清幽雅靜。不想後來命人進來居住,偏生這一所兒又被寶玉佔了。這紅玉雖然是個不諳事的丫頭,卻因他有三分容貌,\begin{note}庚雙夾:有三分容貌尚且不肯受屈,況黛玉等一干才貌者乎?\end{note}心內著實妄想癡心的往上攀高,\begin{note}庚雙夾:爭奪者同來一看。\end{note}每每的要在寶玉面前現弄現弄。只是寶玉身邊一干人,都是伶牙利爪的,\begin{note}庚側:“難說”的原故在此。\end{note}那裏插的下手去。不想今兒纔有些消息,\begin{note}庚側:餘前批不謬。\end{note}又遭秋紋等一場惡意,心內早灰了一半。\begin{note}庚雙夾:爭名奪利者齊來一哭。\end{note}正悶悶的,忽然聽見老嬤嬤說起賈芸來,不覺心中一動,便悶悶的回至房中,睡在牀上暗暗盤算,翻來掉去,正沒個抓尋。忽聽窗外低低的叫道:“紅玉,你的手帕子我拾在這裏呢。”紅玉聽了忙走出來看,不是別人,正是賈芸。紅玉不覺的粉面含羞,問道:“二爺在那裏拾著的?”賈芸笑道:“你過來,我告訴你。”一面說,一面就上來拉他。那紅玉急回身一跑,卻被門檻絆倒。\begin{note}庚側:睡夢中當然一跑,這方是怡紅之鬟。\end{note}要知端的,下回分解。
\end{parag}


\begin{parag}
    \begin{note}庚:《紅樓夢》寫夢章法總不雷同。此夢更寫的新奇,不見後文,不知是夢。\end{note}
\end{parag}


\begin{parag}
    \begin{note}紅玉在怡紅院爲諸環所掩,亦可謂生不遇時,但看後四章供阿鳳驅使可知。\end{note}
\end{parag}


\begin{parag}
    \begin{note}蒙回末總評:冷暖時,只自知,金剛卜氏渾閒事。眼中心,言中意,三生舊債原無底。任你貴比王侯,任你富似郭石,一時間,風流願,不怕死!\end{note}
\end{parag}

