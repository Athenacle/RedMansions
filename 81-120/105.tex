\chap{一百零五}{錦衣軍查抄寧國府 驄馬使彈劾平安州}



\begin{parag}
    話說賈政正在那裏設宴請酒,忽見賴大急忙走上榮禧堂來回賈政道:“有錦衣府堂官趙老爺帶領好幾位司官說來拜望。奴才要取職名來回,趙老爺說:‘我們至好,不用的。’一面就下車來走進來了。請老爺同爺們快接去。”賈政聽了,心想:“趙老爺並無來往,怎麼也來?現在有客,留他不便,不留又不好。”正自思想,賈璉說:“叔叔快去罷,再想一回,人都進來了。”正說着,只見二門上家人又報進來說:“趙老爺已進二門了。”賈政等搶步接去,只見趙堂官滿臉笑容,並不說什麼,一徑走上廳來。後面跟着五六位司官,也有認得的,也有不認得的,但是總不答話。賈政等心裏不得主意,只得跟了上來讓坐。衆親友也有認得趙堂官的,見他仰着臉不大理人,只拉着賈政的手,笑着說了幾句寒溫的話。衆人看見來頭不好,也有躲進裏間屋裏的,也有垂手侍立的。賈政正要帶笑敘話,只見家人慌張報道:“西平王爺到了。”賈政慌忙去接,已見王爺進來。趙堂官搶上去請了安,便說:“王爺已到,隨來各位老爺就該帶領府役把守前後門。”衆官應了出去。賈政等知事不好,連忙跪接。西平郡王用兩手扶起,笑嘻嘻的說道:“無事不敢輕造,有奉旨交辦事件,要赦老接旨。如今滿堂中筵席未散,想有親友在此未便,且請衆位府上親友各散,獨留本宅的人聽候。”趙堂官回說:“王爺雖是恩典,但東邊的事,這位王爺辦事認真,想是早已封門。”衆人知是兩府幹系,恨不能脫身。只見王爺笑道:“衆位只管就請,叫人來給我送出去,告訴錦衣府的官員說,這都是親友,不必盤查,快快放出。”那些親友聽見,就一溜煙如飛的出去了。獨有賈赦賈政一干人唬得面如土色,滿身發顫。不多一回,只見進來無數番役,各門把守。本宅上下人等,一步不能亂走。趙堂官便轉過一付臉來回王爺道:“請爺宣旨意,就好動手。”這些番役卻撩衣勒臂,專等旨意。西平王慢慢的說道:“小王奉旨帶領錦衣府趙全來查看賈赦家產。”賈赦等聽見,俱俯伏在地。王爺便站在上頭說:“有旨意:‘賈赦交通外官,依勢凌弱,辜負朕恩,有忝祖德,着革去世職。欽此。’”趙堂官一迭聲叫:“拿下賈赦,其餘皆看守。”維時賈赦,賈政,賈璉,賈珍,賈蓉,賈薔,賈芝,賈蘭俱在,惟寶玉假說有病,在賈母那邊打鬧,賈環本來不大見人的,所以就將現在幾人看住。趙堂官即叫他的家人:“傳齊司員,帶同番役,分頭按房抄查登帳。”這一言不打緊,唬得賈政上下人等面面相看,喜得番役家人摩拳擦掌,就要往各處動手。西平王道:“聞得赦老與政老同房各爨的,理應遵旨查看賈赦的家資,其餘且按房封鎖,我們復旨去再候定奪。”趙堂官站起來說:“回王爺:賈赦賈政並未分家,聞得他侄兒賈璉現在承總管家,不能不盡行查抄。”西平王聽了,也不言語。趙堂官便說:“賈璉賈赦兩處須得奴才帶領去查抄纔好。”西平王便說:“不必忙,先傳信後宅,且請內眷迴避,再查不遲。”一言未了,老趙家奴番役已經拉着本宅家人領路,分頭查抄去了。王爺喝命:“不許羅皁!待本爵自行查看。”說着,便慢慢的站起來要走,又吩咐說:“跟我的人一個不許動,都給我站在這裏候着,回來一齊瞧着登數。”正說着,只見錦衣司官跪稟說:“在內查出御用衣裙並多少禁用之物,不敢擅動,回來請示王爺。”一回兒又有一起人來攔住王爺,就回說:“東跨所抄出兩箱房地契又一箱借票,卻都是違例取利的。”老趙便說:“好個重利盤剝!很該全抄!請王爺就此坐下,叫奴才去全抄來再候定奪罷。”說着,只見王府長史來稟說:“守門軍傳進來說,主上特命北靜王到這裏宣旨,請爺接去。”趙堂官聽了,心裏喜歡說:“我好晦氣,碰着這個酸王。如今那位來了,我就好施威。”一面想着,也迎出來。
\end{parag}


\begin{parag}
    只見北靜王已到大廳,就向外站着,說:“有旨意,錦衣府趙全聽宣。”說:“奉旨意:‘着錦衣官惟提賈赦質審,餘交西平王遵旨查辦。欽此。’”西平王領了,好不喜歡,便與北靜王坐下,着趙堂官提取賈赦回衙。裏頭那些查抄的人聽得北靜王到,俱一齊出來,及聞趙堂官走了,大家沒趣,只得侍立聽候。北靜王便挑選兩個誠實司官並十來個老年番役,餘者一概逐出。西平王便說:“我正與老趙生氣。幸得王爺到來降旨,不然這裏很喫大虧。”北靜王說:“我在朝內聽見王爺奉旨查抄賈宅,我甚放心,諒這裏不致荼毒。不料老趙這麼混賬。但不知現在政老及寶玉在那裏,裏面不知鬧到怎麼樣了。”衆人回稟:“賈政等在下房看守着,裏面已抄得亂騰騰的了。”西平王便吩咐司員:“快將賈政帶來問話。”衆人命帶了上來。賈政跪了請安,不免含淚乞恩。北靜王便起身拉着,說:“政老放心。”便將旨意說了。賈政感激涕零,望北又謝了恩,仍上來聽候。王爺道:“政老,方纔老趙在這裏的時候,番役呈稟有禁用之物並重利欠票,我們也難掩過。這禁用之物原辦進貴妃用的,我們聲明,也無礙。獨是借券想個什麼法兒纔好。如今政老且帶司員實在將赦老家產呈出,也就了事,切不可再有隱匿,自幹罪戾。”賈政答應道:“犯官再不敢。但犯官祖父遺產並未分過,惟各人所住的房屋有的東西便爲己有。”兩王便說:“這也無妨,惟將赦老那一邊所有的交出就是了。”又吩咐司員等依命行去,不許胡混亂動。司員領命去了。
\end{parag}


\begin{parag}
    且說賈母那邊女眷也擺家宴,王夫人正在那邊說:“寶玉不到外頭,恐他老子生氣。”鳳姐帶病哼哼唧唧的說:“我看寶玉也不是怕人,他見前頭陪客的人也不少了,所以在這裏照應也是有的。倘或老爺想起裏頭少個人在那裏照應,太太便把寶兄弟獻出去,可不是好?”賈母笑道:“鳳丫頭病到這地位,這張嘴還是那麼尖巧。”正說到高興,只聽見邢夫人那邊的人一直聲的嚷進來說:“老太太,太太,不……不好了!多多少少的穿靴帶帽的強……強盜來了,翻箱倒籠的來拿東西。”賈母等聽着發呆。又見平兒披頭散髮拉着巧姐哭啼啼的來說:“不好了,我正與姐兒喫飯,只見來旺被人拴着進來說:‘姑娘快快傳進去,請太太們迴避,外面王爺就進來查抄家產。’我聽了着忙,正要進房拿要緊東西,被一夥人渾推渾趕出來的。咱們這裏該穿該帶的快快收拾。”王邢二夫人等聽得,俱魂飛天外,不知怎樣纔好。獨見鳳姐先前圓睜兩眼聽着,後來便一仰身栽到地下死了。賈母沒有聽完,便嚇得涕淚交流,連話也說不出來。那時一屋子人拉那個,扯那個,正鬧得翻天覆地,又聽見一迭聲嚷說:“叫裏面女眷們迴避,王爺進來了!”
\end{parag}


\begin{parag}
    可憐寶釵寶玉等正在沒法,只見地下這些丫頭婆子亂抬亂扯的時候,賈璉喘吁吁的跑進來說:“好了,好了,幸虧王爺救了我們了!”衆人正要問他,賈璉見鳳姐死在地下,哭着亂叫,又怕老太太嚇壞了,急得死去活來。還虧平兒將鳳姐叫醒,令人扶着,老太太也回過氣來,哭得氣短神昏,躺在炕上。李紈再三寬慰。然後賈璉定神將兩王恩典說明,惟恐賈母邢夫人知道賈赦被拿,又要唬死,暫且不敢明說,只得出來照料自己屋內。
\end{parag}


\begin{parag}
    一進屋門,只見箱開櫃破,物件搶得半空。此時急得兩眼直豎,淌淚發呆。聽見外頭叫,只得出來。見賈政同司員登記物件,一人報說:
\end{parag}


\begin{qute2sp}
    赤金首飾共一百二十三件,珠寶俱全。珍珠十三掛,淡金盤二件,金碗二對,金搶碗二個,金匙四十把,銀大碗八十個,銀盤二十個,三鑲金象牙筋二把,鍍金執壺四把,鍍金折盂三對,茶托二件,銀碟七十六件,銀酒杯三十六個。黑狐皮十八張,青狐六張,貂皮三十六張,黃狐三十張,猞猁猻皮十二張,麻葉皮三張,洋灰皮六十張,灰狐腿皮四十張,醬色羊皮二十張,猢狸皮二張,黃狐腿二把,小白狐皮二十塊,洋呢三十度,畢嘰二十三度,姑絨十二度,香鼠筒子十件,豆鼠皮四方,天鵝絨一卷,梅鹿皮一方,雲狐筒子二件,貉崽皮一卷,鴨皮七把,灰鼠一百六十張,獾子皮八張,虎皮六張,海豹三張,海龍十六張,灰色羊四十把,黑色羊皮六十三張,元狐帽沿十副,倭刀帽沿十二副,貂帽沿二副,小狐皮十六張,江貉皮二張,獺子皮二張,貓皮三十五張,倭股十二度,綢緞一百三十卷,紗綾一百八一卷,羽線縐三十二卷,氆氌三十卷,妝蟒緞八卷,葛布三捆,各色布三捆,各色皮衣一百三十二件,棉夾單紗絹衣三百四十件。玉玩三十二件,帶頭九副,銅錫等物五百餘件,鐘錶十八件,朝珠九掛,各色妝蟒三十四件,上用蟒緞迎手靠背三分,宮妝衣裙八套,脂玉圈帶一條,黃緞十二卷。潮銀五千二百兩,赤金五十兩,錢七千吊。
\end{qute2sp}


\begin{parag}
    一切動用傢伙攢釘登記,以及榮國賜第,俱一一開列,其房地契紙,家人文書,亦俱封裹。賈璉在旁邊竊聽,只不聽見報他的東西,心裏正在疑惑。只聞兩家王爺問賈政道:“所抄家資內有借券,實系盤剝,究是誰行的?政老據實纔好。”賈政聽了,跪在地下碰頭說:“實在犯官不理家務,這些事全不知道。問犯官侄兒賈璉才知。”賈璉連忙走上跪下,稟說:“這一箱文書既在奴才屋內抄出來的,敢說不知道麼。只求王爺開恩,奴才叔叔並不知道的。”兩王道:“你父已經獲罪,只可併案辦理。你今認了也是正理。如此叫人將賈璉看守,餘俱散收宅內。政老,你須小心候旨。我們進內復旨去了,這裏有官役看守。”說着,上轎出門。賈政等就在二門跪送。北靜王把手一伸,說:“請放心。”覺得臉上大有不忍之色。
\end{parag}


\begin{parag}
    此時賈政魂魄方定,猶是發怔。賈蘭便說:“請爺爺進內瞧老太太,再想法兒打聽東府裏的事。”賈政疾忙起身進內。只見各門上婦女亂糟糟的,不知要怎樣。賈政無心查問,一直到賈母房中,只見人人淚痕滿面,王夫人寶玉等圍住賈母,寂靜無言,各各掉淚。惟有邢夫人哭作一團。因見賈政進來,都說:“好了,好了!”便告訴老太太說:“老爺仍舊好好的進來,請老太太安心罷。”賈母奄奄一息的,微開雙目說:“我的兒,不想還見得着你!”一聲未了,便嚎啕的哭起來。於是滿屋裏人俱哭個不住。賈政恐哭壞老母,即收淚說:“老太太放心罷。本來事情原不小,蒙主上天恩,兩位王爺的恩典,萬般軫恤。就是大老爺暫時拘質,等問明白了,主上還有恩典。如今家裏一些也不動了。”賈母見賈赦不在,又傷心起來,賈政再三安慰方止。
\end{parag}


\begin{parag}
    衆人俱不敢走散,獨邢夫人回至自己那邊,見門總封鎖,丫頭婆子亦鎖在幾間屋內。邢夫人無處可走,放聲大哭起來,只得往鳳姐那邊去。見二門旁舍亦上封條,惟有屋門開着,裏頭嗚咽不絕。邢夫人進去,見鳳姐面如紙灰,閤眼躺着,平兒在旁暗哭。邢夫人打諒鳳姐死了,又哭起來。平兒迎上來說:“太太不要哭。奶奶擡回來覺着象是死的了,幸得歇息一回蘇過來,哭了幾聲,如今痰息氣定,略安一安神。太太也請定定神罷。但不知老太太怎樣了?”邢夫人也不答言,仍走到賈母那邊。見眼前俱是賈政的人,自己夫子被拘,媳婦病危,女兒受苦,現在身無所歸,那裏禁得住。衆人勸慰,李紈等令人收拾房屋請邢夫人暫住,王夫人撥人服侍。
\end{parag}


\begin{parag}
    賈政在外,心驚肉跳,拈鬚搓手的等候旨意。聽見外面看守軍人亂嚷道:“你到底是那一邊的?既碰在我們這裏,就記在這裏冊上。拴着他,交給裏頭錦衣府的爺們!”賈政出外看時,見是焦大,便說:“怎麼跑到這裏來?”焦大見問,便號天蹈地的哭道:“我天天勸,這些不長進的爺們,倒拿我當作冤家!連爺還不知道焦大跟着太爺受的苦!今朝弄到這個田地!珍大爺蓉哥兒都叫什麼王爺拿了去了,裏頭女主兒們都被什麼府裏衙役搶得披頭散髮擉在一處空房裏,那些不成材料的狗男女卻象豬狗似的攔起來了。所有的都抄出來擱着,木器釘得破爛,磁器打得粉碎。他們還要把我拴起來。我活了八九十歲,只有跟着太爺捆人的,那裏倒叫人捆起來!我便說我是西府裏,就跑出來。那些人不依,押到這裏,不想這裏也是那麼着。我如今也不要命了,和那些人拚了罷!”說着撞頭。衆役見他年老,又是兩王吩咐,不敢發狠,便說:“你老人家安靜些,這是奉旨的事。你且這裏歇歇,聽個信兒再說。”賈政聽明,雖不理他,但是心裏刀絞似的,便道:“完了,完了!不料我們一敗塗地如此!”正在着急聽候內信,只見薛蝌氣噓噓的跑進來說:“好容易進來了!姨父在那裏。”賈政道:“來得好,但是外頭怎麼放進來的?”薛蝌道:“我再三央說,又許他們錢,所以我才能夠出入的。”賈政便將抄去之事告訴了他,便煩去打聽打聽,”就有好親,在火頭上也不便送信,是你就好通信了。”薛蝌道:“這裏的事我倒想不到,那邊東府的事我已聽見說,完了。”賈政道:“究竟犯什麼事?”薛蝌道:“今朝爲我哥哥打聽決罪的事,在衙內聞得,有兩位御史風聞得珍大爺引誘世家子弟賭博,這款還輕,還有一大款是強佔良民妻女爲妾,因其女不從,凌逼致死。那御史恐怕不準,還將咱們家的鮑二拿去,又還拉出一個姓張的來。只怕連都察院都有不是,爲的是姓張的曾告過的。”賈政尚未聽完,便跺腳道:“了不得!罷了,罷了!”嘆了一口氣,撲簌簌的掉下淚來。
\end{parag}


\begin{parag}
    薛蝌寬慰了幾句,即便又出來打聽去了。隔了半日,仍舊進來說:“事情不好。我在刑科打聽,倒沒有聽見兩王復旨的信,但聽得說李御史今早參奏平安州奉承京官,迎合上司,虐害百姓,好幾大款。”賈政慌道:“那管他人的事,到底打聽我們的怎麼樣?”薛蝌道:“說是平安州就有我們,那參的京官就是赦老爺。說的是包攬詞訟,所以火上澆油。就是同朝這些官府,俱藏躲不迭,誰肯送信。就即如才散的這些親友,有的竟回家去了,也有遠遠兒的歇下打聽的。可恨那些貴本家便在路上說,‘祖宗擲下的功業,弄出事來了,不知道飛到那個頭上,大家也好施威。’”賈政沒有聽完,復又頓足道:“都是我們大爺忒糊塗,東府也忒不成事體。如今老太太與璉兒媳婦是死是活還不知道呢。你再打聽去,我到老太太那邊瞧瞧。若有信,能夠早一步纔好。”正說着,聽見裏頭亂嚷出來說:“老太太不好了!”急得賈政即忙進去。未知生死如何,下回分解。
\end{parag}
