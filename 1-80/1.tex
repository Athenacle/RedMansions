\chap{一}{甄士隱夢幻識通靈 賈雨村風塵懷閨秀}

\begin{parag}
    \begin{note} \begin{subnote} 按:此段批語混入正文。\end{subnote}
        甲、庚、蒙批:此書開卷第一回也,作者自雲:因曾歷過一番夢幻之後,故將真事隱去,而借通靈之說,撰此《石頭記》一書也。故曰“甄士隱夢幻識通靈”。但書中所記何事,又因何而撰是書哉?自又云:今風塵碌碌,一事無成,忽念及當日所有之女子,一一細推了去,覺其行止見識,皆出於我之上。何我堂堂之鬚眉,曾不若彼裙釵哉! 蒙側:何非夢幻,何不通靈?作者託言,原當有自。受氣清濁,本無男女之別。實愧則有餘,悔又無益之大無可奈何之日也!當此時,則自欲將已往所賴,上賴天恩,下承祖德,錦衣紈絝之時、飫甘饜美肥之日,背父母教育之恩,負師兄規訓之德,已至今日一事無成、半生潦倒之罪, 蒙側:明告看者。編述一記,以告普天下人。我之罪固不能免,然閨閣中本自歷歷有人,萬不可因我之不肖,自護其短,則一併使其泯滅也。 蒙側:因爲傳他,並可傳我。雖今日之茆椽蓬牖,瓦竈繩牀,其風晨月夕,階柳庭花,亦未有傷於我之襟懷筆墨者。雖我未學,下筆無文,何爲不用假語村言,敷演出一段故事來,以悅人之耳目哉。故曰“風塵懷閨秀”,乃是第一回題綱正義也。開卷即雲“風塵懷閨秀”,則知作者本意原爲記述當日閨友閨情,並非怨世罵時之書矣。雖一時有涉於世態,然亦不得不敘者,但非其本旨耳,閱者切記之。詩曰:
    \end{note}



    \begin{poem}
        \color{Mahogany}
        \begin{pl}浮生着甚苦奔忙?盛席華筵終散場。\end{pl}

        \begin{pl}悲喜千般同幻渺,古今一夢盡荒唐。\end{pl}

        \begin{pl}謾言紅袖啼痕重,更有情癡抱恨長。\end{pl}

        \begin{pl}字字看來皆是血,十年辛苦不尋常!\end{pl}
    \end{poem}

\end{parag}

\begin{parag}
    \begin{note}楊、庚、覺、舒回前:此回中凡用夢用幻等字,是提醒閱者眼目,亦是此書立意本旨。\end{note}

\end{parag}


\begin{parag}
    列位看官,你道此書從何而來?說起根由雖近荒唐
    \begin{note}甲側:自佔地步。自首荒唐。妙!\end{note},細諳則深有趣味。待在下將此來歷註明,方使聞者瞭然不惑。

\end{parag}


\begin{parag}
    原來,當年女媧氏煉石補天之時\begin{note}甲側:補天濟世,勿認真,用常言。\end{note},於大荒山\begin{note}甲側:荒唐也。\end{note}無稽崖\begin{note}甲側:無稽也。\end{note}煉成高經十二丈\begin{note}甲側:總應十二釵。\end{note}、方經二十四丈\begin{note}甲側:照應副十二釵。\end{note}頑石三萬六千五百零一塊。媧皇氏只用了三萬六千五百塊\begin{note}甲側:合周天之數。\end{note},只單單的剩下了一塊未用\begin{note}甲側:剩了這一塊便生出這許多故事。使當日雖不以此補天,就該去補地之坑陷,使地平坦,而不得有此一部鬼話。蒙側:數足,偏遺我。“不堪入選”句中透出心眼。\end{note},便棄在此山青埂峯下\begin{note}甲眉:妙!自謂落墮情根,故無補天之用。\end{note}。誰知此石自經煅煉之後,靈性已通\begin{note}甲側:煅煉後,性方通。甚哉!人生不能學也。\end{note},因見衆石俱得補天,獨自己無材不堪入選,遂自怨自嗟,日夜悲號慚愧。

\end{parag}


\begin{parag}
    一日,正當嗟悼之際,俄見一僧一道遠遠而來,生得氣骨不凡,丰神迥異,\begin{note}蒙雙:這是真像,非幻像也。靖眉:作者自己形容。\end{note}說說笑笑來至峯下,坐於石邊高談快論。先是說些雲山霧海,神僊玄幻之事,後便說到紅塵中榮華富貴。此石聽了,不覺打動凡心,也想要到人間去享一享這榮華富貴,但自恨粗蠢,不得已,便口吐人言,\begin{note}甲側:竟有人問口生於何處,其無心肝,可笑可恨之極!\end{note}向那僧道說道:“大師!弟子蠢物\begin{note}甲側:豈敢豈敢。\end{note},不能見禮了。適聞二位談那人世間榮耀繁華,心切慕之。弟子質雖粗蠢\begin{note}甲側:豈敢豈敢。\end{note},性卻稍通,況見二師仙形道體,定非凡品,必有補天濟世之材,利物濟人之德。如蒙發一點慈心,攜帶弟子得入紅塵,在那富貴場中、溫柔鄉里受享幾年,自當永佩洪恩,萬劫不忘也。”二仙師聽畢,齊憨笑道:“善哉,善哉!那紅塵中有卻有些樂事,但不能永遠依恃。況又有‘美中不足,好事多魔’八個字緊相連屬。瞬息間則又樂極悲生,人非物換。究竟是到頭一夢,萬境歸空。\begin{note}甲側:四句乃一部之總綱。\end{note}倒不如不去的好。”這石凡心已熾,那裏聽得進這話去,乃復苦求再四。二仙知不可強制,乃嘆道:“此亦靜極思動,無中生有之數也。既如此,我們便攜你去受享受享,只是到不得意時,切莫後悔。”石道:“自然,自然。”那僧又道:“若說你性靈,卻又如此質蠢,並更無奇貴之處,如此也只好踮腳而已。\begin{note}甲側:煅煉過尚與人踮腳,不學者又當如何?\end{note}也罷,我如今大施佛法助你助,待劫終之日,復還本質,以了此案。你道好否?\begin{note}甲側:妙!佛法亦須償還,況世人之債乎?近之賴債者來看此句。所謂遊戲筆墨也。\end{note}”石頭聽了,感謝不盡。那僧便唸咒書符,大展幻\begin{note}甲側:明點幻字。好!\end{note}術,將一塊大石,登時變成一塊鮮明瑩潔的美玉,且又縮成扇墜大小的可佩可拿。\begin{note}甲側:奇詭險怪之文,有如髯蘇《石鍾》《赤壁》用幻處。\end{note}那僧託於掌上,笑道:“形體倒也是個寶物了,\begin{note}甲側:自愧之語。蒙雙:世上人原自據看得見處爲憑。\end{note}還只沒有實在的好處\begin{note}甲側,蒙、戚、覺雙:妙極!今之金玉其外、敗絮其中者,見此大不歡喜。\end{note},得再鐫上數字,使人一見便知是奇物方妙。\begin{note}甲側:世上原宜假,不宜真也。諺雲:“一日賣了三千假,三日賣不出一個真。”信哉!\end{note}然後好攜你到那昌明隆盛之邦\begin{note}甲側:伏長安大都。\end{note},詩禮簪纓之族\begin{note}甲側:伏榮國府。\end{note},花錦繁華之地\begin{note}甲側:伏大觀園。\end{note},溫柔富貴之鄉\begin{note}甲側:伏紫芸軒。\end{note}去安身樂業。\begin{note}甲側:何不再添一句雲:“擇個絕世情癡作主人”?甲眉:昔子房後謁黃石公,惟見一石。子房當時恨不能隨此石去。餘亦恨不能隨此石去也。聊供閱者一笑。\end{note}”石頭聽了,喜不能禁,乃問:“不知賜了弟子那幾件奇處,\begin{note}甲側:可知若果有奇貴之處,自己亦不知者;若自以奇貴而居,究竟是無真奇貴之人。\end{note}又不知攜了弟子到何處?望乞明示,使弟子不惑。”那僧笑道:“你且莫問,日後自然明白的。”說着,便袖了那石,同那道人飄然而去,竟不知投奔何方何捨去了。

\end{parag}


\begin{parag}
    後來,不知又過了幾世幾劫。因有個空空道人訪道求仙,忽從這大荒山無稽崖青埂峯下經過,忽見一大塊石上字跡分明,編述歷歷。空空道人乃從頭一看,原來就是無材補天,幻形入世,\begin{note}甲側:八字便是作者一生慚恨。\end{note}蒙茫茫大士、渺渺真人攜入紅塵,歷盡一番離合悲歡、炎涼世態的一段故事。後面又有一首偈雲:
\end{parag}


\begin{poem}
    \begin{pl} 無材可與補蒼天,\end{pl}\begin{note}甲側:書之本旨。\end{note}

    \begin{pl} 枉入紅塵若許年!\end{pl}\begin{note}甲側:慚愧之言,嗚咽如聞。\end{note}

    \begin{pl} 此係身前身後事,\end{pl}

    \begin{pl} 倩誰寄去作神傳?\end{pl}
\end{poem}


\begin{parag}
    詩後便是此石墮落之鄉,投胎之處,親自經歷的一段陳跡故事。其中家庭閨閣瑣事,以及閒情詩詞到還全備,或\begin{note}甲側:或字謙得好。\end{note}可適情解悶,然朝代年紀、地輿邦國,\begin{note}甲側:若用此套者,胸中必無好文字,手中斷無新筆墨。\end{note}卻反失落無考。\begin{note}甲側:據餘說,卻大有考證。蒙側:妙在無考。\end{note}
\end{parag}


\begin{parag}
    空空道人遂向石頭說道:“石兄,你這一段故事,據你自己說有些趣味,故編寫在此,意欲問世傳奇。據我看來,第一件,無朝代年紀可考,\begin{note}甲側:先駁得妙。\end{note}第二件,幷無大賢大忠理朝廷治風俗的善政,\begin{note}甲側:將世人慾駁之腐言預先代人駁盡。妙!\end{note}其中只不過幾個異樣女子,或情或癡,或小才微善,亦無班姑蔡女之德能。我縱抄去,恐世人不愛看呢。”
\end{parag}


\begin{parag}
    石頭笑答道:“我師何太癡耶!若雲無朝代可考,今我師竟假借漢唐等年紀添綴,又有何難?\begin{note}甲側:所以答得好。\end{note}但我想,歷來野史,皆蹈一轍,莫如我這不借此套者,反倒新奇別緻,不過只取其事體情理罷了,又何必拘拘於朝代年紀哉!再者,市井俗人喜看理治之書者甚少,愛適趣閒文者特多。歷來野史,或訕謗君相,或貶人妻女,\begin{note}甲側:先批其大端。\end{note}姦淫兇惡,不可勝數。更有一種風月筆墨,其淫穢污臭,塗毒筆墨,壞人子弟,又不可勝數。至若佳人才子等書,則又千部共出一套,且其中終不能不涉於淫濫,以致滿紙潘安子建、西子文君,不過作者要寫出自己的那兩首情詩豔賦來,故假擬出男女二人名姓,又必旁出一小人其間撥亂,\begin{note}蒙側:放筆以情趣世人,幷評倒多少傳奇。文氣淋漓,字句切實。\end{note}亦如劇中之小丑然。且嬛婢開口即者也之乎,非文即理。故逐一看去,悉皆自相矛盾,大不近情理之話。竟不如我半世親睹親聞的這幾個女子,雖不敢說強似前代書中所有之人,但事蹟原委,亦可以消愁破悶,也有幾首歪詩熟話,可以噴飯供酒。至若離合悲歡,興衰際遇,則又追蹤躡跡,不敢稍加穿鑿,徒爲供人之目而反失其真傳者。\begin{note}甲眉:事則實事,然亦敘得有間架、有曲折、有順逆、有映帶、有隱有見、有正有閏,以致草蛇灰線、空谷傳聲、一擊兩鳴、明修棧道、暗渡 陳倉、雲龍霧雨、兩山對峙、烘雲托月、背面敷粉、千皴萬染諸奇書中之祕法,亦不復少。餘亦於逐回中搜剔刮剖明白註釋以待高明,再批示誤謬。甲眉:開卷一篇立意真,打破歷來小說窠臼 。閱其筆則是《莊子》《離騷》之亞。甲眉:斯亦太過。\end{note}今之人,貧者日爲衣食所累,富者又懷不足之心,縱然一時稍閒,又有貪淫戀色、好貨尋愁之事,那裏去有工夫看那理治之書?所以我這一段故事,也不願世人稱奇道妙,也不定要世人喜悅檢讀,\begin{note}甲側:轉得更好。\end{note}只願他們當那醉淫飽臥之時,或避世去愁之際,把此一玩,豈不省了些壽命筋力?就比那謀虛逐妄,卻也省了口舌是非之害,腿腳奔忙之苦。再者,亦令世人換新眼目,不比那些胡牽亂扯,忽離忽遇,滿紙才人淑女、子建文君、紅娘小玉等通共熟套之舊稿。我師意爲何如?”\begin{note}甲側:餘代空空道人答曰:“不獨破愁醒盹,且有大益。”\end{note}
\end{parag}


\begin{parag}
    空空道人聽如此說,思忖半晌,將《石頭記》\begin{note}甲側:本名。\end{note}再檢閱一遍,\begin{note}甲側:這空空道人也太小心了,想亦世之一腐儒耳。\end{note}因見上面雖有些指奸責佞貶惡誅邪之語,\begin{note}甲側:亦斷不可少。\end{note}亦非傷時罵世之旨,\begin{note}甲側:要緊句。\end{note}及至君仁臣良父慈子孝,凡倫常所關之處,皆是稱功頌德,眷眷無窮,實非別書之可比。雖其中大旨談情,亦不過實錄其事,又非假擬妄稱,\begin{note}甲側:要緊句。\end{note}一味淫邀艶約、私訂偷盟之可比。因毫不干涉時世,\begin{note}甲側:要緊句。\end{note}方從頭至尾抄錄回來,問世傳奇。從此空空道人因空見色,由色生情,傳情入色,自色悟空,遂易名爲情僧,改《石頭記》爲《情僧錄》。至吳玉峯題曰《紅樓夢》。東魯孔梅溪則題曰《風月寶鑑》。\begin{note}甲眉:雪芹舊有《風月寶鑑》之書,乃其弟棠村序也。今棠村已逝,餘睹新懷舊,故仍因之。\end{note}後因曹雪芹於悼紅軒中披閱十載,增刪五次,纂成目錄,分出章回,則題曰《金陵十二釵》。\begin{note}甲眉:若雲雪芹披閱增刪,然後開卷至此,這一篇楔子又系誰撰?足見作者之筆狡猾之甚。後文如此處者不少。這正是作者用畫家煙雲模糊處,觀者萬不可被作者瞞蔽了去,方是巨眼。\end{note}幷題一絕雲:
\end{parag}


\begin{poem}
    \begin{pl}滿紙荒唐言,一把辛酸淚!\end{pl}

    \begin{pl}都雲作者癡,誰解其中味?\end{pl}
    \begin{note}甲雙夾:此是第一首標題詩。甲眉:能解者方有辛酸之淚,哭成此書。壬午除夕,書未成,芹爲淚盡而逝。餘常哭芹,淚亦待盡。每思覓青埂峯再問石兄,奈不遇癩頭和尚何!悵悵!今而後惟願造化主再出一芹一脂,是書何幸,餘二人亦大快遂心於九泉矣。甲午八日淚筆。\end{note}
\end{poem}


\begin{parag}
    至脂硯齋甲抄閱再評,仍用《石頭記》。出則既明,且看石上是何故事。按那石上書雲:\begin{note}甲側:以下系石上所記之文。\end{note}
\end{parag}


\begin{parag}
    當日地陷東南,這東南一隅有處曰姑蘇,\begin{note}甲側:是金陵。\end{note}有城曰閶門者,最是紅塵中一二等富貴風流之地。\begin{note}甲側:妙極!是石頭口氣,惜米顛不遇此石。\end{note}這閶門外有個十里\begin{note}甲側:開口先雲勢利,是伏甄、封二姓之事。\end{note}街,街內有個仁清\begin{note}甲側:又言人情,總爲士隱火後伏筆。\end{note}巷,巷內有個古廟,因地方窄狹,\begin{note}甲側:世路寬平者甚少。亦鑿。\end{note}人皆呼作葫蘆\begin{note}甲側:糊塗也,故假語從此具焉。\end{note}廟。\begin{note}蒙側:畫的雖不依樣,卻是葫蘆。\end{note}廟旁住著一家鄉宦,\begin{note}甲側:不出榮國大族,先寫鄉宦小家,從小至大,是此書章法。\end{note}姓甄,\begin{note}甲眉:真。後之甄寶玉亦藉此音,後不注。\end{note}名費,\begin{note}甲側:廢。\end{note}字士隱。\begin{note}甲側:託言將真事隱去也。\end{note}嫡妻封\begin{note}甲側:風。因風俗來。\end{note}氏,情性賢淑,深明禮義。\begin{note}甲側:八字正是寫日後之香菱,見其根源不凡。\end{note}家中雖不甚富貴,然本地便也推他爲望族了。\begin{note}甲側:本地推爲望族,寧、榮則天下推爲望族,敘事有層落。\end{note}因這甄士隱稟性恬淡,不以功名爲念,\begin{note}甲側:自是羲皇上人,便可作是書之朝代年紀矣。總寫香菱根基,原與正十二釵無異。蒙側:伏筆。\end{note}每日只以觀花修竹,酌酒吟詩爲樂,倒是神仙一流人品。只是一件不足:如今年已半百,膝下無兒,\begin{note}甲側:所謂“美中不足”也。\end{note}只有一女,乳名英蓮,\begin{note}甲側:設雲“應憐”也。\end{note}年方三歲。
\end{parag}


\begin{parag}
    一日,炎夏永晝。\begin{note}甲側:熱日無多。\end{note}士隱於書房閒坐,至手倦拋書,伏几少憩,不覺朦朧睡去。夢至一處,不辨是何地方。忽見那廂來了一僧一道,\begin{note}甲側:是方從青埂峯袖石而來也,接得無痕。\end{note}且行且談。
\end{parag}


\begin{parag}
    只聽道人問道:“你攜了這蠢物,意欲何往?”那僧笑道:“你放心,如今現有一段風流公案正該了結,這一干風流冤家,尚未投胎入世。趁此機會,就將此蠢物夾帶於中,使他去經歷經歷。”那道人道:“原來近日風流冤孽又將造劫歷世去不成?\begin{note}蒙側:苦惱是“造劫歷世”,又不能不“造劫歷世”,悲夫!\end{note}但不知落於何方何處?”
\end{parag}


\begin{parag}
    那僧笑道:“此事說來好笑,竟是千古未聞的罕事。只因西方靈河岸上三生石畔,\begin{note}甲側:妙!所謂“三生石上舊精魂”也。甲眉:全用幻。情之至,莫如此。今採來壓卷,其後可知。\end{note}有絳\begin{note}甲側:點“紅”字。\end{note}珠\begin{note}甲側:細思“絳珠”二字豈非血淚乎。\end{note}草一株,時有赤瑕\begin{note}甲側:點“紅”字“玉”字二。甲眉:按“瑕”字本注:“玉小赤也,又玉有病也。”以此命名恰極。\end{note}宮神瑛\begin{note}甲側:單點“玉”字二。\end{note}侍者,日以甘露灌溉,這絳珠草便得久延歲月。後來既受天地精華,復得雨露滋養,遂得脫卻草胎木質,得換人形,僅修成個女體,終日遊於離恨天外,飢則食蜜青果爲膳,渴則飲灌愁海水爲湯。\begin{note}甲側:飲食之名奇甚,出身履歷更奇甚,寫黛玉來歷自與別個不同。\end{note}只因尚未酬報灌溉之德,故其五內便鬱結著一段纏綿不盡之意。\begin{note}甲側:妙極!恩怨不清,西方尚如此,況世之人乎?趣甚警甚!甲眉:以頑石草木爲偶,實歷盡風月波瀾,嚐遍情緣滋味,至無可如何,始結此木石因果,以泄胸中悒鬱。古人之“一花一石如有意,不語不笑能留人”,此之謂也。蒙側:點題處,清雅。\end{note}恰近日這神瑛侍者凡心偶熾,\begin{note}甲側:總悔輕舉妄動之意。\end{note}乘此昌明太平朝世,意欲下凡造歷幻\begin{note}甲側:點“幻”字。\end{note}緣,已在警幻\begin{note}甲側:又出一警幻,皆大關鍵處。\end{note}仙子案前掛了號。警幻亦曾問及灌溉之情未償,趁此倒可了結的。那絳珠仙子道:“他是甘露之惠,我幷無此水可還。他既下世爲人,我也去下世爲人,但把我一生所有的眼淚還他,也償還得過他了。”\begin{note}甲側:觀者至此請掩卷思想,歷來小說中可曾有此句?千古未聞之奇文。甲眉:知眼淚還債,大都作者一人耳。餘亦知此意,但不能說得出。蒙側:恩情山海債,唯有淚堪還。\end{note}因此一事,就勾出多少風流冤家來,\begin{note}甲側:餘不及一人者,蓋全部之主惟二玉二人也。\end{note}陪他們去了結此案。”
\end{parag}


\begin{parag}
    那道人道:“果是罕聞,實未聞有還淚之說。\begin{note}蒙側:作想得奇!\end{note}想來這一段故事,比歷來風月事故更加瑣碎細膩了。”那僧道:“歷來幾個風流人物,不過傳其大概以及詩詞篇章而已,至家庭閨閣中一飲一食,總未述記。再者,大半風月故事,不過偷香竊玉、暗約私奔而已,幷不曾將兒女之真情發泄一二。\begin{note}蒙側:所以別緻。\end{note}想這一干人入世,其情癡色鬼,賢愚不肖者,悉與前人傳述不同矣。”
\end{parag}


\begin{parag}
    那道人道:“趁此何不你我也去下世度脫\begin{note}蒙側:“度脫”,請問是幻不是幻?\end{note}幾個,豈不是一場功德?”那僧道:“正合吾意,你且同我到警幻仙子宮中,將蠢物交割清楚,待這一干風流孽鬼下世已完,你我再去。\begin{note}蒙側:幻中幻,何不可幻?情中情,誰又無情?不覺僧道亦入幻中矣。\end{note}如今雖已有一半落塵,然猶未全集。”\begin{note}甲側:若從頭逐個寫去,成何文字?《石頭記》得力處在此。丁亥春。\end{note}
\end{parag}


\begin{parag}
    道人道:“既如此,便隨你去來。”
\end{parag}


\begin{parag}
    卻說甄士隱俱聽得明白,但不知所云蠢物系何東西。遂不禁上前施禮,笑問道:“二仙師請了。”那僧道也忙答禮相問。士隱因說道:“適聞仙師所談因果,實人世罕聞者。但弟子愚濁,不能洞悉明白,若蒙大開癡頑,備細一聞,弟子則洗耳諦聽,稍能警省,亦可免沉倫之苦。”二仙笑道:“此乃玄機不可預泄者。到那時不要忘了我二人,便可跳出火坑矣。”士隱聽了,不便再問。因笑道:“玄機不可預泄,但適雲‘蠢物’,不知爲何,或可一見否?”那僧道:“若問此物,倒有一面之緣。”說著,取出遞與士隱。士隱接了看時,原來是塊鮮明美玉,上面字跡分明,鐫著“通靈寶玉”四字,\begin{note}甲側:凡三四次始出明玉形,隱屈之至。\end{note}後面還有幾行小字。正欲細看時,那僧便說已到幻境,\begin{note}甲側:又點“幻”字,雲書已入幻境矣。蒙側:幻中言幻,何等法門。\end{note}便強從手中奪了去,與道人竟過一大石牌坊,上書四個大字,乃是“太虛幻境”。\begin{note}甲側:四字可思。\end{note}兩邊又有一幅對聯,道是:\begin{note}蒙雙夾:無極太極之輪轉,色空之相生,四季之隨行,皆不過如此。\end{note}
\end{parag}


\begin{poem}
    \begin{pl}假作真時真亦假,無爲有處有還無。\end{pl}
    \begin{note}甲夾:疊用真假有無字,妙!\end{note}
\end{poem}


\begin{parag}
    士隱意欲也跟了過去,方舉步時,忽聽一聲霹靂,有若山崩地陷。士隱大叫一聲,定睛一看,\begin{note}蒙側:真是大警覺大轉身。\end{note}只見烈日炎炎,芭蕉冉冉,\begin{note}甲側:醒得無痕,不落舊套。\end{note}所夢之事便忘了對半。\begin{note}甲側:妙極!若記得,便是俗筆了。\end{note}
\end{parag}


\begin{parag}
    又見奶母正抱了英蓮走來。士隱見女兒越發生得粉妝玉琢,乖覺可喜,便伸手接來,抱在懷內,鬥他頑耍一回,又帶至街前,看那過會的熱鬧。方欲進來時,只見從那邊來了一僧一道,\begin{note}甲側:所謂“萬境都如夢境看”也。\end{note}那僧則癩頭跣腳,那道則跛足蓬頭,\begin{note}甲側:此則是幻像。\end{note}瘋瘋癲癲,揮霍談笑而至。及至到了他門前,看見士隱抱著英蓮,那僧便大哭起來,\begin{note}甲側:奇怪!所謂情僧也。\end{note}又向士隱道:“施主,你把這有命無運,累及爹孃之物,抱在懷內作甚?”\begin{note}甲眉:八個字屈死多少英雄?屈死多少忠臣孝子?屈死多少仁人志士?屈死多少詞客騷人?今又被作者將此一把眼淚灑與閨閣之中,見得裙釵尚遭逢此數,況天下之男子乎?看他所寫開卷之第一個女子便用此二語以定終身,則知託言寓意之旨,誰謂獨寄興於一“情”字耶!武侯之三分,武穆之二帝,二賢之恨,及今不盡,況今之草芥乎?家國君父事有大小之殊,其理其運其數則略無差異。知運知數者則必諒而後嘆也。\end{note}士隱聽了,知是瘋話,也不去睬他。那僧還說:“舍我罷,舍我罷!”士隱不耐煩,便抱女兒撤身要進去,\begin{note}蒙側:如果捨出,則不成幻境矣。行文至此,又不得不有此一語。\end{note}那僧乃指著他大笑,口內唸了四句言詞道:
\end{parag}


\begin{poem}
    \begin{pl}慣養嬌生笑你癡,\end{pl}\begin{note}甲側:爲天下父母癡心一哭。\end{note}

    \begin{pl}菱花空對雪澌澌。\end{pl}\begin{note}甲側:生不遇時。遇又非偶。\end{note}

    \begin{pl}好防佳節元宵後,\end{pl}\begin{note}甲側:前後一樣,不直雲前而云後,是諱知者。\end{note}

    \begin{pl}便是煙消火滅時!\end{pl}\begin{note}甲側:伏後文。\end{note}
\end{poem}


\begin{parag}
    士隱聽得明白,心下猶豫,意欲問他們來歷。只聽道人說道:“你我不必同行,就此分手,各幹營生去罷。三劫後,\begin{note}甲眉:佛以世謂“劫”,凡三十年爲一世。三劫者,想以九十春光寓言也。\end{note}我在北邙山等你,會齊了同往太虛幻境銷號。”那僧道:“妙妙妙!”說畢,二人一去,再不見個蹤影了。士隱心中此時自忖:這兩個人必有來歷,該試一問,如今悔卻晚也。
\end{parag}


\begin{parag}
    這士隱正癡想,忽見隔壁\begin{note}甲側:“隔壁”二字極細極險,記清。\end{note}葫蘆廟內寄居的一個窮儒,姓賈名化,\begin{note}甲側:假話。妙!\end{note}表字時飛,\begin{note}甲側:實非。妙!\end{note}別號雨村\begin{note}甲側:雨村者,村言粗語也。言以村粗之言演出一段假話也。\end{note}者走了出來。這賈雨村原系胡州\begin{note}甲側:胡謅也。\end{note}人氏,也是詩書仕宦之族,因他生於末世,\begin{note}甲側:又寫一末世男子。\end{note}父母祖宗根基已盡,人口衰喪,只剩得他一身一口,在家鄉無益。\begin{note}蒙側:形容落破詩書子弟,逼真。\end{note}因進京求取功名,再整基業。自前歲來此,又淹蹇住了,暫寄廟中安身,每日賣字作文爲生,\begin{note}蒙側:“廟中安身”、“賣字爲生”,想是過午不食的了。\end{note}故士隱常與他交接。\begin{note}甲側:又夾寫士隱實是翰林文苑,非守錢虜也,直灌入“慕雅女雅集苦吟詩”一回。\end{note}當下雨村見了士隱,忙施禮陪笑道:“老先生倚門佇望,敢是街市上有甚新聞否?”士隱笑道:“非也,適因小女啼哭,引他出來作耍,正是無聊之甚,兄來得正妙,請入小齋一談,彼此皆可消此永晝。”說著,便令人送女兒進去,自與雨村攜手來至書房中。小童獻茶。方談得三五句話,忽家人飛報:“嚴\begin{note}甲側:“炎”也。炎既來,火將至矣。\end{note}老爺來拜。”士隱慌的忙起身謝罪道:“恕誑駕之罪,略坐,弟即來陪。”雨村忙起身亦讓道:“老先生請便。晚生乃常造之客,稍候何妨。”\begin{note}蒙側:世態人情,如聞其聲。\end{note}說著,士隱已出前廳去了。
\end{parag}


\begin{parag}
    這裏雨村且翻弄書籍解悶。忽聽得窗外有女子嗽聲,雨村遂起身往窗外一看,原來是一個丫嬛,在那裏擷花,生得儀容不俗,眉目清明,\begin{note}甲側:八字足矣。\end{note}雖無十分姿色,卻亦有動人之處。\begin{note}甲眉:更好。這便是真正情理之文。可笑近之小說中滿紙“羞花閉月”等字。這是雨村目中,又不與後之人相似。\end{note}雨村不覺看的呆了。\begin{note}甲側:今古窮酸色心最重。\end{note}那甄家丫嬛擷了花,方欲走時,猛抬頭見窗內有人,敝巾舊服,雖是貧窘,然生得腰圓背厚,面闊口方,更兼劍眉星眼,直鼻權腮。\begin{note}甲側:是莽操遺容。甲眉:最可笑世之小說中,凡寫奸人則用“鼠耳鷹腮”等語。\end{note}這丫嬛忙轉身迴避,心下乃想:“這人生的這樣雄壯,卻又這樣襤褸,想他定是我家主人常說的什麼賈雨村了,每有意幫助賙濟,只是沒甚機會。我家幷無這樣貧窘親友,想定是此人無疑了。怪道又說他必非久困之人。”如此想來,不免又回頭兩次。\begin{note}甲眉:這方是女兒心中意中正文。又最恨近之小說中滿紙紅拂紫煙。蒙側:如此忖度,豈得爲無情?\end{note}雨村見他回了頭,便自爲這女子心中有意於他,\begin{note}甲側:今古窮酸皆會替女婦心中取中自己。\end{note}便狂喜不盡,自爲此女子必是個巨眼英雄,風塵中之知己也。\begin{note}蒙側:在此處已把種點出。\end{note}一時小童進來,雨村打聽得前面留飯,不可久待,遂從夾道中自便出門去了。士隱待客既散,知雨村自便,也不去再邀。
\end{parag}


\begin{parag}
    一日,早又中秋佳節。士隱家宴已畢,乃又另具一席於書房,卻自己步月至廟中來邀雨村。\begin{note}甲側:寫士隱愛才好客。\end{note}原來雨村自那日見了甄家之婢曾回顧他兩次,自爲是個知己,便時刻放在心上。\begin{note}蒙側:也是不得不留心。不獨因好色,多半感知音。\end{note}今又正值中秋,不免對月有懷,因而口占五言一律雲:\begin{note}甲雙夾:這是第一首詩。後文香奩閨情皆不落空。餘謂雪芹撰此書,中亦有傳詩之意。\end{note}
\end{parag}


\begin{poem}
    \begin{pl}未卜三生願,頻添一段愁。\end{pl}

    \begin{pl}悶來時斂額,行去幾回頭。\end{pl}

    \begin{pl}自顧風前影,誰堪月下儔?\end{pl}

    \begin{pl}蟾光如有意,先上玉人樓。\end{pl}
\end{poem}


\begin{parag}
    雨村吟罷,因又思及平生抱負,苦未逢時,乃又搔首對天長嘆,復高吟一聯曰:
\end{parag}


\begin{poem}
    \begin{pl}玉在匱中求善價,釵於奩內待時飛。\begin{note}甲側:表過黛玉則緊接上寶釵。甲夾:前用二玉合傳,今用二寶合傳,自是書中正眼。蒙側:偏有些脂氣。\end{note}\end{pl}\end{poem}


\begin{parag}
    恰值士隱走來聽見,笑道:“雨村兄真抱負不淺也!”雨村忙笑道:“不過偶吟前人之句,何敢狂誕至此。”因問:“老先生何興至此?”士隱笑道:“今夜中秋,俗謂‘團圓之節’,想尊兄旅寄僧房,不無寂寥之感,故特具小酌,邀兄到敝齋一飲,不知可納芹意否?”雨村聽了,幷不推辭,\begin{note}蒙側:“不推辭”語便不入估(俗)矣。\end{note}便笑道:“既蒙厚愛,何敢拂此盛情。”\begin{note}甲側:寫雨村豁達,氣象不俗。\end{note}說著,便同士隱復過這邊書院中來。
\end{parag}


\begin{parag}
    須臾茶畢,早已設下杯盤,那美酒佳餚自不必說。二人歸坐,先是款斟漫飲,次漸談至興濃,不覺飛觥限斝起來。當時街坊上家家簫管,戶戶絃歌,當頭一輪明月,飛彩凝輝,二人愈添豪興,酒到杯乾。雨村此時已有七八分酒意,狂興不禁,乃對月寓懷,口號一絕雲:
\end{parag}


\begin{poem}
    \begin{pl}時逢三五便團圓,\end{pl}\begin{note}甲側:是將發之機。\end{note}

    \begin{pl}滿把晴光護玉欄。\end{pl}\begin{note}甲側:奸雄心事,不覺露出。\end{note}

    \begin{pl}天上一輪才捧出,\end{pl}

    \begin{pl}人間萬姓仰頭看。\end{pl}\begin{note}甲眉:這首詩非本旨,不過欲出雨村,不得不有者。用中秋詩起,用中秋詩收,又用起詩社於秋日。所嘆者三春也,卻用三秋作關鍵。\end{note}
\end{poem}


\begin{parag}
    士隱聽了,大叫:“妙哉!吾每謂兄必非久居人下者,今所吟之句,飛騰之兆已見,不日可接履於雲霓之上矣。可賀,可賀!”\begin{note}蒙側:伏筆,作□言語。妙!\end{note}乃親斟一斗爲賀。\begin{note}甲側:這個“鬥”字莫作升斗之鬥看,可笑。\end{note}雨村因幹過,嘆道:“非晚生酒後狂言,若論時尚之學,\begin{note}甲側:四字新而含蓄最廣,若必指明,則又落套矣。\end{note}晚生也或可去充數沽名,只是目今行囊路費一概無措,神京路遠,非賴賣字撰文即能到者。”士隱不待說完,便道:“兄何不早言。愚每有此心,但每遇兄時,兄幷未談及,愚故未敢唐突。今既及此,愚雖不才,‘義利’二字卻還識得。\begin{note}蒙側:“義利”二字,時人故自不識。\end{note}且喜明歲正當大比,兄宜作速入都,春闈一戰,方不負兄之所學也。其盤費餘事,弟自代爲處置,亦不枉兄之謬識矣!”當下即命小童進去,速封五十兩白銀,幷兩套冬衣。\begin{note}甲眉:寫士隱如此豪爽,又無一些粘皮帶骨之氣相,愧殺近之讀書假道學矣。\end{note}又云:“十九日乃黃道之期,兄可即買舟西上,待雄飛高舉,明冬再晤,豈非大快之事耶!”雨村收了銀衣,不過略謝一語,幷不介意,仍是喫酒談笑。\begin{note}甲側:寫雨村真是個英雄。蒙側:託大處,即遇此等人,又不得太瑣細。\end{note}那天已交了三更,二人方散。
\end{parag}


\begin{parag}
    士隱送雨村去後,回房一覺,直至紅日三竿方醒。\begin{note}甲側:是宿酒。\end{note}因思昨夜之事,意欲再寫兩封薦書與雨村帶至神都,使雨村投謁個仕宦之家爲寄足之地。\begin{note}甲側:又周到如此。\end{note}因使人過去請時,那家人去了回來說:“和尚說,賈爺今日五鼓已進京去了,也曾留下話與和尚轉達老爺,說:‘讀書人不在黃道黑道,總以事理爲要,不及面辭了。’”\begin{note}甲側:寫雨村真令人爽快。\end{note}士隱聽了,也只得罷了。
\end{parag}


\begin{parag}
    真是閒處光陰易過,倏忽又是元霄佳節矣。士隱命家人霍啓\begin{note}甲側:妙!禍起也。此因事而命名。\end{note}抱了英蓮去看社火花燈,半夜中,霍啓因要小解,便將英蓮放在一家門檻上坐著。待他小解完了來抱時,那有英蓮的蹤影?急得霍啓直尋了半夜,至天明不見,那霍啓也就不敢回來見主人,便逃往他鄉去了。那士隱夫婦,見女兒一夜不歸,便知有些不妥,再使幾人去尋找,回來皆雲連音響皆無。夫妻二人,半世只生此女,一旦失落,豈不思想,因此晝夜啼哭,幾乎不曾尋死。\begin{note}甲眉:喝醒天下父母之癡心。蒙側:天下作子弟的,看了想去。\end{note}看看的一月,士隱先就得了一病,當時封氏孺人也因思女構疾,日日請醫療治。
\end{parag}


\begin{parag}
    不想這日三月十五,葫蘆廟中炸供,那些和尚不加小心,致使油鍋火逸,便燒著窗紙。此方人家多用竹籬木壁者,\begin{note}甲側:土俗人風。蒙側:交竹滑溜婉轉。\end{note}大抵也因劫數,於是接二連三,牽五掛四,將一條街燒得如火焰山一般。\begin{note}甲眉:寫出南直召禍之實病。\end{note}彼時雖有軍民來救,那火已成了勢,如何救得下?直燒了一夜,方漸漸的熄去,也不知燒了幾家。只可憐甄家在隔壁,早已燒成一片瓦礫場了。只有他夫婦幷幾個家人的性命不曾傷了。急得士隱惟跌足長嘆而已。只得與妻子商議,且到田莊上去安身。偏值近年水旱不收,鼠盜蜂起,無非搶田奪地,鼠竊狗偷,民不安生,因此官兵剿捕,難以安身。士隱只得將田莊都折變了,便攜了妻子與兩個丫嬛投他岳丈家去。
\end{parag}


\begin{parag}
    他岳丈名喚封肅,\begin{note}蒙雙夾:風俗。\end{note}本貫大如州人氏,\begin{note}甲眉:託言大概如此之風俗也。\end{note}雖是務農,家中都還殷實。今見女婿這等狼狽而來,心中便有些不樂。\begin{note}甲側:所以大概之人情如是,風俗如是也。蒙側:大都不過如此。\end{note}幸而\begin{note}蒙側:若非“幸而”,則有不留之意。\end{note}士隱還有折變田地的銀子未曾用完,拿出來託他隨分就價薄置些須房地,爲後日衣食之計。那封肅便半哄半賺,些須與他些薄田朽屋。士隱乃讀書之人,不慣生理稼穡等事,勉強支持了一二年,越覺窮了下去。封肅每見面時,便說些現成話,且人前人後又怨他們不善過活,只一味好喫懶作\begin{note}甲側:此等人何多之極。\end{note}等語。士隱知投人不著,心中未免悔恨,再兼上年驚唬,急忿怨痛,已有積傷,暮年之人,貧病交攻,竟漸漸的露出那下世的光景來。\begin{note}蒙側:几几乎。世人則不能止於几几乎,可悲!觀至此,不……\end{note}
\end{parag}


\begin{parag}
    可巧這日,拄了柺杖掙挫到街前散散心時,忽見那邊來了一個跛足道人,瘋癲落脫,麻屣鶉衣,口內念著幾句言詞,道是:
\end{parag}


\begin{poem}
    \begin{pl}世人都曉神仙好,惟有功名忘不了;\end{pl}

    \begin{pl}古今將相在何方?荒冢一堆草沒了!\end{pl}

    \begin{pl}世人都曉神仙好,只有金銀忘不了;\end{pl}

    \begin{pl}終朝只恨聚無多,及到多時眼閉了。\end{pl}

    \begin{pl}世人都曉神仙好,只有姣妻忘不了;\end{pl}
    \begin{note}蒙雙夾:要寫情要寫幻境,偏先寫出一篇奇人奇境來。\end{note}

    \begin{pl}夫妻日日說恩情,夫死又隨人去了。\end{pl}

    \begin{pl}世人都曉神仙好,只有兒孫忘不了;\end{pl}

    \begin{pl}癡心父母古來多,孝順子孫誰見了!\end{pl}
\end{poem}


\begin{parag}
    士隱聽了,便迎上來道:“你滿口說些什麼?只聽見些‘好’‘了’‘好’‘了’。那道人笑道:“你若果聽見‘好’‘了’二字,還算你明白。可知世上萬般,好便是了,了便是好。若不了,便不好,若要好,須是了。我這歌兒,便名《好了歌》”士隱本是有宿慧的,一聞此言,心中早已徹悟。因笑道:“且住!待我將你這《好了歌》解注出來何如?”道人笑道:“你解,你解。”士隱乃說道:
\end{parag}


\begin{poem}
    \begin{pl}陋室空堂,當年笏滿牀,\end{pl}\begin{note}甲側:寧、榮未有之先。\end{note}

    \begin{pl}衰草枯楊,曾爲歌舞場。\end{pl}\begin{note}甲側:寧、榮既敗之後。\end{note}

    \begin{pl}蛛絲兒結滿雕樑,\end{pl}\begin{note}甲側:瀟湘館、紫芸軒等處。\end{note}

    \begin{pl}綠紗今又糊在蓬窗上。\end{pl}\begin{note}甲側:雨村等一干新榮暴發之家。甲眉:先說場面,忽新忽敗,忽麗忽朽,已見得反覆不了。\end{note}

    \begin{pl}說什麼脂正濃,粉正香,如何兩鬢又成霜?\end{pl}\begin{note}甲側:寶釵、湘雲一干人。\end{note}

    \begin{pl}昨日黃土隴頭堆白骨,\end{pl}\begin{note}甲側:黛玉、晴雯一干人。\end{note}

    \begin{pl}今宵紅燈帳底臥鴛鴦。\end{pl}\begin{note}甲眉:一段妻妾迎新送死,倏恩倏愛,倏痛倏悲,纏綿不了。\end{note}

    \begin{pl}金滿箱,銀滿箱,\end{pl}\begin{note}甲側:熙鳳一干人。\end{note}

    \begin{pl}展眼乞丐人皆謗。\end{pl}\begin{note}甲側:甄玉、賈玉一干人。\end{note}

    \begin{pl}正嘆他人命不長,那知自已歸來喪!\end{pl}\begin{note}甲眉:一段石火光陰,悲喜不了。風露草霜,富貴嗜慾,貪婪不了。\end{note}

    \begin{pl}訓有方,保不定日後\end{pl}\begin{note}甲側:言父母死後之日。\end{note}作強梁。\begin{note}甲側:柳湘蓮一干人。\end{note}

    \begin{pl}擇膏粱,誰承望流落在煙花巷!\end{pl}\begin{note}甲眉:一段兒女死後無憑,生前空爲籌劃計算,癡心不了。\end{note}

    \begin{pl}因嫌紗帽小,致使鎖枷槓,\end{pl}\begin{note}甲側:賈赦、雨村一干人。\end{note}

    \begin{pl}昨憐破襖寒,今嫌紫蟒長。\end{pl}\begin{note}甲側:賈蘭、賈菌一干人。甲眉:一段功名升黜無時,強奪苦爭,喜懼不了。\end{note}
    \begin{pl}亂烘烘你方唱罷我登場,\end{pl}\begin{note}甲側:總收。甲眉:總收古今億兆癡人,共歷幻場,此幻事擾擾紛紛,無日可了。\end{note}

    \begin{pl}反認他鄉是故鄉。\end{pl}\begin{note}甲側:太虛幻境青埂峯一幷結住。\end{note}

    \begin{pl}甚荒唐,到頭來都是爲他人作嫁衣裳!\end{pl}\begin{note}甲側:語雖舊句,用於此妥極是極。苟能如此,便能了得。甲眉:此等歌謠原不宜太雅,恐其不能通俗,故只此便妙極。其說得痛切處,又非一味俗語可到。蒙雙夾:誰不解得世事如此,有龍象力者方能放得下。\end{note}
\end{poem}


\begin{parag}
    那瘋跛道人聽了,拍掌笑道:“解得切,解得切!”士隱便笑一聲“走罷!”\begin{note}甲側:如聞如見。甲眉:“走罷”二字真懸崖撒手,若個能行?蒙側:一轉念間登彼岸。靖眉:“走罷”二字,如見如聞,真懸崖撒手。非過來人,若個能行?\end{note}將道人肩上褡褳搶了過來背著,竟不回家,同了瘋道人飄飄而去。
\end{parag}


\begin{parag}
    當下烘動街坊,衆人當作一件新聞傳說。封氏聞得此信,哭個死去活來,只得與父親商議,遣人各處訪尋,那討音信?無奈何,少不得依靠著他父母度日。幸而身邊還有兩個舊日的丫嬛伏侍,主僕三人,日夜作些針線發賣,幫著父親用度。那封肅雖然日日報怨,也無可奈何了。
\end{parag}


\begin{parag}
    這日,那甄家大丫嬛在門前買線,忽聽得街上喝道之聲,衆人都說新太爺到任。丫嬛於是隱在門內看時,只見軍牢快手,一對一對的過去,俄而大轎抬著一個烏帽猩袍的官府過去。\begin{note}甲側:雨村別來無恙否?可賀可賀。甲眉:所謂“亂哄哄,你方唱罷我登場”是也。\end{note}丫嬛倒發了個怔,自思這官好面善,倒象在那裏見過的。於是進入房中,也就丟過不在心上。\begin{note}甲側:是無兒女之情,故有夫人之分。蒙側:起初到底有心乎?無心乎?\end{note}至晚間,正待歇息之時,忽聽一片聲打的門響,許多人亂嚷,說:“本府太爺差人來傳人問話。”\begin{note}蒙側:不忘情的先寫出頭一位來了。\end{note}封肅聽了,唬得目瞪口呆,不知有何禍事。
\end{parag}


\begin{parag}
    \begin{note}蒙:出口神奇,幻中不幻。文勢跳躍,情裏生情。借幻說法,而幻中更自多情,因情捉筆,而情裏偏成癡幻。試問君家識得否,色空空色兩無干。\end{note}
\end{parag}

