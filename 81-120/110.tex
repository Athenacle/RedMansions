\chap{一百一十}{史太君寿终归地府 王凤姐力诎失人心}



\begin{parag}
    却说贾母坐起说道:“我到你们家已经六十多年了。从年轻的时候到老来,福也享尽了。自你们老爷起,儿子孙子也都算是好的了。就是宝玉呢,我疼了他一场。”说到那里,拿眼满地下瞅着。王夫人便推宝玉走到床前。贾母从被窝里伸出手来拉着宝玉道:“我的儿,你要争气才好!”宝玉嘴里答应,心里一酸,那眼泪便要流下来,又不敢哭,只得站着,听贾母说道:“我想再见一个重孙子我就安心了。我的兰儿在那里呢?”李纨也推贾兰上去。贾母放了宝玉,拉着贾兰道:“你母亲是要孝顺的,将来你成了人,也叫你母亲风光风光。凤丫头呢?”凤姐本来站在贾母旁边,赶忙走到眼前说:“在这里呢。”贾母道:“我的儿,你是太聪明了,将来修修福罢。我也没有修什么,不过心实吃亏,那些吃斋念佛的事我也不大干,就是旧年叫人写了些《金刚经》送送人,不知送完了没有?”凤姐道:“没有呢。”贾母道:“早该施舍完了才好。我们大老爷和珍儿是在外头乐了,最可恶的是史丫头没良心,怎么总不来瞧我。”鸳鸯等明知其故,都不言语。贾母又瞧了一瞧宝钗,叹了口气,只见脸上发红。贾政知是回光返照,即忙进上参汤。贾母的牙关已经紧了,合了一回眼,又睁着满屋里瞧了一瞧。王夫人宝钗上去轻轻扶着,邢夫人凤姐等便忙穿衣,地下婆子们已将床安设停当,铺了被褥,听见贾母喉间略一响动,脸变笑容,竟是去了,享年八十三岁。众婆子疾忙停床。于是贾政等在外一边跪着,邢夫人等在内一边跪着,一齐举起哀来。外面家人各样预备齐全,只听里头信儿一传出来,从荣府大门起至内宅门扇扇大开,一色净白纸糊了,孝棚高起,大门前的牌楼立时竖起,上下人等登时成服。贾政报了丁忧。礼部奏闻,主上深仁厚泽,念及世代功勋,又系元妃祖母,赏银一千两,谕礼部主祭。家人们各处报丧。众亲友虽知贾家势败,今见圣恩隆重,都来探丧。择了吉时成殓,停灵正寝。贾赦不在家,贾政为长,宝玉,贾环,贾兰是亲孙,年纪又小,都应守灵。贾琏虽也是亲孙,带着贾蓉尚可分派家人办事。虽请了些男女外亲来照应,内里邢王二夫人,李纨,凤姐,宝钗等是应灵旁哭泣的,尤氏虽可照应,他贾珍外出依住荣府,一向总不上前,且又荣府的事不甚谙练。贾蓉的媳妇更不必说了。惜春年小,虽在这里长的,他于家事全不知道。所以内里竟无一人支持,只有凤姐可以照管里头的事。况又贾琏在外作主,里外他二人倒也相宜。
\end{parag}


\begin{parag}
    凤姐先前仗着自己的才干,原打量老太太死了他大有一番作用。邢王二夫人等本知他曾办过秦氏的事,必是妥当,于是仍叫凤姐总理里头的事。凤姐本不应辞,自然应了,心想:“这里的事本是我管的,那些家人更是我手下的人,太太和珍大嫂子的人本来难使唤些,如今他们都去了。银项虽没有了对牌,这种银子是现成的。外头的事又是他办着。虽说我现今身子不好,想来也不致落褒贬,必是比宁府里还得办些。”心下已定,且待明日接了三,后日一早便叫周瑞家的传出话去,将花名册取上来。凤姐一一的瞧了,统共只有男仆二十一人,女仆只有十九人,余者俱是些丫头,连各房算上,也不过三十多人,难以点派差使。心里想道:“这回老太太的事倒没有东府里的人多。”又将庄上的弄出几个,也不敷差遣。正在思算,只见一个小丫头过来说:“鸳鸯姐姐请奶奶。”凤姐只得过去。只见鸳鸯哭得泪人一般,一把拉着凤姐儿说道:“二奶奶请坐,我给二奶奶磕个头。虽说服中不行礼,这个头是要磕的。”鸳鸯说着跪下。慌的凤姐赶忙拉住,说道:“这是什么礼,有话好好的说。”鸳鸯跪着,凤姐便拉起来。鸳鸯说道:“老太太的事一应内外都是二爷和二奶奶办,这种银子是老太太留下的。老太太这一辈子也没有糟踏过什么银钱,如今临了这件大事,必得求二奶奶体体面面的办一办才好。我方才听见老爷说什么诗云子曰,我不懂,又说什么‘丧与其易,宁戚’,我听了不明白。我问宝二奶奶,说是老爷的意思老太太的丧事只要悲切才是真孝,不必糜费图好看的念头。我想老太太这样一个人,怎么不该体面些!我虽是奴才丫头,敢说什么,只是老太太疼二奶奶和我这一场,临死了还不叫他风光风光!我想二奶奶是能办大事的,故此我请二奶奶来求作个主。我生是跟老太太的人,老太太死了我也是跟老太太的,若是瞧不见老太太的事怎么办,将来怎么见老太太呢!”凤姐听了这话来的古怪,便说:“你放心,要体面是不难的。况且老爷虽说要省,那势派也错不得。便拿这项银子都花在老太太身上,也是该当的。”鸳鸯道:“老太太的遗言说,所有剩下的东西是给我们的,二奶奶倘或用着不够,只管拿这个去折变补上。就是老爷说什么,我也不好违老太太的遗言。那日老太太分派的时候不是老爷在这里听见的么。”凤姐道:“你素来最明白的,怎么这会子那样的着急起来了。”鸳鸯道:“不是我着急,为的是大太太是不管事的,老爷是怕招摇的,若是二奶奶心里也是老爷的想头,说抄过家的人家丧事还是这么好,将来又要抄起来,也就不顾起老太太来,怎么处!在我呢是个丫头,好歹碍不着,到底是这里的声名。”凤姐道:“我知道了,你只管放心,有我呢!”鸳鸯千恩万谢的托了凤姐。
\end{parag}


\begin{parag}
    那凤姐出来想道:“鸳鸯这东西好古怪,不知打了什么主意,论理老太太身上本该体面些。嗳,不要管他,且按着咱们家先前的样子办去。”于是叫了旺儿家的来把话传出去请二爷进来。不多时,贾琏进来,说道:“怎么找我?你在里头照应着些就是了。横竖作主是咱们二老爷,他说怎么着咱们就怎么着。”凤姐道:“你也说起这个话来了,可不是鸳鸯说的话应验了么。”贾琏道:“什么鸳鸯的话?”凤姐便将鸳鸯请进去的话述了一遍。贾琏道:“他们的话算什么。才刚二老爷叫我去,说老太太的事固要认真办理,但是知道的呢,说是老太太自己结果自己,不知道的只说咱们都隐匿起来了,如今很宽裕。老太太的这种银子用不了谁还要么,仍旧该用在老太太身上。老太太是在南边的坟地虽有,阴宅却没有。老太太的柩是要归到南边去的,留这银子在祖坟上盖起些房屋来,再余下的置买几顷祭田。咱们回去也好,就是不回去,也叫这些贫穷族中住着,也好按时按节早晚上香,时常祭扫祭扫。你想这些话可不是正经主意?据你这个话,难道都花了罢?”凤姐道:“银子发出来了没有?”贾琏道:“谁见过银子!我听见咱们太太听见了二老爷的话,极力的窜掇二太太和二老爷,说这是好主意。叫我怎么着!现在外头棚杠上要支几百银子,这会子还没有发出来。我要去,他们都说有,先叫外头办了回来再算。你想这些奴才们有钱的早溜了,按着册子叫去,有的说告病,有的说下庄子去了。走不动的有几个,只有赚钱的能耐,还有赔钱的本事么!”凤姐听了,呆了半天,说道:“这还办什么!”正说着,见来了一个丫头说:“大太太的话问二奶奶,今儿第三天了,里头还很乱,供了饭还叫亲戚们等着吗?叫了半天,来了菜,短了饭,这是什么办事的道理!”凤姐急忙进去,吆喝人来伺候,胡弄着将早饭打发了。偏偏那日人来的多,里头的人都死眉瞪眼的。凤姐只得在那里照料了一会子,又惦记着派人,赶着出来叫了旺儿家的传齐了家人女人们,一一分派了。众人都答应着不动。凤姐道:“什么时候,还不供饭!”众人道:“传饭是容易的,只要将里头的东西发出来,我们才好照管去。”凤姐道:“糊涂东西,派定了你们少不得有的。”众人只得勉强应着。凤姐即往上房取发应用之物,要去请示邢王二夫人,见人多难说,看那时候已经日渐平西了,只得找了鸳鸯,说要老太太存的这一分家伙。鸳鸯道:“你还问我呢,那一年二爷当了赎了来了么!”凤姐道:“不用银的金的,只要这一分平常使的。”鸳鸯道:“大太太珍大奶奶屋里使的是那里来的!”凤姐一想不差,转身就走,只得到王夫人那边找了玉钏彩云,才拿了一分出来,急忙叫彩明登帐,发与众人收管。
\end{parag}


\begin{parag}
    鸳鸯见凤姐这样慌张,又不好叫他回来,心想:“他头里作事何等爽利周到,如今怎么掣肘的这个样儿。我看这两三天连一点头脑都没有,不是老太太白疼了他了吗!”那里知邢夫人一听贾政的话,正合着将来家计艰难的心,巴不得留一点子作个收局。况且老太太的事原是长房作主,贾赦虽不在家,贾政又是拘泥的人,有件事便说请大奶奶的主意。邢夫人素知凤姐手脚大,贾琏的闹鬼,所以死拿住不放松。鸳鸯只道已将这项银两交了出去了,故见凤姐掣肘如此,便疑为不肯用心,便在贾母灵前唠唠叨叨哭个不了。邢夫人等听了话中有话,不想到自己不令凤姐便宜行事,反说凤丫头果然有些不用心。王夫人到了晚上叫了凤姐过来说:“咱们家虽说不济,外头的体面是要的。这两三日人来人往,我瞧着那些人都照应不到,想是你没有吩咐。还得你替我们操点心儿才好。”凤姐听了,呆了一会,要将银两不凑手的话说出,但是银钱是外头管的,王夫人说的是照应不到,凤姐也不敢辨,只好不言语。邢夫人在旁说道:“论理该是我们做媳妇的操心,本不是孙子媳妇的事。但是我们动不得身,所以托你的,你是打不得撒手的。”凤姐紫涨了脸,正要回说,只听外头鼓乐一奏,是烧黄昏纸的时候了,大家举起哀来,又不得说,凤姐原想回来再说,王夫人催他出去料理,说道:“这里有我们的,你快快儿的去料理明儿的事罢。”
\end{parag}


\begin{parag}
    凤姐不敢再言,只得含悲忍泣的出来,又叫人传齐了众人,又吩咐了一会,说:“大娘婶子们可怜我罢!我上头捱了好些说,为的是你们不齐截,叫人笑话。明儿你们豁出些辛苦来罢。”那些人回道:“奶奶办事不是今儿个一遭儿了,我们敢违拗吗。只是这回的事上头过于累赘。只说打发这顿饭罢,有的在这里吃,有的要在家里吃,请了那位太太,又是那位奶奶不来。诸如此类,那得齐全。还求奶奶劝劝那些姑娘们不要挑饬就好了。”凤姐道:“头一层是老太太的丫头们是难缠的,太太们的也难说话,叫我说谁去呢。”众人道:“从前奶奶在东府里还是署事,要打要骂,怎么这样锋利,谁敢不依。如今这些姑娘们都压不住了?”凤姐叹道:“东府里的事虽说托办的,太太虽在那里,不好意思说什么。如今是自己的事情,又是公中的,人人说得话。再者外头的银钱也叫不灵,即如棚里要一件东西,传了出来总不见拿进来。这叫我什么法儿呢。”众人道:“二爷在外头倒怕不应付么?”凤姐道:“还提那个,他也是那里为难。第一件银钱不在他手里,要一件得回一件,那里凑手。”众人道:“老太太这项银子不在二爷手里吗?”凤姐道:“你们回来问管事的便知道了。”众人道:“怨不得我们听见外头男人抱怨说:‘这么件大事,咱们一点摸不着,净当苦差!’叫人怎么能齐心呢?”凤姐道:“如今不用说了,眼面前的事大家留些神罢。倘或闹的上头有了什么说的,我和你们不依的。”众人道:“奶奶要怎么样他们敢抱怨吗,只是上头一人一个主意,我们实在难周到的。”凤姐听了没法,只得央说道:“好大娘们!明儿且帮我一天,等我把姑娘们闹明白了再说罢咧。”众人听命而去。
\end{parag}


\begin{parag}
    凤姐一肚子的委屈,愈想愈气,直到天亮又得上去。要把各处的人整理整理,又恐邢夫人生气,要和王夫人说,怎奈邢夫人挑唆。这些丫头们见邢夫人等不助着凤姐的威风,更加作践起他来。幸得平儿替凤姐排解,说是“二奶奶巴不得要好,只是老爷太太们吩咐了外头,不许糜费,所以我们二奶奶不能应付到了。”说过几次才得安静些。虽说僧经道忏,上祭挂帐,络绎不绝,终是银钱吝啬,谁肯踊跃,不过草草了事。连日王妃诰命也来得不少,凤姐也不能上去照应,只好在底下张罗,叫了那个,走了这个,发一回急,央及一会,胡弄过了一起,又打发一起。别说鸳鸯等看去不象样,连凤姐自己心里也过不去了。
\end{parag}


\begin{parag}
    邢夫人虽说是冢妇,仗着“悲戚为孝”四个字,倒也都不理会。王夫人落得跟了邢夫人行事,余者更不必说了。独有李纨瞧出凤姐的苦处,也不敢替他说话,只自叹道:“俗话说的,‘牡丹虽好,全仗绿叶扶持’,太太们不亏了凤丫头,那些人还帮着吗!若是三姑娘在家还好,如今只有他几个自己的人瞎张罗,面前背后的也抱怨说是一个钱摸不着,脸面也不能剩一点儿。老爷是一味的尽孝,庶务上头不大明白,这样的一件大事,不撒散几个钱就办的开了吗!可怜凤丫头闹了几年,不想在老太太的事上,只怕保不住脸了。”于是抽空儿叫了他的人来吩咐道:“你们别看着人家的样儿,也糟踏起琏二奶奶来。别打量什么穿孝守灵就算了大事了,不过混过几天就是了。看见那些人张罗不开,便插个手儿也未为不可,这也是公事,大家都该出力的。”那些素服李纨的人都答应着说:“大奶奶说得很是。我们也不敢那么着,只听见鸳鸯姐姐们的口话儿好象怪琏二奶奶的似的。”李纨道:“就是鸳鸯我也告诉过他,我说琏二奶奶并不是在老太太的事上不用心,只是银子钱都不在他手里,叫他巧媳妇还作的上没米的粥来吗?如今鸳鸯也知道了,所以他不怪他了。只是鸳鸯的样子竟是不象从前了,这也奇怪,那时候有老太太疼他倒没有作过什么威福,如今老太太死了,没有了仗腰子的了,我看他倒有些气质不大好了。我先前替他愁,这会子幸喜大老爷不在家才躲过去了,不然他有什么法儿。”
\end{parag}


\begin{parag}
    说着,只见贾兰走来说:“妈妈睡罢,一天到晚人来客去的也乏了,歇歇罢。我这几天总没有摸摸书本儿,今儿爷爷叫我家里睡,我喜欢的很,要理个一两本书才好。别等脱了孝再都忘了。”李纨道:“好孩子,看书呢自然是好的。今儿且歇歇罢,等老太太送了殡再看罢。”贾兰道:“妈妈要睡,我也就睡在被窝里头想想也罢了。”众人听了都夸道:“好哥儿,怎么这点年纪得了空儿就想到书上!不象宝二爷娶了亲的人还是那么孩子气,这几日跟着老爷跪着,瞧他很不受用,巴不得老爷一动身就跑过来找二奶奶,不知唧唧咕咕的说些什么,甚至弄的二奶奶都不理他了。他又去找琴姑娘,琴姑娘也远避他。邢姑娘也不很同他说话。倒是咱们本家的什么喜姑娘咧四姑娘咧,哥哥长哥哥短的和他亲蜜。我们看那宝二爷除了和奶奶姑娘们混混,只怕他心里也没有别的事,白过费了老太太的心,疼了他这么大,那里及兰哥儿一零儿呢。大奶奶,你将来是不愁的了。”李纨道:“就好也还小,只怕到他大了,咱们家还不知怎么样了呢!环哥儿你们瞧着怎么样?”众人道:“这一个更不象样儿了!两个眼睛倒象个活猴儿似的,东溜溜,西看看,虽在那里嚎丧,见了奶奶姑娘们来了,他在孝幔子里头净偷着眼儿瞧人呢。”李纨道:“他的年纪其实也不小了。前日听见说还要给他说亲呢,如今又得等着了。嗳,还有一件事,——咱们家这些人,我看来也是说不清的,且不必说闲话,——后日送殡各房的车辆是怎么样了?”众人道:“琏二奶奶这几天闹的象失魂落魄的样儿了,也没见传出去。昨儿听见我的男人说,琏二爷派了蔷二爷料理,说是咱们家的车也不够,赶车的也少,要到亲戚家去借去呢。”李纨笑道:“车也都是借得的么?”众人道:“奶奶说笑话儿了,车怎么借不得?只是那一日所有的亲戚都用车,只怕难借,想来还得雇呢。”李纨道:“底下人的只得雇,上头白车也有雇的么?”众人道:“现在大太太东府里的大奶奶小蓉奶奶都没有车了,不雇那里来的呢?”李纨听了叹息道:“先前见有咱们家儿的太太奶奶们坐了雇的车来咱们都笑话,如今轮到自己头上了。你明儿去告诉你的男人,我们的车马早早儿的预备好了,省得挤。”众人答应了出去。不题。
\end{parag}


\begin{parag}
    且说史湘云因他女婿病着,贾母死后只来的一次,屈指算是后日送殡,不能不去。又见他女婿的病已成痨症,暂且不妨,只得坐夜前一日过来。想起贾母素日疼他,又想到自己命苦,刚配了一个才貌双全的男人,性情又好,偏偏的得了冤孽症候,不过捱日子罢了。于是更加悲痛,直哭了半夜。鸳鸯等再三劝慰不止。宝玉瞅着也不胜悲伤,又不好上前去劝,见他淡妆素服,不敷脂粉,更比未出嫁的时候犹胜几分。转念又看宝琴等淡素装饰,自有一种天生丰韵。独有宝钗浑身孝服,那知道比寻常穿颜色时更有一番雅致。心里想道:“所以千红万紫终让梅花为魁,殊不知并非为梅花开的早,竟是‘洁白清香’四字是不可及的了。但只这时候若有林妹妹也是这样打扮,又不知怎样的丰韵了!”想到这里,不觉的心酸起来,那泪珠便直滚滚的下来了,趁着贾母的事,不妨放声大哭。众人正劝湘云不止,外间又添出一个哭的来了。大家只道是想着贾母疼他的好处,所以伤悲,岂知他们两个人各自有各自的心事。这场大哭,不禁满屋的人无不下泪。还是薛姨妈李婶娘等劝住。
\end{parag}


\begin{parag}
    明日是坐夜之期,更加热闹。凤姐这日竟支撑不住,也无方法,只得用尽心力,甚至咽喉嚷破敷衍过了半日。到了下半天,人客更多了,事情也更繁了,瞻前不能顾后。正在着急,只见一个小丫头跑来说:“二奶奶在这里呢,怪不得大太太说,里头人多照应不过来,二奶奶是躲着受用去了。”凤姐听了这话,一口气撞上来,往下一咽,眼泪直流,只觉得眼前一黑,嗓子里一甜,便喷出鲜红的血来,身子站不住,就蹲倒在地。幸亏平儿急忙过来扶住。只见凤姐的血吐个不住。未知性命如何,下回分解。
\end{parag}