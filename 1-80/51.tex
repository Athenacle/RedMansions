\chap{五十一}{薛小妹新編懷古詩 胡庸醫亂用虎狼藥}


\begin{parag}
    \begin{note}蒙回前總:文有一語寫出大景者,如“園中不見一女子”句,儼然大家規模。“疑是姑娘”一語,又儼然庸醫口角,新醫行徑。筆大如椽。\end{note}
\end{parag}


\begin{parag}
    衆人聞得寶琴將素習所經過各省內的古蹟爲題,作了十首懷古絕句,內隱十物,皆說這自然新巧。都爭著看時,只見寫道是:
\end{parag}


\begin{poem}
    \begin{pl}赤壁懷古 \authorr{其一}\end{pl}

    \begin{pl}赤壁沉埋水不流,徒留名姓載空舟。\end{pl}

    \begin{pl}喧闐一炬悲風冷,無限英魂在內遊。\end{pl}

    \emptypl

    \begin{pl}交趾懷古 \authorr{其二}\end{pl}

    \begin{pl}銅鑄金鏞振紀綱,聲傳海外播戎羌。\end{pl}

    \begin{pl}馬援自是功勞大,鐵笛無煩說子房。\end{pl}

    \emptypl

    \begin{pl}鐘山懷古 \authorr{其三}\end{pl}

    \begin{pl}名利何曾伴汝身,無端被詔出凡塵。\end{pl}

    \begin{pl}牽連大抵難休絕,莫怨他人嘲笑頻。\end{pl}

    \emptypl
    \begin{pl}淮陰懷古 \authorr{其四}\end{pl}

    \begin{pl}壯士須防惡犬欺,三齊位定蓋棺時。\end{pl}

    \begin{pl}寄言世俗休輕鄙,一飯之恩死也知。\end{pl}

    \emptypl

    \begin{pl}廣陵懷古 \authorr{其五}\end{pl}

    \begin{pl}蟬噪鴉棲轉眼過,隋堤風景近如何。\end{pl}

    \begin{pl}只緣佔得風流號,惹得紛紛口舌多。\end{pl}

    \emptypl

    \begin{pl}桃葉渡懷古 \authorr{其六}\end{pl}

    \begin{pl}衰草閒花映淺池,桃枝桃葉總分離。\end{pl}

    \begin{pl}六朝樑棟多如許,小照空懸壁上題。\end{pl}

    \emptypl
    \begin{pl}青冢懷古 \authorr{其七}\end{pl}

    \begin{pl}黑水茫茫咽不流,冰弦撥盡曲中愁。\end{pl}

    \begin{pl}漢家制度誠堪嘆,樗櫟應慚萬古羞。\end{pl}

    \emptypl

    \begin{pl}馬嵬懷古 \authorr{其八}\end{pl}

    \begin{pl}寂寞脂痕漬汗光,溫柔一旦付東洋。\end{pl}

    \begin{pl}只因遺得風流跡,此日衣衾尚有香。\end{pl}

    \emptypl
    \begin{pl}蒲東寺懷古 \authorr{其九}\end{pl}

    \begin{pl}小紅骨賤最身輕,私掖偷攜強撮成。\end{pl}

    \begin{pl}雖被夫人時吊起,已經勾引彼同行。\end{pl}

    \emptypl
    \begin{pl}梅花觀懷古 \authorr{其十}\end{pl}

    \begin{pl}不在梅邊在柳邊,箇中誰拾畫嬋娟。\end{pl}

    \begin{pl}團圓莫憶春香到,一別西風又一年。\end{pl}

\end{poem}


\begin{parag}
    衆人看了,都稱奇道妙。寶釵先說道:“前八首都是史鑑上有據的;後二首卻無考,我們也不大懂得,不如另作兩首爲是。”\begin{note}庚雙夾:如何必得寶釵此駁方是好文,後文若真另作亦必無趣,若不另作,又有何法省之,看他下文如何。\end{note}黛玉忙攔道:\begin{note}庚雙夾:好極!非黛玉不可。脂硯。\end{note}“這寶姐姐也忒 ‘膠柱鼓瑟’,矯揉造作了。這兩首雖於史鑑上無考,咱們雖不曾看這些外傳,不知底裏,難道咱們連兩本戲也沒有見過不成?那三歲孩子也知道,何況咱們?”探春便道:“這話正是了。”\begin{note}庚雙夾:餘謂顰兒必有尖語來諷,不望竟有此飾詞代爲解釋,此則真心以待寶釵也。\end{note}李紈又道:“況且他原是到過這個地方的。這兩件事雖無考,古往今來,以訛傳訛,好事者竟故意的弄出這古蹟來以愚人。比如那年上京的時節,單是關夫子的墳,倒見了三四處。關夫子一生事業,皆是有據的,如何又有許多的墳?自然是後來人敬愛他生前爲人,只怕從這敬愛上穿鑿出來,也是有的。及至看《廣輿記》上,不止關夫子的墳多,自古來有些名望的人,墳就不少,無考的古蹟更多。如今這兩首雖無考,凡說書唱戲,甚至於求的簽上皆有註批,老小男女,俗語口頭,人人皆知皆說的。況且又並不是看了《西廂》《牡丹》的詞曲,怕看了邪書。這竟無妨,只管留著。”寶釵聽說,方罷了。\begin{note}庚雙夾:此爲三染無痕也,妙極!天花無縫之文。\end{note}大家猜了一回,皆不是。
\end{parag}


\begin{parag}
    冬日天短,不覺又是前頭喫晚飯之時,一齊前來喫飯。因有人回王夫人說:“襲人的哥哥花自芳進來說,他母親病重了,想他女兒。他來求恩典,接襲人家去走走。”王夫人聽了,便道:“人家母女一場,豈有不許他去的。”一面就叫了鳳姐兒來,告訴了鳳姐兒,命酌量去辦理。
\end{parag}


\begin{parag}
    鳳姐兒答應了,回至房中,便命周瑞家的去告訴襲人原故。又吩咐周瑞家的:“再將跟著出門的媳婦傳一個,你兩個人,再帶兩個小丫頭子,跟了襲人去。外頭派四個有年紀跟車的。要一輛大車,你們帶著坐;要一輛小車,給丫頭們坐。”周瑞家的答應了,纔要去,鳳姐兒又道:“那襲人是個省事的,你告訴他說我的話:叫他穿幾件顏色好衣裳,大大的包一包袱衣裳拿著,包袱也要好好的,手爐也要拿好的。臨走時,叫他先來我瞧瞧。”周瑞家的答應去了。
\end{parag}


\begin{parag}
    半日,果見襲人穿戴來了,兩個丫頭與周瑞家的拿著手爐與衣包。鳳姐兒看襲人頭上戴著幾枝金釵珠釧,倒華麗;又看身上穿著桃紅百子刻絲銀鼠襖子,蔥綠盤金彩繡綿裙,外面穿著青緞灰鼠褂。鳳姐兒笑道:“這三件衣裳都是太太的,賞了你倒是好的;但只這褂子太素了些,如今穿著也冷,你該穿一件大毛的。”襲人笑道:“太太就只給了這灰鼠的,還有一件銀鼠的。說趕年下再給大毛的,還沒有得呢。”鳳姐兒笑道:“我倒有一件大毛的,我嫌風毛兒出不好了,正要改去。也罷,先給你穿去罷。等年下太太給作的時節我再作罷,只當你還我一樣。”衆人都笑道:“奶奶慣會說這話。成年家大手大腳的,替太太不知背地裏賠墊了多少東西,真真的賠的是說不出來,那裏又和太太算去?偏這會子又說這小氣話取笑兒。”鳳姐兒笑道:“太太那裏想的到這些?究竟這又不是正經事,再不照管,也是大家的體面。說不得我自己喫些虧,把衆人打扮體統了,寧可我得個好名也罷了。一個一個象‘燒糊了的卷子’似的,人先笑話我當家倒把人弄出個花子來。”衆人聽了,都嘆說:“誰似奶奶這樣聖明!在上體貼太太,在下又疼顧下人。”一面說,一面只見鳳姐兒命平兒將昨日那件石青刻絲八團天馬皮褂子拿出來,與了襲人。又看包袱,只得一個彈墨花綾水紅綢裏的夾包袱,裏面只包著兩件半舊棉襖與皮褂。鳳姐兒又命平兒把一個玉色綢裏的哆羅呢的包袱拿出來,又命包上一件雪褂子。
\end{parag}


\begin{parag}
    平兒走去拿了出來,一件是半舊大紅猩猩氈的,一件是大紅羽紗的。襲道:“一件就當不起了。”平兒笑道:“你拿這猩猩氈的。把這件順手拿將出來,叫人給邢大姑娘送去。昨兒那麼大雪,人人都是有的,不是猩猩氈就是羽緞羽紗的,十來件大紅衣裳,映著大雪好不齊整。就只他穿著那件舊氈鬥蓬,越發顯的拱肩縮背,好不可憐見的。如今把這件給他罷。”鳳姐兒笑道:“我的東西,他私自就要給人。我一個還花不夠,再添上你提著,更好了!”衆人笑道:“這都是奶奶素日孝敬太太,疼愛下人。若是奶奶素日是小氣的,只以東西爲事,不顧下人的,姑娘那裏還敢這樣了。”鳳姐兒笑道:“所以知道我的心的,也就是他還知三分罷了。”說著,又囑咐襲人道:“你媽若好了就罷;若不中用了,只管住下,打發人來回我,我再另打發人給你送鋪蓋去。可別使人家的鋪蓋和梳頭的傢伙。”又吩咐周瑞家的道:“你們自然也知道這裏的規矩的,也不用我囑咐了。”周瑞家的答應:“都知道。我們這去到那裏,總叫他們的人迴避。若住下,必是另要一兩間內房的。”說著,跟了襲人出去,又吩咐預備燈籠,遂坐車往花自芳家來,不在話下。
\end{parag}


\begin{parag}
    這裏鳳姐又將怡紅院的嬤嬤喚了兩個來,吩咐道:“襲人只怕不來家,你們素日知道那大丫頭們,那兩個知好歹,派出來在寶玉屋裏上夜。你們也好生照管著,別由著寶玉胡鬧。”兩個嬤嬤去了,一時來回說:“派了晴雯和麝月在屋裏,我們四個人原是輪流著帶管上夜的。”鳳姐兒聽了,點頭道:“晚上催他早睡,早上催他早起。”老嬤嬤們答應了,自回園去。一時果有周瑞家的帶了信回鳳姐兒說:“襲人之母業已停牀,不能回來。”鳳姐兒回明瞭王夫人,一面著人往大觀園去取他的鋪蓋妝奩。
\end{parag}


\begin{parag}
    寶玉看著晴雯麝月二人打點妥當,送去之後,晴雯麝月皆卸罷殘妝,脫換過裙襖。晴雯只在熏籠上圍坐。麝月笑道:“你今兒別裝小姐了,我勸你也動一動兒。”晴雯道:“等你們都去盡了,我再動不遲。有你們一日,我且受用一日。”麝月笑道:“好姐姐,我鋪牀,你把那穿衣鏡的套子放下來,上頭的划子劃上,你的身量比我高些。”說著,便去與寶玉鋪牀。晴雯嗐了一聲,笑道:“人家才坐暖和了,你就來鬧。”此時寶玉正坐著納悶,想襲人之母不知是死是活,忽聽見晴雯如此說,便自己起身出去,放下鏡套,劃上消息,進來笑道:“你們暖和罷,都完了。”晴雯笑道:“終久暖和不成的,我又想起來湯婆子還沒拿來呢。”麝月道: “這難爲你想著!他素日又不要湯婆子,咱們那熏籠上暖和,比不得那屋裏炕冷,今兒可以不用。”寶玉笑道:“這個話,你們兩個都在那上頭睡了,我這外邊沒個人,我怪怕的,一夜也睡不著。”晴雯道:“我是在這裏。麝月往他外邊睡去。”說話之間,天已二更,麝月早已放下簾幔,移燈炷香,伏侍寶玉臥下,二人方睡。
\end{parag}


\begin{parag}
    晴雯自在熏籠上,麝月便在暖閣外邊。至三更以後,寶玉睡夢之中,便叫襲人。叫了兩聲,無人答應,自己醒了,方想起襲人不在家,自己也好笑起來。晴雯已醒,因笑喚麝月道:“連我都醒了,他守在旁邊還不知道,真是個挺死屍的。”麝月翻身打個哈氣笑道:“他叫襲人,與我什麼相干!”因問作什麼。寶玉要喫茶,麝月忙起來,單穿紅綢小棉襖兒。寶玉道:“披上我的襖兒再去,仔細冷著。”麝月聽說,回手便把寶玉披著起夜的一件貂頦滿襟暖襖披上,下去向盆內洗手,先倒了一鍾溫水,拿了大漱盂,寶玉漱了一口;然後才向茶格上取了茶碗,先用溫水涮了一涮,向暖壺中倒了半碗茶,遞與寶玉吃了;自己也漱了一漱,吃了半碗。晴雯笑道:“好妹子,也賞我一口兒。”麝月笑道:“越發上臉兒了!”晴雯道:“好妹妹,明兒晚上你別動,我伏侍你一夜,如何?”麝月聽說,只得也伏侍他漱了口,倒了半碗茶與他喫過。麝月笑道:“你們兩個別睡,說著話兒,我出去走走回來。”晴雯笑道:“外頭有個鬼等著你呢。” 寶玉道:“外頭自然有大月亮的,我們說話,你只管去。”一面說,一面便嗽了兩聲。
\end{parag}


\begin{parag}
    麝月便開了後門,揭起氈簾一看,果然好月色。晴雯等他出去,便欲唬他玩耍。仗著素日比別人氣壯,不畏寒冷,也不披衣,只穿著小襖,便躡手躡腳的下了薰籠,隨後出來。寶玉笑勸道:“看凍著,不是頑的。”晴雯只擺手,隨後出了房門。只見月光如水,忽然一陣微風,只覺侵肌透骨,不禁毛骨森然。心下自思道: “怪道人說熱身子不可被風吹,這一冷果然利害。”一面正要唬麝月,只聽寶玉高聲在內道:“晴雯出去了!”晴雯忙回身進來,笑道:“那裏就唬死了他?偏你慣會這蠍蠍螫螫老婆漢像的!”寶玉笑道:“倒不爲唬壞了他,頭一則你凍著也不好;二則他不防,不免一喊,倘或唬醒了別人,不說咱們是頑意,倒反說襲人才去了一夜,你們就見神見鬼的。你來把我的這邊被掖一掖。”晴雯聽說,便上來掖了掖,伸手進去渥一渥時,寶玉笑道:“好冷手!我說看凍著。”一面又見晴雯兩腮如胭脂一般,用手摸了一摸,也覺冰冷。寶玉道:“快進被來來渥渥罷。”一語未了,只聽咯噔的一聲門響,麝月慌慌張張的笑了進來,說道:“嚇了我一跳好的。黑影子裏,山子石後頭,只見一個人蹲著。我纔要叫喊,原來是那個大錦雞,見了人一飛,飛到亮處來,我纔看真了。若冒冒失失一嚷,倒鬧起人來。”一面說,一面洗手,又笑道:“晴雯出去我怎麼不見?一定是要唬我去了。”寶玉笑道:“這不是他,在這裏渥呢!我若不叫的快,可是倒唬一跳。”晴雯笑道:“也不用我唬去,這小蹄子已經自怪自驚的了。”一面說,一面仍回自己被中去了。麝月道:“你就這麼‘跑解馬’似的打扮得伶伶俐俐的出去了不成?”寶玉笑道:“可不就這麼去了。”麝月道:“你死不揀好日子!你出去站一站,把皮不凍破了你的。”說著,又將火盆上的銅罩揭起,拿灰鍬重將熟炭埋了一埋,拈了兩塊素香放上,仍舊罩了,至屏後重剔了燈,方纔睡下。
\end{parag}


\begin{parag}
    晴雯因方纔一冷,如今又一暖,不覺打了兩個噴嚏。寶玉嘆道:“如何?到底傷了風了。”麝月笑道:“他早起就嚷不受用,一日也沒喫飯。他這會還不保養些,還要捉弄人。明兒病了,叫他自作自受。”寶玉問:“頭上可熱?”晴雯嗽了兩聲,說道:“不相干,那裏這麼嬌嫩起來了。”說著,只聽外間房中十錦格上的自鳴鐘噹噹兩聲,外間值宿的老嬤嬤嗽了兩聲,因說道:“姑娘們睡罷,明兒再說罷。”寶玉方悄悄的笑道:“咱們別說話了,又惹他們說話。”說著,方大家睡了。
\end{parag}


\begin{parag}
    至次日起來,晴雯果覺有些鼻塞聲重,懶怠動彈。寶玉道:“快不要聲張!太太知道,又叫你搬了家去養息。家去雖好,到底冷些,不如在這裏。你就在裏間屋裏躺著,我叫人請了大夫,悄悄的從後門來瞧瞧就是了。”晴雯道:“雖如此說,你到底要告訴大奶奶一聲兒,不然一時大夫來了,人問起來,怎麼說呢?”寶玉聽了有理,便喚一個老嬤嬤吩咐道:“你回大奶奶去,就說晴雯白冷著了些,不是什麼大病。襲人又不在家,他若家去養病,這裏更沒有人了。傳一個大夫,悄悄的從後門進來瞧瞧,別回太太罷了。”老嬤嬤去了半日,來回說:“大奶奶知道了,說兩劑藥喫好了便罷,若不好時,還是出去爲是。如今時氣不好,恐沾帶了別人事小,姑娘們的身子要緊的。”晴雯睡在暖閣裏,只管咳嗽,聽了這話,氣的喊道:“我那裏就害瘟病了,只怕過了人!我離了這裏,看你們這一輩子都別頭疼腦熱的。”說著,便真要起來。寶玉忙按他,笑道:“別生氣,這原是他的責任,唯恐太太知道了說他不是,白說一句。你素習好生氣,如今肝火自然盛了。”
\end{parag}


\begin{parag}
    正說時,人回大夫來了。寶玉便走過來,避在書架之後。只見兩三個後門口的老嬤嬤帶了一個大夫進來。這裏的丫鬟都回避了,有三四個老嬤嬤放下暖閣上的大紅繡幔,晴雯從幔中單伸出手去。那大夫見這隻手上有兩根指甲,足有三寸長,尚有金鳳花染的通紅的痕跡,便忙回過頭來。有一個老嬤嬤忙拿了一塊手帕掩了。那大夫方診了一回脈,起身到外間,向嬤嬤們說道:“小姐的症是外感內滯,近日時氣不好,竟算是個小傷寒。幸虧是小姐素日飲食有限,風寒也不大,不過是血氣原弱,偶然沾帶了些,喫兩劑藥疏散疏散就好了。”說著,便又隨婆子們出去。
\end{parag}


\begin{parag}
    彼時,李紈已遣人知會過後門上的人及各處丫鬟迴避,那大夫只見了園中的景緻,並不曾見一女子。一時出了園門,就在守園門的小廝們的班房內坐了,開了藥方。老嬤嬤道:“你老爺且別去,我們小爺羅唆,恐怕還有話說。”大夫忙道:“方纔不是小姐,是位爺不成?那屋子竟是繡房一樣,又是放下幔子來的,如何是位爺呢?”老嬤嬤悄悄笑道:“我的老爺,怪道小廝們才說今兒請了一位新大夫來了,真不知我們家的事。那屋子是我們小哥兒的,那人是他屋裏的丫頭,倒是個大姐,那裏的小姐?若是小姐的繡房,小姐病了,你那麼容易就進去了?”說著,拿了藥方進去。
\end{parag}


\begin{parag}
    寶玉看時,上面有紫蘇、桔梗、防風、荊芥等藥,後面又有枳實、麻黃。寶玉道:“該死,該死,他拿著女孩兒們也象我們一樣的治,如何使得!憑他有什麼內滯,這枳實、麻黃如何禁得。誰請了來的?快打發他去罷!再請一個熟的來。”老婆子道:“用藥好不好,我們不知道這理。如今再叫小廝去請王太醫去倒容易,只是這大夫又不是告訴總管房請來的,這轎馬錢是要給他的。”寶玉道:“給他多少?”婆子道:“少了不好看,也得一兩銀子,纔是我們這門戶的禮。”寶玉道: “王太醫來了給他多少?”婆子笑道:“王太醫和張太醫每常來了,也並沒個給錢的,不過每年四節大躉送禮,那是一定的年例。這人新來了一次,須得給他一兩銀子去。”寶玉聽說,便命麝月去取銀子。麝月道:“花大奶奶還不知擱在那裏呢?”寶玉道:“我常見他在螺甸小櫃子裏取錢,我和你找去。”說著,二人來至寶玉堆東西的房子,開了螺甸櫃子,上一格子都是些筆墨、扇子、香餅、各色荷包、汗巾等物;下一格卻是幾串錢。於是開了抽屜,纔看見一個小簸籮內放著幾塊銀子,倒也有一把戥子。麝月便拿了一塊銀子,提起戥子來問寶玉:“那是一兩的星兒?”寶玉笑道:“你問我?有趣,你倒成了纔來的了。”麝月也笑了,又要去問人。寶玉道:“揀那大的給他一塊就是了。又不作買賣,算這些做什麼!”麝月聽了,便放下戥子,揀了一塊掂了一掂,笑道:“這一塊只怕是一兩了。寧可多些好,別少了,叫那窮小子笑話,不說咱們不識戥子,倒說咱們有心小器似的。”那婆子站在外頭臺磯上,笑道:“那是五兩的錠子夾了半邊,這一塊至少還有二兩呢!這會子又沒夾剪,姑娘收了這塊,再揀一塊小些的罷。”麝月早掩了櫃子出來,笑道:“誰又找去!多了些你拿了去罷。”寶玉道:“你只快叫茗煙再請王大夫去就是了。”婆子接了銀子,自去料理。
\end{parag}


\begin{parag}
    一時茗煙果請了王太醫來,診了脈後,說的病症與前相仿,只是方上果沒有枳實、麻黃等藥,倒有當歸、陳皮、白芍等,藥之分量較先也減了些。寶玉喜道: “這纔是女孩兒們的藥,雖然疏散,也不可太過。舊年我病了,卻是傷寒內裏飲食停滯,他瞧了,還說我禁不起麻黃、石膏、枳實等狼虎藥。我和你們一比,我就如那野墳圈子裏長的幾十年的一棵老楊樹,你們就如秋天芸兒進我的那纔開的白海棠,連我禁不起的藥,你們如何禁得起。”麝月等笑道:“野墳裏只有楊樹不成?難道就沒有松柏?我最嫌的是楊樹,那麼大笨樹,葉子只一點子,沒一絲風,他也是亂響。你偏比他,也太下流了。” 寶玉笑道:“松柏不敢比。連孔子都說:‘歲寒然後知松柏之後凋也。’可知這兩件東西高雅,不怕羞臊的纔拿他混比呢。”
\end{parag}


\begin{parag}
    說著,只見老婆子取了藥來。寶玉命把煎藥的銀吊子找了出來,\begin{note}庚雙夾:“找”字神理,乃不常用之物也。\end{note}就命在火盆上煎。晴雯因說:“正經給他們茶房裏煎去,弄得這屋裏藥氣,如何使得。”寶玉道:“藥氣比一切的花香果子香都雅。神仙採藥燒藥,再者高人逸士採藥治藥,最妙的一件東西。這屋裏我正想各色都齊了,就只少藥香,如今恰好全了。”一面說,一面早命人煨上。又囑咐麝月打點東西,遣老嬤嬤去看襲人,勸他少哭。一一妥當,方過前邊來賈母王夫人處問安喫飯。
\end{parag}


\begin{parag}
    正值鳳姐兒和賈母王夫人商議說:“天又短又冷,不如以後大嫂子帶著姑娘們在園子裏喫飯一樣。等天長暖和了,再來回的跑也不妨。”王夫人笑道:“這也是好主意。刮風下雪倒便宜。喫些東西受了冷氣也不好;空心走來,一肚子冷風,壓上些東西也不好。不如後園門裏頭的五間大房子,橫豎有女人們上夜的,挑兩個廚子女人在那裏,單給他姊妹們弄飯。新鮮菜蔬是有分例的,在總管房裏支去,或要錢,或要東西;那些野雞、獐、狍各樣野味,分些給他們就是了。”賈母道:“我也正想著呢,就怕又添一個廚房多事些。”鳳姐道:“並不多事。一樣的分例,這裏添了,那裏減了。就便多費些事,小姑娘們冷風朔氣的,\begin{note}庚雙夾: “朔”字又妙!“朔”作韶上音也,用此音奇想奇想。\end{note}別人還可,第一林妹妹如何禁得住?就連寶兄弟也禁不住,何況衆位姑娘。”賈母道:“正是這話了。上次我要說這話,我見你們的大事太多了,如今又添出這些事來,……”要知端的──
\end{parag}


\begin{parag}
    \begin{note}蒙回末總:此回再從猜謎著色,便與前回末犯重,且又是一幅即景聯詩圖矣,成何趣味?就燈謎中生一番讖評,別有情思,迥非凡豔。\end{note}
\end{parag}


\begin{parag}
    \begin{note}蒙回末總:閣起燈謎,接入襲人了卻不就襲人一面寫照,作者大有苦心。蓋襲人不盛飾則非大家威儀,如盛飾又豈有其母臨危?而盛飾者乎,在春(鳳)姐一面,衣服車馬僕從房屋鋪蓋等物一一點檢色色親囑,即得掌家人體統,而襲人之俊俏風神畢現。\end{note}
\end{parag}


\begin{parag}
    \begin{note}蒙回末總:文有數千言寫一瑣事者,如一喫茶,偏能於未喫之前既喫以後,細細描寫;如一拿銀,偏能於開櫃時生無數波折,平銀時又生無數波折。心細如髮。\end{note}
\end{parag}
