\chap{四十三}{闲取乐偶攒金庆寿 不了情暂撮土为香}
\begin{parag}
    \begin{note}蒙回前总:了与不了在心头,迷却原来难自由。如有如无谁解得,相生相灭苐传流。\end{note}
\end{parag}


\begin{parag}
    话说王夫人因见贾母那日在大观园不过著了些风寒,不是什么大病,请医生吃了两剂药也就好了,便放了心,因命凤姐来吩咐他预备给贾政带送东西。正商议著,只见贾母打发人来请,王夫人忙引著凤姐儿过来。王夫人又请问:“这会子可又觉大安些”?贾母道:“今日可大好了。方才你们送来野鸡崽子汤,我尝了一尝,倒有味儿,又吃了两块肉,心里很受用。”王夫人笑道:“这是凤丫头孝敬老太太的。算他的孝心虔,不枉了素日老太太疼他。”贾母点头笑道:“难为他想著。若是还有生的,再炸上两块,咸浸浸的,吃粥有味儿。那汤虽好,就只不对稀饭。”凤姐听了,连忙答应,命人去厨房传话。
\end{parag}


\begin{parag}
    这里贾母又向王夫人笑道:“我打发人请你来,不为别的。初二是凤丫头的生日,上两年我原早想替他做生日,偏到跟前有大事,就混过去了。今年人又齐全,料著又没事,咱们大家好生乐一日。”\begin{note}庚辰双夹:贾母犹云“好生乐一日”,可见逐日虽乐,皆还不称心也。所以世人不论贫富,各有愁肠,终不能时时遂心如意。此是至理,非不足语也。\end{note}王夫人笑道:“我也想著呢。既是老太太高兴,何不就商议定了?”贾母笑道:“我想往年不拘谁作生日,都是各自送各自的礼,这个也俗了,也觉生分的似的。今儿我出个新法子,又不生分,又可取笑。”王夫人忙道:“老太太怎么想著好,就是怎么样行。”贾母笑道:“我想著,咱们也学那小家子大家凑分子,\begin{note}庚辰双夹:原来请分子是小家的事,近见多少人家红白事一出且筹算分子之多寡,不知何说。\end{note}多少尽著这钱去办,你道好顽不好顽?”\begin{note}庚辰双夹:看他写与宝钗作生日,后又偏写与凤姐作生日。阿凤何人也,岂不为彼之华诞大用一回笔墨哉?只是亏他如何想来。特写于宝钗之后,较姊妹胜而有余;于贾母之前,较诸父母相去不远。一部书中若一个一个只管写过生日,复成何文哉?故起用宝钗,盛用阿凤,终用贾母,各有妙文,各有妙景。余者诸人或一笔不写,或偶因一语带过,或丰或简,其情当理合,不表可知。岂必谆谆死笔按数而写众人之生日哉?迥不犯宝钗。\end{note}王夫人笑道:“这个很好,但不知怎么凑法?”贾母听说,益发高兴起来,忙遣人去请薛姨妈邢夫人等,\begin{note}蒙侧:世家之长上多犯此等“办寿也要请人”毛病。\end{note}又叫请姑娘们并宝玉,那府里珍儿媳妇并赖大家的等有头脸管事的媳妇也都叫了来。
\end{parag}


\begin{parag}
    众丫头婆子见贾母十分高兴也都高兴,忙忙的各自分头去请的请,传的传,没顿饭的工夫,老的少的,上的下的,乌压压挤了一屋子。只薛姨妈和贾母对坐,邢夫人王夫人只坐在房门前两张椅子上,宝钗姊妹等五六个人坐在炕上,宝玉坐在贾母怀前,地下满满的站了一地。贾母忙命拿几个小杌子来,给赖大母亲等几个高年有体面的妈妈坐了。贾府风俗,年高服侍过父母的家人,比年轻的主子还有体面,所以尤氏凤姐儿等只管地下站著,那赖大的母亲等三四个妈妈告个罪,都坐在小杌子上了。
\end{parag}


\begin{parag}
    贾母笑著把方才一席话说与众人听了。众人谁不凑这趣儿?再也有和凤姐儿好的,有情愿这样的;有畏惧凤姐儿的,巴不得来奉承的:况且都是拿的出来的,所以一闻此言,都欣然应诺。贾母先道:“我出二十两。”薛姨妈笑道:“我随著老太太,也是二十两了。”邢夫人王夫人笑道:“我们不敢和老太太并肩,自然矮一等,每人十六两罢了。”尤氏李纨也笑道:“我们自然又矮一等,每人十二两罢。”贾母忙和李纨道:“你寡妇失业的,那里还拉你出这个钱,我替你出了罢。”\begin{note}庚辰双夹:必如是方妙。\end{note}凤姐忙笑道:“老太太别高兴,且算一算账再揽事。老太太身上已有两分呢,这会子又替大嫂子出十二两,说著高兴,一会子回想又心疼了。过后儿又说:‘都是为凤丫头花了钱。’使个巧法子,哄著我拿出三四分子来暗里补上,我还做梦呢。”说的众人都笑了。贾母笑道:“依你怎么样呢?”\begin{note}庚辰双夹:又写阿凤一样,更妙。若一笔直下,有何趣哉?\end{note}凤姐笑道:“生日没到,我这会子已经折受的不受用了。我一个钱饶不出,惊动这些人实在不安,不如大嫂子这一分我替他出了罢了。我到了那一日多吃些东西,就享了福了。”邢夫人等听了,都说:“很是。”贾母方允了。凤姐儿又笑道:“我还有一句话呢。我想老祖宗自己二十两,又有林妹妹宝兄弟的两分子。姨妈自己二十两,又有宝妹妹的一分子,这倒也公道。只是二位太太每位十六两,自己又少,又不替人出,这有些不公道。老祖宗吃了亏了!”贾母听了,忙笑道:“倒是我的凤姐儿向著我,这说的很是。要不是你,我叫他们又哄了去了。”凤姐笑道:“老祖宗只把他姐儿两个交给两位太太,一位占一个,派多派少,每位替出一分就是了。”贾母忙说:“这很公道,就是这样。”赖大的母亲忙站起来笑说道:“这可反了!我替二位太太生气。在那边是儿子媳妇,在这边是内侄女儿,倒不向著婆婆姑娘,倒向著别人。这儿媳妇成了陌路人,内侄女儿竟成了个外侄女儿了。”说的贾母与众人都大笑起来了。\begin{note}庚辰双夹:写阿凤全副精神,虽一戏,亦人想不到之文。\end{note}赖大之母因又问道:“少奶奶们十二两,我们自然也该矮一等了。”贾母听说,道:“这使不得。你们虽该矮一等,我知道你们这几个都是财主,分位虽低,钱却比他们多。\begin{note}庚辰双夹:惊魂夺魄只此一句。所以一部书全是老婆舌头,全是讽刺世事,反面春秋也。所谓“痴子弟正照风月鉴”,若单看了家常老婆舌头,岂非痴子弟乎?\end{note}你们和他们一例才使得。”众妈妈听了,连忙答应。贾母又道: “姑娘们不过应个景儿,每人照一个月的月例就是了。”又回头叫鸳鸯来,“你们也凑几个人,商议凑了来。”鸳鸯答应著,去不多时带了平儿、袭人、彩霞等还有几个小丫鬟来,也有二两的,也有一两的。贾母因问平儿:“你难道不替你主子作生日,还入在这里头?”平儿笑道:“我那个私自另外有了,这是官中的,也该出一分。”贾母笑道:“这才是好孩子。”凤姐又笑道:“上下都全了。还有二位姨奶奶,他出不出,也问一声儿。尽到他们是理,,不然,他们只当小看了他们了。”\begin{note}庚辰双夹:纯写阿凤以衬后文。\end{note}贾母听了,忙说:“可是呢,怎么倒忘了他们!只怕他们不得闲儿,叫一个丫头问问去。”说著,早有丫头去了,半日回来说道:“每位也出二两。”贾母喜道:“拿笔砚来算明,共计多少。”尤氏因悄骂凤姐道:“我把你这没足厌的小蹄子!这么些婆婆婶子来凑银子给你过生日,你还不足,又拉上两个苦瓠子作什么?”凤姐也悄笑道:“你少胡说,一会子离了这里,我才和你算账。他们两个为什么苦呢?有了钱也是白填送别人,不如拘来咱们乐。”\begin{note}庚辰双夹:纯写阿凤以衬后文,二人形景如见,语言如闻,真描画得到。\end{note}
\end{parag}


\begin{parag}
    说著,早已合算了,共凑了一百五十两有余。贾母道:“一日戏酒用不了。”尤氏道:“既不请客,酒席又不多,两三日的用度都够了。头等,戏不用钱,省在这上头。”贾母道:“凤丫头说那一班好,就传那一班。”凤姐儿道:“咱们家的班子都听熟了,倒是花几个钱叫一班来听听罢。”贾母道:“这件事我交给珍哥媳妇了。越性叫凤丫头别操一点心,受用一日才算。”\begin{note}庚辰双夹:所以特受用了,才有琏卿之变。乐极生悲,自然之理。\end{note}尤氏答应著。又说了一回话,都知贾母乏了,才渐渐的都散出来。
\end{parag}


\begin{parag}
    尤氏等送邢夫人王夫人二人散去,便往凤姐房里来商议怎么办生日的话。凤姐儿道:“你不用问我,你只看老太太的眼色行事就完了。”尤氏笑道:“你这阿物儿,也忒行了大运了。我当有什么事叫我们去,原来单为这个。出了钱不算,还要我来操心,你怎么谢我?”凤姐笑道:“你别扯臊,我又没叫你来,谢你什么!你怕操心?你这会子就回老太太去,再派一个就是了。”尤氏笑道:“你瞧他兴的这样儿!我劝你收著些儿好。太满了就泼出来了。”二人又说了一回方散。
\end{parag}


\begin{parag}
    次日将银子送到宁国府来,尤氏方才起来梳洗,因问是谁送过来的,丫鬟们回说:“是林大娘。”尤氏便命叫了他来。丫鬟走至下房,叫了林之孝家的过来。尤氏命他脚踏上坐了,一面忙著梳洗,一面问他:“这一包银子共多少?”林之孝家的回说:“这是我们底下人的银子,凑了先送过来。老太太和太太们的还没有呢。” 正说著,丫鬟们回说:“那府里太太和姨太太打发人送分子来了。”尤氏笑骂道:“小蹄子们,专会记得这些没要紧的话。昨儿不过老太太一时高兴,故意的要学那小家子凑分子,你们就记得,到了你们嘴里当正经的说。\begin{note}蒙侧:世家风调。\end{note}还不快接了进来好生待茶,再打发他们去。”丫鬟应著,忙接了进来,一共两封,连宝钗黛玉的都有了。尤氏问还少谁的,林之孝家的道:“还少老太太、太太、姑娘们的和底下姑娘们的。”尤氏道:“还有你们大奶奶的呢?”林之孝家的道: “奶奶过去,这银子都从二奶奶手里发,\begin{note}蒙侧:伏线。\end{note}一共都有了。”
\end{parag}


\begin{parag}
    说著,尤氏已梳洗了,命人伺候车辆。一时来至荣府,先来见凤姐。只见凤姐已将银子封好,正要送去。尤氏问:“都齐了?”凤姐儿笑道:\begin{note}庚辰双夹: “笑”字就有神情。\end{note}“都有了,快拿了去罢,丢了我不管。”尤氏笑道:“我有些信不及,倒要当面点一点。”说著果然按数一点,只没有李纨的一分。\begin{note}蒙侧:点明题目。\end{note}尤氏笑道:“我说你肏鬼呢,怎么你大嫂子的没有?”凤姐儿笑道:“那么些还不够使?短一分儿也罢了,等不够了我再给你。”\begin{note}庚辰双夹:可见阿凤处处心机。\end{note}尤氏道:“昨儿你在人跟前作人,今儿又来和我赖,这个断不依你。我只和老太太要去。”凤姐儿笑道:“我看你利害。明儿有了事,我也‘丁是丁卯是卯’的,你也别抱怨。”尤氏笑道:“你一般的也怕。不看你素日孝敬我,我才是不依你呢。”\begin{note}蒙侧:处处是世情作趣,处处是随笔埋伏。\end{note}说著,把平儿的一分拿了出来,说道:“平儿,来!把你的收起去,等不够了,我替你添上。”平儿会意,因说道:“奶奶先使著,若剩下了再赏我一样。”尤氏笑道:“只许你那主子作弊,就不许我作情儿。”\begin{note}蒙侧:请看。\end{note}平儿只得收了。尤氏又道:“我看著你主子这么细致,弄这些钱那里使去!使不了,明儿带了棺材里使去。”\begin{note}庚辰双夹:此言不假,伏下后文短命。尤氏亦能干事矣,惜不能劝夫治家,惜哉痛哉!\end{note}
\end{parag}


\begin{parag}
    一面说著,一面又往贾母处来。先请了安,大概说了两句话,便走到鸳鸯房中和鸳鸯商议,只听鸳鸯的主意行事,何以讨贾母的喜欢。二人计议妥当。尤氏临走时,也把鸳鸯二两银子还他,说:“这还使不了呢。”说著,一径出来,又至王夫人跟前说了一回话。因王夫人进了佛堂,把彩云一分也还了他。见凤姐不在跟前,一时把周、赵二人的也还了。\begin{note}蒙侧:另是一番作用。\end{note}他两个还不敢收。尤氏道:“你们可怜见的,那里有这些闲钱?凤丫头便知道了,有我应著呢。”二人听说,千恩万谢的方收了。\begin{note}庚辰双夹:尤氏亦可谓有才矣。论有德比阿凤高十倍,惜乎不能谏夫治家,所谓“人各有当”也。此方是至理至情,最恨近之野史中,恶则无往不恶,美则无一不美,何不近情理之如是耶?\end{note}于是尤氏一径出来,坐车回家。不在话下。
\end{parag}


\begin{parag}
    展眼已是九月初二日,园中人都打听得尤氏办得十分热闹,不但有戏,连耍百戏并说书的男女先儿全有,\begin{note}蒙侧:剩笔且影射能事者不独阿凤。\end{note}都打点取乐顽耍。李纨又向众姊妹道:“今儿是正经社日,可别忘了。\begin{note}庚辰双夹:看书者已忘,批书者亦已忘了,作者竟未忘,忽写此事,真忙中愈忙、紧处愈紧也。\end{note}宝玉也不来,想必他只图热闹,把清雅就丢开了。”\begin{note}庚辰双夹:此独宝玉乎?亦骂世人。余亦为宝玉忘了,不然何不来耶?\end{note}说著,便命丫鬟去瞧作什么,快请了来。丫鬟去了半日,回说:“花大姐姐说,今儿一早就出门去了。”\begin{note}庚辰双夹:奇文。\end{note}众人听了,都诧异说:“再没有出门之理。这丫头糊涂,不知说话。”因又命翠墨去。一时翠墨回来说:“可不真出了门了。说有个朋友死了,出去探丧去了。”\begin{note}庚辰双夹:奇文。信有之乎?花团锦簇之日偏如此写法。\end{note}探春道:“断然没有的事。凭他什么,再没今日出门之理。你叫袭人来,我问他。”刚说著,只见袭人走来。李纨等都说道:“今儿凭他有什么事,也不该出门。头一件,你二奶奶的生日,老太太都这等高兴,两府上下众人来凑热闹,他倒走了;\begin{note}蒙侧:因行文不肯平下一反笔,则文语并奇,好看煞人。\end{note}第二件,又是头一社的正日子,他也不告假,就私自去了!”袭人叹道:“昨儿晚上就说了,今儿一早起有要紧的事到北静王府里去,就赶回来的。劝他不要去,他必不依。今儿一早起来,又要素衣裳穿,想必是北静王府里的要紧姬妾没了,也未可知。”李纨等道:“若果如此,也该去走走,只是也该回来了。”说著,大家又商议:“咱们只管作诗,等他回来罚他。”刚说著,只见贾母已打发人来请,便都往前头来了。袭人回明宝玉的事,贾母不乐,便命人去接。
\end{parag}


\begin{parag}
    原来宝玉心里有件私事,于头一日就吩咐茗烟:“明日一早要出门,备下两匹马在后门口等著,不要别一个跟著。说给李贵,我往北府里去了。倘或要有人找我,叫他拦住不用找,只说北府里留下了,横竖就来的。”茗烟也摸不著头脑,只得依言说了。今儿一早,果然备了两匹马在园后门等著。天亮了,只见宝玉遍体纯素,从角门出来,一语不发跨上马,一弯腰,顺著街就颠下去了。茗烟也只得跨马加鞭赶上,在后面忙问:“往那里去?”宝玉道:“这条路是往那里去的?”茗烟道:“这是出北门的大道。出去了冷清清没有可顽的。”宝玉听说,点头道:“正要冷清清的地方好。”说著,越性加了鞭,那马早已转了两个弯子,出了城门。茗烟越发不得主意,只得紧紧跟著。
\end{parag}


\begin{parag}
    一气跑了七八里路出来,人烟渐渐稀少,宝玉方勒住马,回头问茗烟道:“这里可有卖香的?”焙茗道:“香倒有,不知是那一样?” 宝玉想道:“别的香不好,须得檀、芸、降三样。”茗烟笑道:“这三样可难得。”宝玉为难。茗烟见他为难,因问道:“要香作什么使?我见二爷时常小荷包有散香,何不找一找。”一句提醒了宝玉,便回手向衣襟上拉出一个荷包来,摸了一摸,竟有两星沉速,心内欢喜:“只是不恭些。”再想自己亲身带的,倒比买的又好些。于是又问炉炭。茗烟道:“这可罢了。荒郊野外那里有?用这些何不早说,带了来岂不便宜。”宝玉道:“糊涂东西,若可带了来,又不这样没命的跑了。”\begin{note}庚辰双夹:奇奇怪怪不知为何,看他下文怎样。\end{note}茗烟想了半日,笑道:“我得了个主意,不知二爷心下如何?我想二爷不只用这个呢,只怕还要用别的。这也不是事。如今我们往前再走二里地,就是水仙庵了。”宝玉听了忙问:“水仙庵就在这里?更好了,我们就去。”说著,就加鞭前行,一面回头向茗烟道:“这水仙庵的姑子长往咱们家去,咱们这一去到那里,和他借香炉使使,他自然是肯的。”茗烟道:“别说他是咱们家的香火,就是平白不认识的庙里,和他借,他也不敢驳回。只是一件,我常见二爷最厌这水仙庵的,如何今儿又这样喜欢了?”宝玉道:“我素日因恨俗人不知原故,混供神混盖庙,这都是当日有钱的老公们和那些有钱的愚妇们听见有个神,就盖起庙来供著,也不知那神是何人,因听些野史小说,便信真了。\begin{note}庚辰双夹:近闻刚丙庙又有三教庵,以如来为尊,太上为次,先师为末,真杀有余辜,所谓此书救世之溺不假。\end{note}比如这水仙庵里面因供的是洛神,故名水仙庵,殊不知古来并没有个洛神,那原是曹子建的谎话,谁知这起愚人就塑了像供著。今儿却合我的心事,故借他一用。”
\end{parag}


\begin{parag}
    说著早已来至门前。那老姑子见宝玉来了,事出意外,竟象天上掉下个活龙来的一般,忙上来问好,命老道来接马。宝玉进去,也不拜洛神之像,却只管赏鉴。虽是泥塑的,却真有“翩若惊鸿,婉若游龙”之态,“荷出绿波,日映朝霞”之姿。\begin{note}庚辰双夹:妙计!用《洛神赋》谮洛神本地风光,愈觉新奇。\end{note}宝玉不觉滴下泪来。老姑子献了茶。宝玉因和他借香炉。那姑子去了半日,连香供纸马都预备了来。宝玉道:“一概不用。”便命茗烟捧著炉出至后园中,拣一块干净地方儿,竟拣不出。茗烟道:“那井台儿上如何?”宝玉点头,一齐来至井台上,将炉放下。\begin{note}庚辰双夹:妙极之文。宝玉心中拣定是井台上了,故意使茗烟说出,使彼不犯疑猜矣。宝玉亦有欺人之才,盖不用耳。\end{note}
\end{parag}


\begin{parag}
    茗烟站过一旁。宝玉掏出香来焚上,含泪施了半礼,\begin{note}庚辰双夹:奇文。只云“施半礼”,终不知为何事也。\end{note}回身命收了去。茗烟答应,且不收,忙爬下磕了几个头,口内祝道:“我茗烟跟二爷这几年,二爷的心事,我没有不知道的,只有今儿这一祭祀没有告诉我,我也不敢问。只是这受祭的阴魂虽不知名姓,想来自然是那人间有一、天上无双,极聪明极俊雅的一位姐姐妹妹了。二爷心事不能出口,让我代祝:若芳魂有感,香魄多情,虽然阴阳间隔,既是知己之间,时常来望候二爷,未尝不可。你在阴间保佑二爷来生也变个女孩儿,和你们一处相伴,再不可又托生这须眉浊物了。”说毕,又磕几个头,才爬起来。\begin{note}庚辰双夹:忽插入茗烟一偏流言,粗看则小儿戏语,亦甚无味。细玩则大有深意,试思宝玉之为人岂不应有一极伶俐乖巧之小童哉?此一祝亦如《西厢记》中双文降香,第三柱则不语,红娘则代祝数语,直将双文心事道破。此处若写宝玉一祝,则成何文字?若不祝则成一哑迷,如何散场?故写茗烟一戏直戏入宝玉心中,又发出前文,又可收后文,又写茗烟素日之乖觉可人,且衬出宝玉直似一个守礼代嫁的女儿一般,其素日脂香粉气不待写而全现出矣。今看此回,直欲将宝玉当作一个极清俊羞怯的女儿,看茗烟则极乖觉可人之丫鬟也。 该 批:这方是作者真意。\end{note}
\end{parag}


\begin{parag}
    宝玉听他没说完,便撑不住笑了,\begin{note}庚辰双夹:方一笑,盖原可发笑,且说得合心,愈见可笑也。\end{note}因踢他道:“休胡说,看人听见笑话。”\begin{note}庚辰双夹:也知人笑,更奇。\end{note}茗烟起来收过香炉,和宝玉走著,因道:“我已经和姑子说了,二爷还没用饭,叫他随便收拾了些东西,二爷勉强吃些。我知道今儿咱们里头大排筵宴,热闹非常,二爷为此才躲了出来的。横竖在这里清净一天,也就尽到礼了。若不吃东西,断使不得。”宝玉道:“戏酒既不吃,这随便素的吃些何妨。”茗烟道:“这便才是。还有一说,咱们来了,还有人不放心。若没有人不放心,便晚了进城何妨?若有人不放心,二爷须得进城回家去才是。第一老太太、太太也放了心,第二礼也尽了,不过如此。就是家去了看戏吃酒,也并不是二爷有意,原不过陪著父母尽孝道。二爷若单为了这个不顾老太太、太太悬心,就是方才那受祭的阴魂也不安生。二爷想我这话如何?”宝玉笑道:“你的意思我猜著了,你想著只你一个跟了我出来,回来你怕担不是,所以拿这大题目来劝我。\begin{note}庚辰双夹:亦知这个大,妙极!\end{note}我才来了,不过为尽个礼,再去吃酒看戏,并没说一日不进城。这已完了心愿,赶著进城,大家放心,岂不两尽其道。”\begin{note}庚辰双夹:这是大通的意见,世人不及的去处。\end{note}茗烟道:“这更好了。”说著二人来至禅堂,果然那姑子收拾了一桌素菜,宝玉胡乱吃了些,茗烟也吃了。
\end{parag}


\begin{parag}
    二人便上马仍回旧路。茗烟在后面只嘱咐:“二爷好生骑著,这马总没大骑的,手里提紧著。”\begin{note}庚辰双夹:看他偏不写凤姐那样热闹,却写这般清冷,真世人意料不到这一篇文字也。\end{note}一面说著,早已进了城,仍从后门进去,忙忙来至怡红院中。袭人等都不在房里,只有几个老婆子看屋子,见他来了,都喜的眉开眼笑,说:“阿弥陀佛,可来了!把花姑娘急疯了!上头正坐席呢,二爷快去罢。”宝玉听说忙将素服脱了,自去寻了华服换上,问在什么地方坐席,老婆子回说在新盖的大花厅上。
\end{parag}


\begin{parag}
    宝玉听说,一径往花厅来,耳内早已隐隐闻得歌管之声。刚至穿堂那边,只见玉钏儿独坐在廊檐下垂泪,\begin{note}庚辰双夹:总是千奇百怪的文字。\end{note}一见他来,便收泪说道:“凤凰来了,快进去罢。再一会子不来,都反了。”\begin{note}庚辰双夹:是平常言语,却是无限文章,无限情理。看至后文在细思此言,则可知矣。\end{note}宝玉陪笑道:“你猜我往那里去了?”玉钏儿不答,只管擦泪。\begin{note}庚辰双夹:无限情理。\end{note}宝玉忙进厅里,见了贾母王夫人等,众人真如得了凤凰一般。宝玉忙赶著与凤姐儿行礼。贾母王夫人都说他不知道好歹,“怎么也不说声就私自跑了,这还了得!明儿再这样,等老爷回家来,必告诉他打你。”说著又骂跟的小厮们都偏听他的话,说那里去就去,也不回一声儿。一面又问他到底那去了,可吃了什么,可唬著了。\begin{note}庚辰双夹:奇文,毕肖。\end{note}宝玉只回说:“北静王的一个爱妾昨日没了,给他道恼去。他哭的那样,不好撇下就回来,所以多等了一会子。”贾母道:“以后再私自出门,不先告诉我们,一定叫你老子打你。”宝玉答应著。因又要打跟的小子们,众人又忙说情,又劝道:“老太太也不必过虑了,他已经回来,大家该放心乐一回了。”贾母先不放心,自然发狠,如今见他来了,喜且有余,那里还恨,也就不提了;还怕他不受用,或者别处没吃饱,路上著了惊怕,反百般的哄他。袭人早过来伏侍。大家仍旧看戏。当日演的是《荆钗记》。贾母薛姨妈等都看的心酸落泪,也有叹的,也有骂的。要知端的,下回分解。
\end{parag}


\begin{parag}
    \begin{note}蒙回末总:攒金办寿家常乐,素服焚香无限情。\end{note}
\end{parag}


\begin{parag}
    \begin{note}蒙回末总:写办事不独熙凤,写多情不漏亡人,情之所钟必让若辈。此所谓情情者也。\end{note}
\end{parag}

