\chap{一百零八}{强欢笑蘅芜庆生辰 死缠绵潇湘闻鬼哭}



\begin{parag}
    却说贾政先前曾将房产并大观园奏请入官,内廷不收,又无人居住,只好封锁。因园子接连尤氏惜春住宅,太觉旷阔无人,遂将包勇罚看荒园。此时贾政理家,又奉了贾母之命将人口渐次减少,诸凡省俭,尚且不能支持。幸喜凤姐为贾母疼惜,王夫人等虽则不大喜欢,若说治家办事尚能出力,所以将内事仍交凤姐办理。但近来因被抄以后,诸事运用不来,也是每形拮据。那些房头上下人等原是宽裕惯的,如今较之往日,十去其七,怎能周到,不免怨言不绝。凤姐也不敢推迟,扶病承欢贾母。过了些时,贾赦贾珍各到当差地方,恃有用度,暂且自安,写书回家,都言安逸,家中不必挂念。于是贾母放心,邢夫人尤氏也略略宽怀。
\end{parag}


\begin{parag}
    一日,史湘云出嫁回门,来贾母这边请安。贾母提起他女婿甚好,史湘云也将那里过日平安的话说了,请老太太放心。又提起黛玉去世,不免大家泪落。贾母又想起迎春苦楚,越觉悲伤起来。史湘云劝解一回,又到各家请安问好毕,仍到贾母房中安歇,言及”薛家这样人家被薛大哥闹的家破人亡。今年虽是缓决人犯,明年不知可能减等?”贾母道:“你还不知道呢,昨儿蟠儿媳妇死的不明白,几乎又闹出一场大事来。还幸亏老佛爷有眼,叫他带来的丫头自己供出来了,那夏奶奶才没的闹了,自家拦住相验。你姨妈这里才将皮裹肉的打发出去了。你说说,真真是六亲同运!薛家是这样了,姨太太守着薛蝌过日,为这孩子有良心他说哥哥在监里尚未结局,不肯娶亲。你邢妹妹在大太太那边也就很苦。琴姑娘为他公公死了尚未满服,梅家尚未娶去。二太太的娘家舅太爷一死,凤丫头的哥哥也不成人,那二舅太爷也是个小气的,又是官项不清,也是打饥荒。甄家自从抄家以后别无信息。”湘云道:“三姐姐去了曾有书字回家么?”贾母道:“自从嫁了去,二老爷回来说,你三姐姐在海疆甚好。只是没有书信,我也日夜惦记,为着我们家连连的出些不好事,所以我也顾不来。如今四丫头也没有给他提亲。环儿呢,谁有功夫提起他来。如今我们家的日子比你从前在这里的时侯更苦些。只可怜你宝姐姐,自过了门,没过一天安逸日子。你二哥哥还是这样疯疯颠颠,这怎么处呢!”湘云道:“我从小儿在这里长大的,这里那些人的脾气我都知道的。这一回来了,竟都改了样子了。我打量我隔了好些时没来,他们生疏我。我细想起来,竟不是的,就是见了我,瞧他们的意思原要象先前一样的热闹,不知道怎么,说说就伤心起来了。我所以坐坐就到老太太这里来了。”贾母道:“如今这样日子在我也罢了,你们年轻轻儿的人还了得!我正要想个法儿叫他们还热闹一天才好,只是打不起这个精神来。”湘云道:“我想起来了,宝姐姐不是后儿的生日吗,我多住一天,给他拜过寿,大家热闹一天。不知老太太怎么样?”贾母道:“我真正气糊涂了。你不提我竟忘了,后日可不是他的生日!我明日拿出钱来,给他办个生日。他没有定亲的时侯倒做过好几次,如今他过了门,倒没有做。宝玉这孩子头里很伶俐很淘气,如今为着家里的事不好,把这孩子越发弄的话都没有了。倒是珠儿媳妇还好,他有的时侯是这么着,没的时侯他也是这么着,带着兰儿静静儿的过日子,倒难为他。”湘云道:“别人还不离,独有琏二嫂子连模样儿都改了,说话也不伶俐了。明日等我来引导他们,看他们怎么样。但是他们嘴里不说,心里要抱怨我,说我有了——”湘云说到那里,却把脸飞红了。贾母会意,道:“这怕什么。原来姊妹们都是在一处乐惯了的,说说笑笑,再别要留这些心。大凡一个人,有也罢没也罢,总要受得富贵耐得贫贱才好。你宝姐姐生来是个大方的人,头里他家这样好,他也一点儿不骄傲,后来他家坏了事,他也是舒舒坦坦的。如今在我家里,宝玉待他好,他也是那样安顿,一时待他不好,不见他有什么烦恼。我看这孩子倒是个有福气的。你林姐姐那是个最小性儿又多心的,所以到底不长命。凤丫头也见过些事,很不该略见些风波就改了样子,他若这样没见识,也就是小器了。后儿宝丫头的生日,我替另拿出银子来,热热闹闹给他做个生日,也叫他欢喜这一天。”湘云答应道:“老太太说得很是。索性把那些姐妹们都请来了,大家叙一叙。”贾母道:“自然要请的。”一时高兴道:“叫鸳鸯拿出一百银子来交给外头,叫他明日起预备两天的酒饭。”鸳鸯领命,叫婆子交了出去。一宿无话。次日传话出去,打发人去接迎春,又请了薛姨妈宝琴,叫带了香菱来。又请李婶娘。不多半日,李纹李绮都来了。宝钗本没有知道,听见老太太的丫头来请,说:“薛姨太太来了,请二奶奶过去呢。”宝钗心里喜欢,便是随身衣服过去,要见他母亲。只见他妹子宝琴并香菱都在这里,又见李婶娘等人也都来了。心想:“那些人必是知道我们家的事情完了,所以来问侯的。”便去问了李婶娘好,见了贾母,然后与他母亲说了几句话,便与李家姐妹们问好。湘云在旁说道:“太太们请都坐下,让我们姐妹们给姐姐拜寿。”宝钗听了倒呆了一呆,回来一想:“可不是明日是我的生日吗!”便说:“妹妹们过来瞧老太太是该的,若说为我的生日,是断断不敢的。”正推让着,宝玉也来请薛姨妈李婶娘的安。听见宝钗自己推让,他心里本早打算过宝钗生日,因家中闹得七颠八倒,也不敢在贾母处提起,今见湘云等众人要拜寿,便喜欢道:“明日才是生日,我正要告诉老太太来。”湘云笑道:“扯臊,老太太还等你告诉。你打量这些人为什么来?是老太太请的!”宝钗听了,心下未信。只听贾母合他母亲道:“可怜宝丫头做了一年新媳妇,家里接二连三的有事,总没有给他做过生日。今日我给他做个生日,请姨太太,太太们来大家说说话儿。”薛姨妈道:“老太太这些时心里才安,他小人儿家还没有孝敬老太太,倒要老太太操心。”湘云道:“老太太最疼的孙子是二哥哥,难道二嫂子就不疼了么!况且宝姐姐也配老太太给他做生日。”宝钗低头不语。宝玉心里想道:“我只说史妹妹出了阁是换了一个人了,我所以不敢亲近他,他也不来理我。如今听他的话,原是和先前一样的。为什么我们那个过了门更觉得腼腆了,话都说不出来了呢?”正想着,小丫头进来说:“二姑奶奶回来了。”随后李纨凤姐都进来,大家厮见一番。迎春提起他父亲出门,说:“本要赶来见见,只是他拦着不许来,说是咱们家正是晦气时侯,不要沾染在身上。我扭不过,没有来,直哭了两三天。”凤姐道:“今儿为什么肯放你回来?”迎春道:“他又说咱们家二老爷又袭了职,还可以走走,不妨事的,所以才放我来。”说着,又哭起来。贾母道:“我原为气得慌,今日接你们来给孙子媳妇过生日,说说笑笑解个闷儿。你们又提起这些烦事来,又招起我的烦恼来了。”迎春等都不敢作声了。凤姐虽勉强说了几句有兴的话,终不似先前爽利,招人发笑。贾母心里要宝钗喜欢,故意的呕凤姐儿说话。凤姐也知贾母之意,便竭力张罗,说道:“今儿老太太喜欢些了。你看这些人好几时没有聚在一处,今儿齐全。”说着回过头去,看见婆婆尤氏不在这里,又缩住了口。贾母为着“齐全”两字,也想邢夫人等,叫人请去。邢夫人,尤氏惜春等听见老太太叫,不敢不来,心内也十分不愿意,想着家业零败,偏又高兴给宝钗做生日,到底老太太偏心,便来了也是无精打采的。贾母问起岫烟来,邢夫人假说病着不来。贾母会意,知薛姨妈在这里有些不便,也不提了。
\end{parag}


\begin{parag}
    一时摆下果酒。贾母说:“也不送到外头,今日只许咱们娘儿们乐一乐。”宝玉虽然娶过亲的人,因贾母疼爱,仍在里头打混,但不与湘云宝琴等同席,便在贾母身旁设着一个坐儿,他代宝钗轮流敬酒。贾母道:“如今且坐下大家喝酒,到挨晚儿再到各处行礼去。若如今行起来了,大家又闹规矩,把我的兴头打回去就没趣了。”宝钗便依言坐下。贾母又叫人来道:“咱们今儿索性洒脱些,各留一两个人伺侯。我叫鸳鸯带了彩云,莺儿,袭人,平儿等在后间去,也喝一钟酒。”鸳鸯等说:“我们还没有给二奶奶磕头,怎么就好喝酒去呢。”贾母道:“我说了,你们只管去,用的着你们再来。”鸳鸯等去了。这里贾母才让薛姨妈等喝酒,见他们都不是往常的样子,贾母着急道:“你们到底是怎么着?大家高兴些才好。”湘云道:“我们又吃又喝,还要怎样!”凤姐道:“他们小的时侯儿都高兴,如今都碍着脸不敢混说,所以老太太瞧着冷净了。”
\end{parag}


\begin{parag}
    宝玉轻轻的告诉贾母道:“话是没有什么说的,再说就说到不好的上头来了。不如老太太出个主意,叫他们行个令儿罢。”贾母侧着耳朵听了,笑道:“若是行令,又得叫鸳鸯去。”宝玉听了,不待再说,就出席到后间去找鸳鸯,说:“老太太要行令,叫姐姐去呢。”鸳鸯道:“小爷,让我们舒舒服服的喝一杯罢,何苦来又来搅什么。”宝玉道:“当真老太太说,得叫你去呢,与我什么相干。”鸳鸯没法,说道:“你们只管喝,我去了就来。”便到贾母那边。老太太道:“你来了,不是要行令吗。”鸳鸯道:“听见宝二爷说老太太叫,我敢不来吗。不知老太太要行什么令儿?”贾母道:“那文的怪闷的慌,武的又不好,你倒是想个新鲜顽意儿才好。”鸳鸯想了想道:“如今姨太太有了年纪,不肯费心,倒不如拿出令盘骰子来,大家掷个曲牌名儿赌输赢酒罢。”贾母道:“这也使得。”便命人取骰盆放在桌上。鸳鸯说:“如今用四个骰子掷去,掷不出名儿来的罚一杯,掷出名儿来,每人喝酒的杯数儿掷出来再定。”众人听了道:“这是容易的,我们都随着。”鸳鸯便打点儿。众人叫鸳鸯喝了一杯,就在他身上数起,恰是薛姨妈先掷。薛姨妈便掷了一下,却是四个么。鸳鸯道:“这是有名的,叫做‘商山四皓’。有年纪的喝一杯。”于是贾母,李婶娘,邢王二夫人都该喝。贾母举酒要喝,鸳鸯道:“这是姨太太掷的,还该姨太太说个曲牌名儿,下家儿接一句《千家诗》。说不出的罚一杯。”薛姨妈道:“你又来算计我了,我那里说得上来。”贾母道:“不说到底寂寞,还是说一句的好。下家儿就是我了,若说不出来,我陪姨太太喝一钟就是了。”薛姨妈便道:“我说个‘临老入花丛’。”贾母点点头儿道:“将谓偷闲学少年。”说完,骰盆过到李纹,便掷了两个四两个二。鸳鸯说:“也有名了,这叫作‘刘阮入天台’。”李纹便接着说了个”二士入桃源。”下手儿便是李纨,说道:“寻得桃源好避秦。”大家又喝了一口。骰盆又过到贾母跟前,便掷了两个二两个三。贾母道:“这要喝酒了?”鸳鸯道:“有名儿的,这是‘江燕引雏’。众人都该喝一杯。”凤姐道:“雏是雏,倒飞了好些了。”众人瞅了他一眼,凤姐便不言语。贾母道:“我说什么呢,‘公领孙’罢。”下手是李绮,便说道:“闲看儿童捉柳花。”众人都说好。宝玉巴不得要说,只是令盆轮不到,正想着,恰好到了跟前,便掷了一个二两个三一个么,便说道:“这是什么?”鸳鸯笑道:“这是个‘臭’,先喝一杯再掷罢。”宝玉只得喝了又掷,这一掷掷了两个三两个四,鸳鸯道:“有了,这叫做‘张敞画眉’。”宝玉明白打趣他,宝钗的脸也飞红了。凤姐不大懂得,还说:“二兄弟快说了,再找下家儿是谁。”宝玉明知难说,自认“罚了罢,我也没下家。”过了令盆轮到李纨,便掷了一下儿。鸳鸯道:“大奶奶掷的是‘十二金钗’。”宝玉听了,赶到李纨身旁看时,只见红绿对开,便说:“这一个好看得很。”忽然想起十二钗的梦来,便呆呆的退到自己座上,心里想,”这十二钗说是金陵的,怎么家里这些人如今七大八小的就剩了这几个。”复又看看湘云宝钗,虽说都在,只是不见了黛玉,一时按捺不住,眼泪便要下来。恐人看见,便说身上躁的很,脱脱衣服去,挂了筹出席去了。这史湘云看见宝玉这般光景,打量宝玉掷不出好的,被别人掷了去,心里不喜欢,便去了,又嫌那个令儿没趣,便有些烦。只见李纨道:“我不说了,席间的人也不齐,不如罚我一杯。”贾母道:“这个令儿也不热闹,不如蠲了罢。让鸳鸯掷一下,看掷出个什么来。”小丫头便把令盆放在鸳鸯跟前。鸳鸯依命便掷了两个二一个五,那一个骰子在盆中只管转,鸳鸯叫道:“不要五!”那骰子单单转出一个五来。鸳鸯道:“了不得!我输了。”贾母道:“这是不算什么的吗?”鸳鸯道:“名儿倒有,只是我说不上曲牌名来。”贾母道:“你说名儿,我给你诌。”鸳鸯道:“这是浪扫浮萍。”贾母道:“这也不难,我替你说个‘秋鱼入菱窠’。”鸳鸯下手的就是湘云,便道:“白萍吟尽楚江秋。”众人都道:“这句很确。”贾母道:“这令完了。咱们喝两杯吃饭罢。”回头一看,见宝玉还没进来,便问道:“宝玉那里去了,还不来?”鸳鸯道:“换衣服去了。”贾母道:“谁跟了去的?”那莺儿便上来回道:“我看见二爷出去,我叫袭人姐姐跟了去了。”贾母王夫人才放心。
\end{parag}


\begin{parag}
    等了一回,王夫人叫人去找来。小丫头子到了新房,只见五儿在那里插蜡。小丫头便问:“宝二爷那里去了?”五儿道:“在老太太那边喝酒呢。”小丫头道:“我在老太太那里,太太叫我来找的。岂有在那里倒叫我来找的理。”五儿道:“这就不知道了,你到别处找去罢。”小丫头没法,只得回来,遇见秋纹,便道:“你见二爷那里去了?”秋纹道:“我也找他。太太们等他吃饭,这会子那里去了呢?你快去回老太太去,不必说不在家,只说喝了酒不大受用不吃饭了,略躺一躺再来,请老太太们吃饭罢。”小丫头依言回去告诉珍珠,珍珠依言回了贾母。贾母道:“他本来吃不多,不吃也罢了。叫他歇歇罢。告诉他今儿不必过来,有他媳妇在这里。”珍珠便向小丫头道:“你听见了?”小丫头答应着,不便说明,只得在别处转了一转,说告诉了。众人也不理会,便吃毕饭,大家散坐说话。不题。
\end{parag}


\begin{parag}
    且说宝玉一时伤心,走了出来,正无主意,只见袭人赶来,问是怎么了。宝玉道:“不怎么,只是心里烦得慌。何不趁他们喝酒咱们两个到珍大奶奶那里逛逛去。”袭人道:“珍大奶奶在这里,去找谁?”宝玉道:“不找谁,瞧瞧他现在这里住的房屋怎么样。”袭人只得跟着,一面走,一面说。走到尤氏那边,又一个小门儿半开半掩,宝玉也不进去。只见看园门的两个婆子坐在门坎上说话儿。宝玉问道:“这小门开着么?”婆子道:“天天是不开的。今儿有人出来说,今日预备老太太要用园里的果子,故开着门等着。”宝玉便慢慢的走到那边,果见腰门半开,宝玉便走了进去。袭人忙拉住道:“不用去,园里不干净,常没有人去,不要撞见什么。”宝玉仗着酒气,说:“我不怕那些。”袭人苦苦的拉住不容他去。婆子们上来说道:“如今这园子安静的了。自从那日道士拿了妖去,我们摘花儿,打果子一个人常走的。二爷要去,咱们都跟着,有这些人怕什么。”宝玉喜欢,袭人也不便相强,只得跟着。
\end{parag}


\begin{parag}
    宝玉进得园来,只见满目凄凉,那些花木枯萎,更有几处亭馆,彩色久经剥落,远远望见一丛修竹,倒还茂盛。宝玉一想,说:“我自病时出园住在后边,一连几个月不准我到这里,瞬息荒凉。你看独有那几杆翠竹菁葱,这不是潇湘馆么!”袭人道:“你几个月没来,连方向都忘了。咱们只管说话,不觉将怡红院走过了。”回过头来用手指着道:“这才是潇湘馆呢。”宝玉顺着袭人的手一瞧,道:“可不是过了吗!咱们回去瞧瞧。”袭人道:“天晚了,老太太必是等着吃饭,该回去了。”宝玉不言,找着旧路,竟往前走。
\end{parag}


\begin{parag}
    你道宝玉虽离了大观园将及一载,岂遂忘了路径?只因袭人恐他见了潇湘馆,想起黛玉又要伤心,所以用言混过。岂知宝玉只望里走,天又晚,恐招了邪气,故宝玉问他,只说已走过了,欲宝玉不去。不料宝玉的心惟在潇湘馆内。袭人见他往前急走,只得赶上,见宝玉站着,似有所见,如有所闻,便道:“你听什么?”宝玉道:“潇湘馆倒有人住着么?”袭人道:“大约没有人罢。”宝玉道:“我明明听见有人在内啼哭,怎么没有人!”袭人道:“你是疑心。素常你到这里,常听见林姑娘伤心,所以如今还是那样。”宝玉不信,还要听去。婆子们赶上说道:“二爷快回去罢。天已晚了,别处我们还敢走走,只是这里路又隐僻,又听得人说这里林姑娘死后常听见有哭声,所以人都不敢走的。”宝玉袭人听说,都吃了一惊。宝玉道:“可不是。”说着,便滴下泪来,说:“林妹妹,林妹妹,好好儿的是我害了你了!你别怨我,只是父母作主,并不是我负心。”愈说愈痛,便大哭起来。袭人正在没法,只见秋纹带着些人赶来对袭人道:“你好大胆,怎么领了二爷到这里来!老太太,太太他们打发人各处都找到了,刚才腰门上有人说是你同二爷到这里来了,唬得老太太,太太们了不得,骂着我,叫我带人赶来,还不快回去么!”宝玉犹自痛哭。袭人也不顾他哭,两个人拉着就走,一面替他拭眼泪,告诉他老太太着急。宝玉没法,只得回来。
\end{parag}


\begin{parag}
    袭人知老太太不放心,将宝玉仍送到贾母那边。众人都等着未散。贾母便说:“袭人,我素常知你明白,才把宝玉交给你,怎么今儿带他园里去!他的病才好,倘或撞着什么,又闹起来,这便怎么处?”袭人也不敢分辩,只得低头不语。宝钗看宝玉颜色不好,心里着实的吃惊。倒还是宝玉恐袭人受委屈,说道:“青天白日怕什么。我因为好些时没到园里逛逛,今儿趁着酒兴走走。那里就撞着什么了呢!”凤姐在园里吃过大亏的,听到那里寒毛倒竖,说:“宝兄弟胆子忒大了。”湘云道:“不是胆大,倒是心实。不知是会芙蓉神去了,还是寻什么仙去了。”宝玉听着,也不答言。独有王夫人急的一言不发。贾母问道:“你到园里可曾唬着么?这回不用说了,以后要逛,到底多带几个人才好。不然大家早散了。回去好好的睡一夜,明日一早过来,我还要找补,叫你们再乐一天呢。不要为他又闹出什么原故来。”众人听说,辞了贾母出来。薛姨妈便到王夫人那里住下。史湘云仍在贾母房中。迎春便往惜春那里去了。余者各自回去。不题。独有宝玉回到房中,嗳声叹气。宝钗明知其故,也不理他,只是怕他忧闷,勾出旧病来,便进里间叫袭人来细问他宝玉到园怎么的光景。未知袭人怎生回说,下回分解。
\end{parag}