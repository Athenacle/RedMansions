\chap{三十三}{手足耽耽小動脣舌 不肖種種大遭笞撻}


\begin{parag}
    \begin{note}蒙回前總:富貴公子,侯王應襲,容易在紅粉場中作罪。風流情性,詩賦文詞,偏只爲鶯花路間留滯。笑嘻嘻,哭啼啼,總是一般情事。\end{note}
\end{parag}


\begin{parag}
    卻說王夫人喚他母親上來,拿幾件簪環當面賞與,又吩咐請幾衆僧人唸經超度。他母親磕頭謝了出去。
\end{parag}


\begin{parag}
    原來寶玉會過雨村回來聽見了,便知金釧兒含羞賭氣自盡,心中早又五內摧傷,進來被王夫人數落教訓,也無可回說。見寶釵進來,方得便出來,茫然不知何 往,背著手,低頭一面感嘆,一面慢慢的走著,信步來至廳上。剛轉過屏門,不想對面來了一人正往裏走,可巧兒撞了個滿懷。只聽那人喝了一聲“站住!”寶玉唬了一跳,抬頭一看,不是別人,卻是他父親,不覺的倒抽了一口氣,只得垂手一旁站了。賈政道:“好端端的,你垂頭喪氣嗐些什麼?方纔雨村來了要見你,叫你那半天你纔出來;既出來了,全無一點慷慨揮灑談吐,仍是葳葳蕤蕤。我看你臉上一團思欲愁悶氣色,這會子又咳聲嘆氣。你那些還不足,還不自在?無故這樣,卻是爲何?”寶玉素日雖是口角伶俐,只是此時一心總爲金釧兒感傷,恨不得此時也身亡命殞,\begin{note}蒙側:真有此情,真有此理。\end{note}跟了金釧兒去。如今見了他父親說這些話,究竟不曾聽見,只是怔呵呵的站著。
\end{parag}


\begin{parag}
    賈政見他惶悚,應對不似往日,原本無氣的,這一來倒生了三分氣。方欲說話,忽有回事人來回:“忠順親王府裏有人來,要見老爺。”賈政聽了,心下疑惑, 暗暗思忖道:“素日並不和忠順府來往,爲什麼今日打發人來?”一面想,一面令“快請”,急走出來看時,卻是忠順府長史官,忙接進廳上坐了獻茶。未及敘談, 那長史官先就說道:“下官此來,並非擅造潭府,皆因奉王命而來,有一件事相求。看王爺面上,敢煩老大人作主,不但王爺知情,且連下官輩亦感謝不盡。”賈政 聽了這話,抓不住頭腦,忙陪笑起身問道:“大人既奉王命而來,不知有何見諭,望大人宣明,學生好遵諭承辦。”那長史官便冷笑道:“也不必承辦,只用大人一句話就完了。我們府裏有一個做小旦的琪官,一向好好在府裏,如今竟三五日不見回去,各處去找,又摸不著他的道路,因此各處訪察。這一城內,十停人倒有八停人都說,他近日和銜玉的那位令郎相與甚厚。下官輩等聽了,尊府不比別家,可以擅入索取,因此啓明王爺。王爺亦云:‘若是別的戲子呢,一百個也罷了;只是這 琪官隨機應答,謹慎老誠,甚合我老人家的心,竟斷斷少不得此人。’故此求老大人轉諭令郎,請將琪官放回,一則可慰王爺諄諄奉懇,二則下官輩也可免操勞求覓之苦。”說畢,忙打一躬。
\end{parag}


\begin{parag}
    賈政聽了這話,又驚又氣,即命喚寶玉來。寶玉也不知是何原故,忙趕來時,賈政便問:“該死的奴才!你在家不讀書也罷了,怎麼又做出這些無法無天的事來!那琪官現是忠順王爺駕前承奉的人,你是何等草芥,無故引逗他出來,如今禍及於我。”寶玉聽了唬了一跳,忙回道:“實在不知此事。究竟連‘琪官’兩個字不知爲何物,豈更又加‘引逗’二字!”說著便哭了。賈政未及開言,只見那長史官冷笑道:“公子也不必掩飾。或隱藏在家,或知其下落,早說了出來,我們也少受些辛苦,豈不念公子之德?”寶玉連說不知,“恐是訛傳,也未見得。”那長史官冷笑道:“現有據證,何必還賴?必定當著老大人說了出來,公子豈不喫虧?既雲不知此人,那紅汗巾子怎麼到了公子腰裏?”寶玉聽了這話,不覺轟去魂魄,目瞪口呆,心下自思:“這話他如何得知!他既連這樣機密事都知道了,大約別的瞞他不過,不如打發他去了,免的再說出別的事來。”因說道:“大人既知他的底細,如何連他置買房舍這樣大事倒不曉得了?聽得說他如今在東郊離城二十里有個什 麼紫檀堡,他在那裏置了幾畝田地幾間房舍。想是在那裏也未可知。”那長史官聽了,笑道:“這樣說,一定是在那裏。我且去找一回,若有了便罷,若沒有,還要 來請教。”\begin{note}蒙側:寶玉其人,愛之有餘,豈可撻之者?用此等文章逼之,能不使人肝膽憤烈以成下文之嚴酷耶?\end{note}說著,便忙忙的走了。
\end{parag}


\begin{parag}
    賈政此時氣的目瞪口歪,一面送那長史官,一面回頭命寶玉“不許動!回來有話問你!”一直送那官員去了。纔回身,忽見賈環帶著幾個小廝一陣亂跑。賈政喝令小廝“快打,快打!”賈環見了他父親,唬的骨軟筋酥,忙低頭站住。賈政便問:“你跑什麼?帶著你的那些人都不管你,不知往那裏逛去,由你野馬一般!”喝令叫跟上學的人來。賈環見他父親盛怒,便乘機說道:“方纔原不曾跑,只因從那井邊一過,那井裏淹死了一個丫頭,我看見人頭這樣大,身子這樣粗,泡的實在可 怕,所以才趕著跑了過來。”賈政聽了驚疑,問道:“好端端的,誰去跳井?我家從無這樣事情,自祖宗以來,皆是寬柔以待下人。──大約我近年於家務疏懶,自然執事人操克奪之權,致使生出這暴殄輕生的禍患。若外人知道,祖宗顏面何在!”喝令快叫賈璉、賴大、來興。小廝們答應了一聲,方欲叫去,賈環忙上前拉住賈政的袍襟,貼膝跪下道:“父親不用生氣。此事除太太房裏的人,別人一點也不知道。我聽見我母親說……”說到這裏,便回頭四顧一看。賈政知意,將眼一看衆小 廝,小廝們明白,都往兩邊後面退去。賈環便悄悄說道:“我母親告訴我說,寶玉哥哥前日在太太屋裏,拉著太太的丫頭金釧兒強姦不遂,\begin{note}蒙側:再逼下文,有 不得不盡情苦打之勢。\end{note}打了一頓。那金釧兒便賭氣投井死了。”話未說完,把個賈政氣的面如金紙,大喝:“快拿寶玉來!”一面說,一面便往裏邊書房裏去,喝 令:“今日再有人勸我,我把這冠帶傢俬一應交與他與寶玉過去!我免不得做個罪人,把這幾根煩惱鬢毛剃去,尋個乾淨去處自了,也免得上辱先人下生逆子之罪。\begin{note}蒙側:一激再激,實文實事。\end{note}”衆門客僕從見賈政這個形景,便知又是爲寶玉了,一個個都是啖指咬舌,連忙退出。那賈政喘吁吁直挺挺坐在椅子上,滿面淚痕,\begin{note}蒙側:爲天下父母一哭。\end{note}一疊聲“拿寶玉!拿大棍!拿索子捆上!把各門都關上!有人傳信往裏頭去,立刻打死!”衆小廝們只得齊聲答應,有幾個來找 寶玉。
\end{parag}


\begin{parag}
    那寶玉聽見賈政吩咐他“不許動”,早知多凶少吉,那裏承望賈環又添了許多的話。正在廳上幹轉,怎得個人來往裏頭去捎信,偏生沒個人,連焙茗也不知在那裏。正盼望時,只見一個老姆姆出來。寶玉如得了珍寶,便趕上來拉他,說道:“快進去告訴:老爺要打我呢!快去,快去!要緊,要緊!”寶玉一則急了,說話不明白;二則老婆子偏生又聾,竟不曾聽見是什麼話,把“要緊”二字只聽作“跳井”二字,便笑道:“跳井讓他跳去,二爺怕什麼?”寶玉見是個聾子,便著急道: “你出去叫我的小廝來罷。”那婆子道:“有什麼不了的事?老早的完了。太太又賞了衣服,又賞了銀子,怎麼不了事的!”\begin{note}蒙側:寫老婆子處,說“無要緊的”,真如見其人,如聞其聾。\end{note}
\end{parag}


\begin{parag}
    寶玉急的跺腳,正沒抓尋處,只見賈政的小廝走來,逼著他出去了。賈政一見,眼都紅紫了,也不暇問他在外流蕩優伶,表贈私物,在家荒疏學業,淫辱母婢等語,只喝令:“堵起嘴來,著實打死!”小廝們不敢違拗,只得將寶玉按在凳上,舉起大板打了十來下。賈政猶嫌打輕了,一腳踢開掌板的,自己奪過來,咬著牙狠 命蓋了三四十下。衆門客見打的不祥了,忙上前奪勸。賈政那裏肯聽,說道:“你們問問他乾的勾當可饒不可饒!素日皆是你們這些人把他釀壞了,到這步田地還來 解勸。明日釀到他弒君殺父,你們纔不勸不成!”
\end{parag}


\begin{parag}
    衆人聽這話不好聽,知道氣急了,忙又退出,只得覓人進去給信。王夫人不敢先回賈母,只得忙穿衣出來,也不顧有人沒人,忙忙趕往書房中來,\begin{note}蒙側:爲 天下慈母一哭。\end{note}慌的衆門客小廝等避之不及。王夫人一進房來,賈政更如火上澆油一般,那板子越發下去的又狠又快。按寶玉的兩個小廝忙鬆了手走開,寶玉早已 動彈不得了。賈政還欲打時,早被王夫人抱住板子。賈政道:“罷了,罷了!今日必定要氣死我才罷!”王夫人哭道:“寶玉雖然該打,老爺也要自重。況且炎天暑 日的,老太太身上也不大好,打死寶玉事小,倘或老太太一時不自在了,豈不事大!\begin{note}蒙側:父母之心,昊天罔極。賈政王夫人異地則皆然。\end{note}” 賈政冷笑道:“倒休提這話。我養了這不肖的孽障,已經不孝;教訓他一番,又有衆人護持;不如趁今日一發勒死了,以絕將來之患!”說著,便要繩索來勒死。王夫人連忙抱住哭道:“老爺雖然應當管教兒子,也要看夫妻分上。我如今已將五十歲的人,只有這個孽障,必定苦苦的以他爲法,我也不敢深勸。今日越發要他死, 豈不是有意絕我。既要勒死他,快拿繩子來先勒死我,再勒死他。我們娘兒們不敢含怨,到底在陰司裏得個依靠。\begin{note}蒙側:使人讀之聲哽咽而淚如雨下。\end{note}\begin{note}蒙雙夾:未喪母者來細玩,即喪母者來痛哭。\end{note}” 說畢,爬在寶玉身上大哭起來。賈政聽了此話,不覺長嘆一聲,向椅上坐了,淚如雨下。王夫人抱著寶玉,只見他面白氣弱,底下穿著一條綠紗小衣皆是血漬。禁不住解下汗巾看,由臀至脛,或青或紫,或整或破,竟無一點好處,不覺失聲大哭起來,“苦命的兒嚇!”因哭出“苦命兒”來,忽又想起賈珠來,便叫著賈珠哭道: “若有你活著,便死一百個我也不管了。”此時裏面的人聞得王夫人出來,那李宮裁王熙鳳與迎春姊妹早已出來了。王夫人哭著賈珠的名字,\begin{note}蒙側:慈母如畫。\end{note}別人還可,惟有宮裁禁不住也放聲哭了。賈政聽了,那淚珠更似滾瓜一般滾了下來。
\end{parag}


\begin{parag}
    正沒開交處,忽聽丫鬟來說:“老太太來了。”一句話未了,只聽窗外顫巍巍的聲氣說道:\begin{note}蒙側:老人家神影活現。\end{note}“先打死我,再打死他,豈不乾淨了!”賈政見他母親來了,又急又痛,連忙迎接出來,只見賈母扶著丫頭,喘吁吁的走來。賈政上前躬身陪笑道:“大暑熱天,母親有何生氣親自走來?有話只該叫 了兒子進去吩咐。”賈母聽說,便止住步喘息一回,\begin{note}蒙側:大家規模,一絲不亂。\end{note}厲聲說道:“你原來是和我說話!我倒有話吩咐,只是可憐我一生沒養個好兒子,卻教我和誰說去!”賈政聽這話不象,忙跪下含淚說道:“爲兒的教訓兒子,也爲的是光宗耀祖。母親這話,我做兒的如何禁得起?”賈母聽說,便啐了一口,說道:“我說一句話,你就禁不起,你那樣下死手的板子,難道寶玉就禁得起了?\begin{note}蒙側:如此礙犯文字,隨景生情,毫無牽滯。\end{note}你說教訓兒子是光宗耀 祖,當初你父親怎麼教訓你來!”說著,不覺就滾下淚來。賈政又陪笑道:“母親也不必傷感,皆是作兒的一時性起,從此以後再不打他了。” 賈母便冷笑道:“你也不必和我使性子賭氣的。你的兒子,我也不該管你打不打。我猜著你也厭煩我們娘兒們。不如我們趕早兒離了你,大家乾淨!”說著便令人去看轎馬,“我和你太太寶玉立刻回南京去!”家下人只得幹答應著。賈母又叫王夫人道:“你也不必哭了。如今寶玉年紀小,你疼他,他將來長大成人,爲官作宰的,也未必想著你是他母親了。你如今倒不要疼他,只怕將來還少生一口氣呢。”賈政聽說,忙叩頭哭道:“母親如此說,賈政無立足之地。”賈母冷笑道:“你分明使我無立足之地,你反說起你來!只是我們回去了,你心裏乾淨,看有誰來許你打。”一面說,一面只令快打點行李車轎回去。賈政苦苦叩求認罪。
\end{parag}


\begin{parag}
    賈母一面說話,一面又記掛寶玉,忙進來看時,只見今日這頓打不比往日,又是心疼,又是生氣,也抱著哭個不了。王夫人與鳳姐等解勸了一會,方漸漸的止住。早有丫鬟媳婦等上來,要攙寶玉,鳳姐便罵道:\begin{note}蒙側:能事者自不凡。\end{note}“糊塗東西,也不睜開眼瞧瞧!打的這麼個樣兒,還要攙著走!還不快進去把那藤 屜子春凳擡出來呢。”衆人聽說連忙進去,果然擡出春凳來,將寶玉抬放凳上,隨著賈母王夫人等進去,送至賈母房中。
\end{parag}


\begin{parag}
    彼時賈政見賈母氣未全消,不敢自便,也跟了進去。看看寶玉,果然打重了。再看看王夫人,“兒”一聲,“肉”一聲,“你替珠兒早死了,留著珠兒,免你父親生氣,我也不白操這半世的心了。這會子你倘或有個好歹,丟下我,叫我靠那一個!”數落一場,又哭“不爭氣的兒”。賈政聽了,也就灰心,\begin{note}蒙側:天下作 父兄者教子弟時,亦當留意。\end{note}自悔不該下毒手打到如此地步。先勸賈母,賈母含淚說道:“你不出去,還在這裏做什麼!難道於心不足,還要眼看著他死了纔去不 成!\begin{note}蒙側:遣去有法。\end{note}”賈政聽說,方退了出來。
\end{parag}


\begin{parag}
    此時薛姨媽同寶釵、香菱、襲人、史湘雲也都在這裏。襲人滿心委屈,只不好十分使出來,見衆人圍著,灌水的灌水,打扇的打扇,自己插不下手去,便越性走 出來到二門前,令小廝們找了焙茗來細問:\begin{note}蒙側:各自有各自一番作用。\end{note}“方纔好端端的,爲什麼打起來?你也不早來透個信兒!”焙茗急的說:“偏生我沒在跟前,打到半中間我才聽見了。忙打聽原故,卻是爲琪官金釧姐姐的事。”襲人道:“老爺怎麼得知道的?”焙茗道:“那琪官的事,多半是薛大爺素日喫醋,沒 法兒出氣,不知在外頭唆挑了誰來,在老爺跟前下的火。那金釧兒的事是三爺說的,我也是聽見老爺的人說的。”襲人聽了這兩件事都對景,心中也就信了八九分。 然後回來,只見衆人都替寶玉療治。調停完備,賈母令“好生抬到他房內去”。衆人答應,七手八腳,忙把寶玉送入怡紅院內自己牀上臥好。又亂了半日,衆人漸漸散去,襲人方進前來經心服侍,問他端的。且聽下回分解。
\end{parag}


\begin{parag}
    \begin{note}蒙回末總評:嚴酷其刑以教子,不請中十分用情。牽連不斷以思婢,有恩處一等無恩。嚴父慈母一般愛子,親優溺婢總是乖淫。矇頭花柳,誰解春光。跳出樊籠,一場笑話。\end{note}
\end{parag}

