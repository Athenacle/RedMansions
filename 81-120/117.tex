\chap{一百一十七}{阻超凡佳人雙護玉 欣聚黨惡子獨承家}



\begin{parag}
    話說王夫人打發人來叫寶釵過去商量,寶玉聽見說是和尚在外頭,趕忙的獨自一人走到前頭,嘴裏亂嚷道:“我的師父在那裏?”叫了半天,並不見有和尚,只得走到外面。見李貴將和尚攔住,不放他進來。寶玉便說道:“太太叫我請師父進去。”李貴聽了鬆了手,那和尚便搖搖擺擺的進去。寶玉看見那僧的形狀與他死去時所見的一般,心裏早有些明白了,便上前施禮,連叫:“師父,弟子迎候來遲。”那僧說:“我不要你們接待,只要銀子,拿了來我就走。”寶玉聽來又不象有道行的話,看他滿頭癩瘡,混身腌臢破爛,心裏想道:“自古說‘真人不露相,露相不真人’,也不可當面錯過,我且應了他謝銀,並探探他的口氣。”便說道:“師父不必性急,現在家母料理,請師父坐下略等片刻。弟子請問,師父可是從‘太虛幻境’而來?”那和尚道:“什麼幻境,不過是來處來去處去罷了!我是送還你的玉來的。我且問你,那玉是從那裏來的?”寶玉一時對答不來。那僧笑道:“你自己的來路還不知,便來問我!”寶玉本來穎悟,又經點化,早把紅塵看破,只是自己的底裏未知,一聞那僧問起玉來,好象當頭一棒,便說道:“你也不用銀子了,我把那玉還你罷。”那僧笑道:“也該還我了。”
\end{parag}


\begin{parag}
    寶玉也不答言,往裏就跑,走到自己院內,見寶釵襲人等都到王夫人那裏去了,忙向自己牀邊取了那玉便走出來。迎面碰見了襲人,撞了一個滿懷,把襲人唬了一跳,說道:“太太說,你陪着和尚坐着很好,太太在那裏打算送他些銀兩。你又回來做什麼?”寶玉道:“你快去回太太,說不用張羅銀兩了,我把這玉還了他就是了。”襲人聽說,即忙拉住寶玉道:“這斷使不得的!那玉就是你的命,若是他拿去了,你又要病着了。”寶玉道:“如今不再病的了,我已經有了心了,要那玉何用!”摔脫襲人,便要想走。襲人急得趕着嚷道:“你回來,我告訴你一句話。”寶玉回過頭來道:“沒有什麼說的了。”襲人顧不得什麼,一面趕着跑,一面嚷道:“上回丟了玉,幾乎沒有把我的命要了!剛剛兒的有了,你拿了去,你也活不成,我也活不成了!你要還他,除非是叫我死了!”說着,趕上一把拉住。寶玉急了道:“你死也要還,你不死也要還!”狠命的把襲人一推,抽身要走。怎奈襲人兩隻手繞着寶玉的帶子不放鬆,哭喊着坐在地下。裏面的丫頭聽見連忙趕來,瞧見他兩個人的神情不好,只聽見襲人哭道:“快告訴太太去,寶二爺要把那玉去還和尚呢!”丫頭趕忙飛報王夫人。那寶玉更加生氣,用手來掰開了襲人的手,幸虧襲人忍痛不放。紫鵑在屋裏聽見寶玉要把玉給人,這一急比別人更甚,把素日冷淡寶玉的主意都忘在九霄雲外了,連忙跑出來幫着抱住寶玉。那寶玉雖是個男人,用力摔打,怎奈兩個人死命的抱住不放,也難脫身,嘆口氣道:“爲一塊玉這樣死命的不放,若是我一個人走了,又待怎麼樣呢?”襲人紫鵑聽到那裏,不禁嚎啕大哭起來。正在難分難解,王夫人寶釵急忙趕來,見是這樣形景,便哭着喝道:“寶玉,你又瘋了嗎!”寶玉見王夫人來了,明知不能脫身,只得陪笑說道:“這當什麼,又叫太太着急。他們總是這樣大驚小怪的,我說那和尚不近人情,他必要一萬銀子,少一個不能。我生氣進來拿這玉還他,就說是假的,要這玉幹什麼。他見得我們不希罕那玉,便隨意給他些就過去了。”王夫人道:“我打諒真要還他,這也罷了。爲什麼不告訴明白了他們,叫他們哭哭喊喊的象什麼。”寶釵道:“這麼說呢倒還使得。要是真拿那玉給他,那和尚有些古怪,倘或一給了他,又鬧到家口不寧,豈不是不成事了麼?至於銀錢呢,就把我的頭面折變了,也還夠了呢。”王夫人聽了道:“也罷了,且就這麼辦罷。”寶玉也不回答。只見寶釵走上來在寶玉手裏拿了這玉,說道:“你也不用出去,我合太太給他錢就是了。”寶玉道:“玉不還他也使得,只是我還得當面見他一見纔好。”襲人等仍不肯放手,到底寶釵明決,說:“放了手由他去就是了。”襲人只得放手。寶玉笑道:“你們這些人原來重玉不重人哪。你們既放了我,我便跟着他走了,看你們就守着那塊玉怎麼樣!”襲人心裏又着急起來,仍要拉他,只礙着王夫人和寶釵的面前,又不好太露輕薄。恰好寶玉一撒手就走了。襲人忙叫小丫頭在三門口傳了焙茗等,”告訴外頭照應着二爺,他有些瘋了。”小丫頭答應了出去。
\end{parag}


\begin{parag}
    王夫人寶釵等進來坐下,問起襲人來由,襲人便將寶玉的話細細說了。王夫人寶釵甚是不放心,又叫人出去吩咐衆人伺候,聽着和尚說些什麼。回來小丫頭傳話進來回王夫人道:“二爺真有些瘋了。外頭小廝們說,裏頭不給他玉,他也沒法,如今身子出來了,求着那和尚帶了他去。”王夫人聽了說道:“這還了得!那和尚說什麼來着?”小丫頭回道:“和尚說要玉不要人。”寶釵道:“不要銀子了麼?”小丫頭道:“沒聽見說,後來和尚和二爺兩個人說着笑着,有好些話外頭小廝們都不大懂。”王夫人道:“糊塗東西,聽不出來,學是自然學得來的。”便叫小丫頭:“你把那小廝叫進來。”小丫頭連忙出去叫進那小廝,站在廊下,隔着窗戶請了安。王夫人便問道:“和尚和二爺的話你們不懂,難道學也學不來嗎?”那小廝回道:“我們只聽見說什麼‘大荒山’,什麼‘青埂峯’,又說什麼‘太虛境’,‘斬斷塵緣’這些話。”王夫人聽了也不懂。寶釵聽了,唬得兩眼直瞪,半句話都沒有了。正要叫人出去拉寶玉進來,只見寶玉笑嘻嘻的進來說:“好了,好了。”寶釵仍是發怔。王夫人道:“你瘋瘋顛顛的說的是什麼?”寶玉道:“正經話又說我瘋顛。那和尚與我原是認得的,他不過也是要來見我一見。他何嘗是真要銀子呢,也只當化個善緣就是了。所以說明了他自己就飄然而去了。這可不是好了麼!”王夫人不信,又隔着窗戶問那小廝。那小廝連忙出去問了門上的人,進來回說:“果然和尚走了。說請太太們放心,我原不要銀子,只要寶二爺時常到他那裏去去就是了。諸事只要隨緣,自有一定的道理。”王夫人道:“原來是個好和尚,你們曾問住在那裏?”門上道:“奴才也問來着,他說我們二爺是知道的。”王夫人問寶玉道:“他到底住在那裏?”寶玉笑道:“這個地方說遠就遠,說近就近。”寶釵不待說完,便道:“你醒醒兒罷,別盡着迷在裏頭。現在老爺太太就疼你一個人,老爺還吩咐叫你幹功名長進呢。”寶玉道:“我說的不是功名麼!你們不知道,‘一子出家,七祖昇天’呢。”王夫人聽到那裏,不覺傷心起來,說:“我們的家運怎麼好,一個四丫頭口口聲聲要出家,如今又添出一個來了。我這樣個日子過他做什麼!”說着,大哭起來。寶釵見王夫人傷心,只得上前苦勸。寶玉笑道:“我說了這一句頑話,太太又認起真來了。”王夫人止住哭聲道:“這些話也是混說的麼!”正鬧着,只見丫頭來回話:“璉二爺回來了,顏色大變,說請太太回去說話。”王夫人又吃了一驚,說道:“將就些,叫他進來罷,小嬸子也是舊親,不用迴避了。”賈璉進來,見了王夫人請了安。寶釵迎着也問了賈璉的安。回說道:“剛纔接了我父親的書信,說是病重的很,叫我就去,若遲了恐怕不能見面。”說到那裏,眼淚便掉下來了。王夫人道:“書上寫的是什麼病?”賈璉道:“寫的是感冒風寒起來的,如今成了癆病了。現在危急,專差一個人連日連夜趕來的,說如若再耽擱一兩天就不能見面了。故來回太太,侄兒必得就去纔好。只是家裏沒人照管。薔兒芸兒雖說糊塗,到底是個男人,外頭有了事來還可傳個話。侄兒家裏倒沒有什麼事,秋桐是天天哭着喊着不願意在這裏,侄兒叫了他孃家的人來領了去了,倒省了平兒好些氣。雖是巧姐沒人照應,還虧平兒的心不很壞。妞兒心裏也明白,只是性氣比他娘還剛硬些,求太太時常管教管教他。”說着眼圈兒一紅,連忙把腰裏拴檳榔荷包的小絹子拉下來擦眼。王夫人道:“放着他親祖母在那裏,託我做什麼。”賈璉輕輕的說道:“太太要說這個話,侄兒就該活活兒的打死了。沒什麼說的,總求太太始終疼侄兒就是了。”說着,就跪下來了。王夫人也眼圈兒紅了,說:“你快起來,娘兒們說話兒,這是怎麼說。只是一件,孩子也大了,倘或你父親有個一差二錯又耽擱住了,或者有個門當戶對的來說親,還是等你回來,還是你太太作主?”賈璉道:“現在太太們在家,自然是太太們做主,不必等我。”王夫人道:“你要去,就寫了稟帖給二老爺送個信,說家下無人,你父親不知怎樣,快請二老爺將老太太的大事早早的完結,快快回來。”賈璉答應了“是”,正要走出去,復轉回來回說道:“咱們家的家下人家裏還夠使喚,只是園裏沒有人太空了。包勇又跟了他們老爺去了。姨太太住的房子,薛二爺已搬到自己的房子內住了。園裏一帶屋子都空着,忒沒照應,還得太太叫人常查看查看。那櫳翠庵原是咱們家的地基,如今妙玉不知那裏去了,所有的根基他的當家女尼不敢自己作主,要求府裏一個人管理管理。”王夫人道:“自己的事還鬧不清,還擱得住外頭的事麼。這句話好歹別叫四丫頭知道,若是他知道了,又要吵着出家的念頭出來了。你想咱們傢什麼樣的人家,好好的姑娘出了家,還了得!”賈璉道:“太太不提起侄兒也不敢說,四妹妹到底是東府裏的,又沒有父母,他親哥哥又在外頭,他親嫂子又不大說的上話。侄兒聽見要尋死覓活了好幾次。他既是心裏這麼着的了,若是牛着他,將來倘或認真尋了死,比出家更不好了。”王夫人聽了點頭道:“這件事真真叫我也難擔。我也做不得主,由他大嫂子去就是了。”
\end{parag}


\begin{parag}
    賈璉又說了幾句纔出來,叫了衆家人來交待清楚,寫了書,收拾了行裝,平兒等不免叮嚀了好些話。只有巧姐兒慘傷的了不得,賈璉又欲託王仁照應,巧姐到底不願意,聽見外頭託了芸薔二人,心裏更不受用,嘴裏卻說不出來,只得送了他父親,謹謹慎慎的隨着平兒過日子。豐兒小紅因鳳姐去世,告假的告假,告病的告病,平兒意欲接了家中一個姑娘來,一則給巧姐作伴,二則可以帶量他。遍想無人,只有喜鸞四姐兒是賈母舊日鍾愛的,偏偏四姐兒新近出了嫁了,喜鸞也有了人家兒,不日就要出閣,也只得罷了。
\end{parag}


\begin{parag}
    且說賈芸賈薔送了賈璉,便進來見了邢王二夫人。他兩個倒替着在外書房住下,日間便與家人廝鬧,有時找了幾個朋友喫個車箍轆會,甚至聚賭,裏頭那裏知道。一日邢大舅王仁來,瞧見了賈芸賈薔住在這裏,知他熱鬧,也就藉着照看的名兒時常在外書房設局賭錢喝酒。所有幾個正經的家人,賈政帶了幾個去,賈璉又跟去了幾個,只有那賴林諸家的兒子侄兒。那些少年託着老子孃的福喫喝慣了的,那知當家立計的道理。況且他們長輩都不在家,便是沒籠頭的馬了,又有兩個旁主人慫恿,無不樂爲。這一鬧,把個榮國府鬧得沒上沒下,沒裏沒外。那賈薔還想勾引寶玉,賈芸攔住道:“寶二爺那個人沒運氣的,不用惹他。那一年我給他說了一門子絕好的親,父親在外頭做稅官,家裏開幾個當鋪,姑娘長的比仙女兒還好看。我巴巴兒的細細的寫了一封書子給他,誰知他沒造化,——”說到這裏,瞧了瞧左右無人,又說:“他心裏早和咱們這個二嬸孃好上了。你沒聽見說,還有一個林姑娘呢,弄的害了相思病死的,誰不知道。這也罷了,各自的姻緣罷咧。誰知他爲這件事倒惱了我了,總不大理。他打諒誰必是借誰的光兒呢。”賈薔聽了點點頭,才把這個心歇了。
\end{parag}


\begin{parag}
    他兩個還不知道寶玉自會那和尚以後,他是欲斷塵緣。一則在王夫人跟前不敢任性,已與寶釵襲人等皆不大款洽了。那些丫頭不知道,還要逗他,寶玉那裏看得到眼裏。他也並不將家事放在心裏。時常王夫人寶釵勸他念書,他便假作攻書,一心想着那個和尚引他到那仙境的機關。心目中觸處皆爲俗人,卻在家難受,閒來倒與惜春閒講。他們兩個人講得上了,那種心更加準了幾分,那裏還管賈環賈蘭等。那賈環爲他父親不在家,趙姨娘已死,王夫人不大理會他,便入了賈薔一路。倒是彩雲時常規勸,反被賈環辱罵。玉釧兒見寶玉瘋顛更甚,早和他娘說了要求着出去。如今寶玉賈環他哥兒兩個各有一種脾氣,鬧得人人不理。獨有賈蘭跟着他母親上緊攻書,作了文字送到學裏請教代儒。因近來代儒老病在牀,只得自己刻苦。李紈是素來沉靜,除了請王夫人的安,會會寶釵,餘者一步不走,只有看着賈蘭攻書。所以榮府住的人雖不少,竟是各自過各自的,誰也不肯做誰的主。賈環賈薔等愈鬧的不象事了,甚至偷典偷賣,不一而足。賈環更加宿娼濫賭,無所不爲。
\end{parag}


\begin{parag}
    一日邢大舅王仁都在賈家外書房喝酒,一時高興,叫了幾個陪酒的來唱着喝着勸酒。賈薔便說:“你們鬧的太俗。我要行個令兒。”衆人道:“使得。”賈薔道:“咱們‘月’字流觴罷。我先說起‘月’字,數到那個便是那個喝酒,還要酒面酒底。須得依着令官,不依者罰三大杯。”衆人都依了。賈薔喝了一杯令酒,便說:“飛羽觴而醉月。”順飲數到賈環。賈薔說:“酒面要個‘桂’字。”賈環便說道“‘冷露無聲溼桂花’。酒底呢?”賈薔道:“說個‘香’字。”賈環道:“天香雲外飄。”大舅說道:“沒趣,沒趣。你又懂得什麼字了,也假斯文起來!這不是取樂,竟是慪人了。咱們都蠲了,倒是搳搳拳,輸家喝輸家唱,叫做‘苦中苦’。若是不會唱的,說個笑話兒也使得,只要有趣。”衆人都道:“使得。”於是亂搳起來。王仁輸了,喝了一杯,唱了一個。衆人道好,又搳起來了。是個陪酒的輸了,唱了一個什麼“小姐小姐多豐彩”。以後邢大舅輸了,衆人要他唱曲兒,他道:“我唱不上來的,我說個笑話兒罷。”賈薔道:“若說不笑仍要罰的。”邢大舅就喝了杯,便說道:“諸位聽着:村莊上有一座元帝廟,旁邊有個土地祠。那元帝老爺常叫土地來說閒話兒。一日元帝廟裏被了盜,便叫土地去查訪。土地稟道:‘這地方沒有賊的,必是神將不小心,被外賊偷了東西去。’元帝道:‘胡說,你是土地,失了盜不問你問誰去呢?你倒不去拿賊,反說我的神將不小心嗎?’土地稟道:‘雖說是不小心,到底是廟裏的風水不好。’元帝道:‘你倒會看風水麼?’土地道:‘待小神看看。’那土地向各處瞧了一會,便來回稟道:‘老爺坐的身子背後兩扇紅門就不謹慎。小神坐的背後是砌的牆,自然東西丟不了。以後老爺的背後亦改了牆就好了。’元帝老爺聽來有理,便叫神將派人打牆。衆神將嘆口氣道:‘如今香火一炷也沒有,那裏有磚灰人工來打牆!’元帝老爺沒法,叫衆神將作法,卻都沒有主意。那元帝老爺腳下的龜將軍站起來道:‘你們不中用,我有主意。你們將紅門拆下來,到了夜裏拿我的肚子墊住這門口,難道當不得一堵牆麼?’衆神將都說道:‘好,又不花錢,又便當結實。’於是龜將軍便當這個差使,竟安靜了。豈知過了幾天,那廟裏又丟了東西。衆神將叫了土地來說道:‘你說砌了牆就不丟東西,怎麼如今有了牆還要丟?’那土地道:‘這牆砌的不結實。’衆神將道:‘你瞧去。’土地一看,果然是一堵好牆,怎麼還有失事?把手摸了一摸道:‘我打諒是真牆,那裏知道是個假牆!’”衆人聽了大笑起來。賈薔也忍不住的笑,說道:“傻大舅,你好!我沒有罵你,你爲什麼罵我!快拿杯來罰一大杯。”邢大舅喝了,已有醉意。衆人又喝了幾杯,都醉起來。邢大舅說他姐姐不好,王仁說他妹妹不好,都說的狠狠毒毒的。賈環聽了,趁着酒興也說鳳姐不好,怎樣苛刻我們,怎麼樣踏我們的頭。衆人道:“大凡做個人,原要厚道些。看鳳姑娘仗着老太太這樣的利害,如今焦了尾巴梢子了,只剩了一個姐兒,只怕也要現世現報呢。”賈芸想着鳳姐待他不好,又想起巧姐兒見他就哭,也信着嘴兒混說。還是賈薔道:“喝酒罷,說人家做什麼。”那兩個陪酒的道:“這位姑娘多大年紀了?長得怎麼樣?”賈薔道:“模樣兒是好的很的。年紀也有十三四歲了。”那陪酒的說道:“可惜這樣人生在府裏這樣人家,若生在小戶人家,父母兄弟都做了官,還發了財呢。”衆人道:“怎麼樣?”那陪酒的說:“現今有個外藩王爺,最是有情的,要選一個妃子。若合了式,父母兄弟都跟了去。可不是好事兒嗎?”衆人都不大理會,只有王仁心裏略動了一動,仍舊喝酒。
\end{parag}


\begin{parag}
    只見外頭走進賴林兩家的子弟來,說:“爺們好樂呀!”衆人站起來說道:“老大老三怎麼這時候纔來?叫我們好等!”那兩個人說道:“今早聽見一個謠言,說是咱們家又鬧出事來了,心裏着急,趕到裏頭打聽去,並不是咱們。”衆人道:“不是咱們就完了,爲什麼不就來?”那兩個說道:“雖不是咱們,也有些干係。你們知道是誰,就是賈雨村老爺。我們今兒進去,看見帶着鎖子,說要解到三法司衙門裏審問去呢。我們見他常在咱們家裏來往,恐有什麼事,便跟了去打聽。”賈芸道:“到底老大用心,原該打聽打聽。你且坐下喝一杯再說。”兩人讓了一回,便坐下,喝着酒道:“這位雨村老爺人也能幹,也會鑽營,官也不小了,只是貪財,被人家參了個婪索屬員的幾款。如今的萬歲爺是最聖明最仁慈的,獨聽了一個‘貪’字,或因糟蹋了百姓,或因恃勢欺良,是極生氣的,所以旨意便叫拿問。若是問出來了,只怕擱不住。若是沒有的事,那參的人也不便。如今真真是好時候,只要有造化做個官兒就好。”衆人道:“你的哥哥就是有造化的,現做知縣還不好麼。”賴家的說道:“我哥哥雖是做了知縣,他的行爲只怕也保不住怎麼樣呢。”衆人道:“手也長麼?”賴家的點點頭兒,便舉起杯來喝酒。衆人又道:“裏頭還聽見什麼新聞?”兩人道:“別的事沒有,只聽見海疆的賊寇拿住了好些,也解到法司衙門裏審問。還審出好些賊寇,也有藏在城裏的,打聽消息,抽空兒就劫搶人家,如今知道朝裏那些老爺們都是能文能武,出力報效,所到之處早就消滅了。”衆人道:“你聽見有在城裏的,不知審出咱們家失盜了一案來沒有?”兩人道:“倒沒有聽見。恍惚有人說是有個內地裏的人,城裏犯了事,搶了一個女人下海去了。那女人不依,被這賊寇殺了。那賊寇正要跳出關去,被官兵拿住了,就在拿獲的地方正了法了。”衆人道:“咱們櫳翠庵的什麼妙玉不是叫人搶去,不要就是他罷?”賈環道:“必是他!”衆人道:“你怎麼知道?”賈環道:“妙玉這個東西是最討人嫌的。他一日家捏酸,見了寶玉就眉開眼笑了。我若見了他,他從不拿正眼瞧我一瞧。真要是他,我才趁願呢!”衆人道:“搶的人也不少,那裏就是他。”賈芸道:“有點信兒。前日有個人說,他庵裏的道婆做夢,說看見是妙玉叫人殺了。”衆人笑道:“夢話算不得。”邢大舅道:“管他夢不夢,咱們快喫飯罷。今夜做個大輸贏。”衆人願意,便喫畢了飯,大賭起來。
\end{parag}


\begin{parag}
    賭到三更多天,只聽見裏頭亂嚷,說是四姑娘合珍大奶奶拌嘴,把頭髮都絞掉了,趕到邢夫人王夫人那裏去磕了頭,說是要求容他做尼姑呢,送他一個地方,若不容他他就死在眼前。那邢王兩位太太沒主意,叫請薔大爺芸二爺進去。賈芸聽了,便知是那回看家的時候起的念頭,想來是勸不過來的了,便合賈薔商議道:“太太叫我們進去,我們是做不得主的。況且也不好做主,只好勸去。若勸不住,只好由他們罷。咱們商量了寫封書給璉二叔,便卸了我們的干係了。”兩人商量定了主意,進去見了邢王兩位太太,便假意的勸了一回。無奈惜春立意必要出家,就不放他出去,只求一兩間淨屋子給他誦經拜佛。尤氏見他兩個不肯作主,又怕惜春尋死,自己便硬做主張,說是:“這個不是索性我耽了罷。說我做嫂子的容不下小姑子,逼他出了家了就完了。若說到外頭去呢,斷斷使不得。若在家裏呢,太太們都在這裏,算我的主意罷。叫薔哥兒寫封書子給你珍大爺璉二叔就是了。”賈薔等答應了。不知邢王二夫人依與不依,下回分解。
\end{parag}