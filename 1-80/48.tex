\chap{四十八}{濫情人情誤思遊藝 慕雅女雅集苦吟詩}

\begin{parag}
    \begin{note}庚:題曰“柳湘蓮走他鄉”,必謂寫湘蓮如何走,今卻不寫,反寫阿呆兄之遊藝,了卻柳湘蓮之分內走者而不細寫其走,反寫阿呆不應走而寫其走,文牽歧路,令人不識者如此。\end{note}
\end{parag}


\begin{parag}
    \begin{note}庚:至“情小妹”回中方寫湘蓮文字,真神化之筆。\end{note}
\end{parag}


\begin{parag}
    \begin{note}蒙回前總:心地聰明性自靈,喜同雅品講詩經,嬌柔倍覺可憐形。皓齒朱脣真嬝嬝,癡情專意更娉娉,宜人解語小星星。\end{note}
\end{parag}


\begin{parag}
    且說薛蟠聽見如此說了,氣方漸平。三五日後,疼痛雖愈,傷痕未平,只裝病在家,愧見親友。
\end{parag}


\begin{parag}
    展眼已到十月,因有各鋪面夥計內有算年帳要回家的,少不得家內治酒餞行。內有一個張德輝,年過六十,自幼在薛家當鋪內攬總,家內也有二三千金的過活,今歲也要回家,明春方來。因說起“今年紙札香料短少,明年必是貴的。明年先打發大小兒上來當鋪內照管,趕端陽前我順路販些紙札香扇來賣。除去關稅花銷,亦可以剩得幾倍利息。”薛蟠聽了,心中忖度:“我如今捱了打,正難見人,想著要躲個一年半載,又沒處去躲。天天裝病,也不是事。況且我長了這麼大,文又不文,武又不武,雖說做買賣,究竟戥子算盤從沒拿過,地土風俗遠近道路又不知道,不如也打點幾個本錢,和張德輝逛一年來。賺錢也罷,不賺錢也罷,且躲躲羞去。二則逛逛山水也是好的。”心內主意已定,至酒席散後,便和張德輝說知,命他等一二日一同前往。
\end{parag}


\begin{parag}
    晚間薛蟠告訴了他母親。薛姨媽聽了雖是歡喜,但又恐他在外生事,花了本錢倒是末事,因此不命他去,只說:“好歹你守著我,我還能放心些。況且也不用做這買賣,也不等著這幾百銀子來用。你在家裏安分守己的,就強似這幾百銀子了。”薛蟠主意已定,那裏肯依,只說:“天天又說我不知世事,這個也不知,那個也不學。如今我發狠把那些沒要緊的都斷了,如今要成人立事,學習著做買賣,又不准我了,叫我怎麼樣呢?我又不是個丫頭,把我關在家裏,何日是個了日?況且那張德輝又是個年高有德的,咱們和他世交,我同他去,怎麼得有舛錯?我就一時半刻有不好的去處,他自然說我勸我。就是東西貴賤行情,他是知道的,自然色色問他,何等順利,倒不叫我去。過兩日我不告訴家裏,私自打點了一走,明年發了財回家,那時才知道我呢。”說畢,賭氣睡覺去了。
\end{parag}


\begin{parag}
    薛姨媽聽他如此說,因和寶釵商議。寶釵笑道:“哥哥果然要經歷正事,正是好的了。只是他在家時說著好聽,到了外頭舊病復犯,越發難拘束他了。但也愁不得許多。他若是真改了,是他一生的福。若不改,媽也不能又有別的法子。一半盡人力,一半聽天命罷了。這麼大人了,若只管怕他不知世路,出不得門,幹不得事,今年關在家裏,明年還是這個樣兒。他既說的名正言順,媽就打諒著丟了八百一千銀子,竟交與他試一試。橫豎有夥計們幫著,也未必好意思哄騙他的。二則他出去了,左右沒有助興的人,又沒了倚仗的人,到了外頭,誰還怕誰,有了的喫,沒了的餓著,舉眼無靠,他見這樣,只怕比在家裏省了事也未可知。”\begin{note}庚雙夾:作書者曾喫此虧,批書者亦曾喫此虧,故特於此註明,使後來人深思默戒。脂硯齋。\end{note}薛姨媽聽了,思忖半晌說道:“倒是你說的是。花兩個錢,叫他學些乖來也值了。”商議已定,一宿無話。
\end{parag}


\begin{parag}
    至次日,薛姨媽命人請了張德輝來,在書房中命薛蟠款待酒飯,自己在後廊下,隔著窗子,向裏千言萬語囑託張德輝照管薛蟠。張德輝滿口應承,喫過飯告辭,又回說:“十四日是上好出行日期,大世兄即刻打點行李,僱下騾子,十四一早就長行了。”薛蟠喜之不盡,將此話告訴了薛姨媽。薛姨媽便和寶釵香菱並兩個老年的嬤嬤連日打點行裝,派下薛蟠之乳父老蒼頭一名,當年諳事舊僕二名,外有薛蟠隨身常使小廝二人,主僕一共六人,僱了三輛大車,單拉行李使物,又僱了四個長行騾子。薛蟠自騎一匹家內養的鐵青大走騾,外備一匹坐馬。諸事完畢,薛姨媽寶釵等連夜勸戒之言,自不必備說。
\end{parag}


\begin{parag}
    至十三日,薛蟠先去辭了他舅舅,然後過來辭了賈宅諸人。賈珍等未免又有餞行之說,也不必細述。至十四日一早,薛姨媽寶釵等直同薛蟠出了儀門,母女兩個四隻淚眼看他去了,方回來。
\end{parag}


\begin{parag}
    薛姨媽上京帶來的家人不過四五房,並兩三個老嬤嬤小丫頭,今跟了薛蟠一去,外面只剩了一兩個男子。因此薛姨媽即日到書房,將一應陳設玩器並簾幔等物盡行搬了進來收貯,命那兩個跟去的男子之妻一併也進來睡覺。又命香菱將他屋裏也收拾嚴緊,“將門鎖了,晚間和我去睡。”寶釵道:“媽既有這些人作伴,不如叫菱姐姐和我作伴去。我們園裏又空,夜長了,我每夜作活,越多一個人豈不越好。”薛姨媽聽了,笑道:“正是我忘了,原該叫他同你去纔是。我前日還同你哥哥說,文杏又小,道三不著兩,鶯兒一個人不夠伏侍的,還要買一個丫頭來你使。”寶釵道:“買的不知底裏,倘或走了眼,花了錢小事,沒的淘氣。倒是慢慢的打聽著,有知道來歷的,買個還罷了。”\begin{note}庚雙夾:閒言過耳無跡,然又伏下一事矣。\end{note}一面說,一面命香菱收拾了衾褥妝奩,命一個老嬤嬤並臻兒送至蘅蕪苑去,然後寶釵和香菱才同回園中來。\begin{note}庚雙夾:細想香菱之爲人也,根基不讓迎、探,容貌不讓鳳、秦,端雅不讓紈、釵,風流不讓湘、黛,賢惠不讓襲、平,所惜者青年罹禍,命運乖蹇,至爲側室,且雖曾讀書,不能與林、湘輩並馳於海棠之社耳。然此一人豈可不入園哉?故欲令入園,終無可入之隙,籌劃再四,欲令入園必呆兄遠行後方可。然阿呆兄又如何方可遠行?曰名,不可;利,不可;無事,不可;必得萬人想不到,自己忽發一機之事方可。因此思及“情”之一字及呆素所誤者,故借“情誤”二字生出一事,使阿呆遊藝之志已堅,則菱卿入園之隙方妥。回思因欲香菱入園,是寫阿呆情誤,因欲阿呆情誤,先寫一賴尚榮,實委婉嚴密之甚也。脂硯齋評。\end{note}\begin{note}靖眉:此批甚當。\end{note}
\end{parag}


\begin{parag}
    香菱道:“我原要和奶奶說的,大爺去了,我和姑娘作伴兒去。又恐怕奶奶多心,說我貪著園裏來頑;誰知你竟說了。”寶釵笑道:“我知道你心裏羨慕這園子不是一日兩日了,只是沒個空兒。就每日來一趟,慌慌張張的,也沒趣兒。所以趁著機會,越性住上一年,我也多個作伴的,你也遂了心。”香菱笑道:“好姑娘,你趁著這個功夫,教給我作詩罷。”\begin{note}庚雙夾:寫得何其有趣,今忽見菱卿此句,合卷從紙上另走出一嬌小美人來,並不是湘、林、探、鳳等一樣口氣聲色。真神駿之技,雖驅馳萬里而不見有倦怠之色。\end{note}寶釵笑道:“我說你‘得隴望蜀’呢。我勸你今兒頭一日進來,先出園東角門,從老太太起,各處各人你都瞧瞧,問候一聲兒,也不必特意告訴他們說搬進園來。若有提起因由,你只帶口說我帶了你進來作伴兒就完了。回來進了園,再到各姑娘房裏走走。”
\end{parag}


\begin{parag}
    香菱應著纔要走時,只見平兒忙忙的走來。\begin{note}庚雙夾:“忙忙”二字奇,不知有何妙文。\end{note}香菱忙問了好,平兒只得陪笑相問。寶釵因向平兒笑道:“我今兒帶了他來作伴兒,正要去回你奶奶一聲兒。”平兒笑道:“姑娘說的是那裏話?我竟沒話答言了。” 寶釵道:“這纔是正理。店房也有個主人,廟裏也有個住持。雖不是大事,到底告訴一聲,便是園裏坐更上夜的人知道添了他兩個,也好關門候戶的了。你回去告訴一聲罷,我不打發人去了。”平兒答應著,因又向香菱笑道:“你既來了,也不拜一拜街坊鄰舍去?”\begin{note}庚雙夾:是極,恰是戲言,實欲支出香菱去也。\end{note}寶釵笑道:“我正叫他去呢。”平兒道:“你且不必往我們家去,二爺病了在家裏呢。”香菱答應著去了,先從賈母處來,不在話下。
\end{parag}


\begin{parag}
    且說平兒見香菱去了,便拉寶釵忙說道:“姑娘可聽見我們的新聞了?”寶釵道:“我沒聽見新聞。因連日打發我哥哥出門,所以你們這裏的事,一概也不知道,連姊妹們這兩日也沒見。”平兒笑道:“老爺把二爺打了個動不得,難道姑娘就沒聽見?”寶釵道:“早起恍惚聽見了一句,也信不真。我也正要瞧你奶奶去呢,不想你來了。又是爲了什麼打他?”平兒咬牙罵道:“都是那賈雨村什麼風村,半路途中那裏來的餓不死的野雜種!認了不到十年,生了多少事出來!今年春天,老爺不知在那個地方看見了幾把舊扇子,回家看家裏所有收著的這些好扇子都不中用了,立刻叫人各處搜求。誰知就有一個不知死的冤家,混號兒世人叫他作石呆子,窮的連飯也沒的喫,偏他家就有二十把舊扇子,死也不肯拿出大門來。二爺好容易煩了多少情,見了這個人,說之再三,把二爺請到他家裏坐著,拿出這扇子略瞧了一瞧。據二爺說,原是不能再有的,全是湘妃、棕竹、麋鹿、玉竹的,皆是古人寫畫真跡,因來告訴了老爺。老爺便叫買他的,要多少銀子給他多少。偏那石呆子說:‘我餓死凍死,一千兩銀子一把我也不賣!’老爺沒法子,天天罵二爺沒能爲。已經許了他五百兩,先兌銀子後拿扇子。他只是不賣,只說:‘要扇子,先要我的命!’姑娘想想,這有什麼法子?誰知雨村那沒天理的聽見了,便設了個法子,訛他拖欠了官銀,拿他到衙門裏去,說所欠官銀,變賣家產賠補,把這扇子抄了來,作了官價送了來。那石呆子如今不知是死是活。老爺拿著扇子問著二爺說:‘人家怎麼弄了來?’二爺只說了一句:‘爲這點子小事,弄得人坑家敗業,也不算什麼能爲!’老爺聽了就生了氣,說二爺拿話堵老爺,因此這是第一件大的。這幾日還有幾件小的,我也記不清,所以都湊在一處,就打起來了。也沒拉倒用板子棍子,就站著,不知拿什麼混打一頓,臉上打破了兩處。我們聽見姨太太這裏有一種丸藥,上棒瘡的,姑娘快尋一丸子給我。”寶釵聽了,忙命鶯兒去要了一丸來與平兒。寶釵道:“既這樣,替我問候罷,我就不去了。”平兒答應著去了,不在話下。
\end{parag}


\begin{parag}
    且說香菱見過衆人之後,喫過晚飯,寶釵等都往賈母處去了,自己便往瀟湘館中來。此時黛玉已好了大半,見香菱也進園來住,自是歡喜。香菱因笑道:“我這一進來了,也得了空兒,好歹教給我作詩,就是我的造化了!”黛玉笑道:“既要作詩,你就拜我作師。我雖不通,大略也還教得起你。”香菱笑道:“果然這樣,我就拜你作師。你可不許膩煩的。”黛玉道:“什麼難事,也值得去學!不過是起承轉合,當中承轉是兩副對子,平聲對仄聲,虛的對實的,實的對虛的,若是果有了奇句,連平仄虛實不對都使得的。”香菱笑道:“怪道我常弄一本舊詩偷空兒看一兩首,又有對的極工的,又有不對的,又聽見說‘一三五不論,二四六分明’。看古人的詩上亦有順的,亦有二四六上錯了的,所以天天疑惑。如今聽你一說,原來這些格調規矩竟是末事,只要詞句新奇爲上。”黛玉道:“正是這個道理。詞句究竟還是末事,第一立意要緊。若意趣真了,連詞句不用修飾,自是好的,這叫做‘不以詞害意’。”香菱笑道:“我只愛陸放翁的詩‘重簾不卷留香久,古硯微凹聚墨多’,說的真有趣!”黛玉道:“斷不可學這樣的詩。你們因不知詩,所以見了這淺近的就愛,一入了這個格局,再學不出來的。你只聽我說,你若真心要學,我這裏有《王摩詰全集》,你且把他的五言律讀一百首,細心揣摩透熟了,然後再讀一二百首老杜的七言律,次再李青蓮的七言絕句讀一二百首。肚子裏先有了這三個人作了底子,然後再把陶淵明、應瑒、謝、阮、庾、鮑等人的一看。你又是一個極聰敏伶俐的人,不用一年的工夫,不愁不是詩翁了!”香菱聽了,笑道:“既這樣,好姑娘,你就把這書給我拿出來,我帶回去夜裏念幾首也是好的。”黛玉聽說,便命紫鵑將王右丞的五言律拿來,遞與香菱,又道:“你只看有紅圈的都是我選的,有一首念一首。不明白的問你姑娘,或者遇見我,我講與你就是了。” 香菱拿了詩,回至蘅蕪苑中,諸事不顧,只向燈下一首一首的讀起來。寶釵連催他數次睡覺,他也不睡。寶釵見他這般苦心,只得隨他去了。
\end{parag}


\begin{parag}
    一日,黛玉方梳洗完了,只見香菱笑吟吟的送了書來,又要換杜律。黛玉笑道:“共記得多少首?”香菱笑道:“凡紅圈選的我盡讀了。”黛玉道:“可領略了些滋味沒有?”香菱笑道:“領略了些滋味,不知可是不是,說與你聽聽。”黛玉笑道:“正要講究討論,方能長進。你且說來我聽。”香菱笑道:“據我看來,詩的好處,有口裏說不出來的意思,想去卻是逼真的。有似乎無理的,想去竟是有理有情的。”黛玉笑道:“這話有了些意思,但不知你從何處見得?”香菱笑道: “我看他《塞上》一首,那一聯雲:‘大漠孤煙直,長河落日圓。’想來煙如何直?日自然是圓的:這‘直’字似無理,‘圓’字似太俗。合上書一想,倒象是見了這景的。若說再找兩個字換這兩個,竟再找不出兩個字來。再還有‘日落江湖白,潮來天地青’,這‘白’‘青’兩個字也似無理。想來,必得這兩個字才形容得盡,念在嘴裏倒象有幾千斤重的一個橄欖。還有‘渡頭餘落日,墟里上孤煙’,這‘餘’字和‘上’字,難爲他怎麼想來!我們那年上京來,那日下晚便灣住船,岸上又沒有人,只有幾棵樹,遠遠的幾家人家作晚飯,那個煙竟是碧青,連雲直上。誰知我昨日晚上讀了這兩句,倒象我又到了那個地方去了。”
\end{parag}


\begin{parag}
    正說著,寶玉和探春也來了,也都入坐聽他講詩。寶玉笑道:“既是這樣,也不用看詩。會心處不在多,聽你說了這兩句,可知三昧你已得了。”黛玉笑道: “你說他這‘上孤煙’ 好,你還不知他這一句還是套了前人的來。我給你這一句瞧瞧,更比這個淡而現成。”說著便把陶淵明的“曖曖遠人村,依依墟里煙”翻了出來,遞與香菱。香菱瞧了,點頭歎賞,笑道:“原來‘上’字是從‘依依’兩個字上化出來的。”寶玉大笑道:“你已得了,不用再講,越發倒學雜了。你就作起來,必是好的。”探春笑道:“明兒我補一個柬來,請你入社。”香菱笑道:“姑娘何苦打趣我,我不過是心裏羨慕,才學著頑罷了。”探春黛玉都笑道:“誰不是頑?難道我們是認真作詩呢!若說我們認真成了詩,出了這園子,把人的牙還笑倒了呢。”寶玉道:“這也算自暴自棄了。前日我在外頭和相公們商議畫兒,他們聽見咱們起詩社,求我把稿子給他們瞧瞧。我就寫了幾首給他們看看,誰不真心歎服。他們都抄了刻去了。”探春黛玉忙問道:“這是真話麼?”寶玉笑道:“說謊的是那架上的鸚哥。”黛玉探春聽說,都道:“你真真胡鬧!且別說那不成詩,便是成詩,我們的筆墨也不該傳到外頭去。”寶玉道:“這怕什麼!古來閨閣中的筆墨不要傳出去,如今也沒有人知道了。”說著,只見惜春打發了入畫來請寶玉,寶玉方去了。香菱又逼著黛玉換出杜律來,又央黛玉探春二人:“出個題目,讓我謅去,謅了來,替我改正。” 黛玉道:“昨夜的月最好,我正要謅一首,竟未謅成,你竟作一首來。‘十四寒’的韻,由你愛用那幾個字去。”
\end{parag}


\begin{parag}
    香菱聽了,喜的拿回詩來,又苦思一回作兩句詩,又捨不得杜詩,又讀兩首。如此茶飯無心,坐臥不定。寶釵道:“何苦自尋煩惱。都是顰兒引的你,我和他算賬去。你本來呆頭呆腦的,再添上這個,越發弄成個呆子了。”\begin{note}庚雙夾:“呆頭呆腦的”有趣之至!最恨野史有一百個女子皆曰“聰敏伶俐”,究竟看來,他行爲也只平平。今以“呆”字爲香菱定評,何等嫵媚之至也。\end{note}香菱笑道:“好姑娘,別混我。”\begin{note}庚雙夾:如聞如見。\end{note}一面說,一面作了一首,先與寶釵看。寶釵看了笑道:“這個不好,不是這個作法。你別怕臊,只管拿了給他瞧去,看他是怎麼說。”香菱聽了,便拿了詩找黛玉。黛玉看時,只見寫道是:
\end{parag}


\begin{poem}
    \begin{pl}月掛中天夜色寒,清光皎皎影團團。\end{pl}

    \begin{pl}詩人助興常思玩,野客添愁不忍觀。\end{pl}

    \begin{pl}翡翠樓邊懸玉鏡,珍珠簾外掛冰盤。\end{pl}

    \begin{pl}良宵何用燒銀燭,晴彩輝煌映畫欄。\end{pl}
\end{poem}


\begin{parag}
    黛玉笑道:“意思卻有,只是措詞不雅。皆因你看的詩少,被他縛住了。把這首丟開,再作一首。只管放開膽子去作。”
\end{parag}


\begin{parag}
    香菱聽了,默默的回來,越性連房也不入,只在池邊樹下,或坐在山石上出神,或蹲在地下摳土,來往的人都詫異。李紈、寶釵、探春、寶玉等聽得此信,都遠遠的站在山坡上瞧看他。只見他皺一回眉,又自己含笑一回。寶釵笑道:“這個人定要瘋了!昨夜嘟嘟噥噥直鬧到五更天才睡下,沒一頓飯的工夫天就亮了。我就聽見他起來了,忙忙碌碌梳了頭就找顰兒去。一回來了,呆了一日,作了一首又不好,這會子自然另作呢。”寶玉笑道:“這正是‘地靈人傑’,老天生人再不虛賦情性的。我們成日嘆說可惜他這麼個人竟俗了,誰知到底有今日。可見天地至公。”寶釵笑道:“你能夠像他這苦心就好了,學什麼有個不成的。”寶玉不答。
\end{parag}


\begin{parag}
    只見香菱興興頭頭的又往黛玉那邊去了。探春笑道:“咱們跟了去,看他有些意思沒有。”說著,一齊都往瀟湘館來。只見黛玉正拿著詩和他講究。衆人因問黛玉作的如何。黛玉道:“自然算難爲他了,只是還不好。這一首過於穿鑿了,還得另作。”衆人因要詩看時,只見作道:
\end{parag}


\begin{poem}
    \begin{pl}非銀非水映窗寒,試看晴空護玉盤。\end{pl}

    \begin{pl}淡淡梅花香欲染,絲絲柳帶露初幹。\end{pl}

    \begin{pl}只疑殘粉塗金砌,恍若輕霜抹玉欄。\end{pl}

    \begin{pl}夢醒西樓人跡絕,餘容猶可隔簾看。\end{pl}
\end{poem}


\begin{parag}
    寶釵笑道:“不象吟月了,月字底下添一個‘色’字倒還使得,你看句句倒是月色。這也罷了,原來詩從胡說來,再遲幾天就好了。”香菱自爲這首妙絕,聽如此說,自己掃了興,不肯丟開手,便要思索起來。因見他姊妹們說笑,便自己走至階前竹下閒步,挖心搜膽,耳不旁聽,目不別視。一時探春隔窗笑說道:“菱姑娘,你閒閒罷。”香菱怔怔答道:“‘閒’字是‘十五刪’的,你錯了韻了。”衆人聽了,不覺大笑起來。寶釵道:“可真是詩魔了。都是顰兒引的他!”黛玉笑道:“聖人說:‘誨人不倦。’他又來問我,我豈有不說之理。”李紈笑道:“咱們拉了他往四姑娘房裏去,引他瞧瞧畫兒,叫他醒一醒纔好。”
\end{parag}


\begin{parag}
    說著,真個出來拉了他過藕香榭,至暖香塢中。惜春正乏倦,在牀上歪著睡午覺,畫繒立在壁間,用紗罩著。衆人喚醒了惜春,揭紗看時,十停方有了三停。香菱見畫上有幾個美人,因指著笑道:“這一個是我們姑娘,那一個是林姑娘。”探春笑道:“凡會作詩的都畫在上頭,快學罷。”說著,頑笑了一回。
\end{parag}


\begin{parag}
    各自散後,香菱滿心中還是想詩。至晚間對燈出了一回神,至三更以後上牀臥下,兩眼鰥鰥,直到五更方纔朦朧睡去了。一時天亮,寶釵醒了,聽了一聽,他安穩睡了,心下想:“他翻騰了一夜,不知可作成了?這會子乏了,且別叫他。”正想著,只聽香菱從夢中笑道:“可是有了,難道這一首還不好?”寶釵聽了,又是可嘆,又是可笑,連忙喚醒了他,問他:“得了什麼?你這誠心都通了仙了。學不成詩,還弄出病來呢。”一面說,一面梳洗了,會同姊妹往賈母處來。原來香菱苦志學詩,精血誠聚,日間做不出,忽於夢中得了八句。梳洗已畢,便忙錄出來,自己並不知好歹,便拿來又找黛玉。剛到沁芳亭,只見李紈與衆姊妹方從王夫人處回來,寶釵正告訴他們說他夢中作詩說夢話。\begin{note}庚雙夾:一部大書起是夢,寶玉情是夢,賈瑞淫又是夢,秦之家計長策又是夢,今作詩也是夢,一併“風月鑑” 亦從夢中所有,故“紅樓夢”也。餘今批評亦在夢中,特爲夢中之人作此一大夢也。脂硯齋。\end{note}衆人正笑,抬頭見他來了,便都爭著要詩看。且聽下回分解。
\end{parag}


\begin{parag}
    \begin{note}蒙回末總:一扇之微,而害人如此,其毒藏之者。故自無味,構求者更覺可笑。多少沒天理處,全不自覺。可見,好愛之端,斷不可生求古董於古墳,爭盆景而蕩產勢。所以至可不慎諸。\end{note}
\end{parag}

