\chap{七十二}{王熙鳳恃強羞說病 來旺婦倚勢霸成親}


\begin{parag}
    \begin{note}蒙回前總:此回似著意似不著意,似接續似不接續,在畫師爲濃淡相間,在墨客爲骨肉停勻,在樂工爲笙歌間作,在文壇爲養局爲別調。前後文氣,至此一歇。\end{note}
\end{parag}


\begin{parag}
    且說鴛鴦出了角門,臉上猶紅,心內突突的,真是意外之事。因想這事非常,若說出來,奸盜相連,關係人命,還保不住帶累了旁人。橫豎與自己無干,且藏在心內,不說與一人知道。回房復了賈母的命,大家安息。從此凡晚間便不大往園中來。因思園中尚有這樣奇事,何況別處,因此連別處也不大輕走動了。
\end{parag}


\begin{parag}
    原來那司棋因從小兒和他姑表兄弟在一處頑笑起住時,小兒戲言,便都訂下將來不娶不嫁。近年大了,彼此又出落的品貌風流,常時司棋回家時,二人眉來眼去,舊情不忘,只不能入手。又彼此生怕父母不從,二人便設法彼此裏外買囑園內老婆子們留門看道,今日趁亂方初次入港。雖未成雙,卻也海誓山盟,私傳表記,已有無限風情了。忽被鴛鴦驚散,那小廝早穿花度柳,從角門出去了。司棋一夜不曾睡著,又後悔不來。至次日見了鴛鴦,自是臉上一紅一白,百般過不去。心內懷著鬼胎,茶飯無心,起坐恍惚。捱了兩日,竟不聽見有動靜,方略放下了心。這日晚間,忽有個婆子來悄告訴他道:“你兄弟竟逃走了,三四天沒歸家。如今打發人四處找他呢。”司棋聽了,氣個倒仰,因思道:“縱是鬧了出來,也該死在一處。他自爲是男人,先就走了,可見是個沒情意的。”因此又添了一層氣。次日便覺心內不快,百般支持不住,一頭睡倒,懨懨的成了大病。
\end{parag}


\begin{parag}
    鴛鴦聞知那邊無故走了一個小廝,園內司棋又病重,要往外挪,心下料定是二人懼罪之故,“生怕我說出來,方嚇到這樣。”因此自己反過意不去,指著來望候司棋,支出人去,反自己立身發誓,與司棋說:“我告訴一個人,立刻現死現報!你只管放心養病,別白糟踏了小命兒。”司棋一把拉住,哭道:“我的姐姐,咱們從小兒耳鬢廝磨,你不曾拿我當外人待,我也不敢待慢了你。如今我雖一著走錯,你若果然不告訴一個人,你就是我的親孃一樣。從此後我活一日是你給我一日,我的病好之後,把你立個長生牌位,我天天焚香禮拜,保佑你一生福壽雙全。我若死了時,變驢變狗報答你。再俗語說:‘千里搭長棚,沒有不散的筵席。’再過三二年,咱們都是要離這裏的。俗語又說:‘浮萍尚有相逢日,人豈全無見面時。’倘或日後咱們遇見了,那時我又怎麼報你的德行。”一面說,一面哭。這一席話反把鴛鴦說的心酸,也哭起來了。因點頭道:“正是這話。我又不是管事的人,何苦我壞你的聲名,我白去獻勤。況且這事我自己也不便開口向人說。你只放心。從此養好了,可要安分守己,再不許胡行亂作了。”司棋在枕上點首不絕。
\end{parag}


\begin{parag}
    鴛鴦又安慰了他一番,方出來。因知賈璉不在家中,又因這兩日鳳姐兒聲色怠惰了些,不似往日一樣,因順路也來望候。因進入鳳姐院門,二門上的人見是他來,便立身待他進去。鴛鴦剛至堂屋中,只見平兒從裏間出來,見了他來,忙上來悄聲笑道:“才吃了一口飯歇了午睡,你且這屋裏略坐坐。”鴛鴦聽了,只得同平兒到東邊房裏來。小丫頭倒了茶來。鴛鴦因悄問:“你奶奶這兩日是怎麼了?我看他懶懶的。”平兒見問,因房內無人,便嘆道:“他這懶懶的也不止今日了,這有一月之前便是這樣。又兼這幾日忙亂了幾天,又受了些閒氣,從新又勾起來。這兩日比先又添了些病,所以支持不住,便露出馬腳來了。”鴛鴦忙道:“既這樣,怎麼不早請大夫來治?”平兒嘆道:“我的姐姐,你還不知道他的脾氣的。別說請大夫來吃藥。我看不過,白問了一聲身上覺怎麼樣,他就動了氣,反說我咒他病了。饒這樣,天天還是察三訪四,自己再不肯看破些且養身子。”鴛鴦道:“雖然如此,到底該請大夫來瞧瞧是什麼病,也都好放心。”平兒道:“我的姐姐,說起病來,據我看也不是什麼小症候。”鴛鴦忙道:“是什麼病呢?”平兒見問,又往前湊了一湊,向耳邊說道:“只從上月行了經之後,這一個月竟淅淅瀝瀝的沒有止住。這可是大病不是?”鴛鴦聽了,忙答道:“噯喲!依你這話,這可不成了血山崩了。”平兒忙啐了一口,又悄笑道:“你女孩兒家,這是怎麼說的,倒會咒人呢。”鴛鴦見說,不禁紅了臉,又悄笑道:“究竟我也不知什麼是崩不崩的,你倒忘了不成,先我姐姐不是害這病死了。我也不知是什麼病,因無心聽見媽和親家媽說,我還納悶,後來也是聽見媽細說原故,才明白了一二分。”平兒笑道:“你該知道的,我竟也忘了。”
\end{parag}


\begin{parag}
    二人正說著,只見小丫頭進來向平兒道:“方纔朱大娘又來了。我們回了他奶奶才歇午覺,他往太太上頭去了。”平兒聽了點頭。鴛鴦問:“那一個朱大娘?” 平兒道:“就是官媒婆那朱嫂子。因有什麼孫大人家來和咱們求親,所以他這兩日天天弄個帖子來賴死賴活。”一語未了,小丫頭跑來說:“二爺進來了。”說話之間,賈璉已走至堂屋門,口內喚平兒。平兒答應著才迎出去,賈璉已找至這間房內來。至門前,忽見鴛鴦坐在炕上,便煞住腳,笑道:“鴛鴦姐姐,今兒貴腳踏賤地。”鴛鴦只坐著,笑道:“來請爺奶奶的安,偏又不在家的不在家,睡覺的睡覺。”賈璉笑道:“姐姐一年到頭辛苦伏侍老太太,我還沒看你去,那裏還敢勞動來看我們。正是巧的很,我纔要找姐姐去。因爲穿著這袍子熱,先來換了夾袍子再過去找姐姐,不想天可憐,省我走這一趟,姐姐先在這裏等我了。”一面說,一面在椅上坐下。鴛鴦因問:“又有什麼說的?”賈璉未語先笑道:“因有一件事,我竟忘了,只怕姐姐還記得。上年老太太生日,曾有一個外路和尚來孝敬一個臘油凍的佛手,因老太太愛,就即刻拿過來擺著了。因前日老太太生日,我看古董帳上還有這一筆,卻不知此時這件東西著落何方。古董房裏的人也回過我兩次,等我問準了好註上一筆。所以我問姐姐,如今還是老太太擺著呢,還是交到誰手裏去了呢?”鴛鴦聽說,便道:“老太太擺了幾日厭煩了,就給了你們奶奶。你這會子又問我來。我連日子還記得,還是我打發了老王家的送來的。你忘了,或是問你們奶奶和平兒。”平兒正拿衣服,聽見如此說,忙出來回說:“交過來了,現在樓上放著呢。奶奶已經打發過人出去說過給了這屋裏,他們發昏,沒記上,又來叨登這些沒要緊的事。”賈璉聽說,笑道:“既然給了你奶奶,我怎麼不知道,你們就昧下了。”平兒道:“奶奶告訴二爺,二爺還要送人,奶奶不肯,好容易留下的。這會子自己忘了,倒說我們昧下。那是什麼好東西,什麼沒有的物兒。比那強十倍的東西也沒昧下一遭,這會子愛上那不值錢的!”賈璉垂頭含笑想了一想,拍手道:“我如今竟糊塗了!丟三忘四,惹人抱怨,竟大不象先了。”鴛鴦笑道:“也怨不得。事情又多,口舌又雜,你再喝上兩杯酒,那裏清楚的許多。”一面說,一面就起身要去。
\end{parag}


\begin{parag}
    賈璉忙也立身說道:“好姐姐,再坐一坐,兄弟還有事相求。”說著便罵小丫頭:“怎麼不沏好茶來!快拿乾淨蓋碗,把昨兒進上的新茶沏一碗來。”說著向鴛鴦道:“這兩日因老太太的千秋,所有的幾千兩銀子都使了。幾處房租地稅通在九月才得,這會子竟接不上。明兒又要送南安府裏的禮,又要預備娘娘的重陽節禮,還有幾家紅白大禮,至少還得三二千兩銀子用,一時難去支借。俗語說,‘求人不如求己’。說不得,姐姐擔個不是,暫且把老太太查不著的金銀傢伙偷著搬運出一箱子來,暫押千數兩銀子支騰過去。不上半年的光景,銀子來了,我就贖了交還,斷不能叫姐姐落不是。”鴛鴦聽了,笑道:“你倒會變法兒,虧你怎麼想來。”賈璉笑道:“不是我扯謊,若論除了姐姐,也還有人手裏管的起千數兩銀子的,只是他們爲人都不如你明白有膽量。我若和他們一說,反嚇住了他們。所以我‘寧撞金鐘一下,不打破鼓三千’。”一語未了,忽有賈母那邊的小丫頭子忙忙走來找鴛鴦,說:“老太太找姐姐半日,我們那裏沒找到,卻在這裏。”鴛鴦聽說,忙的且去見賈母。
\end{parag}


\begin{parag}
    賈璉見他去了,只得回來瞧鳳姐。誰知鳳姐已醒了,聽他和鴛鴦借當,自己不便答話,只躺在榻上。聽見鴛鴦去了,賈璉進來,鳳姐因問道:“他可應準了?” 賈璉笑道:“雖然未應準,卻有幾分成手,須得你晚上再和他一說,就十成了。”鳳姐笑道:“我不管這事。倘或說準了,這會子說得好聽,到有了錢的時節,你就丟在脖子後頭,誰去和你打饑荒去。倘或老太太知道了,倒把我這幾年的臉面都丟了。”賈璉笑道:“好人,你若說定了,我謝你如何?”鳳姐笑道:“你說,謝我什麼?”賈璉笑道:“你說要什麼就給你什麼。”平兒一旁笑道:“奶奶倒不要謝的。昨兒正說,要作一件什麼事,恰少一二百銀子使,不如借了來,奶奶拿一二百銀子,豈不兩全其美。”鳳姐笑道:“幸虧提起我來,就是這樣也罷。”賈璉笑道:“你們太也狠了。你們這會子別說一千兩的當頭,就是現銀子要三五千,只怕也難不倒。我不和你們借就罷了。這會子煩你說一句話,還要個利錢,真真了不得。”鳳姐聽了,翻身起來說:“我有三千五萬,不是賺的你的。如今裏裏外外上上下下背著我嚼說我的不少,就差你來說了,可知沒家親引不出外鬼來。我們王家可那裏來的錢,都是你們賈家賺的。別叫我噁心了。你們看著你傢什麼石崇鄧通。把我王家的地縫子掃一掃,就夠你們過一輩子呢。說出來的話也不怕臊!現有對證:把太太和我的嫁妝細看看,比一比你們的,那一樣是配不上你們的。”賈璉笑道: “說句頑話就急了。這有什麼這樣的,要使一二百兩銀子值什麼,多的沒有,這還有,先拿進來,你使了再說,如何?”鳳姐道:“我又不等著銜口墊背,忙了什麼。”賈璉道:“何苦來,不犯著這樣肝火盛。”鳳姐聽了,又自笑起來,“不是我著急,你說的話戳人的心。我因爲我想著後日是尤二姐的週年,我們好了一場,雖不能別的,到底給他上個墳燒張紙,也是姊妹一場。他雖沒留下個男女,也要‘前人撒土迷了後人的眼’纔是。”一語倒把賈璉說沒了話,低頭打算了半晌,方道:“難爲你想的周全,我竟忘了。既是後日才用,若明日得了這個,你隨便使多少就是了。”
\end{parag}


\begin{parag}
    一語未了,只見旺兒媳婦走進來。鳳姐便問:“可成了沒有?”旺兒媳婦道:“竟不中用。我說須得奶奶作主就成了。”賈璉便問:“又是什麼事?”鳳姐兒見問,便說道:“不是什麼大事。旺兒有個小子,今年十七歲了,還沒得女人,因要求太太房裏的彩霞,不知太太心裏怎麼樣,就沒有計較得。前日太太見彩霞大了,二則又多病多災的,因此開恩打發他出去了,給他老子娘隨便自己揀女婿去罷。因此旺兒媳婦來求我。我想他兩家也就算門當戶對的,一說去自然成的,誰知他這會子來了,說不中用。”賈璉道:“這是什麼大事,比彩霞好的多著呢。”旺兒家的陪笑道:“爺雖如此說,連他家還看不起我們,別人越發看不起我們了。好容易相看準一個媳婦,我只說求爺奶奶的恩典,替作成了。奶奶又說他必肯的,我就煩了人走過去試一試,誰知白討了沒趣。若論那孩子倒好,據我素日私意兒試他,他心裏沒有甚說的,只是他老子娘兩個老東西太心高了些。”一語戳動了鳳姐和賈璉,鳳姐因見賈璉在此,且不作一聲,只看賈璉的光景。賈璉心中有事,那裏把這點子事放在心裏。待要不管,只是看著他是鳳姐兒的陪房,且又素日出過力的,臉上實在過不去,因說道:“什麼大事,只管咕咕唧唧的。你放心且去,我明兒作媒打發兩個有體面的人,一面說,一面帶著定禮去,就說我的主意。他十分不依,叫他來見我。”旺兒家的看著鳳姐,鳳姐便扭嘴兒。旺兒家的會意,忙爬下就給賈璉磕頭謝恩。賈璉忙道:“你只給你姑娘磕頭。我雖如此說了這樣行,到底也得你姑娘打發個人叫他女人上來,和他好說更好些。雖然他們必依,然這事也不可霸道了。” 鳳姐忙道:“連你還這樣開恩操心呢,我倒反袖手旁觀不成。旺兒家你聽見,說了這事,你也忙忙的給我完了事來。說給你男人,外頭所有的帳,一概趕今年年底下收了進來,少一個錢我也不依的。我的名聲不好,再放一年,都要生吃了我呢。”旺兒媳婦笑道:“奶奶也太膽小了。誰敢議論奶奶,若收了時,公道說,我們倒還省些事,不大得罪人。”鳳姐冷笑道:“我也是一場癡心白使了。我真個的還等錢作什麼,不過爲的是日用出的多,進的少。這屋裏有的沒的,我和你姑爺一月的月錢,再連上四個丫頭的月錢,通共一二十兩銀子,還不夠三五天的使用呢。若不是我千湊萬挪的,早不知道到什麼破窯裏去了。如今倒落了一個放帳破落戶的名兒。\begin{note}庚雙夾:可知放帳乃發,所謂此家兒如恥惡之事也。\end{note}既這樣,我就收了回來。我比誰不會花錢,咱們以後就坐著花,到多早晚是多早晚。這不是樣兒:前兒老太太生日,太太急了兩個月,想不出法兒來,還是我提了一句,後樓上現有些沒要緊的大銅錫傢伙四五箱子,拿去弄了三百銀子,才把太太遮羞禮兒搪過去了。我是你們知道的,那一個金自鳴鐘賣了五百六十兩銀子。沒有半個月,大事小事倒有十來件,白填在裏頭。今兒外頭也短住了,不知是誰的主意,搜尋上老太太了。明兒再過一年,各人搜尋到頭面衣服,可就好了!”旺兒媳婦笑道:“那一位太太奶奶的頭面衣服折變了不夠過一輩子的,只是不肯罷了。”\begin{note}庚雙夾:閒語,補出近日諸事。\end{note}鳳姐道:“不是我說沒了能奈的話,要象這樣,我竟不能了。昨晚上忽然作了一個夢,說來也可笑,\begin{note}庚雙夾:反說“可笑”,則思□ 落套,妙甚!若必以此夢爲凶兆,非紅樓之夢矣。\end{note}夢見一個人,雖然面善,卻又不知名姓,\begin{note}庚雙夾:是以前授方相之舊,數十年後矣。\end{note}找我。問他作什麼,他說娘娘打發他來要一百匹錦。我問他是那一位娘娘,他說的又不是咱們家的娘娘。我就不肯給他,他就上來奪。正奪著,就醒了。”\begin{note}庚雙夾:妙!實家常觸景閒夢必有之理,卻是江淹才盡之兆也,可傷。\end{note}旺兒家的笑道:“這是奶奶的日間操心,常應候宮裏的事。”\begin{note}庚雙夾:淡淡抹去,妙!\end{note}
\end{parag}


\begin{parag}
    一語未了,人回:“夏太府打發了一個小內監來說話。”賈璉聽了,忙皺眉道:“又是什麼話,一年他們也搬夠了。”鳳姐道:“你藏起來,等我見他,若是小事罷了,若是大事,我自有話回他。”賈璉便躲入內套間去。這裏鳳姐命人帶進小太監來,讓他椅子上坐了喫茶,因問何事。那小太監便說:“夏爺爺因今兒偶見一所房子,如今竟短二百兩銀子,打發我來問舅奶奶家裏,有現成的銀子暫借一二百,過一兩日就送過來。”\begin{note}庚雙夾:可謂“密處不容針”。\end{note}鳳姐兒聽了,笑道:“什麼是送過來,有的是銀子,只管先兌了去。改日等我們短了,再借去也是一樣。”小太監道:“夏爺爺還說了,上兩回還有一千二百兩銀子沒送來,等今年年底下,自然一齊都送過來。”鳳姐笑道:“你夏爺爺好小氣,這也值得提在心上。我說一句話,不怕他多心,若都這樣記清了還我們,不知還了多少了。只怕沒有,若有,只管拿去。”因叫旺兒媳婦來,“出去不管那裏先支二百兩來。”旺兒媳婦會意,因笑道:“我才因別處支不動,纔來和奶奶支的。”鳳姐道:“你們只會裏頭來要錢,叫你們外頭算去就不能了。”說著叫平兒,“把我那兩個金項圈拿出去,暫且押四百兩銀子。”平兒答應了,去半日,果然拿了一個錦盒子來,裏面兩個錦袱包著。打開時,一個金累絲攢珠的,那珍珠都有蓮子大小,一個點翠嵌寶石的。兩個都與宮中之物不離上下。\begin{note}庚雙夾:是太監眼中看、心中評。\end{note}一時拿去,果然拿了四百兩銀子來。鳳姐命與小太監打疊起一半,那一半命人與了旺兒媳婦,命他拿去辦八月中秋的節。\begin{note}庚雙夾:過下伏脈。\end{note}那小太監便告辭了,鳳姐命人替他拿著銀子,送出大門去了。這裏賈璉出來笑道:“這一起外祟何日是了!”鳳姐笑道:“剛說著,就來了一股子。”賈璉道:“昨兒周太監來,張口一千兩。我略應慢了些,他就不自在。將來得罪人之處不少。這會子再發個三二百萬的財就好了。”一面說,一面平兒伏侍鳳姐另洗了面,更衣往賈母處去伺候晚飯。
\end{parag}


\begin{parag}
    這裏賈璉出來,剛至外書房,忽見林之孝走來。賈璉因問何事。林之孝說道:“方纔聽得雨村降了,卻不知因何事,只怕未必真。” 賈璉道:“真不真,他那官兒也未必保得長。將來有事,只怕未必不連累咱們,寧可疏遠著他好。”林之孝道:“何嘗不是,只是一時難以疏遠。如今東府大爺和他更好,老爺又喜歡他,時常來往,那個不知。”賈璉道:“橫豎不和他謀事,也不相干。你去再打聽真了,是爲什麼。”林之孝答應了,卻不動身,坐在下面椅子上,且說些閒話。因又說起家道艱難,便趁勢又說:“人口太重了。不如揀個空日回明老太太老爺,把這些出過力的老家人用不著的,開恩放幾家出去。一則他們各有營運,二則家裏一年也省些口糧月錢。再者裏頭的姑娘也太多。俗語說:‘一時比不得一時。’如今說不得先時的例了,少不得大家委屈些,該使八個的使六個,該使四個的便使兩個。若各房算起來,一年也可以省得許多月米月錢。況且裏頭的女孩子們一半都太大了,也該配人的配人。成了房,豈不又孳生出人來。”賈璉道:“我也這樣想著,只是老爺纔回家來,多少大事未回,那裏議到這個上頭。前兒官媒拿了個庚帖來求親,太太還說老爺纔來家,每日歡天喜地的說骨肉完聚,忽然就提起這事,恐老爺又傷心,所以且不叫提這事。”林之孝道:“這也是正理,太太想的周到。”賈璉道:“正是,提起這話我想起了一件事來。我們旺兒的小子要說太太房裏的彩霞。他昨兒求我,我想什麼大事,不管誰去說一聲去。這會子有誰閒著,我打發個人去說一聲,就說我的話。”林之孝聽了,只得應著,半晌笑道:“依我說,二爺竟別管這件事。旺兒的那小兒子雖然年輕,在外頭喫酒賭錢,無所不至。雖說都是奴才們,到底是一輩子的事。彩霞那孩子這幾年我雖沒見,聽得越發出挑的好了,何苦來白糟踏一個人。”賈璉道:“他小兒子原會喫酒,不成人?”林之孝冷笑道:“豈只喫酒賭錢,在外頭無所不爲。我們看他是奶奶的人,也只見一半不見一半罷了。”賈璉道:“我竟不知道這些事。既這樣,那裏還給他老婆,且給他一頓棍,鎖起來,再問他老子娘。”林之孝笑道:“何必在這一時。那是錯也等他再生事,我們自然回爺處治。如今且恕他。”賈璉不語,一時林之孝出去。
\end{parag}


\begin{parag}
    晚間鳳姐已命人喚了彩霞之母來說媒。那彩霞之母滿心縱不願意,見鳳姐親自和他說,何等體面,\begin{note}庚雙夾:今時人女兒因圖此現在體面誤了多少,此正是因今時女兒一笑。\end{note}便心不由意的滿口應了出去。今鳳姐問賈璉可說了沒有,賈璉因說:“我原要說的,打聽得他小兒子大不成人,故還不曾說。若果然不成人,且管教他兩日,再給他老婆不遲。”鳳姐聽說,便說:“你聽見誰說他不成人?”賈璉道:“不過是家裏的人,還有誰。”鳳姐笑道:“我們王家的人,連我還不中你們的意,何況奴才呢。我才已竟和他母親說了,他娘已經歡天喜地應了,難道又叫進他來不要了不成?”賈璉道:“既你說了,又何必退,明兒說給他老子好生管他就是了。”這裏說話不提。
\end{parag}


\begin{parag}
    且說彩霞因前日出去,等父母擇人,心中雖是與賈環有舊,尚未作準。今日又見旺兒每每來求親,早聞得旺兒之子酗酒賭博,而且容顏醜陋,一技不知,自此心中越發懊惱。生恐旺兒仗鳳姐之勢,一時作成,終身爲患,不免心中急躁。遂至晚間悄命他妹子小霞\begin{note}庚雙夾:霞大小,奇奇怪怪之文,更覺有趣。\end{note}進二門來找趙姨娘,問了端的。趙姨娘素日深與彩霞契合,巴不得與了賈環,方有個膀臂,不承望王夫人又放了出去。每唆賈環去討,一則賈環羞口難開,二則賈環也不大甚在意,不過是個丫頭,他去了,將來自然還有,\begin{note}庚雙夾:這是世人之情,亦是丈夫之情。\end{note}遂遷延住不說,意思便丟開。無奈趙姨娘又不捨,又見他妹子來問,是晚得空,便先求了賈政。\begin{note}庚雙夾:這是世人想不到之文,卻是大家必有之事。\end{note}賈政因說道:“且忙什麼,等他們再念一二年書再放人不遲。我已經看中了兩個丫頭,一個與寶玉,一個給環兒。只是年紀還小,又怕他們誤了書,所以再等一二年。”\begin{note}庚雙夾:妙文,又寫出賈老兒女之情。細思一部書總不寫賈老,則不若文,然不如此寫,則又非賈老。\end{note}趙姨娘道:“寶玉已有了二年了,老爺還不知道?”賈政聽了忙問道:“誰給的?”趙姨娘方欲說話,只聽外面一聲響,不知何物,大家吃了一驚不小。要知端的,且聽下回分解。
\end{parag}


\begin{parag}
    \begin{note}蒙回末總:夏雨冬風常不解,其何自來?何自去?鴛鴦與司棋相哭發誓,事已丸釋冰消,及平地風波一起,措手不及,亦不解何自來何自去!\end{note}
\end{parag}
