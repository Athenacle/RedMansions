\chap{五}{開生面夢演紅樓夢 立新場情傳幻境情}



\begin{parag}
    \begin{note}蒙:萬種豪華原是幻,何嘗造孽,何是風流?曲終人散有誰留,爲甚營求?只愛蠅頭!一番遭遇幾多愁,點水根由,泉湧難酬!
    \end{note}
\end{parag}


\begin{parag}
    題曰:
\end{parag}


\begin{poem}
    \begin{pl}春困葳蕤擁繡衾,恍隨仙子別紅塵。\end{pl}

    \begin{pl}問誰幻入華胥境,千古風流造孽人。\end{pl}
\end{poem}


\begin{parag}
    卻說薛家母子在榮府中寄居等事略已表明,此回則暫不能寫矣。\begin{note}甲側:此等處實又非別部小說之熟套起法。\end{note}
\end{parag}


\begin{parag}
    如今且說林黛玉\begin{note}甲眉:不敘寶釵,反仍敘黛玉。蓋前回只不過欲出寶釵,非實寫之文耳,此回若仍續寫,則將二玉高擱矣,故急轉筆仍舊至黛玉,使榮府正文方不至於冷落也。今寫黛玉神妙之至,何也?因寫黛玉實是寫寶釵,非真有意去寫黛玉,幾乎又被作者瞞過。\end{note}自在榮府以來,賈母萬般憐愛,寢食起居,一如寶玉,\begin{note}甲側:妙極!所謂一擊兩鳴法,寶玉身分可知。\end{note}迎春、探春、惜春三個親孫女倒且靠後。\begin{note}甲側:此句寫賈母。\end{note}便是寶玉和黛玉二人之親密友愛處,亦自較別個不同,\begin{note}甲側:此句妙,細思有多少文章。\end{note}日則同行同坐,夜則同息同止,真是言和意順,略無參商。不想如今忽然來了一個薛寶釵,\begin{note}甲側:總是奇峻之筆,寫來健拔,似新出一人耳。甲眉:此處如此寫寶釵,前回中略不一寫,可知前回迥非十二釵之正文也。欲出寶釵便不肯從寶釵身上寫來,卻先款款敘出二玉,陡然轉出寶釵,三人方可鼎立。行文之法又一變體。\end{note}年歲雖大不多,然品格端方,容貌豐美,人多謂黛玉所不及。\begin{note}甲側:此句定評,想世人目中各有所取也。按黛玉寶釵二人,一如姣花,一如纖柳,各極其妙者,然世人性分甘苦不同之故耳。\end{note}而且寶釵行爲豁達,隨分從時,不比黛玉孤高自許,目無下塵,故比黛玉大得下人之心。\begin{note}甲側:將兩個行止攝總一寫,實是難寫,亦實系千部小說中未敢說寫者。\end{note}便是那些小丫頭子們,亦多喜與寶釵去頑。因此黛玉心中便有些悒鬱不忿之意,\begin{note}甲側:此一句是今古才人通病,如人人皆如我黛玉之爲人,方許他妒。此是黛玉缺處。\end{note}寶釵卻渾然不覺。\begin{note}甲側:這還是天性,後文中則是又加學力了。\end{note}那寶玉亦在孩提之間,況自天性所稟來的一片愚拙偏僻,\begin{note}甲側:四字是極不好,卻是極妙。只不要被作者瞞過。\end{note}視姊妹弟兄皆出一意,並無親疏遠近之別。\begin{note}甲側:如此反謂“愚癡”,正從世人意中寫也。\end{note}其中因與黛玉同隨賈母一處坐臥,故略比別個姊妹熟慣些。既熟慣,則更覺親密,既親密,則不免一時有求全之毀,不虞之隙。\begin{note}甲側:八字定評,有趣。不獨黛玉、寶玉二人,亦可爲古今天下親密人當頭一喝。甲眉:八字爲二玉一生文字之綱。\end{note}這日不知爲何,他二人言語有些不合起來,黛玉又\begin{note}甲側:“又”字妙極!補出近日無限垂淚之事矣,此仍淡淡寫來,使後文來得不突然。\end{note}氣的獨在房中垂淚,寶玉又\begin{note}甲側:“又”字妙極!凡用二“又”字,如雙峯對峙,總補二玉正文。\end{note}自悔言語冒撞,前去俯就,那黛玉方漸漸的迴轉來。
\end{parag}


\begin{parag}
    因東邊寧府中花園內梅花盛開,\begin{note}甲側:元春消息動矣。\end{note}賈珍之妻尤氏乃治酒,請賈母、邢夫人、王夫人等賞花。是日先攜了賈蓉之妻,二人來面請。賈母等於早飯後過來,就在會芳園\begin{note}甲側:隨筆帶出,妙!字意可思。\end{note}遊頑,先茶後酒,不過皆是寧榮二府女眷家宴小集,並無別樣新文趣事可記。\begin{note}甲側:這是第一家宴,偏如此草草寫。此如晉人倒食甘蔗,漸入佳境一樣。\end{note}
\end{parag}


\begin{parag}
    一時寶玉倦怠,欲睡中覺,賈母命人好生哄著,歇一回再來。賈蓉之妻秦氏便忙笑回道:“我們這裏有給寶叔收拾下的屋子,老祖宗放心,只管交與我就是了。”又向寶玉的奶孃丫鬟等道:“嬤嬤姐姐們,請寶叔隨我這裏來。”賈母素知秦氏是個極妥當的人,\begin{note}甲側:借賈母心中定評。\end{note}生的嫋娜纖巧,行事又溫柔和平,乃重孫媳中第一個得意之人,\begin{note}甲側:又夾寫出秦氏來。\end{note}見他去安置寶玉,自是安穩的。
\end{parag}


\begin{parag}
    當下秦氏引了一簇人來至上房內間。寶玉抬頭看見一幅畫貼在上面,畫的人物固好,其故事乃是“燃藜圖”,也不看系何人所畫,心中便有些不快。\begin{note}甲眉:如此畫聯,焉能入夢?\end{note}又有一幅對聯,寫的是:
\end{parag}


\begin{poem}
    \begin{pl}世事洞明皆學問,人情練達即文章。\end{pl}
    \begin{note}甲雙夾:看此聯極俗,用於此則極妙。蓋作者正因古今王孫公子,劈頭先下金針。\end{note}
\end{poem}


\begin{parag}
    及看了這兩句,縱然室宇精美,鋪陳華麗,亦斷斷不肯在這裏了,忙說:“出去,出去!”秦氏聽了笑道:“這裏還不好,可往那裏去呢?不然往我屋裏去吧。”寶玉點頭微笑。有一個嬤嬤說道:“那裏有個叔叔往侄兒房裏睡覺的理?”秦氏笑道:“噯喲喲!不怕他惱。他能多大呢,就忌諱這些個!上月你沒看見我那個兄弟來了,\begin{note}甲眉:伏下秦鍾,妙!\end{note}雖然與寶叔同年,兩個人若站在一處,只怕那個還高些呢。”\begin{note}甲側:又伏下一人,隨筆便出,得隙便入,精細之極。\end{note}寶玉道:“我怎麼沒見過?你帶他來我瞧瞧。”\begin{note}甲側:侯門少年紈絝活跳下來。\end{note}衆人笑道:“隔著二三十里,往那裏帶去,見的日子有呢。”說著大家來至秦氏房中。剛至房門,便有一股細細的甜香\begin{note}甲側:此香名“引夢香”。\end{note}襲人而來。寶玉覺得眼餳骨軟,連說: “好香!”\begin{note}甲側:刻骨吸髓之情景,如何想得來,又如何寫得來?[進房如夢境。]\end{note}入房向壁上看時,有唐伯虎畫的《海棠春睡圖》,\begin{note}甲側:妙圖。\end{note}兩邊有宋學士秦太虛寫的一副對聯,其聯雲:
\end{parag}


\begin{poem}
    \begin{pl}嫩寒鎖夢因春冷,\end{pl}
    \begin{note}甲夾:豔極,淫極!\end{note}

    \begin{pl}芳氣籠人是酒香。\end{pl}
    \begin{note}甲夾:已入夢境矣。\end{note}

\end{poem}


\begin{parag}
    案上設著武則天當日鏡室中設的寶鏡,\begin{note}甲側:設譬調侃耳,若真以爲然,則又被作者瞞過。\end{note}一邊擺著飛燕立著舞過的金盤,盤內盛著安祿山擲過傷了太真乳的木瓜。上面設著壽昌公主於含章殿下臥的榻,懸的是同昌公主制的聯珠帳。寶玉含笑連說:“這裏好!”\begin{note}擺設就合著他的意。\end{note}秦氏笑道:“我這屋子大約神仙也可以住得了。”說著親自展開了西子浣過的紗衾,移了紅娘抱過的鴛枕,\begin{note}甲側:一路設譬之文,迥非《石頭記》大筆所屑,別有他屬,餘所不知。\end{note}於是衆奶母伏侍寶玉臥好,款款散了,只留襲人、\begin{note}甲側:一個再見。\end{note}媚人、\begin{note}甲側:二新出。\end{note}晴雯、\begin{note}甲側:三新出,名妙而文。\end{note}麝月\begin{note}甲側:四新出,尤妙。看此四婢之名,則知歷來小說難與並肩。\end{note}四個丫鬟爲伴。\begin{note}甲眉:文至此不知從何處想來。\end{note}秦氏便分咐小丫鬟們,好生在廊檐下看著貓兒狗兒打架。\begin{note}甲側:細極。\end{note}
\end{parag}


\begin{parag}
    那寶玉剛合上眼,便惚惚的睡去,猶似秦氏在前,遂悠悠盪盪,隨了秦氏,至一所在。\begin{note}甲側:此夢文情固佳,然必用秦氏引夢,又用秦氏出夢,竟不知立意何屬?惟批書人知之。\end{note}但見朱欄白石,綠樹清溪,真是人跡希逢,飛塵不到。\begin{note}甲側:一篇《蓬萊賦》。\end{note}寶玉在夢中歡喜,想道:“這個去處有趣,我就在這裏過一生,縱然失了家也願意,強如天天被父母師傅打呢。”\begin{note}甲側:一句忙裏點出小兒心性。\end{note}正胡思之間,忽聽山後有人作歌曰:
\end{parag}


\begin{poem}
    \begin{pl}春夢隨雲散,\end{pl}
    \begin{note}甲雙夾:開口拿“春”字,最緊要!\end{note}

    \begin{pl}飛花逐水流。\end{pl}\begin{note}甲夾:二句比也。\end{note}

    \begin{pl}寄言衆兒女,何必覓閒愁。\end{pl}
    \begin{note}甲夾:將通部人一喝。\end{note}
\end{poem}


\begin{parag}
    寶玉聽了是女子的聲音。\begin{note}甲側:寫出終日與女兒廝混最熟。\end{note}歌聲未息,早見那邊走出一個人來,蹁躚嫋娜,端的與人不同。有賦爲證:
\end{parag}


\begin{qute2sp}
    \textbf{
        方離柳塢,乍出花房。但行處,鳥驚庭樹;將到時,影度迴廊。仙袂乍飄兮,聞麝蘭之馥郁;荷衣欲動兮,聽環佩之鏗鏘。靨笑春桃兮,雲堆翠髻;脣綻櫻顆兮,榴齒含香。纖腰之楚楚兮,迴風舞雪;珠翠之輝輝兮,滿額鵝黃。出沒花間兮,宜嗔宜喜;徘徊池上兮,若飛若揚。蛾眉顰笑兮,將言而未語;蓮步乍移兮,待止而欲行。羨彼之良質兮,冰清玉潤;羨彼之華服兮,閃灼文章;愛彼之貌容兮,香培玉琢;美彼之態度兮,鳳翥龍翔。其素若何?春梅綻雪。其潔若何?秋菊被霜。其靜若何?松生空谷。其豔若何?霞映澄塘。其文若何?龍游曲沼。其神若何?月射寒江。應慚西子,實愧王嬙。奇矣哉,生於孰地,來自何方?信矣乎,瑤池不二,紫府無雙。果何人哉?如斯之美也!}
    \begin{note}甲眉:按此書凡例本無贊賦閒文,前有寶玉二詞,今復見此一賦,何也?蓋此二人乃通部大綱,不得不用此套。前詞卻是作者別有深意,故見其妙。此賦則不見長,然亦不可無者也。\end{note}

\end{qute2sp}


\begin{parag}
    寶玉見是一個仙姑,喜的忙來作揖問道:“神仙姐姐,\begin{note}甲側:千古未聞之奇稱,寫來竟成千古未聞之奇語。故是千古未有之奇文。\end{note}不知從那裏來,如今要往那裏去?也不知這是何處,望乞攜帶攜帶。”那仙姑笑道:“吾居離恨天之上,灌愁海之中,乃放春山遣香洞太虛幻境警幻仙姑是也。\begin{note}甲側:與首回中甄士隱夢景一照。\end{note}司人間之風情月債,掌塵世之女怨男癡。因近來風流冤孽,\begin{note}甲側:四字可畏。\end{note}纏綿於此處,是以前來訪察機會,佈散相思。今忽與爾相逢,亦非偶然。此離吾境不遠,別無他物,僅有自採仙茗一盞,親釀美酒一甕,素練魔舞歌姬數人,新填《紅樓夢》\begin{note}甲側:點題。蓋作者自雲所歷不過紅樓一夢耳。\end{note}仙曲十二支,試隨吾一遊否?”寶玉聽說,便忘了秦氏在何處,\begin{note}甲側:細極。\end{note}竟隨了仙姑,至一所在,有石牌橫建,上書“太虛幻境”四個大字,兩邊一副對聯,\begin{note}甲側:士隱曾見此匾對,而僧道不能領入,留此回警幻邀寶玉後文。\end{note}乃是:
\end{parag}


\begin{poem}
    \begin{pl}假作真時真亦假,無爲有處有還無。\end{pl}\begin{note}甲雙:正恐觀者忘卻首回,故特將甄士隱夢景重一滃染。\end{note}
\end{poem}


\begin{parag}
    轉過牌坊,便是一座宮門,上面橫書四個大字,道是“孽海情天”。又有一副對聯,大書雲:
\end{parag}


\begin{poem}
    \begin{pl}厚地高天,堪嘆古今情不盡;癡男怨女,可憐風月債難償\end{pl}
\end{poem}


\begin{parag}
    寶玉看了,\begin{note}甲眉:菩薩天尊皆因僧道而有,以點俗人,獨不許幻造太虛幻境以警情者乎?觀者惡其荒唐,餘則喜其新鮮。有修廟造塔祈福者,餘今意欲起太虛幻境以較修七十二司更有功德。\end{note}心下自思道:“原來如此。但不知何爲古今之情,何爲風月之債?從今倒要領略領略。”寶玉只顧如此一想,不料早把些邪魔招入膏肓了。\begin{note}甲側:奇極妙文。\end{note}當下隨了仙姑進入二層門內,至兩邊配殿,皆有匾額對聯,一時看不盡許多,惟見有幾處寫的是:“癡情司”、“結怨司”、“朝啼司”、“夜怨司”、“春感司”、“秋悲司”。\begin{note}甲側:虛陪六個。\end{note}看了,因向仙姑道:“敢煩仙姑引我到那各司中游玩遊玩,不知可使得?”仙姑道:“此各司中皆貯的是普天之下所有的女子過去未來的簿冊。爾凡眼塵軀,未便先知的。”寶玉聽了,那裏肯依,復央之再四。仙姑無奈,說: “也罷,就在此司內略隨喜隨喜罷了。”寶玉喜不自勝,抬頭看這司的匾上,乃是“薄命司”\begin{note}甲側:正文。\end{note}三字,兩邊對聯寫的是:
\end{parag}


\begin{poem}\begin{pl}春恨秋悲皆自惹,花容月貌爲誰妍。\end{pl}\end{poem}


\begin{parag}
    寶玉看了,便知\begin{note}甲側:便知二字是字法,最爲緊要之至。\end{note}感嘆。進入門來,只見有數十個大廚,皆用封條封着。看那封條上,皆是各省地名。寶玉一心只揀自己的家鄉的封條看,遂無心看別省的了。只見那邊廚上封條上大書七字雲:金陵十二釵正冊。\begin{note}甲側:正文題。\end{note}寶玉因問:“何爲金陵十二釵正冊?”警幻道:“即貴省中十二冠首女子之冊,故爲正冊。”寶玉道:“常聽\begin{note}甲側:常聽二字,神理極妙。\end{note}人說,金陵極大,怎麼只十二個女子?如今單我們家裏,上上下下就有幾百女孩子呢。”\begin{note}甲側:貴公子口聲。\end{note}警幻冷笑道:“貴省女子固多,不過擇其緊要者錄之。下邊二廚則又次之。餘者庸愚之輩,則無冊可錄矣。”寶玉聽說,再看下首二廚上,果然一個寫着金陵十二釵副冊,又一個寫着金陵十二釵又副冊。寶玉便伸手先將又副冊廚開了,拿出一本冊來,揭開一看,只見這首頁上畫着一副畫,又非人物,亦無山水,不過水墨滃染的滿紙烏雲濁霧而已。後有幾行字,寫的是:
\end{parag}


\begin{poem}
    \begin{pl}霽月難逢,彩雲易散。\end{pl}

    \begin{pl}心比天高,身爲下賤。\end{pl}

    \begin{pl}風流靈巧招人怨。\end{pl}

    \begin{pl}壽夭多因誹謗生,多情公子空牽念。\end{pl}
    \begin{note}甲雙:恰極之至!「病補雀金裘」回中與此合看。\end{note}

\end{poem}


\begin{parag}
    寶玉看了,又見後面畫着一簇鮮花,一牀破席。也有幾句言詞,寫道是:
\end{parag}


\begin{poem}
    \begin{pl}枉自溫柔和順,空雲似桂如蘭。\end{pl}

    \begin{pl}堪羨優伶有福,誰知公子無緣。\end{pl}
    \begin{note}甲雙:罵死寶玉,卻是自悔。\end{note}

\end{poem}


\begin{parag}
    寶玉看了不解。遂擲下這個,又去開了副冊廚門,拿起一本冊來,揭開看時,只見畫着一株桂花,下面有一池沼,其中水涸泥幹,蓮枯藕敗。畫後書雲:
\end{parag}


\begin{poem}
    \begin{pl}根並荷花一莖香,\end{pl}
    \begin{note}甲雙:卻是詠菱妙句。\end{note}

    \begin{pl}平生遭際實堪傷。\end{pl}

    \begin{pl}自從兩地生孤木,\end{pl}
    \begin{note}甲夾:折(拆)字法。\end{note}

    \begin{pl}致使香魂返故鄉。\end{pl}
\end{poem}


\begin{parag}
    寶玉看了仍不解他。又擲下,再去取正冊看。只見頭一頁上便畫着兩株枯木,木上懸着一圍玉帶,又有一堆雪,雪下一股金簪。也有四句言詞道:
\end{parag}


\begin{poem}
    \begin{pl}可嘆停機德,\end{pl}\begin{note}甲夾:此句薛。\end{note}

    \begin{pl}堪憐詠絮才。\end{pl}\begin{note}甲夾:此句林。\end{note}

    \begin{pl}玉帶林中掛,\end{pl}

    \begin{pl}金簪雪裏埋。\end{pl}\begin{note}甲雙:寓意深遠,皆非生其地之意。\end{note}

\end{poem}


\begin{parag}
    寶玉看了仍不解。待要問時,情知他必不肯泄漏;待要丟下,又不捨。遂又往後看時,只見畫著一張弓,弓上掛一香櫞。也有一首歌詞雲:\begin{note}甲眉:世之好事者爭傳《推背圖》之說,想前人斷不肯煽惑愚迷,即有此說,亦非常人供談之物。此回悉借其法,爲衆女子數運之機。無可以供茶酒之物,亦無干涉政事,真奇想奇筆。\end{note}
\end{parag}


\begin{poem}
    \begin{pl}二十年來辨是誰,\end{pl}

    \begin{pl}榴花開處照宮闈;\end{pl}

    \begin{pl}三春爭及初春景,\end{pl}\begin{note}甲夾:顯極。\end{note}

    \begin{pl}虎兕相逢大夢歸。\end{pl}
\end{poem}


\begin{parag}
    後面又畫著兩人放風箏,一片大海,一隻大船,船中有一女子掩面泣涕之狀。也有四句寫雲:
\end{parag}


\begin{poem}
    \begin{pl}才自精明志自高,生於末世運偏消。\end{pl}
    \begin{note}甲雙:感嘆句,自寓。\end{note}

    \begin{pl}清明涕送江邊艦,千里東風一望遙。\end{pl}
    \begin{note}甲夾:好句!\end{note}
\end{poem}


\begin{parag}
    後面又畫幾縷飛雲,一灣逝水。其詞曰:
\end{parag}


\begin{poem}
    \begin{pl}富貴又何爲?襁褓之間父母違;\end{pl}

    \begin{pl}展眼吊斜暉,湘江水逝楚雲飛。\end{pl}

\end{poem}


\begin{parag}
    後面又畫著一塊美玉,落在泥垢之中。其斷語云:
\end{parag}


\begin{poem}
    \begin{pl}欲潔何曾潔,雲空未必空!\end{pl}

    \begin{pl}可憐金玉質,落陷污泥中。\end{pl}
\end{poem}


\begin{parag}
    後面忽見畫著個惡狼,追撲一美女,欲啖之意。其書雲:
\end{parag}


\begin{poem}
    \begin{pl}子系中山狼,\end{pl}

    \begin{pl}得志便猖狂。\end{pl}\begin{note}甲夾:好句。\end{note}

    \begin{pl}金閨花柳質,\end{pl}

    \begin{pl}一載赴黃梁。\end{pl}

\end{poem}


\begin{parag}
    後面便是一所古廟,裏面有一美人在內看經獨坐。其判雲:
\end{parag}


\begin{poem}
    \begin{pl}勘破三春景不長,緇衣頓改昔年妝。\end{pl}

    \begin{pl}可憐繡戶侯門女,獨臥青燈古佛傍。\end{pl}
    \begin{note}甲夾:好句。\end{note}
\end{poem}


\begin{parag}
    後面便是一片冰山,上面有一隻雌鳳。其判曰:
\end{parag}


\begin{poem}
    \begin{pl}凡鳥偏從末世來,\end{pl}

    \begin{pl}都知愛慕此身才。\end{pl}

    \begin{pl}一從二令三人木,\end{pl}\begin{note}甲夾:拆字法。\end{note}

    \begin{pl}哭向金陵事更哀。\end{pl}
\end{poem}


\begin{parag}
    後面又是一座荒村野店,有一美人在那裏紡績。其判雲:
\end{parag}


\begin{poem}
    \begin{pl}事敗休雲貴,\end{pl}

    \begin{pl}家亡莫論親。\end{pl}\begin{note}甲雙:非經歷過者,此二句則雲紙上談兵。過來人那得不哭!\end{note}

    \begin{pl}偶因濟劉氏,\end{pl}

    \begin{pl}巧得遇恩人。\end{pl}

\end{poem}


\begin{parag}
    後面又畫著一盆茂蘭,旁有一位鳳冠霞帔的美人。也有判雲:
\end{parag}


\begin{poem}
    \begin{pl}桃李春風結子完,到頭誰似一盆蘭?\end{pl}

    \begin{pl}爲冰爲水空相妒,枉與他人作話談。\end{pl}
    \begin{note}甲雙:真心實語。\end{note}

\end{poem}


\begin{parag}
    後面又畫著高樓大廈,有一美人懸樑自縊。其判雲:
\end{parag}


\begin{poem}
    \begin{pl}情天情海幻情身,情既相逢必主淫。\end{pl}

    \begin{pl}謾言不肖皆榮出,造釁開端實在寧。\end{pl}
\end{poem}


\begin{parag}
    寶玉還欲看時,那仙姑知他天分高明,性情穎慧,\begin{note}甲眉:通部中筆筆貶寶玉,人人嘲寶玉,語語謗寶玉,今卻於警幻意中忽寫出此八字來,真是意外之意。此法亦別書中所無。\end{note}恐把仙機泄漏,遂掩了卷冊,笑向寶玉道:“且隨我去遊玩奇景,\begin{note}甲側:是哄小兒語,細甚。\end{note}何必在此打這悶葫蘆!”\begin{note}甲側:爲前文“葫蘆廟”一點。\end{note}
\end{parag}


\begin{parag}
    寶玉恍恍惚惚,不覺棄了卷冊,\begin{note}甲側:是夢中景況,細極。\end{note}又隨了警幻來至後面。但見珠簾繡幕,畫棟雕檐,說不盡那光搖朱戶金鋪地,雪照瓊窗玉作宮。更見仙花馥郁,異草芬芳,真好個所在。\begin{note}甲側:已爲省親別墅畫下圖式矣。\end{note}又聽警幻笑道:“你們快出來迎接貴客!”一語未了,只見房中又走出幾個仙子來,皆是荷袂蹁躚,羽衣飄舞,姣若春花,媚如秋月。一見了寶玉,都怨謗警幻道:“我們不知系何‘貴客’,忙的接了出來!姐姐曾說今日今時必有絳珠妹子\begin{note}甲側:絳珠爲誰氏?請觀者細思首回。\end{note}的生魂前來遊玩,故我等久待。何故反引這濁物來污染這清淨女兒之境?”\begin{note}甲眉:奇筆攄奇文。作書者視女兒珍貴之至,不知今時女兒可知?餘爲作者癡心一哭,又爲近之自棄自敗之女兒一恨。\end{note}寶玉聽如此說,便嚇得欲退不能退,\begin{note}甲側:貴公子不怒而反退,卻是寶玉天分中一段情癡。\end{note}果覺自形污穢不堪。警幻忙攜住寶玉的手,\begin{note}甲側:妙!警幻自是個多情種子。\end{note}向衆姊妹道:“你等不知原委:今日原欲往榮府去接絳珠,適從寧府所過,偶遇寧榮二公之靈,囑吾雲:‘吾家自國朝定鼎以來,功名奕世,富貴傳流,雖歷百年,奈運終數盡,不可挽回者。故遺之子孫雖多,竟無可以繼業。\begin{note}甲側:這是作者真正一把眼淚。\end{note}其中惟嫡孫寶玉一人,稟性乖張,生性怪譎,雖聰明靈慧,略可望成,無奈吾家運數合終,恐無人規引入正。幸仙姑偶來,萬望先以情慾聲色等事警其癡頑,\begin{note}甲側:二公真無可奈何,開一覺世覺人之路也。\end{note}或能使彼跳出迷人圈子,然後入於正路,亦吾兄弟之幸矣。’如此囑吾,故發慈心,引彼至此。先以彼家上中下三等女子之終身冊籍,令彼熟玩,尚未覺悟。故引彼再至此處,令其再歷飲饌聲色之幻,或冀將來一悟,亦未可知也。”\begin{note}甲側:一段敘出寧、榮二公,足見作者深意。\end{note}
\end{parag}


\begin{parag}
    說畢,攜了寶玉入室。但聞一縷幽香,竟不知其所焚何物。寶玉遂不禁相問,警幻冷笑道:“此香塵世中既無,爾何能知!此香乃系諸名山勝境內初生異卉之精,合各種寶林珠樹之油所制,名‘羣芳髓’。”\begin{note}甲側:好香!\end{note}寶玉聽了,自是羨慕而已。大家入座,小丫鬟捧上茶來。寶玉自覺清香異味,純美非常,因又問何名。警幻道:“此茶出在放春山遣香洞,又以仙花靈葉上所帶之宿露而烹。此茶名曰‘千紅一窟’。”\begin{note}甲側:隱“哭”字。\end{note}寶玉聽了,點頭稱賞。因看房內,瑤琴、寶鼎、古畫、新詩,無所不有,更喜窗下亦有唾絨,奩間時漬粉污。壁上也見懸著一副對聯,書雲:
\end{parag}


\begin{poem}
    \begin{pl}幽微靈秀地,\end{pl}\begin{note}甲雙:女兒之心,女兒之境。\end{note}

    \begin{pl}無可奈何天。\end{pl}\begin{note}甲雙:兩句盡矣。撰通部大書不難,最難是此等處,可知皆從無可奈何而有。\end{note}
\end{poem}


\begin{parag}
    寶玉看畢,無不羨慕。因又請問衆仙姑姓名:一名癡夢仙姑,一名鍾情大士,一名引愁金女,一名度恨菩提,各各道號不一。少刻,有小丫鬟來調桌安椅,設擺酒饌。真是:
\end{parag}


\begin{poem}
    \begin{pl} 瓊漿滿泛玻璃盞,玉液濃斟琥珀杯。\end{pl}
\end{poem}


\begin{parag}
    更不用再說那餚饌之盛。寶玉因聞得此酒清香甘冽,異乎尋常,又不禁相問。警幻道:“此酒乃以百花之蕊,萬木之汁,加以麟髓之醅,鳳乳之曲釀成,因名爲‘萬豔同杯’。”\begin{note}甲側:與千紅一窟一對,隱悲字。\end{note}寶玉稱賞不迭。
\end{parag}


\begin{parag}
    飲酒間,又有十二個舞女上來,請問演何詞曲。警幻道:“就將新制《紅樓夢》十二支演上來。”舞女們答應了,便輕敲檀板,款按銀箏。聽他歌道是:
\end{parag}


\begin{qute2sp}

    開闢鴻蒙……\begin{note}甲夾:故作頓挫搖擺。\end{note}
\end{qute2sp}


\begin{parag}
    方歌了一句,警幻便說道:“此曲不比塵世中所填傳奇之曲,必有生旦淨末之則,又有南北九宮之限。此或詠歎一人,或感懷一事,偶成一曲,即可譜入管絃。若非箇中人,\begin{note}甲側:三字要緊。不知誰是個中人。寶玉即箇中人乎?然則石頭亦箇中人乎?作者亦繫個中人乎?觀者亦箇中人乎?\end{note}不知其中之妙。料爾亦未必深明此調,若不先閱其稿,後聽其歌,翻成嚼蠟矣。”\begin{note}甲眉:警幻是個極會看戲人。近之大老觀戲,必先翻閱角本。目睹其詞,耳聽彼歌,卻從警幻處學來。\end{note}說畢,回頭命小丫鬟取了《紅樓夢》原稿來,遞與寶玉。寶玉接來,一面目視其文,一面耳聆其歌曰:\begin{note}甲眉:作者能處,慣於自站地步,又慣於陡起波瀾,又慣於故爲曲折,最是行文祕訣。\end{note}
\end{parag}


\begin{qute2sp}
    \song{紅樓夢 引子}
    開闢鴻蒙,誰爲情種?\begin{note}甲側:非作者爲誰。餘又曰:“亦非作者,乃石頭耳。”\end{note}都只爲風月情濃。趁着這奈何天、傷懷日、寂寞時,試遣愚衷\begin{note}甲側:愚字自謙得妙。\end{note}。因此上,演出這懷金悼玉的《紅樓夢》。\begin{note}甲雙:讀此幾句,翻厭近之傳奇中必用開場副末等套,累贅太甚。甲眉:懷金悼玉,大有深意。\end{note}
\end{qute2sp}


\begin{qute2sp}
    \song{終身誤}
    都道是金玉良姻,俺只念木石前盟。空對著,山中高士晶瑩雪;終不忘,世外仙姝寂寞林。嘆人間,美中不足今方信。縱然是齊眉舉案,到底意難平。\begin{note}甲眉:語句潑撒,不負自創北曲。\end{note}
\end{qute2sp}


\begin{qute2sp}
    \song{枉凝眉}
    一個是閬苑仙葩,一個是美玉無瑕。若說沒奇緣,今生偏又遇着他;若說有奇緣,如何心事終須化!一個枉自嗟呀,一個空勞牽掛。一個是水中月,一個是鏡中花。想眼中,能有多少淚珠兒,怎經得,秋流到冬盡春流到夏。
\end{qute2sp}


\begin{parag}
    寶玉聽了此曲,散漫無稽,不見得好處,\begin{note}甲側:自批駁,妙極!\end{note}但其聲韻悽惋,竟能銷魂醉魄。因此也不察其原委,問其來歷,就暫以此釋悶而已。\begin{note}甲眉:妙!設言世人亦應如此法看此《紅樓夢》一書,更不必追究其隱寓。\end{note}因又看下道:
\end{parag}


\begin{qute2sp}
    \song{恨無常}
    喜榮華正好,恨無常又到。眼睜睜,把萬事全拋;盪悠悠,把芳魂消耗。望家鄉,路遠山遙。故向爹孃夢裏相尋告:兒命已入黃泉,天倫呵,須要退步抽身早。\begin{note}甲夾:悲險之至!\end{note}
\end{qute2sp}


\begin{qute2sp}
    \song{分骨肉}
    一帆風雨路三千,把骨肉家園齊來拋閃。恐哭損殘年。告爹孃,莫把兒懸念。自古窮通皆有命,離合豈無緣。從今分兩地,各自保平安。奴去也,莫牽連。
\end{qute2sp}


\begin{qute2sp}
    \song{樂中悲}
    襁褓中,父母嘆雙亡。\begin{note}甲側:意真辭切,過來人見之不免失聲。\end{note}縱居那綺羅叢,誰知嬌養?幸生來,英雄闊大寬宏量,從未將兒女私情略縈心上。好一似,霽月光風耀玉堂。廝配得才貌仙郎,博得個地久天長,准折得幼年時坎坷形狀。終久是雲散高唐,水涸湘江。這是塵寰中消長數應當,何必枉悲傷!\begin{note}甲眉:悲壯之極,北曲中不能多得。\end{note}
\end{qute2sp}


\begin{qute2sp}
    \song{世難容}
    氣質美如蘭,才華阜比仙。\begin{note}甲側:妙卿實當得起。\end{note}天生成孤癖人皆罕。你道是啖肉食腥羶,\begin{note}甲側:絕妙!曲文填詞中不能多見。\end{note}視綺羅俗厭;卻不知太高人愈妒,過潔世同嫌。\begin{note}甲夾:至語。\end{note}可嘆這,青燈古殿人將老;辜負了,紅粉朱樓春色闌。到頭來,依舊是風塵骯髒違心願;好一似,無瑕美玉遭泥陷。又何須,王孫公子嘆無緣。
\end{qute2sp}


\begin{qute2sp}
    \song{喜冤家}
    中山狼,無情獸,全不念當日根由。一味的,驕奢淫蕩貪還構。覷著那,侯門豔質同蒲柳;作踐的,公府千金似下流。嘆芳魂豔魄,一載盪悠悠。\begin{note}甲雙:題只十二釵,卻無人不有,無事不備。\end{note}
\end{qute2sp}


\begin{qute2sp}
    \song{虛花悟}
    將那三春看破,桃紅柳綠待如何?把這韶華打滅,覓那情淡天和。說什麼,天上夭桃盛,雲中杏蕊多!到頭來,誰見把秋捱過?則看那,白楊村裏人嗚咽,青楓林下鬼吟哦。更兼着,連天衰草遮墳墓。這的是,昨貧今富人勞碌,春榮秋謝花折磨。似這般,生關死劫誰能躲?聞道說,西方寶樹喚婆娑,上結著長生果。\begin{note}甲夾:末句、開句、收句。\end{note}
\end{qute2sp}


\begin{qute2sp}
    \song{聰明累}
    機關算盡太聰明,反算了卿卿性命。\begin{note}甲側:警拔之句。\end{note}生前心已碎,死後性靈空。家富人寧,終有個,家亡人散各奔騰。枉費了,意憖憖半世心;好一似,盪悠悠三更夢。\begin{note}甲眉:過來人睹此,寧不放聲一哭?\end{note}忽喇喇如大廈傾,昏慘慘似燈將盡。呀!一場歡喜忽悲辛。嘆人世,終難定!\begin{note}甲夾:見得到。\end{note}
\end{qute2sp}


\begin{qute2sp}
    \song{留餘慶}
    留餘慶,留餘慶,忽遇恩人;幸孃親,幸孃親,積得陰功。勸人生,濟困扶窮,休似俺那銀錢上,忘骨肉的狠舅奸兄!正是乘除加減,上有蒼穹。
\end{qute2sp}


\begin{qute2sp}
    \song{晚韶華}
    鏡裏恩情,\begin{note}甲夾:起得妙!\end{note}更那堪夢裏功名!那美韶華去之何迅!再休提繡帳鴛衾。只這戴珠冠,披鳳襖,也抵不了無常性命。雖說是,人生莫受老來貧,也須要陰騭積兒孫。氣昂昂頭戴簪纓,光閃閃腰懸金印;威赫赫爵位高登,昏慘慘黃泉路近。問古來將相可還存?也只是虛名兒與後人欽敬。
\end{qute2sp}


\begin{qute2sp}
    \song{好事終}
    畫梁春盡落香塵。\begin{note}甲側:六朝妙句。\end{note}擅風情,秉月貌,便是敗家的根本。箕裘頹墮皆從敬,\begin{note}甲側:深意他人不解。\end{note}家事消亡首罪寧。宿孽總因情。\begin{note}甲雙:是作者具菩薩之心,秉刀斧之筆,撰成此書,一字不可更,一語不可少。\end{note}
\end{qute2sp}


\begin{qute2sp}
    \song{收尾 飛鳥各投林}\begin{note}甲雙:收尾愈覺悲慘可畏。\end{note}
    爲官的,家業凋零;富貴的,金銀散盡。\begin{note}甲側:二句先總寧榮。\end{note}有恩的,死裏逃生;無情的,分明照應。欠命的,命已還;欠淚的,淚已盡。冤冤相報實非輕,分離合聚皆前定。欲知命短問前生,老來富貴也真僥倖。看破的,遁入空門;癡迷的,枉送了性命。\begin{note}甲側:將通部女子一總。\end{note}好一似食盡鳥投林,落了片白茫茫大地真乾淨!\begin{note}甲夾:又照看葫蘆廟。與樹倒猢猻散反照。\end{note}
\end{qute2sp}


\begin{parag}
    歌畢,還又歌別曲。\begin{note}甲側:是極!香菱、晴雯輩豈可無,亦不必再。\end{note}警幻見寶玉甚無趣味,因嘆:“癡兒竟尚未悟!”那寶玉忙止歌姬不必再曲,自覺朦朧恍惚,告醉求臥。警幻便命撤去殘席,送寶玉至一香閨繡閣之中,其間鋪陳之盛,乃素所未見之物。更可駭者,早有一位女子在內,其鮮豔嫵媚,有似乎寶釵,風流嫋娜,則又如黛玉。\begin{note}甲側:難得雙兼,妙極!\end{note}正不知何意。忽警幻道:“塵世中多少富貴之家,那些綠窗風月,繡閣煙霞,皆被淫污紈絝與那些流蕩女子悉皆玷辱。\begin{note}甲側:真極!\end{note}更可恨者,自古來多少輕薄浪子,皆以好色不淫爲飾,又以情而不淫作案,此皆飾非掩醜之語也。好色即淫,知情更淫。是以巫山之會,雲雨之歡,皆由既悅其色,復戀其情所致也。\begin{note}甲側:“好色而不淫”,今翻案,奇甚!\end{note}吾所愛汝者,乃天下古今第一淫人也。”\begin{note}甲側:多大膽量敢作如此之文!甲眉:絳芸軒中諸事情景由此而生。\end{note}寶玉聽了,唬的忙答道:“仙姑差了。我因懶於讀書,家父母尚每垂訓飭,豈敢再冒淫字?況且年紀尚小。不知淫字爲何物。”警幻道:“非也。淫雖一理。意則有別。如世之好淫者,不過悅容貌,喜歌舞,調笑無厭,雲雨無時,恨不能盡天下之美女供我片時之趣興,\begin{note}甲側:說得懇切恰當之至!\end{note}此皆皮膚淫濫之蠢物耳。如爾則天分中生成一段癡情,吾輩推之爲‘意淫’。\begin{note}甲側:二字新雅。\end{note}‘意淫’二字,惟心會而不可口傳,可神通而不能語達。\begin{note}甲側:按寶玉一生心性,只不過是體貼二字,故曰意淫。\end{note}汝今獨得此二字,在閨闥中,固可爲良友,然於世道中未免迂闊怪詭,百口嘲謗,萬目睚眥。今既遇令祖寧榮二公剖腹深囑,吾不忍君獨爲我閨閣增光,見棄於世道,是特引前來,醉以靈酒,沁以仙茗,警以妙曲,再將吾妹一人,乳名兼美\begin{note}甲側:妙!蓋指薛林而言也。\end{note}字可卿者,許配於汝。今夕良時,即可成姻。不過領汝領略此仙閨幻境之風光,尚然如此,何況塵境之情哉?今而後萬萬解釋,改悟前情,將謹勤有用的工夫,置身於經濟之道。”說畢便祕授以雲雨之事,推寶玉入帳。那寶玉恍恍惚惚,依警幻所囑之言,未免有陽臺巫峽之會。數日來,柔情綣繾,軟語溫存,與可卿難解難分。
\end{parag}


\begin{parag}
    那日,警幻攜寶玉、可卿閒遊至一個所在,但見荊榛遍地,狼虎同羣,忽爾大河阻路,黑水淌洋,又無橋樑可通。\begin{note}甲側:若有橋樑可通,則世路人情猶不算艱難。\end{note}寶玉正自徬徨,只聽警幻道:“寶玉再休前進,作速回頭要緊!”\begin{note}甲側:機鋒。點醒世人。\end{note}寶玉忙止步問道:“此係何處?”警幻道:“此即迷津也。深有萬丈,遙亙千里,中無舟楫可通,只有一個木筏,乃木居士掌舵,灰侍者撐篙,不受金銀之謝,但遇有緣者渡之。爾今偶遊至此,如墮落其中,則深負我從前一番以情悟道、守理衷情之言。”寶玉方欲回言,只聽迷津內水響如雷,竟有一夜叉般怪物攛出,直撲而來。嚇得寶玉汗下如雨,一面失聲喊叫:“可卿救我!可卿救我!”慌得襲人、媚人等上來扶起,拉手說:“寶玉別怕,我們在這裏!”秦氏在外聽見,連忙進來,一面說ㄚ鬟們好生看着貓兒狗兒打架,又聞寶玉口中連叫可卿救我,\begin{note}甲側:雲龍作雨,不知何爲龍,何爲雲,何爲雨。\end{note}因納悶道:“我的小名,這裏沒人知道,他如何從夢裏叫出來?”正是:
\end{parag}


\begin{poem}
    \begin{pl}一場幽夢同誰訴,千古情人獨我知。\end{pl}
\end{poem}

