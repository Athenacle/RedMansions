\chap{六}{贾宝玉初试云雨情 刘姥姥一进荣国府}


\begin{parag}
    \begin{note}甲:宝玉、袭人亦大家常事耳,写得是已全领警幻意淫之训。此回借刘妪,却是写阿凤正传,并非泛文,且伏“二进”“三进”及巧姐之归著。\end{note}
\end{parag}


\begin{parag}
    \begin{note}此回刘妪一进荣国府,用周瑞家的,又过下回无痕,是无一笔写一人文字之笔。\end{note}
\end{parag}


\begin{parag}
    \begin{note}蒙:风流真假一般看,借贷亲疏触眼酸。总是幻情无了处,银灯挑尽泪漫漫。\end{note}
\end{parag}


\begin{parag}
    题曰:
\end{parag}


\begin{poem}
    \begin{pl}朝叩富儿门,富儿犹未足。\end{pl}

    \begin{pl}虽无千金酬,嗟彼胜骨肉。\end{pl}
\end{poem}


\begin{parag}
    却说秦氏因听见宝玉从梦中唤他的乳名,心中自是纳闷,又不好细问。彼时宝玉迷迷惑惑,若有所失。众人忙端上桂圆汤来,呷了两口,遂起身整衣。袭人伸手与他系裤带时,不觉伸手至大腿处,只觉冰凉一片沾湿。唬的忙退出手来,问是怎么了。宝玉红涨了脸,把他的手一捻。袭人本是个聪明女子,年纪本又比宝玉大两岁,近来也渐通人事,今见宝玉如此光景,心中便觉察一半了,不觉也羞的红涨了脸面,不敢再问。仍旧理好衣裳,遂至贾母处来,胡乱吃毕了晚饭,过这边来。袭人忙趁众奶娘丫鬟不在旁时,另取出一件中衣来与宝玉换上。宝玉含羞央告道:“好姐姐,千万别告诉人。”袭人亦含羞笑问道:“你梦见什么故事了?是那里流出来的那些脏东西?”宝玉道:“一言难尽。”说著便把梦中之事细说与袭人听了,然后说至警幻所授云雨之情,羞的袭人掩面伏身而笑。宝玉亦素喜袭人柔媚娇俏,遂强袭人同领警幻所训云雨之事。\begin{note}甲侧:数句文完一回提纲文字。\end{note}袭人素知贾母已将自己与了宝玉的,今便如此,亦不为越礼,\begin{note}甲双夹:写出袭人身份。\end{note}遂和宝玉偷试一番,幸得无人撞见。自此宝玉视袭人更比别个不同,\begin{note}甲双夹:伏下晴雯。\end{note}袭人待宝玉更为尽心。\begin{note}甲双夹:一段小儿女之态,可谓追魂摄魄之笔。\end{note}暂且别无话说。\begin{note}甲双夹:一句接住上回“红楼梦”大篇文字,另起本回正文。\end{note}
\end{parag}


\begin{parag}
    按荣府中一宅人合算起来,人口虽不多,从上至下也有三四百丁,虽事不多,一天也有一二十件,竟如乱麻一般,并无个头绪可作纲领。正寻思从那一件事自那一个人写起方妙,恰好忽从千里之外,芥豆之微,小小一个人家,因与荣府略有些瓜葛,\begin{note}甲侧:略有些瓜葛,是数十回后之正脉也。真千里伏线。 \end{note}这日正往荣府中来,因此便就此一家说来,倒还是头绪。你道这一家姓甚名谁,又与荣府有甚瓜葛?诸公若嫌琐碎粗鄙呢,则快掷下此书,另觅好书去醒目;若谓聊可破闷时,待蠢物\begin{note}甲双夹:妙谦,是石头口角。\end{note}逐细言来。
\end{parag}


\begin{parag}
    方才所说的这小小之家,乃本地人氏,姓王,祖上曾作过小小的一个京官,昔年与凤姐之祖王夫人之父认识。因贪王家的势利,便连了宗认作侄儿。\begin{note}甲双夹:与贾雨村遥遥相对。\end{note}那时只有王夫人之大兄凤姐之父\begin{note}甲双夹:两呼两起,不过欲观者自醒。\end{note}与王夫人随在京中的,知有此一门连宗之族,余者皆不认识。目今其祖已故,只有一个儿子,名唤王成,因家业萧条,仍搬出城外原乡中住去了。王成新近亦因病故,只有其子,小名狗儿。狗儿亦生一子,小名板儿,嫡妻刘氏,又生一女,名唤青儿。\begin{note}甲双夹:《石头记》中公勋世宦之家以及草莽庸俗之族,无所不有,自能各得其妙。\end{note}一家四口,仍以务农为业,因狗儿白日间又作些生计,刘氏又操井臼等事,青板姊妹两个无人看管,狗儿遂将岳母刘姥姥\begin{note}甲双夹:音老,出《谐声字笺》。称呼毕肖。\end{note}接来一处过活。这刘姥姥乃是个积年的老寡妇,膝下又无儿女,只靠两亩薄田度日。今者女婿接来养活,岂不愿意,遂一心一计,帮趁著女儿女婿过活起来。
\end{parag}


\begin{parag}
    因这年秋尽冬初,天气冷将上来,家中冬事未办,狗儿未免心中烦虑,吃了几杯闷酒,在家闲寻气恼,\begin{note}甲双夹:病此病人不少,请来看狗儿。\end{note}刘氏也不敢顶撞。\begin{note}甲眉:自“红楼梦”一回至此,则珍馐中之虀耳,好看煞!\end{note}因此刘姥姥看不过,乃劝道:“姑爷,你别嗔著我多嘴。咱们村庄人,那一个不是老老诚诚的,守多大碗儿吃多大的饭。\begin{note}甲侧:能两亩薄田度日,方说的出来。\end{note}你皆因年小的时候,托著你那老的福,\begin{note}甲双夹:妙称,何肖之至!\end{note}吃喝惯了,如今所以把持不住。有了钱就顾头不顾尾,没了钱就瞎生气,成个什么男子汉大丈夫呢!\begin{note}甲侧:此口气自何处得来?甲双夹:为纨绔下针,却先从此等小处写来。\end{note}如今咱们虽离城住著,终是天子脚下。这长安城中,遍地都是钱,只可惜没人会去拿去罢了。在家跳蹋会子也不中用。”狗儿听说,便急道:“你老只会炕头儿上混说,难道叫我打劫偷去不成?”刘姥姥道:“谁叫你偷去呢。也到底想法儿大家裁度,不然那银子钱自己跑到咱家来不成?”狗儿冷笑道:“有法儿还等到这会子呢。我又没有收税的亲戚,\begin{note}甲双夹:骂死。\end{note}作官的朋友,\begin{note}甲双夹:骂死\end{note}\begin{note}脂批:骂死世人,可叹可悲!\end{note}有什么法子可想的?便有,也只怕他们未必来理我们呢!”
\end{parag}


\begin{parag}
    刘姥姥道:“这倒不然。谋事在人,成事在天。咱们谋到了,看菩萨的保佑,有些机会,也未可知。我倒替你们想出一个机会来。当日你们原是和金陵王家\begin{note}甲双夹:四字便抵一篇世家传。\end{note}连过宗的,二十年前,他们看承你们还好,如今自然是你们拉硬屎,不肯去亲近他,故疏远起来。想当初我和女儿还去过一遭。\begin{note}甲双夹:补前文之未到处。\end{note}他们家的二小姐著实响快,会待人,倒不拿大。如今现是荣国府贾二老爷的夫人。听得说,如今上了年纪,越发怜贫恤老,最爱斋僧敬道,舍米舍钱的。如今王府虽升了边任,只怕这二姑太太还认得咱们。你何不去走动走动,或者他念旧,有些好处,也未可知。要是他发一点好心,拔一根寒毛比咱们的腰还粗呢。”刘氏一旁接口道:“你老虽说的是,但只你我这样个嘴脸,怎样好到他门上去的。先不先,他们那些门上的人也未必肯去通信。没的去打嘴现世。”
\end{parag}


\begin{parag}
    谁知狗儿利名心最重,\begin{note}甲双夹:调侃语。\end{note}听如此一说,心下便有些活动起来。又听他妻子这话,便笑接道:“姥姥既如此说,况且当年你又见过这姑太太一次,何不你老人家明日就走一趟,先试试风头再说。”刘姥姥道:“嗳呦呦!\begin{note}甲侧:口声如闻。\end{note}可是说的,‘侯门深似海’,我是个什么东西,他家人又不认得我,我去了也是白去的。”狗儿笑道:“不妨,我教你老人家一个法子:你竟带了外孙子板儿,先去找陪房周瑞,若见了他,就有些意思了。这周瑞先时曾和我父亲交过一件事,我们极好的。”\begin{note}甲双夹:欲赴豪门,必先交其仆。写来一叹。\end{note}刘姥姥道:“我也知道他的。只是许多时不走动,知道他如今是怎样。这也说不得了,你又是个男人,又这样个嘴脸,自然去不得,我们姑娘年轻媳妇子,也难卖头卖脚的,倒还是舍著我这付老脸去碰一碰。果然有些好处,大家都有益,便是没银子来,我也到那公府侯门见一见世面,也不枉我一生。”说毕,大家笑了一回。当晚计议已定。
\end{parag}


\begin{parag}
    次日天未明,刘姥姥便起来梳洗了,又将板儿教训了几句。那板儿才五六岁的孩子,一无所知,听见刘姥姥带他进城逛去,\begin{note}甲双夹:音光,去声。游也。出《谐声字笺》。\end{note}便喜的无不应承。于是刘姥姥带他进城,找至宁荣街。\begin{note}甲双夹:街名。本地风光,妙!\end{note}来至荣府大门石狮子前,只见簇簇轿马,刘姥姥便不敢过去,且掸了掸衣服,又教了板儿几句话,然后蹭\begin{note}甲侧:“蹭”字神理。\end{note}到角门前。只见几个挺胸叠肚指手画脚的人,坐在大板凳上,说东谈西呢。\begin{note}甲双夹:不知如何想来,又为侯门三等豪奴写照。\end{note}刘姥姥只得蹭上来问:“太爷们纳福。”众人打量了他一会,便问“那里来的?”刘姥姥陪笑道:“我找太太的陪房周大爷的,烦那位太爷替我请他老出来。”那些人听了,都不瞅睬,半日方说道:“你远远的在那墙角下等著,一会子他们家有人就出来的。”内中有一老年人说道:“不要误他的事,何苦耍他。”因向刘姥姥道:“那周大爷已往南边去了。他在后一带住著,他娘子却在家。你要找时,从这边绕到后街上后门上去问就是了。”\begin{note}甲双夹:有年纪人诚厚,亦是自然之理。\end{note}
\end{parag}


\begin{parag}
    刘姥姥听了谢过,遂携了板儿,绕到后门上。只见门前歇著些生意担子,也有卖吃的,也有卖顽耍物件的,闹吵吵三二十个小孩子在那里厮闹。\begin{note}甲双夹:如何想来?合眼如见。\end{note}刘姥姥便拉住一个道:“我问哥儿一声,有个周大娘可在家么?”孩子们道:“那个周大娘?我们这里周大娘有三个呢,还有两个周奶奶,不知是那一行当的?”刘姥姥道:“是太太的陪房周瑞。”孩子道:“这个容易,你跟我来。”说著,跳跳蹿蹿的引著刘姥姥进了后门,\begin{note}甲侧:因女眷,又是后门,故容易引入。\end{note}至一院墙边,指与刘姥姥道:“这就是他家。”又叫道:“周大娘,有个老奶奶来找你呢,我带了来了。”
\end{parag}


\begin{parag}
    周瑞家的在内听说,忙迎了出来,问:“是那位?”刘姥姥忙迎上来问道:“好呀,周嫂子!”周瑞家的认了半日,方笑道: “刘姥姥,你好呀!你说说,能几年,我就忘了。\begin{note}甲侧:如此口角,从何处出来?\end{note}请家里来坐罢。”刘姥姥一壁里走著,一壁笑说道:“你老是贵人多忘事,那里还记得我们呢。”说著,来至房中。周瑞家的命雇的小丫头倒上茶来吃著,周瑞家的又问板儿道:“你都长这们大了!”又问些别后闲话。又问刘姥姥:“今日还是路过,还是特来的?”\begin{note}甲侧:问的有情理。\end{note}刘姥姥便说:“原是特来瞧瞧嫂子你,二则也请请姑太太的安。若可以领我见一见更好,若不能,便藉重嫂子转致意罢了。”\begin{note}甲双夹:刘婆亦善于权变应酬矣。\end{note}
\end{parag}


\begin{parag}
    周瑞家的听了,便已猜著几分来意。只因昔年他丈夫周瑞争买田地一事,其中多得狗儿之力,今见刘姥姥如此而来,心中难却其意,\begin{note}甲双夹:在今世,周瑞夫妇算是个怀情不忘的正人。\end{note}二则也要显弄自己的体面。\begin{note}甲眉:“也要显弄”句为后文作地步,也陪房本心本意实事。\end{note}听如此说,便笑说道:“姥姥你放心,\begin{note}甲侧:自是有宠人声口。\end{note}大远的诚心诚意来了,岂有个不教你见个真佛去的呢?\begin{note}甲双夹:好口角。\end{note}论理,人来客至回话,却不与我相干。我们这里都是各占一样儿:\begin{note}甲侧:略将荣府中带一带。\end{note}我们男的只管春秋两季地租子,闲时只带著小爷们出门子就完了,我只管跟太太奶奶们出门的事。皆因你原是太太的亲戚,又拿我当个人,投奔了我来,我就破个例,给你通个信去。但只一件,姥姥有所不知,我们这里又不比五年前了。如今太太竟不大管事,都是琏二奶奶管家了。你道这琏二奶奶是谁?就是太太的内侄女,当日大舅老爷的女儿,小名凤哥的。”刘姥姥听了,罕问道:“原来是他!怪道呢,我当日就说他不错呢。\begin{note}甲双夹:我亦说不错。\end{note}这等说来,我今儿还得见他了。”周瑞家的道:“这自然的。如今太太事多心烦,有客来了,略可推得去的就推过去了,都是凤姑娘周旋迎待。今儿宁可不会太太,倒要见他一面,才不枉这里来一遭。”刘姥姥道:“阿弥陀佛!全仗嫂子方便了。”周瑞家的道:“说那里话。俗语说的:‘与人方便,自己方便。’不过用我说一句话罢了,害著我什么。”说著,便叫小丫头到倒厅上\begin{note}甲双夹:一丝不乱。\end{note}悄悄的打听打听,老太太屋里摆了饭了没有。小丫头去了。这里二人又说些闲话。
\end{parag}


\begin{parag}
    刘姥姥因说:“这凤姑娘今年大还不过二十岁罢了,就这等有本事,当这样的家,可是难得的。”周瑞家的听了道:“我的姥姥,告诉不得你呢。这位凤姑娘年纪虽小,行事却比世人都大呢。如今出挑的美人一样的模样儿,少说些有一万个心眼子。再要赌口齿,十个会说话的男人也说他不过。回来你见了就信了。就只一件,待下人未免太严些个。”\begin{note}甲双夹:略点一句,伏下后文。\end{note}说著,只见小丫头回来说:“老太太屋里已摆完了饭了,二奶奶在太太屋里呢。”周瑞家的听了,连忙起身,催著刘姥姥说:“快走,快走。这一下来他吃饭是个空子,咱们先赶著去。若迟一步,回事的人也多了,难说话。再歇了中觉,越发没了时候了。”\begin{note}甲双夹:写出阿凤勤劳冗杂,并骄矜珍贵等事来。甲眉:写阿凤勤劳等事,然却是虚笔,故于后文不犯。蒙侧:非身临其境者不知。\end{note}说著一齐下了炕,打扫打扫衣服,又教了板儿几句话,随著周瑞家的,逶迤往贾琏的住处来。
\end{parag}


\begin{parag}
    先到了倒厅,周瑞家的将刘姥姥安插在那里略等一等。自己先过了影壁,进了院门,知凤姐未下来,先找著凤姐的一个心腹通房大丫头,\begin{note}甲双夹:著眼。这也是书中一要紧人。《红楼梦》曲内虽未见有名,想亦在副册内者也。\end{note}\begin{note}脂批:观警幻情榜方知言余不谬。\end{note}名唤平儿的。\begin{note}甲双夹:名字真极,文雅则假。\end{note}周瑞家的先将刘姥姥起初来历说明,\begin{note}甲双夹:细!盖平儿原不知有此一人耳。\end{note}又说:“今日大远的特来请安。当日太太是常会的,今日不可不见,所以我带了他进来了。等奶奶下来,我细细回明,奶奶想也不责备我莽撞的。”平儿听了,便作了主意:“叫他们进来,先在这里坐著就是了。”\begin{note}甲双夹:暗透平儿身份。\end{note}周瑞家的听了,方出去引他两个进入院来。上了正房台矶,小丫头打起猩红毡帘,\begin{note}甲双夹:是冬日。\end{note}才入堂屋,只闻一阵香扑了脸来,\begin{note}甲双夹:是刘姥姥鼻中。 \end{note}竟不辨是何气味,身子如在云端里一般。\begin{note}甲双夹:是刘姥姥身子。 \end{note}满屋中之物都耀眼争光的,使人头悬目眩。\begin{note}甲双夹:是刘姥姥头目。\end{note}刘姥姥此时惟点头咂嘴念佛而已。\begin{note}甲双夹:六字尽矣,如何想来。\end{note}于是来至东边这间屋内,乃是贾琏的女儿大姐儿睡觉之所。\begin{note}甲双夹:记清。\end{note}平儿站在炕沿边,打量了刘姥姥两眼,\begin{note}甲双夹:写豪门侍儿。\end{note}只得\begin{note}甲双夹:字法。\end{note}问个好让坐。刘姥姥见平儿遍身绫罗,插金带银,花容玉貌的,\begin{note}甲双夹:从刘姥姥心中目中略一写,非平儿正传。 \end{note}便当是凤姐儿了。\begin{note}甲双夹:毕肖。\end{note}才要称姑奶奶,忽见周瑞家的称他是平姑娘,又见平儿赶著周瑞家的称周大娘,方知不过是个有些体面的丫头了。于是让刘姥姥和板儿上了炕,平儿和周瑞家的对面坐在炕沿上,小丫头子斟了茶来吃茶。
\end{parag}


\begin{parag}
    刘姥姥只听见“咯当”“咯当”的响声,大有似乎打箩柜筛面的一般,\begin{note}甲双夹:从刘姥姥心中意中幻拟出奇怪文字。\end{note}不免东瞧西望的。忽见堂屋中柱子上挂著一个匣子,底下又坠著一个秤砣般一物,却不住的乱幌。\begin{note}甲双夹:从刘姥姥心中目中设譬拟想,真是镜花水月。\end{note}刘姥姥心中想著:“这是什么爱物儿?有甚用呢?”正呆时,\begin{note}甲双夹:三字有劲。\end{note}只听得“当”的一声,又若金钟铜磬一般,不防倒唬的一展眼。接著又是一连八九下。\begin{note}甲侧:写得出。甲双夹:细!是巳时。\end{note}方欲问时,只见小丫头子们齐乱跑,说:“奶奶下来了。”周瑞家的与平儿忙起身,命刘姥姥:“只管等著,是时候我们来请你。”说著,都迎出去了。
\end{parag}


\begin{parag}
    刘姥姥屏声侧耳默候。只听远远有人笑声,\begin{note}甲侧:写得侍仆妇。\end{note}约有一二十妇人,衣裙窣窣,渐入堂屋,往那边屋内去了。又见两三个妇人,都捧著大漆捧盒,进这边来等候。听得那边说了声“摆饭”,渐渐的人才散出,只有伺候端菜的几个人。半日鸦雀不闻之后,忽见二人抬了一张炕桌来,放在这边炕上,桌上碗盘森列,仍是满满的鱼肉在内,不过略动了几样。板儿一见了,便吵著要肉吃,刘姥姥一巴掌打了他去。忽见周瑞家的笑嘻嘻走过来,招手儿叫他。刘姥姥会意,于是带了板儿下炕,至堂屋中,周瑞家的又和他唧咕了一会,方过这边屋里来。
\end{parag}


\begin{parag}
    只见门外錾铜钩上悬著大红撒花软帘,\begin{note}甲侧:从门外写来。\end{note}南窗下是炕,炕上大红毡条,靠东边板壁立著一个锁子锦靠背与一个引枕,铺著金心绿闪缎大坐褥,旁边有雕漆痰盒。那凤姐儿家常带著秋板貂鼠昭君套,围著攒珠勒子,穿著桃红撒花袄,石青刻丝灰鼠披风,大红洋绉银鼠皮裙,粉光脂艳,端端正正坐在那里,\begin{note}甲双夹:一段阿凤房室起居器皿家常正传,奢侈珍贵好奇货注脚,写来真是好看。\end{note}手内拿著小铜火箸儿拨手炉内的灰。\begin{note}甲侧:至平,实至奇,稗官中未见此笔。甲双夹:这一句是天然地设,非别文杜撰妄拟者。\end{note}平儿站在炕沿边,捧著小小的一个填漆茶盘,盘内一个小盖钟。凤姐也不接茶,也不抬头,\begin{note}甲侧:神情宛肖。\end{note}只管拨手炉内的灰,慢慢的问道:“怎么还不请进来?”\begin{note}甲侧:此等笔墨,真可谓追魂摄魄。蒙侧:“还不请进来”五字,写尽天下富贵人待穷亲戚的态度。\end{note}一面说,一面抬身要茶时,只见周瑞家的已带了两个人在地下站著呢。这才忙欲起身,犹未起身,满面春风的问好,又嗔周瑞家的不早说。刘姥姥在地下已是拜了数拜,“问姑奶奶安。”凤姐忙说:“周姐姐,快搀住不拜罢。请坐。我年轻,不大认得,可也不知是什么辈数,不敢称呼。”周瑞家的忙回道:“这就是我才回的那姥姥了。”\begin{note}甲侧:凤姐云“不敢称呼”,周瑞家的云“那个姥姥”。凡三四句一气读下,方是凤姐声口。\end{note}凤姐点头。刘姥姥已在炕沿上坐了,板儿便躲在背后,百般的哄他出来作揖,他死也不肯。
\end{parag}


\begin{parag}
    凤姐儿笑\begin{note}甲侧:二笑。\end{note}道:“亲戚们不大走动,都疏远了。知道的呢,说你们弃厌我们,不肯常来,\begin{note}甲侧:阿凤真真可畏可恶。\end{note}不知道的那起小人,还只当我们眼里没人似的。”刘姥姥忙念佛\begin{note}甲侧:如闻。\end{note}道:“我们家道艰难,走不起,来了这里,没的给姑奶奶打嘴,就是管家爷们看著也不像。”凤姐儿笑\begin{note}甲侧:三笑。\end{note}道:“这话没的叫人恶心。不过借赖著祖父虚名,作个穷官儿,谁家有什么,不过是个旧日的空架子。俗语说,‘朝廷还有三门子穷亲戚’呢,何况你我。”说著,又问周瑞家的回了太太了没有。\begin{note}甲侧:一笔不肯落空,的是阿凤。\end{note}周瑞家的道:“如今等奶奶的示下。”凤姐道:“你去瞧瞧,要是有人有事就罢,得闲儿呢就回,看怎么说。”周瑞家的答应著去了。
\end{parag}


\begin{parag}
    这里凤姐叫人抓些果子与板儿吃,刚问些闲话时,就有家下许多媳妇管事的来回话。\begin{note}甲侧:不落空家务事,却不实写。妙极!妙极!\end{note}平儿回了,凤姐道:“我这里陪客呢,晚上再来回。若有很要紧的,你就带进来现办。”平儿出去了,一会进来说:“我都问了,没什么紧事,我就叫他们散了。”凤姐点头。只见周瑞家的回来,向凤姐道:“太太说了,今日不得闲,二奶奶陪著便是一样。多谢费心想著。白来逛逛呢便罢,若有甚说的,只管告诉二奶奶,都是一样。”刘姥姥道:“也没甚说的,不过是来瞧瞧姑太太,姑奶奶,也是亲戚们的情分。”周瑞家的道:“没甚说的便罢,若有话,只管回二奶奶,是和太太一样的。”\begin{note}甲侧:周妇系真心为老妪也,可谓得方便。\end{note}一面说,一面递眼色与刘姥姥。\begin{note}甲侧:何如?余批不谬。\end{note}刘姥姥会意,未语先飞红的脸,\begin{note}蒙侧:开口告人难。\end{note}欲待不说,今日又所为何来?只得忍耻\begin{note}甲眉:老妪有忍耻之心,故后有招大姐之事。作者并非泛写,且为求亲靠友下一棒喝。\end{note}说道:“论理今儿初次见姑奶奶,却不该说,只是大远的奔了你老这里来,也少不的说了。”刚说到这里,只听二门上小厮们回说:“东府里的小大爷进来了。”凤姐忙止刘姥姥:“不必说了。”一面便问:“你蓉大爷在那里呢?”\begin{note}甲侧:惯用此等横云断山法。\end{note}只听一路靴子脚响,进来了一个十七八岁的少年,面目清秀,身材俊俏,轻裘宝带,美服华冠。\begin{note}甲侧:如纨绔写照。\end{note}刘姥姥此时坐不是,立不是,藏没处藏。凤姐笑道:“你只管坐著,这是我侄儿。”刘姥姥方扭扭捏捏在炕沿上坐了。
\end{parag}


\begin{parag}
    贾蓉笑道:“我父亲打发我来求婶子,说上回老舅太太给婶子的那架玻璃炕屏,明日请一个要紧的客,借了略摆一摆就送过来的。”\begin{note}甲侧:夹写凤姐好奖誉。\end{note}凤姐道:“说迟了一日,昨儿已经给了人了。”贾蓉听著,嘻嘻的笑著,在炕沿上半跪道:“婶子若不借,又说我不会说话了,又挨一顿好打呢。婶子只当可怜侄儿罢。”凤姐笑\begin{note}甲侧:又一笑,凡五。\end{note}道:“也没见我们王家的东西都是好的不成?一般你们那里放著那些东西,只是看不见我的才罢。”贾蓉笑道:“那里有这个好呢!只求开恩罢。”凤姐道:“若碰一点儿,你可仔细你的皮!”因命平儿拿了楼房的钥匙,传几个妥当人抬去。贾蓉喜的眉开眼笑,说:“我亲自带了人拿去,别由他们乱碰。”说著便起身出去了。
\end{parag}


\begin{parag}
    这里凤姐忽又想起一事来,便向窗外叫:“蓉哥回来。”外面几个人接声说:“蓉大爷快回来。”贾蓉忙复身转来,垂手侍立,听何指示。\begin{note}甲眉:传神之笔,写阿凤跃跃纸上。\end{note}那凤姐只管慢慢的吃茶,出了半日的神,又笑道:“罢了,你且去罢。晚饭后你来再说罢。这会子有人,我也没精神了。”贾蓉应了一声,方慢慢的退去。\begin{note}甲侧:妙!却是从刘姥姥身边目中写来。度至下回。\end{note}
\end{parag}


\begin{parag}
    这里刘姥姥心神方定,才又说道:“今日我带了你侄儿来,也不为别的,只因他老子娘在家里,连吃的都没有。如今天又冷了,越想没个派头儿,只得带了你侄儿奔了你老来。”说著又推板儿道:“你那爹在家怎么教你来?打发咱们作煞事来?只顾吃果子咧。”凤姐早已明白了,听他不会说话,因笑止道:\begin{note}甲双夹:又一笑,凡六。自刘姥姥来凡笑五次,写得阿凤乖滑伶俐,合眼如立在前。若会说话之人便听他说了,阿凤厉害处正在此。问看官常有将挪移借贷已说明白了,彼仍推聋装哑,这人为阿凤若何?呵呵,一叹!\end{note}“不必说了,我知道了。”因问周瑞家的:“这姥姥不知可用了早饭没有?”刘姥姥忙说道:“一早就往这里赶咧,那里还有吃饭的工夫咧。”凤姐听说,忙命快传饭来。一时周瑞家的传了一桌客饭来,摆在东边屋内,过来带了刘姥姥和板儿过去吃饭。凤姐说道:“周姐姐,好生让著些儿,我不能陪了。”于是过东边房里来。又叫过周瑞家的去,问他才回了太太,说了些什么?周瑞家的道:“太太说,他们家原不是一家子,不过因出一姓,当年又与太老爷在一处作官,偶然连了宗的。这几年来也不大走动。当时他们来一遭,却也没空了他们。今儿既来了瞧瞧我们,是他的好意思,\begin{note}甲侧:穷亲戚来看是“好意思”,余又自《石头记》中见了,叹叹!\end{note}也不可简慢了他。便是有什么说的,叫奶奶裁度著就是了。”\begin{note}甲眉:王夫人数语令余几哭出。\end{note}凤姐听了说道:“我说呢,既是一家子,我如何连影儿也不知道。”
\end{parag}


\begin{parag}
    说话时,刘姥姥已吃毕了饭,拉了板儿过来,舚舌咂嘴的道谢。凤姐笑道:“且请坐下,听我告诉你老人家。方才的意思,我已知道了。若论亲戚之间,原该不等上门来就该有照应才是。但如今家内杂事太烦,太太渐上了年纪,一时想不到也是有的。\begin{note}甲侧:点“不待上门就该有照应”数语,此亦于《石头记》再见话头。\end{note}况是我近来接著管些事,都不知道这些亲戚们。二则外头看著虽是烈烈轰轰的,殊不知大有大的艰难去处,说与人也未必信罢。今儿你既老远的来了,又是头一次见我张口,怎好叫你空回去呢。\begin{note}甲侧:也是《石头记》再见了,叹叹!\end{note}可巧昨儿太太给我的丫头们做衣裳的二十两银子,我还没动呢,你若不嫌少,就暂且先拿了去罢。”那刘姥姥先听见告艰难,只当是没有,心里便突突的,\begin{note}甲侧:可怜可叹!\end{note}后来听见给他二十两,喜的又浑身发痒起来,\begin{note}甲侧:可怜可叹!\end{note}说道:“嗳,我也是知道艰难的。但俗语说的,‘瘦死的骆驼比马大’,凭他怎样,你老拔根寒毛比我们的腰还粗呢!”周瑞家的见他说的粗鄙,只管使眼色止他。凤姐看见,笑而不睬,只命平儿把昨儿那包银子拿来,再拿一吊钱来,\begin{note}甲侧:这样常例亦再见。\end{note}都送到刘姥姥的跟前。凤姐乃道:“这是二十两银子,暂且给这孩子做件冬衣罢。若不拿著,就真是怪我了。这钱雇车坐罢。改日无事,只管来逛逛,方是亲戚们的意思。天也晚了,也不虚留你们了,到家里该问好的问个好儿罢。”一面说,一面就站了起来。
\end{parag}


\begin{parag}
    刘姥姥只管千恩万谢,拿了银钱,随了周瑞家的来至外面。周瑞家的方道:“我的娘啊!你见了他怎么倒不会说话了?开口就是‘你侄儿’。我说句不怕你恼的话,便是亲侄儿,也要说和软些。那蓉大爷才是他的正经侄儿呢,他怎么又跑出这么个侄儿来了。”\begin{note}甲双夹:与前“眼色”针对,可见文章中无一个闲字。为财势一哭。\end{note}刘姥姥笑道:“我的嫂子,\begin{note}甲侧:赧颜如见。\end{note}我见了他,心眼儿里爱还爱不过来,那里还说的上话来呢。”二人说著,又到周瑞家坐了片时。刘姥姥便要留下一块银子与周瑞家孩子们买果子吃,周瑞家的如何放在眼里,执意不肯。刘姥姥感谢不尽,仍从后门去了。正是:
\end{parag}


\begin{poem}
    \begin{pl}得意浓时易接济,受恩深处胜亲朋。\end{pl}
\end{poem}


\begin{parag}
    \begin{note}甲:一进荣府一回,曲折顿挫,笔如游龙,且将豪华举止令观者已得大概,想作者应是心花欲开之候。借刘妪入阿凤正文,“送宫花”写“金玉初聚”为引,作者真笔似游龙,变幻难测,非细究至再三再四不记数,那能领会也?叹叹!蒙:梦里风流,醒后风流,试问何真何假?刘姆乞谋,蓉儿借求,多少颠倒相酬。英雄反正用计筹,不是死生看守。\end{note}
\end{parag}
