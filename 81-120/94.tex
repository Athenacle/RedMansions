\chap{九十四}{宴海棠贾母赏花妖 失宝玉通灵知奇祸}



\begin{parag}
    话说赖大带了贾芹出来,一宿无话,静候贾政回来。单是那些女尼女道重进园来,都喜欢的了不得,欲要到各处逛逛,明日预备进宫。不料赖大便吩咐了看院的婆子并小厮看守,惟给了些饮食,却是一步不准走开。那些女孩子摸不着头脑,只得坐着等到天亮。园里各处的丫头虽都知道拉进女尼们来预备宫里使唤,却也不能深知原委。
\end{parag}


\begin{parag}
    到了明日早起,贾政正要下班,因堂上发下两省城工估销册子立刻要查核,一时不能回家,便叫人告诉贾琏说:“赖大回来,你务必查问明白。该如何办就如何办了,不必等我。”贾琏奉命,先替芹儿喜欢,又想道:若是办得一点影儿都没有,又恐贾政生疑,”不如回明二太太讨个主意办去,便是不合老爷的心,我也不至甚担干系。”主意定了,进内去见王夫人,陈说:“昨日老爷见了揭帖生气,把芹儿和女尼女道等都叫进府来查办。今日老爷没空问这种不成体统的事,叫我来回太太,该怎么便怎么样。我所以来请示太太,这件事如何办理?”王夫人听了,诧异道:“这是怎么说!若是芹儿这么样起来,这还成咱们家的人了么!但只这个贴帖儿的也可恶,这些话可是混嚼说得的么。你到底问了芹儿有这件事没有呢?”贾琏道:“刚才也问过了。太太想,别说他干了没有,就是干了,一个人干了混账事也肯应承么?但只我想芹儿也不敢行此事,知道那些女孩子都是娘娘一时要叫的,倘或闹出事来,怎么样呢?依侄儿的主见,要问也不难,若问出来,太太怎么个办法呢?”王夫人道:“如今那些女孩子在那里?”贾琏道:“都在园里锁着呢。”王夫人道:“姑娘们知道不知道?”贾琏道:“大约姑娘们也都知道是预备宫里头的话,外头并没提起别的来。”王夫人道:“很是。这些东西一刻也是留不得的。头里我原要打发他们去来着,都是你们说留着好,如今不是弄出事来了么。你竟叫赖大那些人带去,细细的问他的本家有人没有,将文书查出,花上几十两银子,雇只船,派个妥当人送到本地,一概连文书发还了,也落得无事。若是为着一两个不好,个个都押着他们还俗,那又太造孽了。若在这里发给官媒,虽然我们不要身价,他们弄去卖钱,那里顾人的死活呢。芹儿呢,你便狠狠的说他一顿。除了祭祀喜庆,无事叫他不用到这里来,看仔细碰在老爷气头儿上,那可就吃不了兜着走了。并说与账房儿里,把这一项钱粮档子销了。还打发个人到水月庵,说老爷的谕:除了上坟烧纸,若有本家爷们到他那里去,不许接待。若再有一点不好风声,连老姑子一并撵出去。”
\end{parag}


\begin{parag}
    贾琏一一答应了,出去将王夫人的话告诉赖大,说:“是太太主意,叫你这么办去。办完了,告诉我去回太太。你快办去罢。回来老爷来,你也按着太太的话回去。”赖大听说,便道:“我们太太真正是个佛心。这班东西着人送回去。既是太太好心,不得不挑个好人。芹哥儿竟交给二爷开发了罢。那个贴帖儿的,奴才想法儿查出来,重重的收拾他才好。”贾琏点头说:“是了。”即刻将贾芹发落。赖大也赶着把女尼等领出,按着主意办去了。晚上贾政回家,贾琏赖大回明贾政。贾政本是省事的人,听了也便撂开手了。独有那些无赖之徒,听得贾府发出二十四个女孩子出来,那个不想。究竟那些人能够回家不能,未知着落,亦难虚拟。
\end{parag}


\begin{parag}
    且说紫鹃因黛玉渐好,园中无事,听见女尼等预备宫内使唤,不知何事,便到贾母那边打听打听,恰遇着鸳鸯下来,闲着坐下说闲话儿,提起女尼的事。鸳鸯诧异道:“我并没有听见,回来问问二奶奶就知道了。”正说着,只见傅试家两个女人过来请贾母的安,鸳鸯要陪了上去。那两个女人因贾母正睡晌觉,就与鸳鸯说了一声儿回去了。紫鹃问:“这是谁家差来的?”鸳鸯道:“好讨人嫌。家里有了一个女孩儿生得好些,便献宝的似的,常常在老太太面前夸他家姑娘长得怎么好,心地怎么好,礼貌上又能,说话儿又简绝,做活计儿手儿又巧,会写会算,尊长上头最孝敬的,就是待下人也是极和平的。来了就编这么一大套,常常说给老太太听。我听着很烦。这几个老婆子真讨人嫌。我们老太太偏爱听那些个话。老太太也罢了,还有宝玉,素常见了老婆子便很厌烦的,偏见了他们家的老婆子便不厌烦。你说奇不奇!前儿还来说,他们姑娘现有多少人家儿来求亲,他们老爷总不肯应,心里只要和咱们这种人家作亲才肯。一回夸奖,一回奉承,把老太太的心都说活了。”紫鹃听了一呆,便假意道:“若老太太喜欢,为什么不就给宝玉定了呢?”鸳鸯正要说出原故,听见上头说:“老太太醒了。”鸳鸯赶着上去。
\end{parag}


\begin{parag}
    紫鹃只得起身出来,回到园里。一头走,一头想道:“天下莫非只有一个宝玉,你也想他,我也想他。我们家的那一位越发痴心起来了,看他的那个神情儿,是一定在宝玉身上的了。三番五次的病,可不是为着这个是什么!这家里金的银的还闹不清,若添了一个什么傅姑娘,更了不得了。我看宝玉的心也在我们那一位的身上,听着鸳鸯的说话竟是见一个爱一个的。这不是我们姑娘白操了心了吗?”紫鹃本是想着黛玉,往下一想,连自己也不得主意了,不免掉下泪来。要想叫黛玉不用瞎操心呢,又恐怕他烦恼,若是看着他这样,又可怜见儿的。左思右想,一时烦躁起来,自己啐自己道:“你替人耽什么忧!就是林姑娘真配了宝玉,他的那性情儿也是难伏侍的。宝玉性情虽好,又是贪多嚼不烂的。我倒劝人不必瞎操心,我自己才是瞎操心呢。从今以后,我尽我的心伏侍姑娘,其余的事全不管!”这么一想,心里倒觉清净。回到潇湘馆来,见黛玉独自一人坐在炕上,理从前做过的诗文词稿。抬头见紫鹃来,便问:“你到那里去了?”紫鹃道:“我今儿瞧了瞧姐妹们去。”黛玉道:“敢是找袭人姐姐去么?”紫鹃道:“我找他做什么。”黛玉一想这话,怎么顺嘴说了出来,反觉不好意思,便啐道:“你找谁与我什么相干!倒茶去罢。”
\end{parag}


\begin{parag}
    紫鹃也心里暗笑,出来倒茶。只听见园里的一迭声乱嚷,不知何故,一面倒茶,一面叫人去打听。回来说道:“怡红院里的海棠本来萎了几棵,也没人去浇灌他。昨日宝玉走去,瞧见枝头上好象有了骨朵儿似的。人都不信,没有理他。忽然今日开得很好的海棠花,众人诧异,都争着去看。连老太太,太太都哄动了来瞧花儿呢,所以大奶奶叫人收拾园里败叶枯枝,这些人在那里传唤。”黛玉也听见了,知道老太太来,便更了衣,叫雪雁去打听,”若是老太太来了,即来告诉我。”雪雁去不多时,便跑来说:“老太太,太太好些人都来了,请姑娘就去罢。”黛玉略自照了一照镜子,掠了一掠鬓发,便扶着紫鹃到怡红院来。已见老太太坐在宝玉常卧的榻上,黛玉便说道:“请老太太安。”退后,便见了邢王二夫人,回来与李纨,探春,惜春,邢岫烟彼此问了好。只有凤姐因病未来,史湘云因他叔叔调任回京,接了家去,薛宝琴跟他姐姐家去住了,李家姐妹因见园内多事,李婶娘带了在外居住:所以黛玉今日见的只有数人。大家说笑了一回,讲究这花开得古怪。贾母道:“这花儿应在三月里开的,如今虽是十一月,因节气迟,还算十月,应着小阳春的天气,这花开因为和暖是有的。”王夫人道:“老太太见的多,说得是。也不为奇。”邢夫人道:“我听见这花已经萎了一年,怎么这回不应时候儿开了,必有个原故。”李纨笑道:“老太太与太太说得都是。据我的糊涂想头,必是宝玉有喜事来了,此花先来报信。”探春虽不言语,心内想:“此花必非好兆。大凡顺者昌,逆者亡。草木知运,不时而发,必是妖孽。”只不好说出来。独有黛玉听说是喜事,心里触动,便高兴说道:“当初田家有荆树一棵,三个弟兄因分了家,那荆树便枯了。后来感动了他弟兄们仍旧在一处,那荆树也就荣了。可知草木也随人的。如今二哥哥认真念书,舅舅喜欢,那棵树也就发了。”贾母王夫人听了喜欢,便说:“林姑娘比方得有理,很有意思。”正说着,贾赦,贾政,贾环,贾兰都进来看花。贾赦便说:“据我的主意,把他砍去,必是花妖作怪。”贾政道:“见怪不怪,其怪自败。不用砍他,随他去就是了。”贾母听见,便说:“谁在这里混说!人家有喜事好处,什么怪不怪的。若有好事,你们享去,若是不好,我一个人当去。你们不许混说。”贾政听了,不敢言语,讪讪的同贾赦等走了出来。
\end{parag}


\begin{parag}
    那贾母高兴,叫人传话到厨房里,快快预备酒席,大家赏花。叫:“宝玉,环儿,兰儿各人做一首诗志喜。林姑娘的病才好,不要他费心,若高兴,给你们改改。”对着李纨道:“你们都陪我喝酒。”李纨答应了”是”,便笑对探春笑道:“都是你闹的。”探春道:“饶不叫我们做诗,怎么我们闹的。”李纨道:“海棠社不是你起的么,如今那棵海棠也要来入社了。”大家听着都笑了。一时摆上酒菜,一面喝着,彼此都要讨老太太的欢喜,大家说些兴头话。宝玉上来,斟了酒,便立成了四句诗,写出来念与贾母听道:
\end{parag}


\begin{poem}
    \begin{pl}
        海棠何事忽摧隤,今日繁花为底开?
    \end{pl}


    \begin{pl}
        应是北堂增寿考,一阳旋复占先梅。
    \end{pl}

\end{poem}


\begin{parag}
    贾环也写了来念道:
\end{parag}

\begin{poem}
    \begin{pl}
        草木逢春当茁芽,海棠未发候偏差。
    \end{pl}


    \begin{pl}
        人间奇事知多少,冬月开花独我家。
    \end{pl}
\end{poem}


\begin{parag}
    贾兰恭楷誊正,呈与贾母,贾母命李纨念道:
\end{parag}


\begin{poem}


    \begin{pl}
        烟凝媚色春前萎,霜浥微红雪后开。
    \end{pl}


    \begin{pl}
        莫道此花知识浅,欣荣预佐合欢杯。
    \end{pl}

\end{poem}

\begin{parag}
    贾母听毕,便说:“我不大懂诗,听去倒是兰儿的好,环儿做得不好。都上来吃饭罢。”宝玉看见贾母喜欢,更是兴头。因想起:“晴雯死的那年海棠死的,今日海棠复荣,我们院内这些人自然都好。但是晴雯不能象花的死而复生了。”顿觉转喜为悲。忽又想起前日巧姐提凤姐要把五儿补入,或此花为他而开,也未可知,却又转悲为喜,依旧说笑。
\end{parag}


\begin{parag}
    贾母还坐了半天,然后扶了珍珠回去了。王夫人等跟着过来。只见平儿笑嘻嘻的迎上来说:“我们奶奶知道老太太在这里赏花,自己不得来,叫奴才来伏侍老太太,太太们,还有两匹红送给宝二爷包裹这花,当作贺礼。”袭人过来接了,呈与贾母看。贾母笑道:“偏是凤丫头行出点事儿来,叫人看着又体面,又新鲜,很有趣儿。”袭人笑着向平儿道:“回去替宝二爷给二奶奶道谢。要有喜大家喜。”贾母听了笑道:“嗳哟,我还忘了呢,凤丫头虽病着,还是他想得到,送得也巧。”一面说着,众人就随着去了。平儿私与袭人道:“奶奶说,这花开得奇怪,叫你铰块红绸子挂挂,便应在喜事上去了。以后也不必只管当作奇事混说。”袭人点头答应,送了平儿出去。不题。
\end{parag}


\begin{parag}
    且说那日宝玉本来穿着一裹圆的皮袄在家歇息,因见花开,只管出来看一回,赏一回,叹一回,爱一回的,心中无数悲喜离合,都弄到这株花上去了。忽然听说贾母要来,便去换了一件狐腋箭袖,罩一件元狐腿外褂,出来迎接贾母。匆匆穿换,未将通灵宝玉挂上。及至后来贾母去了,仍旧换衣。袭人见宝玉脖子上没有挂着,便问:“那块玉呢?”宝玉道:“才刚忙乱换衣,摘下来放在炕桌上,我没有带。”袭人回看桌上并没有玉,便向各处找寻,踪影全无,吓得袭人满身冷汗。宝玉道:“不用着急,少不得在屋里的。问他们就知道了。”袭人当作麝月等藏起吓他顽,便向麝月等笑着说道:“小蹄子们,顽呢到底有个顽法。把这件东西藏在那里了?别真弄丢了,那可就大家活不成了。”麝月等都正色道:“这是那里的话!顽是顽笑是笑,这个事非同儿戏,你可别混说。你自己昏了心了,想想罢,想想搁在那里了。这会子又混赖人了。”袭人见他这般光景,不象是顽话,便着急道:“皇天菩萨小祖宗,到底你摆在那里去了?”宝玉道:“我记得明明放在炕桌上的,你们到底找啊。”袭人,麝月,秋纹等也不敢叫人知道,大家偷偷儿的各处搜寻。闹了大半天,毫无影响,甚至翻箱倒笼,实在没处去找,便疑到方才这些人进来,不知谁捡了去了。袭人说道:“进来的谁不知道这玉是性命似的东西呢,谁敢捡了去呢。你们好歹先别声张,快到各处问去。若有姐妹们捡着吓我们顽呢,你们给他磕头要了回来,若是小丫头偷了去,问出来也不回上头,不论把什么送给他换了出来都使得的。这可不是小事,真要丢了这个,比丢了宝二爷的还利害呢。”麝月秋纹刚要往外走,袭人又赶出来嘱咐道:“头里在这里吃饭的倒先别问去,找不成再惹出些风波来,更不好了。”麝月等依言分头各处追问,人人不晓,个个惊疑。麝月等回来,俱目瞪口呆,面面相窥。宝玉也吓怔了。袭人急的只是干哭。找是没处找,回又不敢回,怡红院里的人吓得个个象木雕泥塑一般。
\end{parag}


\begin{parag}
    大家正在发呆,只见各处知道的都来了。探春叫把园门关上,先命个老婆子带着两个丫头,再往各处去寻去,一面又叫告诉众人:若谁找出来,重重的赏银。大家头宗要脱干系,二宗听见重赏,不顾命的混找了一遍,甚至于茅厮里都找到。谁知那块玉竟象绣花针儿一般,找了一天,总无影响。李纨急了,说:“这件事不是顽的,我要说句无礼的话了。”众人道:“什么呢?”李纨道:“事情到了这里,也顾不得了。现在园里除了宝玉,都是女人,要求各位姐姐,妹妹,姑娘都要叫跟来的丫头脱了衣服,大家搜一搜。若没有,再叫丫头们去搜那些老婆子并粗使的丫头。”大家说道:“这话也说的有理。现在人多手乱,鱼龙混杂,倒是这么一来,你们也洗洗清。”探春独不言语。那些丫头们也都愿意洗净自己。先是平儿起,平儿说道:“打我先搜起。”于是各人自己解怀,李纨一气儿混搜。探春嗔着李纨道:“大嫂子,你也学那起不成材料的样子来了。那个人既偷了去,还肯藏在身上?况且这件东西在家里是宝,到了外头,不知道的是废物,偷他做什么?我想来必是有人使促狭。”众人听说,又见环儿不在这里,昨儿是他满屋里乱跑,都疑到他身上,只是不肯说出来。探春又道:“使促狭的只有环儿。你们叫个人去悄悄的叫了他来,背地里哄着他,叫他拿出来,然后吓着他,叫他不要声张。这就完了。”大家点头称是。
\end{parag}


\begin{parag}
    李纨便向平儿道:“这件事还是得你去才弄得明白。”平儿答应,就赶着去了。不多时同了环儿来了。众人假意装出没事的样子,叫人沏了碗茶搁在里间屋里,众人故意搭讪走开。原叫平儿哄他,平儿便笑着向环儿道:“你二哥哥的玉丢了,你瞧见了没有?”贾环便急得紫涨了脸,瞪着眼说道:“人家丢了东西,你怎么又叫我来查问,疑我。我是犯过案的贼么!”平儿见这样子,倒不敢再问,便又陪笑道:“不是这么说,怕三爷要拿了去吓他们,所以白问问瞧见了没有,好叫他们找。”贾环道:“他的玉在他身上,看见不看见该问他,怎么问我。捧着他的人多着咧!得了什么不来问我,丢了东西就来问我!”说着,起身就走。众人不好拦他。这里宝玉倒急了,说道:“都是这劳什子闹事,我也不要他了。你们也不用闹了。环儿一去,必是嚷得满院里都知道了,这可不是闹事了么。”袭人等急得又哭道:“小祖宗,你看这玉丢了没要紧,若是上头知道了,我们这些人就要粉身碎骨了!”说着,便嚎啕大哭起来。
\end{parag}


\begin{parag}
    众人更加伤感,明知此事掩饰不来,只得要商议定了话,回来好回贾母诸人。宝玉道:“你们竟也不用商议,硬说我砸了就完了。”平儿道:“我的爷,好轻巧话儿!上头要问为什么砸的呢,他们也是个死啊。倘或要起砸破的碴儿来,那又怎么样呢?”宝玉道:“不然便说我前日出门丢了。”众人一想,这句话倒还混得过去,但是这两天又没上学,又没往别处去。宝玉道:“怎么没有,大前儿还到南安王府里听戏去了呢,便说那日丢的。”探春道:“那也不妥。既是前儿丢的,为什么当日不来回。”众人正在胡思乱想,要装点撒谎,只听得赵姨娘的声儿哭着喊着走来说:“你们丢了东西自己不找,怎么叫人背地里拷问环儿。我把环儿带了来,索性交给你们这一起洑上水的,该杀该剐,随你们罢。”说着,将环儿一推说:“你是个贼,快快的招罢!”气得环儿也哭喊起来。
\end{parag}


\begin{parag}
    李纨正要劝解,丫头来说:“太太来了。”袭人等此时无地可容,宝玉等赶忙出来迎接。赵姨娘暂且也不敢作声,跟了出来。王夫人见众人都有惊惶之色,才信方才听见的话,便道:“那块玉真丢了么?”众人都不敢作声,王夫人走进屋里坐下,便叫袭人。慌得袭人连忙跪下,含泪要禀。王夫人道:“你起来,快快叫人细细找去,一忙乱倒不好了。”袭人哽咽难言。宝玉生恐袭人真告诉出来,便说道:“太太,这事不与袭人相干。是我前日到南安王府那里听戏,在路上丢了。”王夫人道:“为什么那日不找?”宝玉道:“我怕他们知道,没有告诉他们。我叫焙茗等在外头各处找过的。”王夫人道:“胡说!如今脱换衣服不是袭人他们伏侍的么。大凡哥儿出门回来,手巾荷包短了,还要问个明白,何况这块玉不见了,便不问的么!”宝玉无言可答。赵姨娘听见,便得意了,忙接过口道:“外头丢了东西,也赖环儿!”话未说完,被王夫人喝道:“这里说这个,你且说那些没要紧的话!”赵姨娘便不敢言语了。还是李纨探春从实的告诉了王夫人一遍,王夫人也急得泪如雨下,索性要回明贾母,去问邢夫人那边跟来的这些人去。
\end{parag}


\begin{parag}
    凤姐病中也听见宝玉失玉,知道王夫人过来,料躲不住,便扶了丰儿来到园里。正值王夫人起身要走,凤姐娇怯怯的说:“请太太安。”宝玉等过来问了凤姐好。王夫人因说道:“你也听见了么,这可不是奇事吗?刚才眼错不见就丢了,再找不着。你去想想,打从老太太那边丫头起至你们平儿,谁的手不稳,谁的心促狭。我要回了老太太,认真的查出来才好。不然是断了宝玉的命根子了。”凤姐回道:“咱们家人多手杂,自古说的,‘知人知面不知心’,那里保得住谁是好的。但是一吵嚷已经都知道了,偷玉的人若叫太太查出来,明知是死无葬身之地,他着了急,反要毁坏了灭口,那时可怎么处呢。据我的糊涂想头,只说宝玉本不爱他,撂丢了,也没有什么要紧。只要大家严密些,别叫老太太老爷知道。这么说了,暗暗的派人去各处察访,哄骗出来,那时玉也可得,罪名也好定。不知太太心里怎么样?”王夫人迟了半日,才说道:“你这话虽也有理,但只是老爷跟前怎么瞒的过呢。”便叫环儿过来道:“你二哥哥的玉丢了,白问了你一句,怎么你就乱嚷。若是嚷破了,人家把那个毁坏了,我看你活得活不得!”贾环吓得哭道:“我再不敢嚷了。”赵姨娘听了,那里还敢言语。王夫人便吩咐众人道:“想来自然有没找到的地方儿,好端端的在家里的,还怕他飞到那里去不成。只是不许声张。限袭人三天内给我找出来,要是三天找不着,只怕也瞒不住,大家那就不用过安静日子了。”说着,便叫凤姐儿跟到邢夫人那边商议踩缉。不题。
\end{parag}


\begin{parag}
    这里李纨等纷纷议论,便传唤看园子的一干人来,叫把园门锁上,快传林之孝家的来,悄悄儿的告诉了他,叫他吩咐前后门上,三天之内,不论男女下人从里头可以走动,要出时一概不许放出,只说里头丢了东西,待这件东西有了着落,然后放人出来。林之孝家的答应了”是”,因说:“前儿奴才家里也丢了一件不要紧的东西,林之孝必要明白,上街去找了一个测字的,那人叫做什么刘铁嘴,测了一个字,说的很明白,回来依旧一找便找着了。”袭人听见,便央及林家的道:“好林奶奶,出去快求林大爷替我们问问去。”那林之孝家的答应着出去了。邢岫烟道:“若说那外头测字打卦的,是不中用的。我在南边闻妙玉能扶乩,何不烦他问一问。况且我听见说这块玉原有仙机,想来问得出来。”众人都诧异道:“咱们常见的,从没有听他说起。”麝月便忙问岫烟道:“想来别人求他是不肯的,好姑娘,我给姑娘磕个头,求姑娘就去,若问出来了,我一辈子总不忘你的恩。”说着,赶忙就要磕下头去,岫烟连忙拦住。黛玉等也都怂恿着岫烟速往栊翠庵去。一面林之孝家的进来说道:“姑娘们大喜。林之孝测了字回来说,这玉是丢不了的,将来横竖有人送还来的。”众人听了,也都半信半疑,惟有袭人麝月喜欢的了不得。探春便问:“测的是什么字?”林之孝家的道:“他的话多,奴才也学不上来,记得是拈了个赏人东西的‘赏’字。那刘铁嘴也不问,便说:‘丢了东西不是?’”李纨道:“这就算好。”林之孝家的道:“他还说,‘赏’字上头一个‘小’字,底下一个‘口’字,这件东西很可嘴里放得,必是个珠子宝石。”众人听了,夸赞道:“真是神仙。往下怎么说?”林之孝家的道:“他说底下‘贝’字,拆开不成一个‘见’字,可不是‘不见’了?因上头拆了‘当’字,叫快到当铺里找去。‘赏’字加一‘人’字,可不是‘偿’字?只要找着当铺就有人,有了人便赎了来,可不是偿还了吗。”众人道:“既这么着,就先往左近找起,横竖几个当铺都找遍了,少不得就有了。咱们有了东西,再问人就容易了。”李纨道:“只要东西,那怕不问人都使得。林嫂子,烦你就把测字的话快去告诉二奶奶,回了太太,先叫太太放心。就叫二奶奶快派人查去。”林家的答应了便走。
\end{parag}


\begin{parag}
    众人略安了一点儿神,呆呆的等岫烟回来。正呆等,只见跟宝玉的焙茗在门外招手儿,叫小丫头子快出来。那小丫头赶忙的出去了。焙茗便说道:“你快进去告诉我们二爷和里头太太奶奶姑娘们天大喜事。”那小丫头子道:“你快说罢,怎么这么累赘。”焙茗笑着拍手道:“我告诉姑娘,姑娘进去回了,咱们两个人都得赏钱呢。你打量什么,宝二爷的那块玉呀,我得了准信来了。”未知如何,下回分解。
\end{parag}