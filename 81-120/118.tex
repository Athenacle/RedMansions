\chap{一百一十八}{记微嫌舅兄欺弱女 惊谜语妻妾谏痴人}



\begin{parag}
    说话邢王二夫人听尤氏一段话,明知也难挽回。王夫人只得说道:“姑娘要行善,这也是前生的夙根,我们也实在拦不住。只是咱们这样人家的姑娘出了家,不成了事体。如今你嫂子说了准你修行,也是好处。却有一句话要说,那头发可以不剃的,只要自己的心真,那在头发上头呢。你想妙玉也是带发修行的,不知他怎样凡心一动,才闹到那个分儿。姑娘执意如此,我们就把姑娘住的房子便算了姑娘的静室。所有服侍姑娘的人也得叫他们来问:他若愿意跟的,就讲不得说亲配人,若不愿意跟的,另打主意。”惜春听了,收了泪,拜谢了邢王二夫人,李纨,尤氏等。王夫人说了,便问彩屏等谁愿跟姑娘修行。彩屏等回道:“太太们派谁就是谁。”王夫人知道不愿意,正在想人。袭人立在宝玉身后,想来宝玉必要大哭,防着他的旧病。岂知宝玉叹道:“真真难得。”袭人心里更自伤悲。宝钗虽不言语,遇事试探,见是执迷不醒,只得暗中落泪。王夫人才要叫了众丫头来问。忽见紫鹃走上前去,在王夫人面前跪下,回道:“刚才太太问跟四姑娘的姐姐,太太看着怎么样?”王夫人道:“这个如何强派得人的,谁愿意他自然就说出来了。”紫鹃道:“姑娘修行自然姑娘愿意,并不是别的姐姐们的意思。我有句话回太太,我也并不是拆开姐姐们,各人有各人的心。我服侍林姑娘一场,林姑娘待我也是太太们知道的,实在恩重如山,无以可报。他死了,我恨不得跟了他去。但是他不是这里的人,我又受主子家的恩典,难以从死。如今四姑娘既要修行,我就求太太们将我派了跟着姑娘,服侍姑娘一辈子。不知太太们准不准。若准了,就是我的造化了。”邢王二夫人尚未答言,只见宝玉听到那里,想起黛玉一阵心酸,眼泪早下来了。众人才要问他时,他又哈哈的大笑,走上来道:“我不该说的。这紫鹃蒙太太派给我屋里,我才敢说。求太太准了他罢,全了他的好心。”王夫人道:“你头里姊妹出了嫁,还哭得死去活来,如今看见四妹妹要出家,不但不劝,倒说好事,你如今到底是怎么个意思,我索性不明白了。”宝玉道:“四妹妹修行是已经准的了,四妹妹也是一定主意了。若是真的,我有一句话告诉太太,若是不定的,我就不敢混说了。”惜春道:“二哥哥说话也好笑,一个人主意不定便扭得过太太们来了?我也是象紫鹃的话,容我呢,是我的造化,不容我呢。还有一个死呢。那怕什么!二哥哥既有话,只管说。”宝玉道:“我这也不算什么泄露了,这也是一定的。我念一首诗给你们听听罢!”众人道:“人家苦得很的时侯,你倒来做诗。怄人!”宝玉道:“不是做诗,我到一个地方儿看了来的。你们听听罢。”众人道:“使得。你就念念,别顺着嘴儿胡诌。”宝玉也不分辩,便说道:
\end{parag}


\begin{poem}

    \begin{pl}
        勘破三春景不长,缁衣顿改昔年妆。
    \end{pl}


    \begin{pl}
        可怜绣户侯门女,独卧青灯古佛旁!
    \end{pl}

\end{poem}


\begin{parag}
    李纨宝钗听了,诧异道:“不好了,这人入了迷了。”王夫人听了这话,点头叹息,便问宝玉:“你到底是那里看来的?”宝玉不便说出来,回道:“太太也不必问,我自有见的地方。”王夫人回过味来,细细一想,便更哭起来道:“你说前儿是顽话,怎么忽然有这首诗?罢了,我知道了,你们叫我怎么样呢!我也没有法儿了,也只得由着你们罢!但是要等我合上了眼,各自干各自的就完了!”宝钗一面劝着,这个心比刀绞更甚,也掌不住便放声大哭起来。袭人已经哭的死去活来,幸亏秋纹扶着。宝玉也不啼哭,也不相劝,只不言语。贾兰贾环听到那里,各自走开。李纨竭力的解说:“总是宝兄弟见四妹妹修行,他想来是痛极了,不顾前后的疯话,这也作不得准的。独有紫鹃的事情准不准,好叫他起来。”王夫人道:“什么依不依,横竖一个人的主意定了,那也扭不过来的。可是宝玉说的也是一定的了。”紫鹃听了磕头。惜春又谢了王夫人。紫鹃又给宝玉宝钗磕了头。宝玉念声“阿弥陀佛!难得,难得。不料你倒先好了!”宝钗虽然有把持,也难掌住。只有袭人,也顾不得王夫人在上,便痛哭不止,说:“我也愿意跟了四姑娘去修行。”宝玉笑道:“你也是好心,但是你不能享这个清福的。”袭人哭道:“这么说,我是要死的了!”宝玉听到那里,倒觉伤心,只是说不出来。因时已五更,宝玉请王夫人安歇,李纨等各自散去。彩屏等暂且伏侍惜春回去,后来指配了人家。紫鹃终身伏侍,毫不改初。此是后话。
\end{parag}


\begin{parag}
    且言贾政扶了贾母灵柩一路南行,因遇着班师的兵将船只过境,河道拥挤,不能速行,在道实在心焦。幸喜遇见了海疆的官员,闻得镇海统制钦召回京,想来探春一定回家,略略解些烦心。只打听不出起程的日期,心里又烦燥。想到盘费算来不敷,不得已写书一封,差人到赖尚荣任上借银五百,叫人沿途迎上来应需用。那人去了几日,贾政的船才行得十数里。那家人回来,迎上船只,将赖尚荣的禀启呈上。书内告了多少苦处,备上白银五十两。贾政看了生气,即命家人立刻送还,将原书发回,叫他不必费心。那家人无奈,只得回到赖尚荣任所。
\end{parag}


\begin{parag}
    赖尚荣接到原书银两,心中烦闷,知事办得不周到,又添了一百,央求来人带回,帮着说些好话。岂知那人不肯带回,撂下就走了。赖尚荣心下不安,立刻修书到家,回明他父亲,叫他设法告假赎出身来。于是赖家托了贾蔷贾芸等在王夫人面前乞恩放出。贾蔷明知不能,过了一日,假说王夫人不依的话回复了。赖家一面告假,一面差人到赖尚荣任上,叫他告病辞官。王夫人并不知道。
\end{parag}


\begin{parag}
    那贾芸听见贾蔷的假话,心里便没想头,连日在外又输了好些银钱,无所抵偿,便和贾环相商。贾环本是一个钱没有的,虽是赵姨娘积蓄些微,早被他弄光了,那能照应人家。便想起凤姐待他刻薄,要趁贾琏不在家要摆布巧姐出气,遂把这个当叫贾芸来上,故意的埋怨贾芸道:“你们年纪又大,放着弄银钱的事又不敢办,倒和我没有钱的人相商。”贾芸道:“三叔,你这话说的倒好笑,咱们一块儿顽,一块儿闹,那里有银钱的事。”贾环道:“不是前儿有人说是外藩要买个偏房,你们何不和王大舅商量把巧姐说给他呢?”贾芸道:“叔叔,我说句招你生气的话,外藩花了钱买人,还想能和咱们走动么。”贾环在贾芸耳边说了些话,贾芸虽然点头,只道贾环是小孩子的话,也不当事。恰好王仁走来说道:“你们两个人商量些什么,瞒着我么?”贾芸便将贾环的话附耳低言的说了。王仁拍手道:“这倒是一种好事,又有银子。只怕你们不能,若是你们敢办,我是亲舅舅,做得主的。只要环老三在大太太跟前那么一说,我找邢大舅再一说,太太们问起来你们齐打伙说好就是了。”贾环等商议定了,王仁便去找邢大舅,贾芸便去回邢王二夫人,说得锦上添花。
\end{parag}


\begin{parag}
    王夫人听了虽然入耳,只是不信。邢夫人听得邢大舅知道,心里愿意,便打发人找了邢大舅来问他。那邢大舅已经听了王仁的话,又可分肥,便在邢夫人跟前说道:“若说这位郡王,是极有体面的。若应了这门亲事,虽说是不是正配,保管一过了门,姊夫的官早复了,这里的声势又好了。”邢夫人本是没主意人,被傻大舅一番假话,哄得心动,请了王仁来一问,更说得热闹。于是邢夫人倒叫人出去追着贾芸去说。王仁即刻找了人去到外藩公馆说了。那外藩不知底细,便要打发人来相看。贾芸又钻了相看的人,说明“原是瞒着合宅的,只是王府相亲。等到成了,他祖母作主,亲舅舅的保山,是不怕的。”那相看的人应了。贾芸便送信与邢夫人,并回了王夫人。那李纨宝钗等不知原故,只道是件好事,也都欢喜。
\end{parag}


\begin{parag}
    那日果然来了几个女人,都是艳妆丽服。邢夫人接了进去,叙了些闲话。那来人本知是个诰命,也不敢待慢。邢夫人因事未定,也没有和巧姐说明,只说有亲戚来瞧,叫他去见。那巧姐到底是个小孩子,那管这些,便跟了奶妈过来。平儿不放心,也跟着来。只见有两个宫人打扮的,见了巧姐便浑身上下一看,更又起身来拉着巧姐的手又瞧了一遍,略坐了一坐就走了。倒把巧姐看得羞臊,回到房中纳闷,想来没有这门亲戚,便问平儿。平儿先看见来头,却也猜着八九必是相亲的。“但是二爷不在家,大太太作主,到底不知是那府里的。若说是对头亲,不该这样相看。瞧那几个人的来头,不象是本支王府,好象是外头路数如今且不必和姑娘说明,且打听明白再说。”
\end{parag}


\begin{parag}
    平儿心下留神打听。那些丫头婆子都是平儿使过的,平儿一问,所有听见外头的风声都告诉了。平儿便吓的没了主意,虽不和巧姐说,便赶着去告诉了李纨宝钗,求他二人告诉王夫人。王夫人知道这事不好,便和邢夫人说知。怎奈邢夫人信了兄弟并王仁的话,反疑心王夫人不是好意,便说:“孙女儿也大了,现在琏儿不在家,这件事我还做得主。况且是他亲舅爷爷和他亲舅舅打听的,难道倒比别人不真么!我横竖是愿意的。倘有什么不好,我和琏儿也抱怨不着别人!”
\end{parag}


\begin{parag}
    王夫人听了这些话,心下暗暗生气,勉强说些闲话,便走了出来,告诉了宝钗,自己落泪。宝玉劝道:“太太别烦恼,这件事我看来是不成的。这又是巧姐儿命里所招,只求太太不管就是了。”王夫人道:“你一开口就是疯话。人家说定了就要接过去。若依平儿的话,你琏二哥可不抱怨我么。别说自己的侄孙女儿,就是亲戚家的,也是要好才好。邢姑娘是我们作媒的,配了你二大舅子,如今和和顺顺的过日子不好么。那琴姑娘梅家娶了去,听见说是丰衣足食的很好。就是史姑娘是他叔叔的主意,头里原好,如今姑爷痨病死了,你史妹妹立志守寡,也就苦了。若是巧姐儿错给了人家儿,可不是我的心坏?”正说着,平儿过来瞧宝钗,并探听邢夫人的口气。王夫人将邢夫人的话说了一遍。平儿呆了半天,跪下求道:“巧姐儿终身全仗着太太。若信了人家的话,不但姑娘一辈子受了苦,便是琏二爷回来怎么说呢!”王夫人道:“你是个明白人,起来,听我说。巧姐儿到底是大太太孙女儿,他要作主,我能够拦他么?”宝玉劝道:“无妨碍的,只要明白就是了。”平儿生怕宝玉疯颠嚷出来,也并不言语,回了王夫人竟自去了。
\end{parag}


\begin{parag}
    这里王夫人想到烦闷,一阵心痛,叫丫头扶着勉强回到自己房中躺下,不叫宝玉宝钗过来,说睡睡就好的。自己却也烦闷,听见说李婶娘来了也不及接待。只见贾兰进来请了安,回道:“今早爷爷那里打发人带了一封书子来,外头小子们传进来的。我母亲接了正要过来,因我老娘来了,叫我先呈给太太瞧,回来我母亲就过来来回太太。还说我老娘要过来呢。”说着,一面把书子呈上。王夫人一面接书,一面问道:“你老娘来作什么?”贾兰道:“我也不知道。我只见我老娘说,我三姨儿的婆婆家有什么信儿来了。”王夫人听了,想起来还是前次给甄宝玉说了李绮,后来放定下茶,想来此时甄家要娶过门,所以李婶娘来商量这件事情,便点点头儿。一面拆开书信,见上面写着道:
\end{parag}


\begin{parag}
    近因沿途俱系海疆凯旋船只,不能迅速前行。闻探姐随翁婿来都,不知曾有信否?前接到琏侄手禀,知大老爷身体欠安,亦不知已有确信否?宝玉兰哥场期已近,务须实心用功,不可怠惰。老太太灵柩抵家,尚需日时。我身体平善,不必挂念。此谕宝玉等知道。月日手书。蓉儿另禀。王夫人看了,仍旧递给贾兰,说:“你拿去给你二叔瞧瞧,还交给你母亲罢。”正说着,李纨同李婶过来。请安问好毕,王夫人让了坐。李婶娘便将甄家要娶李绮的话说了一遍。大家商议了一会子。李纨因问王夫人道:“老爷的书子太太看过了么?”王夫人道:“看过了。”贾兰便拿着给他母亲瞧。李纨看了道:“三姑娘出门了好几年,总没有来,如今要回京了。太太也放了好些心。”王夫人道:“我本是心痛,看见探丫头要回来了,心里略好些。只是不知几时才到。”李婶娘便问了贾政在路好。李纨因向贾兰道:“哥儿瞧见了?场期近了,你爷爷掂记的什么似的。你快拿了去给二叔叔瞧去罢。”李婶娘道:“他们爷儿两个又没进过学,怎么能下场呢?”王夫人道:“他爷爷做粮道的起身时,给他们爷儿两个援了例监了。”李婶娘点头。贾兰一面拿著书子出来,来找宝玉。
\end{parag}


\begin{parag}
    却说宝玉送了王夫人去后,正拿着《秋水》一篇在那里细玩。宝钗从里间走出,见他看的得意忘言,便走过来一看,见是这个,心里着实烦闷。细想他只顾把这些出世离群的话当作一件正经事,终久不妥。看他这种光景,料劝不过来,便坐在宝玉旁边怔怔的坐着。宝玉见他这般,便道:“你这又是为什么?”宝钗道:“我想你我既为夫妇,你便是我终身的倚靠,却不在情欲之私。论起荣华富贵,原不过是过眼烟云,但自古圣贤,以人品根柢为重。”宝玉也没听完,把那书本搁在旁边,微微的笑道:“据你说人品根柢,又是什么古圣贤,你可知古圣贤说过‘不失其赤子之心’。那赤子有什么好处,不过是无知无识无贪无忌。我们生来已陷溺在贪嗔痴爱中,犹如污泥一般,怎么能跳出这般尘网。如今才晓得‘聚散浮生’四字,古人说了,不曾提醒一个。既要讲到人品根柢,谁是到那太初一步地位的!”宝钗道:“你既说‘赤子之心’,古圣贤原以忠孝为赤子之心,并不是遁世离群无关无系为赤子之心。尧舜禹汤周孔时刻以救民济世为心,所谓赤子之心,原不过是‘不忍’二字。若你方才所说的,忍于抛弃天伦,还成什么道理?”宝玉点头笑道:“尧舜不强巢许,武周不强夷齐。”宝钗不等他说完,便道:“你这个话益发不是了。古来若都是巢许夷齐,为什么如今人又把尧舜周孔称为圣贤呢!况且你自比夷齐,更不成话,伯夷叔齐原是生在商末世,有许多难处之事,所以才有托而逃。当此圣世,咱们世受国恩,祖父锦衣玉食,况你自有生以来,自去世的老太太以及老爷太太视如珍宝。你方才所说,自己想一想是与不是。”宝玉听了也不答言,只有仰头微笑。宝钗因又劝道:“你既理屈词穷,我劝你从此把心收一收,好好的用用功。但能搏得一第,便是从此而止,也不枉天恩祖德了。”宝玉点了点头,叹了口气说道:“一第呢,其实也不是什么难事,倒是你这个‘从此而止,不枉天恩祖德’却还不离其宗。”宝钗未及答言,袭人过来说道:“刚才二奶奶说的古圣先贤,我们也不懂。我只想着我们这些人从小儿辛辛苦苦跟着二爷,不知陪了多少小心,论起理来原该当的,但只二爷也该体谅体谅。况二奶奶替二爷在老爷太太跟前行了多少孝道,就是二爷不以夫妻为事,也不可太辜负了人心。至于神仙那一层更是谎话,谁见过有走到凡间来的神仙呢!那里来的这么个和尚,说了些混话,二爷就信了真。二爷是读书的人,难道他的话比老爷太太还重么!”宝玉听了,低头不语。
\end{parag}


\begin{parag}
    袭人还要说时,只听外面脚步走响,隔着窗户问道:“二叔在屋里呢么?”宝玉听了,是贾兰的声音,便站起来笑道:“你进来罢。”宝钗也站起来。贾兰进来笑容可掬的给宝玉宝钗请了安,问了袭人的好,——袭人也问了好——便把书子呈给宝玉瞧。宝玉接在手中看了,便道:“你三姑姑回来了。”贾兰道:“爷爷既如此写,自然是回来的了。”宝玉点头不语,默默如有所思。贾兰便问:“叔叔看见爷爷后头写的叫咱们好生念书了?叔叔这一程子只怕总没作文章罢?”宝玉笑道:“我也要作几篇熟一熟手,好去诓这个功名。”贾兰道:“叔叔既这样,就拟几个题目,我跟着叔叔作作,也好进去混场,别到那时交了白卷子惹人笑话。不但笑话我,人家连叔叔都要笑话了。”宝玉道:“你也不至如此。”说着,宝钗命贾兰坐下。宝玉仍坐在原处,贾兰侧身坐了。两个谈了一回文,不觉喜动颜色。宝钗见他爷儿两个谈得高兴,便仍进屋里去了。心中细想宝玉此时光景,或者醒悟过来了,只是刚才说话,他把那“从此而止”四字单单的许可,这又不知是什么意思了。宝钗尚自犹豫,惟有袭人看他爱讲文章,提到下场,更又欣然。心里想道:“阿弥陀佛!好容易讲四书似的才讲过来了!”这里宝玉和贾兰讲文,莺儿沏过茶来,贾兰站起来接了。又说了一会子下场的规矩并请甄宝玉在一处的话,宝玉也甚似愿意。一时贾兰回去,便将书子留给宝玉了。
\end{parag}


\begin{parag}
    那宝玉拿著书子,笑嘻嘻走进来递给麝月收了,便出来将那本《庄子》收了,把几部向来最得意的,如《参同契》《元命苞》《五灯会元》之类,叫出麝月秋纹莺儿等都搬了搁在一边。宝钗见他这番举动,甚为罕异,因欲试探他,便笑问道:“不看他倒是正经,但又何必搬开呢。”宝玉道:“如今才明白过来了。这些书都算不得什么,我还要一火焚之,方为干净。”宝钗听了更欣喜异常。只听宝玉口中微吟道:“内典语中无佛性,金丹法外有仙丹。”宝钗也没很听真,只听得”无佛性”“有仙丹”几个字,心中转又狐疑,且看他作何光景。宝玉便命麝月秋纹等收拾一间静室,把那些语录名稿及应制诗之类都找出来搁在静室中,自己却当真静静的用起功来。宝钗这才放了心。
\end{parag}


\begin{parag}
    那袭人此时真是闻所未闻,见所未见,便悄悄的笑着向宝钗道:“到底奶奶说话透彻,只一路讲究,就把二爷劝明白了。就只可惜迟了一点儿,临场太近了。”宝钗点头微笑道:“功名自有定数,中与不中倒也不在用功的迟早。但愿他从此一心巴结正路,把从前那些邪魔永不沾染就是好了。”说到这里,见房里无人,便悄说道:“这一番悔悟回来固然很好,但只一件,怕又犯了前头的旧病,和女孩儿们打起交道来,也是不好。”袭人道:“奶奶说的也是。二爷自从信了和尚,才把这些姐妹冷淡了,如今不信和尚,真怕又要犯了前头的旧病呢。我想奶奶和我二爷原不大理会,紫鹃去了,如今只他们四个,这里头就是五儿有些个狐媚子,听见说他妈求了大奶奶和奶奶,说要讨出去给人家儿呢。但是这两天到底在这里呢。麝月秋纹虽没别的,只是二爷那几年也都有些顽顽皮皮的。如今算来只有莺儿二爷倒不大理会,况且莺儿也稳重。我想倒茶弄水只叫莺儿带着小丫头们伏侍就够了,不知奶奶心里怎么样。”宝钗道:“我也虑的是这些,你说的倒也罢了。”从此便派莺儿带着小丫头伏侍。
\end{parag}


\begin{parag}
    那宝玉却也不出房门,天天只差人去给王夫人请安。王夫人听见他这番光景,那一种欣慰之情,更不待言了。到了八月初三,这一日正是贾母的冥寿。宝玉早晨过来磕了头,便回去,仍到静室中去了。饭后,宝钗袭人等都和姊妹们跟着邢王二夫人在前面屋里说闲话儿。宝玉自在静室冥心危坐,忽见莺儿端了一盘瓜果进来说:“太太叫人送来给二爷吃的。这是老太太的克什。”宝玉站起来答应了,复又坐下,便道:“搁在那里罢。”莺儿一面放下瓜果一面悄悄向宝玉道:“太太那里夸二爷呢。”宝玉微笑。莺儿又道:“太太说了,二爷这一用功,明儿进场中了出来,明年再中了进士,作了官,老爷太太可就不枉了盼二爷了。”宝玉也只点头微笑。莺儿忽然想起那年给宝玉打络子的时候宝玉说的话来,便道:“真要二爷中了,那可是我们姑奶奶的造化了。二爷还记得那一年在园子里,不是二爷叫我打梅花络子时说的,我们姑奶奶后来带着我不知到那一个有造化的人家儿去呢。如今二爷可是有造化的罢咧。”宝玉听到这里,又觉尘心一动,连忙敛神定息,微微的笑道:“据你说来,我是有造化的,你们姑娘也是有造化的,你呢?”莺儿把脸飞红了,勉强道:“我们不过当丫头一辈子罢咧,有什么造化呢!”宝玉笑道:“果然能够一辈子是丫头,你这个造化比我们还大呢!”莺儿听见这话似乎又是疯话了,恐怕自己招出宝玉的病根来,打算着要走。只见宝玉笑着说道:“傻丫头,我告诉你罢。”未知宝玉又说出什么话来,且听下回分解。
\end{parag}