\chap{九十一}{纵淫心宝蟾工设计 布疑阵宝玉妄谈禅}



\begin{parag}
    话说薛蝌正在狐疑,忽听窗外一笑,唬了一跳,心中想道:“不是宝蟾,定是金桂。只不理他们,看他们有什么法儿。”听了半日,却又寂然无声。自己也不敢吃那酒果。掩上房门,刚要脱衣时,只听见窗纸上微微一响。薛蝌此时被宝蟾鬼混了一阵,心中七上八下,竟不知是如何是可。听见窗纸微响,细看时,又无动静,自己反倒疑心起来,掩了怀,坐在灯前,呆呆的细想,又把那果子拿了一块,翻来覆去的细看。猛回头,看见窗上纸湿了一块,走过来觑着眼看时,冷不防外面往里一吹,把薛蝌唬了一大跳。听得吱吱的笑声,薛蝌连忙把灯吹灭了,屏息而卧。只听外面一个人说道:“二爷为什么不喝酒吃果子,就睡了?”这句话仍是宝蟾的语音。薛蝌只不作声装睡。又隔有两句话时,又听得外面似有恨声道:“天下那里有这样没造化的人。”薛蝌听了是宝蟾又似是金桂的语音,这才知道他们原来是这一番意思,翻来覆去,直到五更后才睡着了。刚到天明,早有人来扣门。薛蝌忙问是谁,外面也不答应。薛蝌只得起来,开了门看时,却是宝蟾,拢着头发,掩着怀,穿一件片锦边琵琶襟小紧身,上面系一条松花绿半新的汗巾,下面并未穿裙,正露着石榴红洒花夹裤,一双新绣红鞋。原来宝蟾尚未梳洗,恐怕人见,赶早来取家伙。薛蝌见他这样打扮,便走进来,心中又是一动,只得陪笑问道:“怎么这样早就起来了?”宝蟾把脸红着,并不答言,只管把果子折在一个碟子里,端着就走。薛蝌见他这般,知是昨晚的原故,心里想道:“这也罢了。倒是他们恼了,索性死了心,也省得来缠。”于是把心放下,唤人舀水洗脸。自己打算在家里静坐两天,一则养养心神,二则出去怕人找他。原来和薛蟠好的那些人因见薛家无人,只有薛蝌在那里办事,年纪又轻,便生许多觊觎之心。也有想插在里头做跑腿的,也有能做状子的,认得一二个书役的,要给他上下打点的,甚至有叫他在内趁钱的,也有造作谣言恐吓的:种种不一。薛蝌见了这些人,远远躲避,又不敢面辞,恐怕激出意外之变,只好藏在家中,听候传详。不提。
\end{parag}


\begin{parag}
    且说金桂昨夜打发宝蟾送了些酒果去探探薛蝌的消息,宝蟾回来将薛蝌的光景一一的说了。金桂见事有些不大投机,便怕白闹一场,反被宝蟾瞧不起,欲把两三句话遮饰改过口来,又可惜了这个人,心里倒没了主意,怔怔的坐着。那知宝蟾亦知薛蟠难以回家,正欲寻个头路,因怕金桂拿他,所以不敢透漏。今见金桂所为先已开了端了,他便乐得借风使船,先弄薛蝌到手,不怕金桂不依,所以用言挑拨。见薛蝌似非无情,又不甚兜揽,一时也不敢造次,后来见薛蝌吹灯自睡,大觉扫兴,回来告诉金桂,看金桂有甚方法,再作道理。及见金桂怔怔的,似乎无技可施,他也只得陪金桂收拾睡了。夜里那里睡得着,翻来覆去,想出一个法子来:不如明儿一早起来,先去取了家伙,却自己换上一两件动人的衣服,也不梳洗,越显出一番娇媚来。只看薛蝌的神情,自己反倒装出一番恼意,索性不理他。那薛蝌若有悔心,自然移船泊岸,不愁不先到手。及至见了薛蝌,仍是昨晚这般光景,并无邪僻之意,自己只得以假为真,端了碟子回来,却故意留下酒壶,以为再来搭转之地。只见金桂问道:“你拿东西去有人碰见么?”宝蟾道:“没有。”“二爷也没问你什么?”宝蟾道:“也没有。”金桂因一夜不曾睡着,也想不出一个法子来,只得回思道:“若作此事,别人可瞒,宝蟾如何能瞒?不如我分惠于他,他自然没有不尽心的。我又不能自去,少不得要他作脚,倒不如和他商量一个稳便主意。”因带笑说道:“你看二爷到底是个怎么样的人?”宝蟾道:“倒象个糊涂人。”金桂听了笑道:“你如何说起爷们来了。”宝蟾也笑道:“他辜负奶奶的心,我就说得他。”金桂道:“他怎么辜负我的心,你倒得说说。”宝蟾道:“奶奶给他好东西吃,他倒不吃,这不是辜负奶奶的心么。”说着,却把眼溜着金桂一笑。金桂道:“你别胡想。我给他送东西,为大爷的事不辞劳苦,我所以敬他,又怕人说瞎话,所以问你。你这些话向我说,我不懂是什么意思。”宝蟾笑道:“奶奶别多心,我是跟奶奶的,还有两个心么。但是事情要密些,倘或声张起来,不是顽的。”金桂也觉得脸飞红了,因说道:“你这个丫头就不是个好货!想来你心里看上了,却拿我作筏子,是不是呢?”宝蟾道:“只是奶奶那么想罢咧,我倒是替奶奶难受。奶奶要真瞧二爷好,我倒有个主意。奶奶想,那个耗子不偷油呢,他也不过怕事情不密,大家闹出乱子来不好看。依我想,奶奶且别性急,时常在他身上不周不备的去处张罗张罗。他是个小叔子,又没娶媳妇儿,奶奶就多尽点心儿和他贴个好儿,别人也说不出什么来。过几天他感奶奶的情,他自然要谢候奶奶。那时奶奶再备点东西儿在咱们屋里,我帮着奶奶灌醉了他,怕跑了他?他要不应,咱们索性闹起来,就说他调戏奶奶。他害怕,他自然得顺着咱们的手儿。他再不应,他也不是人,咱们也不至白丢了脸面。奶奶想怎么样?”金桂听了这话,两颧早已红晕了,笑骂道:“小蹄子,你倒偷过多少汉子的似的,怪不得大爷在家时离不开你。”宝蟾把嘴一撇,笑说道:“罢哟,人家倒替奶奶拉纤,奶奶倒往我们说这个话咧。”从此金桂一心笼络薛蝌,倒无心混闹了。家中也少觉安静。
\end{parag}


\begin{parag}
    当日宝蟾自去取了酒壶,仍是稳稳重重一脸的正气。薛蝌偷眼看了,反倒后悔,疑心或者是自己错想了他们,也未可知。果然如此,倒辜负了他这一番美意,保不住日后倒要和自己也闹起来,岂非自惹的呢。过了两天,甚觉安静。薛蝌遇见宝蟾,宝蟾便低头走了,连眼皮儿也不抬,遇见金桂,金桂却一盆火儿的赶着。薛蝌见这般光景,反倒过意不去。这且不表。
\end{parag}


\begin{parag}
    且说宝钗母女觉得金桂几天安静,待人忽亲热起来,一家子都为罕事。薛姨妈十分欢喜,想到必是薛蟠娶这媳妇时冲犯了什么,才败坏了这几年。目今闹出这样事来,亏得家里有钱,贾府出力,方才有了指望。媳妇儿忽然安静起来,或者是蟠儿转过运气来了,也未可知,于是自己心里倒以为希有之奇。这日饭后扶了同贵过来,到金桂房里瞧瞧。走到院中,只听一个男人和金桂说话。同贵知机,便说道:“大奶奶,老太太过来了。”说着已到门口。只见一个人影儿在房门后一躲,薛姨妈一吓,倒退了出来。金桂道:“太太请里头坐。没有外人,他就是我的过继兄弟,本住在屯里,不惯见人,因没有见过太太。今儿才来,还没去请太太的安。”薛姨妈道:“既是舅爷,不妨见见。”金桂叫兄弟出来,见了薛姨妈,作了一个揖,问了好。薛姨妈也问了好,坐下叙起话来。薛姨妈道:“舅爷上京几时了?”那夏三道:“前月我妈没有人管家,把我过继来的。前日才进京,今日来瞧姐姐。”薛姨妈看那人不尴尬,于是略坐坐儿,便起身道:“舅爷坐着罢。”回头向金桂道:“舅爷头上末下的来,留在咱们这里吃了饭再去罢。”金桂答应着,薛姨妈自去了。金桂见婆婆去了,便向夏三道:“你坐着,今日可是过了明路的了,省得我们二爷查考你。我今日还叫你买些东西,只别叫众人看见。”夏三道:“这个交给我就完了。你要什么,只要有钱,我就买得来。”金桂道:“且别说嘴,你买上了当,我可不收。”说着,二人又笑了一回,然后金桂陪夏三吃了晚饭,又告诉他买的东西,又嘱咐一回,夏三自去。从此夏三往来不绝。虽有个年老的门上人,知是舅爷,也不常回,从此生出无限风波,这是后话。不表。
\end{parag}


\begin{parag}
    一日薛蟠有信寄回,薛姨妈打开叫宝钗看时,上写:
\end{parag}


\begin{qute2sp}
    男在县里也不受苦,母亲放心。但昨日县里书办说,府里已经准详,想是我们的情到了。岂知府里详上去,道里反驳下来。亏得县里主文相公好,即刻做了回文顶上去了。那道里却把知县申饬。现在道里要亲提,若一上去,又要吃苦。必是道里没有托到。母亲见字,快快托人求道爷去。还叫兄弟快来,不然就要解道。银子短不得。火速,火速。
\end{qute2sp}


\begin{parag}
    薛姨妈听了,又哭了一场,自不必说。薛蝌一面劝慰,一面说道:“事不宜迟。”薛姨妈没法,只得叫薛蝌到县照料,命人即便收拾行李,兑了银子,家人李祥本在那里照应的,薛蝌又同了一个当中伙计连夜起程。
\end{parag}


\begin{parag}
    那时手忙脚乱,虽有下人办理,宝钗又恐他们思想不到,亲来帮着,直闹至四更才歇。到底富家女子娇养惯的,心上又急,又苦劳了一会,晚上就发烧。到了明日,汤水都吃不下。莺儿去回了薛姨妈。薛姨妈急来看时,只见宝钗满面通红,身如燔灼,话都不说。薛姨妈慌了手脚,便哭得死去活来。宝琴扶着劝薛姨妈。秋菱也泪如泉涌,只管叫着。宝钗不能说话,手也不能摇动,眼干鼻塞。叫人请医调治,渐渐苏醒回来。薛姨妈等大家略略放心。早惊动荣宁两府的人,先是凤姐打发人送十香返魂丹来,随后王夫人又送至宝丹来。贾母邢王二夫人以及尤氏等都打发丫头来问候,却都不叫宝玉知道。一连治了七八天,终不见效,还是他自己想起冷香丸,吃了三丸,才得病好。后来宝玉也知道了,因病好了,没有瞧去。
\end{parag}


\begin{parag}
    那时薛蝌又有信回来,薛姨妈看了,怕宝钗耽忧,也不叫他知道。自己来求王夫人,并述了一会子宝钗的病。薛姨妈去后,王夫人又求贾政。贾政道:“此事上头可托,底下难托,必须打点才好。”王夫人又提起宝钗的事来,因说道:“这孩子也苦了。既是我家的人了,也该早些娶了过来才是,别叫他糟踏坏了身子。”贾政道:“我也是这么想。但是他家乱忙,况且如今到了冬底,已经年近岁逼,不无各自要料理些家务。今冬且放了定,明春再过礼,过了老太太的生日,就定日子娶。你把这番话先告诉薛姨太太。”王夫人答应了。到了明日,王夫人将贾政的话向薛姨妈述了。薛姨妈想着也是。到了饭后,王夫人陪着来到贾母房中,大家让了坐。贾母道:“姨太太才过来?”薛姨妈道:“还是昨儿过来的。因为晚了,没得过来给老太太请安。”王夫人便把贾政昨夜所说的话向贾母述了一遍,贾母甚喜。说着,宝玉进来了。贾母便问道:“吃了饭了没有?”宝玉道:“才打学房里回来,吃了要往学房里去,先见见老太太。又听见说姨妈来了,过来给姨妈请请安。”因问:“宝姐姐可大好了?”薛姨妈笑道:“好了。”原来方才大家正说着,见宝玉进来,都煞住了。宝玉坐了坐,见薛姨妈情形不似从前亲热,”虽是此刻没有心情,也不犯大家都不言语。”满腹猜疑,自往学中去了。
\end{parag}


\begin{parag}
    晚间回来,都见过了,便往潇湘馆来。掀帘进去,紫鹃接着,见里间屋内无人,宝玉道:“姑娘那里去了?”紫鹃道:“上屋里去了。知道姨太太过来,姑娘请安去了。二爷没有到上屋里去么?”宝玉道:“我去了来的,没有见你姑娘。”紫鹃道:“这也奇了。”宝玉问:“姑娘到底那里去了?”紫鹃道:“不定。”宝玉往外便走。刚出屋门,只见黛玉带着雪雁,冉冉而来。宝玉道:“妹妹回来了。”缩身退步进来。
\end{parag}


\begin{parag}
    黛玉进来,走入里间屋内,便请宝玉里头坐。紫鹃拿了一件外罩换上,然后坐下,问道:“你上去看见姨妈没有?”宝玉道:“见过了。”黛玉道:“姨妈说起我没有?”宝玉道:“不但没有说起你,连见了我也不象先时亲热。今日我问起宝姐姐病来,他不过笑了一笑,并不答言。难道怪我这两天没有去瞧他么。”黛玉笑了一笑道:“你去瞧过没有?”宝玉道:“头几天不知道,这两天知道了,也没有去。”黛玉道:“可不是。”宝玉道:“老太太不叫我去,太太也不叫我去,老爷又不叫我去,我如何敢去。若是象从前这扇小门走得通的时候,要我一天瞧他十趟也不难。如今把门堵了,要打前头过去,自然不便了。”黛玉道:“他那里知道这个原故。”宝玉道:“宝姐姐为人是最体谅我的。”黛玉道:“你不要自己打错了主意。若论宝姐姐,更不体谅,又不是姨妈病,是宝姐姐病。向来在园中,做诗赏花饮酒,何等热闹,如今隔开了,你看见他家里有事了,他病到那步田地,你象没事人一般,他怎么不恼呢。”宝玉道:“这样难道宝姐姐便不和我好了不成?”黛玉道:“他和你好不好我却不知,我也不过是照理而论。”宝玉听了,瞪着眼呆了半晌。黛玉看见宝玉这样光景,也不睬他,只是自己叫人添了香,又翻出书来细看了一会。只见宝玉把眉一皱,把脚一跺道:“我想这个人生他做什么!天地间没有了我,倒也干净!”黛玉道:“原是有了我,便有了人,有了人,便有无数的烦恼生出来,恐怖,颠倒,梦想,更有许多缠碍。——才刚我说的都是顽话,你不过是看见姨妈没精打彩,如何便疑到宝姐姐身上去?姨妈过来原为他的官司事情心绪不宁,那里还来应酬你?都是你自己心上胡思乱想,钻入魔道里去了。”宝玉豁然开朗,笑道:“很是,很是。你的性灵比我竟强远了,怨不得前年我生气的时候,你和我说过几句禅语,我实在对不上来。我虽丈六金身,还借你一茎所化。”黛玉乘此机会说道:“我便问你一句话,你如何回答?”宝玉盘着腿,合着手,闭着眼,嘘着嘴道:“讲来。”黛玉道:“宝姐姐和你好你怎么样?宝姐姐不和你好你怎么样?宝姐姐前儿和你好,如今不和你好你怎么样?今儿和你好,后来不和你好你怎么样?你和他好他偏不和你好你怎么样?你不和他好他偏要和你好你怎么样?”宝玉呆了半晌,忽然大笑道:“任凭弱水三千,我只取一瓢饮。”黛玉道:“瓢之漂水奈何?”宝玉道:“非瓢漂水,水自流,瓢自漂耳!”黛玉道:“水止珠沉,奈何?”宝玉道:“禅心已作沾泥絮,莫向春风舞鹧鸪。”黛玉道:“禅门第一戒是不打诳语的。”宝玉道:“有如三宝。”黛玉低头不语。只听见檐外老鸹呱呱的叫了几声,便飞向东南上去,宝玉道:“不知主何吉凶。”黛玉道:“人有吉凶事,不在鸟声中。”忽见秋纹走来说道:“请二爷回去。老爷叫人到园里来问过,说二爷打学里回来了没有。袭人姐姐只说已经来了。快去罢。”吓得宝玉站起身来往外忙走,黛玉也不敢相留。未知何事,下回分解。
\end{parag}