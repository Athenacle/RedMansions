\chap{一百一十二}{活冤孽妙尼遭大劫 死讎仇趙妾赴冥曹}



\begin{parag}
    話說鳳姐命捆起上夜衆女人送營審問,女人跪地哀求。林之孝同賈芸道:“你們求也無益。老爺派我們看家,沒有事是造化,如今有了事,上下都擔不是,誰救得你。若說是周瑞的乾兒子,連太太起,裏裏外外的都不乾淨。”鳳姐喘吁吁的說道:“這都是命裏所招,和他們說什麼,帶了他們去就是了。這丟的東西你告訴營裏去說,實在是老太太的東西,問老爺們才知道。等我們報了去,請了老爺們回來,自然開了失單送來。文官衙門裏我們也是這樣報。”賈芸林之孝答應出去。
\end{parag}


\begin{parag}
    惜春一句話也沒有,只是哭道:“這些事我從來沒有聽見過,爲什麼偏偏碰在咱們兩個人身上!明兒老爺太太回來叫我怎麼見人!說把家裏交給咱們,如今鬧到這個分兒,還想活着麼!”鳳姐道:“咱們願意嗎!現在有上夜的人在那裏。”惜春道:“你還能說,況且你又病着。我是沒有說的。這都是我大嫂子害了我的,他攛掇着太太派我看家的。如今我的臉擱在那裏呢!”說着,又痛哭起來。鳳姐道:“姑娘,你快別這麼想,若說沒臉,大家一樣的。你若這麼糊塗想頭,我更擱不住了。”
\end{parag}


\begin{parag}
    二人正說着,只聽見外頭院子裏有人大嚷的說道:“我說那三姑六婆是再要不得的,我們甄府裏從來是一概不許上門的,不想這府裏倒不講究這個呢。昨兒老太太的殯纔出去,那個什麼庵裏的尼姑死要到咱們這裏來,我吆喝着不准他們進來,腰門上的老婆子倒罵我,死央及叫放那姑子進去。那腰門子一會兒開着,一會兒關着,不知做什麼,我不放心沒敢睡,聽到四更這裏就嚷起來。我來叫門倒不開了,我聽見聲兒緊了,打開了門,見西邊院子裏有人站着,我便趕走打死了。我今兒才知道,這是四姑奶奶的屋子。那個姑子就在裏頭,今兒天沒亮溜出去了,可不是那姑子引進來的賊麼。”平兒等聽着,都說:“這是誰這麼沒規矩?姑娘奶奶都在這裏,敢在外頭混嚷嗎。”鳳姐道:“你聽見說‘他甄府裏’,別就是甄家薦來的那個厭物罷。”惜春聽得明白,更加心裏過不的。鳳姐接着問惜春道:“那個人混說什麼姑子,你們那裏弄了個姑子住下了?”惜春便將妙玉來瞧他留着下棋守夜的話說了。鳳姐道:“是他麼,他怎麼肯這樣,是再沒有的話。但是叫這討人嫌的東西嚷出來,老爺知道了也不好。”惜春愈想愈怕,站起來要走。鳳姐雖說坐不住,又怕惜春害怕弄出事來,只得叫他先別走。“且看着人把偷剩下的東西收起來,再派了人看着纔好走呢。”平兒道:“咱們不敢收,等衙門裏來了踏看了纔好收呢。咱們只好看着。但只不知老爺那裏有人去了沒有?”鳳姐道:“你叫老婆子問去。”一回進來說:“林之孝是走不開,家下人要伺候查驗的,再有的是說不清楚的,已經芸二爺去了。”鳳姐點頭,同惜春坐着發愁。
\end{parag}


\begin{parag}
    且說那夥賊原是何三等邀的,偷搶了好些金銀財寶接運出去,見人追趕,知道都是那些不中用的人,要往西邊屋內偷去,在窗外看見裏面燈光底下兩個美人:一個姑娘,一個姑子。那些賊那顧性命,頓起不良,就要踹進來,因見包勇來趕,才獲贓而逃。只不見了何三。大家且躲入窩家。到第二天打聽動靜,知是何三被他們打死,已經報了文武衙門。這裏是躲不住的,便商量趁早規入海洋大盜一處,去若遲了,通緝文書一行,關津上就過不去了。
\end{parag}


\begin{parag}
    內中一個人膽子極大,便說:“咱們走是走,我就只捨不得那個姑子,長的實在好看。不知是那個庵裏的雛兒呢?”一個人道:“啊呀,我想起來了,必就是賈府園裏的什麼櫳翠庵裏的姑子。不是前年外頭說他和他們傢什麼寶二爺有原故,後來不知怎麼又害起相思病來了,請大夫吃藥的就是他。”那一個人聽了,說:“咱們今日躲一天,叫咱們大哥借錢置辦些買賣行頭,明兒亮鐘時候陸續出關。你們在關外二十里坡等我。”衆賊議定,分贓俵散。不題。
\end{parag}


\begin{parag}
    且說賈政等送殯,到了寺內安厝畢,親友散去。賈政在外廂房伴靈,邢王二夫人等在內,一宿無非哭泣。到了第二日,重新上祭。正擺飯時,只見賈芸進來,在老太太靈前磕了個頭,忙忙的跑到賈政跟前跪下請了安,喘吁吁的將昨夜被盜,將老太太上房的東西都偷去,包勇趕賊打死了一個,已經呈報文武衙門的話說了一遍。賈政聽了發怔。邢王二夫人等在裏頭也聽見了,都唬得魂不附體,並無一言,只有啼哭。賈政過了一會子問失單怎樣開的,賈芸回道:“家裏的人都不知道,還沒有開單。”賈政道:“還好,咱們動過家的,若開出好的來反擔罪名。快叫璉兒。”
\end{parag}


\begin{parag}
    賈璉領了寶玉等去別處上祭未回,賈政叫人趕了回來。賈璉聽了,急得直跳,一見芸兒,也不顧賈政在那裏,便把賈芸狠狠的罵了一頓說:“不配抬舉的東西,我將這樣重任託你,押着人上夜巡更,你是死人麼!虧你還有臉來告訴!”說着,往賈芸臉上啐了幾口。賈芸垂手站着,不敢回一言。賈政道:“你罵他也無益了。”賈璉然後跪下說:“這便怎麼樣?”賈政道:“也沒法兒,只有報官緝賊。但只有一件:老太太遺下的東西咱們都沒動,你說要銀子,我想老太太死得幾天,誰忍得動他那一項銀子。原打諒完了事算了帳還人家,再有的在這裏和南邊置墳產的,再有東西也沒見數兒。如今說文武衙門要失單,若將幾件好的東西開上恐有礙,若說金銀若干,衣飾若干,又沒有實在數目,謊開使不得。倒可笑你如今竟換了一個人了,爲什麼這樣料理不開!你跪在這裏是怎麼樣呢!”賈璉也不敢答言,只得站起來就走。賈政又叫道:“你那裏去?”賈璉又跪下道:“趕回去料理清楚再來回。”賈政哼的一聲,賈璉把頭低下。賈政道:“你進去回了你母親,叫了老太太的一兩個丫頭去,叫他們細細的想了開單子。”賈璉心裏明知老太太的東西都是鴛鴦經管,他死了問誰?就問珍珠,他們那裏記得清楚。只不敢駁回,連連的答應了,起來走到裏頭。邢王夫人又埋怨了一頓,叫賈璉快回去,問他們這些看家的說“明兒怎麼見我們!”賈璉也只得答應了出來,一面命人套車預備琥珀等進城,自己騎上騾子,跟了幾個小廝,如飛的回去。賈芸也不敢再回賈政,斜簽着身子慢慢的溜出來,騎上了馬來趕賈璉。一路無話。
\end{parag}


\begin{parag}
    到回了家中,林之孝請了安,一直跟了進來。賈璉到了老太太上屋,見了鳳姐惜春在那裏,心裏又恨又說不出來,便問林之孝道:“衙門裏瞧了沒有?”林之孝自知有罪,便跪下回道:“文武衙門都瞧了,來蹤去跡也看了,屍也驗了。”賈璉喫驚道:“又驗什麼屍?”林之孝又將包勇打死的夥賊似周瑞的乾兒子的話回了賈璉。賈璉道:“叫芸兒。”賈芸進來也跪着聽話。賈璉道:“你見老爺時怎麼沒有回周瑞的乾兒子做了賊被包勇打死的話?”賈芸說道:“上夜的人說象他的,恐怕不真,所以沒有回。”賈璉道:“好糊塗東西!你若告訴了我,就帶了周瑞來一認可不就知道了。”林之孝回道:“如今衙門裏把屍首放在市口兒招認去了。”賈璉道:“這又是個糊塗東西,誰家的人做了賊,被人打死,要償命麼!”林之孝回道:“這不用人家認,奴才就認得是他。”賈璉聽了想道:“是啊,我記得珍大爺那一年要打的可不是周瑞家的麼。”林之孝回說:“他和鮑二打架來着,還見過的呢。”賈璉聽了更生氣,便要打上夜的人。林之孝哀告道:“請二爺息怒,那些上夜的人,派了他們,還敢偷懶?只是爺府上的規矩,三門裏一個男人不敢進去的,就是奴才們,裏頭不叫,也不敢進去。奴才在外同芸哥兒刻刻查點,見三門關的嚴嚴的,外頭的門一重沒有開。那賊是從後夾道子來的。”賈璉道:“裏頭上夜的女人呢。”林之孝將分更上夜奉奶奶的命捆着等爺審問的話回了。賈璉又問“包勇呢?”林之孝說:“又往園裏去了。”賈璉便說:“去叫來。”小廝們便將包勇帶來。說:“還虧你在這裏,若沒有你,只怕所有房屋裏的東西都搶了去了呢。”包勇也不言語。惜春恐他說出那話,心下着急。鳳姐也不敢言語。只見外頭說:“琥珀姐姐等回來了。”大家見了,不免又哭一場。
\end{parag}


\begin{parag}
    賈璉叫人檢點偷剩下的東西,只有些衣服尺頭錢箱未動,餘者都沒有了。賈璉心裏更加着急,想着“外頭的棚槓銀,廚房的錢都沒有付給,明兒拿什麼還呢!”便呆想了一會。只見琥珀等進去,哭了一會,見箱櫃開着,所有的東西怎能記憶,便胡亂想猜,虛擬了一張失單,命人即送到文武衙門。賈璉復又派人上夜。鳳姐惜春各自回房。賈璉不敢在家安歇,也不及埋怨鳳姐,竟自騎馬趕出城外。這裏鳳姐又恐惜春短見,又打發了豐兒過去安慰。
\end{parag}


\begin{parag}
    天已二更。不言這裏賊去關門,衆人更加小心,誰敢睡覺。且說夥賊一心想着妙玉,知是孤庵女衆,不難欺負。到了三更夜靜,便拿了短兵器,帶了些悶香,跳上高牆。遠遠瞧見櫳翠庵內燈光猶亮,便潛身溜下,藏在房頭僻處。
\end{parag}


\begin{parag}
    等到四更,見裏頭只有一盞海燈,妙玉一人在蒲團上打坐。歇了一會,便噯聲嘆氣的說道:“我自元墓到京,原想傳個名的,爲這裏請來,不能又棲他處。昨兒好心去瞧四姑娘,反受了這蠢人的氣,夜裏又受了大驚。今日回來,那蒲團再坐不穩,只覺肉跳心驚。”因素常一個打坐的,今日又不肯叫人相伴。豈知到了五更,寒顫起來。正要叫人,只聽見窗外一響,想起昨晚的事,更加害怕,不免叫人。豈知那些婆子都不答應。自己坐着,覺得一股香氣透入滷門,便手足麻木,不能動彈,口裏也說不出話來,心中更自着急。只見一個人拿着明晃晃的刀進來。此時妙玉心中卻是明白,只不能動,想是要殺自己,索性橫了心,倒也不怕。那知那個人把刀插在背後,騰出手來將妙玉輕輕的抱起,輕薄了一會子,便拖起背在身上。此時妙玉心中只是如醉如癡。可憐一個極潔極淨的女兒,被這強盜的悶香薰住,由着他掇弄了去了。
\end{parag}


\begin{parag}
    卻說這賊背了妙玉來到園後牆邊,搭了軟梯,爬上牆跳出去了。外邊早有夥計弄了車輛在園外等着,那人將妙玉放倒在車上,反打起官銜燈籠,叫開柵欄,急急行到城門,正是開門之時。門官只知是有公幹出城的,也不及查詰。趕出城去,那夥賊加鞭趕到二十里坡和衆強徒打了照面,各自分頭奔南海而去。不知妙玉被劫或是甘受污辱,還是不屈而死,不知下落,也難妄擬。
\end{parag}


\begin{parag}
    只言櫳翠庵一個跟妙玉的女尼,他本住在靜室後面,睡到五更,聽見前面有人聲響,只道妙玉打坐不安。後來聽見有男人腳步,門窗響動,欲要起來瞧看,只是身子發軟懶怠開口,又不聽見妙玉言語,只睜着兩眼聽着。到了天亮,終覺得心裏清楚,披衣起來,叫了道婆預備妙玉茶水,他便往前面來看妙玉。豈知妙玉的蹤跡全無,門窗大開。心裏詫異,昨晚響動甚是疑心,說:“這樣早,他到那裏去了?”走出院門一看,有一個軟梯靠牆立着,地下還有一把刀鞘,一條搭膊,便道:“不好了,昨晚是賊燒了悶香了!”急叫人起來查看,庵門仍是緊閉。那些婆子女侍們都說:“昨夜煤氣燻着了,今早都起不起來,這麼早叫我們做什麼。”那女尼道:“師父不知那裏去了。”衆人道:“在觀音堂打坐呢。”女尼道:“你們還做夢呢,你來瞧瞧。”衆人不知,也都着忙,開了庵門,滿園裏都找到了,“想來或是到四姑娘那裏去了。”
\end{parag}


\begin{parag}
    衆人來叩腰門,又被包勇罵了一頓。衆人說道:“我們妙師父昨晚不知去向,所以來找。求你老人家叫開腰門,問一問來了沒來就是了。”包勇道:“你們師父引了賊來偷我們,已經偷到手了,他跟了賊受用去了。”衆人道:“阿彌陀佛,說這些話的防着下割舌地獄!”包勇生氣道:“胡說,你們再鬧我就要打了。”衆人陪笑央告道:“求爺叫開門我們瞧瞧,若沒有,再不敢驚動你太爺了。”包勇道:“你不信你去找,若沒有,回來問你們。”包勇說着叫開腰門,衆人找到惜春那裏。
\end{parag}


\begin{parag}
    惜春正是愁悶,惦着“妙玉清早去後不知聽見我們姓包的話了沒有,只怕又得罪了他,以後總不肯來。我的知己是沒有了。況我現在實難見人。父母早死,嫂子嫌我,頭裏有老太太,到底還疼我些,如今也死了,留下我孤苦伶仃,如何了局!”想到:“迎春姐姐磨折死了,史姐姐守着病人,三姐姐遠去,這都是命裏所招,不能自由。獨有妙玉如閒雲野鶴,無拘無束。我能學他,就造化不小了。但我是世家之女,怎能遂意。這回看家已大擔不是,還有何顏在這裏。又恐太太們不知我的心事,將來的後事如何呢?”想到其間,便要把自己的青絲絞去,要想出家。彩屏等聽見,急忙來勸,豈知已將一半頭髮絞去。彩屏愈加着忙,說道:“一事不了又出一事,這可怎麼好呢!”
\end{parag}


\begin{parag}
    正在吵鬧,只見妙玉的道婆來找妙玉。彩屏問起來由,先唬了一跳,說是昨日一早去了沒來。裏面惜春聽見,急忙問道:“那裏去了?”道婆們將昨夜聽見的響動,被煤氣燻着,今早不見有妙玉,庵內軟梯刀鞘的話說了一遍。惜春驚疑不定,想起昨日包勇的話來,必是那些強盜看見了他,昨晚搶去了也未可知。但是他素來孤潔的很,豈肯惜命?”怎麼你們都沒聽見麼?”衆人道:“怎麼不聽見!只是我們這些人都是睜着眼連一句話也說不出,必是那賊子燒了悶香。妙姑一人想也被賊悶住,不能言語,況且賊人必多,拿刀弄杖威逼着,他還敢聲喊麼?”正說着,包勇又在腰門那裏嚷,說:“裏頭快把這些混賬的婆子趕了出來罷,快關腰門!”彩屏聽見恐擔不是,只得叫婆子出去,叫人關了腰門。惜春於是更加苦楚,無奈彩屏等再三以禮相勸,仍舊將一半青絲籠起。大家商議不必聲張,就是妙玉被搶也當作不知,且等老爺太太回來再說。惜春心裏的死定下一個出家的念頭,暫且不提。
\end{parag}


\begin{parag}
    且說賈璉回到鐵檻寺,將到家中查點了上夜的人,開了失單報去的話回了。賈政道:“怎樣開的?”賈璉便將琥珀所記得的數目單子呈出,並說:“這上頭元妃賜的東西已經註明。還有那人家不大有的東西不便開上,等侄兒脫了孝出去託人細細的緝訪,少不得弄出來的。”賈政聽了合意,就點頭不言。賈璉進內見了邢王二夫人,商量着“勸老爺早些回家纔好呢,不然都是亂麻似的。”邢夫人道:“可不是,我們在這裏也是驚心吊膽。”賈璉道:“這是我們不敢說的,還是太太的主意二老爺是依的。”邢夫人便與王夫人商議妥了。
\end{parag}


\begin{parag}
    過了一夜,賈政也不放心,打發寶玉進來說:“請太太們今日回家,過兩三日再來。家人們已經派定了,裏頭請太太們派人罷。”邢夫人派了鸚哥等一干人伴靈,將周瑞家的等人派了總管,其餘上下人等都回去。一時忙亂套車備馬。賈政等在賈母靈前辭別,衆人又哭了一場。
\end{parag}


\begin{parag}
    都起來正要走時,只見趙姨娘還爬在地下不起。周姨娘打諒他還哭,便去拉他。豈知趙姨娘滿嘴白沫,眼睛直豎,把舌頭吐出,反把家人唬了一大跳。賈環過來亂嚷。趙姨娘醒來說道:“我是不回去的,跟着老太太回南去。”衆人道:“老太太那用你來!”趙姨娘道:“我跟了一輩子老太太,大老爺還不依,弄神弄鬼的來算計我。——我想仗着馬道婆要出出我的氣,銀子白花了好些,也沒有弄死了一個。如今我回去了,又不知誰來算計我。”衆人聽見,早知是鴛鴦附在他身上。邢王二夫人都不言語瞅着。只有彩雲等代他央告道:“鴛鴦姐姐,你死是自己願意的,與趙姨娘什麼相干,放了他罷。”見邢夫人在這裏,也不敢說別的。趙姨娘道:“我不是鴛鴦,他早到仙界去了。我是閻王差人拿我去的,要問我爲什麼和馬婆子用魘魔法的案件。”說着便叫“好璉二奶奶,你在這裏老爺面前少頂一句兒罷,我有一千日的不好還有一天的好呢。好二奶奶,親二奶奶,並不是我要害你,我一時糊塗,聽了那個老娼婦的話。”
\end{parag}


\begin{parag}
    正鬧着,賈政打發人進來叫環兒。婆子們去回說:“趙姨娘中了邪了,三爺看着呢。”賈政道:“沒有的事,我們先走了。”於是爺們等先回。這裏趙姨娘還是混說,一時救不過來。邢夫人恐他又說出什麼來,便說:“多派幾個人在這裏瞧着他,咱們先走,到了城裏打發大夫出來瞧罷。”王夫人本嫌他,也打撒手兒。寶釵本是仁厚的人,雖想着他害寶玉的事,心裏究竟過不去,背地裏託了周姨娘在這裏照應。周姨娘也是個好人,便應承了。李紈說道:“我也在這裏罷。”王夫人道:“可以不必。”於是大家都要起身。賈環急忙道:“我也在這裏嗎?”王夫人啐道:“糊塗東西!你姨媽的死活都不知,你還要走嗎!”賈環就不敢言語了。寶玉道:“好兄弟,你是走不得的。我進了城打發人來瞧你。”說畢,都上車回家。寺裏只有趙姨娘,賈環,鸚鵡,等人。
\end{parag}


\begin{parag}
    賈政邢夫人等先後到家,到了上房哭了一場。林之孝帶了家下衆人請了安,跪着。賈政喝道:“去罷!明日問你!”鳳姐那日發暈了幾次,竟不能出接,只有惜春見了,覺得滿面慚愧。邢夫人也不理他,王夫人仍是照常,李紈,寶釵拉着手說了幾句話。獨有尤氏說道:“姑娘,你操心了,倒照應了好幾天!”惜春一言不答,只漲紫了臉。寶釵將尤氏一拉,使了個眼色,尤氏等各自歸房去了。賈政略略地看了看,嘆了口氣,並不言語,到書房席地坐下,叫了賈璉,賈蓉,賈芸吩咐了幾句話。寶玉要在書房來陪賈政,賈政道:“不必。”蘭兒仍跟着他母親,一宿無話。
\end{parag}


\begin{parag}
    次日,林之孝一早進書房跪着,賈政將前後被盜的事問了一遍,並將周瑞供了出來,又說:“衙門拿住了鮑二,身邊搜出了失單上的東西,現在夾訊,要在他身上要這一夥賊呢。”賈政聽了,大怒道:“家奴負恩,引賊偷竊家主,真是反了!”立刻叫人到城外將周瑞捆了,送到衙門審問。林之孝只管跪着,不敢起來。賈政道:“你還跪着幹什麼!”林之孝到:“奴才該死,求老爺開恩。”正說着,賴大等一干辦事家人上來請安,呈上喪事帳薄。賈政道:“交給璉二爺算明瞭來回。”吆喝着林之孝起來出去了。
\end{parag}


\begin{parag}
    賈璉一腿跪着,在賈政身邊說了一句話。賈政把眼一瞪道:“胡說!老太太的事,銀兩被賊偷去,難道就該罰奴才拿出來麼?”賈璉紅了臉,不敢言語,站起來也不敢動。賈政道:“你媳婦怎麼樣了?”賈璉又跪下說:“看來是不中用了。”賈政嘆了口氣道:“我不料家運衰敗,一至如此!況且環哥他媽尚在廟中病着,也不知是什麼症候。你們知道不知道?”賈璉也不敢言語。賈政道:“傳出話去,讓人帶了大夫瞧瞧去。”賈璉急忙答應着出來,叫人帶了大夫到鐵檻寺去瞧趙姨娘。未知死活,下回分解。
\end{parag}