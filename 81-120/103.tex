\chap{一百零三}{施毒计金桂自焚身 昧真禅雨村空遇旧}



\begin{parag}
    话说贾琏到了王夫人那边,一一的说了。次日到了部里打点停妥,回来又到王夫人那边,将打点吏部之事告知。王夫人便道:“打听准了么?果然这样,老爷也愿意,合家也放心。那外任是何尝做得的!若不是那样的参回来,只怕叫那些混账东西把老爷的性命都坑了呢!”贾琏道:“太太那里知道?”王夫人道:“自从你二叔放了外任,并没有一个钱拿回来,把家里的倒掏摸了好些去了。你瞧那些跟老爷去的人,他男人在外头不多几时,那些小老婆子们便金头银面的妆扮起来了,可不是在外头瞒着老爷弄钱?你叔叔便由着他们闹去,若弄出事来,不但自己的官做不成,只怕连祖上的官也要抹掉了呢。”贾琏道:“婶子说得很是。方才我听见参了,吓的了不得,直等打听明白才放心。也愿意老爷做个京官,安安逸逸的做几年,才保得住一辈子的声名。就是老太太知道了,倒也是放心的,只要太太说得宽缓些。”王夫人道:“我知道。你到底再去打听打听。”
\end{parag}


\begin{parag}
    贾琏答应了,才要出来,只见薛姨妈家的老婆子慌慌张张的走来,到王夫人里间屋内,也没说请安,便道:“我们太太叫我来告诉这里的姨太太,说我们家了不得了,又闹出事来了。”王夫人听了,便问:“闹出什么事来?”那婆子又说:“了不得,了不得!”王夫人哼道:“糊涂东西!有要紧事你到底说啊!”婆子便说:“我们家二爷不在家,一个男人也没有。这件事情出来怎么办!要求太太打发几位爷们去料理料理。”王夫人听着不懂,便急着道:“究竟要爷们去干什么事?”婆子道:“我们大奶奶死了。”王夫人听了,便啐道:“这种女人死,死了罢咧,也值得大惊小怪的!”婆子道:“不是好好儿死的,是混闹死的。快求太太打发人去办办。”说着就要走。王夫人又生气,又好笑,说:“这婆子好混账。琏哥儿,倒不如你过去瞧瞧,别理那糊涂东西。”那婆子没听见打发人去,只听见说别理他,他便赌气跑回去了。这里薛姨妈正在着急,再等不来,好容易见那婆子来了,便问:“姨太太打发谁来?”婆子叹说道:“人最不要有急难事,什么好亲好眷,看来也不中用。姨太太不但不肯照应我们,倒骂我糊涂。”薛姨妈听了,又气又急道:“姨太太不管,你姑奶奶怎么说了?”婆子道:“姨太太既不管,我们家的姑奶奶自然更不管了。没有去告诉。”薛姨妈啐道:“姨太太是外人,姑娘是我养的,怎么不管!”婆子一时省悟道:“是啊,这么着我还去。”
\end{parag}


\begin{parag}
    正说着,只见贾琏来了,给薛姨妈请了安,道了恼,回说:“我婶子知道弟妇死了,问老婆子,再说不明,着急得很,打发我来问个明白,还叫我在这里料理。该怎么样,姨太太只管说了办去。”薛姨妈本来气得干哭,听见贾琏的话,便笑着说:“倒要二爷费心。我说姨太太是待我们最好的,都是这老货说不清,几乎误了事。请二爷坐下,等我慢慢的告诉你。”便说:“不为别的事,为的是媳妇不是好死的。”贾琏道:“想是为兄弟犯事怨命死的?”薛姨妈道:“若这样倒好了。前几个月头里,他天天蓬头赤脚的疯闹。后来听见你兄弟问了死罪,他虽哭了一场,以后倒擦脂抹粉的起来。我若说他,又要吵个了不得,我总不理他。有一天不知怎么样来要香菱去作伴,我说:‘你放着宝蟾,还要香菱做什么,况且香菱是你不爱的,何苦招气生。’他必不依。我没法儿,便叫香菱到他屋里去。可怜这香菱不敢违我的话,带着病就去了。谁知道他待香菱很好,我倒喜欢。你大妹妹知道了,说:‘只怕不是好心罢。’我也不理会。头几天香菱病着,他倒亲手去做汤给他吃,那知香菱没福,刚端到跟前,他自己烫了手,连碗都砸了。我只说必要迁怒在香菱身上,他倒没生气,自己还拿笤帚扫了,拿水泼净了地,仍旧两个人很好。昨儿晚上,又叫宝蟾去做了两碗汤来,自己说同香菱一块儿喝。隔了一回,听见他屋里两只脚蹬响,宝蟾急的乱嚷,以后香菱也嚷着扶着墙出来叫人。我忙着看去,只见媳妇鼻子眼睛里都流出血来,在地下乱滚,两手在心口乱抓,两脚乱蹬,把我就吓死了,问他也说不出来,只管直嚷,闹了一回就死了。我瞧那光景是服了毒的。宝蟾便哭着来揪香菱,说他把药药死了奶奶了。我看香菱也不是这么样的人,再者他病的起还起不来,怎么能药人呢。无奈宝蟾一口咬定。我的二爷,这叫我怎么办!只得硬着心肠叫老婆子们把香菱捆了,交给宝蟾,便把房门反扣了。我同你二妹妹守了一夜,等府里的门开了才告诉去的。二爷你是明白人,这件事怎么好?”贾琏道:“夏家知道了没有?”薛姨妈道:“也得撕掳明白了才好报啊。”贾琏道:“据我看起来,必要经官才了得下来。我们自然疑在宝蟾身上,别人便说宝蟾为什么药死他奶奶,也是没答对的。若说在香菱身上,竟还装得上。”正说着,只见荣府女人们进来说:“我们二奶奶来了。”贾琏虽是大伯子,因从小儿见的,也不回避。宝钗进来见了母亲,又见了贾琏,便往里间屋里同宝琴坐下。薛姨妈也将前事告诉一遍。宝钗便说:“若把香菱捆了,可不是我们也说是香菱药死的了么?妈妈说这汤是宝蟾做的,就该捆起宝蟾来问他呀。一面便该打发人报夏家去,一面报官的是。”薛姨妈听见有理,便问贾琏。贾琏道:“二妹子说得很是。报官还得我去,托了刑部里的人,相验问口供的时候有照应得。只是要捆宝蟾放香菱倒怕难些。”薛姨妈道:“并不是我要捆香菱,我恐怕香菱病中受怨着急,一时寻死,又添了一条人命,才捆了交给宝蟾,也是一个主意。”贾琏道:“虽是这么说,我们倒帮了宝蟾了。若要放都放,要捆都捆,他们三个人是一处的。只要叫人安慰香菱就是了。”薛姨妈便叫人开门进去,宝钗就派了带来几个女人帮着捆宝蟾。只见香菱已哭得死去活来,宝蟾反得意洋洋。以后见人要捆他,便乱嚷起来。那禁得荣府的人吆喝着,也就捆了。竟开着门,好叫人看着。这里报夏家的人已经去了。
\end{parag}


\begin{parag}
    那夏家先前不住在京里,因近年消索,又记挂女儿,新近搬进京来。父亲已没,只有母亲,又过继了一个混账儿子,把家业都花完了,不时的常到薛家。那金桂原是个水性人儿,那里守得住空房,况兼天天心里想念薛蝌,便有些饥不择食的光景。无奈他这一干兄弟又是个蠢货,虽也有些知觉,只是尚未入港。所以金桂时常回去,也帮贴他些银钱。这些时正盼金桂回家,只见薛家的人来,心里就想又拿什么东西来了。不料说这里姑娘服毒死了,他便气得乱嚷乱叫。金桂的母亲听见了,更哭喊起来,说:“好端端的女孩儿在他家,为什么服了毒呢!”哭着喊着的,带了儿子,也等不得雇车,便要走来。那夏家本是买卖人家,如今没了钱,那顾什么脸面。儿子头里就走,他跟了一个破老婆子出了门,在街上啼啼哭哭的雇了一辆破车,便跑到薛家。
\end{parag}


\begin{parag}
    进门也不打话,便儿一声肉一声的要讨人命。那时贾琏到刑部托人,家里只有薛姨妈,宝钗,宝琴,何曾见过个阵仗,都吓得不敢则声。便要与他讲理,他们也不听,只说:“我女孩儿在你家得过什么好处,两口朝打暮骂的。闹了几时,还不容他两口子在一处,你们商量着把女婿弄在监里,永不见面。你们娘儿们仗着好亲戚受用也罢了,还嫌他碍眼,叫人药死了他,倒说是服毒!他为什么服毒!”说着,直奔着薛姨妈来。薛姨妈只得后退,说:“亲家太太且请瞧瞧你女儿,问问宝蟾,再说歪话不迟。”那宝钗宝琴因外面有夏家的儿子,难以出来拦护,只在里边着急。恰好王夫人打发周瑞家的照看,一进门来,见一个老婆子指着薛姨妈的脸哭骂。周瑞家的知道必是金桂的母亲,便走上来说:“这位是亲家太太么?大奶奶自己服毒死的,与我们姨太太什么相干,也不犯这么遭塌呀。”那金桂的母亲问:“你是谁?”薛姨妈见有了人,胆子略壮了些,便说:“这就是我亲戚贾府里的。”金桂的母亲便说道:“谁不知道,你们有仗腰子的亲戚,才能够叫姑爷坐在监里。如今我的女孩儿倒白死了不成!”说着,便拉薛姨妈说:“你到底把我女儿怎样弄杀了?给我瞧瞧!”周瑞家的一面劝说:“只管瞧瞧,用不着拉拉扯扯。”便把手一推。夏家的儿子便跑进来不依道:“你仗着府里的势头儿来打我母亲么!”说着,便将椅子打去,却没有打着。里头跟宝钗的人听见外头闹起来,赶着来瞧,恐怕周瑞家的吃亏,齐打伙的上去半劝半喝。那夏家的母子索性撒起泼来,说:“知道你们荣府的势头儿。我们家的姑娘已经死了,如今也都不要命了!”说着,仍奔薛姨妈拼命。地下的人虽多,那里挡得住,自古说的“一人拼命,万夫莫当。”
\end{parag}


\begin{parag}
    正闹到危急之际,贾琏带了七八个家人进来,见是如此,便叫人先把夏家的儿子拉出去,便说:“你们不许闹,有话好好儿的说。快将家里收拾收拾,刑部里头的老爷们就来相验了。”金桂的母亲正在撒泼,只见来了一位老爷,几个在头里吆喝,那些人都垂手侍立。金桂的母亲见这个光景,也不知是贾府何人,又见他儿子已被人揪住,又听见说刑部来验,他心里原想看见女儿尸首先闹了一个稀烂再去喊官去,不承望这里先报了官,也便软了些。薛姨妈已吓糊涂了。还是周瑞家的回说:“他们来了,也没有去瞧他姑娘,便作践起姨太太来了。我们为好劝他,那里跑进一个野男人,在奶奶们里头混撒村混打,这可不是没有王法了!”贾琏道:“这回子不用和他讲理,等一会子打着问他,说:男人有男人的所在,里头都是些姑娘奶奶们,况且有他母亲还瞧不见他们姑娘么,他跑进来不是要打抢来了么!”家人们做好做歹压伏住了。周瑞家的仗着人多,便说:“夏太太,你不懂事,既来了,该问个青红皂白。你们姑娘是自己服毒死了,不然便是宝蟾药死他主子了,怎么不问明白,又不看尸首,就想讹人来了呢,我们就肯叫一个媳妇儿白死了不成!现在把宝蟾捆着,因为你们姑娘必要点病儿,所以叫香菱陪着他,也在一个屋里住,故此两个人都看守在那里,原等你们来眼看看刑部相验,问出道理来才是啊。”
\end{parag}


\begin{parag}
    金桂的母亲此时势孤,也只得跟着周瑞家的到他女孩儿屋里,只见满脸黑血,直挺挺的躺在炕上,便叫哭起来。宝蟾见是他家的人来,便哭喊说:“我们姑娘好意待香菱,叫他在一块儿住,他倒抽空儿药死我们姑娘!”那时薛家上下人等俱在,便齐声吆喝道:“胡说,昨日奶奶喝了汤才药死的,这汤可不是你做的!”宝蟾道:“汤是我做的,端了来我有事走了,不知香菱起来放些什么在里头药死的。”金桂的母亲听未说完,就奔香菱。众人拦住。薛姨妈便道:“这样子是砒霜药的,家里决无此物。不管香菱宝蟾,终有替他买的,回来刑部少不得问出来,才赖不去。如今把媳妇权放平正,好等官来相验。”众婆子上来抬放。宝钗道:“都是男人进来,你们将女人动用的东西检点检点。”只见炕褥底下有一个揉成团的纸包儿。金桂的母亲瞧见便拾起,打开看时,并没有什么,便撩开了。宝蟾看见道:“可不是有了凭据了。这个纸包儿我认得,头几天耗子闹得慌,奶奶家去与舅爷要的,拿回来搁在首饰匣内,必是香菱看见了拿来药死奶奶的。若不信,你们看看首饰匣里有没有了。”
\end{parag}


\begin{parag}
    金桂的母亲便依着宝蟾的所在取出匣子,只有几支银簪子。薛姨妈便说:“怎么好些首饰都没有了?”宝钗叫人打开箱柜,俱是空的,便道:“嫂子这些东西被谁拿去,这可要问宝蟾。”金桂的母亲心里也虚了好些,见薛姨妈查问宝蟾,便说:“姑娘的东西他那里知道。”周瑞家的道:“亲家太太别这么说呢。我知道宝姑娘是天天跟着大奶奶的,怎么说不知!”这宝蟾见问得紧,又不好胡赖,只得说道:“奶奶自己每每带回家去,我管得么。”众人便说:“好个亲家太太!哄着拿姑娘的东西,哄完了叫他寻死来讹我们。好罢了,回来相验便是这么说。”宝钗叫人:“到外头告诉琏二爷说,别放了夏家的人。”
\end{parag}


\begin{parag}
    里面金桂的母亲忙了手脚,便骂宝蟾道:“小蹄子别嚼舌头了!姑娘几时拿东西到我家去。”宝蟾道:“如今东西是小,给姑娘偿命是大。”宝琴道:“有了东西就有偿命的人了。快请琏二哥哥问准了夏家的儿子买砒霜的话,回来好回刑部里的话。”金桂的母亲着了急道:“这宝蟾必是撞见鬼了,混说起来。我们姑娘何尝买过砒霜。若这么说,必是宝蟾药死了的。”宝蟾急的乱嚷说:“别人赖我也罢了,怎么你们也赖起我来呢!你们不是常和姑娘说,叫他别受委屈,闹得他们家破人亡,那时将东西卷包儿一走,再配一个好姑爷。这个话是有的没有?”金桂的母亲还未及答言,周瑞家的便接口说道:“这是你们家的人说的,还赖什么呢。”金桂的母亲恨的咬牙切齿的骂宝蟾说:“我待你不错呀,为什么你倒拿话来葬送我呢!回来见了官,我就说是你药死姑娘的。”宝蟾气得瞪着眼说:“请太太放了香菱罢,不犯着白害别人。我见官自有我的话。”
\end{parag}


\begin{parag}
    宝钗听出这个话头儿来了,便叫人反倒放开了宝蟾,说:“你原是个爽快人,何苦白冤在里头。你有话索性说了,大家明白,岂不完了事了呢。”宝蟾也怕见官受苦,便说:“我们奶奶天天抱怨说:‘我这样人,为什么碰着这个瞎眼的娘,不配给二爷,偏给了这么个混账糊涂行子。要是能够同二爷过一天,死了也是愿意的。’说到那里,便恨香菱。我起初不理会,后来看见与香菱好了,我只道是香菱教他什么了,不承望昨儿的汤不是好意。”金桂的母亲接说道:“益发胡说了,若是要药香菱,为什么倒药了自己呢?”宝钗便问道:“香菱,昨日你喝汤来着没有?”香菱道:“头几天我病得抬不起头来,奶奶叫我喝汤,我不敢说不喝,刚要扎挣起来,那碗汤已经洒了,倒叫奶奶收拾了个难,我心里很过不去。昨儿听见叫我喝汤,我喝不下去,没有法儿正要喝的时候儿呢,偏又头晕起来。只见宝蟾姐姐端了去,我正喜欢,刚合上眼,奶奶自己喝着汤,叫我尝尝,我便勉强也喝了。”宝蟾不待说完,便道:“是了,我老实说罢。昨儿奶奶叫我做两碗汤,说是和香菱同喝。我气不过,心里想着香菱那里配我做汤给他喝呢。我故意的一碗里头多抓了一把盐,记了暗记儿,原想给香菱喝的。刚端进来,奶奶却拦着我到外头叫小子们雇车,说今日回家去。我出去说了,回来见盐多的这碗汤在奶奶跟前呢,我恐怕奶奶喝着咸,又要骂我。正没法的时候,奶奶往后头走动,我眼错不见就把香菱这碗汤换了过来。也是合该如此,奶奶回来就拿了汤去到香菱床边喝着,说:‘你到底尝尝。’那香菱也不觉咸。两个人都喝完了。我正笑香菱没嘴道儿,那里知道这死鬼奶奶要药香菱,必定趁我不在将砒霜撒上了,也不知道我换碗,这可就是天理昭彰,自害其身了。”于是众人往前后一想,真正一丝不错,便将香菱也放了,扶着他仍旧睡在床上。
\end{parag}


\begin{parag}
    不说香菱得放,且说金桂母亲心虚事实,还想辩赖。薛姨妈等你言我语,反要他儿子偿还金桂之命。正然吵嚷,贾琏在外嚷说:“不用多说了,快收拾停当,刑部老爷就到了。”此时惟有夏家母子着忙,想来总要吃亏的,不得已反求薛姨妈道:“千不是万不是,终是我死的女孩儿不长进,这也是自作自受。若是刑部相验,到底府上脸面不好看。求亲家太太息了这件事罢。”宝钗道:“那可使不得,已经报了,怎么能息呢。”周瑞家的等人大家做好做歹的劝说:“若要息事,除非夏亲家太太自己出去拦验,我们不提长短罢了。”贾琏在外也将他儿子吓住,他情愿迎到刑部具结拦验。众人依允。薛姨妈命人买棺成殓。不提。
\end{parag}


\begin{parag}
    且说贾雨村升了京兆府尹兼管税务,一日出都查勘开垦地亩,路过知机县,到了急流津。正要渡过彼岸,因待人夫,暂且停轿。只见村旁有一座小庙,墙壁坍颓,露出几株古松,倒也苍老。雨村下轿,闲步进庙,但见庙内神像金身脱落,殿宇歪斜,旁有断碣,字迹模糊,也看不明白。意欲行至后殿,只见一翠柏下荫着一间茅庐,庐中有一个道士合眼打坐。雨村走近看时,面貌甚熟,想着倒象在那里见来的,一时再想不出来。从人便欲吆喝。雨村止住,徐步向前叫一声:“老道。”那道士双眼微启,微微的笑道:“贵官何事?”雨村便道:“本府出都查勘事件,路过此地,见老道静修自得,想来道行深通,意欲冒昧请教。”那道人说:“来自有地,去自有方。”雨村知是有些来历的,便长揖请问:“老道从何处修来,在此结庐?此庙何名?庙中共有几人?或欲真修,岂无名山,或欲结缘,何不通衢?”那道人道:“葫芦尚可安身,何必名山结舍。庙名久隐,断碣犹存。形影相随,何须修募。岂似那‘玉在椟中求善价,钗于奁内待时飞’之辈耶!”
\end{parag}


\begin{parag}
    雨村原是个颖悟人,初听见“葫芦”两字,后闻“玉钗”一对,忽然想起甄士隐的事来。重复将那道士端详一回,见他容貌依然,便屏退从人,问道:“君家莫非甄老先生么?”那道人从容笑道:“什么真,什么假!要知道真即是假,假即是真。”雨村听说出贾字来,益发无疑,便从新施礼道:“学生自蒙慨赠到都,托庇获隽公交车,受任贵乡,始知老先生超悟尘凡,飘举仙境。学生虽溯洄思切,自念风尘俗吏,未由再觐仙颜。今何幸于此处相遇,求老仙翁指示愚蒙。倘荷不弃,京寓甚近,学生当得供奉,得以朝夕聆教。”那道人也站起来回礼道:“我于蒲团之外,不知天地间尚有何物。适才尊官所言,贫道一概不解。”说毕,依旧坐下。雨村复又心疑:“想去若非士隐,何貌言相似若此?离别来十九载,面色如旧,必是修炼有成,未肯将前身说破。但我既遇恩公,又不可当面错过。看来不能以富贵动之,那妻女之私更不必说了。”想罢又道:“仙师既不肯说破前因,弟子于心何忍!”正要下礼,只见从人进来,禀说天色将晚,快请渡河。雨村正无主意,那道人道:“请尊官速登彼岸,见面有期,迟则风浪顿起。果蒙不弃,贫道他日尚在渡头候教。”说毕,仍合眼打坐。雨村无奈,只得辞了道人出庙。正要过渡,只见一人飞奔而来。未知何事,下回分解。
\end{parag}