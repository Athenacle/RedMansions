\chap{一百零二}{寧國府骨肉病災襟 大觀園符水驅妖孽}



\begin{parag}
    話說王夫人打發人來喚寶釵,寶釵連忙過來,請了安。王夫人道:“你三妹妹如今要出嫁了,只得你們作嫂子的大家開導開導他,也是你們姊妹之情。況且他也是個明白孩子,我看你們兩個也很合的來。只是我聽見說寶玉聽見他三妹妹出門子,哭的了不的,你也該勸勸他。如今我的身子是十病九痛的,你二嫂子也是三日好兩日不好。你還心地明白些,諸事也別說只管吞着不肯得罪人,將來這一番家事,都是你的擔子。”寶釵答應着。王夫人又說道:“還有一件事,你二嫂子昨兒帶了柳家媳婦的丫頭來,說補在你們屋裏。”寶釵道:“今日平兒才帶過來,說是太太和二奶奶的主意。”王夫人道:“是呦,你二嫂子和我說,我想也沒要緊,不便駁他的回。只是一件,我見那孩子眉眼兒上頭也不是個很安頓的。起先爲寶玉房裏的丫頭狐狸似的,我攆了幾個,那時候你也知道,不然你怎麼搬回家去了呢。如今有你,自然不比先前了。我告訴你,不過留點神兒就是了。你們屋裏就是襲人那孩子還可以使得。”寶釵答應了,又說了幾句話,便過來了。飯後到了探春那邊,自有一番殷勤勸慰之言,不必細說。
\end{parag}


\begin{parag}
    次日,探春將要起身,又來辭寶玉。寶玉自然難割難分。探春便將綱常大體的話,說的寶玉始而低頭不語,後來轉悲作喜,似有醒悟之意。於是探春放心,辭別衆人,竟上轎登程,水舟車陸而去。
\end{parag}


\begin{parag}
    先前衆姊妹們都住在大觀園中,後來賈妃薨後,也不修葺。到了寶玉娶親,林黛玉一死,史湘雲回去,寶琴在家住着,園中人少,況兼天氣寒冷,李紈姊妹,探春,惜春等俱挪回舊所。到了花朝月夕,依舊相約頑耍。如今探春一去,寶玉病後不出屋門,益發沒有高興的人了。所以園中寂寞,只有幾家看園的人住着,那日尤氏過來送探春起身,因天晚省得套車,便從前年在園裏開通寧府的那個便門裏走過去了。覺得淒涼滿目,臺榭依然,女牆一帶都種作園地一般,心中悵然如有所失,因到家中,便有些身上發熱,扎掙一兩天,竟躺倒了。日間的發燒猶可,夜裏身熱異常,便譫語綿綿。賈珍連忙請了大夫看視。說感冒起的,如今纏經,入了足陽明胃經,所以譫語不清,如有所見,有了大穢即可身安。尤氏服了兩劑,並不稍減,更加發起狂來。
\end{parag}


\begin{parag}
    賈珍着急,便叫賈蓉來打聽外頭有好醫生再請幾位來瞧瞧。賈蓉回道:“前兒這位太醫是最興時的了。只怕我母親的病不是藥治得好的。”賈珍道:“胡說,不吃藥難道由他去罷。”賈蓉道:“不是說不治。爲的是前日母親從西府去,回來是穿着園子裏走來家的,一到了家就身上發燒,別是撞客着了罷?外頭有個毛半仙,是南方人,卦起的很靈,不如請他來占卦占卦。看有信兒呢,就依着他,要是不中用,再請別的好大夫來。”賈珍聽了,即刻叫人請來。坐在書房內喝了茶,便說:“府上叫我,不知佔什麼事?”賈蓉道:“家母有病,請教一卦。”毛半仙道:“既如此,取淨水洗手,設下香案。讓我起出一課來看就是了。”一時下人安排定了。他便懷裏掏出卦筒來,走到上頭恭恭敬敬的作了一個揖,手內搖着卦筒,口裏念道:“伏以太極兩儀,絪縕交感。圖書出而變化不窮,神聖作而誠求必應。茲有信官賈某,爲因母病,虔請伏羲,文王,周公,孔子四大聖人,鑑臨在上,誠感則靈,有兇報兇,有吉報吉。先請內象三爻。”說着,將筒內的錢倒在盤內,說“有靈的頭一爻就是交。”拿起來又搖了一搖,倒出來說是單。第三爻又是交。檢起錢來,嘴裏說是:“內爻已示,更請外象三爻,完成一卦。”起出來是單拆單。那毛半仙收了卦筒和銅錢,便坐下問道:“請坐,請坐。讓我來細細的看看。這個卦乃是‘未濟’之卦。世爻是第三爻,午火兄弟劫財,晦氣是一定該有的。如今尊駕爲母問病,用神是初爻,真是父母爻動出官鬼來。五爻上又有一層官鬼,我看令堂太夫人的病是不輕的。還好,還好,如今子亥之水休囚,寅木動而生火。世爻上動出一個子孫來,倒是克鬼的。況且日月生身,再隔兩日子水官鬼落空,交到戌日就好了。但是父母爻上變鬼,恐怕令尊大人也有些關礙。就是本身世爻比劫過重,到了水旺土衰的日子也不好。”說完了,便撅着鬍子坐着。賈蓉起先聽他搗鬼,心裏忍不住要笑,聽他講的卦理明白,又說生怕父親也不好,便說道:“卦是極高明的,但不知我母親到底是什麼病?”毛半仙道:“據這卦上世爻午火變水相剋,必是寒火凝結。若要斷得清楚,揲蓍也不大明白,除非用大六壬才斷得準。”賈蓉道:“先生都高明的麼?”毛半仙道:“知道些。”賈蓉便要請教,報了一個時辰。毛先生便畫了盤子,將神將排定。”算去是戌上白虎,這課叫做‘魄化課’。大凡白虎乃是兇將,乘旺象氣受制,便不能爲害。如今乘着死神死煞及時令囚死,則爲餓虎,定是傷人。就如魄神受驚消散,故名‘魄化’。這課象說是人身喪鬼,憂患相仍,病多喪死,訟有憂驚。按象有日暮虎臨,必定是傍晚得病的。象內說,凡佔此課,必定舊宅有伏虎作怪,或有形響。如今尊駕爲大人而佔,正合着虎在陽憂男,在陰憂女。此課十分兇險呢。”賈蓉沒有聽完,唬得面上失色道:“先生說得很是。但與那卦又不大相合,到底有妨礙麼?”毛半仙道:“你不用慌,待我慢慢的再看。”低着頭又咕噥了一會子,便說“好了,有救星了!算出巳上有貴神救解,謂之‘魄化魂歸’。先憂後喜,是不妨事的。只要小心些就是了。”
\end{parag}


\begin{parag}
    賈蓉奉上卦金,送了出去,回稟賈珍,說是:“母親的病是在舊宅傍晚得的,爲撞着什麼伏屍白虎。”賈珍道:“你說你母親前日從園裏走回來的,可不是那裏撞着的。你還記得你二嬸孃到園裏去,回來就病了。他雖沒有見什麼,後來那些丫頭老婆們都說是山子上一個毛烘烘的東西,眼睛有燈籠大,還會說話,把他二奶奶趕了回來,唬出一場病來。”賈蓉道:“怎麼不記得。我還聽見寶叔家的茗煙說,晴雯是做了園裏芙蓉花的神了,林姑娘死了半空裏有音樂,必定他也是管什麼花兒了。想這許多妖怪在園裏,還了得!頭裏人多陽氣重,常來常往不打緊。如今冷落的時候,母親打那裏走,還不知踹了什麼花兒呢,不然就是撞着那一個。那卦也還算是準的。”賈珍道:“到底說有妨礙沒有呢?”賈蓉道:“據他說,到了戌日就好了。只願早兩天好,或除兩天才好。”賈珍道:“這又是什麼意思?”賈蓉道:“那先生若是這樣準,生怕老爺也有些不自在。”正說着,裏頭喊說“奶奶要坐起到那邊園裏去,丫頭們都按捺不住。”賈珍等進去安慰定了。只聞尤氏嘴裏亂說:“穿紅的來叫我,穿綠的來趕我。”地下這些人又怕又好笑。賈珍便命人買些紙錢送到園裏燒化,果然那夜出了汗,便安靜些。到了戌日,也就漸漸的好起來。由是一人傳十,十人傳百,都說大觀園中有了妖怪。唬得那些看園的人也不修花補樹,灌溉果蔬。起先晚上不敢行走,以致鳥獸逼人,甚至日裏也是約伴持械而行。過了些時,果然賈珍患病。竟不請醫調治,輕則到園化紙許願,重則詳星拜斗。賈珍方好,賈蓉等相繼而病。如此接連數月,鬧得兩府俱怕。從此風聲鶴唳,草木皆妖。園中出息,一概全蠲,各房月例重新添起,反弄得榮府中更加拮据。那些看園的沒有了想頭,個個要離此處,每每造言生事,便將花妖樹怪編派起來,各要搬出,將園門封固,再無人敢到園中。以致崇樓高閣,瓊館瑤臺,皆爲禽獸所棲。
\end{parag}


\begin{parag}
    卻說晴雯的表兄吳貴正住在園門口,他媳婦自從晴雯死後,聽見說作了花神,每日晚間便不敢出門。這一日吳貴出門買東西,回來晚了。那媳婦子本有些感冒着了,日間喫錯了藥,晚上吳貴到家,已死在炕上。外面的人因那媳婦子不妥當,便都說妖怪爬過牆吸了精去死的。於是老太太着急的了不得,替另派了好些人將寶玉的住房圍住,巡邏打更。這些小丫頭們還說,有的看見紅臉的,有的看見很俊的女人的,吵嚷不休。唬得寶玉天天害怕。虧得寶釵有把持的,聽得丫頭們混說,便唬嚇着要打,所以那些謠言略好些。無奈各房的人都是疑人疑鬼的不安靜,也添了人坐更,於是更加了好些食用。獨有賈赦不大很信,說:“好好園子,那裏有什麼鬼怪!”挑了個風清日暖的日子,帶了好幾個家人,手內持着器械,到園踹看動靜。衆人勸他不依。到了園中,果然陰氣逼人。賈赦還扎掙前走,跟的人都探頭縮腦。內中有個年輕的家人,心內已經害怕,只聽呼的一聲,回過頭來,只見五色燦爛的一件東西跳過去了,唬得噯喲一聲,腿子發軟,便躺倒了。賈赦回身查問,那小子喘噓噓的回道:“親眼看見一個黃臉紅須綠衣青裳一個妖怪走到樹林子後頭山窟窿裏去了。”賈赦聽了,便也有些膽怯,問道:“你們都看見麼?”有幾個推順水船兒的回說:“怎麼沒瞧見,因老爺在頭裏,不敢驚動罷了。奴才們還撐得住。”說得賈赦害怕,也不敢再走,急急的回來,吩咐小子們:“不要提及,只說看遍了,沒有什麼東西。”心裏實也相信,要到真人府裏請法官驅邪。豈知那些家人無事還要生事,今見賈赦怕了,不但不瞞着,反添些穿鑿,說得人人吐舌。
\end{parag}


\begin{parag}
    賈赦沒法,只得請道士到園作法事驅邪逐妖。擇吉日先在省親正殿上鋪排起壇場,上供三清聖像,旁設二十八宿並馬,趙,溫,週四大將,下排三十六天將圖像。香花燈燭設滿一堂,鐘鼓法器排兩邊,插着五方旗號。道紀司派定四十九位道衆的執事,淨了一天的壇。三位法官行香取水畢,然後擂起法鼓,法師們俱戴上七星冠,披上九宮八卦的法衣,踏着登雲履,手執牙笏,便拜表請聖。又唸了一天的消災驅邪接福的《洞元經》,以後便出榜召將。榜上大書“太乙混元上清三境靈寶符錄演教大法師行文敕令本境諸神到壇聽用。”
\end{parag}


\begin{parag}
    那日兩府上下爺們仗着法師擒妖,都到園中觀看,都說:“好大法令!呼神遣將的鬧起來,不管有多少妖怪也唬跑了。”大家都擠到壇前。只見小道士們將旗幡舉起,按定五方站住,伺候法師號令。三位法師,一位手提寶劍拿着法水,一位捧着七星皁旗,一位舉着桃木打妖鞭,立在壇前。只聽法器一停,上頭令牌三下,口中唸唸有詞,那五方旗便團團散佈。法師下壇,叫本家領着到各處樓閣殿亭房廊屋舍山崖水畔灑了法水,將劍指畫了一回,回來連擊牌令,將七星旗祭起,衆道士將旗幡一聚,接下打怪鞭望空打了三下。本家衆人都道拿住妖怪,爭着要看,及到跟前,並不見有什麼形響。只見法師叫衆道士拿取瓶罐,將妖收下,加上封條。法師硃筆書符收禁,令人帶回在本觀塔下鎮住,一面撤壇謝將。
\end{parag}


\begin{parag}
    賈赦恭敬叩謝了法師。賈蓉等小弟兄背地都笑個不住,說:“這樣的大排場,我打量拿着妖怪給我們瞧瞧到底是些什麼東西,那裏知道是這樣收羅,究竟妖怪拿去了沒有?”賈珍聽見罵道:“糊塗東西,妖怪原是聚則成形,散則成氣,如今多少神將在這裏,還敢現形嗎!無非把這妖氣收了,便不作祟,就是法力了。”衆人將信將疑,且等不見響動再說。那些下人只知妖怪被擒,疑心去了,便不大驚小怪,往後果然沒人提起了。賈珍等病癒復原,都道法師神力。獨有一個小子笑說道:“頭裏那些響動我也不知道,就是跟着大老爺進園這一日,明明是個大公野雞飛過去了,拴兒嚇離了眼,說得活象。我們都替他圓了個謊,大老爺就認真起來。倒瞧了個很熱鬧的壇場。”衆人雖然聽見,那裏肯信,究無人住。
\end{parag}


\begin{parag}
    一日,賈赦無事,正想要叫幾個家下人搬住園中,看守房屋,惟恐夜晚藏匿奸人。方欲傳出話去,只見賈璉進來,請了安,回說今日到他大舅家去聽見一個荒信,“說是二叔被節度使參進來,爲的是失察屬員,重徵糧米,請旨革職的事。”賈赦聽了喫驚道:“只怕是謠言罷。前兒你二叔帶書子來說,探春於某日到了任所,擇了某日吉時送了你妹子到了海疆,路上風恬浪靜,閤家不必掛念。還說節度認親,倒設席賀喜,那裏有做了親戚倒提參起來的。且不必言語,快到吏部打聽明白就來回我。”
\end{parag}


\begin{parag}
    賈璉即刻出去,不到半日回來便說:“纔到吏部打聽,果然二叔被參。題本上去,虧得皇上的恩典,沒有交部,便下旨意,說是失察屬員,重徵糧米,苛虐百姓,本應革職,姑念初膺外任,不諳吏治,被屬員矇蔽,着降三級,加恩仍以工部員外上行走,並令即日回京。這信是準的。正在吏部說話的時候,來了一個江西引見知縣,說起我們二叔,是很感激的,但說是個好上司,只是用人不當,那些家人在外招搖撞騙,欺凌屬員,已經把好名聲都弄壞了。節度大人早已知道,也說我們二叔是個好人。不知怎麼樣這回又參了。想是忒鬧得不好,恐將來弄出大禍,所以借了一件失察的事情參的,倒是避重就輕的意思也未可知。”賈赦未聽說完,便叫賈璉:“先去告訴你嬸子知道,且不必告訴老太太就是了。”賈璉去回王夫人。未知有何話說,下回分解。
\end{parag}