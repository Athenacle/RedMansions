\chap{四十九}{琉璃世界白雪紅梅 脂粉香娃割腥啖羶}


\begin{parag}
    \begin{note}庚:此回系大觀園集十二正釵之文。\end{note}
\end{parag}


\begin{parag}
    話說香菱見衆人正說笑,他便迎上去笑道:“你們看這一首。若使得,我便還學;若還不好,我就死了這作詩的心了。”\begin{note}蒙側:說“死了心不學”方是才人“語不驚人死不休”本懷!\end{note}說著,把詩遞與黛玉及衆人看時,只見寫道是:
\end{parag}


\begin{poem}
    \begin{pl}精華欲掩料應難,影自娟娟魄自寒。\end{pl}

    \begin{pl}一片砧敲千里白,半輪雞唱五更殘。\end{pl}

    \begin{pl}綠蓑江上秋聞笛,紅袖樓頭夜倚欄。\end{pl}

    \begin{pl}博得嫦娥應借問,何緣不使永團圓?\end{pl}


\end{poem}


\begin{parag}
    衆人看了笑道:“這首不但好,而且新巧有意趣。可知俗語說‘天下無難事,只怕有心人’,社裏一定請你了。”香菱聽了心下不信\begin{note}蒙側:聽了不信方是才人虛心。香菱可愛。\end{note},料著是他們瞞哄自己的話,還只管問黛玉寶釵等。
\end{parag}


\begin{parag}
    正說之間,只見幾個小丫頭並老婆子忙忙的走來,都笑道:“來了好些姑娘奶奶們,我們都不認得,奶奶姑娘們快認親去。”李紈笑道:“這是那裏的話?你到底說明白了是誰的親戚?”那婆子丫頭都笑道:“奶奶的兩位妹子都來了。還有一位姑娘,說是薛大姑娘的妹妹,還有一位爺,說是薛大爺的兄弟。我這會子請姨太太去呢,奶奶和姑娘們先上去罷。”說著,一逕去了。寶釵笑道:“我們薛蝌和他妹妹來了不成?”李紈也笑道:“我們嬸子又上京來了不成?他們也不能湊在一處,這可是奇事。”大家納悶,來至王夫人上房,只見烏壓壓一地的人。
\end{parag}


\begin{parag}
    原來邢夫人之兄嫂帶了女兒岫煙進京來投邢夫人的,可巧鳳姐之兄王仁也正進京,兩親家一處打幫來了。走至半路泊船時,正遇見李紈之寡嬸帶著兩個女兒── 大名李紋,次名李綺──也上京。大家敘起來又是親戚,因此三家一路同行。後有薛蟠之從弟薛蝌,因當年父親在京時已將胞妹薛寶琴許配都中梅翰林之子爲婚,\begin{note}蒙側:寶琴許配梅門,於敘事內先逗一筆,後方不突。實此等法脈,識者著眼。\end{note}正欲進京發嫁,聞得王仁進京,他也帶了妹子隨後趕來。所以今日會齊了來訪投各人親戚。
\end{parag}


\begin{parag}
    於是大家見禮敘過,賈母王夫人都歡喜非常。賈母因笑道:“怪道昨日晚上燈花爆了又爆,結了又結,\begin{note}蒙側:燈花二語,何等扯淡,何等包括有趣著。吾筆則語喇喇而不休矣。\end{note}原來應到今日。”一面敘些家常,一面收看帶來的禮物,一面命留酒飯。鳳姐兒自不必說,忙上加忙。李紈寶釵自然和嬸母姊妹敘離別之情。黛玉見了,先是歡喜,\begin{note}蒙側:黛玉先喜後悲,不悲非情,不喜又非情作。\end{note}次後想起衆人皆有親眷,獨自己孤單,無個親眷,不免又去垂淚。寶玉深知其情,十分勸慰了一番方罷。
\end{parag}


\begin{parag}
    然後寶玉忙忙來至怡紅院中,向襲人、麝月、晴雯等笑道:“你們還不快看人去!誰知寶姐姐的親哥哥是那個樣子,他這叔伯兄弟形容舉止另是一樣了,倒象是寶姐姐的同胞弟兄似的。更奇在你們成日家只說寶姐姐是絕色的人物,你們如今瞧瞧他這妹子,更有大嫂嫂這兩個妹子,我竟形容不出了。老天,老天,你有多少精華靈秀,生出這些人上之人來!可知我井底之蛙,成日家自說現在的這幾個人是有一無二的,誰知不必遠尋,就是本地風光,一個賽似一個,如今我又長了一層學問了。除了這幾個,難道還有幾個不成?”一面說,一面自笑自嘆。襲人見他又有了魔意,便不肯去瞧。晴雯等早去瞧了一遍回來,嘻嘻笑向襲人道:“你快瞧瞧去!大太太的一個侄女兒,寶姑娘一個妹妹,大奶奶兩個妹妹,倒象一把子四根水蔥兒。”
\end{parag}


\begin{parag}
    一語未了,只見探春也笑著進來找寶玉,因說道:“咱們的詩社可興旺了。”寶玉笑道:“正是呢。這是你一高興起詩社,所以鬼使神差來了這些人。但只一件,不知他們可學過作詩不曾?”探春道:“我才都問了問他們,雖是他們自謙,看其光景,沒有不會的。便是不會也沒難處,你看香菱就知道了。”襲人笑道: “他們說薛大姑娘的妹妹更好,三姑娘看著怎麼樣?”探春道:“果然的話。據我看,連他姐姐並這些人總不及他。”襲人聽了,又是詫異,又笑道:“這也奇了,還從那裏再好的去呢?我倒要瞧瞧去。”探春道:“老太太一見了,喜歡的無可不可,已經逼著太太認了乾女兒了。老太太要養活,纔剛已經定了。”寶玉喜的忙問:“這果然的?”探春道:“我幾時說過謊!”又笑道:“有了這個好孫女兒,就忘了這孫子了。”寶玉笑道:“這倒不妨,原該多疼女兒些纔是正理。明兒十六,咱們可該起社了。”探春道:“林丫頭剛起來了,二姐姐又病了,終是七上八下的。”寶玉道:“二姐姐又不大作詩,沒有他又何妨。”探春道:“越性等幾天,他們新來的混熟了,咱們邀上他們豈不好?這會子大嫂子寶姐姐心裏自然沒有詩興的,況且湘雲沒來,顰兒剛好了,人人不合式。不如等著雲丫頭來了,這幾個新的也熟了,顰兒也大好了,大嫂子和寶姐姐心也閒了,香菱詩也長進了,如此邀一滿社豈不好?咱們兩個如今且往老太太那裏去聽聽,除寶姐姐的妹妹不算外,他一定是在咱們家住定了的。倘或那三個要不在咱們這裏住,咱們央告著老太太留下他們在園子裏住下,咱們豈不多添幾個人,越發有趣了。”寶玉聽了,喜的眉開眼笑,忙說道:“倒是你明白。\begin{note}蒙側:觀寶玉“到底是你”數語,胸中純是一團活潑潑天機。\end{note}我終久是個糊塗心腸,空喜歡一會子,卻想不到這上頭來。”
\end{parag}


\begin{parag}
    說著,兄妹兩個一齊往賈母處來。果然王夫人已認了寶琴作乾女兒,賈母歡喜非常,連園中也不命住,晚上跟著賈母一處安寢。薛蝌自向薛蟠書房中住下。賈母便和邢夫人說:“你侄女兒也不必家去了,園裏住幾天,逛逛再去。”邢夫人兄嫂家中原艱難,這一上京,原仗的是邢夫人與他們治房舍,幫盤纏,聽如此說,豈不願意。邢夫人便將岫煙交與鳳姐兒。鳳姐兒籌算得園中姊妹多,性情不一,\begin{note}蒙側:鳳姐一番籌算,總爲與自己無干。奸雄每每如此,我愛之,我恨之!\end{note}且又不便另設一處,莫若送到迎春一處去,倘日後邢岫煙有些不遂意的事,縱然邢夫人知道了,與自己無干。從此後若邢岫煙家去住的日期不算,若在大觀園住到一個月上,鳳姐兒亦照迎春的分例送一分與岫煙。鳳姐兒冷眼敁敠岫煙心性爲人,\begin{note}蒙側:先敘岫煙,後敘李紈,又敘李紋李綺,亦何精緻可玩。\end{note}竟不象邢夫人及他的父母一樣,卻是溫厚可疼的人。因此鳳姐兒又憐他家貧命苦,比別的姊妹多疼他些,邢夫人倒不大理論了。
\end{parag}


\begin{parag}
    賈母王夫人因素喜李紈賢惠,且年輕守節,令人敬伏,今見他寡嬸來了,便不肯令他外頭去住。那李嬸雖十分不肯,無奈賈母執意不從,只得帶著李紋李綺在稻香村住下來。
\end{parag}


\begin{parag}
    當下安插既定,誰知保齡侯史鼐又遷委了外省大員,不日要帶家眷去上任。\begin{note}蒙側:史鼐未必左遷,但欲湘雲赴社,故作此一折耳,莫被他混過。\end{note}賈母因捨不得湘雲,便留下他了,接到家中,原要命鳳姐兒另設一處與他住。史湘雲執意不肯,只要與寶釵一處住,因此就罷了。
\end{parag}


\begin{parag}
    此時大觀園中比先更熱鬧了多少。\begin{note}蒙側:此時大觀園數行收拾,是大手筆。\end{note}李紈爲首,餘者迎春、探春、惜春、寶釵、黛玉、湘雲、李紋、李綺、寶琴、邢岫煙,再添上鳳姐兒和寶玉,一共十三個。敘起年庚,除李紈年紀最長,他十二個人皆不過十五六七歲,或有這三個同年,或有那五個共歲,或有這兩個同月同日,那兩個同刻同時,所差者大半是時刻月分而已。連他們自己也不能細細分晰,不過是 “弟”“兄”“姊”“妹”四個字隨便亂叫。
\end{parag}


\begin{parag}
    如今香菱正滿心滿意只想作詩,又不敢十分羅唣寶釵,可巧來了個史湘雲。那史湘雲又是極愛說話的,那裏禁得起香菱又請教他談詩,越發高了興,沒晝沒夜高談闊論起來。寶釵因笑道:“我實在聒噪的受不得了。一個女孩兒家,只管拿著詩作正經事講起來,叫有學問的人聽了,反笑話說不守本分的。一個香菱沒鬧清,偏又添了你這麼個話口袋子,滿嘴裏說的是什麼:怎麼是杜工部之沈鬱,韋蘇州之淡雅,又怎麼是溫八叉之綺靡,李義山之隱僻。放著兩個現成的詩家不知道,提那些死人做什麼!”湘雲聽了,忙笑問道:“是那兩個?好姐姐,你告訴我。”寶釵笑道:“呆香菱之心苦,瘋湘雲之話多。”湘雲香菱聽了,都笑起來。
\end{parag}


\begin{parag}
    正說著,只見寶琴來了,披著一領斗篷,金翠輝煌,不知何物。寶釵忙問:“這是那裏的?”寶琴笑道:“因下雪珠兒,老太太找了這一件給我的。”香菱上來瞧道:“怪道這麼好看,原來是孔雀毛織的。”湘雲道:“那裏是孔雀毛,就是野鴨子頭上的毛作的。可見老太太疼你了,這樣疼寶玉,也沒給他穿。”寶釵道: “真俗語說‘各人有緣法’。他也再想不到他這會子來,既來了,又有老太太這麼疼他。”湘雲道:“你除了在老太太跟前,就在園裏來,這兩處只管頑笑喫喝。到了太太屋裏,若太太在屋裏,只管和太太說笑,多坐一回無妨;若太太不在屋裏,你別進去,那屋裏人多心壞,都是要害咱們的。”說的寶釵、寶琴、香菱、鶯兒等都笑了。寶釵笑道:“說你沒心,卻又有心;雖然有心,到底嘴太直了。我們這琴兒就有些象你。你天天說要我作親姐姐,我今兒竟叫你認他作親妹妹罷了。”湘雲又瞅了寶琴半日,笑道:“這一件衣裳也只配他穿,別人穿了,實在不配。”正說著,只見琥珀走來笑道:“老太太說了,叫寶姑娘別管緊了琴姑娘。他還小呢,讓他愛怎麼樣就怎麼樣。要什麼東西只管要去,別多心。”寶釵忙起身答應了,又推寶琴笑道:“你也不知是那裏來的福氣!你倒去罷,仔細我們委曲著你。我就不信我那些兒不如你。”說話之間,寶玉黛玉都進來了,寶釵猶自嘲笑。湘雲因笑道:“寶姐姐,你這話雖是頑話,恰有人真心是這樣想呢。”琥珀笑道:“真心惱的再沒別人,就只是他。”口裏說,手指著寶玉。寶釵湘雲都笑道:“他倒不是這樣人。”琥珀又笑道:“不是他,就是他。”說著又指著黛玉。湘雲便不則聲。\begin{note}庚雙夾:是不知黛玉病中相談送燕窩之事也。脂硯。\end{note}寶釵忙笑道:“更不是了。我的妹妹和他的妹妹一樣。他喜歡的比我還疼呢,那裏還惱?你信口兒混說。他的那嘴有什麼實據。”寶玉素習深知黛玉有些小性兒,且尚不知近日黛玉和寶釵之事,正恐賈母疼寶琴他心中不自在,今見湘雲如此說了,寶釵又如此答,再審度黛玉聲色亦不似往時,果然與寶釵之說相符,心中悶悶不解。因想:“他兩個素日不是這樣的好,今看來竟更比他人好十倍。”一時林黛玉又趕著寶琴叫妹妹,並不提名道姓,直是親姊妹一般。那寶琴年輕心熱,\begin{note}庚雙夾:四字道盡,不犯寶釵。脂硯齋評。\end{note}且本性聰敏,自幼讀書識字,\begin{note}庚雙夾:我批此書竟得一祕訣以告諸公幾:野史中所云“才貌雙全佳人”者,細細通審之,只得一個粗知筆墨之女子耳。此書凡雲“知書識字”者便是上等才女,不信時只看他通部行爲及詩詞、詼諧皆可知。妙在此書從不肯自下評註雲此人系何等人,只借書中人閒評一二語,故不得有未密之縫被看書者指出,真狡猾之筆耳。\end{note}今在賈府住了兩日,大概人物已知。又見諸姊妹都不是那輕薄脂粉,且又和姐姐皆和契,故也不肯怠慢,其中又見林黛玉是個出類拔萃的,便更與黛玉親敬異常。寶玉看著只是暗暗的納罕。
\end{parag}


\begin{parag}
    一時寶釵姊妹往薛姨媽房內去後,湘雲往賈母處來,林黛玉回房歇著。寶玉便找了黛玉來,笑道:“我雖看了《西廂記》,也曾有明白的幾句,說了取笑,你曾惱過。如今想來,竟有一句不解,我念出來你講講我聽。”黛玉聽了,便知有文章,因笑道:“你念出來我聽聽。”寶玉笑道:“那《鬧簡》上有一句說得最好, ‘是幾時孟光接了梁鴻案?’這句最妙。‘孟光接了梁鴻案’這五個字,不過是現成的典,難爲他這‘是幾時’三個虛字問的有趣。是幾時接了?你說說我聽聽。” 黛玉聽了,禁不住也笑起來,因笑道:“這原問的好。他也問的好,你也問的好。”寶玉道:“先時你只疑我,如今你也沒的說,我反落了單。”黛玉笑道:“誰知他竟真是個好人,我素日只當他藏奸。”因把說錯了酒令起,連送燕窩病中所談之事,細細告訴了寶玉。寶玉方知緣故,因笑道:“我說呢,正納悶‘是幾時孟光接了梁鴻案’,原來是從‘小孩兒口沒遮攔’就接了案了。”黛玉因又說起寶琴來,想起自己沒有姊妹,不免又哭了。寶玉忙勸道:“你又自尋煩惱了。你瞧瞧,今年比舊年越發瘦了,你還不保養。每天好好的,你必是自尋煩惱,哭一會子,纔算完了這一天的事。”黛玉拭淚道:“近來我只覺心酸,眼淚卻象比舊年少了些的。心裏只管痠痛,眼淚卻不多。”寶玉道:“這是你哭慣了心裏疑的,豈有眼淚會少的!”
\end{parag}


\begin{parag}
    正說著,只見他屋裏的小丫頭子送了猩猩氈斗篷來,又說:“大奶奶纔打發人來說,下了雪,要商議明日請人作詩呢。”一語未了,只見李紈的丫頭走來請黛玉。寶玉便邀著黛玉同往稻香村來。黛玉換上掐金挖雲紅香羊皮小靴,罩了一件大紅羽紗面白狐狸裏的鶴氅,束一條青金閃綠雙環四合如意絛,頭上罩了雪帽。二人一齊踏雪行來。只見衆姊妹都在那邊,都是一色大紅猩猩氈與羽毛緞斗篷,獨李紈穿一件青哆羅呢對襟褂子,薛寶釵穿一件蓮青斗紋錦上添花洋線番羓絲的鶴氅;邢岫煙仍是家常舊衣,並無避雪之衣。一時史湘雲來了,穿著賈母與他的一件貂鼠腦袋面子大毛黑灰鼠裏子裏外發燒大褂子,頭上帶著一頂挖雲鵝黃片金裏大紅尚燒昭君套,又圍著大貂鼠風領。黛玉先笑道:“你們瞧瞧,孫行者來了。他一般的也拿著雪褂子,故意裝出個小騷達子來。”湘雲笑道:“你們瞧我裏頭打扮的。”一面說,一面脫了褂子。只見他裏頭穿著一件半新的靠色三鑲領袖秋香色盤金五色繡龍窄褃小袖掩衿銀鼠短襖,裏面短短的一件水紅裝緞狐肷褶子,腰裏緊緊束著一條蝴蝶結子長穗五色宮絛,腳下也穿著腳下也穿著麀皮小靴,越顯的蜂腰猿背,鶴勢螂形。\begin{note}庚雙夾:近之拳譜中有“坐馬式”,便似螂之蹲立。昔人愛輕捷便俏,閒取一螂觀其仰頸疊胸之勢。今四字無出處卻寫盡矣。脂硯齋評。\end{note}衆人都笑道:“偏他只愛打扮成個小子的樣兒,原比他打扮女兒更俏麗了些。”湘雲道:“快商議作詩!我聽聽是誰的東家?”李紈道:“我的主意。想來昨兒的正日已過了,再等正日又太遠,可巧又下雪,不如大家湊個社,又替他們接風,又可以作詩。你們意思怎麼樣?”寶玉先道:“這話很是。只是今日晚了,若到明兒,晴了又無趣。”衆人看道,“這雪未必晴,縱晴了,這一夜下的也夠賞了。”李紈道:“我這裏雖好,又不如蘆雪廣 \begin{subnote}按:廣,音眼。就山築成之房屋。韓愈《陪杜侍御遊湘西兩寺》詩:“剖竹走泉源,開廊架崖广。”各本或作“庵” “庭”“廬”,皆非。今從庚本改。\end{subnote}好。我已經打發人籠地炕去了,咱們大家擁爐作詩。老太太想來未必高興,況且咱們小頑意兒,單給鳳丫頭個信兒就是了。你們每人一兩銀子就夠了,送到我這裏來。”指著香菱、寶琴、李紋、李綺、岫煙,“五個不算外,咱們裏頭二丫頭病了不算,四丫頭告了假也不算,你們四分子送了來,我包總五六兩銀子也儘夠了。”寶釵等一齊應諾。因又擬題限韻,李紈笑道:“我心裏自己定了,等到了明日臨期,橫豎知道。”說畢,大家又閒話了一回,方往賈母處來。本日無話。
\end{parag}


\begin{parag}
    到了次日一早,寶玉因心裏記掛著這事,一夜沒好生得睡,天亮了就爬起來。掀開帳子一看,雖門窗尚掩,只見窗上光輝奪目,心內早躊躇起來,埋怨定是晴了,日光已出。一面忙起來揭起窗屜,從玻璃窗內往外一看,原來不是日光,竟是一夜大雪,下將有一尺多厚,天上仍是搓綿扯絮一般。寶玉此時歡喜非常,忙喚人起來,盥漱已畢,只穿一件茄色哆羅呢狐皮襖子,罩一件海龍皮小小鷹膀褂,束了腰,披了玉針蓑,戴上金藤笠,登上沙棠屐,忙忙的往蘆雪廣來。出了院門,四顧一望,並無二色,遠遠的是青松翠竹,自己卻如裝在玻璃盒內一般。於是走至山坡之下,順著山腳剛轉過去,已聞得一股寒香拂鼻。回頭一看,恰是妙玉門前櫳翠庵中有十數株紅梅如胭脂一般,映著雪色,分外顯得精神,好不有趣!寶玉便立住,細細的賞玩一回方走。只見蜂腰板橋上一個人打著傘走來,是李紈打發了請鳳姐兒去的人。
\end{parag}


\begin{parag}
    寶玉來至蘆雪廣,只見丫鬟婆子正在那裏掃雪開徑。原來這蘆雪廣蓋在傍山臨水河灘之上,一帶幾間,茅檐土壁,槿籬竹牖,推窗便可垂釣,四面都是蘆葦掩覆,一條去徑逶迤穿蘆度葦過去,便是藕香榭的竹橋了。衆丫鬟婆子見他披蓑戴笠而來,卻笑道:“我們才說正少一個漁翁,如今都全了。姑娘們吃了飯纔來呢,你也太性急了。”寶玉聽了,只得回來。剛至沁芳亭,見探春正從秋爽齋來,圍著大紅猩猩氈斗篷,戴著觀音兜,扶著小丫頭,後面一個婦人打著青綢油傘。寶玉知他往賈母處去,便立在亭邊,等他來到,二人一同出園前去。寶琴正在裏間房內梳洗更衣。
\end{parag}


\begin{parag}
    一時衆姊妹來齊,寶玉只嚷餓了,連連催飯。好容易等擺上來,頭一樣菜便是牛乳蒸羊羔。賈母便說;“這是我們有年紀的人的藥,沒見天日的東西,可惜你們小孩子們喫不得。今兒另外有新鮮鹿肉,你們等著喫。”衆人答應了。寶玉卻等不得,只拿茶泡了一碗飯,就著野雞瓜齏忙忙的咽完了。賈母道:“我知道你們今兒又有事情,連飯也不顧吃了。”便叫“留著鹿肉與他晚上喫”,鳳姐忙說“還有呢”\begin{note}蒙側:喫殘了的倒\end{note},方纔罷了。史湘雲便悄和寶玉計較道:“有新鮮鹿肉,不如咱們要一塊,自己拿了園裏弄著,又頑又喫。”寶玉聽了,巴不得一聲兒,便真和鳳姐要了一塊,命婆子送入園去。
\end{parag}


\begin{parag}
    一時大家散後,進園齊往蘆雪廣來,聽李紈出題限韻,獨不見湘雲寶玉二人。黛玉道:“他兩個再到不了一處,若到一處,生出多少故事來。這會子一定算計那塊鹿肉去了。”\begin{note}庚雙夾:聯詩極雅之事,偏於雅前寫出小兒啖羶茹血極腌臢的事來,爲“錦心繡口”作配。\end{note}正說著,只見李嬸也走來看熱鬧,因問李紈道:“怎麼一個帶玉的哥兒和那一個掛金麒麟的姐兒,那樣乾淨清秀,又不少喫的,他兩個在那裏商議著要喫生肉呢,說的有來有去的。我只不信肉也生喫得的。” 衆人聽了,都笑道:“了不得,快拿了他兩個來。”黛玉笑道:“這可是雲丫頭鬧的,我的卦再不錯。”
\end{parag}


\begin{parag}
    李紈等忙出來找著他兩個說道:“你們兩個要喫生的,我送你們到老太太那裏喫去。那怕喫一隻生鹿,撐病了不與我相干。這麼大雪,怪冷的,替我作禍呢。” 寶玉笑道:“沒有的事,我們燒著喫呢。”李紈道:“這還罷了。”只見老婆們了拿了鐵爐、鐵叉、鐵絲蒙來,李紈道:“仔細割了手,不許哭!”說著,同探春進去了。
\end{parag}


\begin{parag}
    鳳姐打發了平兒來回復不能來,爲發放年例正忙。湘雲見了平兒,那裏肯放。平兒也是個好頑的,素日跟著鳳姐兒無所不至,見如此有趣,樂得頑笑,因而褪去手上的鐲子,三個圍著火爐兒,便要先燒三塊喫。那邊寶釵黛玉平素看慣了,不以爲異,寶琴等及李嬸深爲罕事。探春與李紈等已議定了題韻。探春笑道:“你聞聞,香氣這裏都聞見了,我也喫去。”說著,也找了他們來。李紈也隨來說:“客已齊了,你們還喫不夠?”湘雲一面喫,一面說道:“我喫這個方愛喫酒,吃了酒纔有詩。若不是這鹿肉,今兒斷不能作詩。”說著,只見寶琴披著鳧靨裘站在那裏笑。湘雲笑道:“傻子,過來嚐嚐。”寶琴笑說:“怪髒的。”寶釵道:“你嚐嚐去,好喫的。你林姐姐弱,吃了不消化,不然他也愛喫。”寶琴聽了,便過去吃了一塊,果然好喫,便也喫起來。一時鳳姐兒打發小丫頭來叫平兒。平兒說:“史姑娘拉著我呢,你先走罷。”小丫頭去了。一時只見鳳姐也披了斗篷走來,笑道:“喫這樣好東西,也不告訴我!”說著也湊著一處喫起來。黛玉笑道:“那裏找這一羣花子去!罷了,罷了,今日蘆雪廣遭劫,生生被雲丫頭作踐了。我爲蘆雪廣一大哭!”\begin{note}庚雙夾:大約此話不獨黛玉,觀書者亦如此。\end{note}湘雲冷笑道:“你知道什麼!‘是真名士自風流’,你們都是假清高,最可厭的。我們這會子腥羶大喫大嚼,回來卻是錦心繡口。”寶釵笑道:“你回來若作的不好了,把那肉掏了出來,就把這雪壓的蘆葦子揌上些,以完此劫。”
\end{parag}


\begin{parag}
    說著,喫畢,洗漱了一回。平兒帶鐲子時卻少了一個,左右前後亂找了一番,蹤跡全無。衆人都詫異。鳳姐兒笑道:“我知道這鐲子的去向。你們只管作詩去,我們也不用找,只管前頭去,不出三日包管就有了。”說著又問:“你們今兒做什麼詩?老太太說了,離年又近了,正月裏還該作些燈謎兒大家頑笑。”衆人聽了,都笑道:“可是倒忘了。如今趕著作幾個好的,預備正月裏頑。”說著,一齊來至地炕屋內,只見杯盤果菜俱已擺齊,牆上已貼出詩題、韻腳、格式來了。寶玉湘雲二人忙看時,只見題目是“即景聯句,五言排律一首,限‘二蕭’韻”。後面尚未列次序。李紈道:“我不大會作詩,我只起三句罷,然後誰先得了誰先聯。”寶釵道:“到底分個次序。”要知端的,且聽下回分解。
\end{parag}


\begin{parag}
    \begin{note}蒙回末總:此文線索在斗篷。寶琴翠羽斗篷,賈母所賜,言其親也;寶玉紅猩猩氈斗篷,爲後雪披一襯也;黛玉白狐皮斗篷,明其弱也;李宮裁斗篷是哆羅呢,昭其質也;寶釵斗篷是蓮青斗紋錦,致其文也;賈母是大斗篷,尊之詞也;鳳姐是披著斗篷,恰似掌家人也;湘雲有斗篷不穿,著其異樣行動也;岫煙無斗篷,敘其窮也。只一斗篷,寫得前後照耀生色。\end{note}
\end{parag}


\begin{parag}
    \begin{note}蒙回末總:一片含梅咀雪圖,偏從雞肉、鹿肉、鵪鶉肉上以渲染之,點成異樣筆墨。較之雪吟、雪賦諸作更覺幽秀。\end{note}
\end{parag}

