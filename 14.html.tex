\chap{一十四}{林如海捐馆扬州城 贾宝玉路谒北静王}

\begin{parag}
    \begin{note}甲戌:凤姐用彩明,因自识字不多,且彩明系未冠之童。\end{note}
\end{parag}


\begin{parag}
    \begin{note}甲戌:写凤姐之珍贵,写凤姐之英气,写凤姐之声势,写凤姐之心机,写凤姐之骄大。\end{note}
\end{parag}


\begin{parag}
    \begin{note}甲戌:昭儿回,并非林文、琏文,是黛玉正文。\end{note}
\end{parag}


\begin{parag}
    \begin{note}甲戌:牛,丑也。清,属水,子也。柳拆卯字。彪拆虎字,寅字寓焉。陈即辰。翼火为蛇;巳字寓焉。马,午也。魁拆鬼,鬼,金羊,未字寓焉。侯、猴同音,申也。晓鸣,鸡也,酉字寓焉。石即豕,亥字寓焉。其祖曰守业,即守夜也,犬字寓焉。此所谓十二支寓焉。\end{note}
\end{parag}


\begin{parag}
    \begin{note}甲戌:路谒北静王,是宝玉正文。\end{note}
\end{parag}


\begin{parag}
    \begin{note}蒙:家书一纸千金重,勾引难防嘱下人。任你无双肝胆烈,多情奋起自眉颦。\end{note}
\end{parag}


\begin{parag}
    话说宁国府中都总管来升闻得里面委请了凤姐,因传齐了同事人等说道:“如今请了西府里琏二奶奶管理内事,倘或他来支取东西,或是说话,我们须要比往日小心些。每日大家早来晚散,宁可辛苦这一个月,过后再歇著,不要把老脸丢了。\begin{note}庚辰侧:此是都总管的话头。\end{note}那是个有名的烈货,脸酸心硬,一时恼了,不认人的。”众人都道:“有理。”又有一个笑道:“论理,我们里面也须得他来整理整理,\begin{note}庚辰侧:伏线在二十板之误差妇人。\end{note}都忒不像了。”正说著,只见来旺媳妇拿了对牌来领取呈文京榜纸札,票上批著数目。众人连忙让坐倒茶,一面命人按数取纸来抱著,同来旺媳妇一路来至仪门口,方交与来旺媳妇自己抱进去了。
\end{parag}


\begin{parag}
    凤姐即命彩明钉造簿册。\begin{note}甲戌眉:宁府如此大家,阿凤如此身份,岂有使贴身丫头与家里男人答话交事之理呢?此作者忽略之处。\end{note}\begin{note}庚辰眉:彩明系未冠小童,阿凤便于出入使令者。老兄并未前后看明,是男是女,乱加批驳。可笑。\end{note}\begin{note}庚辰眉:且明写阿凤不识字之故。壬午春。\end{note}即时传来升媳妇,兼要家口花名册来查看,又限于明日一早传齐家人媳妇进来听差等语。大概点了一点数目单册,\begin{note}甲戌侧:已有成见。\end{note}问了来升媳妇几句话,便坐车回家。一宿无话。
\end{parag}


\begin{parag}
    至次日,卯正二刻便过来了。那宁国府中婆娘媳妇闻得到齐,只见凤姐正与来升媳妇分派,众人不敢擅入,只在窗外听觑。\begin{note}甲戌侧:传神之笔。\end{note}只听凤姐与来升媳妇道:“既托了我,我就说不得要讨你们嫌了。\begin{note}甲戌侧:先站地步。\end{note}我可比不得你们奶奶好性儿,由著你们去,再不要说你们‘这府里原是这样’的话,\begin{note}甲戌侧:此话听熟了。一叹!\end{note}\begin{note}蒙侧:“不要说”,“原是这样的话”,破尽痼弊根底。\end{note}如今可要依著我行,\begin{note}甲戌侧:婉转得妙!\end{note}错我半点儿,管不得谁是有脸的,谁是没脸的,一例现清白处治。”说著,便吩咐彩明念花名册,按名一个一个的唤进来看视。\begin{note}庚辰侧:量才而用之意。\end{note}
\end{parag}


\begin{parag}
    一时看完,便又吩咐道:“这二十个分作两班,一班十个,每日在里头单管人客来往倒茶,别的事不用他们管。这二十个也分作两班,每日单管本家亲戚茶饭,别的事也不用他们管。这四十个人也分作两班,单在灵前上香添油,挂幔守灵,供茶供饭,随起举哀,别的事也不与他们相干。这四个人单在内茶房收管杯碟茶器,若少一件,便叫他四个人描赔。这四个人单管酒饭器皿,少一件,也是他四个人描赔。这八个人单管监收祭礼。这八个人单管各处灯油、蜡烛、纸札,我总支了来,交与你八个,然后按我的定数再往各处去分派。这三十个每日轮流各处上夜,照管门户,监察火烛,打扫地方。这下剩的按著房屋分开,某人守某处,某处所有桌椅古董起,至于痰盒掸帚,一草一苗,或丢或坏,就和守这处的人算帐描赔。来升家的每日揽总查看,或有偷懒的,赌钱吃酒的,打架拌嘴的,立刻来回我。你有徇情,经我查出,三四辈子的老脸就顾不成了。如今都有定规,以后那一行乱了,只和那一行说话。素日跟我的人,随身自有钟表,不论大小事,我是皆有一定的时辰。横竖你们上房里也有时辰钟。卯正二刻我来点卯,巳正吃早饭,凡有领牌回事的,只在午初刻,戍初烧过黄昏纸,我亲到各处查一遍,回来上夜的交明钥匙。第二日仍是卯正二刻过来。说不得咱们大家辛苦这几日,\begin{note}甲戌侧:是协理口气,好听之至!\end{note}\begin{note}庚辰侧:所谓先礼后兵是也。\end{note}事完了,你们家大爷自然赏你们。”\begin{note}庚辰侧:滑贼,好收煞。\end{note}
\end{parag}


\begin{parag}
    说罢,又吩咐按数发与茶叶、油烛、鸡毛掸子、笤帚等物。一面又搬取家伙:桌围、椅搭、坐褥、毡席、痰盒、脚踏之类。一面交发,一面提笔登记,某人管某处,某人领某物,开得十分清楚。众人领了去,也都有了投奔,不似先时只拣便宜的做,剩下的苦差没个招揽。各房中也不能趁乱失迷东西。便是人来客往,也都安静了,不比先前一个正摆茶,又去端饭,正陪举哀,又顾接客。如这些无头绪,荒乱、推托、偷闲、窃取等弊,次日一概都蠲了。
\end{parag}


\begin{parag}
    凤姐儿见自己威重令行,心中十分得意。因见尤氏犯病,贾珍又过于悲哀,不大进饮食,自己每日从那府中煎了各样细粥,精致小菜,命人送来劝食。\begin{note}庚辰眉:写凤之心机。\end{note}贾珍也另外吩咐每日送上等菜到抱厦内,单与凤姐吃。\begin{note}庚辰眉:写凤之珍贵。\end{note}那凤姐不畏勤劳,\begin{note}蒙双夹:不畏勤劳者,一则任专而易办,一则技痒而莫遏。士为知己者死。不过勤劳,有何可畏?\end{note}天天于卯正二刻就过来点卯理事,\begin{note}庚辰眉:写凤之英勇。\end{note}独在抱厦内起坐,不与众妯娌合群,便有堂客来往,也不迎会。\begin{note}庚辰眉:写凤之骄大。如此写得可叹可笑。\end{note}
\end{parag}


\begin{parag}
    这日乃五七正五日上,那应佛僧正开方破狱,传灯照亡,参阎君,拘都鬼,延请地藏王,开金桥,引幢幡;那道士们正伏章申表,朝三清,叩玉帝;禅僧们行香,放焰口,拜水忏;又有十三众尼僧,搭绣衣,趿红鞋,在灵前默诵接引诸咒,十分热闹。那凤姐必知今日人客不少,在家中歇宿一夜,至寅正,平儿便请起来梳洗。及收拾完备,更衣盥手,吃了几口奶子糖粳粥,漱口已毕,已是卯正二刻了。来旺媳妇率领诸人伺候已久。凤姐出至厅前,上了车,前面打了一对明角灯,大书 “荣国府”三个大字,款款来至宁府。大门上门灯朗挂,两边一色戳灯,照如白昼,白汪汪穿孝仆从两边侍立。请车至正门上,小厮等退去,众媳妇上来揭起车帘。凤姐下了车,一手扶著丰儿,两个媳妇执著手把灯罩,簇拥著凤姐进来。宁府诸媳妇迎来请安接待。凤姐缓缓走入会芳园中登仙阁灵前,一见了棺材,那眼泪恰似断线之珠,滚将下来。院中许多小厮垂手伺候烧纸。凤姐吩咐得一声:“供茶烧纸。”只听一棒锣呜,诸乐齐奏,早有人端过一张大圈椅来,放在灵前,凤姐坐了,放声大哭。\begin{note}庚辰侧:谁家行事,宁不堕泪?\end{note}于是里外男女上下,见凤姐出声,都忙忙接声嚎哭。
\end{parag}


\begin{parag}
    一时贾珍尤氏遣人来劝,凤姐方才止住。来旺媳妇献茶漱口毕,凤姐方起身,别过族中诸人,自入抱厦内来,按名查点,各项人数都已到齐,只有迎送亲客上的一人未到。\begin{note}庚辰侧:须得如此,方见文章妙用。余前批非谬。\end{note}即命传到。那人已张惶愧惧。凤姐冷笑\begin{note}甲戌侧:凡凤姐恼时,偏偏用“笑”字,是章法。\end{note}道:“我说是谁误了,原来是你!\begin{note}庚辰侧:四字有神,是有名姓上等人口气。\end{note}你原比他们有体面,所以才不听我的话。”那人道:“小的天天来的早,只有今儿,醒了觉得早些,因又睡迷了,来迟了一步,求奶奶饶过这次。”正说著,只见荣府中的王兴媳妇来了,\begin{note}甲戌侧:惯起波澜,惯能忙中写闲,又惯用曲笔,又惯综错,真妙!\end{note}\begin{note}庚辰侧:偏用这等闲文间住。\end{note}在前探头。
\end{parag}


\begin{parag}
    凤姐且不发放这人,\begin{note}庚辰侧:的是凤姐作派。\end{note}却先问:“王兴媳妇作什么?”王兴媳妇巴不得先问他完了事,连忙进去说:“领牌取线,打车轿上网络。”\begin{note}庚辰侧:是丧事中用物,闲闲写却。\end{note}说著,将个帖儿递上去。凤姐命彩明念道:“大轿两顶,小轿四顶,车四辆,共用大小络子若干根,用珠儿线若干斤。”凤姐听了,数目相合,便命彩明登记,取荣国府对牌掷下。王兴家的去了。
\end{parag}


\begin{parag}
    凤姐方欲说话时,见荣国府的四个执事人进来,都是要支领东西领牌来的。凤姐命彩明要了帖念过,听了一共四件,指两件说道:“这两件开销错了,再算清了来取。”\begin{note}庚辰侧:好看煞,这等文字。\end{note}说著掷下帖子来。那二人扫兴而去。
\end{parag}


\begin{parag}
    凤姐因见张材家的在旁,\begin{note}庚辰侧:又一顿挫。\end{note}因问:“你有什么事?”张材家的忙取帖儿回说:“就是方才车轿围作成,领取裁缝工银若干两。”凤姐听了,便收了帖子,命彩明登记。待王兴家的交过牌,得了买办的回押相符,然后方与张材家的去领。一面又命念那一个,是为宝玉外书房完竣,支买纸料糊裱。\begin{note}庚辰侧:却从闲中,又引出一件关系文字来。\end{note}凤姐听了,即命收帖儿登记,待张材家的缴清,又发与这人去了。
\end{parag}


\begin{parag}
    凤姐便说道:“明儿他也睡迷了,后儿我也睡迷了,\begin{note}甲戌侧:接上文,一点痕迹俱无,且是仍与方才诸人说话神色口角。庚辰侧:接的紧,且无痕迹,是山断云连法也。\end{note}将来都没了人了。本来要饶你,只是我头一次宽了,下次人就难管,不如现开发的好。”登时放下脸来,喝令:“带出去,打二十板子!”一面又掷下宁国府对牌:“出去说与来升,革他一月银米!”众人听说,又见凤姐眉立,\begin{note}庚辰侧:二字如神。\end{note}知是恼了,不敢怠慢,拖人的出去拖人,执牌传谕的忙去传谕。那人身不由己,已拖出去挨了二十大板,还要进来叩谢。凤姐道:“明日再有误的,打四十,后日的六十,有挨打的,只管误!”说著,吩咐:“散了罢。”窗外众人听说,方各自执事去了。彼时宁国荣国两处执事领牌交牌的,人来人往不绝,那抱愧被打之人含羞去了,\begin{note}甲戌侧:又伏下文,非独为阿凤之威势费此一段笔墨。\end{note}这才知道凤姐利害。众人不敢偷闲,自此兢兢业业,\begin{note}庚辰侧:收拾得好。\end{note}执事保全。不在话下。
\end{parag}


\begin{parag}
    如今且说宝玉\begin{note}庚辰侧:忙中闲笔。\end{note}因见今日人众,恐秦钟受了委曲,因默与他商议,要同他往凤姐处来坐。秦钟道:“他的事多,况且不喜人去,咱们去了,他岂不烦腻。”\begin{note}甲戌侧:纯是体贴人情。\end{note}宝玉道:“他怎好腻我们,不相干,只管跟我来。”说著,便拉了秦钟,直至抱厦。凤姐才吃饭,见他们来了,便笑道:“好长腿子,快上来罢。”宝玉道:“我们偏了。”\begin{note}庚辰侧:家常戏言,毕肖之至!\end{note}凤姐道:“在这边外头吃的,还是那边吃的?”宝玉道:“这边同那些浑人\begin{note}甲戌侧:奇称。试问谁是清人?\end{note}吃什么!原是那边,我们两个同老太太吃了来的。”一面归坐。
\end{parag}


\begin{parag}
    凤姐吃毕,就有宁国府中的一个媳妇来领牌,为支取香灯事。凤姐笑道:“我算著你们今儿该来支取,总不见来,想是忘了。这会子到底来取,要忘了,自然是你们包出来,都便宜了我。”那媳妇笑道:“何尝不是忘了,\begin{note}甲戌侧:此妇亦善迎合。庚辰侧:下人迎合凑趣,毕真。\end{note}方才想起来,再迟一步,也领不成了!”说罢,领牌而去。
\end{parag}


\begin{parag}
    一时登记交牌。秦钟因笑道:“你们两府里都是这牌,倘或别人私弄一个,支了银子跑了,怎样?”\begin{note}庚辰侧:小人语。\end{note}凤姐笑道:“依你说,都没王法了。”宝玉道:“怎么咱们家没人领牌子做东西?”\begin{note}庚辰侧:写不理家务公子之语。\end{note}凤姐道:“人家来领的时候,你还做梦呢。\begin{note}庚辰侧:言甚是也。\end{note}我且问你,你们这夜书多早晚才念呢?”\begin{note}庚辰侧:补前文之未到。\end{note}宝玉道:“巴不得这如今就念才好,他们只是不快给收拾出书房来,这也无法。”凤姐笑道: “你请我一请,包管就快了。”宝玉道:“你要快也不中用。他们该作到那里的,自然就有了。”凤姐笑道:“便是他们作,也得要东西,搁不住我不给对牌是难的。”宝玉听说,便猴\begin{note}庚辰侧:诗中知有炼字一法,不期于《石头记》中多得其妙。\end{note}向凤姐身上立刻要牌,说:“好姐姐,给出牌子来,叫他们要东西去。” 凤姐道:“我乏的身子上生疼,还搁的住揉搓。你放心罢,今儿才领了纸裱糊去了,他们该要的还等叫呢,可不傻了?”宝玉不信,凤姐便叫彩明查册子与宝玉看了。
\end{parag}


\begin{parag}
    正闹著,人回:“苏州去的人昭儿来了。”\begin{note}甲戌侧:接得好!\end{note}凤姐急命唤进来。昭儿打千儿请安。凤姐便问:“回来做什么的?”昭儿道:“二爷打发回来的。林姑老爷是九月初三日巳时没的。\begin{note}甲戌眉:颦儿方可长居荣府之文。\end{note}二爷带了林姑娘\begin{note}庚辰侧:暗写黛玉。\end{note}同送林姑老爷灵到苏州,大约赶年底就回来。二爷打发小的来报个信请安,讨老太太示下,还瞧瞧奶奶家里好,叫把大毛服带几件去。”凤姐道:“你见过别人了没有?”昭儿道:“都见过了。”说毕,连忙退去。凤姐向宝玉笑道:“你林妹妹可在咱们家住长了。”\begin{note}庚辰侧:此系无意中之有意,妙!\end{note}宝玉道:“了不得,想来这几日他不知哭的怎样呢!”说著,蹙眉长叹。
\end{parag}


\begin{parag}
    凤姐见昭儿回来,因当著人未及细问贾琏,心中自是记挂,待要回去,争奈事情繁杂,一时去了,恐有延迟失误,惹人笑话。少不得耐到晚上回来,复令昭儿进来,细问一路平安信息。连夜打点大毛衣服,和平儿亲自检点包裹,再细细追想所需\begin{note}蒙侧:“追想所需”四字,写尽能事者之所以为能事者之底蕴。\end{note}何物,一并包藏交付昭儿。又细细吩咐昭儿“在外好生小心伏侍,不要惹你二爷生气;时时劝他少吃酒,别勾引他认得浑账老婆,\begin{note}甲戌侧:切心事耶?\end{note}”“回来打折你的腿”\begin{note}甲戌侧:此一句最要紧。\end{note}等语。赶乱完了,天已四更将尽,总睡下又走了困,\begin{note}庚辰侧:此为病源伏线。后文方不突然。\end{note}不觉又是天明鸡唱,忙梳洗过宁府中来。
\end{parag}


\begin{parag}
    那贾珍因见发引日近,亲自坐车,带了阴阳司吏,往铁槛寺来踏看寄灵所在。又一一嘱咐住持色空,好生领备新鲜陈设,多请名僧,以备接灵使用。色空忙看晚斋。贾珍也无心茶饭,因天晚不得进城,就在净室胡乱歇了一夜。次日早,便进城来料理出殡之事,一面又派人先往铁槛寺,连夜另外修饰停灵之处,并厨茶等项接灵人口坐落。
\end{parag}


\begin{parag}
    里面凤姐见日期有限,也预先逐细分派料理,一面又派荣府中车轿人从跟王夫人送殡,又顾自己送殡去占下处。目今正值缮国公诰命亡故,王邢二夫人又去打祭送殡;西安郡王妃华诞,送寿礼;镇国公诰命生了长男,预备贺礼;又有胞兄王仁连家眷回南,一面写家信禀叩父母并带往之物;又有迎春染病,每日请医服药,看医生启帖、症源、药案等事,亦难尽述。又兼发引在迩,因此忙的凤姐茶饭也没工夫吃得,坐卧不得清净。\begin{note}庚辰眉:总得好。\end{note}刚到了宁府,荣府的人又跟到宁府;既回到荣府,宁府的人又找到荣府。凤姐见如此,心中倒十分欢喜,并不偷安推托,恐落人褒贬,因此日夜不暇,筹理得十分的整肃。于是合族上下无不称叹者。
\end{parag}


\begin{parag}
    这日伴宿之夕,里面两班小戏并耍百戏的与亲朋堂客伴宿,尤氏犹卧内于室,一应张罗款待,独是凤姐一人周全承应。合族中虽有许多妯娌,但或有羞口的,或有羞脚的,或有不惯见人的,也有惧贵怯官的,种种之类,俱不及凤姐举止舒徐,言语慷慨,珍贵宽大;因此也不把众人放在眼里,挥霍指示,任其所为,目若无人。\begin{note}甲戌侧:写秦氏之丧,却只为凤姐一人。\end{note}一夜中灯明火彩,客送官迎,那百般热闹,自不用说的。至天明,吉时已到,一般六十四名青衣请灵,前面铭旌上大书“奉天洪建兆年不易之朝\begin{note}庚辰眉:“兆年不易之朝,永治太平之国”,奇甚妙甚!\end{note}诰封一等宁国公冢孙妇防护内廷紫禁道御前侍卫龙禁尉享强寿贾门秦氏恭人之灵柩”。一应执事陈设,皆系现赶著新做出来的,一色光艳夺目。宝珠自行未嫁女之礼外,摔丧驾灵,十分哀苦。
\end{parag}


\begin{parag}
    那时官客送殡的,有镇国公牛清之孙现袭一等伯牛继宗,理国公柳彪之孙现袭一等子柳芳,齐国公陈翼之孙世袭三品威镇将军陈瑞文,治国公马魁之孙世袭三品威远将军马尚,修国公侯明之孙世袭一等子侯孝康;缮国公诰命亡故,其孙石光珠守孝不曾来得。\begin{note}庚辰眉:牛,丑也。清,属水,子也。柳拆卯字。彪拆虎字,寅字寓焉。陈即辰。翼火为蛇;巳字寓焉。马,午也。魁拆鬼,鬼,金羊,未字寓焉。侯、猴同音,申也。晓鸣,鸡也,酉字寓焉。石即豕,亥字寓焉。其祖曰守业,即守夜也,犬字寓焉。此所谓十二支寓焉。\end{note}这六家与荣宁二家,当日所称“八公”的便是。余者更有南安郡王之孙,西宁郡王之孙,忠靖侯史鼎,平原侯之孙世袭二等男蒋子宁,定城侯之孙世袭二等男兼京营游击谢鲸,襄阳侯之孙世袭二等男戚建辉,景田侯之孙五城兵马司裘良。余者锦乡侯公子韩奇,神威将军公子冯紫英,卫若兰等诸王孙公子,不可枚数。堂客算来亦有十来顶大轿,三四十小轿,连家下大小轿车辆,不下百十余乘。连前面各色执事、陈设、百耍,浩浩荡荡,一带摆出三四里远来。
\end{parag}


\begin{parag}
    走不多时,路旁彩棚高搭,设席张筵,和音奏乐,俱是各家路祭:第一座是王府东平王府祭棚,第二座是南安郡王祭棚,第三座是西宁郡王,第四座是北静郡王的。原来这四王,当日惟北静王功高,及今子孙犹袭王爵。现今北静王水溶年未弱冠,生得形容秀美,性情谦和。近闻宁国公冢孙媳告殂,因想当日彼此祖父相与之情,同难同荣,未以异姓相视,因此不以王位自居,上日也曾探丧上祭,如今又设路祭,命麾下的各官在此伺候。自己五更入朝,公事一毕,便换了素服,坐大轿鸣锣张伞而来,至棚前落轿。手下各官两旁拥侍,军民人众不得往还。
\end{parag}


\begin{parag}
    一时只见府大殡浩浩荡荡、压地银山一般从北而至。\begin{note}庚辰眉:数字道尽声势。壬午春。畸笏老人。\end{note}早有宁府开路传事人看见,连忙回去报与贾珍。贾珍急命前面驻扎,同贾赦贾政三人连忙迎来,以国礼相见。水溶在轿内欠身含笑答礼,仍以世交称呼接待,并不妄自尊大。贾珍道:“犬妇之丧,累蒙郡驾下临,荫生辈何以克当。” 水溶笑道:“世交之谊,何出此言。”遂回头命长府官主祭代奠。贾赦等一旁还礼毕,复身又来谢恩。
\end{parag}


\begin{parag}
    水溶十分谦逊,因问贾政道:“那一位是衔玉而诞者?\begin{note}庚辰眉:忙中闲笔,点缀玉兄,方不是正文中之正人。作者良苦。壬午春。畸笏。\end{note}几次要见一见,都为杂冗所阻,想今日是来的,何不请来一会?”贾政听说,忙回去,急命宝玉脱去孝服,领他前来。那宝玉素日就曾听得父兄亲友人等说闲话时,赞水溶是个贤王,\begin{note}蒙侧:宝玉见北静王,是为后文伏线。\end{note}且生得才貌双全,风流潇洒,每不以官俗国体所缚。每思相会,只是父亲拘束严密,无由得会,今日反来叫他,自是喜欢。一面走,一面早瞥见那水溶坐在轿内,好个仪表人才。不知近看时又是怎样,且听下回分解。
\end{parag}


\begin{parag}
    \begin{note}庚辰:此回将大家丧事详细剔尽,如见其气概,如闻其声音,丝毫不错,作者不负大家后裔。\end{note}
\end{parag}


\begin{parag}
    \begin{note}写秦死之盛,贾珍之奢,实是却写得一个凤姐。\end{note}
\end{parag}


\begin{parag}
    \begin{note}蒙:大抵事之不理,法之不行,多因偏于爱恶,幽柔不断。请看凤姐无私,犹能整齐丧事。况丈夫辈受职于庙堂之上,倘能奉公守法,一毫不苟,承上率下,何安不行?\end{note}
\end{parag}