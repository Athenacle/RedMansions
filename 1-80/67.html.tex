\chap{六十七}{见土仪颦卿思故里 闻秘事凤姐讯家童}

\begin{parag}
    话说尤三姐自戕之后,尤老娘以及尤二姐、贾珍、尤氏并贾蓉、贾琏等俱不胜悲恸伤感,忙着买棺盛殓,送往城外埋葬。柳湘莲见尤三姐身亡,迷性不悟,尚有痴情眷恋,却被道人数句偈言打破迷关,竟自削发出家,随一疯道人飘然而去,不知何往。
\end{parag}


\begin{parag}
    薛姨妈闻知湘莲已说定了尤三姐,正打算替他买房置器,择日迎娶过门,以报他救命之恩。忽有家中小厮吿知尤三姐自戕与柳湘莲出家之事,心甚叹息。时值宝钗从园中过来,听了这些话,并不在意,乃劝道:“俗语说的好,‘天有不测风云,人有旦夕祸福’。这也是他们前生命定。前日妈妈为他救了哥哥,商量著替他料理,如今已经死的死了,走的走了,依我说,也只好由他罢了。妈妈也不必为他们伤感了。倒是自从哥哥打江南回来了一二十日,贩了来的货物,想来也该发完了,妈妈和哥哥商议商议,酬谢酬谢那同去的那张德辉才是。伙计们辛辛苦苦的,回来几个月了,也该请一请,别叫人家看著无礼似的。”
\end{parag}


\begin{parag}
    母女正说话间,见薛蟠自外而入,眼中尚有泪痕,一进门来,便向他母亲拍手说道:“妈妈可知道柳二哥尤三姐的事么?”薛姨妈说:“我才听见说,正在这里合你妹妹说这件公案呢。”薛蟠道:“妈妈可听见说湘莲跟著一个道士出了家了么?”薛姨妈道:“这越发奇了。怎么柳相公那样一个年轻的聪明人,一时胡涂了,就跟著道士去了呢?我想你们好了一场,他又无父母兄弟,只身一人在此,你该各处找找他才是。靠那道士,能往那里远去?左不过是在这方近左右的庙里寺里罢了。”薛蟠说:“何尝不是呢?我一听见这个信儿,就连忙带了小厮们在各处寻找,连一个影儿也没有。又去问人,都说没看见。”
\end{parag}


\begin{parag}
    薛姨妈说:“你既找寻过,没有,也算把你做朋友的心尽了。再者,你妹妹才说你也回家半个多月了,想货物也该发完了,也该摆桌酒,给张德辉和伙计们,道道乏才是。”薛蟠听说,便道:“妈妈说的很是。倒是妹妹想的周到。因这些日子,为各处发货,又为柳二哥的事忙了这几日,把正经事都误了。要不然,定了明儿后儿,下帖儿请罢。”薛姨妈道:“由你办去罢。”
\end{parag}


\begin{parag}
    话犹未了,外面小厮在门外回说:“张管总着人送了两个箱子来。”薛蟠听了,便命小厮央门外几个伙计搬进了两个夹板夹的大棕箱。薛蟠一见说:“特给妈和妹妹带来的东西,不是伙计送家里来,我都忘了。”
\end{parag}


\begin{parag}
    薛姨妈同宝钗问:“是什么好东西,这样捆着夹着的?”便命人挑了绳子,去了夹板,开了锁看时,却是些绸缎、绫锦、洋货等家常应用之物。独有宝钗他的那个箱子里,除了笔、墨、砚、各色笺纸、香袋、香珠、扇子、扇坠、花粉、胭脂、头油等物外,还有虎丘带来的自行人、酒令儿、水银灌的打筋斗的小小子,沙子灯,一出一出的泥人儿的戏,用青纱罩的匣子装着,又有在虎丘山上作的薛蟠的小像,泥捏成的与薛蟠毫无相差,以及许多碎小顽意儿的东西。宝钗一见,拿着薛蟠的小像细细看了,又看看他哥哥捂着嘴微笑,再和母兄说了一回闲话。便吩咐莺儿:“你带几个老婆子,将我的这个箱子,拿到园子里去,我好就近从那边送人。”说着,便起身辞了母兄往园子里去了。这里薛姨妈将自己这个箱子里的东西取出,一份一份的打点清楚,着莺儿送往贾母并王夫人等处。
\end{parag}


\begin{parag}
    宝钗随着箱子到了自己房中,将东西逐件过了目,除将自己留用之外,遂一一配妥当:也有送笔、墨、纸、砚的,也有送香袋、扇子、香坠的,也有送脂粉、头油的,有单送顽意儿的。一一打点完毕,使莺儿同一个老婆子,送往各处。
\end{parag}


\begin{parag}
    宝钗送东西的ㄚ头回来,说:“也有道谢的,也有赏钱的,独有给巧姐儿的那一份,仍旧拿回来了。”宝钗一见,不知何意,便问:“为什么这一份没送去,还是送了去没收呢?”莺儿说:“我方才给环哥儿送东西的时候,见琏二奶奶往老太太房里去了。”宝钗说:“二奶奶不在家,你只管交给丫头们收下,等二奶奶回来,自有他们告诉就是了。”莺儿听了,又与老婆子出了园子,到了凤姐这边,送了东西,回来见宝钗。
\end{parag}


\begin{parag}
    宝钗问道:“你见了琏二奶奶没有?”莺儿说:“我没见。”宝钗说:“二奶奶还没有回来?”莺儿说:“回来是回来了。因丰儿对我说:‘二奶奶自老太太屋里回来,一脸怒气,叫了平儿去,唧唧咕咕的说话,也不叫人听见。你不必见,等我替你回一声儿就是了。’因此丰儿拿进去,回了二奶奶。我们就回来了。”宝钗听了,自己纳闷,想不出凤姐是为什么生气。
\end{parag}


\begin{parag}
    众人不过收了东西,皆说些见面再谢等语而已。惟有林黛玉见是江南家乡之物,便对着挥泪自叹。紫鹃深知黛玉心肠,在一旁劝道:“宝姑娘送来这些东西,姑娘看着该喜欢才是。”
\end{parag}


\begin{parag}
    话犹未毕,只见宝玉已进来。宝玉见黛玉泪痕满面,便问:“妹妹,又是为的什么?”黛玉不答。旁边紫鹃将嘴向床后桌上一努,宝玉会意,便往床上一看,见堆着许多东西,就知道是宝钗送来的。宝玉深知黛玉是因见了江南来的故乡之物,勾起伤感落泪。便道:“妹妹,你放心!等我明年往江南去,与你带两船来。”黛玉听了这话,说道:“你那里知道我的缘故。”说着眼泪又流了下来。宝玉忙走到床前,挨著黛玉坐下,将那些东西一件一件拿起来,摆弄著细瞧,故意问:“这是什么,叫什么名字?那是什么做的,这样齐整?这是什么,要它做什么使用?妹妹,你瞧,这一件可以摆在书阁儿上作陈设,那件放在条案上当古董儿倒好呢!”一味的将些没要紧的话来支吾,搭讪。黛玉见宝玉可笑的样子,稍将烦恼丢开。宝玉便说道:“宝姐姐送东西来给咱们,我想著,咱们也该到她那里道个谢去才是,不知妹妹可去不去?”黛玉道:“自家姐妹,这倒不必。只是到他那边,薛大哥回来了,必然告诉他些南边的新闻故事儿,我去听听,只当回了家乡一趟的。”说著,眼圈儿又红了。宝玉便站著等他。黛玉只得和他出来,往宝钗那里去了。
\end{parag}


\begin{parag}
    二人到宝钗处,道了谢,宝玉又口口称赞泥人儿等物有趣。宝钗笑道:“原不是什么好东西,不过是远路带来的土物儿,大家看著新鲜些就是了。”黛玉道:“这些东西我们小时候倒不理会,如今看见,真是新鲜物儿了。”宝钗因笑道:“妹妹知道,这就是俗语说的‘物离乡贵’,其实可算什么呢。”宝玉听了这话正触著黛玉方才的心事,连忙拿话岔开:“明年大哥哥还去江南吗?——”话没说完,黛玉早接口道:“——姐姐,你瞧,宝哥哥不是给姐姐来道谢,竟又要定下明年的东西来了。”说的宝钗宝玉都笑了。
\end{parag}


\begin{parag}
    三个人又闲话了一回,因提起黛玉的病来,宝钗劝了一回,因说道:“妹妹若觉著身上不爽快,倒要自己勉强扎挣著出来,各处走走逛逛,散散心,比在屋里闷坐著到底好些。我那两日,不是觉著发懒,浑身发热,只是要歪著?也因为时气不好,怕病,因此寻些事情,自己混著。这两日才觉得好些了。”黛玉道:“姐姐说的何尝不是?我也是这么想著呢。”大家又坐了一会子方散。宝玉仍把黛玉送至潇湘馆门首,才各自回去了。
\end{parag}


\begin{parag}
    且说那赵姨娘因见宝钗送环哥儿物件,心中甚喜,满嘴夸奖:“人人都说宝姑娘会行事,很大方,今日看来,果然不错。他哥哥能带了多少东西来,他挨家送到,并不遗漏一处,也不露出谁薄谁厚,连我们他都想到了,若是林姑娘,即或有人带了东西来,那里轮得到我们娘儿俩身上呢!可见人会行事,真真露着各别另样的好。”赵姨娘因环哥儿得了东西,深为得意,不住的托在掌上摆弄瞧看一会。想宝钗乃系王夫人之表侄女,特要在王夫人跟前卖好儿。自己蝎蝎螫螫的拿着那东西,走至王夫人房中,站在一旁说道:“这是宝姑娘才给环哥的,他年轻轻的人想得周到,我还给了送东西的小ㄚ头二百钱。听说姨太太也给太太送来了,不知是什么东西?你们瞧瞧这一个门里头,就是两份儿,能有多少呢?怪不得老太太同太太都夸他疼他,果然招人疼。”说着,将手里的东西递过去与王夫人瞧,谁知王夫人头也没擡,手也没伸,只口内说了声“好,给环哥儿顽去罢”,并无正眼看一看。赵姨娘因招了一鼻子灰,满肚气恼,无精打彩的回房,将东西丢在一边,也无人问他,他却自己咕嘟着嘴,一边子坐着。
\end{parag}


\begin{parag}
    且说薛蟠听了母亲之言,次日请了张德辉与四位伙计,俱已到齐,不免说些贩卖账目发货之事。不一时,上席让坐,薛蟠挨次斟了酒,薛姨妈又使人出来致谢,大家喝著酒说闲话儿。内中一个道:“今儿这席上短了柳二爷。”薛蟠闻言,把眉一皱,叹口气道:“什么是柳二爷,如今不知那里作‘柳道爷’去了。”众人都诧异道:“这是怎么说?”薛蟠便把湘莲前后事体说了一遍。众人听了,越发骇异,因说道:“怪不的。前儿我们在店里,髣髣髴髴也听见人吵嚷,说:‘有一个道士,三言两语,把一个人度了去了。’又说“‘一阵风刮了去了。’只不知是谁。我们正发货,那里有闲工夫打听这个事去?到如今还是似信不信的,谁知就是柳二爷呢?张德辉道:“柳二爷那样个伶俐人,未必是真跟了道士去罢。他原会些武艺,又有力量,或看破那道士的妖术邪法,特意跟他去,在背地摆布他,也未可知。”薛蟠道:“果然如此,倒也罢了。”众伙计随便喝了几杯酒,吃了饭,大家散了。
\end{parag}


\begin{parag}
    话说宝玉回来,想着黛玉的孤苦,不免替他伤感起来。袭人见宝玉从外面进来坐在那发呆,便问:“就回来了?是不是同林姑娘一块去了宝姑娘那儿?”宝玉说:“我会林姑娘同去的——送林姑娘的东西比送我们的多一两倍呢。”说着话儿,便叫取了枕来,要在床上歪着。袭人说:“琏二奶奶自从病了一场之后,我早就想着要到他那里去看看,你同晴雯麝月呆着,我去看看就来。”宝玉说:“你只管去罢。”言毕,袭人遂换了两件新鲜衣服。嘱咐了晴雯、麝月几句,便出了怡红院。
\end{parag}


\begin{parag}
    至沁芳桥上立住,往四下里观看那园中景致。那时正是夏末秋初,园内蝉闹虫鸣;只是花也开败了,芙蓉池中荷叶新残相间,也将残上来了。倒是近着池边,都发了红铺铺的咕嘟子,衬着碧绿的叶儿,着实可爱。于是一壁里瞧着,一壁里下了桥。走了不远,迎见李纨房里的丫头素云捧着个洋漆盒儿走来。袭人便问:“往那里去送东西?”素云说:“这是我们奶奶给三姑娘送去的菱角儿、鸡头米。”袭人说:“这个东西,是咱们园子里河内采的,还是外头买来的呢?”素云说:“是我们那边刘妈妈的女儿从乡下带来孝敬我们奶奶的。因三姑娘在我们那里坐,奶奶叫人剥了让他吃。他说:‘才吃了热茶了,一会子再吃罢。’所以命我给三姑娘送过去。”言毕,各自散了。
\end{parag}


\begin{parag}
    袭人走著,沿堤看顽了一回。猛抬头看见那边葡萄架底下有人拿著掸子在那里掸什么呢,走到跟前,却是老祝妈。那老婆子见了袭人,便笑嘻嘻的迎上来,说道:“姑娘怎么今日得工夫出来逛逛?”袭人道:“可不是。我要到琏二奶奶家去。你在这里做什么呢?”那婆子道:“我在这里赶蜜蜂儿。今年三伏里雨水少,这果子树上都有虫子,把果子吃的疤瘌流星的掉了好些下来。姑娘还不知道呢,这马蜂最可恶的,一嘟噜上只咬破三两个儿,那破的水滴到好的上头,连这一嘟噜都是要烂的。姑娘你瞧,咱们说话的空儿没赶,就落上许多了。”袭人道:“你就是不住手的赶,也赶不了许多。你倒是告诉买办,叫他多多做些小冷布口袋儿,一嘟噜套上一个,又透风,又不遭塌。”婆子笑道:“倒是姑娘说的是。我今年才管上,那里知道这个巧法儿呢。”
\end{parag}


\begin{parag}
    袭人说:“如今这园子里这些果品有好些种,到是那样先熟的快
    些?”老祝婆子说:“如今才入七月的门,果子都是才红上来,要是好
    吃,想来还得月尽头儿才熟透了呢。姑娘不信,我摘一个给姑娘尝尝。”
    袭人正色说道:“这那里使得?不但没熟吃不得,就是熟了,一则没有
    供鲜,二则主子们尚然没吃,我如何先吃得呢?”老婆子忙笑道:“姑娘说得有理。我因
    为姑娘问我,我白这样说。”袭人说:“我方才告诉你要口袋的话,你就回一回二奶奶,叫管事的作去罢。”言毕,遂一直的出了园子的门,就到凤姐这里来了。
\end{parag}


\begin{parag}
    一到院里,只听凤姐说道:“天理良心,我在这屋里熬的越发成了贼了。”袭人听见这话,知道有原故了,又不好回来,又不好进去,遂把脚步放重些,隔著窗子问道:“平姐姐在家里呢么?”平儿忙答应著迎出来。袭人便问:“二奶奶也在家里呢么,身上可大安了?”说著,已走进来。凤姐装著在床上歪著呢,见袭人进来,也笑著站起来,说:“好些了,叫你惦著。怎么这几日不过我们这边坐坐?”袭人道:“奶奶身上欠安,本该天天过来请安才是。但只怕奶奶身上不爽快,倒要静静儿的歇歇儿,我们来了,倒吵的奶奶烦。”凤姐笑道:“常听见平儿说你背地里还惦著我,常常问我。这就是你尽心了。”一面说著,叫平儿挪了张杌子放在床旁边,让袭人坐下。丰儿端进茶来,袭人欠身道:“妹妹坐著罢。”一面说闲话儿。只见一个小丫头子在外间屋里悄悄的和平儿说:“旺儿来了。在二门上伺候著呢。”袭人知他们有事,又说了两句话,便起身要走。凤姐道:“闲来坐坐,说说话儿,我倒开心。”因命平儿:“送送你妹妹。”平儿答应著送出来。只见两三个小丫头子,都在那里屏声息气齐齐的伺候著。袭人不知何事,便自去了。
\end{parag}


\begin{parag}
    却说平儿送出袭人,进来回道:“旺儿才来了,因袭人在这里我叫他先到外头等等儿,这会子还是立刻叫他呢,还是等著?请奶奶的示下。”凤姐道:“叫他来。”平儿忙叫小丫头去传旺儿进来。这里凤姐又问平儿:“你到底是怎么听见说的?”平儿道:“就是头里那小丫头子的话。他说他在二门里头听见外头两个小厮说:‘这个新二奶奶比咱们旧二奶奶还俊呢,脾气儿也好。’不知是旺儿还是谁,吆喝了两个一顿,说:‘什么新奶奶旧奶奶的,还不快悄悄儿的呢,叫里头知道了,把你的舌头还割了呢。’”平儿正说著,只见一个小丫头进来回说:“旺儿在外头伺候著呢。”凤姐听了,冷笑了一声说:“叫他进来。”那小丫头出来说:“奶奶叫呢。”旺儿连忙答应著进来。旺儿请了安,在外间门口垂手侍立。凤姐儿道:“你过来,我问你话。”旺儿才走到里间门旁站著。凤姐儿道:“你二爷在外头弄了人,你知道不知道?”旺儿又打著千儿回道:“奴才天天在二门上听差事,如何能知道二爷外头的事呢。”凤姐冷笑道:“你自然不知道。你要知道,你怎么拦人呢。”旺儿见这话,知道刚才的话已经走了风了,料著瞒不过,便又跪回道:“奴才实在不知。就是头里兴儿和喜儿两个人在那里混说,奴才吆喝了他们两句。内中深情底里奴才不知道,不敢妄回。求奶奶问兴儿,他是长跟二爷出门的。”凤姐听了,下死劲啐了一口,骂道:“你们这一起没良心的混帐忘八崽子!都是一条藤儿,打量我不知道呢。先去给我把兴儿那个忘八崽子叫了来,你也不许走。问明白了他,回来再问你。好,好,好,这才是我使出来的好人呢!”那旺儿只得连声答应几个是,磕了个头爬起来出去,去叫兴儿。
\end{parag}


\begin{parag}
    却说兴儿正在帐房儿里和小厮们玩呢,听见说二奶奶叫,先唬了一跳,却也想不到是这件事发作了,连忙跟著旺儿进来。旺儿先进去,回说:“兴儿来了。”凤姐儿厉声道:“叫他!”那兴儿听见这个声音儿,早已没了主意了,只得乍著胆子进来。凤姐儿一见,便说:“好小子啊!你和你爷办的好事啊!你只实说罢!”兴儿一闻此言,又看见凤姐气色,早唬软了,不觉跪下,只是磕头。凤姐儿道:“论起这事来,我也听见说不与你相干。但只你不早来回我知道,这就是你的不是了。你要实说了,我还饶你;再有一字虚言,你先摸摸你腔子上几个脑袋瓜子!”兴儿战战兢兢的朝上磕头道:“奶奶问的是什么事,奴才同爷办坏了?”凤姐听了,一腔火都发作起来,喝命:“打嘴巴!”旺儿过来才要打时,凤姐儿骂道:“什么糊涂忘八崽子!叫他自己打,用你打吗!一会子你再各人打你那嘴巴子还不迟呢。”那兴儿真个自己左右开弓打了自己十几个嘴巴。凤姐儿喝声“站住”,问道:“你二爷外头娶了什么新奶奶旧奶奶的事,你大概不知道啊。”兴儿见说出这件事来,越发著了慌,连忙把帽子抓下来在砖地上咕咚咕咚碰的头山响,口里说道:“只求奶奶超生,奴才再不敢撒一个字儿的谎。”凤姐道:“快说!”兴儿直蹶蹶的跪起来回道:“这事头里奴才也不知道。就是这一天,东府里大老爷送了殡,俞禄往珍大爷庙里去领银子。二爷同著蓉哥儿到了东府里,道儿上爷儿两个说起珍大奶奶那边的二位姨奶奶来。二爷夸他好,蓉哥儿哄著二爷,说把二姨奶奶说给二爷。”凤姐听到这里,使劲啐道:“呸,没脸的忘八蛋!他是你那一门子的姨奶奶!”兴儿忙又磕头说:“奴才该死!”往上啾著,不敢言语。凤姐儿道:“完了吗?怎么不说了?”兴儿方才又回道:“奶奶恕奴才,奴才才敢回。”凤姐啐道:“放你妈的屁,这还什么恕不恕了。你好生给我往下说,好多著呢。” 兴儿又回道:“二爷听见这个话就喜欢了。后来奴才也不知道怎么就弄真了。”凤姐微微冷笑道:“这个自然么,你可那里知道呢!你知道的只怕都烦了呢。是了,说底下的罢!”兴儿回道:“后来就是蓉哥儿给二爷找了房子。”凤姐忙问道:“如今房子在那里?”兴儿道:“就在府后头。”凤姐儿道:“哦。”回头瞅著平儿道:“咱们都是死人哪。你听听!”平儿也不敢作声。兴儿又回道:“珍大爷那边给了张家不知多少银子,那张家就不问了。”凤姐道:“这里头怎么又扯拉上什么张家李家咧呢?”兴儿回道:“奶奶不知道,这二奶奶……”刚说到这里,又自己打了个嘴巴,想了想,说道: “那珍大奶奶的妹子……”凤姐儿接著道:“怎么样?快说呀。”兴儿道:“那珍大奶奶的妹子原来从小儿有人家的,姓张,叫什么张华,如今穷的待好讨饭。珍大爷许了他银子,他就退了亲了。”凤姐儿听到这里,点了点头儿,回头便望平儿说道:“你都听见了?小忘八崽子,头里他还说他不知道呢!”兴儿又回道: “后来二爷才叫人裱糊了房子,娶过来了。”凤姐道:“打那里娶过来的?”兴儿回道:“就在他老娘家抬过来的。”凤姐又问:“没人送亲么?”兴儿道:“就是蓉哥儿。还有几个丫头老婆子们,没别人。”凤姐道:“你大奶奶没来吗?”兴儿道:“过了两天,大奶奶才拿了些东西来瞧的。”凤姐儿回头向平儿道:“怪道那两天二爷称赞大奶奶不离嘴呢。”掉过脸来又问兴儿,“谁伏侍呢?自然是你了。”兴儿赶著碰头不言语。凤姐又问:“前头那些日子说给那府里办事,想来办的就是这个了。”兴儿回道:“也有办事的时候,也有往新房子里去的时候。”
\end{parag}


\begin{parag}
    凤姐听了这一篇言词,只气得痴呆了半天,面如金纸,两只吊稍丹凤眼越发直竖起来了,浑身乱战。半晌,连话也说不上来,只是发怔。猛低头,见兴儿还在地下跪着,便说道:“你这个猴儿崽子就该打死。这有什么瞒著我的?你想著瞒了我,就在你那糊涂爷跟前讨了好儿了,你新奶奶好疼你。”兴儿道:“未能早回奶奶,是奴才该死!”便叩头有声。
\end{parag}


\begin{parag}
    凤姐又问道:“谁和他住著呢。”兴儿道:“先是和他娘和妹子在一处。就在十几天前,他妹子自己抹了脖子。他娘得病,昨儿也死了。”凤姐道:“这又为什么?”兴儿随将柳湘莲的事说了一遍。凤姐道:“这个人还算造化高,省了当那出名儿的忘八。”因又问道:“没了别的事了么?”兴儿道:“别的事奴才不知道。奴才刚才说的字字是实话,一字虚假,奶奶问出来只管打死奴才,奴才也无怨的。”凤姐低了一回头,便又指著兴儿说道:“我不看你刚才还有点怕惧儿,不敢撒谎,我把你的腿不给你砸折了。”说著喝声“出去!”兴儿瞌了个头,才爬起来,退到外间门口,不敢就走。凤姐道:“过来,我还有话呢。”兴儿赶忙垂手敬听。凤姐道:“你忙什么,新奶奶等著赏你什么呢?”兴儿也不敢抬头。凤姐道:“我什么时候叫你,你什么时候到。迟一步儿,你试试!出去罢。”兴儿忙答应几个“是”,退出门来。凤姐又叫道:“兴儿!”兴儿赶忙站住。凤姐道:“快出去告诉你二爷去,是不是啊?”兴儿回道:“奴才不敢。”凤姐道:“你出去提一个字儿,隄防你的皮!”兴儿连忙答应著才出去了。凤姐又叫:“旺儿呢?”旺儿连忙答应著过来。凤姐把眼直瞪瞪的瞅了两三句话的工夫,才说道:“好旺儿,很好,去罢!外头有人提一个字儿,全在你身上。”旺儿答应著也出去了。
\end{parag}


\begin{parag}
    且说凤姐见兴儿出去,回头向平儿说:“方才兴儿说的话,你都听见了没有?天下那有这样没脸的男人!吃着碗里,看着锅里,见一个,爱一个,真成了喂不饱的狗,实在是个弃旧迎新的坏货。只可惜这五六品的顶带给他!他别想着俗语说的‘家花那有野花香’的话,他要信了这个话,可就大错了。多早晚在外面闹一个没脸、亲戚朋友见不得的事出来,他才罢手呢!”平儿一旁劝道:“奶奶身子才好了,也不可过于气恼。看二爷自从鲍二的女人那一件事之后,倒收了心,好了呢,如今为什么又干起这样事来?这都是珍大爷他的不是。”凤姐说:“珍大爷固有不是,也总因咱们那位下作不堪的爷他眼馋,人家才引诱他的。俗语说‘牛儿不吃水,也强按头么?’珍大爷干这样的事,珍大奶奶也该拦着不依才是。珍大奶奶也不想一想,把一个妹子要许几家子弟才好,先许了姓张的,今又嫁了姓贾的;天下的男人都死绝了,都嫁到贾家来!难道贾家的衣食这样好不成?那妹子本来也不是他亲的,而且听见说原是个混账烂桃。难道珍大奶奶现做着命妇,家中有这样一个打嘴现世的妹子,也不知道羞臊,躲避着些,反倒大面上扬铃打鼓的,在这门里丢丑,也不怕笑话?珍大爷也是做官的人,别的律例不知道也罢了,连个服中娶亲,停妻再娶,使不得的规矩,他也不知道不成?你替他细想一想,他干的这件事,是疼兄弟,还是害兄弟呢?”平儿说:“珍大爷只顾眼前,叫兄弟喜欢,也不管日后的轻重干系了。”凤姐儿冷笑道:“这是什么‘叫兄弟喜欢’,这是给他毒药吃!若论亲叔伯兄弟中,他年纪又最大,又居长,不知教导兄弟学好,反引诱兄弟学不长进,担罪名儿,日后闹出事来,他在一边缸沿儿上站着看热闹,真真我要骂也骂不出口来。他在那边府里的丑事坏名声,已经叫人听不上了,必定也叫兄弟学他一样,才好显不出他的丑来。这是什么作哥哥的道理?倒不如撒泡尿浸死了,替大老爷死了也罢,活着作什么。”
\end{parag}


\begin{parag}
    平儿看凤姐越说越气,便跪在地下,再三苦劝安慰一会子,凤姐才略消了些气恼。喝了口茶,喘息一回,便要了拐枕,歪在床上,闭眼养神。平儿只得悄悄的退出去了。凤姐将前事从头至尾细细的盘算多时,才得了主意,也不吿诉平儿,却作出个嘻笑自若、毫无恼恨妒嫉的样子来。心下早已算定,只待贾琏起程去平安州,再作道理。要知端的,且听下回分解。
\end{parag}

