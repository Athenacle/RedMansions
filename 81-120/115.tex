\chap{一百一十五}{惑偏私惜春矢素志 證同類寶玉失相知}



\begin{parag}
    話說寶玉爲自己失言被寶釵問住,想要掩飾過去,只見秋紋進來說:“外頭老爺叫二爺呢。”寶玉巴不得一聲,便走了。去到賈政那裏,賈政道:“我叫你來不爲別的,現在你穿着孝,不便到學裏去,你在家裏,必要將你念過的文章溫習溫習。我這幾天倒也閒着,隔兩三日要做幾篇文章我瞧瞧,看你這些時進益了沒有。”寶玉只得答應着。賈政又道:“你環兄弟蘭侄兒我也叫他們溫習去了。倘若你作的文章不好,反倒不及他們,那可就不成事了。”寶玉不敢言語,答應了個“是”,站着不動。賈政道:“去罷。”寶玉退了出來,正撞見賴大諸人拿着些冊子進來。
\end{parag}


\begin{parag}
    寶玉一溜煙回到自己房中,寶釵問了知道叫他作文章,倒也喜歡,惟有寶玉不願意,也不敢怠慢。正要坐下靜靜心,見有兩個姑子進來,寶玉看是地藏庵的,來和寶釵說:“請二奶奶安。”寶釵待理不理的說:“你們好?”因叫人來:“倒茶給師父們喝。”寶玉原要和那姑子說話,見寶釵似乎厭惡這些,也不好兜搭。那姑子知道寶釵是個冷人,也不久坐,辭了要去。寶釵道:“再坐坐去罷。”那姑子道:“我們因在鐵檻寺做了功德,好些時沒來請太太奶奶們的安,今日來了,見過了奶奶太太們,還要看四姑娘呢。”寶釵點頭,由他去了。
\end{parag}


\begin{parag}
    那姑子便到惜春那裏,見了彩屏,說:“姑娘在那裏呢?”彩屏道:“不用提了。姑娘這幾天飯都沒喫,只是歪着。”那姑子道:“爲什麼?”彩屏道:“說也話長。你見了姑娘只怕他便和你說了。”惜春早已聽見,急忙坐起來說:“你們兩個人好啊?見我們家事差了,便不來了。”那姑子道:“阿彌陀佛!有也是施主,沒也是施主,別說我們是本家庵裏的,受過老太太多少恩惠呢。如今老太太的事,太太奶奶們都見了,只沒有見姑娘,心裏惦記,今兒是特特的來瞧姑娘來的。”惜春便問起水月庵的姑子來,那姑子道:“他們庵裏鬧了些事,如今門上也不肯常放進來了。”便問惜春道:“前兒聽見說櫳翠庵的妙師父怎麼跟了人去了?”惜春道:“那裏的話!說這個話的人堤防着割舌頭。人家遭了強盜搶去,怎麼還說這樣的壞話。”那姑子道:“妙師父的爲人怪僻,只怕是假惺惺罷。在姑娘面前我們也不好說的。那裏象我們這些粗夯人,只知道諷經唸佛,給人家懺悔,也爲着自己修個善果。”惜春道:“怎麼樣就是善果呢?”那姑子道:“除了咱們家這樣善德人家兒不怕,若是別人家,那些誥命夫人小姐也保不住一輩子的榮華。到了苦難來了,可就救不得了。只有個觀世音菩薩大慈大悲,遇見人家有苦難的就慈心發動,設法兒救濟。爲什麼如今都說大慈大悲救苦救難的觀世音菩薩呢。我們修了行的人,雖說比夫人小姐們苦多着呢,只是沒有險難的了。雖不能成佛作祖,修修來世或者轉個男身,自己也就好了。不象如今脫生了個女人胎子,什麼委屈煩難都說不出來。姑娘你還不知道呢,要是人家姑娘們出了門子,這一輩子跟着人是更沒法兒的。若說修行,也只要修得真。那妙師父自爲才情比我們強,他就嫌我們這些人俗,豈知俗的才能得善緣呢。他如今到底是遭了大劫了。”惜春被那姑子一番話說得合在機上,也顧不得丫頭們在這裏,便將尤氏待他怎樣,前兒看家的事說了一遍。並將頭髮指給他瞧道:“你打諒我是什麼沒主意戀火坑的人麼?早有這樣的心,只是想不出道兒來。”那姑子聽了,假作驚慌道:“姑娘再別說這個話!珍大奶奶聽見還要罵殺我們,攆出庵去呢!姑娘這樣人品,這樣人家,將來配個好姑爺,享一輩子的榮華富貴。”惜春不等說完,便紅了臉說:“珍大奶奶攆得你,我就攆不得麼?”那姑子知是真心,便索性激他一激,說道:“姑娘別怪我們說錯了話,太太奶奶們那裏就依得姑娘的性子呢?那時鬧出沒意思來倒不好。我們倒是爲姑娘的話。”惜春道:“這也瞧罷咧。”彩屏等聽這話頭不好,便使個眼色兒給姑子叫他去。那姑子會意,本來心裏也害怕,不敢挑逗,便告辭出去。惜春也不留他,便冷笑道:“打諒天下就是你們一個地藏庵麼!”那姑子也不敢答言去了。
\end{parag}


\begin{parag}
    彩屏見事不妥,恐擔不是,悄悄的去告訴了尤氏說:“四姑娘絞頭髮的念頭還沒有息呢。他這幾天不是病,竟是怨命。奶奶堤防些,別鬧出事來,那會子歸罪我們身上。”尤氏道:“他那裏是爲要出家,他爲的是大爺不在家,安心和我過不去,也只好由他罷了。”彩屏等沒法,也只好常常勸解。豈知惜春一天一天的不喫飯,只想絞頭髮。彩屏等喫不住,只得到各處告訴。邢王二夫人等也都勸了好幾次,怎奈惜春執迷不解。
\end{parag}


\begin{parag}
    邢王二夫人正要告訴賈政,只聽外頭傳進來說:“甄家的太太帶了他們家的寶玉來了。”衆人急忙接出,便在王夫人處坐下。衆人行禮,敘些溫寒,不必細述。只言王夫人提起甄寶玉與自己的寶玉無二,要請甄寶玉一見。傳話出去,回來說道:“甄少爺在外書房同老爺說話,說的投了機了,打發人來請我們二爺三爺,還叫蘭哥兒,在外頭喫飯。吃了飯進來。”說畢,裏頭也便擺飯。不題。
\end{parag}


\begin{parag}
    且說賈政見甄寶玉相貌果與寶玉一樣,試探他的文才,竟應對如流,甚是心敬,故叫寶玉等三人出來警勵他們。再者倒底叫寶玉來比一比。寶玉聽命,穿了素服,帶了兄弟侄兒出來,見了甄寶玉,竟是舊相識一般。那甄寶玉也象那裏見過的,兩人行了禮,然後賈環賈蘭相見。本來賈政席地而坐,要讓甄寶玉在椅子上坐。甄寶玉因是晚輩,不敢上坐,就在地下鋪了褥子坐下。如今寶玉等出來,又不能同賈政一處坐着,爲甄寶玉又是晚一輩,又不好叫寶玉等站着。賈政知是不便,站着又說了幾句話,叫人擺飯,說:“我失陪,叫小兒輩陪着,大家說說話兒,好叫他們領領大教。”甄寶玉遜謝道:“老伯大人請便。侄兒正欲領世兄們的教呢。”賈政回覆了幾句,便自往內書房去。那甄寶玉反要送出來,賈政攔住。寶玉等先搶了一步出了書房門坎,站立着看賈政進去,然後進來讓甄寶玉坐下。彼此套敘了一回,諸如久慕竭想的話,也不必細述。
\end{parag}


\begin{parag}
    且說賈寶玉見了甄寶玉,想到夢中之景,並且素知甄寶玉爲人必是和他同心,以爲得了知己。因初次見面,不便造次。且又賈環賈蘭在坐,只有極力誇讚說:“久仰芳名,無由親炙。今日見面,真是謫仙一流的人物。”那甄寶玉素來也知賈寶玉的爲人,今日一見,果然不差,”只是可與我共學,不可與你適道,他既和我同名同貌,也是三生石上的舊精魂了。既我略知了些道理,怎麼不和他講講。但是初見,尚不知他的心與我同不同,只好緩緩的來。”便道:“世兄的才名,弟所素知的,在世兄是數萬人的裏頭選出來最清最雅的,在弟是庸庸碌碌一等愚人,忝附同名,殊覺玷辱了這兩個字。”賈寶玉聽了,心想:“這個人果然同我的心一樣的。但是你我都是男人,不比那女孩兒們清潔,怎麼他拿我當作女孩兒看待起來?”便道:“世兄謬讚,實不敢當。弟是至濁至愚,只不過一塊頑石耳,何敢比世兄品望高清,實稱此兩字。”甄寶玉道:“弟少時不知分量,自謂尚可琢磨。豈知家遭消索,數年來更比瓦礫猶殘,雖不敢說歷盡甘苦,然世道人情略略的領悟了好些。世兄是錦衣玉食,無不遂心的,必是文章經濟高出人上,所以老伯鍾愛,將爲席上之珍。弟所以才說尊名方稱。”賈寶玉聽這話頭又近了碌蠹的舊套,想話回答。賈環見未與他說話,心中早不自在。倒是賈蘭聽了這話甚覺合意,便說道:“世叔所言固是太謙,若論到文章經濟,實在從歷練中出來的,方爲真才實學。在小侄年幼,雖不知文章爲何物,然將讀過的細味起來,那膏粱文繡比着令聞廣譽,真是不啻百倍的了。”甄寶玉未及答言,賈寶玉聽了蘭兒的話心裏越發不合,想道:“這孩子從幾時也學了這一派酸論。”便說道:“弟聞得世兄也詆盡流俗,性情中另有一番見解。今日弟幸會芝範,想欲領教一番超凡入聖的道理,從此可以淨洗俗腸,重開眼界,不意視弟爲蠢物,所以將世路的話來酬應。”甄寶玉聽說,心裏曉得”他知我少年的性情,所以疑我爲假。我索性把話說明,或者與我作個知心朋友也是好的。”便說道:“世兄高論,固是真切。但弟少時也曾深惡那些舊套陳言,只是一年長似一年,家君致仕在家,懶於酬應,委弟接待。後來見過那些大人先生盡都是顯親揚名的人,便是著書立說,無非言忠言孝,自有一番立德立言的事業,方不枉生在聖明之時,也不致負了父親師長養育教誨之恩,所以把少時那一派迂想癡情漸漸的淘汰了些。如今尚欲訪師覓友,教導愚蒙,幸會世兄,定當有以教我。適才所言,並非虛意。”賈寶玉愈聽愈不耐煩,又不好冷淡,只得將言語支吾。幸喜裏頭傳出話來說:“若是外頭爺們吃了飯,請甄少爺裏頭去坐呢。”寶玉聽了,趁勢便邀甄寶玉進去。
\end{parag}


\begin{parag}
    那甄寶玉依命前行,賈寶玉等陪着來見王夫人。賈寶玉見是甄太太上坐,便先請過了安,賈環賈蘭也見了。甄寶玉也請了王夫人的安。兩母兩子互相廝認。雖是賈寶玉是娶過親的,那甄夫人年紀已老,又是老親,因見賈寶玉的相貌身材與他兒子一般,不禁親熱起來。王夫人更不用說,拉着甄寶玉問長問短,覺得比自己家的寶玉老成些。回看賈蘭,也是清秀超羣的,雖不能象兩個寶玉的形像,也還隨得上。只有賈環粗夯,未免有偏愛之色。衆人一見兩個寶玉在這裏,都來瞧看,說道:“真真奇事,名字同了也罷,怎麼相貌身材都是一樣的。虧得是我們寶玉穿孝,若是一樣的衣服穿着,一時也認不出來。”內中紫鵑一時癡意發作,便想起黛玉來,心裏說道:“可惜林姑娘死了,若不死時,就將那甄寶玉配了他,只怕也是願意的。”正想着,只聽得甄夫人道:“前日聽得我們老爺回來說,我們寶玉年紀也大了,求這裏老爺留心一門親事。”王夫人正愛甄寶玉,順口便說道:“我也想要與令郎作伐。我家有四個姑娘,那三個都不用說,死的死,嫁的嫁了,還有我們珍大侄兒的妹子,只是年紀過小几歲,恐怕難配。倒是我們大媳婦的兩個堂妹子生得人才齊整,二姑娘呢,已經許了人家,三姑娘正好與令郎爲配。過一天我給令郎作媒,但是他家的家計如今差些。”甄夫人道:“太太這話又客套了。如今我們家還有什麼,只怕人家嫌我們窮罷了。”王夫人道:“現今府上覆又出了差,將來不但復舊,必是比先前更要鼎盛起來。”甄夫人笑着道:“但願依着太太的話更好。這麼着就求太太作個保山。”甄寶玉聽他們說起親事,便告辭出來。賈寶玉等只得陪着來到書房,見賈政已在那裏,復又立談幾句。聽見甄家的人來回甄寶玉道:“太太要走了,請爺回去罷。”於是甄寶玉告辭出來。賈政命寶玉環蘭相送。不題。
\end{parag}


\begin{parag}
    且說寶玉自那日見了甄寶玉之父,知道甄寶玉來京,朝夕盼望。今兒見面原想得一知己,豈知談了半天,竟有些冰炭不投。悶悶的回到自己房中,也不言,也不笑,只管發怔。寶釵便問:“那甄寶玉果然象你麼?”寶玉道:“相貌倒還是一樣的。只是言談間看起來並不知道什麼,不過也是個祿蠹。”寶釵道:“你又編派人家了。怎麼就見得也是個祿蠹呢?”寶玉道:“他說了半天,並沒個明心見性之談,不過說些什麼文章經濟,又說什麼爲忠爲孝,這樣人可不是個祿蠹麼!只可惜他也生了這樣一個相貌。我想來,有了他,我竟要連我這個相貌都不要了。”寶釵見他又發呆話,便說道:“你真真說出句話來叫人發笑,這相貌怎麼能不要呢。況且人家這話是正理,做了一個男人原該要立身揚名的,誰象你一味的柔情私意。不說自己沒有剛烈,倒說人家是祿蠹。”寶玉本聽了甄寶玉的話甚不耐煩,又被寶釵搶白了一場,心中更加不樂,悶悶昏昏,不覺將舊病又勾起來了,並不言語,只是傻笑。寶釵不知,只道是“我的話錯了,他所以冷笑”,也不理他。豈知那日便有些發呆,襲人等慪他也不言語。過了一夜,次日起來只是發呆,竟有前番病的樣子。
\end{parag}


\begin{parag}
    一日,王夫人因爲惜春定要絞發出家,尤氏不能攔阻,看着惜春的樣子是若不依他必要自盡的,雖然晝夜着人看着,終非常事,便告訴了賈政。賈政嘆氣跺腳,只說:“東府裏不知幹了什麼,鬧到如此地位。”叫了賈蓉來說了一頓,叫他去和他母親說,認真勸解勸解。”若是必要這樣,就不是我們家的姑娘了。”豈知尤氏不勸還好,一勸了更要尋死,說:“做了女孩兒終不能在家一輩子的,若象二姐姐一樣,老爺太太們倒要煩心,況且死了。如今譬如我死了似的,放我出了家,乾乾淨淨的一輩子,就是疼我了。況且我又不出門,就是櫳翠庵,原是咱們家的基趾,我就在那裏修行。我有什麼,你們也照應得着。現在妙玉的當家的在那裏。你們依我呢,我就算得了命了;若不依我呢,我也沒法,只有死就完了。我如若遂了自己的心願,那時哥哥回來我和他說,並不是你們逼着我的。若說我死了,未免哥哥回來倒說你們不容我。”尤氏本與惜春不合,聽他的話也似乎有理,只得去回王夫人。
\end{parag}


\begin{parag}
    王夫人已到寶釵那裏,見寶玉神魂失所,心下着忙,便說襲人道:“你們忒不留神,二爺犯了病也不來回我。”襲人道:“二爺的病原來是常有的,一時好,一時不好。天天到太太那裏仍舊請安去,原是好好兒的,今兒才發糊塗些。二奶奶正要來回太太,恐防太太說我們大驚小怪。”寶玉聽見王夫人說他們,心裏一時明白,恐他們受委屈,便說道:“太太放心,我沒什麼病,只是心裏覺着有些悶悶的。”王夫人道:“你是有這病根子,早說了好請大夫瞧瞧,喫兩劑藥好了不好!若再鬧到頭裏丟了玉的時候似的,就費事了。”寶玉道:“太太不放心便叫個人來瞧瞧,我就吃藥。”王夫人便叫丫頭傳話出來請大夫。這一個心思都在寶玉身上,便將惜春的事忘了。遲了一回,大夫看了,服藥。王夫人回去。
\end{parag}


\begin{parag}
    過了幾天,寶玉更糊塗了,甚至於飯食不進,大家着急起來。恰又忙着脫孝,家中無人,又叫了賈芸來照應大夫。賈璉家下無人,請了王仁來在外幫着料理。那巧姐兒是日夜哭母,也是病了。所以榮府中又鬧得馬仰人翻。
\end{parag}


\begin{parag}
    一日又當脫孝來家,王夫人親身又看寶玉,見寶玉人事不醒,急得衆人手足無措。一面哭着,一面告訴賈政說:“大夫回了,不肯下藥,只好預備後事。”賈政嘆氣連連,只得親自看視,見其光景果然不好,便又叫賈璉辦去。賈璉不敢違拗,只得叫人料理。手頭又短,正在爲難,只見一個人跑進來說:“二爺,不好了,又有饑荒來了。”賈璉不知何事,這一唬非同小可,瞪着眼說道:“什麼事?”那小廝道:“門上來了一個和尚,手裏拿着二爺的這塊丟的玉,說要一萬賞銀。”賈璉照臉啐道:“我打量什麼事,這樣慌張。前番那假的你不知道麼!就是真的,現在人要死了,要這玉做什麼!”小廝道:“奴才也說了,那和尚說給他銀子就好了。”又聽着外頭嚷進來說:“這和尚撒野,各自跑進來了,衆人攔他攔不住。”賈璉道:“那裏有這樣怪事,你們還不快打出去呢。”正鬧着,賈政聽見了,也沒了主意了。裏頭又哭出來說:“寶二爺不好了!”賈政益發着急。只見那和尚嚷道:“要命拿銀子來!”賈政忽然想起,頭裏寶玉的病是和尚治好的,這會子和尚來,或者有救星。但是這玉倘或是真,他要起銀子來怎麼樣呢?想了一想,姑且不管他,果真人好了再說。
\end{parag}


\begin{parag}
    賈政叫人去請,那和尚已進來了,也不施禮,也不答話,便往裏就跑。賈璉拉着道:“裏頭都是內眷,你這野東西混跑什麼!”那和尚道:“遲了就不能救了。”賈璉急得一面走一面亂嚷道:“裏頭的人不要哭了,和尚進來了。”王夫人等只顧着哭,那裏理會。賈璉走近來又嚷,王夫人等回過頭來,見一個長大的和尚,唬了一跳,躲避不及。那和尚直走到寶玉炕前,寶釵避過一邊,襲人見王夫人站着,不敢走開。只見那和尚道:“施主們,我是送玉來的。”說着,把那塊玉擎着道:“快把銀子拿出來,我好救他。”王夫人等驚惶無措,也不擇真假,便說道:“若是救活了人,銀子是有的。”那和尚笑道:“拿來。”王夫人道:“你放心,橫豎折變的出來。”和尚哈哈大笑,手拿着玉在寶玉耳邊叫道:“寶玉,寶玉,你的寶玉回來了。”說了這一句,王夫人等見寶玉把眼一睜。襲人說道:“好了。”只見寶玉便問道:“在那裏呢?”那和尚把玉遞給他手裏。寶玉先前緊緊的攥着,後來慢慢的得過手來,放在自己眼前細細的一看說:“噯呀,久違了!”裏外衆人都喜歡的唸佛,連寶釵也顧不得有和尚了。賈璉也走過來一看,果見寶玉回過來了,心裏一喜,疾忙躲出去了。
\end{parag}


\begin{parag}
    那和尚也不言語,趕來拉着賈璉就跑。賈璉只得跟着到了前頭,趕着告訴賈政。賈政聽了喜歡,即找和尚施禮叩謝。和尚還了禮坐下。賈璉心下狐疑:“必是要了銀子才走。”賈政細看那和尚,又非前次見的,便問:“寶剎何方?法師大號?這玉是那裏得的?怎麼小兒一見便會活過來呢?”那和尚微微笑道:“我也不知道,只要拿一萬銀子來就完了。”賈政見這和尚粗魯,也不敢得罪,便說:“有。”和尚道:“有便快拿來罷,我要走了。”賈政道:“略請少坐,待我進內瞧瞧。”和尚道:“你去快出來纔好。”
\end{parag}


\begin{parag}
    賈政果然進去,也不及告訴便走到寶玉炕前。寶玉見是父親來,欲要爬起,因身子虛弱起不來。王夫人按着說道:“不要動。”寶玉笑着拿這玉給賈政瞧道:“寶玉來了。”賈政略略一看,知道此事有些根源,也不細看,便和王夫人道:“寶玉好過來了。這賞銀怎麼樣?”王夫人道:“盡着我所有的折變了給他就是了。”寶玉道:“只怕這和尚不是要銀子的罷。”賈政點頭道:“我也看來古怪,但是他口口聲聲的要銀子。”王夫人道:“老爺出去先款留着他再說。”賈政出來,寶玉便嚷餓了,喝了一碗粥,還說要飯。婆子們果然取了飯來,王夫人還不敢給他喫。寶玉說:“不妨的,我已經好了。”便爬着吃了一碗,漸漸的神氣果然好過來了,便要坐起來。麝月上去輕輕的扶起,因心裏喜歡,忘了情說道:“真是寶貝,纔看見了一會兒就好了。虧的當初沒有砸破。”寶玉聽了這話,神色一變,把玉一撂,身子往後一仰。未知死活,下回分解。
\end{parag}