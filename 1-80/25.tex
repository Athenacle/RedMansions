\chap{二十五}{魘魔法叔嫂逢五鬼 紅樓夢通靈遇雙真}


\begin{parag}
    \begin{note}蒙回前總:有緣的,推不開;如心的,死不改。縱然是,通靈神玉也遭塵敗。夢裏徘徊,醒後疑猜,時時兜(左手右兜)底上心來。怕人窺破笑盈腮,獨自無言偷打嚕。這的是,前生造定今生債。\end{note}
\end{parag}


\begin{parag}
    話說紅玉心神恍惚,情思纏綿,忽朦朧睡去,遇見賈芸要拉他,卻回身一跑,被門檻絆了一跤,唬醒過來,方知是夢。因此翻來覆去,一夜無眠。至次日天明,方纔起來,就有幾個丫頭子來會他去打掃房子地面,提洗臉水。這紅玉也不梳洗,向鏡中胡亂挽了一挽頭髮,洗了洗手,腰內束了一條汗巾子,便來打掃房屋。誰知寶玉昨兒見了紅玉,也就留了心。若要直點名喚他來使用,一則怕襲人等寒心;\begin{note}甲側:是寶玉心中想,不是襲人拈酸。\end{note}二則又不知紅玉是何等行爲,若好還罷了,\begin{note}甲側:不知“好”字是如何講?答曰:在“何等行爲”四字上看便知,玉兒每情不情,況有情者乎?\end{note}若不好起來,那時倒不好退送的。因此心下悶悶的,早起來也不梳洗,只坐著出神。一時下了窗子,隔著紗屜子,向外看的真切,只見好幾個丫頭在那裏掃地,都擦胭抹粉,簪花插柳的,\begin{note}甲側:八字寫盡蠢鬟,是爲襯紅玉,亦如用豪貴人家濃妝豔飾插金戴銀的襯寶釵、黛玉也。\end{note}獨不見昨兒那一個。寶玉便靸了鞋晃出了房門,只裝著看花兒,這裏瞧瞧,那裏望望,\begin{note}庚側:文字有層次。\end{note}一抬頭,只見西南角上游廊底下欄杆上似有一個人倚在那裏,卻恨面前有一株海棠花遮著,看不真切。\begin{note}甲雙夾:餘所謂此書之妙皆從詩詞句中翻出者,皆系此等筆墨也。試問觀者,此非“隔花人遠天涯近”乎?可知上幾回非餘妄擬也。\end{note}只得又轉了一步,仔細一看,可不是昨兒那個丫頭在那裏出神。待要迎上去,又不好去的。正想著,忽見碧痕來催他洗臉,只得進去了。不在話下。
\end{parag}


\begin{parag}
    卻說紅玉正自出神,忽見襲人招手叫他,\begin{note}甲側:此處方寫出襲人來,是襯貼法。\end{note}只得走上前來。襲人笑道:“我們這裏的噴壺還沒有收拾了來呢,你到林姑娘那裏去,把他們的借來使使。”紅玉答應了,便走出來往瀟湘館去。正走上翠煙橋,抬頭一望,只見山坡上高處都是攔著幃幙,方想起今兒有匠役在裏頭種樹。因轉身一望,只見那邊遠遠一簇人在那裏掘土,賈芸正坐在那山子石上。紅玉待要過去,又不敢過去,只得悶悶的向瀟湘館取了噴壺回來,無精打彩自向房內倒著。衆人只說他一時身上不爽快,都不理論。\begin{note}甲側:文字到此一頓,狡猾之甚。\end{note}
\end{parag}


\begin{parag}
    展眼過了一日,\begin{note}甲側:必雲“展眼過了一日”者,是反襯紅玉“挨一刻似一夏”也,知乎?\end{note}原來次日就是王子騰夫人的壽誕,那裏原打發人來請賈母王夫人的,王夫人見賈母不自在,也便不去了。\begin{note}甲側:所謂一筆兩用也!\end{note}倒是薛姨媽同鳳姐兒並賈家幾個姊妹、寶釵、寶玉一齊都去了,至晚方回。
\end{parag}


\begin{parag}
    可巧王夫人見賈環下了學,便命他來抄個《金剛咒》\begin{note}甲側:用《金剛咒》引五鬼法。\end{note}唪誦唪誦。那賈環正在王夫人炕上坐著,命人點燈,拿腔作勢的抄寫。\begin{note}甲側:小人乍得意者齊來一玩。\end{note}一時又叫彩雲倒杯茶來,一時又叫玉釧兒來剪剪蠟花,一時又說金釧兒擋了燈影。衆丫鬟們素日厭惡他,都不答理。只有彩霞還和他合的來,\begin{note}甲側:暗中又伏一風月之隙。\end{note}倒了一鍾茶來遞與他。因見王夫人和人說話兒,他便悄悄的向賈環說道:“你安些分罷,何苦討這個厭那個厭的。”賈環道:“我也知道了,你別哄我。如今你和寶玉好,把我不答理,我也看出來了。”彩霞咬著嘴脣,向賈環頭上戳了一指頭,說道:“沒良心的!狗咬呂洞賓,不識好人心。”\begin{note}甲雙夾:風月之情,皆系彼此業障所牽。雖雲“惺惺惜惺惺”,但亦從業障而來。蠢婦配才郎,世間固不少,然俏女慕村夫者尤多,所謂業障牽魔,不在才貌之論。\end{note}\begin{note}庚眉:此等世俗之言,亦因人而用,妥極當極!壬午孟夏,雨窗。畸笏。\end{note}
\end{parag}


\begin{parag}
    兩人正說著,只見鳳姐來了,拜見過王夫人。王夫人便一長一短的問他,今兒是那幾位堂客,戲文好歹,酒席如何等語。說了不多幾句話,寶玉也來了,進門見了王夫人,不過規規矩矩說了幾句,\begin{note}甲側:是大家子弟模樣。\end{note}便命人除去抹額,脫了袍服,拉了靴子,便一頭滾在王夫人懷裏。\begin{note}甲側:餘几几失聲哭出。\end{note}王夫人便用手滿身滿臉摩挲撫弄他,\begin{note}甲側:普天下幼年喪母者齊來一哭。\end{note}寶玉也搬著王夫人的脖子說長道短的。\begin{note}甲側:慈母嬌兒寫盡矣。\end{note}王夫人道:“我的兒,你又喫多了酒,臉上滾熱。你還只是揉搓,一會鬧上酒來。還不在那裏靜靜的倒一會子呢。”說著,便叫人拿個枕頭來。寶玉聽說便下來,在王夫人身後倒下,又叫彩霞來替他拍著。寶玉便和彩霞說笑,只見彩霞淡淡的,不大答理,兩眼睛只向賈環處看。寶玉便拉他的手笑道:“好姐姐,你也理我理兒呢。”一面說,一面拉他的手,彩霞奪手不肯,便說:“再鬧,我就嚷了。”
\end{parag}


\begin{parag}
    二人正鬧著,原來賈環聽的見,素日原恨寶玉,如今又見他和彩霞鬧,心中越發按不下這口毒氣。雖不敢明言,卻每每暗中算計,\begin{note}甲側:已伏金釧回矣。\end{note}只是不得下手,今見相離甚近,便要用熱油燙瞎他的眼睛。因而故意裝作失手,把那一盞油汪汪的蠟燈向寶玉臉上只一推。只聽寶玉“噯喲”了一聲,滿屋裏衆人都唬了一跳。連忙將地下的戳燈挪過來,又將裏外間屋的燈拿了三四盞看時,只見寶玉滿臉滿頭都是油。王夫人又急又氣,一面命人來替寶玉擦洗,一面又罵賈環。鳳姐三步兩步的上炕去替寶玉收拾著,\begin{note}甲側:阿鳳活現紙上。\end{note}一面笑道:“老三還是這麼慌腳雞似的,我說你上不得高臺盤。趙姨娘時常也該教導教導他。”\begin{note}庚側:爲下文緊一步。\end{note}一句話提醒了王夫人,那王夫人不罵賈環,便叫過趙姨娘來罵道:“養出這樣黑心不知道理下流種子來,也不管管!幾番幾次我都不理論,\begin{note}甲側:補出素日來。\end{note}你們得了意了,越發上來了!”
\end{parag}


\begin{parag}
    那趙姨娘素日雖然常懷嫉妒之心,不忿鳳姐寶玉兩個,也不敢露出來;如今賈環又生了事,受這場惡氣,不但吞聲承受,而且還要走去替寶玉收拾。只見寶玉左邊臉上燙了一溜燎泡出來,幸而眼睛竟沒動。王夫人看了,又是心疼,又怕明日賈母問怎麼回答,急的又把趙姨娘數落一頓。\begin{note}甲側:總是爲楔緊“五鬼”一回文字。\end{note}然後又安慰了寶玉一回,又命取敗毒消腫藥來敷上。寶玉道:“有些疼,還不妨事。明兒老太太問,就說是我自己燙的罷了。”鳳姐笑\begin{note}甲側:兩笑,壞極。庚眉:爲五鬼法作耳,非泛文也。雨窗。\end{note}道:“便說是自己燙的,\begin{note}甲側:玉兄自是悌弟之心性,一嘆。\end{note}也要罵人爲什麼不小心看著,叫你燙了!橫豎有一場氣生的,到明兒憑你怎麼說去罷。”\begin{note}甲側:壞極!總是調唆口吻,趙氏寧不覺乎?\end{note}王夫人命人好生送了寶玉回房去後,襲人等見了,都慌的了不得。
\end{parag}


\begin{parag}
    林黛玉見寶玉出了一天門,就覺悶悶的,沒個可說話的人。至晚正打發人來問了兩三遍回來不曾,這遍方纔回來,又偏生燙了。林黛玉便趕著來瞧,只見寶玉正拿鏡子照呢,左邊臉上滿滿的敷了一臉的藥。林黛玉只當燙的十分利害,忙上來問怎麼燙了,要瞧瞧。寶玉見他來了,忙把臉遮著,搖手叫他出去,不肯叫他看── 知道他的癖性喜潔,見不得這些東西。\begin{note}甲雙夾:寫寶玉文字,此等方是正緊筆墨。\end{note}林黛玉自己也知道自己也有這件癖性,\begin{note}甲雙夾:寫林黛玉文字,此等方是正經筆墨。故二人文字雖多,如此等暗伏淡寫處亦不少,觀者實實看不出者。\end{note}知道寶玉的心內怕他嫌髒,\begin{note}甲側:二人純用體貼功夫。\end{note}\begin{note}甲雙夾:將二人一併,真真寫他二人之心玲瓏七竅。\end{note}因笑道:“我瞧瞧燙了那裏了,有什麼遮著藏著的。”一面說,一面就湊上來,強搬著脖子瞧了一瞧,問他疼的怎麼樣。寶玉道:“也不很疼,養一兩日就好了。”林黛玉坐了一回,悶悶的回房去了。一宿無話。次日,寶玉見了賈母,雖然自己承認是自己燙的,不與別人相干,免不得那賈母又把跟從的人罵一頓。\begin{note}甲側:此原非正文,故草草寫去。\end{note}
\end{parag}


\begin{parag}
    過了一日,就有寶玉寄名的乾孃馬道婆進榮國府來請安。見了寶玉,唬一大跳,問起原由,說是燙的,便點頭嘆息一回,向寶玉臉上用指頭畫了一畫,口內嘟囔囔的又持誦了一回,說道:“管保就好了,這不過是一時飛災。”又向賈母道:“祖宗老菩薩那裏知道,那經典佛法上說的利害,\begin{note}甲側:一段無倫無理信口開河的混話,卻句句都是耳聞目睹者,並非杜撰而有。作者與餘實實經過。\end{note}大凡那王公卿相人家的子弟,只一生長下來,暗裏便有許多促狹鬼跟著他,得空便擰他一下,或掐他一下,或喫飯時打下他的飯碗來,或走著推他一跤,所以往往的那些大家子孫多有長不大的。”賈母聽如此說,便趕著問:“這有什麼佛法解釋沒有呢?”馬道婆道:“這個容易,只是替他多作些因果善事也就罷了。再那經上還說,西方有位大光明普照菩薩,專管照耀陰暗邪祟,若有善男子善女子虔心供奉者,可以永佑兒孫康寧安靜,再無驚恐邪祟撞客之災。”賈母道:“倒不知怎麼個供奉這位菩薩?”馬道婆道:“也不值些什麼,不過除香燭供養之外,一天多添幾斤香油,點上個大海燈。這海燈,便是菩薩現身法像,晝夜不敢息的。”賈母道:“ 一天一夜也得多少油?明白告訴我,我也好作這件功德的。”馬道婆聽如此說,便笑道:“這也不拘,隨施主菩薩們隨心願舍罷了。像我們廟裏,就有好幾處的王妃誥命供奉的:南安郡王府裏的太妃,他許的多,願心大,一天是四十八斤油,一斤燈草,\begin{note}甲側:賊婆先用大鋪排試之。\end{note}那海燈也只比缸略小些;錦田侯的誥命次一等,一天不過二十四斤油;再還有幾家也有五斤的、三斤的、一斤的,都不拘數。那小家子窮人家舍不起這些,就是四兩半斤,也少不得替他點。”賈母聽了,點頭思忖。\begin{note}甲眉:“點頭思忖”是量事之大小,非吝嗇也。日費香油四十八斤,每月油二百五十餘斤,合錢三百餘串。爲一小兒,如何服衆?太君細心若是。\end{note}馬道婆又道:“還有一件,若是爲父母尊親長上的,多舍些不妨;若是象老祖宗如今爲寶玉,若舍多了倒不好,\begin{note}甲側:賊道婆!是自“太君思忖”上來,後用如此數語收之,使太君必心悅誠服願行。賊婆,賊婆,費我作者許多心機摹寫也。\end{note}還怕哥兒禁不起,倒折了福。也不當家花花的,要舍,大則七斤,小則五斤,也就是了。”賈母說:“既是這樣說,你便一日五斤合準了,每月打躉來關了去。”馬道婆唸了一聲“阿彌陀佛慈悲大菩薩”。賈母又命人來吩咐:“以後大凡寶玉出門的日子,拿幾串錢交給他的小子們帶著,遇見僧道窮苦人好舍。”說畢,那馬道婆又坐了一回,便又往各院各房問安,閒逛了一回。一時來至趙姨娘房內,\begin{note}甲側:有“各院各房”,接此方不覺突然。\end{note}二人見過,趙姨娘命小丫頭倒了茶來與他喫。
\end{parag}


\begin{parag}
    馬道婆因見炕上堆著些零碎綢緞灣角,趙姨娘正粘鞋呢。馬道婆道:“可是我正沒了鞋面子了。\begin{note}甲側:見者有分是也。\end{note}趙奶奶你有零碎緞子,不拘什麼顏色的,弄一雙鞋面給我。”趙姨娘聽說,便嘆口氣說道:“你瞧瞧那裏頭,還有那一塊是成樣的?成了樣的東西,也不能到我手裏來!有的沒的都在這裏,你不嫌,就挑兩塊子去。”馬道婆見說,果真便挑了兩塊袖將起來。
\end{parag}


\begin{parag}
    趙姨娘問道:“前日我送了五百錢去,在藥王跟前上供,你可收了沒有?”馬道婆道:“早已替你上了供了。”趙姨娘嘆口氣道:“阿彌陀佛!我手裏但凡從容些,也時常的上個供,只是心有餘力量不足。”馬道婆道:“你只管放心,將來熬的環哥兒大了,得個一官半職,那時你要作多大的功德不能?”趙姨娘聽說,鼻子裏笑了一聲,說道:“罷,罷,再別說起。如今就是個樣兒,我們娘兒們跟的上這屋裏那一個兒!也不是有了寶玉,竟是得了活龍。他還是小孩子家,長的得人意兒,大人偏疼他些也還罷了;\begin{note}甲側:趙嫗數語,可知玉兄之身份,況在背後之言。\end{note}我只不伏這個主兒。”\begin{note}甲側:活現趙嫗。\end{note}一面說,一面伸出兩個指頭兒來。\begin{note}甲側:活現阿鳳。\end{note}馬道婆會意,便問道:“可是璉二奶奶?”趙姨娘唬的忙搖手兒,走到門前,掀簾子向外看看無人,\begin{note}甲側:是心膽俱怕破。\end{note}方進來向馬道婆悄悄說道:“了不得,了不得!提起這個主兒,這一分傢俬要不都叫他搬送到孃家去,我也不是個人。”\begin{note}庚側:這是妒心正題目。\end{note}
\end{parag}


\begin{parag}
    馬道婆見他如此說,便探他口氣說道:\begin{note}庚側:有隙即入,所謂賊婆,是極!\end{note}“我還用你說,難道都看不出來。也虧你們心裏也不理論,只憑他去。倒也妙。”趙姨娘道:“我的娘,不憑他去,難道誰還敢把他怎麼樣呢?”馬道婆聽說,鼻子裏一笑,\begin{note}庚側:二笑。\end{note}半晌說道:“不是我說句造孽的話,你們沒有本事!──也難怪別人。明不敢怎樣,暗裏也就算計了,\begin{note}甲側:賊婆操必勝之券,趙嫗已墮術中,故敢直出明言。可畏可怕!\end{note}還等到這如今!”趙姨娘聞聽這話裏有道理,心內暗暗的歡喜,便說道:“怎麼暗裏算計?我倒有這個意思,只是沒這樣的能幹人。你若教給我這法子,我大大的謝你。”馬道婆聽說這話打攏了一處,便又故意說道:“阿彌陀佛!你快休問我,我那裏知道這些事。罪過,罪過。”\begin{note}甲側:遠一步卻是近一步。賊婆,賊婆!\end{note}趙姨娘道:“你又來了。你是最肯濟困扶危的人,難道就眼睜睜的看人家來擺佈死了我們孃兒兩個不成?難道還怕我不謝你?”馬道婆聽說如此,便笑道:“若說我不忍叫你娘兒們受人委曲還猶可,若說謝我的這兩個字,可是你錯打算盤了。就便是我希圖你謝,靠你有些什麼東西能打動我?”\begin{note}甲側:探謝禮大小是如此說法,可怕可畏!\end{note}趙姨娘聽這話口氣鬆動了,便說道:“你這麼個明白人,怎麼糊塗起來了。你若果然法子靈驗,把他兩個絕了,明日這傢俬不怕不是我環兒的。那時你要什麼不得?” 馬道婆聽了,低了頭,半晌說道:“那時候事情妥了,又無憑據,你還理我呢!”趙姨娘道:“這又何難。如今我雖手裏沒什麼,也零碎攢了幾兩梯己,還有幾件衣服簪子,你先拿些去。下剩的,我寫個欠銀子文契給你,你要什麼保人也有,那時我照數給你。”馬道婆道:“果然這樣?”趙姨娘道:“這如何還撒得謊。”說著便叫過一個心腹婆子來,耳根底下嘁嘁喳喳喳說了幾句話。\begin{note}甲側:所謂狐羣狗黨大家難免,看官著眼。\end{note}那婆子出去了,一時回來,果然寫了個五百兩欠契來。趙姨娘便印了手模,\begin{note}甲側:癡婦,癡婦!\end{note}走到櫥櫃裏將梯己拿了出來,與馬道婆看看,道:“這個你先拿了去做香燭供奉使費,可好不好?”馬道婆看看白花花的一堆銀子,又有欠契,並不顧青紅皁白,\begin{note}甲側:有道婆作乾孃者來看此句。“並不顧”三字怕殺人。千萬件惡事皆從三字生出來。可怕可畏可警,可長存戒之。\end{note}滿口裏應著,伸手先去抓了銀子掖起來,然後收了欠契。又向褲腰裏掏了半晌,掏出十個紙鉸的青面白髮的鬼來,並兩個紙人,\begin{note}甲側:如此現成,更可怕。庚側:如此現成,想賊婆所害之人豈止寶玉、阿鳳二人哉?大家太君夫人誡之慎之。\end{note}遞與趙姨娘,又悄悄的教他道:“把他兩個的年庚八字寫在這兩個紙人身上,一併五個鬼都掖在他們各人的牀上就完了。我只在家裏作法,自有效驗。千萬小心,不要害怕!”\begin{note}甲眉:寶玉乃賊婆之寄名乾兒,一樣下此毒手,況阿鳳乎?三姑六婆之害如此,即賈母之神明,在所不免。其他只知喫齋唸佛之夫人太君,豈能防範的來?此係老太君一大病。作者一片婆心,不避嫌疑,特爲寫出,使看官再四著眼,吾家兒孫慎之戒之!\end{note}正才說著,只見王夫人的丫鬟進來找道:“奶奶可在這裏,太太等你呢。”二人方散了,不在話下。
\end{parag}


\begin{parag}
    卻說林黛玉因見寶玉近日燙了臉,總不出門,倒時常在一處說說話兒。這日飯後看了兩篇書,自覺無趣,便同紫鵑雪雁做了一回針線,更覺煩悶。便倚著房門出了一回神,\begin{note}甲側:所謂“閒倚繡房吹柳絮”是也。\end{note}信步出來,看階下新迸出的稚筍,\begin{note}甲側:妙妙!“筍根稚子無人見”,今得顰兒一見,何幸如之。\end{note}
\end{parag}


\begin{parag}
    不覺出了院門。一望園中,四顧無人,\begin{note}甲側:恐冷落圓亭花柳,故有是十數字也。\end{note}惟見花光柳影,鳥語溪聲。\begin{note}甲側:純用畫家筆寫。\end{note}林黛玉信步便往怡紅院中來,只見幾個丫頭舀水,都在迴廊上圍著看畫眉洗澡呢。\begin{note}甲側:閨中女兒樂事。\end{note}聽見房內有笑聲,林黛玉便入房中看時,原來是李宮裁、鳳姐、寶釵都在這裏呢,一見他進來都笑道:“這不又來了一個。”林黛玉笑道:“今兒齊全,誰下帖子請來的?”鳳姐道:“前兒我打發了丫頭送了兩瓶茶葉去,\begin{note}庚側:有照應。\end{note}你往那去了?”林黛玉笑道:“哦,可是倒忘了,\begin{note}甲側:該雲“我正看《會真記》呢”。一笑。\end{note}多謝多謝。”鳳姐兒又道:“你嚐了可還好不好?”沒有說完,寶玉便說道:“論理可倒罷了,只是我說不大甚好,也不知別人嘗著怎麼樣。”寶釵道:“味倒輕,只是顏色不大好些。”\begin{note}庚眉:二寶答言是補出諸豔俱領過之文。乙酉冬,雪窗。畸笏老人。\end{note}鳳姐道:“那是暹羅進貢來的。我嘗著也沒什麼趣兒,還不如我每日喫的呢。”林黛玉道:“我喫著好,\begin{note}甲側:卿愛因味輕也。卿如何擔的起味厚之物耶?\end{note}不知你們的脾胃是怎樣?”寶玉道:“你果然愛喫,把我這個也拿了去喫罷。”鳳姐笑道:“你要愛喫,我那裏還有呢。”林黛玉道:“果真的,我就打發丫頭取去了。”鳳姐道:“不用取去,我打發人送來就是了。我明兒還有一件事求你,一同打發人送來。”林黛玉聽了笑道:“你們聽聽,這是吃了他們家一點子茶葉,就來使喚人了。”鳳姐笑道:“倒求你,你倒說這些閒話,喫茶喫水的。你既吃了我們家的茶,怎麼還不給我們家作媳婦?”\begin{note}甲側:二玉事在賈府上下諸人即看書人批書人皆信定一段好夫妻,書中常常每每道及,豈具不然,嘆嘆!\end{note}\begin{note}庚側:二玉之配偶在賈府上下諸人即觀者批者作者皆爲無疑,故常常有此等點題語。我也要笑。\end{note}衆人聽了一齊都笑起來。
\end{parag}


\begin{parag}
    林黛玉紅了臉,一聲兒不言語,便回過頭去了。李宮裁笑向寶釵道:“真真我們二嬸子的詼諧是好的。”\begin{note}庚側:好贊!該他贊。\end{note}林黛玉道:“什麼詼諧,不過是貧嘴賤舌討人厭惡罷了。”\begin{note}甲側:此句還要候查。\end{note}說著便啐了一口。
\end{parag}


\begin{parag}
    鳳姐笑道:“你別作夢!你給我們家作了媳婦,少什麼?”指寶玉道:“你瞧瞧,人物兒、門第配不上,\begin{note}甲側:大大一泄,好接後文。\end{note}根基配不上,傢俬配不上?那一點還玷辱了誰呢?”林黛玉抬身就走。寶釵便叫:“顰兒急了,還不回來坐著。走了倒沒意思。”說著便站起來拉住。剛至房門前,只見趙姨娘和周姨娘兩個人進來瞧寶玉。李宮裁、寶釵、寶玉等都讓他兩個坐。獨鳳姐只和林黛玉說笑,正眼也不看他們。寶釵方欲說話時,只見王夫人房內的丫頭來說:“舅太太來了,請奶奶姑娘們出去呢。”李宮裁聽了,連忙叫著鳳姐等走了。趙、周兩個忙辭了寶玉出去。寶玉道:“我也不能出去,你們好歹別叫舅母進來。”又道:“林妹妹,你先略站一站,我說一句話。”鳳姐聽了,回頭向林黛玉笑道:“有人叫你說話呢。”說著便把林黛玉往裏一推,和李紈一同去了。
\end{parag}


\begin{parag}
    這裏寶玉拉著林黛玉的袖子,只是嘻嘻的笑,\begin{note}庚側:此刻好看之至!\end{note}心裏有話,只是口裏說不出來。\begin{note}甲側:是已受鎮,“說不出來”。勿得錯會了意。\end{note}此時林黛玉只是禁不住把臉紅漲了,掙著要走。寶玉忽然“噯喲”了一聲,說:“好頭疼!”\begin{note}甲側:自黛玉看書起分三段寫來,真無容針之空。如夏日烏雲四起,疾閃長雷不絕,不知雨落何時,忽然霹靂一聲,傾盆大注,何快如之,何樂如之,其令人寧不叫絕!\end{note}林黛玉道:“該,阿彌陀佛!”\begin{note}庚眉:黛玉唸佛,是喫茶之語在心故也。然摹寫神妙,一絲不漏如此。己冬夜。\end{note}只見寶玉大叫一聲:“我要死!”將身一縱,離地跳有三四尺高,口內亂嚷亂叫,說起胡話來了。林黛玉並丫頭們都唬慌了,忙去報知王夫人、賈母等。此時王子騰的夫人也在這裏,都一齊來時,寶玉益發拿刀弄杖,尋死覓活的,鬧得天翻地覆。賈母、王夫人見了,唬的抖衣而顫,且“兒”一聲“肉”一聲放聲慟哭。於是驚動諸人,連賈赦、邢夫人、賈珍、賈政、賈璉、賈蓉、賈芸、賈萍、薛姨媽、薛蟠並周瑞家的一干家中上上下下里裏外外衆媳婦丫頭等,都來園內看視。登時園內亂麻一般。\begin{note}甲側:寫玉兄驚動若許人忙亂,正寫太君一人之鐘愛耳。看官勿被作者瞞過。\end{note}正沒個主見,只見鳳姐手持一把明晃晃剛刀砍進園來,見雞殺雞,見狗殺狗,見人就要殺人。\begin{note}甲雙夾:此處焉用雞犬?然輝煌富麗非處家之常也,雞犬閒閒始爲兒孫千年之業,故於此處必用雞犬二字,方時一簇騰騰大舍。\end{note}衆人越發慌了。周瑞媳婦忙帶著幾個有力量的膽壯的婆娘上去抱住,奪下刀來,擡回房去。平兒、豐兒等哭的淚天淚地。賈政等心中也有些煩難,顧了這裏,丟不下那裏。
\end{parag}


\begin{parag}
    別人慌張自不必講,獨有薛蟠更比諸人忙到十分去:\begin{note}甲側:寫呆兄忙是愈覺忙中之愈忙,且避正文之絮煩。好筆仗,寫得出。\end{note}\begin{note}庚側:寫呆兄是躲煩碎文字法。好想頭,好筆力。《石頭記》最得力處在此。\end{note}又恐薛姨媽被人擠倒,又恐薛寶釵被人瞧見,又恐香菱被人臊皮──知道賈珍等是在女人身上做功夫的,\begin{note}甲側:從阿呆兄意中,又寫賈珍一筆,妙!\end{note}因此忙的不堪。忽一眼瞥見了林黛玉風流婉轉,已酥倒在那裏。\begin{note}甲側:忙到容針不能。此似唐突顰兒,卻是寫情字萬不能禁止者,又可知顰兒之丰神若仙子也。\end{note}\begin{note}甲雙夾:忙中寫閒,真大手眼,大章法。\end{note}
\end{parag}


\begin{parag}
    當下衆人七言八語,有的說請端公送祟的,有的說請巫婆跳神的,有的又薦玉皇閣的張真人,種種喧騰不一。也曾百般醫治祈禱,問卜求神,總無效驗。堪堪日落。王子騰夫人告辭去後,次日王子騰也來瞧問。\begin{note}甲側:寫外戚,亦避正文之繁。\end{note}接著小史侯家、邢夫人弟兄輩並各親戚眷屬都來瞧看,也有送符水的,也有薦僧道的,總不見效。他叔嫂二人愈發糊塗,不省人事,睡在牀上,渾身火炭一般,口內無般不說。到夜晚間,那些婆娘媳婦丫頭們都不敢上前。因此把他二人都抬到王夫人的上房內,\begin{note}甲側:收拾得乾淨有著落。庚側:收拾得得體正大。\end{note}夜間派了賈芸帶著小廝們挨次輪班看守。賈母、王夫人、邢夫人、薛姨媽等寸地不離,只圍著乾哭。
\end{parag}


\begin{parag}
    此時賈赦、賈政又恐哭壞了賈母,日夜熬油費火,鬧的人口不安,也都沒了主意。賈赦還各處去尋僧覓道。賈政見不靈效,著實懊惱,\begin{note}甲側:四字寫盡政老矣。\end{note}因阻賈赦道:“兒女之數,皆由天命,非人力可強者。他二人之病出於不意,百般醫治不效,想天意該當如此,也只好由他們去罷。”\begin{note}甲側:唸書人自應如是語。\end{note}賈赦也不理此話,仍是百般忙亂,那裏見些效驗。看看三日光陰,那鳳姐和寶玉躺在牀上,亦發連氣都將沒了。閤家人口無不驚慌,都說沒了指望,忙著將他二人的後世的衣履都治備下了。賈母、王夫人、賈璉、平兒、襲人這幾個人更比諸人哭的忘餐廢寢,覓死尋活。趙姨娘、賈環等自是稱願。\begin{note}甲側:補明趙嫗進怡紅爲作法也。\end{note}
\end{parag}


\begin{parag}
    到了第四日早晨,賈母等正圍著寶玉哭時,只見寶玉睜開眼說道:\begin{note}甲側:“語不驚人死不休”,此之謂也。\end{note}“從今以後,我可不在你家了!快收拾了,打發我走罷。”賈母聽了這話,如同摘心去肝一般。趙姨娘在旁勸道:“老太太也不必過於悲痛。\begin{note}庚側:斷不可少此句。\end{note}哥兒已是不中用了,不如把哥兒的衣服穿好,讓他早些回去,也免些苦;只管捨不得他,這口氣不斷,他在那世裏也受罪不安生。”\begin{note}庚側:大遂心人必有是語。\end{note}這些話沒說完,被賈母照臉啐了一口唾沫,罵道:“爛了舌頭的混帳老婆,誰叫你來多嘴多舌的!你怎麼知道他在那世裏受罪不安生?怎麼見得不中用了?你願他死了,有什麼好處?你別做夢!他死了,我只和你們要命。素日都不是你們調唆著逼他寫字唸書,\begin{note}甲雙夾:奇語,所謂溺愛者不明,然天生必有是一段文字的。\end{note}把膽子唬破了,見了他老子不象個避貓鼠兒?都不是你們這起淫婦調唆的!這會子逼死了,你們遂了心,我饒那一個!”一面罵,一面哭。賈政在旁聽見這些話,心裏越發難過,便喝退趙姨娘,自己上來委婉解勸。一時又有人來回說:“兩口棺槨都做齊了,\begin{note}甲側:偏寫一頭不了又一頭之文,真步步緊之文。\end{note}請老爺出去看。”賈母聽了,如火上澆油一般,便罵:“是誰做了棺槨?”一疊聲只叫把做棺槨的拉來打死。
\end{parag}


\begin{parag}
    正鬧的天翻地覆,沒個開交,只聞得隱隱的木魚聲響,\begin{note}甲側:不費絲毫勉強,輕輕收住數百言文字,《石頭記》得力處全在此處。以幻作真,以真作幻,看書人亦要如是看法爲幸。\end{note}唸了一句:“南無解冤孽菩薩。有那人口不利,家宅顛傾,或逢兇險,或中邪祟者,我們善能醫治。”賈母、王夫人聽見這些話,那裏還耐得住,便命人去快請進來。賈政雖不自在,奈賈母之言如何違拗;想如此深宅,何得聽的這樣真切,\begin{note}甲側:作者是幻筆,合屋俱是幻耳,焉能無聞?\end{note}心中亦希罕,\begin{note}甲側:政老亦落幻中。\end{note}命人請了進來。衆人舉目看時,原來是一個癩頭和尚與一個跛足道人。\begin{note}甲雙夾:僧因鳳姐,道因寶玉,一絲不亂。\end{note}
\end{parag}


\begin{parag}
    見那和尚是怎的模樣:
\end{parag}


\begin{poem}
    \begin{pl}鼻如懸膽兩眉長,目似明星蓄寶光,\end{pl}

    \begin{pl}破衲芒鞋無住跡,腌臢更有滿頭瘡。\end{pl}
\end{poem}


\begin{parag}
    那道人又是怎生模樣:
\end{parag}


\begin{poem}
    \begin{pl}一足高來一足低,渾身帶水又拖泥。\end{pl}

    \begin{pl}相逢若問家何處,卻在蓬萊弱水西。\end{pl}
\end{poem}


\begin{parag}
    賈政問道:“你道友二人在那廟裏焚修。”那僧笑道:“長官不須多話。\begin{note}甲側:避俗套法。\end{note}因聞得府上人口不利,故特來醫治。”賈政道:“倒有兩個人中邪,不知你們有何符水?”那道人笑道:“你家現有希世奇珍,如何還問我們有符水?”賈政聽這話有意思,心中便動了,因說道:“小兒落草時雖帶了一塊寶玉下來,上面說能除邪祟,\begin{note}庚側:點題。\end{note}誰知竟不靈驗。”那僧道:“長官你那裏知道那物的妙用。只因他如今被聲色貨利所迷,\begin{note}甲雙夾:石皆能迷,可知其害不小。觀者著眼,方可讀《石頭記》。\end{note}故不靈驗了。\begin{note}甲側:讀書者觀之。\end{note}你今且取他出來,待我們持誦持誦,只怕就好了。”\begin{note}庚側: “只怕”二字,是不知此石肯聽持誦否?\end{note}
\end{parag}


\begin{parag}
    賈政聽說,便向寶玉項上取下那玉來遞與他二人。那和尚接了過來,擎在掌上,長嘆一聲道:“青埂峯一別,展眼已過十三載矣!\begin{note}庚側:正點題,大荒山手捧時語。\end{note}人世光陰,如此迅速,塵緣滿日,若似彈指!\begin{note}甲雙夾:見此一句,令人可嘆可驚,不忍往後再看矣!\end{note}可羨你當時的那段好處:
\end{parag}


\begin{poem}
    \begin{pl}   天不拘兮地不羈,心頭無喜亦無悲;\end{pl}
    \begin{note}甲側:所謂越不聰明越快活。\end{note}

    \begin{pl}   卻因鍛鍊通靈後,便向人間覓是非。\end{pl}
\end{poem}


\begin{parag}
    可嘆你今日這番經歷:
\end{parag}


\begin{poem}
    \begin{pl} 粉漬脂痕污寶光,綺櫳晝夜困鴛鴦。\end{pl}

    \begin{pl} 沉酣一夢終須醒,\end{pl}\begin{note}甲側:無百年的筵席。\end{note}\begin{pl}冤孽償清好散場!”\end{pl}\begin{note}甲側:三次鍛鍊,焉得不成佛作祖?\end{note}
\end{poem}


\begin{parag}
    念畢,又摩弄一回,說了些瘋話,遞與賈政道:“此物已靈,不可褻瀆,懸於臥室上檻,將他二人安在一室之內,除親身妻母外,不可使外人衝犯。\begin{note}庚側:是要緊語,是不可不寫之套語。\end{note}三十三日之後,包管身安病退,復舊如初。”說著回頭便走了。\begin{note}庚眉:通靈玉除邪,全部百回只此一見,何得再言?僧道蹤跡虛實,幻筆幻想,寫幻人於幻文也。壬午孟夏,雨窗。\end{note}賈政趕著還說話,讓二人坐了喫茶,要送謝禮,他二人早已出去了。賈母等還只管著人去趕,那裏有個蹤影。少不得依言將他二人就安放在王夫人臥室之內,將玉懸在門上。王夫人親身守著,不許別個人進來。
\end{parag}


\begin{parag}
    至晚間他二人竟漸漸醒來,\begin{note}甲側:能領持誦,故如此靈效。\end{note}說腹中飢餓。賈母、王夫人如得了珍寶一般,\begin{note}甲側:昊天罔極之恩如何報得?哭殺幼而喪親者。\end{note}旋熬了米湯與他二人吃了,精神漸長,邪祟稍退,一家子才把心放下來。\begin{note}甲眉:通靈玉聽癩和尚二偈即刻靈應,抵卻前回若干《莊子》及語錄機鋒偈子。正所謂物各有所主也。嘆不得見玉兄“懸崖撒手”文字爲恨。\end{note}李宮裁併賈府三豔、薛寶釵、林黛玉、平兒、襲人等在外間聽信息。聞得吃了米湯,省了人事,別人未開口,林黛玉先就唸了一聲“阿彌陀佛”。\begin{note}甲側:針對得病時那一聲。\end{note}薛寶釵便回頭看了他半日,嗤的一聲笑。衆人都不會意,賈惜春道:“寶姐姐,好好的笑什麼?”寶釵笑道:“我笑如來佛比人還忙:\begin{note}庚側:這一句作正意看,餘皆雅謔,但此一 實 顰兒半部之謔。\end{note}又要講經說法,又要普渡衆生;這如今寶玉,鳳姐姐病了,又燒香還願,賜福消災;今纔好些,又管林姑娘的姻緣了。你說忙的可笑不可笑。”林黛玉不覺的紅了臉,啐了一口道:“你們這起人不是好人,不知怎麼死!再不跟著好人學,只跟那些貧嘴惡舌的人學。”一面說,一面摔簾子出去了。不知端詳,且聽下回分解。
\end{parag}


\begin{parag}
    \begin{note}甲:先寫紅玉數行引接正文,是不作開門見山文字。\end{note}
\end{parag}


\begin{parag}
    \begin{note}甲:燈油引大光明普照菩薩,大光明普照菩薩引五鬼魘魔法是一線貫成。\end{note}
\end{parag}


\begin{parag}
    \begin{note}甲:通靈玉除邪,全部只此一見,卻又不靈,遇癩和尚、跛道人一點方靈應矣。寫利慾之害如此。\end{note}
\end{parag}


\begin{parag}
    \begin{note}甲:此回本意是爲禁三姑六婆進門之害,難以防範。\end{note}
\end{parag}


\begin{parag}
    \begin{note}庚:此回書因才幹乖覺太露,引出事來,作者婆心爲世之乖覺人爲鑑。\end{note}
\end{parag}


\begin{parag}
    \begin{note}蒙回末總評:欲深魔重複可疑,苦海冤河解者誰?結不休時冤日盛,井天甚小性難移。\end{note}
\end{parag}

