\chap{二十一}{賢襲人嬌嗔箴寶玉 俏平兒軟語救賈璉}


\begin{parag}
    賢襲人\begin{note}庚側:當得起\end{note}嬌嗔箴寶玉 俏平兒軟語救賈璉
\end{parag}

\begin{parag}
    \begin{note}庚:有客題《紅樓夢》一律,失其姓氏,惟見其詩意駭警,故錄於斯:“自執金矛又執戈,自相戕戮自張羅。茜紗公子情無限,脂硯先生恨幾多。是幻是真空歷遍,閒風閒月枉吟哦。情機轉得情天破,情不情兮奈我何?”凡是書題者不少,此爲絕調。詩句警拔,且深知擬書底裏,惜乎失名矣!\end{note}
\end{parag}


\begin{parag}
    \begin{note}蒙回前批:按此回之文固妙,然未見後三十回猶不見此之妙。此回“嬌嗔箴寶玉”、“軟語救賈璉”,後文“薛寶釵藉詞含諷諫,王熙鳳知命強英雄”。今只從二婢說起,後則直指其主。然今日之襲人、之寶玉,亦他日之襲人、他日之寶玉也。今日之平兒、之賈璉,亦他日之平兒、他日之賈璉也。何今日之玉猶可箴,他日之玉已不可箴耶?今日之璉猶可救,他日之璉已不能救耶?箴與諫無異也,而襲人安在哉?寧不悲乎!救與強無別也,甚矣!但此日阿鳳英氣何如是也,他日之身微運蹇,亦何如是也?人世之變遷,倏忽如此!\end{note}
\end{parag}


\begin{parag}
    \begin{note}蒙回前批:今日寫襲人,後文寫寶釵;今日寫平兒,後文寫阿鳳。文是一樣情理,景況光陰,事卻天壤矣!多少恨淚灑出此兩回書。\end{note}
\end{parag}


\begin{parag}
    \begin{note}蒙回前批:此回襲人三大功,直與寶玉一生三大病映射。\end{note}
\end{parag}


\begin{parag}
    話說史湘雲跑了出來,怕林黛玉趕上,寶玉在後忙說:“仔細絆跌了!那裏就趕上了?”林黛玉趕到門前,被寶玉叉手在門框上攔住,笑勸道:“饒他這一遭罷。”林黛玉搬著手說道:“我若饒過雲兒,再不活著!”湘雲見寶玉攔住門,料黛玉不能出來,\begin{note}庚雙夾:寫得湘雲與寶玉又親厚之極,卻不見疏遠黛玉,是何情思耶?\end{note}便立住腳笑道:“好姐姐,饒我這一遭罷。”恰值寶釵來在湘雲身後,也笑道:“我勸你兩個看寶兄弟分上,都丟開手罷。”\begin{note}庚雙夾:好極,妙極!玉、顰、雲三人已難解難分,插入寶釵雲“我勸你兩個看寶玉兄弟分上”,話只一句,便將四人一齊籠住,不知孰遠孰近,孰親孰疏,真好文字!\end{note}黛玉道:“我不依。你們是一氣的,都戲弄我不成!”\begin{note}庚雙夾:話是顰兒口吻,雖屬尖利,真實堪愛堪憐。\end{note}寶玉勸道:“誰敢打趣你!你不打趣他,他焉敢說你?”\begin{note}庚雙夾:好!二“你”字連二“他”字,華灼之至!\end{note}四人正難分解,\begin{note}庚雙夾:好!前三人,今忽四人,俱是書中正眼,不可少矣。\end{note}有人來請喫飯,方往前邊來。\begin{note}庚雙夾:好文章!正是閨中女兒口角之事。若只管諄諄不已,則成何文矣!\end{note}
\end{parag}


\begin{parag}
    那天早又掌燈時分,王夫人、李紈、鳳姐、迎、探、惜等都往賈母這邊來,大家閒話了一回,各自歸寢。湘雲仍往黛玉房中安歇。\begin{note}庚雙夾:前文黛玉未來時,湘雲、寶玉則隨賈母。今湘雲已去,黛玉既來,年歲漸成,寶玉各自有房,黛玉亦各有房,故湘雲自應同黛玉一處也。\end{note}
\end{parag}


\begin{parag}
    寶玉送他二人到房,那天已二更多時,襲人來催了幾次,方回自己房中來睡。次日天明時,便披衣靸鞋往黛玉房中來,不見紫鵑、翠縷二人,只見他姊妹兩個尚臥在衾內。那林黛玉\begin{note}庚雙夾:寫黛玉身分。\end{note}嚴嚴密密裹著一幅杏子紅綾被,安穩合目而睡。\begin{note}庚雙夾:一個睡態。\end{note}那史湘雲卻一把青絲拖於枕畔,被只齊胸,一彎雪白的膀子撂於被外,又帶著兩個金鐲子。\begin{note}庚雙夾:又一個睡態。寫黛玉之睡態,儼然就是嬌弱女子,可憐。湘雲之態,則儼然是個嬌態女兒,可愛。真是人人俱盡,個個活跳,吾不知作者胸中埋伏多少裙釵。\end{note}寶玉見了,嘆道:\begin{note}庚雙夾:“嘆”字奇!除玉卿外,世人見之自曰喜也。\end{note}“睡覺還是不老實!回來風吹了,又嚷肩窩疼了。”一面說,一面輕輕的替他蓋上。林黛玉早已醒了,\begin{note}庚側:不醒不是黛玉了。\end{note}覺得有人,就猜著定是寶玉,因翻身一看,果中其料。因說道:“這早晚就跑過來作什麼?”寶玉笑道:“這天還早呢!你起來瞧瞧。”黛玉道:“你先出去,讓我們起來。”\begin{note}庚側:一絲不亂。\end{note}寶玉聽了,轉身出至外邊。
\end{parag}


\begin{parag}
    黛玉起來叫醒湘雲,二人都穿了衣服。寶玉復又進來,坐在鏡臺旁邊,只見紫鵑、雪雁進來伏侍梳洗。湘雲洗了面,翠縷便拿殘水要潑,寶玉道:“站著,我趁勢洗了就完了,省得又過去費事。”說著便走過來,彎腰洗了兩把。\begin{note}庚側:妙在兩把。\end{note}紫鵑遞過香皂去,寶玉道:“這盆裏的就不少,不用搓了。”再洗了兩把,便要手巾。\begin{note}庚側:在怡紅何其費事多多。\end{note}翠縷道:“還是這個毛病兒,多早晚才改。”\begin{note}庚側:冷眼人旁點,一絲不漏。\end{note}寶玉也不理,忙忙的要過青鹽擦了牙,嗽了口,完畢,見湘雲已梳完了頭,便走過來笑道:“好妹妹,替我梳上頭罷。”湘雲道:“這可不能了。”寶玉笑道:“好妹妹,你先時怎麼替我梳了呢?”湘雲道:“如今我忘了,\begin{note}庚眉:“忘了”二字在嬌憨。\end{note}怎麼梳呢?”寶玉道:“橫豎我不出門,又不帶冠子勒子,不過打幾根散辮子就完了。”說著,又千妹妹萬妹妹的央告。\begin{note}庚眉:口中自是應聲而出,捉筆人卻從何處設想而來,成此天然對答。壬午九月。\end{note}湘雲只得扶過他的頭來,一一梳篦。在家不戴冠,並不總角,只將四圍短髮編成小辮,往頂心發上歸了總,編一根大辮,紅絛結住。自發頂至辮梢,一路四顆珍珠,下面有金墜腳。湘雲一面編著,一面說道:“這珠子只三顆了,這一顆不是的。\begin{note}庚側:梳頭亦有文字,前已敘過,今將珠子一穿插,卻天生有是事。\end{note}我記得是一樣的,怎麼少了一顆?”寶玉道:“丟了一顆。”湘雲道:“必定是外頭去掉下來,不防被人揀了去,倒便宜他。”\begin{note}庚雙夾:妙談!道“到便宜他”四字,是大家千金口吻。近日多用 “可惜了的”四字。今失一珠,不聞此四字。妙極!是極!\end{note}\begin{note}庚眉:“到便宜他”四字與“忘了”二字是一氣而來,將一侯府千金白描矣。畸笏。\end{note}黛玉一旁盥手,冷笑道:\begin{note}庚側:純用畫家烘染法。\end{note}“也不知是真丟了,也不知是給了人鑲什麼戴去了!”寶玉不答,\begin{note}庚雙夾:有神理,有文章。\end{note}因鏡臺兩邊俱是妝奩等物,順手拿起來賞玩,\begin{note}庚雙夾:何賞玩也?寫來奇特。\end{note}不覺又順手拈了胭脂,意欲要往口邊送,\begin{note}庚雙夾:是襲人勸後余文。\end{note}因又怕史湘雲說。\begin{note}庚雙夾:好極!的是寶玉也。\end{note}正猶豫間,湘雲果在身後看見,一手掠著辮子,便伸手來“拍”的一下,從手中將胭脂打落,說道:“這不長進的毛病兒,多早晚才改過!”\begin{note}庚側:前翠縷之言並非白寫。\end{note}
\end{parag}


\begin{parag}
    一語未了,只見襲人進來,看見這般光景,知是梳洗過了,只得回來自己梳洗。忽見寶釵走來,因問道:“寶兄弟那去了?”襲人含笑道:“寶兄弟那裏還有在家的工夫!”寶釵聽說,心中明白。又聽襲人嘆道:“姊妹們和氣,也有個分寸禮節,也沒個黑家白日鬧的!憑人怎麼勸,都是耳旁風。”寶釵聽了,心中暗忖道: “倒別看錯了這個丫頭,聽他說話,倒有些識見。”\begin{note}庚雙夾:此是寶卿初試,已下漸成知已,蓋寶卿從此心察得襲人果賢女子也。\end{note}寶釵便在炕上坐了,\begin{note}庚雙夾:好!逐回細看,寶卿待人接物,不疏不親,不遠不近。可厭之人,亦未見冷淡之態,形諸聲色;可喜之人,亦未見醴密之情,形諸聲色。今日“便在炕上坐了”,蓋深取襲卿矣。二人文字,此回爲始。詳批於此,諸公請記之。\end{note}慢慢的閒言中套問他年紀家鄉等語,留神窺察,其言語志量深可敬愛。\begin{note}庚雙夾:四字包羅許多文章筆墨,不似近之開口便雲“非諸女子之可比者”,此句大壞。然襲人故佳矣,不書此句是大手眼。\end{note}
\end{parag}


\begin{parag}
    一時寶玉來了,寶釵方出去。\begin{note}庚雙夾:奇文!寫得釵、玉二人形景較諸人皆近,何也?寶玉之心,凡女子前不論貴賤,皆親密之至,豈於寶釵前反生遠心哉?蓋寶釵之行止端肅恭嚴,不可輕犯,寶玉欲近之,而恐一時有瀆,故不敢狎犯也。寶釵待下愚尚且和平親密,何反於兄弟前有遠心哉?蓋寶玉之形景已泥於閨閣,近之則恐不遜,反成遠離之端也。故二人之遠,實相近之至也。至顰兒於寶玉實近之至矣,卻遠之至也。不然,後文如何反較勝角口諸事皆出於顰哉?以及寶玉砸玉,顰兒之淚枯,種種孽障,種種憂忿,皆情之所陷,更何辯哉?此一回將寶玉、襲人、釵、顰、雲等行止大概一描,已啓後大觀園中文字也。今詳批於此,後久不忽矣。釵與玉遠中近,顰與玉近中遠,是要緊兩大股,不可粗心看過。\end{note}寶玉便問襲人道:“怎麼寶姐姐和你說的這麼熱鬧,見我進來就跑了?”\begin{note}庚側:此問必有。\end{note}問一聲不答,再問時,襲人方道:“你問我麼?我那裏知道你們的原故。”寶玉聽了這話,見他臉上氣色非往日可比,便笑道:“怎麼動了真氣?”\begin{note}庚雙夾:寶玉如此。\end{note}襲人冷笑道:“我那裏敢動氣!只是從今以後別再進這屋子了。橫豎有人伏侍你,再別來支使我。我仍舊還伏侍老太太去。”一面說,一面便在炕上閤眼倒下。\begin{note}蒙側:是醋?是諫?不敢擬定,似在可否之間!\end{note}\begin{note}蒙雙夾:醋妒妍憨假態,至矣盡矣!觀者但莫認真此態爲幸。\end{note}寶玉見了這般景況,深爲駭異,\begin{note}蒙雙夾:好!可知未嘗見襲人之如此技藝也!\end{note}禁不住趕來勸慰。那襲人只管合了眼不理。\begin{note}庚雙夾:與顰兒前番嬌態如何?愈覺可愛猶甚。\end{note}寶玉無了主意,因見麝月進來,\begin{note}庚雙夾:偏麝月來,好文章!\end{note}便問道:“你姐姐怎麼了?”\begin{note}庚雙夾:如見如聞。\end{note}麝月道:“我知道麼?問你自己便明白了。”\begin{note}庚雙夾:又好麝月!\end{note}寶玉聽說,呆了一回,自覺無趣,便起身嘆道:“不理我罷,我也睡去。”說著,便起身下炕,到自己牀上歪下。襲人聽他半日無動靜,微微的打鼾,\begin{note}庚側:真乎?詐乎?\end{note}料他睡著,便起身拿一領鬥蓬來,替他剛壓上,只聽“忽”的一聲,\begin{note}庚側:文是好文,唐突我襲卿,吾不忍也。\end{note}寶玉便掀過去,也仍合目裝睡。\begin{note}庚雙夾:寫得爛熳。\end{note}襲人明知其意,便點頭冷笑道:“你也不用生氣,從此後我只當啞子,再不說你一聲兒,如何?”寶玉禁不住起身問道:“我又怎麼了?你又勸我。你勸我也罷了,纔剛又沒見你勸我,一進來你就不理我,賭氣睡了。我還摸不著是爲什麼,這會子你又說我惱了。\begin{note}庚側:這是委屈了石兄。\end{note}我何嘗聽見你勸我什麼話了。”襲人道:“你心裏還不明白,還等我說呢!”\begin{note}庚側:亦是囫圇語,卻從有生以來肺腑中出,千斤重。\end{note}\begin{note}庚眉:《石頭記》每用囫圇語處,無不精絕奇絕,且總不覺相犯。壬午九月。畸笏。\end{note}
\end{parag}


\begin{parag}
    正鬧著,賈母遣人來叫他喫飯,方往前邊來,胡亂吃了半碗,仍回自己房中。只見襲人睡在外頭炕上,麝月在旁邊抹骨牌。寶玉素知麝月與襲人親厚,一併連麝月也不理,揭起軟簾自往裏間來。麝月只得跟進來。寶玉便推他出去,說:“不敢驚動你們。”麝月只得笑著出來,喚了兩個小丫頭進來。寶玉拿一本書,歪著看了半天,因要茶,抬頭只見兩個小丫頭在地下站著。一個大些兒的生得十分水秀,\begin{note}庚雙夾:二字奇絕!多少嬌態包括一盡。今古野史中無有此文也。\end{note}寶玉便問:“你叫什麼名字?”那丫頭便說:“叫蕙香。”\begin{note}庚雙夾:也好。\end{note}寶玉便問:“是誰起的?”蕙香道:“我原叫芸香的,\begin{note}庚雙夾:原俗。\end{note}是花大姐姐改了蕙香。”寶玉道:“正經該叫‘晦氣’罷了,什麼蕙香呢!”\begin{note}庚雙夾:好極!趣極!\end{note}又問:“你姊妹幾個?”蕙香道:“四個。”寶玉道: “你第幾?”蕙香道:“第四。”寶玉道:“明兒就叫‘四兒’,不必什麼‘蕙香’‘蘭氣’的。那一個配比這些花,沒的玷辱了好名好姓。”\begin{note}庚雙夾: “花襲人”三字在內,說的有趣。\end{note}一面說,一面命他倒了茶來喫。襲人和麝月在外間聽了抿嘴而笑。\begin{note}庚雙夾:一絲不漏,好精神!\end{note}
\end{parag}


\begin{parag}
    這一日,寶玉也不大出房,\begin{note}庚雙夾:此是襲卿第一功勞也。\end{note}也不和姊妹丫頭等廝鬧,\begin{note}庚雙夾:此是襲卿第二功勞也。\end{note}自己悶悶的,只不過拿著書解悶,或弄筆墨,\begin{note}庚雙夾:此雖未必成功,較往日終有微補小益,所謂襲卿有三大功勞也。\end{note}也不使喚衆人,只叫四兒答應。誰知四兒是個聰敏乖巧不過的丫頭,\begin{note}庚雙夾:又是一個有害無益者。作者一生爲此所誤,批者一生亦爲此所誤,於開卷凡見如此人,世人故爲喜,餘反抱恨,蓋四字誤人甚矣。被誤者深感此批。\end{note}見寶玉用他,他變盡方法籠絡寶玉。\begin{note}庚雙夾:他好,但不知襲卿之心思何如?\end{note}至晚飯後,寶玉因吃了兩杯酒,眼餳耳熱之際,若往日則有襲人等大家喜笑有興,今日卻冷清清的一人對燈,好沒興趣。待要趕了他們去,又怕他們得了意,以後越發來勸,\begin{note}庚雙夾:寶玉惡勸,此是第一大病也。\end{note}若拿出做上的規矩來鎮唬,似乎無情太甚。\begin{note}庚雙夾:寶玉重情不重禮,此是第二大病也。\end{note}說不得橫心只當他們死了,橫豎自然也要過的。便權當他們死了,毫無牽掛,反能怡然自悅。\begin{note}庚雙夾:此意卻好,但襲卿輩不應如此棄也。寶玉之情,今古無人可比,固矣。然寶玉有情極之毒,亦世人莫忍爲者,看至後半部則洞明矣。此是寶玉三大病也。寶玉有此世人莫忍爲之毒,故後文方有“懸崖撒手”一回。若他人得寶釵之妻、麝月之婢,豈能棄而爲僧哉?此寶玉一生偏僻處。\end{note}因命四兒剪燈烹茶,自己看一回《南華經》。正看至《外篇•胠篋》一則,其文曰:
\end{parag}


\begin{qute2sp}

    故絕聖棄知,大盜乃止,擿玉毀珠,小盜不起,焚符破璽,而民樸鄙,掊斗折衡,而民不爭,殫殘天下之聖法,而民始可與論議。擢亂六律,鑠絕竽瑟,塞瞽曠之耳,而天下始人含其聰矣;滅文章,散五采,膠離朱之目,而天下始人含其明矣,毀鉤繩而棄規矩,攦工倕之指,而天下始人有其巧矣。\begin{note}庚雙夾:此上語本《莊子》。\end{note}
\end{qute2sp}


\begin{parag}
    看至此,意趣洋洋,趁著酒興,不禁提筆續曰:\begin{note}蒙側:敢續!\end{note}\begin{note}庚眉:趁著酒興不禁而續,是作者自站地步處,謂餘何人耶,敢續《莊子》?然奇極怪極之筆,從何設想,怎不令人叫絕?己冬夜。\end{note}\begin{note}庚眉:這亦暗露玉兄閒窗淨幾、不寂不離之工業。壬午孟夏。\end{note}
\end{parag}


\begin{qute2sp}

    焚花散麝,而閨閣始人含其勸矣,戕寶釵之仙姿,灰黛玉之靈竅,喪減情意,而閨閣之美惡始相類矣。彼含其勸,則無參商之虞矣,戕其仙姿,無戀愛之心矣,灰其靈竅,無才思之情矣。彼釵、玉、花、麝者,皆張其羅而穴其隧,所以迷眩纏陷天下者也。\begin{note}庚雙夾:直似莊老,奇甚怪甚!庚眉:趙香梗先生《秋樹根偶譚》內兗州少陵臺有子美祠爲郡守毀爲已祠。先生嘆子美生遭喪亂,奔走無家,孰料千百年後數椽片瓦猶遭貪吏之毒手。甚矣,才人之厄也!因改公《茅屋爲秋風所破歌》數句,爲少陵解嘲:“少陵遺像太守欺無力,忍能對面爲盜賊,公然折克非已祠,旁人有口呼不得,夢歸來兮聞嘆息,白日無光天地黑。安得曠宅千萬間,太守取之不盡生歡顏,公祠免毀安如山。”讀之令人感慨悲憤,心常耿耿。壬午九月。因索書甚迫,姑志於此,非批《石頭記》也。爲續《莊子因》數句,真是打破胭脂陣,坐透紅粉關,另開生面之文,無可評處。\end{note}
\end{qute2sp}


\begin{parag}
    續畢,擲筆就寢。頭剛著枕便忽睡去,一夜竟不知所之,直至天明方醒。\begin{note}庚雙夾:此猶是襲人餘功也。想每日每夜,寶玉自是心忙身忙口忙之極,今則怡然自適。雖此一刻,於身心無所補益,能有一時之閒閒自若,亦豈非襲卿之所使然耶?\end{note}翻身看時,只見襲人和衣睡在衾上。\begin{note}庚雙夾:神極之筆!試思襲人不來同臥亦不成文字,來同臥更不成文字。卻雲“和衣衾上”,正是來同臥不來同臥之間。何神奇文妙絕矣!好襲人!真好石頭記得真,真好述者述得不錯,真好批者批得出。\end{note}寶玉將昨日的事已付與度外,\begin{note}蒙雙夾:更好!可見玉卿的是天真爛漫之人也!近之所謂□公子又曰“老好人”、“無心道人”是也!殊不知尚古淳風。\end{note}便推他說道:“起來好生睡,看凍著了。”
\end{parag}


\begin{parag}
    原來襲人見他無曉夜和姊妹們廝鬧,若直勸他,料不能改,故用柔情以警之,料他不過半日片刻仍復好了。不想寶玉一日夜竟不迴轉,自己反不得主意,直一夜沒好生睡得。今忽見寶玉如此,料他心意回轉,便越性不睬他。寶玉見他不應,便伸手替他解衣,剛解開了鈕子,被襲人將手推開,\begin{note}庚側:好看煞!\end{note}又自扣了。寶玉無法,只得拉他的手笑道:“你到底怎麼了?”連問幾聲,襲人睜眼說道:“我也不怎麼。你睡醒了,你自過那邊房裏去梳洗,再遲了就趕不上。”\begin{note}庚雙夾:說得好痛快。\end{note}寶玉道:“我過那裏去?”\begin{note}庚雙夾:問得更好。\end{note}襲人冷笑道:“你問我,\begin{note}庚側:三字如聞。\end{note}我知道?你愛往那裏去,就往那裏去。從今咱們兩個丟開手,省得雞聲鵝鬥,叫別人笑。橫豎那邊膩了過來,這邊又有個什麼‘四兒’‘五兒’伏侍。我們這起東西,可是‘白玷辱了好名好姓’的。”寶玉笑道:“你今兒還記著呢!”\begin{note}庚雙夾:非渾一純粹,那能至此!\end{note}襲人道:“一百年還記著呢!比不得你,拿著我的話當耳旁風,夜裏說了,早起就忘了。”\begin{note}庚雙夾:這方是正文,直勾起“花解語”一回文字。\end{note}寶玉見他嬌嗔滿面,情不可禁,\begin{note}庚側:又用幻筆瞞過看官。\end{note}便向枕邊拿起一根玉簪來,一跌兩段,說道:“我再不聽你說,就同這個一樣。”\begin{note}蒙側:迎頭一棒!\end{note}襲人忙的拾了簪子,說道:“大清早起,這是何苦來!聽不聽什麼要緊,\begin{note}庚側:已留後文地步。\end{note}也值得這種樣子。”寶玉道:“你那裏知道我心裏急!”襲人笑道:\begin{note}庚雙夾:自此方笑。\end{note}“你也知道著急麼!可知我心裏怎麼著?快起來洗臉去罷。”\begin{note}庚側:結得一星渣滓全無,且合怡紅常事。\end{note}說著,二人方起來梳洗。
\end{parag}


\begin{parag}
    寶玉往上房去後,誰知黛玉走來,見寶玉不在房中,因翻弄案上書看,可巧翻出昨兒的《莊子》來。看至所續之處,不覺又氣又笑,不禁也提筆續書一絕雲:
\end{parag}


\begin{poem}
    \begin{pl}無端弄筆是何人?作踐南華《莊子因》。\end{pl}

    \begin{pl}不悔自己無見識,卻將醜語怪他人。\end{pl}
    \begin{note}庚側:不用寶玉見此詩,若長若短亦是大手法。庚雙夾:罵得痛快,非顰兒不可。真好顰兒,真好顰兒!好詩!若雲知音者顰兒也。至此方完“箴玉”半回。庚眉:又借阿顰詩自相鄙駁,可見餘前批不謬。己冬夜。庚眉:寶玉不見詩,是後文餘步也,《石頭記》得力所在。丁亥夏。 笏叟。\end{note}
\end{poem}


\begin{parag}
    寫畢,也往上房來見賈母,後往王夫人處來。
\end{parag}


\begin{parag}
    誰知鳳姐之女大姐病了,正亂著請大夫來診脈。大夫便說:“替夫人奶奶們道喜,姐兒發熱是見喜了,並非別病。”王夫人鳳姐聽了,忙遣人問:“可好不好?”醫生回道:“病雖險,卻順,\begin{note}庚側:在“子嗣艱難”化出。\end{note}倒還不妨。預備桑蟲豬尾要緊。”鳳姐聽了,登時忙將起來:一面打掃房屋供奉痘疹娘娘,一面傳與家人忌煎炒等物,一面命平兒打點鋪蓋衣服與賈璉隔房,一面又拿大紅尺頭與奶子丫頭親近人等裁衣。\begin{note}庚雙夾:幾個“一面”,寫得如見其景。\end{note}外面又打掃淨室,款留兩個醫生,輪流斟酌診脈下藥,十二日不放家去。賈璉只得搬出外書房來齋戒,\begin{note}庚側:此二字內生出許多事來。\end{note}鳳姐與平兒都隨著王夫人日日供奉娘娘。
\end{parag}


\begin{parag}
    那個賈璉,只離了鳳姐便要尋事,獨寢了兩夜,便十分難熬,便暫將小廝們內有清俊的選來出火。不想榮國府內有一個極不成器破爛酒頭廚子,名叫多官,\begin{note}庚雙夾:今是多多也,妙名!\end{note}人見他懦弱無能,都喚他作“多渾蟲”。\begin{note}庚雙夾:更好!今之渾蟲更多也。\end{note}因他自小父母替他在外娶了一個媳婦,今年方二十來往年紀,生得有幾分人才,見者無不羨愛。他生性輕浮,最喜拈花惹草,多渾蟲又不理論,只是有酒有肉有錢,便諸事不管了,所以榮寧二府之人都得入手。因這個媳婦美貌異常,輕浮無比,衆人都呼他作“多姑娘兒”。\begin{note}庚雙夾:更妙!\end{note}如今賈璉在外熬煎,往日也曾見過這媳婦,失過魂魄,只是內懼嬌妻,外懼孌寵,不曾下得手。那多姑娘兒也曾有意於賈璉,只恨沒空。今聞賈璉挪在外書房來,他便沒事也要走兩趟去招惹。惹的賈璉似飢鼠一般,少不得和心腹的小廝們計議,合同遮掩謀求,多以金帛相許。小廝們焉有不允之理,況都和這媳婦是好友,一說便成。是夜二鼓人定,多渾蟲醉昏在炕,賈璉便溜了來相會。進門一見其態,早已魄飛魂散,也不用情談款敘,便寬衣動作起來。誰知這媳婦有天生的奇趣,一經男子挨身,便覺遍身筋骨癱軟,\begin{note}庚雙夾:淫極!虧想的出!\end{note}使男子如臥綿上,\begin{note}庚雙夾:如此境界,自勝西方、蓬萊等處。\end{note}更兼淫態\begin{note}庚雙夾:總爲後文寶玉一篇作引。\end{note}浪言,壓倒娼妓,諸男子至此豈有惜命者哉。\begin{note}庚側:涼水灌頂之句。\end{note}那賈璉恨不得連身子化在他身上。\begin{note}庚雙夾:親極之語,趣極之語。\end{note}那媳婦故作浪語,在下說道: “你家女兒出花兒,供著娘娘,你也該忌兩日,倒爲我髒了身子。快離了我這裏罷。”\begin{note}庚側:淫婦勾人,慣加反語,看官著眼。\end{note}賈璉一面大動,一面喘吁吁答道:“你就是娘娘!我那裏管什麼娘娘!”\begin{note}庚側:亂語不倫,的是有之。\end{note}那媳婦越浪,賈璉越醜態畢露。\begin{note}蒙雙夾:可以噴飯!\end{note}一時事畢,兩個又海誓山盟,難分難捨,\begin{note}庚雙夾:著眼,再從前看如何光景。\end{note}此後遂成相契。\begin{note}庚雙夾:趣聞!“相契”作如此用,“相契”掃地矣。庚眉:一部書中,只有此一段醜極太露之文,寫於賈璉身上,恰極當極!己冬夜。\end{note}\begin{note}庚眉:看官熟思:寫珍、璉輩當以何等文方妥方恰也?壬午孟夏。\end{note}\begin{note}庚眉:此段系書中情之瘕疵,寫爲阿鳳生日潑醋回及“夭風流”寶玉悄看晴雯回作引,伏線千里外之筆也。丁亥夏。畸笏。\end{note}
\end{parag}


\begin{parag}
    一日大姐毒盡癍回,\begin{note}庚側:好快日子嚇!\end{note}十二日後送了娘娘,閤家祭天祀祖,還願焚香,慶賀放賞已畢,賈璉仍復搬進臥室。見了鳳姐,正是俗語云“新婚不如遠別”,更有無限恩愛,自不必煩絮。\begin{note}庚側:隱得好。\end{note}
\end{parag}


\begin{parag}
    次日早起,鳳姐往上屋去後,平兒收拾賈璉在外的衣服鋪蓋,不承望枕套中抖出一綹青絲來。平兒會意,忙拽在袖內,\begin{note}庚雙夾:好極!不料平兒大有襲卿之身分,可謂何地無材,蓋遭際有別耳。\end{note}便走至這邊房內來,拿出頭髮來,向賈璉笑道:“這是什麼?”\begin{note}庚雙夾:好看之極!\end{note}賈璉看見著了忙,\begin{note}庚批:也有今日。\end{note}搶上來要奪。平兒便跑,被賈璉一把揪住,按在炕上,掰手要奪,口內笑道:“小蹄子,你不趁早拿出來,我把你膀子橛折了。”\begin{note}庚側:無情太甚!\end{note}平兒笑道:“你就是沒良心的。我好意瞞著他來問,你倒賭狠!你只賭狠,等他回來我告訴他,\begin{note}庚側:有是語,恐卿口不應。\end{note}看你怎麼著。”賈璉聽說,忙陪笑央求道:“好人,賞我罷,我再不賭狠了。”\begin{note}庚雙夾:好聽好看之極,迥不犯襲卿。\end{note}
\end{parag}


\begin{parag}
    一語未了,只聽鳳姐聲音進來。\begin{note}庚側:《石頭記》大法小法累累如是,並不爲厭。驚天駭地之文!如何?不知下文怎樣了結,使賈璉及觀者一齊喪膽。\end{note}賈璉聽見鬆了手,平兒剛起身,鳳姐已走進來,命平兒快開匣子,替太太找樣子。平兒忙答應了找時,鳳姐見了賈璉,忽然想起來,便問平兒:“拿出去的東西都收進來了麼?”平兒道:“收進來了。”鳳姐道:“可少什麼沒有?”平兒道:“我也怕丟下一兩件,細細的查了查,也不少。”鳳姐道:“不少就好,只是別多出來罷?”\begin{note}庚側:看至此,寧不拍案叫絕?庚雙夾:奇!\end{note}平兒笑道:“不丟萬幸,誰還添出來呢?”\begin{note}庚側:可兒可兒,卿亦明知故說耳。\end{note}鳳姐冷笑道:“這半個月難保乾淨,或者有相厚的丟下的東西:戒指、汗巾、香袋兒,再至於頭髮、指甲,都是東西。”\begin{note}庚雙夾:好阿鳳,令人膽寒。\end{note}一席話,說的賈璉臉都黃了。賈璉在鳳姐身後,只望著平兒殺雞抹脖使眼色兒。\begin{note}蒙側:作丈夫者,要當自重!\end{note}平兒只裝著看不見,\begin{note}庚側:餘自有三分主意。\end{note}因笑道:“怎麼我的心就和奶奶的心一樣!我就怕有這些個,留神搜了一搜,竟一點破綻也沒有。奶奶不信時,那些東西我還沒收呢,奶奶親自翻尋一遍去。”\begin{note}庚雙夾:好平兒!遍天下懼內者來感謝。\end{note}鳳姐笑道:“傻丫頭,\begin{note}庚雙夾:可嘆可笑,竟不知誰傻。\end{note}他便有這些東西,那裏就叫咱們翻著了!”\begin{note}庚雙夾:好阿鳳,好文字,雖系閨中女兒口角小事,讀之不無聰明得失癡心真假之感。\end{note}說著,尋了樣子又上去了。
\end{parag}


\begin{parag}
    平兒指著鼻子,\begin{note}庚側:好看煞。\end{note}晃著頭笑道:\begin{note}庚側:可兒,可兒。\end{note}“這件事怎麼回謝我呢?”\begin{note}庚雙夾:姣俏如見,迥不犯襲卿麝月一筆。\end{note}喜的個賈璉身癢難撓,\begin{note}庚側:不但賈兄癢癢,即批書人此刻幾乎落筆。試部看官此際若何光景?\end{note}跑上來摟著,“心肝腸肉”亂叫亂謝。平兒仍拿了頭髮笑道:“這是我一生的把柄了。好就好,不好就抖露出這事來。”賈璉笑道:“你只好生收著罷,千萬別叫他知道。”口裏說著,瞅他不防,便搶了過來,\begin{note}庚側:畢肖。璉兄不分玉石,但負我平姐。奈何,奈何!\end{note}笑道:“你拿著終是禍患,不如我燒了他完事了。”\begin{note}庚雙夾:妙!設使平兒再不致泄露,故仍用賈璉搶回,後文遺失,過脈也。\end{note}一面說著,一面便塞於靴掖內。平兒咬牙道:“沒良心的東西,過了河就拆橋,明兒還想我替你撒謊!”賈璉見他嬌俏動情,便摟著求歡,被平兒奪手跑了,急的賈璉彎著腰恨道:“死促狹小淫婦!一定浪上人的火來,他又跑了。”\begin{note}庚雙夾:醜態如見,淫聲如聞,今古淫書未有之章法。\end{note}平兒在窗外笑道:“我浪我的,誰叫你動火了?\begin{note}庚雙夾:妙極之談。直是理學工夫,所謂不可正照風月鑑也。\end{note}難道圖你\begin{note}庚側:阿平,“你” 字作牽強,餘不畫押。一笑。\end{note}受用一回,叫他知道了,又不待見我。”\begin{note}庚雙夾:鳳姐醋妒,於平兒前猶如是,況他人乎!餘謂鳳姐必是甚於諸人。觀者不信,今平兒說出,然乎?否乎?\end{note}賈璉道:“你不用怕他,等我性子上來,把這醋罐打個稀爛,他才認得我呢!他防我象防賊的,只許他同男人說話,不許我和女人說話,我和女人略近些,他就疑惑,他不論小叔子侄兒,大的小的,說說笑笑,就不怕我喫醋了。\begin{note}蒙側:作者又何必如此想?亦犯此病也!\end{note}以後我也不許他見人!”\begin{note}庚雙夾:無理之甚,卻是妙極趣談,天下懼內者背後之談皆如此。\end{note}平兒道:“他醋你使得,你醋他使不得。他原行的正走的正,你行動便有個壞心,連我也不放心,別說他了。”賈璉道:“你兩個一口賊氣。都是你們行的是,我凡行動都存壞心。\begin{note}蒙側:一片俗氣!\end{note}多早晚都死在我手裏!”
\end{parag}


\begin{parag}
    一句未了,鳳姐走進院來,因見平兒在窗外,就問道:“要說話兩個人不在屋裏說,怎麼跑出一個來,隔著窗子,是什麼意思?”賈璉在窗內接道:“你可問他,倒象屋裏有老虎喫他呢。”\begin{note}庚雙夾:好!庚眉:此等章法是在戲場上得來,一笑。畸笏。\end{note}平兒道:“屋裏一個人沒有,我在他跟前作什麼?”鳳姐兒笑道:“正是沒人才好呢。”平兒聽說,便說道:“這話是說我呢?”鳳姐笑道:\begin{note}蒙雙夾:“笑”字妙!平兒反正色,鳳姐反陪笑,奇極意外之文。\end{note}“不說你說誰?”平兒道:“別叫我說出好話來了。”說著,也不打簾子讓鳳姐,自己先摔簾子進來,\begin{note}庚側:若在屋裏,何敢如此形景,不要加上許多小心?平兒平兒,有你說嘴的。\end{note}往那邊去了。鳳姐自掀簾子進來,說道:“平兒瘋魔了。這蹄子認真要降伏我,仔細你的皮要緊!”賈璉聽了,已絕倒在炕上,\begin{note}庚側:懼內形景寫盡了。\end{note}拍手笑道:“我竟不知平兒這麼利害,從此倒伏他了。”鳳姐道:“都是你慣的他,我只和你說!”賈璉聽說忙道:“你兩個不卯,又拿我來作人。我躲開你們。”鳳姐道:“我看你躲到那裏去。”\begin{note}蒙側:世俗之態燻人。\end{note}賈璉道:“我就來。”鳳姐道:“我有話和你商量。”不知商量何事,且聽下回分解。\begin{note}庚側:收得淡雅之至!\end{note}正是:
\end{parag}


\begin{poem}
    \begin{pl}淑女從來多抱怨,嬌妻自古便含酸。\end{pl}
    \begin{note}庚雙夾:二語包盡古今萬世裙衩。\end{note}
\end{poem}


\begin{parag}
    \begin{note}蒙回末總評:不惜恩愛爲良人,方是溫存一脈真。俗子妒婦渾可笑,語言便自笑風塵。\end{note}
\end{parag}
