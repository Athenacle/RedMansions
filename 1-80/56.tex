\chap{五十六}{敏探春興利除宿弊 識寶釵小惠全大體}


\begin{parag}
    \begin{note}蒙回前總:敘入夢景,極迷離,卻極分明。牛鬼蛇神不犯筆端,全從至情至理中寫來,齊諧莫能載也。\end{note}
\end{parag}


\begin{parag}
    話說平兒陪著鳳姐兒吃了飯,伏侍盥漱畢,方往探春處來。只見院中寂靜,只有丫鬟婆子諸內壼近人在窗外聽候。
\end{parag}


\begin{parag}
    平兒進入廳中,他姊妹三人正議論些家務,說的便是年內賴大家請喫酒他家花園中事故。見他來了,探春便命他腳踏上坐了,因說道:“我想的事不爲別的,因想著我們一月有二兩月銀外,丫頭們又另有月錢。前兒又有人回,要我們一月所用的頭油脂粉,每人又是二兩。這又同纔剛學裏的八兩一樣,重重疊疊,事雖小,錢有限,看起來也不妥當。你奶奶怎麼就沒想到這個?”平兒笑道:“這有個原故:姑娘們所用的這些東西,自然是該有分例。每月買辦買了,令女人們各房交與我們收管,不過預備姑娘們使用就罷了,沒有一個我們天天各人拿錢找人買頭油又是脂粉去的理。所以外頭買辦總領了去,按月使女人按房交與我們的。姑娘們的每月這二兩,原不是爲買這些的,原爲的是一時當家的奶奶太太或不在,或不得閒,姑娘們偶然一時可巧要幾個錢使,省得找人去。這原是恐怕姑娘們受委屈,可知這個錢並不是買這個纔有的。如今我冷眼看著,各房裏的我們的姊妹都是現拿錢買這些東西的,竟有一半。我就疑惑,不是買辦脫了空,遲些日子,就是買的不是正經貨,弄些使不得的東西來搪塞。”探春李紈都笑道:“你也留心看出來了。脫空是沒有的,也不敢,只是遲些日子;催急了,不知那裏弄些來,不過是個名兒,其實使不得,依然得現買。就用這二兩銀子,另叫別人的奶媽子的或是弟兄哥哥的兒子買了來才使得。若使了官中的人,依然是那一樣的。不知他們是什麼法子,是鋪子裏壞了不要的,他們都弄了來,單預備給我們?”平兒笑道:“買辦買的是那樣的,他買了好的來,買辦豈肯和他善開交,又說他使壞心要奪這買辦了。所以他們也只得如此,寧可得罪了裏頭,不肯得罪了外頭辦事的人。姑娘們只能可使奶媽媽們,他們也就不敢閒話了。”探春道:“因此我心中不自在。錢費兩起,東西又白丟一半,通算起來,反費了兩摺子,不如竟把買辦的每月蠲了爲是。此是一件事。第二件,年裏往賴大家去,你也去的,你看他那小園子比咱們這個如何?”平兒笑道: “還沒有咱們這一半大,樹木花草也少多了。”探春道:“我因和他家女兒說閒話兒,誰知那麼個園子,除他們帶的花、喫的筍菜魚蝦之外,一年還有人包了去,年終足有二百兩銀子剩。從那日我才知道,一個破荷葉,一根枯草根子,都是值錢的。”
\end{parag}


\begin{parag}
    寶釵笑道:“真真膏粱紈絝之談。雖是千金小姐,原不知這事,但你們都念過書識字的,竟沒看見朱夫子有一篇《不自棄文》不成?”探春笑道:“雖看過,那不過是勉人自勵,虛比浮詞,那裏都真有的?”寶釵道:“朱子都有虛比浮詞?那句句都是有的。你才辦了兩天時事,就利慾薰心,把朱子都看虛浮了。你再出去見了那些利弊大事,越發把孔子也看虛了!”探春笑道:“你這樣一個通人,竟沒看見子書?當日《姬子》有云:‘登利祿之場,處運籌之界者,堯舜之詞,背孔孟之道。’”寶釵笑道:“底下一句呢?”探春笑道:“如今只斷章取意,念出底下一句,我自己罵我自己不成?”寶釵道:“天下沒有不可用的東西;既可用,便值錢。難爲你是個聰敏人,這些正事大節目事竟沒經歷,也可惜遲了。”\begin{note}庚雙夾:反點題,文法中又一變體也。\end{note}李紈笑道:“叫了人家來,不說正事,且你們對講學問。”寶釵道:“學問中便是正事。此刻於小事上用學問一提,那小事越發作高一層了。不拿學問提著,便都流入市俗去了。”
\end{parag}


\begin{parag}
    三人只是取笑之談,說了笑了一回,便仍談正事。\begin{note}庚雙夾:作者又用金蟬脫殼之法。\end{note}探春因又接說道:“咱們這園子只算比他們的多一半,加一倍算,一年就有四百銀子的利息。若此時也出脫生髮銀子,自然小器,不是咱們這樣人家的事。若派出兩個一定的人來,既有許多值錢之物,一味任人作踐,也似乎暴殄天物。不如在園子裏所有的老媽媽中,揀出幾個本分老誠能知園圃的事,派準他們收拾料理,也不必要他們交租納稅,只問他們一年可以孝敬些什麼。一則園子有專定之人修理,花木自有一年好似一年的,也不用臨時忙亂;二則也不至作踐,白辜負了東西;三則老媽媽們也可藉此小補,不枉年日在園中辛苦;四則亦可以省了這些花兒匠山子匠打掃人等的工費。將此有餘,以補不足,未爲不可。”寶釵正在地下看壁上的字畫,聽如此說一則,便點一回頭,說完,便笑道:“善哉,三年之內無饑饉矣!”李紈笑道:“好主意。這果一行,太太必喜歡。省錢事小,第一有人打掃,專司其職,又許他們去賣錢。使之以權,動之以利,再無不盡職的了。” 平兒道:“這件事須得姑娘說出來。我們奶奶雖有此心,也未必好出口。此刻姑娘們在園裏住著,不能多弄些玩意兒去陪襯,反叫人去監管修理,圖省錢,這話斷不好出口。”寶釵忙走過來,摸著他的臉笑道:“你張開嘴,我瞧瞧你的牙齒舌頭是什麼作的。從早起來到這會子,你說這些話,一套一個樣子,也不奉承三姑娘,也沒見你說奶奶才短想不到,也並沒有三姑娘說一句,你就說一句是;橫豎三姑娘一套話出,你就有一套話進去;總是三姑娘想的到的,你奶奶也想到了,只是必有個不可辦的原故。這會子又是因姑娘住的園子,不好因省錢令人去監管。你們想想這話,若果真交與人弄錢去的,那人自然是一枝花也不許掐,一個果子也不許動了,姑娘們分中自然不敢,天天與小姑娘們就吵不清。他這遠愁近慮,不亢不卑。他奶奶便不是和咱們好,聽他這一番話,也必要自愧的變好了,不和也變和了。”探春笑道:“我早起一肚子氣,聽他來了,忽然想他主子來,素日當家使出來的好撒野的人,我見了他便生了氣。誰知他來了,避貓鼠兒似的站了半日,怪可憐的。接著又說了那麼些話,不說他主子待我好,倒說‘不枉姑娘待我們奶奶素日的情意了’。這一句,不但沒了氣,我倒愧了,又傷起心來。我細想,我一個女孩兒家,自己還鬧得沒人疼沒人顧的,我那裏還有好處去待人。”口內說到這裏,不免又流下淚來。李紈等見他說的懇切,又想他素日趙姨娘每生誹謗,在王夫人跟前亦爲趙姨娘所累,亦都不免流下淚來,都忙勸道:“趁今日清淨,大家商議兩件興利剔弊的事,也不枉太太委託一場。又提這沒要緊的事做什麼?”平兒忙道:“我已明白了。姑娘竟說誰好,竟一派人就完了。”探春道:“雖如此說,也須得回你奶奶一聲。我們這裏搜剔小遺,已經不當,皆因你奶奶是個明白人,我才這樣行,若是糊塗多蠱多妒的,我也不肯,倒象抓他乖一般。豈可不商議了行。”平兒笑道:“既這樣,我去告訴一聲。” 說著去了,半日方回來,笑說:“我說是白走一趟,這樣好事,奶奶豈有不依的。”
\end{parag}


\begin{parag}
    探春聽了,便和李紈命人將園中所有婆子的名單要來,大家參度,大概定了幾個。又將他們一齊傳來,李紈大概告訴與他們。衆人聽了,無不願意,也有說: “那一片竹子單交給我,一年工夫,明年又是一片。除了家裏喫的筍,一年還可交些錢糧。”這一個說:“那一片稻地交給我,一年這些頑的大小雀鳥的糧食不必動官中錢糧,我還可以交錢糧。”探春纔要說話,人回:“大夫來了,進園瞧姑娘。”衆婆子只得去接大夫。平兒忙說:“單你們,有一百個也不成個體統,難道沒有兩個管事的頭腦帶進大夫來?”回事的那人說:“有,吳大娘和單大娘他兩個在西南角上聚錦門等著呢。”平兒聽說,方罷了。
\end{parag}


\begin{parag}
    衆婆子去後,探春問寶釵如何。寶釵笑答道:“幸於始者怠於終,繕其辭者嗜其利。”探春聽了點頭稱讚,便向冊上指出幾人來與他三人看。平兒忙去取筆硯來。他三人說道:“這一個老祝媽是個妥當的,況他老頭子和他兒子代代都是管打掃竹子,如今竟把這所有的竹子交與他。這一個老田媽本是種莊稼的,稻香村一帶凡有菜蔬稻稗之類,雖是頑意兒,不必認真大治大耕,也須得他去,再一按時加些培植,豈不更好?” 探春又笑道:“可惜,蘅蕪苑和怡紅院這兩處大地方竟沒有出利息之物。”李紈忙笑道:“ 課咴更利害。如今香料鋪並大市大廟賣的各處香料香草兒,都不是這些東西?算起來比別的利息更大。怡紅院別說別的,單隻說春夏天一季玫瑰花,共下多少花?還有一帶籬笆上薔薇、月季、寶相、金銀藤,單這沒要緊的草花幹了,賣到茶葉鋪藥鋪去,也值幾個錢。”探春笑道:“原來如此。只是弄香草的沒有在行的人。”平兒忙笑道:“跟寶姑娘的鶯兒他媽就是會弄這個的,上回他還採了些曬乾了編成花籃葫蘆給我頑的,姑娘倒忘了不成?”寶釵笑道:“我才贊你,你到來捉弄我了。”三人都詫異,都問這是爲何。寶釵道:“斷斷使不得!你們這裏多少得用的人,一個一個閒著沒事辦,這會子我又弄個人來,叫那起人連我也看小了。我倒替你們想出一個人來:怡紅院有個老葉媽,他就是茗煙的娘。那是個誠實老人家,他又和我們鶯兒的娘極好,不如把這事交與葉媽。他有不知的,不必咱們說,他就找鶯兒的娘去商議了。那怕葉媽全不管,竟交與那一個,那是他們私情兒,有人說閒話,也就怨不到咱們身上了。如此一行,你們辦的又至公,於事又甚妥。”李紈平兒都道:“是極。”\begin{note}庚雙夾:寶釵此等非與鳳姐一樣,此則隨時俯仰,彼則逸才逾蹈耳。\end{note}探春笑道:“雖如此,只怕他們見利忘義。”\begin{note}庚雙夾:這是探春敏智過人處,此諷亦不可少。\end{note}平兒笑道:“不相干,前兒鶯兒還認了葉媽做乾孃,請喫飯喫酒,兩家和厚的好的很呢。”\begin{note}庚雙夾:夾寫大觀園中多少兒女家常閒景,此亦補前文之不足也。\end{note}探春聽了,方罷了。又共同斟酌出幾人來,俱是他四人素昔冷眼取中的,用筆圈出。
\end{parag}


\begin{parag}
    一時婆子們來回大夫已去,將藥方送上去。三人看了,一面遣人送出去取藥,監派調服,一面探春與李紈明示諸人:某人管某處,按四季除家中定例用多少外,餘者任憑你們採取了去取利,年終算帳。探春笑道:“我又想起一件事:若年終算帳歸錢時,自然歸到帳房,仍是上頭又添一層管主,還在他們手心裏,又剝一層皮。這如今我們興出這事來派了你們,已是跨過他們的頭去了,心裏有氣,只說不出來;你們年終去歸帳,他還不捉弄你們等什麼?再者,這一年間管什麼的,主子有一全分,他們就得半分。這是家裏的舊例,人所共知的,別的偷著的在外。如今這園子裏是我的新創,竟別入他們手,每年歸帳,竟歸到裏頭來纔好。”寶釵笑道:“依我說,裏頭也不用歸帳。這個多了那個少了,倒多了事。不如問他們誰領這一分的,他就攬一宗事去。不過是園裏的人的動用。我替你們算出來了,有限的幾宗事:不過是頭油、胭粉、香、紙,每一位姑娘幾個丫頭,都是有定例的;再者,各處笤帚、撮簸、撣子並大小禽鳥、鹿、兔喫的糧食。不過這幾樣,都是他們包了去,不用帳房去領錢。你算算,就省下多少來?”平兒笑道:“這幾宗雖小,一年通共算了,也省的下四百兩銀子。”寶釵笑道:“卻又來,一年四百,二年八百兩,取租的房子也能看得了幾間,薄地也可添幾畝。雖然還有敷餘的,但他們既辛苦鬧一年,也要叫他們剩些,粘補粘補自家。雖是興利節用爲綱,然亦不可太嗇。縱再省上二三百銀子,失了大體統也不象。所以如此一行,外頭帳房裏一年少出四五百銀子,也不覺得很艱嗇了,他們裏頭卻也得些小補。這些沒營生的媽媽們也寬裕了,園子裏花木,也可以每年滋長蕃盛,你們也得了可使之物。這庶幾不失大體。若一味要省時,那裏不搜尋出幾個錢來。凡有些餘利的,一概入了官中,那時裏外怨聲載道,豈不失了你們這樣人家的大體?如今這園裏幾十個老媽媽們,若只給了這個,那剩的也必抱怨不公。我才說的,他們只供給這個幾樣,也未免太寬裕了。一年竟除這個之外,他每人不論有餘無餘,只叫他拿出若干貫錢來,大家湊齊,單散與園中這些媽媽們。他們雖不料理這些,卻日夜也是在園中照看當差之人,關門閉戶,起早睡晚,大雨大雪,姑娘們出入,抬轎子,撐船,拉冰牀,一應粗糙活計,都是他們的差使。一年在園裏辛苦到頭,這園內既有出息,也是分內該沾帶些的。還有一句至小的話,越發說破了:你們只管了自己寬裕,不分與他們些,他們雖不敢明怨,心裏卻都不服,只用假公濟私的多摘你們幾個果子,多掐幾枝花兒,你們有冤還沒處訴。他們也沾帶了些利息,你們有照顧不到,他們就替你照顧了。”
\end{parag}


\begin{parag}
    衆婆子聽了這個議論,又去了帳房受轄制,又不與鳳姐兒去算帳,一年不過多拿出若干貫錢來,各各歡喜異常,都齊說:“願意。強如出去被他揉搓著,還得拿出錢來呢。”那不得管地的聽了每年終又無故得分錢,也都喜歡起來,口內說:“他們辛苦收拾,是該剩些錢粘補的。我們怎麼好‘穩坐喫三注’的?”寶釵笑道: “媽媽們也別推辭了,這原是分內應當的。你們只要日夜辛苦些,別躲懶縱放人喫酒賭錢就是了。不然,我也不該管這事;你們一般聽見,姨娘親口囑託我三五回,說大奶奶如今又不得閒兒,別的姑娘又小,託我照看照看。我若不依,分明是叫姨娘操心。你們奶奶又多病多痛,家務也忙。我原是個閒人,便是個街坊鄰居,也要幫著些,何況是親姨娘託我。我免不得去小就大,講不起衆人嫌我。倘或我只顧了小分沽名釣譽,那時酒醉賭博生出事來,我怎麼見姨娘?你們那時後悔也遲了,就連你們素日的老臉也都丟了。這些姑娘小姐們,這麼一所大花園子,都是你們照看,皆因看得你們是三四代的老媽媽,最是循規遵矩的,原該大家齊心,顧些體統。你們反縱放別人任意喫酒賭博,姨娘聽見了,教訓一場猶可,倘或被那幾個管家娘子聽見了,他們也不用回姨娘,竟教導你們一番。你們這年老的反受了年小的教訓,雖是他們是管家,管的著你們,何如自己存些體統,他們如何得來作踐。所以我如今替你們想出這個額外的進益來,也爲大家齊心把這園裏周全的謹謹慎慎,使那些有權執事的看見這般嚴肅謹慎,且不用他們操心,他們心裏豈不敬伏。也不枉替你們籌劃進益,既能奪他們之權,生你們之利,豈不能行無爲之治,分他們之憂。你們去細想想這話。”家人都歡聲鼎沸說:“姑娘說的很是。從此姑娘奶奶只管放心,姑娘奶奶這樣疼顧我們,我們再要不體上情,天地也不容了。”
\end{parag}


\begin{parag}
    剛說著,只見林之孝家的進來說:“江南甄府裏家眷昨日到京,今日進宮朝賀。此刻先遣人來送禮請安。”說著,便將禮單送上去。探春接了,看道是:“上用的妝緞蟒緞十二匹,上用雜色緞十二匹,上用各色紗十二匹,上用宮綢十二匹,官用各色緞紗綢綾二十四匹。”李紈也看過,說:“用上等封兒賞他。”因又命人回了賈母。賈母便命人叫李紈、探春、寶釵等也都過來,將禮物看了。李紈收過,一邊吩咐內庫上人說:“等太太回來看了再收。”賈母因說:“這甄家又不與別家相同,上等賞封賞男人,只怕展眼又打發女人來請安,預備下尺頭。” 一語未完,果然人回:“甄府四個女人來請安。”賈母聽了,忙命人帶進來。
\end{parag}


\begin{parag}
    那四個人都是四十往上的年紀,穿戴之物,皆比主子不甚差別。請安問好畢,賈母命拿了四個腳踏來,他四人謝了坐,待寶釵等坐了,方都坐下。賈母便問: “多早晚進京的?”四人忙起身回說:“昨兒進的京。今日太太帶了姑娘進宮請安去了,故令女人們來請安,問候姑娘們。”賈母笑問道:“這些年沒進京,也不想到今年來。”四人也都笑回道:“正是,今年是奉旨進京的。”賈母問道:“家眷都來了?”四人回說:“老太太和哥兒、兩位小姐並別位太太都沒來,就只太太帶了三姑娘來了。”賈母道:“有人家沒有?”四人道:“尚沒有。”賈母笑道:“你們大姑娘和二姑娘這兩家,都和我們家甚好。”四人笑道:“正是。每年姑娘們有信回去說,全虧府上照看。”賈母笑道:“什麼照看,原是世交,又是老親,原應當的。你們二姑娘更好,更不自尊自大,所以我們才走的親密。”四人笑道: “這是老太太過謙了。”賈母又問:“你這哥兒也跟著你們老太太?”四人回說:“也是跟著老太太。”賈母道:“幾歲了?”又問:“上學不曾?”四人笑說: “今年十三歲。因長得齊整,老太太很疼。自幼淘氣異常,天天逃學,老爺太太也不便十分管教。”賈母笑道:“也不成了我們家的了!你這哥兒叫什麼名字?”四人道:“因老太太當作寶貝一樣,他又生的白,老太太便叫作寶玉。”賈母便向李紈等道:“偏也叫作個寶玉。”李紈忙欠身笑道:“從古至今,同時隔代重名的很多。”四人也笑道:“起了這小名兒之後,我們上下都疑惑,不知那位親友家也倒似曾有一個的。只是這十來年沒進京來,卻記不得真了。”賈母笑道:“豈敢,就是我的孫子。人來。”衆媳婦丫頭答應了一聲,走近幾步。賈母笑道:“園裏把咱們的寶玉叫了來,給這四個管家娘子瞧瞧,比他們的寶玉如何?”
\end{parag}


\begin{parag}
    衆媳婦聽了,忙去了,半刻圍了寶玉進來。四人一見,忙起身笑道:“唬了我們一跳。若是我們不進府來,倘若別處遇見,還只道我們的寶玉後趕著也進了京了呢。”一面說,一面都上來拉他的手,問長問短。寶玉忙也笑問好。賈母笑道:“比你們的長的如何?”李紈等笑道:“四位媽媽才一說,可知是模樣相仿了。”賈母笑道:“那有這樣巧事?大家子孩子們再養的嬌嫩,除了臉上有殘疾十分黑醜的,大概看去都是一樣的齊整。這也沒有什麼怪處。”四人笑道:“如今看來,模樣是一樣。據老太太說,淘氣也一樣。我們看來,這位哥兒性情卻比我們的好些。”賈母忙問:“怎見得?”四人笑道:“方纔我們拉哥兒的手說話便知。我們那一個只說我們糊塗,慢說拉手,他的東西我們略動一動也不依。所使喚的人都是女孩子們。” 四人未說完,李紈姊妹等禁不住都失聲笑出來。賈母也笑道:“我們這會子也打發人去見了你們寶玉,若拉他的手,他也自然勉強忍耐一時。可知你我這樣人家的孩子們,憑他們有什麼刁鑽古怪的毛病兒,見了外人,必是要還出正經禮數來的。若他不還正經禮數,也斷不容他刁鑽去了。就是大人溺愛的,是他一則生的得人意,二則見人禮數竟比大人行出來的不錯,使人見了可愛可憐,背地裏所以才縱他一點子。若一味他只管沒裏沒外,不與大人爭光,憑他生的怎樣,也是該打死的。”四人聽了,都笑道:“老太太這話正是。雖然我們寶玉淘氣古怪,有時見了人客,規矩禮數更比大人有禮。所以無人見了不愛,只說爲什麼還打他。殊不知他在家裏無法無天,大人想不到的話偏會說,想不到的事他偏要行,所以老爺太太恨的無法。就是弄性,也是小孩子的常情,胡亂花費,這也是公子哥兒的常情,怕上學,也是小孩子的常情,都還治的過來。第一,天生下來這一種刁鑽古怪的脾氣,如何使得。”一語未了,人回:“太太回來了。”王夫人進來問過安。他四人請了安,大概說了兩句。賈母便命歇歇去。王夫人親捧過茶,方退出。四人告辭了賈母,便往王夫人處來,說了一會家務,打發他們回去,不必細說。
\end{parag}


\begin{parag}
    這裏賈母喜的逢人便告訴,也有一個寶玉,也卻一般行景。衆人都爲天下之大,世宦之多,同名者也甚多,祖母溺愛孫者也古今所有常事耳,不是什麼罕事,故皆不介意。獨寶玉是個迂闊呆公子的性情,自爲是那四人承悅賈母之詞。後至蘅蕪苑去看湘雲病去,史湘雲說他:“你放心鬧罷,先是‘單絲不成線,獨樹不成林 ’,如今有了個對子,鬧急了,再打很了,你逃走到南京找那一個去。”寶玉道:“那裏的謊話你也信了,偏又有個寶玉了?”湘雲道:“怎麼列國有個藺相如,漢朝又有個司馬相如呢?”寶玉笑道:“這也罷了,偏又模樣兒也一樣,這是沒有的事。”湘雲道:“怎麼匡人看見孔子,只當是陽虎呢?”寶玉笑道:“孔子、陽虎雖同貌,卻不同名;藺與司馬雖同名,而又不同貌;偏我和他就兩樣俱同不成?”湘雲沒了話答對,因笑道:“你只會胡攪,我也不和你分證。有也罷,沒也罷,與我無干。”說著便睡下了。
\end{parag}


\begin{parag}
    寶玉心中便又疑惑起來:若說必無,然亦似有;若說必有,又並無目睹。心中悶了,回至房中榻上默默盤算,不覺就忽忽的睡去,不覺竟到了一座花園之內。寶玉詫異道:“除了我們大觀園,竟又有這一個園子?”\begin{note}庚雙夾:寫園可知。\end{note}正疑惑間,從那邊來了幾個女兒,都是丫鬟。寶玉又詫異道:“除了鴛鴦、襲人、平兒之外,也竟還有這一干人?”\begin{note}庚雙夾:寫人可知。妙在更不說“更強”二字。\end{note}只見那些丫鬟笑道:“寶玉怎麼跑到這裏來了?”寶玉只當是說他,自己忙來陪笑說道:“因我偶步到此,不知是那位世交的花園,好姐姐們,帶我逛逛。”衆丫鬟都笑道:“原來不是咱家的寶玉。他生的倒也還乾淨,\begin{note}庚雙夾:妙在玉卿身上只落了這兩個字,亦不奇了。\end{note}嘴兒也倒乖覺。”寶玉聽了,忙道:“姐姐們,這裏也更還有個寶玉?”丫鬟們忙道:“寶玉二字,我們是奉老太太、太太之命,爲保佑他延壽消災的。我叫他,他聽見喜歡。你是那裏遠方來的臭小廝,也亂叫起他來。仔細你的臭肉,打不爛你的。”又一個丫鬟笑道:“咱們快走罷,別叫寶玉看見,又說同這臭小廝說了話,把咱燻臭了。”說著一徑去了。
\end{parag}


\begin{parag}
    寶玉納悶道:“從來沒有人如此塗毒我,他們如何更這樣?真亦有我這樣一個人不成?”一面想,一面順步早到了一所院內。寶玉又詫異道:“除了怡紅院,也更還有這麼一個院落。”忽上了臺磯,進入屋內,只見榻上有一個人臥著,那邊有幾個女孩兒做針線,也有嘻笑頑耍的。只見榻上那個少年嘆了一聲。一個丫鬟笑問道:“寶玉,你不睡又嘆什麼?想必爲你妹妹病了,你又胡愁亂恨呢。”寶玉聽說,心下也便喫驚。只見榻上少年說道:“我聽見老太太說,長安都中也有個寶玉,和我一樣的性情,我只不信。我才作了一個夢,竟夢中到了都中一個花園子裏頭,遇見幾個姐姐,都叫我臭小廝,不理我。好容易找到他房裏頭,偏他睡覺,空有皮囊,真性不知那去了。”寶玉聽說,忙說道:“我因找寶玉來到這裏。原來你就是寶玉?”榻上的忙下來拉住:“原來你就是寶玉?這可不是夢裏了。”寶玉道: “這如何是夢?真而又真了。”一語未了,只見人來說:“老爺叫寶玉。”唬得二人皆慌了。一個寶玉就走,一個寶玉便忙叫:“寶玉快回來,快回來!”
\end{parag}


\begin{parag}
    襲人在旁聽他夢中自喚,忙推醒他,笑問道:“寶玉在那裏?”此時寶玉雖醒,神意尚恍惚,因向門外指說:“纔出去了。”襲人笑道:“那是你夢迷了。你揉眼細瞧,是鏡子裏照的你影兒。”寶玉向前瞧了一瞧,原是那嵌的大鏡對面相照,自己也笑了。早有人捧過漱盂茶滷來,漱了口。麝月道:“怪道老太太常囑咐說小人屋裏不可多有鏡子。小人魂不全,有鏡子照多了,睡覺驚恐作胡夢。如今倒在大鏡子那裏安了一張牀。有時放下鏡套還好;往前去,天熱睏倦不定,那裏想的到放他,比如方纔就忘了。自然是先躺下照著影兒頑的,一時合上眼,自然是胡夢顛倒;不然如何得看著自己叫著自己的名字?不如明兒挪進牀來是正經。”一語未了,只見王夫人遣人來叫寶玉,不知有何話說──\begin{note}庚雙夾:此下緊接“慧紫鵑試忙玉”。\end{note}
\end{parag}
