\chap{一十八}{林黛玉误剪香囊袋 贾元春归省庆元宵}

\begin{parag}
    \begin{note}戚本回前总评:一物珍藏见致情,豪华每向闹中争。黛林宝薛传佳句,豪宴仙缘留趣名。为剪荷包绾两意,屈从优女结三生。可怜转眼皆虚话,云自飘飘月自明。\end{note}
\end{parag}


\begin{parag}
    至院外,就有跟贾政的几个小厮上来拦腰抱住,都说:“今儿亏我们,老爷才喜欢,老太太打发人出来问了几遍,都亏我们回说喜欢;\begin{note}庚辰侧批:下人口气毕肖。\end{note}不然,若老太太叫你进去,就不得展才了。人人都说,你才那些诗比世人的都强。今儿得了这样的彩头,该赏我们了。”宝玉笑道:“每人一吊钱。”众人道:“谁没见那一吊钱!\begin{note}庚辰侧批:钱亦有没用处。\end{note}把这荷包赏了罢。”说著,一个上来解荷包,那一个就解扇囊,不容分说,将宝玉所佩之物尽行解去。又道:“好生送上去罢。”一个抱了起来,几个围绕,送至贾母二门前。\begin{note}庚辰侧批:好收煞。\end{note}那时贾母已命人看了几次。众奶娘丫鬟跟上来,见过贾母,知道不曾难为著他,心中自是喜欢。
\end{parag}


\begin{parag}
    少时袭人倒了茶来,见身边佩物一件无存,\begin{note}庚辰侧批:袭人在玉兄一身无时不照察到。\end{note}因笑道:“带的东西又是那起没脸的东西们解了去了。”林黛玉听说,走来瞧瞧,果然一件无存,因向宝玉道:“我给你的那个荷包也给他们了?\begin{note}庚辰侧批:又起楼阁。\end{note}你明儿再想我的东西,可不能够了!”说毕,赌气回房,将前日宝玉所烦他作的那个香袋儿,做了一半,赌气拿过来就铰。宝玉见他生气,便知不妥,忙赶过来,早剪破了。宝玉已见过这香囊,虽尚未完,却十分精巧,费了许多工夫,今见无故剪了,却也可气。因忙把衣领解了,从里面红袄襟上将黛玉所给的那荷包解了下来,递与黛玉瞧道:“你瞧瞧,这是什么!我那一回把你的东西给人了?”林黛玉见他如此珍重,带在里面,\begin{note}庚辰双行夹批:按理论之,则是“天下本无事,庸人自扰之”。若以儿女之情论之,则事必有之,事必有之,理又系今古小说中不能道得写得,谈情者不能说出讲出,情痴之至文也!\end{note}可知是怕人拿去之意,因此又自悔莽撞,未见皂白就剪了香袋,\begin{note}庚辰双行夹批:情痴之至!若无此悔便是一庸俗小性之女子矣。\end{note}因此又愧又气,低头一言不发。宝玉道:“你也不用剪,我知道你是懒待给我东西。我连这荷包奉还,何如?”说著,掷向他怀中便走。\begin{note}庚辰双行夹批:这确是难怪。\end{note}黛玉见如此,越发气起来,声咽气堵,又汪汪的滚下泪来,\begin{note}庚辰双行夹批:怒之极正是情之极。\end{note}拿起荷包来又剪。宝玉见他如此,忙回身抢住,笑道:“好妹妹,饶了他罢!”\begin{note}庚辰双行夹批:这方是宝玉。\end{note}黛玉将剪子一摔,拭泪说道:“你不用同我好一阵歹一阵的,要恼,就撂开手。这当了什么!”说著,赌气上床,面向里倒下拭泪。禁不住宝玉上来“妹妹”长“妹妹”短赔不是。
\end{parag}


\begin{parag}
    前面贾母一片声找宝玉。众奶娘丫鬟们忙回说:“在林姑娘房里呢。”贾母听说道:“好,好,好!让他们姊妹们一处顽顽罢。才他老子拘了他这半天,让他开心一会子罢。只别叫他们拌嘴,不许扭了他。”众人答应著。黛玉被宝玉缠不过,只得起来道:“你的意思不叫我安生,我就离了你。”说著往外就走。宝玉笑道:“你到那里,我跟到那里。”一面仍拿起荷包来带上。黛玉伸手抢道:“你说不要了,这会子又带上,我也替你怪臊的!”说著,嗤的一声笑了。宝玉道:“好妹妹,明日另替我作个香袋儿罢。”黛玉道:“那也只瞧我的高兴罢了。”一面说,一面二人出房,到王夫人上房中去了,\begin{note}庚辰双行夹批:一段点过二玉公案,断不可少。\end{note}可巧宝钗亦在那里。
\end{parag}


\begin{parag}
    此时王夫人那边热闹非常。\begin{note}庚辰双行夹批:四字特补近日千忙万冗多少花团锦簇文字。\end{note}原来贾蔷已从姑苏采买了十二个女孩子,并聘了教习,以及行头等事来了。那时薛姨妈另迁于东北上一所幽静房舍居住,将梨香院早已腾挪出来,另行修理了,就令教习在此教演女戏。又另派家中旧有曾演学过歌唱的众女人们,如今皆已皤然老妪了,\begin{note}庚辰双行夹批:又补出当日宁、荣在世之事,所谓此是末世之时也。\end{note}著他们带领管理。就令贾蔷总理其日用出入银钱等事,以及诸凡大小所需之物料帐目。\begin{note}庚辰双行夹批:补出女戏一段,又伏一案。\end{note}又有林之孝家的来回:“采访聘买的十个小尼姑、小道姑都有了,连新作的二十分道袍也有了。外有一个带发修行的,本是苏州人氏,祖上也是读书仕宦之家。因生了这位姑娘自小多病,买了许多替身儿皆不中用,到底这位姑娘亲自入了空门,方才好了,所以带发修行,今年才十八岁,法名妙玉。\begin{note}庚辰双行夹批:妙卿出现。至此细数十二钗,以贾家四艳再加薛林二冠有六,添秦可卿有七,熙凤有八,李纨有九,今又加妙玉仅得十人矣。后有史湘云与熙凤之女巧姐儿者共十二人,雪芹题曰“金陵十二钗”是本宗《红楼梦》十二曲之意。后宝琴、岫烟、李纹、李绮皆陪客也,《红楼梦》中所谓副十二钗是也。又有又副册三断词乃晴雯、袭人、香菱三人,余未多及,想为金钏、玉钏、鸳鸯、苗云\begin{subnote}按:书中不见此人,想是彩云?\end{subnote}、平儿等人无疑矣。观者不待言可知,故不必多费笔墨。\end{note}\begin{note}庚辰眉批:妙玉世外人也,故笔笔带写,妙极妥极!畸笏。\end{note}\begin{note}庚辰眉批:是处引十二钗总未的确,皆系漫拟也。至回末警幻情榜方知正、副、再副及三四副芳讳。壬午季春。畸笏。\end{note}如今父母俱已亡故,身边只有两个老嬷嬷,一个小丫头伏侍。文墨也极通,经文也不用学了,模样儿又极好。因听见长安都中有观音遗迹并贝叶遗文,去岁随了师父上来,\begin{note}庚辰双行夹批:因此方使妙卿入都。\end{note}现在西门外牟尼院住著。他师父极精演先天神数,于去冬圆寂了。妙玉本欲扶灵回乡的,他师父临寂遗言,说他‘衣食起居不宜回乡,在此静居,后来自有你的结果 ’。所以他竟未回乡。”王夫人不等回完,便说:“既这样,我们何不接了他来。”林之孝家的回道:“请他,他说:‘侯门公府,必以贵势压人,我再不去的。 ’”\begin{note}庚辰双行夹批:补出妙卿身世不凡心性高洁。\end{note}王夫人道:“他既是官宦小姐,自然骄傲些,就下个帖子请他何妨。”林之孝家的答应了出去,命书启相公写请帖去请妙玉。次日遣人备车轿去接等后话,暂且搁过,此时不能表白。\begin{note}庚辰双行夹批:补尼道一段,又伏一案。\end{note}\begin{note}己卯眉批:“不能表白”后是第十八回的起头。\end{note}
\end{parag}


\begin{parag}
    当下又人回,工程上等著糊东西的纱绫,请凤姐去开楼拣纱绫;又有人来回,请凤姐开库,收金银器皿。连王夫人并上房丫鬟等众,皆一时不得闲的。宝钗便说:“咱们别在这里碍手碍脚,找探丫头去。”说著,同宝玉黛玉往迎春等房中来闲顽,无话。
\end{parag}


\begin{parag}
    王夫人等日日忙乱,直到十月将尽,幸皆全备:各处监管都交清帐目;各处古董文玩,皆已陈设齐备;采办鸟雀的,自仙鹤、孔雀以及鹿、兔、鸡、鹅等类,悉已买全,交于园中各处像景饲养;贾蔷那边也演出二十出杂戏来;小尼姑、道姑也都学会了念几卷经咒。贾政方略心意宽畅,\begin{note}蒙双行夹批:好极!可见智者心无一时痴怠!\end{note}又请贾母等进园,色色斟酌,点缀妥当,再无一些遗漏不当之处了。于是贾政方择日题本。\begin{note}蒙双行夹批:至此方完大观园工程公案,观者则为大观园费尽精神,余则为若笔墨却只因一个葬花冢。\end{note}本上之日,奉朱批准奏:次年正月十五日上元之日,恩准贵妃省亲。贾府领了此恩旨,益发昼夜不闲,年也不曾好生过的。\begin{note}庚辰双行夹批:一语带过。是以“岁首祭宗祀,元宵开夜宴”一回留在后文细写。\end{note}
\end{parag}


\begin{parag}
    展眼元宵在迩,自正月初八日,就有太监出来先看方向:何处更衣,何处燕坐,何处受礼,何处开宴,何处退息。又有巡察地方总理关防太监等,带了许多小太监出来,各处关防,挡围幕,指示贾宅人员何处退,何处跪,何处进膳,何处启事,种种仪注不一。外面又有工部官员并五城兵备道打扫街道,撵逐闲人。贾赦等督率匠人扎花灯烟火之类,至十四日,俱已停妥。这一夜,上下通不曾睡。
\end{parag}


\begin{parag}
    至十五日五鼓,自贾母等有爵者,俱各按品服大妆。园内各处,帐舞龙蟠,帘飞彩凤,金银焕彩,珠宝争辉,\begin{note}庚辰双行夹批:是元宵之夕,不写灯月而灯光月色满纸矣。\end{note}鼎焚百合之香,瓶插长春之蕊,\begin{note}庚辰双行夹批:抵一篇大赋。\end{note}静悄无人咳嗽。\begin{note}庚辰双行夹批:有此句方足。\end{note}贾赦等在西街门外,贾母等在荣府大门外。街头巷口,俱系围幕挡严。正等的不耐烦,忽一太监坐大马而来,\begin{note}庚辰双行夹批:有是理。\end{note}贾母忙接入,问其消息。太监道:“早多著呢!未初刻用过晚膳,未正二刻还到宝灵宫拜佛,\begin{note}庚辰双行夹批:暗贴王夫人,细。\end{note}酉初刻进太明宫领宴看灯方请旨,只怕戍初才起身呢。”凤姐听了道:\begin{note}庚辰侧批:自然当家人先说话。\end{note}“既是这么著,老太太、太太且请回房,等是时候再来也不迟。”于是贾母等暂且自便,园中悉赖凤姐照理。又命执事人带领太监们去吃酒饭。
\end{parag}


\begin{parag}
    一时传人一担一担的挑进蜡烛来,各处点灯。方点完时,忽听外边马跑之声。\begin{note}庚辰双行夹批:静极故闻之。细极。\end{note} 一时,有十来个太监都喘吁吁跑来拍手儿。\begin{note}庚辰双行夹批:画出内家风范。《石头记》最难之处别书中摸不著。\end{note}这些太监会意,\begin{note}庚辰侧批:难得他写的出,是经过之人也。\end{note}都知道是“来了,来了”,各按方向站住。贾赦领合族子侄在西街门外,贾母领合族女眷在大门外迎接。半日静悄悄的。忽见一对红衣太监骑马缓缓的走来,\begin{note}庚辰双行夹批:形容毕肖。\end{note}至西街门下了马,将马赶出围幕之外,便垂手面西站住。\begin{note}庚辰双行夹批:形容毕肖。\end{note}半日又是一对,亦是如此。少时便来了十来对,方闻得隐隐细乐之声。一对对龙旌凤翣,雉羽夔头,又有销金提炉焚著御香;然后一把曲柄七凤金黄伞过来,便是冠袍带履。又有值事太监捧著香珠、绣帕、漱盂、拂尘等类。一队队过完,后面方是八个太监抬著一顶金顶金黄绣凤版舆,缓缓行来。贾母等连忙路旁跪下。 \begin{note}庚辰侧批:一丝不乱。\end{note}早飞跑过几个太监来,扶起贾母、邢夫人、王夫人来。那版舆抬进大门、入仪门往东去,到一所院落门前,有执拂太监跪请下舆更衣。于是抬舆入门,太监等散去,只有昭容、彩嫔等引领元春下舆。只见院内各色花灯熌灼,\begin{note}庚辰侧批:元春月中。 \end{note}皆系纱绫扎成,精致非常。上面有一匾灯,写著“体仁沐德”四字。元春入室,更衣毕复出,上舆进园。只见园中香烟缭绕,花彩缤纷,处处灯光相映,时时细乐声喧,说不尽这太平景象,富贵风流。——此时自己回想当初在大荒山中,青埂峰下,那等凄凉寂寞;若不亏癞僧、跛道二人携来到此,又安能得见这般世面。本欲作一篇《灯月赋》、《省亲颂》,以志今日之事,但又恐入了别书的俗套。按此时之景,即作一赋一赞,也不能形容得尽其妙;即不作赋赞,其豪华富丽,观者诸公亦可想而知矣。所以倒是省了这工夫纸墨,且说正经的为是。\begin{note}庚辰双行夹批:自“此时”以下皆石头之语,真是千奇百怪之文。\end{note}\begin{note}庚辰眉批:如此繁华盛极花团锦簇之文忽用石兄自语截住,是何笔力!令人安得不拍案叫绝。试阅历来诸小说中有如此章法乎?\end{note}
\end{parag}


\begin{parag}
    且说贾妃在轿内看此园内外如此豪华,因默默叹息奢华过费。忽又见执拂太监跪请登舟。贾妃乃下舆。只见清流一带,势若游龙,两边石栏上,皆系水晶玻璃各色风灯,点的如银光雪浪;上面柳杏诸树虽无花叶,然皆用通草绸绫纸绢依势作成,粘于枝上的,每一株悬灯数盏;更兼池中荷荇凫鹭之属,亦皆系螺蚌羽毛之类作就的。诸灯上下争辉,真系玻璃世界,珠宝乾坤。船上亦系各种精致盆景诸灯,珠帘绣幕,桂楫兰桡,自不必说。已而入一石港,港上一面匾灯,明现著“蓼汀花溆” 四字。按此四字,并“有凤来仪”等处,皆系上回贾政偶然一试宝玉之课艺才情耳,何今日认真用此匾联?况贾政世代诗书,来往诸客屏侍坐陪者,悉皆才技之流,岂无一名手题撰,竟用小儿一戏之辞苟且搪塞?\begin{note}庚辰眉批:驳得好!\end{note}真似暴发新荣之家,滥使银钱,一味抹油涂朱,毕则大书“前门绿柳垂金锁,后户青山列锦屏”之类,则以为大雅可观,岂《石头记》中通部所表之宁荣贾府所为哉!据此论之,竟大相矛盾了。\begin{note}庚辰双行夹批:石兄自谦,妙!可代答云“岂敢!”\end{note}将原委说明,大家方知。\begin{note}庚辰眉批:《石头记》惯用特犯不犯之笔,读之真令人惊心骇目。\end{note}
\end{parag}


\begin{parag}
    当日这贾妃未入宫时,自幼亦系贾母教养。后来添了宝玉,贾妃乃长姊,宝玉为弱弟,贾妃之心上念母年将迈,始得此弟,是以怜爱宝玉,与诸弟待之不同。且同随贾母,刻未离。那宝玉未入学堂之先,三四岁时,已得贾妃手引口传,教授了几本书、数千字在腹内了。\begin{note}庚辰侧批:批书人领过此教,故批至此竟放声大哭,俺先姊仙逝太早,不然余何得为废人耶?\end{note}其名分虽系姊弟,其情状有如母子。自入宫后,时时带信出来与父母说:“千万好生扶养,不严不能成器,过严恐生不虞,且致父母之忧。”眷念切爱之心,刻未能忘。前日贾政闻塾师背后赞宝玉偏才尽有,贾政未信,适巧遇园已落成,令其题撰,聊一试其情思之清浊。其所拟之匾联虽非妙句,在幼童为之,亦或可取。即另使名公大笔为之,固不费难,然想来倒不如这本家风味有趣。\begin{note}庚辰侧批:转得好。\end{note}更使贾妃见之,知系其爱弟所为,亦或不负其素日切望之意。\begin{note}庚辰侧批:有是论。\end{note}\begin{note}庚辰双行夹批:一驳一解,跌宕摇曳,且写得父母兄弟体贴恋爱之情,淋漓痛切,真是天伦至情。\end{note}因有这段原委,故此竟用了宝玉所题之联额。那日虽未曾题完,后来亦曾补拟。\begin{note}庚辰双行夹批:一句补前文之不暇,启后文之苗裔。至后文凹晶馆黛玉口中又一补,所谓“一击空谷,八方皆应”。\end{note}
\end{parag}


\begin{parag}
    闲文少叙,且说贾妃看了四字,笑道:“‘花溆’二字便妥,何必‘蓼汀’?”侍坐太监听了,忙下小舟登岸,飞传与贾政。贾政听了,即忙移换。\begin{note}庚辰双行夹批:每的周到可悦。\end{note}一时,舟临内岸,复弃舟上舆,便见琳宫绰约,桂殿巍峨。石牌坊上明显“天仙宝镜”四字,\begin{note}庚辰双行夹批:不得不用俗。\end{note}贾妃忙命换“省亲别墅”四字。\begin{note}庚辰双行夹批:妙!是特留此四字与彼自命。\end{note}于是进入行宫。但见庭燎烧空,\begin{note}庚辰双行夹批:庭燎最俗。\end{note}香屑布地,火树琪花,金窗玉槛。说不尽帘卷虾须,毯铺鱼獭,鼎飘麝脑之香,屏列雉尾之扇。真是:
\end{parag}
\begin{poem}
    \begin{pl}金门玉户神仙府,桂殿兰宫妃子家。\end{pl}
\end{poem}
\begin{parag}
    贾妃乃问:“此殿何无匾额?”随侍太监跪启曰:“此系正殿,外臣未敢擅拟。”贾妃点头不语。礼仪太监跪请升座受礼,两陛乐起。礼仪太监二人引贾赦、贾政等于月台下排班,殿上昭容传谕曰:“免。”太监引贾赦等退出。又有太监引荣国太君及女眷等自东阶升月台上排班,\begin{note}庚辰双行夹批:一丝不乱,精致大方。有如欧阳公九九。\end{note}昭容再谕曰:“免。”于是引退。
\end{parag}


\begin{parag}
    茶已三献,贾妃降座,乐止。退入侧殿更衣,方备省亲车驾出园。至贾母正室,欲行家礼,贾母等俱跪止不迭。贾妃满眼垂泪,方彼此上前厮见,一手搀贾母,一手搀王夫人,三个人满心里皆有许多话,只是俱说不出,只管呜咽对泪。\begin{note}庚辰双行夹批:《石头记》得力擅长全是此等地方。庚辰眉批:非经历过如何写得出!壬午春。\end{note}邢夫人、李纨、王熙凤、迎、探、惜三姊妹等,俱在旁围绕,垂泪无言。半日,贾妃方忍悲强笑,安慰贾母、王夫人道:“当日既送我到那不得见人的去处,好容易今日回家娘儿们一会,不说说笑笑,反倒哭起来。一会子我去了,又不知多早晚才来!”说到这句,不觉又哽咽起来。\begin{note}庚辰双行夹批:追魂摄魄,《石头记》传神摸影全在此等地方,他书中不得有此见识。\end{note}邢夫人忙上来解劝。\begin{note}庚辰双行夹批:说完不可,不先说不可,说之不痛不可,最难说者是此时贾妃口中之语。只如此一说,千贴万妥,一字不可更改,一字不可增减,入情入神之至!\end{note}贾母等让贾妃归座,又逐次一一见过,又不免哭泣一番。然后东西两府掌家执事人丁等在厅外行礼,及两府掌家执事媳妇领丫鬟等行礼毕。贾妃因问:“薛姨妈、宝钗、黛玉因何不见?”王夫人启曰:“外眷无职,未敢擅入。”\begin{note}庚辰双行夹批:所谓诗书世家,守礼如此。偏是暴发,骄妄自大。\end{note}贾妃听了,忙命快请。\begin{note}庚辰双行夹批:又谦之如此,真是好界好人物。\end{note}一时薛姨妈等进来,欲行国礼,亦命免过,上前各叙阔别寒温。又有贾妃原带进宫去的丫鬟抱琴等\begin{note}庚辰双行夹批:前所谓贾家四钗之鬟暗以琴棋书画排行,至此始全。\end{note}上来叩见,贾母等连忙扶起,命人别室款待。执事太监及彩嫔、昭容各侍从人等,宁国府及贾赦那宅两处自有人款待,只留三四个小太监答应。母女姊妹深叙些离别情景,\begin{note}庚辰双行夹批:“深”字妙!\end{note}及家务私情。
\end{parag}


\begin{parag}
    又有贾政至帘外问安,贾妃垂帘行参拜等事。又隔帘含泪谓其父曰:“田舍之家,虽齑盐布帛,终能聚天伦之乐;今虽富贵已极,骨肉各方,然终无意趣!”贾政亦含泪启道:“臣,草莽寒门,鸠群鸦属之中,岂意得征凤鸾之瑞。\begin{note}庚辰侧批:此语犹在耳。\end{note}今贵人上锡天恩,下昭祖德,此皆山川日月之精奇、祖宗之远德钟于一人,幸及政夫妇。且今上启天地生物之大德,垂古今未有之旷恩,虽肝脑涂地,臣子岂能得报于万一!惟朝乾夕惕,忠于厥职外,愿我君万寿千秋,乃天下苍生之同幸也。贵妃切勿以政夫妇残年为念,懑愤金怀,更祈自加珍爱。惟业业兢兢,勤慎恭肃以侍上,庶不负上体贴眷爱如此之隆恩也。”贾妃亦嘱“只以国事为重,暇时保养,切勿记念”等语。贾政又启:“园中所有亭台轩馆,皆系宝玉所题;如果有一二稍可寓目者,请别赐名为幸。”元妃听了宝玉能题,便含笑说:“果进益了。”贾政退出。贾妃见宝、林二人亦发比别姊妹不同,真是姣花软玉一般。因问:“宝玉为何不进见?”\begin{note}庚辰双行夹批:至此方出宝玉。\end{note}贾母乃启:“无谕,外男不敢擅入。”元妃命快引进来。小太监出去引宝玉进来,先行国礼毕,元妃命他进前,携手拦揽于怀内,又抚其头颈,\begin{note}庚辰侧批:作书人将批书人哭坏了。\end{note}笑道:“比先竟长了好些……”一语未终,泪如雨下。\begin{note}庚辰双行夹批:至此一句便补足前面许多文字。\end{note}
\end{parag}


\begin{parag}
    尤氏、凤姐等上来启道:“筵宴齐备,请贵妃游幸。”元妃等起身,命宝玉导引,遂同诸人步至园门前。早见灯光火树之中,诸般罗列非常。进园来先从“有凤来仪”、“红香绿玉”、“杏帘在望”、“蘅芷清芬”等处,登楼步阁,涉水缘山,百般眺览徘徊。一处处铺陈不一,一桩桩点缀新奇。贾妃极加奖赞,又劝:“以后不可太奢,此皆过分之极。”已而至正殿,谕免礼归座,大开筵宴。贾母等在下相陪,尤氏、李纨、凤姐等亲捧羹把盏。
\end{parag}


\begin{parag}
    元妃乃命传笔砚伺候,亲搦湘管,择其几处最喜者赐名。按其书云:
\end{parag}
\begin{qute}
    \begin{parag}
        “顾恩思义”匾额\newline
        天地启宏慈,赤子苍头同感戴;\newline
        古今垂旷典,九州万国被恩荣。\newline
        此一匾一联书于正殿。\begin{note}庚辰双行夹批:是贵妃口气。\end{note}
    \end{parag}

    \begin{parag}
        “大观园”园之名\newline
        “有凤来仪”赐名曰“潇湘馆”。 \newline
        “红香绿玉”改作“怡红快绿”。即名曰“怡红院”。\newline
        “蘅芷清芳”赐名曰“蘅芜苑”。 \newline
        “杏帘在望”赐名曰“浣葛山庄”。 \newline
    \end{parag}
\end{qute}
\begin{parag}
    正楼曰“大观楼”,东面飞楼曰“缀锦阁”,西面斜楼曰“含芳阁”;更有“蓼风轩”、“藕香榭”、\begin{note}庚辰双行夹批:雅而新。\end{note}“紫菱洲”、“ 叶渚”等名;又有四字的匾额十数个,诸如“梨花春雨”、“桐剪秋风”、“荻芦夜雪”等名,此时悉难全记。\begin{note}庚辰双行夹批:故意留下秋爽斋、凸碧山堂、凹晶溪馆、暖香坞等处为后文另换眼目之地步。\end{note}又命旧有匾联者俱不必摘去。于是先题一绝云:
\end{parag}
\begin{poem}

    \begin{pl}衔山抱水建来精,\end{pl}

    \begin{pl}多少工夫筑始成。\end{pl}

    \begin{pl}天上人间诸景备,\end{pl}

    \begin{pl}芳园应锡大观名。\end{pl}
    \begin{note}庚辰双行夹批:诗却平平,盖彼不长于此也,故只如此。\end{note}
\end{poem}

\begin{parag}
    写毕,向诸姐妹笑道:“我素乏捷才,且不长于吟咏,妹辈素所深知。今夜聊以塞责,不负斯景而已。异日少暇,必补撰《大观园记》并《省亲颂》等文,以记今日之事。妹辈亦各题一匾一诗,随才之长短,亦暂吟成,不可因我微才所缚。且喜宝玉竟知题咏,是我意外之想。此中‘潇湘馆’、‘蘅芜院’二处,我所极爱,次之‘怡红院’、‘浣葛山庄’,此四大处,必得别有章句题咏方妙。前所题之联虽佳,如今再各赋五言律一首,使我当面试过,方不负我自幼教授之苦心。”宝玉只得答应了,下来自去构思。
\end{parag}


\begin{parag}
    迎、探、惜三人之中,要算探春又出于姊妹之上,然自忖亦难与薛林争衡,\begin{note}庚辰双行夹批:只一语便写出宝黛二人,又写出探卿知己知彼,伏下后文多少地步。\end{note}只得勉强随众塞责而已。李纨也勉强凑成一律。\begin{note}庚辰双行夹批:不表薛林可知。\end{note}贾妃先挨次看姊妹们的,写道是:
\end{parag}


\begin{poem}
    \begin{pl}旷性怡情匾额 迎春 \end{pl}

    \begin{pl}园成景备特精奇, \end{pl}

    \begin{pl}奉命羞题额旷怡。 \end{pl}

    \begin{pl}谁信世间有此景, \end{pl}

    \begin{pl}游来宁不畅神思? \end{pl}
\end{poem}
\begin{poem}
    \begin{pl}万象争辉匾额 探春 \end{pl}

    \begin{pl}名园筑出势巍巍, \end{pl}

    \begin{pl}奉命何惭学浅微。 \end{pl}

    \begin{pl}精妙一时言不出, \end{pl}

    \begin{pl}果然万物有光辉。 \end{pl}
\end{poem}
\begin{poem}
    \begin{pl}文章造化匾额 惜春 \end{pl}

    \begin{pl}山水横拖千里外, \end{pl}

    \begin{pl}楼台高起五云中。 \end{pl}

    \begin{pl}园修日月光辉里, \end{pl}

    \begin{pl}景夺文章造化功 \end{pl} \begin{note}庚辰双行夹批:更牵强。三首之中还算探卿略有作意,故后文写出许多意外妙文。\end{note}
\end{poem}
\begin{poem}
    \begin{pl}文采风流匾额 李纨\end{pl}

    \begin{pl}秀水明山抱复回,\end{pl}

    \begin{pl}风流文采胜蓬莱。    \end{pl}      \begin{note}庚辰双行夹批:超妙!\end{note}

    \begin{pl}绿裁歌扇迷芳草,\end{pl}

    \begin{pl}红衬湘裙舞落梅。    \end{pl}      \begin{note}庚辰双行夹批:凑成。\end{note}

    \begin{pl}珠玉自应传盛世,\end{pl}

    \begin{pl}神仙何幸下瑶台。\end{pl}

    \begin{pl}名园一自邀游赏,\end{pl}

    \begin{pl}未许凡人到此来。    \end{pl}      \begin{note}庚辰双行夹批:此四诗列于前正为滃托下韵也。\end{note}

\end{poem}

\begin{poem}

    \begin{pl}凝晖钟瑞匾额      \end{pl}        \begin{note}庚辰双行夹批:便又含蓄。\end{note} \begin{pl}薛宝钗\end{pl}

    \begin{pl}芳园筑向帝城西,\end{pl}

    \begin{pl}华日祥云笼罩奇。\end{pl}

    \begin{pl}高柳喜迁莺出谷,\end{pl}

    \begin{pl}修篁时待凤来仪。    \end{pl}      \begin{note}庚辰双行夹批:恰极!\end{note}

    \begin{pl}文风已着宸游夕,\end{pl}

    \begin{pl}孝化应隆遍省时。\end{pl}

    \begin{pl}睿藻仙才盈彩笔,\end{pl}

    \begin{pl}自惭何敢再为辞?    \end{pl}      \begin{note}庚辰双行夹批:好诗!此不过颂圣应酬耳,未见长,以后渐知。\end{note}

\end{poem}

\begin{poem}

    \begin{pl}世外仙园匾额      \end{pl}        \begin{note}庚辰双行夹批:落思便不与人同。\end{note}\begin{pl}林黛玉\end{pl}

    \begin{pl}名园筑何处,\end{pl}

    \begin{pl}仙境别红尘。\end{pl}

    \begin{pl}借得山川秀,\end{pl}

    \begin{pl}添来景物新。      \end{pl}        \begin{note}庚辰双行夹批:所谓「信手拈来无不是」,阿颦自是一种心思。\end{note}

    \begin{pl}香融金谷酒,\end{pl}

    \begin{pl}花媚玉堂人。\end{pl}

    \begin{pl}何幸邀恩宠,\end{pl}

    \begin{pl}宫车过往频?      \end{pl}    \begin{note}庚辰双行夹批:末二首是应制诗。余谓宝林二作未见长,何也?该后文别有惊人之句也。在宝卿有不屑为此,在黛卿实不足一为。\end{note}

\end{poem}

\begin{parag}
    贾妃看毕,称赏一番,又笑道:“终是薛林二妹之作与众不同,非愚姊妹可同列者。”原来林黛玉安心今夜大展奇才,将众人压倒,\begin{note}庚辰双行夹批:这却何必,然尤物方如此。\end{note}不想贾妃只命一匾一咏,倒不好违谕多作,只胡乱作一首五律应景罢了。\begin{note}庚辰双行夹批:请看前诗,却云是胡乱应景。\end{note}
\end{parag}


\begin{parag}
    彼时宝玉尚未作完,只刚做了“潇湘馆”与“蘅芜苑”二首,正作“怡红院”一首,起草内有“绿玉春犹卷”一句。宝钗转眼瞥见,便趁众人不理论,急忙回身悄推他道:“他\begin{note}庚辰双行夹批:此“他”字指贾妃。\end{note}因不喜‘红香绿玉’四字,改了‘怡红快绿’;你这会子偏用‘绿玉’二字,岂不是有意和他争驰了?况且蕉叶之说也颇多,再想一个改了罢。”宝玉见宝钗如此说,便拭汗说道:\begin{note}庚辰双行夹批:想见其构思之苦方是至情。最厌近之小说中满纸“神童”“天分”等语。\end{note}“我这会子总想不起什么典故出处来。”宝钗笑道:“你只把‘绿玉’的‘玉’字改作‘蜡’字就是了。”宝玉道:“‘绿蜡’\begin{note}庚辰侧批:好极!\end{note}可有出处?”宝钗见问,悄悄的咂嘴点头\begin{note}庚辰侧批:媚极!韵极!\end{note}笑道:“亏你今夜不过如此,将来金殿对策,你大约连‘赵钱孙李’都忘了呢!\begin{note}庚辰双行夹批:有得宝卿奚落,但就谓宝卿无情,只是较阿颦施之特正耳。\end{note}唐钱珝咏芭蕉诗头一句‘冷烛无烟绿蜡干’,你都忘了不成?”\begin{note}庚辰双行夹批:此等处便是用硬证实处,最是大力量,但不知是何心思,是从何落思,穿插到此玲珑锦绣地步。庚辰眉批:如此章法又是不曾见过的。如此穿插安得不令人拍案叫绝!壬午季春。\end{note}宝玉听了,不觉洞开心臆,笑道:“该死,该死!现成眼前之物偏倒想不起来了,真可谓‘一字师’了。从此后我只叫你师父,再不叫姐姐了。”宝钗亦悄悄的笑道:“还不快作上去,只管姐姐妹妹的。谁是你姐姐?那上头穿黄袍的才是你姐姐,你又认我这姐姐来了。”一面说笑,因说笑又怕他耽延工夫,遂抽身走开了。\begin{note}庚辰双行夹批:一段忙中闲文,已是好看之极,出人意外。\end{note}宝玉只得续成,共有了三首。
\end{parag}


\begin{parag}
    此时林黛玉未得展其抱负,自是不快。因见宝玉独作四律,大费神思,何不代他作两首,也省他些精神不到之处。\begin{note}庚辰双行夹批:写黛玉之情思,待宝玉却又如此,是与前文特犯不犯之处。庚辰眉批:偏又写一样,是何心意构思而得?畸笏。\end{note}想著,便也走至宝玉案旁,悄问:“可都有了?”宝玉道:“才有了三首,只少‘杏帘在望’一首了。”黛玉道:“既如此,你只抄录前三首罢。赶你写完那三首,我也替你作出这首了。”说毕,低头一想,早已吟成一律,\begin{note}庚辰双行夹批:瞧他写阿颦只如此便妙极。\end{note}便写在纸条上,搓成个团子,掷在他跟前。\begin{note}庚辰眉批:纸条送迭系童生秘诀,黛卿自何处学得?一笑。丁亥春。\end{note}宝玉打开一看,只觉此首比自己所作的三首高过十倍,真是喜出望外,\begin{note}庚辰双行夹批:这等文字亦是观书者望外之想。\end{note}遂忙恭楷呈上。贾妃看道:
\end{parag}
\begin{poem}

    \begin{pl}有凤来仪臣宝玉谨题\end{pl}

    \begin{pl}秀玉初成实,\end{pl}

    \begin{pl}堪宜待凤凰。\end{pl}\begin{note}庚辰双行夹批:起便拿得住。\end{note}

    \begin{pl}竿竿青欲滴,\end{pl}

    \begin{pl}个个绿生凉。\end{pl}

    \begin{pl}迸砌防阶水,\end{pl}

    \begin{pl}穿帘碍鼎香。\end{pl}\begin{note}庚辰双行夹批:妙句!古云:「竹密何妨水过?」,今偏翻案。\end{note}

    \begin{pl}莫摇清碎影,\end{pl}

    \begin{pl}好梦昼初长。\end{pl}

\end{poem}

\begin{poem}

    \begin{pl}蘅芷清芬\end{pl}

    \begin{pl}蘅芜满净苑,\end{pl}

    \begin{pl}萝薜助芬芳。\end{pl}\begin{note}庚辰双行夹批:「助」字妙!通部书所以皆善炼字。\end{note}

    \begin{pl}软衬三春草,\end{pl}

    \begin{pl}柔拖一缕香。\end{pl}\begin{note}庚辰双行夹批:刻画入妙。\end{note}

    \begin{pl}轻烟迷曲径,\end{pl}

    \begin{pl}冷翠滴回廊。\end{pl}\begin{note}庚辰双行夹批:甜脆满颊。\end{note}

    \begin{pl}谁谓池塘曲,\end{pl}

    \begin{pl}谢家幽梦长。\end{pl}

\end{poem}

\begin{poem}

    \begin{pl}怡红快绿\end{pl}

    \begin{pl}深庭长日静,\end{pl}

    \begin{pl}两两出婵娟。\end{pl}\begin{note}庚辰双行夹批:双起双敲,读此首始信前云「有蕉无棠不可,有棠无蕉更不可」等批非泛泛妄批驳他人,到自己身上则无能为之论也。\end{note}

    \begin{pl}绿蜡\end{pl}\begin{note}庚辰双行夹批:本是「玉」字,此尊宝卿改,似较「玉」字佳。\end{note}\begin{pl}春犹卷,\end{pl}\begin{note}庚辰双行夹批:是蕉。\end{note}

    \begin{pl}红妆夜未眠。\end{pl}\begin{note}庚辰双行夹批:是海棠。\end{note}

    \begin{pl}凭栏垂绛袖,\end{pl}\begin{note}庚辰双行夹批:是海棠之情。\end{note}

    \begin{pl}倚石护青烟。\end{pl}\begin{note}庚辰双行夹批:是芭蕉之神。何得如此工恰自然?真是好诗,却是好书。\end{note}

    \begin{pl}对立东风里,\end{pl}\begin{note}庚辰双行夹批:双收。\end{note}

    \begin{pl}主人应解怜。\end{pl}\begin{note}庚辰双行夹批:归到主人方不落空。王梅隐云:「咏物体又难双承双落,一味双拿则不免牵强。」此首可谓诗题两称,极工、极切、极流利妩媚。\end{note}

\end{poem}

\begin{poem}

    \begin{pl}杏帘在望\end{pl}

    \begin{pl}杏帘招客饮,\end{pl}

    \begin{pl}在望有山庄。\end{pl}\begin{note}庚辰双行夹批:分题作一气呵成,格调熟练,自是阿颦口气。\end{note}

    \begin{pl}菱荇鹅儿水,\end{pl}

    \begin{pl}桑榆燕子梁。\end{pl}\begin{note}庚辰双行夹批:阿颦之心臆才情原与人别,亦不是从读书中得来。\end{note}

    \begin{pl}一畦春韭熟,\end{pl}

    \begin{pl}十里稻花香。\end{pl}

    \begin{pl}盛世无饥馁,\end{pl}

    \begin{pl}何须耕织忙。\end{pl}\begin{note}庚辰双行夹批:以幻入幻,顺水推舟,且不失应制,所以称阿颦。\end{note}

\end{poem}
\begin{parag}
    贾妃看毕,喜之不尽,说:“果然进益了!”又指“杏帘”一首为前三首之冠。遂将“浣葛山庄”改为“稻香村”。\begin{note}庚辰双行夹批:如此服善,妙!庚辰眉批:仍用玉兄前拟“稻香村”,却如此幻笔幻体,文章之格式至矣尽矣!壬午春。\end{note}又命探春另以彩笺誊录出方才一共十数首诗,出令太监传与外厢。贾政等看了,都称颂不已。贾政又进《归省颂》。元妃又命以琼酥金脍等物,赐与宝玉并贾兰。\begin{note}庚辰双行夹批:忙中点出贾兰,一人不落。\end{note}此时贾兰极幼,未达诸事,只不过随母依叔行礼,故无别传。贾环从年内染病未痊,自有闲处调养,故亦无传。\begin{note}庚辰双行夹批:补明方不遗失。\end{note}
\end{parag}


\begin{parag}
    那时贾蔷带领十二个女戏,在楼下正等的不耐烦,只见一太监飞来说:“作完了诗,快拿戏目来!”贾蔷急将锦册呈上,并十二个花名单子。少时,太监出来,只点了四出戏:
\end{parag}


\begin{parag}
    第一出《豪宴》;\begin{note}庚辰双行夹批:《一捧雪》中伏贾家之败。\end{note}
\end{parag}


\begin{parag}
    第二出《乞巧》;\begin{note}庚辰双行夹批:《长生殿》中伏元妃之死。\end{note}
\end{parag}


\begin{parag}
    第三出《仙缘》;\begin{note}庚辰双行夹批:《邯郸梦》中伏甄宝玉送玉。\end{note}
\end{parag}


\begin{parag}
    第四出《离魂》。\begin{note}庚辰双行夹批:《牡丹亭》中伏黛玉死。所点之戏剧伏四事,乃通部书之大过节、大关键。\end{note}
\end{parag}


\begin{parag}
    贾蔷忙张罗扮演起来。一个个歌欺裂石之音,舞有天魔之态。虽是妆演的形容,却作尽悲欢情状。\begin{note}庚辰双行夹批:二句毕矣。\end{note}刚演完了,一太监执一金盘糕点之属进来,问:“谁是龄官?”贾蔷便知是赐龄官之物,喜的忙接了,\begin{note}庚辰双行夹批:何喜之有?伏下后面许多文字只用一“喜”字。\end{note}命龄官叩头。太监又道:“贵妃有谕,说:‘龄官极好,再作两出戏,不拘那两出就是了。’”贾蔷忙答应了,因命龄官做《游园》、《惊梦》二出。龄官自为此二出原非本角之戏,执意不作,定要作《相约》《相骂》二出。\begin{note}庚辰双行夹批:《钗钏记》中总隐后文不尽风月等文。\end{note}\begin{note}庚辰双行夹批:按近之俗语云:“宁养千军,不养一戏。”盖甚言优伶之不可养之意也。大抵一班之中此一人技业稍出众,此一人则拿腔作势、辖众恃能种种可恶,使主人逐之不舍责之不可,虽欲不怜而实不能不怜,虽欲不爱而实不能不爱。余历梨园弟子广矣,个个皆然,亦曾与惯养梨园诸世家兄弟谈议及此,众皆知其事而皆不能言。今阅《石头记》至“原非本角之戏,执意不作”二语,便见其恃能压众、乔酸娇妒,淋漓满纸矣。复至“情悟梨香院”一回更将和盘托出,与余三十年前目睹身亲之人现形于纸上。使言《石头记》之为书,情之至极、言之至恰,然非领略过乃事、迷蹈过乃情,即观此,茫然嚼蜡,亦不知其神妙也。\end{note}贾蔷扭他不过,\begin{note}庚辰双行夹批:如何反扭他不过?其中隐许多文字。\end{note}只得依他作了。贾妃甚喜,命“不可难为了这女孩子,好生教习”,\begin{note}庚辰双行夹批:可知尤物了。\end{note}额外赏了两匹宫缎、两个荷包并金银锞子、食物之类。\begin{note}庚辰双行夹批:有伏下一个尤物,一段新文。\end{note}然后撤筵,将未到之处复又游顽。忽见山环佛寺,忙另盥手进去焚香拜佛,又题一匾云:“苦海慈航”。\begin{note}庚辰双行夹批:写通部人事一篇热文,却如此冷收。\end{note}又额外加恩与一班幽尼女道。
\end{parag}


\begin{parag}
    少时,太监跪启:“赐物俱齐,请验等例。”乃呈上略节。贾妃从头看了,俱甚妥协,即命照此遵行。太监听了,下来一一发放。原来贾母的是金、玉如意各一柄,沉香拐拄一根,伽楠念珠一串,“富贵长春”宫缎四匹,“福寿绵长”宫绸四匹,紫金“笔锭如意”锞十锭,“吉庆有鱼”银锞十锭。邢夫人、王夫人二分,只减了如意、拐、珠四样。贾敬、贾赦、贾政等,每分御制新书二部,宝墨二匣,金、银爵各二支,表礼按前。宝钗、黛玉诸姊妹等,每人新书一部,宝砚一方,新样格式金银锞二对。宝玉亦同此。\begin{note}庚辰双行夹批:此中忽夹上宝玉,可思。\end{note}贾兰则是金银项圈二个,金银锞二对。尤氏、李纨、凤姐等,皆金银锞四锭,表礼四端。外表礼二十四端,清钱一百串,是赐与贾母、王夫人及诸姊妹房中奶娘众丫鬟的。贾珍、贾琏、贾环、贾蓉等,皆是表礼一分,金锞一双。其余彩缎百端,金银千两,御酒华筵,是赐东西两府凡园中管理工程、陈设、答应及司戏、掌灯诸人的。外有清钱五百串,是 统 役、优伶、百戏、杂行人丁的。
\end{parag}


\begin{parag}
    众人谢恩已毕,执事太监启道:“时已丑正三刻,请驾回銮。”贾妃听了,不由的满眼又滚下泪来。却又勉强堆笑,拉住贾母、王夫人的手,紧紧的不忍释放,\begin{note}庚辰双行夹批:使人鼻酸。\end{note}再四叮咛:“不须记挂,好生自养。如今天恩浩荡,一月许进内省视一次,见面是尽有的,何必伤惨。倘明岁天恩仍许归省,万不可如此奢华靡费了。”\begin{note}庚辰双行夹批:妙极之谶,试看别书中专能故用一不祥之语为谶?今偏不然,只有如此现成一语,便是不再之谶,只看他用一“倘”字便隐讳,自然之至。\end{note}贾母等已哭的哽噎难言。贾妃虽不忍别,怎奈皇家规范,违错不得,只得忍心上舆去了。这里诸人好容易将贾母、王夫人安慰解劝,搀扶出园去了。\begin{note}庚辰眉批:一回离合悲欢夹写之文,正如山阴道上令人应接不暇,尚有许多忙中闲、闲中忙小波澜,一丝不漏,一笔不苟。\end{note}
\end{parag}


\begin{parag}
    \begin{note}蒙回末总批:此回铺陈,非身经历,开巨眼,伸文笔,则必有所滞罣牵强,岂能如此触处成趣,立后文之根,足本文之情者。且借象说法,学我佛开经,代天女散花,已成此奇文妙趣,惟不得与四才子书之作者同时讨论臧否,为可恨恨耳。\end{note}
\end{parag}