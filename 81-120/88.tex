\chap{八十八}{博庭歡寶玉贊孤兒 正家法賈珍鞭悍僕}



\begin{parag}
    卻說惜春正在那裏揣摩棋譜,忽聽院內有人叫彩屏,不是別人卻是鴛鴦的聲兒。彩屏出去,同着鴛鴦進來。那鴛鴦卻帶着一個小丫頭,提了一個小黃絹包兒。惜春笑問道:“什麼事?”鴛鴦道:“老太太因明年八十一歲,是個暗九。許下一場九晝夜的功德,發心要寫三千六百五十零一部《金剛經》。這已發出外面人寫了。但是俗說《金剛經》就象那道家的符殼,《心經》纔算是符膽。故此《金剛經》內必要插着《心經》,更有功德。老太太因《心經》是更要緊的,觀自在又是女菩薩,所以要幾個親丁奶奶姑娘們寫上三百六十五部,如此又虔誠,又潔淨。咱們家中除了二奶奶,頭一宗他當家沒有空兒,二宗他也寫不上來,其餘會寫字的,不論寫得多少,連東府珍大奶奶姨娘們都分了去,本家裏頭自不用說。”惜春聽了,點頭道:“別的我做不來,若要寫經,我最信心的。你擱下喝茶罷。”鴛鴦纔將那小包兒擱在桌上,同惜春坐下。彩屏倒了一鍾茶來。惜春笑問道:“你寫不寫?”鴛鴦道:“姑娘又說笑話了。那幾年還好,這三四年來姑娘見我還拿了拿筆兒麼。”惜春道:“這卻是有功德的。”鴛鴦道:“我也有一件事:向來服侍老太太安歇後,自己念上米佛,已經唸了三年多了。我把這個米收好,等老太太做功德的時候,我將他襯在裏頭供佛施食,也是我一點誠心。”惜春道:“這樣說來,老太太做了觀音,你就是龍女了。”鴛鴦道:“那裏跟得上這個分兒。卻是除了老太太,別的也服侍不來,不曉得前世什麼緣分兒。”說着要走,叫小丫頭把小絹包打開,拿出來道:“這素紙一紮是寫《心經》的。”又拿起一子兒藏香道:“這是叫寫經時點着寫的。”惜春都應了。
\end{parag}


\begin{parag}
    鴛鴦遂辭了出來,同小丫頭來至賈母房中,回了一遍。看見賈母與李紈打雙陸,鴛鴦旁邊瞧着。李紈的骰子好,擲下去把老太太的錘打下了好幾個去。鴛鴦抿着嘴兒笑。忽見寶玉進來,手中提了兩個細蔑絲的小籠子,籠內有幾個蟈蟈兒,說道:“我聽說老太太夜裏睡不着,我給老太太留下解解悶。”賈母笑道:“你別瞅着你老子不在家,你只管淘氣。”寶玉笑道:“我沒有淘氣。”賈母道:“你沒淘氣,不在學房裏唸書,爲什麼又弄這個東西呢。”寶玉道:“不是我自己弄的。今兒因師父叫環兒和蘭兒對對子,環兒對不來,我悄悄的告訴了他。他說了,師父喜歡,誇了他兩句。他感激我的情,買了來孝敬我的。我纔拿了來孝敬老太太的。”賈母道:“他沒有天天唸書麼,爲什麼對不上來?對不上來就叫你儒大爺爺打他的嘴巴子,看他臊不臊。你也夠受了,不記得你老子在家時,一叫做詩做詞,唬的倒象個小鬼兒似的,這會子又說嘴了。那環兒小子更沒出息,求人替做了,就變着方法兒打點人。這麼點子孩子就鬧鬼鬧神的,也不害臊,趕大了還不知是個什麼東西呢。”說的滿屋子人都笑了。賈母又問道:“蘭小子呢,做上來了沒有?這該環兒替他了,他又比他小了。是不是?”寶玉笑道:“他倒沒有,卻是自己對的。”賈母道:“我不信,不然就也是你鬧了鬼了。如今你還了得,‘羊羣裏跑出駱駝來了,就只你大。’你又會做文章了。”寶玉笑道:“實在是他作的。師父還誇他明兒一定有出息呢。老太太不信,就打發人叫了他來親自試試,老太太就知道了。”賈母道:“果然這麼着我才喜歡。我不過怕你撒謊。既是他做的,這孩子明兒大概還有一點兒出息。”因看着李紈,又想起賈珠來,”這也不枉你大哥哥死了,你大嫂子拉扯他一場,日後也替你大哥哥頂門壯戶。”說到這裏,不禁流下淚來。李紈聽了這話,卻也動心,只是賈母已經傷心,自己連忙忍住淚笑勸道:“這是老祖宗的餘德,我們託着老祖宗的福罷咧。只要他應得了老祖宗的話,就是我們的造化了。老祖宗看着也喜歡,怎麼倒傷起心來呢。”因又回頭向寶玉道:“寶叔叔明兒別這麼誇他,他多大孩子,知道什麼。你不過是愛惜他的意思,他那裏懂得,一來二去,眼大心肥,那裏還能夠有長進呢。”賈母道:“你嫂子這也說的是。就只他還太小呢,也別逼梏緊了他。小孩子膽兒小,一時逼急了,弄出點子毛病來,書倒唸不成,把你的工夫都白糟踏了。”賈母說到這裏,李紈卻忍不住撲簌簌掉下淚來,連忙擦了。
\end{parag}


\begin{parag}
    只見賈環賈蘭也都進來給賈母請了安。賈蘭又見過他母親,然後過來在賈母旁邊侍立。賈母道:“我剛纔聽見你叔叔說你對的好對子,師父誇你來着。”賈蘭也不言語,只管抿着嘴兒笑。鴛鴦過來說道:“請示老太太,晚飯伺候下了。”賈母道:“請你姨太太去罷。”琥珀接着便叫人去王夫人那邊請薛姨媽。這裏寶玉賈環退出。素雲和小丫頭們過來把雙陸收起。李紈尚等着伺候賈母的晚飯,賈蘭便跟着他母親站着。賈母道:“你們孃兒兩個跟着我喫罷。”李紈答應了。一時擺上飯來,丫鬟回來稟道:“太太叫回老太太,姨太太這幾天浮來暫去,不能過來回老太太,今日飯後家去了。”於是賈母叫賈蘭在身旁邊坐下,大家喫飯,不必細述。
\end{parag}


\begin{parag}
    卻說賈母剛喫完了飯,盥漱了,歪在牀上說閒話兒。只見小丫頭子告訴琥珀,琥珀過來回賈母道:“東府大爺請晚安來了。”賈母道:“你們告訴他,如今他辦理家務乏乏的,叫他歇着去罷。我知道了。”小丫頭告訴老婆子們,老婆子才告訴賈珍。賈珍然後退出。到了次日,賈珍過來料理諸事。門上小廝陸續回了幾件事,又一個小廝回道:“莊頭送果子來了。”賈珍道:“單子呢?”那小廝連忙呈上。賈珍看時,上面寫着不過是時鮮果品,還夾帶菜蔬野味若干在內。賈珍看完,問向來經管的是誰。門上的回道:“是周瑞。”便叫周瑞:“照帳點清,送往裏頭交代。等我把來帳抄下一個底子,留着好對。”又叫”告訴廚房,把下菜中添幾宗給送果子的來人,照常賞飯給錢。”周瑞答應了。一面叫人搬至鳳姐兒院子裏去,又把莊上的帳同果子交代明白。出去了一回兒,又進來回賈珍道:“纔剛來的果子,大爺曾點過數目沒有?”賈珍道:“我那裏有工夫點這個呢。給了你帳,你照帳點就是了。”周瑞道:“小的曾點過,也沒有少,也不能多出來。大爺既留下底子,再叫送果子來的人問問,他這帳是真的假的。”賈珍道:“這是怎麼說,不過是幾個果子罷咧,有什麼要緊。我又沒有疑你。”說着,只見鮑二走來,磕了一個頭,說道:“求大爺原舊放小的在外頭伺候罷。”賈珍道:“你們這又是怎麼着?”鮑二道:“奴才在這裏又說不上話來。”賈珍道:“誰叫你說話。”鮑二道:“何苦來,在這裏作眼睛珠兒。”周瑞接口道:“奴才在這裏經管地租莊子,銀錢出入每年也有三五十萬來往,老爺太太奶奶們從沒有說過話的,何況這些零星東西。若照鮑二說起來,爺們家裏的田地房產都被奴才們弄完了。”賈珍想道:“必是鮑二在這裏拌嘴,不如叫他出去。”因向鮑二說道:“快滾罷。”又告訴周瑞說:“你也不用說了,你幹你的事罷。”二人各自散了。
\end{parag}


\begin{parag}
    賈珍正在廂房裏歇着,聽見門上鬧的翻江攪海。叫人去查問,回來說道:“鮑二和周瑞的乾兒子打架。”賈珍道:“周瑞的乾兒子是誰?”門上的回道:“他叫何三,本來是個沒味兒的,天天在家裏喝酒鬧事,常來門上坐着。聽見鮑二與周瑞拌嘴,他就插在裏頭。”賈珍道:“這卻可惡。把鮑二和那個什麼何幾給我一塊兒捆起來!周瑞呢?”門上的回道:“打架時他先走了。”賈珍道:“給我拿了來!這還了得了!”衆人答應了。正嚷着,賈璉也回來了,賈珍便告訴了一遍。賈璉道:“這還了得!”又添了人去拿周瑞。周瑞知道躲不過,也找到了。賈珍便叫都捆上。賈璉便向周瑞道:“你們前頭的話也不要緊,大爺說開了,很是了。爲什麼外頭又打架!你們打架已經使不得,又弄個野雜種什麼何三來鬧,你不壓伏壓伏他們,倒竟走了。”就把周瑞踢了幾腳。賈珍道:“單打周瑞不中用。”喝命人把鮑二和何三各人打了五十鞭子,攆了出去,方和賈璉兩個商量正事。下人背地裏便生出許多議論來:也有說賈珍護短的,也有說不會調停的,也有說他本不是好人,前兒尤家姊妹弄出許多醜事來,那鮑二不是他調停着二爺叫了來的嗎,這會子又嫌鮑二不濟事,必是鮑二的女人伏侍不到了。人多嘴雜,紛紛不一。
\end{parag}


\begin{parag}
    卻說賈政自從在工部掌印,家人中盡有發財的。那賈芸聽見了,也要插手弄一點事兒,便在外頭說了幾個工頭,講了成數,便買了些時新繡貨,要走鳳姐兒門子。鳳姐正在房中聽見丫頭們說:“大爺二爺都生了氣,在外頭打人呢。”鳳姐聽了,不知何故,正要叫人去問問,只見賈璉已進來了,把外面的事告訴了一遍。鳳姐道:“事情雖不要緊,但這風俗兒斷不可長。此刻還算咱們家裏正旺的時候兒,他們就敢打架。以後小輩兒們當了家,他們越發難制伏了。前年我在東府裏,親眼見過焦大喫的爛醉,躺在臺階子底下罵人,不管上上下下一混湯子的混罵。他雖是有過功的人,到底主子奴才的名分,也要存點兒體統纔好。珍大奶奶不是我說是個老實頭,個個人都叫他養得無法無天的。如今又弄出一個什麼鮑二,我還聽見是你和珍大爺得用的人,爲什麼今兒又打他呢?”賈璉聽了這話刺心,便覺訕訕的,拿話來支開,借有事,說着就走了。
\end{parag}


\begin{parag}
    小紅進來回道:“芸二爺在外頭要見奶奶。”鳳姐一想,”他又來做什麼?”便道:“叫他進來罷。”小紅出來,瞅着賈芸微微一笑。賈芸趕忙湊近一步問道:“姑娘替我回了沒有?”小紅紅了臉,說道:“我就是見二爺的事多。”賈芸道:“何曾有多少事能到裏頭來勞動姑娘呢。就是那一年姑娘在寶二叔房裏,我才和姑娘——”小紅怕人撞見,不等說完,趕忙問道:“那年我換給二爺的一塊絹子,二爺見了沒有?”那賈芸聽了這句話,喜的心花俱開,纔要說話,只見一個小丫頭從裏面出來,賈芸連忙同着小紅往裏走。兩個人一左一右,相離不遠,賈芸悄悄的道:“回來我出來還是你送出我來,我告訴你還有笑話兒呢。”小紅聽了,把臉飛紅,瞅了賈芸一眼,也不答言。同他到了鳳姐門口,自己先進去回了,然後出來,掀起簾子點手兒,口中卻故意說道:“奶奶請芸二爺進來呢。”
\end{parag}


\begin{parag}
    賈芸笑了一笑,跟着他走進房來,見了鳳姐兒,請了安,並說:“母親叫問好。”鳳姐也問了他母親好。鳳姐道:“你來有什麼事?”賈芸道:“侄兒從前承嬸孃疼愛,心上時刻想着,總過意不去。欲要孝敬嬸孃,又怕嬸孃多想。如今重陽時候,略備了一點兒東西。嬸孃這裏那一件沒有,不過是侄兒一點孝心。只怕嬸孃不肯賞臉。”鳳姐兒笑道:“有話坐下說。”賈芸才側身坐了,連忙將東西捧着擱在旁邊桌上。鳳姐又道:“你不是什麼有餘的人,何苦又去花錢。我又不等着使。你今日來意是怎麼個想頭兒,你倒是實說。”賈芸道:“並沒有別的想頭兒,不過感念嬸孃的恩惠,過意不去罷咧。”說着微微的笑了。鳳姐道:“不是這麼說。你手裏窄,我很知道,我何苦白白兒使你的。你要我收下這個東西,須先和我說明白了。要是這麼含着骨頭露着肉的,我倒不收。”賈芸沒法兒,只得站起來陪着笑兒說道:“並不是有什麼妄想。前幾日聽見老爺總辦陵工,侄兒有幾個朋友辦過好些工程,極妥當的,要求嬸孃在老爺跟前提一提。辦得一兩種,侄兒再忘不了嬸孃的恩典。若是家裏用得着,侄兒也能給嬸孃出力。”鳳姐道:“若是別的我卻可以作主。至於衙門裏的事,上頭呢,都是堂官司員定的,底下呢,都是那些書辦衙役們辦的。別人只怕插不上手。連自己的家人,也不過跟着老爺伏侍伏侍。就是你二叔去,亦只是爲的是各自家裏的事,他也並不能攙越公事。論家事,這裏是踩一頭兒橇一頭兒的,連珍大爺還彈壓不住,你的年紀兒又輕,輩數兒又小,那裏纏的清這些人呢。況且衙門裏頭的事差不多兒也要完了,不過喫飯瞎跑。你在家裏什麼事作不得,難道沒了這碗飯喫不成。我這是實在話,你自己回去想想就知道了。你的情意我已經領了,把東西快拿回去,是那裏弄來的,仍舊給人家送了去罷。”正說着,只見奶媽子一大起帶了巧姐兒進來。那巧姐兒身上穿得錦團花簇,手裏拿着好些頑意兒,笑嘻嘻走到鳳姐身邊學舌。賈芸一見,便站起來笑盈盈的趕着說道:“這就是大妹妹麼?你要什麼好東西不要?”那巧姐兒便啞的一聲哭了。賈芸連忙退下。鳳姐道:“乖乖不怕。”連忙將巧姐攬在懷裏道:“這是你芸大哥哥,怎麼認起生來了。”賈芸道:“妹妹生得好相貌,將來又是個有大造化的。”那巧姐兒回頭把賈芸一瞧,又哭起來,迭連幾次。賈芸看這光景坐不住,便起身告辭要走。鳳姐道:“你把東西帶了去罷。”賈芸道:“這一點子嬸孃還不賞臉?”鳳姐道:“你不帶去,我便叫人送到你家去。芸哥兒,你不要這麼樣,你又不是外人,我這裏有機會,少不得打發人去叫你,沒有事也沒法兒,不在乎這些東東西西上的。”賈芸看見鳳姐執意不受,只得紅着臉道:“既這麼着,我再找得用的東西來孝敬嬸孃罷。”鳳姐兒便叫小紅拿了東西,跟着賈芸送出來。
\end{parag}


\begin{parag}
    賈芸走着,一面心中想道:“人說二奶奶利害,果然利害。一點兒都不漏縫,真正斬釘截鐵,怪不得沒有後世。這巧姐兒更怪,見了我好象前世的冤家似的。真正晦氣,白鬧了這麼一天。”小紅見賈芸沒得彩頭,也不高興,拿着東西跟出來。賈芸接過來,打開包兒揀了兩件,悄悄的遞給小紅。小紅不接,嘴裏說道:“二爺別這麼着,看奶奶知道了,大家倒不好看。”賈芸道:“你好生收着罷,怕什麼,那裏就知道了呢。你若不要,就是瞧不起我了。”小紅微微一笑,才接過來,說道:“誰要你這些東西,算什麼呢。”說了這句話,把臉又飛紅了。賈芸也笑道:“我也不是爲東西,況且那東西也算不了什麼。”說着話兒,兩個已走到二門口。賈芸把下剩的仍舊揣在懷內。小紅催着賈芸道:“你先去罷,有什麼事情,只管來找我。我今日在這院裏了,又不隔手。”賈芸點點頭兒,說道:“二奶奶太利害,我可惜不能長來。剛纔我說的話,你橫豎心裏明白,得了空兒再告訴你罷。”小紅滿臉羞紅,說道:“你去罷,明兒也長來走走。誰叫你和他生疏呢。”賈芸道:“知道了。”賈芸說着出了院門。這裏小紅站在門口,怔怔的看他去遠了,纔回來了。
\end{parag}


\begin{parag}
    卻說鳳姐在房中吩咐預備晚飯,因又問道:“你們熬了粥了沒有?”丫鬟們連忙去問,回來回道:“預備了。”鳳姐道:“你們把那南邊來的糟東西弄一兩碟來罷。”秋桐答應了,叫丫頭們伺候。平兒走來笑道:“我倒忘了,今兒晌午奶奶在上頭老太太那邊的時候,水月庵的師父打發人來,要向奶奶討兩瓶南小菜,還要支用幾個月的月銀,說是身上不受用。我問那道婆來着:‘師父怎麼不受用?’他說:‘四五天了,前兒夜裏因那些小沙彌小道士裏頭有幾個女孩子睡覺沒有吹燈,他說了幾次不聽。那一夜看見他們三更以後燈還點着呢,他便叫他們吹燈,個個都睡着了,沒有人答應,只得自己親自起來給他們吹滅了。回到炕上,只見有兩個人,一男一女,坐在炕上。他趕着問是誰,那裏把一根繩子往他脖子上一套,他便叫起人來。衆人聽見,點上燈火一齊趕來,已經躺在地下,滿口吐白沫子,幸虧救醒了。此時還不能喫東西,所以叫來尋些小菜兒的。’我因奶奶不在房中,不便給他。我說:‘奶奶此時沒有空兒,在上頭呢,回來告訴。’便打發他回去了。纔剛聽見說起南菜,方想起來了,不然就忘了。”鳳姐聽了,呆了一呆,說道:“南菜不是還有呢,叫人送些去就是了。那銀子過一天叫芹哥來領就是了。”又見小紅進來回道:“纔剛二爺差人來,說是今晚城外有事,不能回來,先通知一聲。”鳳姐道:“是了。”
\end{parag}


\begin{parag}
    說着,只聽見小丫頭從後面喘吁吁的嚷着直跑到院子裏來,外面平兒接着,還有幾個丫頭們,咕咕唧唧的說話。鳳姐道:“你們說什麼呢?”平兒道:“小丫頭子有些膽怯,說鬼話。”鳳姐叫那一個小丫頭進來,問道:“什麼鬼話?”那丫頭道:“我纔剛到後邊去叫打雜兒的添煤,只聽得三間空屋子裏譁喇譁喇的響,我還道是貓兒耗子,又聽得噯的一聲,象個人出氣兒的似的。我害怕,就跑回來了。”鳳姐罵道:“胡說!我這裏斷不興說神說鬼,我從來不信這些個話。快滾出去罷。”那小丫頭出去了。鳳姐便叫彩明將一天零碎日用帳對過一遍,時已將近二更。大家又歇了一回,略說些閒話,遂叫各人安歇去罷。鳳姐也睡下了。將近三更,鳳姐似睡不睡,覺得身上寒毛一乍,自己驚醒了,越躺着越發起滲來,因叫平兒秋桐過來作伴。二人也不解何意。那秋桐本來不順鳳姐,後來賈璉因尤二姐之事不大愛惜他了,鳳姐又籠絡他,如今倒也安靜,只是心裏比平兒差多了,外面情兒。今見鳳姐不受用,只得端上茶來。鳳姐喝了一口,道:“難爲你,睡去罷,只留平兒在這裏就夠了。”秋桐卻要獻勤兒,因說道:“奶奶睡不着,倒是我們兩個輪流坐坐也使得。”鳳姐一面說,一面睡着了。平兒秋桐看見鳳姐已睡,只聽得遠遠的雞叫了,二人方都穿着衣服略躺了一躺,就天亮了,連忙起來伏侍鳳姐梳洗。鳳姐因夜中之事,心神恍惚不寧,只是一味要強,仍然扎掙起來。正坐着納悶,忽聽個小丫頭子在院裏問道:“平姑娘在屋裏麼?”平兒答應了一聲,那小丫頭掀起簾子進來,卻是王夫人打發過來來找賈璉,說:“外頭有人回要緊的官事。老爺纔出了門,太太叫快請二爺過去呢。”鳳姐聽見唬了一跳。未知何事,下回分解。
\end{parag}