\chap{三十}{寶釵借扇機帶雙敲 齡官劃薔癡及局外}


\begin{parag}
    \begin{note}庚:指扇敲雙玉是寫寶釵金蟬脫殼。\end{note}
\end{parag}


\begin{parag}
    \begin{note}庚:銀釵畫“薔”字是癡女夢中說夢。\end{note}
\end{parag}


\begin{parag}
    \begin{note}庚:腳踢襲人是斷無是理,竟有是事。\end{note}
\end{parag}


\begin{parag}
    \begin{note}靖:無限文字,癡情畫薔,可知前緣有定,非人力強求。\end{note}
\end{parag}


\begin{parag}
    話說林黛玉與寶玉角口後,也自後悔,但又無去就他之理,因此日夜悶悶,如有所失。紫鵑度其意,乃勸道:“若論前日之事,竟是姑娘太浮躁了些。別人不知寶玉那脾氣,難道咱們也不知道的。爲那玉也不是鬧了一遭兩遭了。”黛玉啐道:“你倒來替人派我的不是。我怎麼浮躁了?”紫鵑笑道:“好好的,爲什麼又剪了那穗子?豈不是寶玉只有三分不是,姑娘倒有七分不是。我看他素日在姑娘身上就好,皆因姑娘小性兒,常要歪派他,才這麼樣。”
\end{parag}


\begin{parag}
    林黛玉正欲答話,只聽院外叫門。紫鵑聽了一聽,笑道:“這是寶玉的聲音,想必是來賠不是來了。”林黛玉聽了道:“不許開門!”紫鵑道:“姑娘又不是了。這麼熱天毒日頭地下,曬壞了他如何使得呢!”口裏說著,便出去開門,果然是寶玉。一面讓他進來,一面笑道:“我只當是寶二爺再不上我們這門了,誰知這會子又來了。”寶玉笑道:“你們把極小的事倒說大了。好好的爲什麼不來?我便死了,魂也要一日來一百遭。妹妹可大好了?”紫鵑道:“身上病好了,只是心裏氣不大好。”寶玉笑道:“我曉得有什麼氣。”一面說著,一面進來,只見林黛玉又在牀上哭。
\end{parag}


\begin{parag}
    那林黛玉本不曾哭,聽見寶玉來,由不得傷了心,止不住滾下淚來。寶玉笑著走近牀來,道:“妹妹身上可大好了?”林黛玉只顧拭淚,並不答應。寶玉因便挨在牀沿上坐了,一面笑道:“我知道妹妹不惱我。但只是我不來,叫旁人看著,倒象是咱們又拌了嘴的似的。若等他們來勸咱們,那時節豈不咱們倒覺生分了?不如這會子,你要打要罵,憑著你怎麼樣,千萬別不理我。”說著,又把“好妹妹”叫了幾萬聲。林黛玉心裏原是再不理寶玉的,這會子見寶玉說別叫人知道他們拌了嘴就生分了似的這一句話,又可見得比人原親近,因又掌不住哭道:“你也不用哄我。從今以後,我也不敢親近二爺,二爺也全當我去了。”寶玉聽了笑道:“你往那去呢?”林黛玉道:“我回家去。”寶玉笑道:“我跟了你去。”林黛玉道:“我死了。”寶玉道:“你死了,我做和尚!”林黛玉一聞此言,登時將臉放下來,問道:“想是你要死了,胡說的是什麼!你家倒有幾個親姐姐親妹妹呢,明兒都死了,你幾個身子去作和尚?明兒我倒把這話告訴別人去評評。”
\end{parag}


\begin{parag}
    寶玉自知這話說的造次了,後悔不來,登時臉上紅脹起來,低著頭不敢則一聲。幸而屋裏沒人。林黛玉直瞪瞪的瞅了他半天,氣的一聲兒也說不出來。見寶玉憋的臉上紫脹,便咬著牙用指頭狠命的在他額顱上戳了一下,哼了一聲,咬牙說道:“你這──”剛說了兩個字,便又嘆了一口氣,仍拿起手帕子來擦眼淚。寶玉心裏原有無限的心事,又兼說錯了話,正自後悔;又見黛玉戳他一下,要說又說不出來,自嘆自泣,因此自己也有所感,不覺滾下淚來。要用帕子揩拭,不想又忘了帶來,便用衫袖去擦。林黛玉雖然哭著,卻一眼看見了,見他穿著簇新藕合紗衫,竟去拭淚,便一面自己拭著淚,一面回身將枕邊搭的一方綃帕子拿起來,向寶玉懷裏一摔,一語不發,仍掩面自泣。寶玉見他摔了帕子來,忙接住拭了淚,又挨近前些,伸手拉了林黛玉一隻手,笑道:“我的五臟都碎了,你還只是哭。走罷,我同你往老太太跟前去。”林黛玉將手一摔道:“誰同你拉拉扯扯的。一天大似一天的,還這麼涎皮賴臉的,連個道理也不知道。”
\end{parag}


\begin{parag}
    一句沒說完,只聽喊道:“好了!”寶林二人不防,都唬了一跳,回頭看時,只見鳳姐兒跳了進來,笑道:“老太太在那裏抱怨天抱怨地,只叫我來瞧瞧你們好了沒有。我說不用瞧,過不了三天,他們自己就好了。老太太罵我,說我懶。我來了,果然應了我的話了。也沒見你們兩個人有些什麼可拌的,三日好了,兩日惱了,越大越成了孩子了!有這會子拉著手哭的,昨兒爲什麼又成了烏眼雞呢!還不跟我走,到老太太跟前,叫老人家也放些心。”說著拉了林黛玉就走。林黛玉回頭叫丫頭們,一個也沒有。鳳姐道:“又叫他們作什麼,有我伏侍你呢。”一面說,一面拉了就走。寶玉在後面跟著出了園門。到了賈母跟前,鳳姐笑道:“我說他們不用人費心,自己就會好的。老祖宗不信,一定叫我去說合。我及至到那裏要說合,誰知兩個人倒在一處對賠不是了。對笑對訴,倒象‘黃鷹抓住了鷂子的腳’,兩個都扣了環了,那裏還要人去說合。”說的滿屋裏都笑起來。
\end{parag}


\begin{parag}
    此時寶釵正在這裏。那林黛玉只一言不發,挨著賈母坐下。寶玉沒甚說的,便向寶釵笑道:“大哥哥好日子,偏生我又不好了,沒別的禮送,連個頭也不得磕去。大哥哥不知我病,倒像我懶,推故不去的。倘或明兒惱了,姐姐替我分辨分辨。”寶釵笑道:“這也多事。你便要去也不敢驚動,何況身上不好,弟兄們日日一處,要存這個心倒生分了。”寶玉又笑道:“姐姐知道體諒我就好了。”又道:“姐姐怎麼不看戲去?”寶釵道:“我怕熱,看了兩出,熱的很。要走,客又不散。我少不得推身上不好,就來了。”寶玉聽說,自己由不得臉上沒意思,只得又搭訕笑道:“怪不得他們拿姐姐比楊妃,原來也體豐怯熱。”寶釵聽說,不由的大怒,待要怎樣,又不好怎樣。回思了一回,臉紅起來,便冷笑了兩聲,說道:“我倒象楊妃,只是沒一個好哥哥好兄弟可以作得楊國忠的!”二人正說著,可巧小丫頭靛兒因不見了扇子,和寶釵笑道:“必是寶姑娘藏了我的。好姑娘,賞我罷。”寶釵指他道:“你要仔細!我和你頑過,你再疑我。和你素日嘻皮笑臉的那些姑娘們跟前,你該問他們去。”說的個靛兒跑了。寶玉自知又把話說造次了,當著許多人,更比才在林黛玉跟前更不好意思,便急回身又同別人搭訕去了。
\end{parag}


\begin{parag}
    林黛玉聽見寶玉奚落寶釵,心中著實得意,纔要搭言也趁勢兒取個笑,不想靛兒因找扇子,寶釵又發了兩句話,他便改口笑道:“寶姐姐,你聽了兩出什麼戲?”寶釵因見林黛玉面上有得意之態,一定是聽了寶玉方纔奚落之言,遂了他的心願,忽又見問他這話,便笑道:“我看的是李逵罵了宋江,後來又賠不是。”寶玉便笑道:“姐姐通今博古,色色都知道,怎麼連這一齣戲的名字也不知道,就說了這麼一串子。這叫《負荊請罪》。”寶釵笑道:“原來這叫作《負荊請罪》!你們通今博古,才知道‘負荊請罪’,我不知道什麼是‘負荊請罪’!”一句話還未說完,寶玉林黛玉二人心裏有病,聽了這話早把臉羞紅了。鳳姐於這些上雖不通達,但只見他三人形景,便知其意,便也笑著問人道:“你們大暑天,誰還喫生薑呢?”衆人不解其意,便說道:“沒有喫生薑。”鳳姐故意用手摸著腮,詫異道: “既沒人喫生薑,怎麼這麼辣辣的?”寶玉黛玉二人聽見這話,越發不好過了。寶釵再要說話,見寶玉十分討愧,形景改變,也就不好再說,只得一笑收住。別人總未解得他四個人的言語,因此付之流水。
\end{parag}


\begin{parag}
    一時寶釵鳳姐去了,林黛玉笑向寶玉道:“你也試著比我利害的人了。誰都像我心拙口笨的,由著人說呢。”寶玉正因寶釵多了心,自己沒趣,又見林黛玉來問著他,越發沒好氣起來。待要說兩句,又恐林黛玉多心,說不得忍著氣,無精打采一直出來。
\end{parag}


\begin{parag}
    誰知目今盛暑之時,又當早飯已過,各處主僕人等多半都因日長神倦之時,寶玉背著手,到一處,一處鴉雀無聞。從賈母這裏出來,往西走過了穿堂,便是鳳姐的院落。到他們院門前,只見院門掩著。知道鳳姐素日的規矩,每到天熱,午間要歇一個時辰的,進去不便,遂進角門,來到王夫人上房內。只見幾個丫頭子手裏拿著針線,卻打盹兒呢。王夫人在裏間涼榻上睡著,金釧兒坐在旁邊捶腿,也乜斜著眼亂恍。
\end{parag}


\begin{parag}
    寶玉輕輕的走到跟前,把他耳上帶的墜子一摘,金釧兒睜開眼,見是寶玉。寶玉悄悄的笑道:“就困的這麼著?”金釧抿嘴一笑,擺手令他出去,仍合上眼。寶玉見了他,就有些戀戀不捨的,悄悄的探頭瞧瞧王夫人合著眼,便自己向身邊荷包裏帶的香雪潤津丹掏了出來,便向金釧兒口裏一送。金釧兒並不睜眼,只管噙了。寶玉上來便拉著手,悄悄的笑道:“我明日和太太討你,咱們在一處罷。”金釧兒不答。寶玉又道:“不然,等太太醒了我就討。”金釧兒睜開眼,將寶玉一推,笑道:“你忙什麼!‘金簪子掉在井裏頭,有你的只是有你的’,連這句話語難道也不明白?我倒告訴你個巧宗兒,你往東小院子裏拿環哥兒同彩雲去。”寶玉笑道: “憑他怎麼去罷,我只守著你。”只見王夫人翻身起來,照金釧兒臉上就打了個嘴巴子,指著罵道:“下作小娼婦,好好的爺們,都叫你教壞了。”寶玉見王夫人起來,早一溜煙去了。
\end{parag}


\begin{parag}
    這裏金釧兒半邊臉火熱,一聲不敢言語。登時衆丫頭聽見王夫人醒了,都忙進來。王夫人便叫玉釧兒:“把你媽叫來,帶出你姐姐去。”金釧兒聽說,忙跪下哭道:“我再不敢了。太太要打罵,只管發落,別叫我出去就是天恩了。我跟了太太十來年,這會子攆出去,我還見人不見人呢!”王夫人固然是個寬仁慈厚的人,從來不曾打過丫頭們一下,今忽見金釧兒行此無恥之事,此乃平生最恨者,故氣忿不過,打了一下,罵了幾句。雖金釧兒苦求,亦不肯收留,到底喚了金釧兒之母白老媳婦來領了下去。那金釧兒含羞忍辱的出去,不在話下。
\end{parag}


\begin{parag}
    且說那寶玉見王夫人醒來,自己沒趣,忙進大觀園來。只見赤日當空,樹陰合地,滿耳蟬聲,靜無人語。剛到了薔薇花架,只聽有人哽噎之聲。寶玉心中疑惑,便站住細聽,果然架下那邊有人。如今五月之際,那薔薇正是花葉茂盛之際,寶玉便悄悄的隔著籬笆洞兒一看,只見一個女孩子蹲在花下,手裏拿著根綰頭的簪子在地下摳土,一面悄悄的流淚。寶玉心中想道:“難道這也是個癡丫頭,又象顰兒來葬花不成?”因又自嘆道:“若真也葬花,可謂‘東施效顰’,不但不爲新特,且更可厭了。”想畢,便要叫那女子,說:“你不用跟著那林姑娘學了。”話未出口,幸而再看時,這女孩子面生,不是個侍兒,倒象是那十二個學戲的女孩子之內的,卻辨不出他是生旦淨醜那一個角色來。寶玉忙把舌頭一伸,將口掩住,自己想道:“幸而不曾造次。上兩次皆因造次了,顰兒也生氣,寶兒也多心,如今再得罪了他們,越發沒意思了。”
\end{parag}


\begin{parag}
    一面想,一面又恨認不得這個是誰。再留神細看,只見這女孩子眉蹙春山,眼顰秋水,面薄腰纖,嫋嫋婷婷,大有林黛玉之態。寶玉早又不忍棄他而去,只管癡看。只見他雖然用金簪劃地,並不是掘土埋花,竟是向土上畫字。寶玉用眼隨著簪子的起落,一直一畫一點一勾的看了去,數一數,十八筆。自己又在手心裏用指頭按著他方纔下筆的規矩寫了,猜是個什麼字。寫成一想,原來就是個薔薇花的“薔”字。寶玉想道:“必定是他也要作詩填詞。這會子見了這花,因有所感,或者偶成了兩句,一時興至恐忘,在地下畫著推敲,也未可知。且看他底下再寫什麼。”一面想,一面又看,只見那女孩子還在那裏畫呢,畫來畫去,還是個“薔”字。再看,還是個“薔”字。裏面的原是早已癡了,畫完一個又畫一個,已經畫了有幾千個“薔”。外面的不覺也看癡了,兩個眼睛珠兒只管隨著簪子動,心裏卻想:“這女孩子一定有什麼話說不出來的大心事,才這樣個形景。外面既是這個形景,心裏不知怎麼熬煎。看他的模樣兒這般單薄,心裏那裏還擱的住熬煎。可恨我不能替你分些過來。”
\end{parag}


\begin{parag}
    伏中陰晴不定,片雲可以致雨,忽一陣涼風過了,唰唰的落下一陣雨來。寶玉看著那女子頭上滴下水來,紗衣裳登時溼了。寶玉想道:“這時下雨。他這個身子,如何禁得驟雨一激!”因此禁不住便說道:“不用寫了。你看下大雨,身上都溼了。”那女孩子聽說倒唬了一跳,抬頭一看,只見花外一個人叫他不要寫了,下大雨了。一則寶玉臉面俊秀;二則花葉繁茂,上下俱被枝葉隱住,剛露著半邊臉,那女孩子只當是個丫頭,再不想是寶玉,因笑道:“多謝姐姐提醒了我。難道姐姐在外頭有什麼遮雨的?”一句提醒了寶玉,“噯喲”了一聲,才覺得渾身冰涼。低頭一看,自己身上也都溼了。說聲“不好”,只得一氣跑回怡紅院去了,心裏卻還記掛著那女孩子沒處避雨。
\end{parag}


\begin{parag}
    原來明日是端陽節,那文官等十二個女子都放了學,進園來各處頑耍。可巧小生寶官、正旦玉官兩個女孩子,正在怡紅院和襲人玩笑,被大雨阻住。大家把溝堵了,水積在院內,把些綠頭鴨、花鸂鶒、彩鴛鴦,捉的捉,趕的趕,縫了翅膀,放在院內頑耍,將院門關了。襲人等都在遊廊上嘻笑。
\end{parag}


\begin{parag}
    寶玉見關著門,便以手扣門,裏面諸人只顧笑,那裏聽見。叫了半日,拍的門山響,裏面方聽見了,估諒著寶玉這會子再不回來的。襲人笑道:“誰這會子叫門,沒人開去。”寶玉道:“是我。”麝月道:“是寶姑娘的聲音。”晴雯道:“胡說!寶姑娘這會子做什麼來。”襲人道:“讓我隔著門縫兒瞧瞧,可開就開,要不可開,叫他淋著去。”說著,便順著遊廊到門前,往外一瞧,只見寶玉淋的雨打雞一般。襲人見了又是著忙又是可笑,忙開了門,笑的彎著腰拍手道:“這麼大雨地裏跑什麼?那裏知道爺回來了。”寶玉一肚子沒好氣,滿心裏要把開門的踢幾腳,及開了門,並不看真是誰,還只當是那些小丫頭子們,便抬腿踢在肋上。襲人 “噯喲”了一聲。寶玉還罵道:“下流東西們!我素日擔待你們得了意,一點兒也不怕,越發拿我取笑兒了。”口裏說著,一低頭見是襲人哭了,方知踢錯了,忙笑道:“噯喲,是你來了!踢在那裏了?”襲人從來不曾受過大話的,今兒忽見寶玉生氣踢他一下,又當著許多人,又是羞,又是氣,又是疼,真一時置身無地。待要怎麼樣,料著寶玉未必是安心踢他,少不得忍著說道:“沒有踢著。還不換衣裳去。”寶玉一面進房來解衣,一面笑道:“我長了這麼大,今日是頭一遭兒生氣打人,不想就偏遇見了你!”襲人一面忍痛換衣裳,一面笑道:“我是個起頭兒的人,不論事大事小事好事歹,自然也該從我起。但只是別說打了我,明兒順了手也打起別人來。”寶玉道:“我才也不是安心。”襲人道:“誰說你是安心了!素日開門關門,都是那起小丫頭子們的事。他們是憨皮慣了的,早已恨的人牙癢癢,他們也沒個怕懼兒。你當是他們,踢一下子,唬唬他們也好些。纔剛是我淘氣,不叫開門的。”
\end{parag}


\begin{parag}
    說著,那雨已住了,寶官、玉官也早去了。襲人只覺肋下疼的心裏發鬧,晚飯也不曾好生喫。至晚間洗澡時脫了衣服,只見肋上青了碗大一塊,自己倒唬了一跳,又不好聲張。一時睡下,夢中作痛,由不得“噯喲”之聲從睡中哼出。寶玉雖說不是安心,因見襲人懶懶的,也睡不安穩。忽夜間聽得“噯喲”,便知踢重了,自己下牀那牡秉燈來照。剛到牀前,只見襲人嗽了兩聲,吐出一口痰來,“噯喲”一聲,睜開眼見了寶玉,倒唬了一跳道:“作什麼?”寶玉道:“你夢裏‘噯喲’,必定踢重了。我瞧瞧。”襲人道:“我頭上發暈,嗓子裏又腥又甜,你倒照一照地下罷。”寶玉聽說,果然持燈向地下一照,只見一口鮮血在地。寶玉慌了,只說:“了不得了!”襲人見了,也就心冷了半截。要知端的,且聽下回分解。
\end{parag}


\begin{parag}
    \begin{note}蒙回末總評:愛衆不長,多情不壽;風月情懷,醉人如酒。\end{note}
\end{parag}