\chap{二十九}{享福人福深還禱福 癡情女情重愈斟情}


\begin{parag}
    \begin{note}庚:清虛觀賈母鳳姐原意大適意大快樂,偏寫出多少不適意事來,此亦天然至情至理必有之事。\end{note}
\end{parag}


\begin{parag}
    \begin{note}庚:二玉心事此回大書,是難了割,卻用太君一言以定,是道悉通部書之大旨。\end{note}
\end{parag}


\begin{parag}
    話說寶玉正自發怔,不想黛玉將手帕子甩了來,正碰在眼睛上,倒唬了一跳,問是誰。林黛玉搖著頭兒笑道:“不敢,是我失了手。因爲寶姐姐要看呆雁,我比給他看,不想失了手。”寶玉揉著眼睛,待要說什麼,又不好說的。
\end{parag}


\begin{parag}
    一時,鳳姐兒來了,因說起初一日在清虛觀打醮的事來,遂約著寶釵、寶玉、黛玉等看戲去。寶釵笑道:“罷,罷,怪熱的。什麼沒看過的戲,我就不去了。” 鳳姐兒道:“他們那裏涼快,兩邊又有樓。咱們要去,我頭幾天打發人去,把那些道士都趕出去,把樓打掃乾淨,掛起簾子來,一個閒人不許放進廟去,纔是好呢。我已經回了太太了,你們不去我去。這些日子也悶的很了。家裏唱動戲,我又不得舒舒服服的看。”
\end{parag}


\begin{parag}
    賈母聽說,笑道:“既這麼著,我同你去。”鳳姐聽說,笑道:“老祖宗也去,敢情好了!就只是我又不得受用了。”賈母道:“到明兒,我在正面樓上,你在旁邊樓上,你也不用到我這邊來立規矩,可好不好?”鳳姐兒笑道:“這就是老祖宗疼我了。”賈母因又寶釵道:“你也去,連你母親也去。長天老日的,在家裏也是睡覺。”寶釵只得答應著。
\end{parag}


\begin{parag}
    賈母又打發人去請了薛姨媽,順路告訴王夫人,要帶了他們姊妹去。王夫人因一則身上不好,二則預備著元春有人出來,早已回了不去的;聽賈母如今這樣說,笑道:“還是這麼高興。”因打發人去到園裏告訴:“有要逛的,只管初一跟了老太太逛去。”這個話一傳開了,別人都還可已,只是那些丫頭們天天不得出門檻子,聽了這話,誰不要去。便是各人的主子懶怠去,他也百般攛掇了去,因此李宮裁等都說去。賈母越發心中喜歡,早已吩咐人去打掃安置,都不必細說。
\end{parag}


\begin{parag}
    單表到了初一這一日,榮國府門前車輛紛紛,人馬簇簇。那底下凡執事人等,聞得是貴妃作好事,賈母親去拈香,正是初一日乃月之首日,況是端陽節間,因此凡動用的什物,一色都是齊全的,不同往日。少時,賈母等出來。賈母坐一乘八人大轎,李氏、鳳姐兒、薛姨媽每人一乘四人轎,寶釵、黛玉二人共坐一輛翠蓋珠纓八寶車,迎春、探春、惜春三人共坐一輛朱輪華蓋車。然後賈母的丫頭鴛鴦、鸚鵡、琥珀、珍珠,林黛玉的丫頭紫鵑、雪雁、春纖,寶釵的丫頭鶯兒、文杏,迎春的丫頭司棋、繡桔,探春的丫頭侍書、翠墨,惜春的丫頭入畫、彩屏,薛姨媽的丫頭同喜、同貴,外帶著香菱,香菱的丫頭臻兒,李氏的丫頭素雲、碧月,鳳姐兒的丫頭平兒、豐兒、小紅,並王夫人兩個丫頭也要跟了鳳姐兒去的金釧、彩雲,奶子抱著大姐兒帶著巧姐兒另在一車,還有兩個丫頭,一共又連上各房的老嬤嬤奶孃並跟出門的家人媳婦子,烏壓壓的佔了一街的車。賈母等已經坐轎去了多遠,這門前尚未坐完。這個說“我不同你在一處”,那個說“你壓了我們奶奶的包袱”,那邊車上又說“蹭了我的花兒”,這邊又說“碰折了我的扇子”,咭咭呱呱,說笑不絕。周瑞家的走來過去的說道:“姑娘們,這是街上,看人笑話。”說了兩遍,方覺好了。前頭的全副執事擺開,早已到了清虛觀了。寶玉騎著馬,在賈母轎前。街上人都站在兩邊。
\end{parag}


\begin{parag}
    將至觀前,只聽鐘鳴鼓響,早有張法官執香披衣,帶領衆道士在路旁迎接。賈母的轎剛至山門以內,賈母在轎內因看見有守門大帥並千里眼、順風耳、當方土地、本境城隍各位泥胎聖像,便命住轎。賈珍帶領各子弟上來迎接。鳳姐兒知道鴛鴦等在後面,趕不上來攙賈母,自己下了轎,忙要上來攙。可巧有個十二三歲的小道士兒,拿著剪筒,照管剪各處蠟花,正欲得便且藏出去,不想一頭撞在鳳姐兒懷裏。鳳姐便一揚手,照臉一下,把那小孩子打了一個筋斗,罵道:“野牛肏的,胡朝那裏跑!”那小道士也不顧拾燭剪,爬起來往外還要跑。正值寶釵等下車,衆婆娘媳婦正圍隨的風雨不透,但見一個小道士滾了出來,都喝聲叫“拿,拿,拿!打,打,打!”
\end{parag}


\begin{parag}
    賈母聽了忙問:“是怎麼了?”賈珍忙出來問。鳳姐上去攙住賈母,就回說:“一個小道士兒,剪燈花的,沒躲出去,這會子混鑽呢。”賈母聽說,忙道:“快帶了那孩子來,別唬著他。小門小戶的孩子,都是嬌生慣養的,那裏見的這個勢派。倘或唬著他,倒怪可憐見的,他老子娘豈不疼的慌?”說著,便叫賈珍去好生帶了來。賈珍只得去拉了那孩子來。那孩子還一手拿著蠟剪,跪在地下亂戰。賈母命賈珍拉起來,叫他別怕,問他幾歲了。那孩子通說不出話來。賈母還說“可憐見的”,又向賈珍道:“珍哥兒,帶他去罷。給他些錢買果子喫,別叫人難爲了他。”賈珍答應,領他去了。這裏賈母帶著衆人,一層一層的瞻拜觀玩。外面小廝們見賈母等進入二層山門,忽見賈珍領了一個小道士出來,叫人來帶去,給他幾百錢,不要難爲了他。家人聽說,忙上來領了下去。
\end{parag}


\begin{parag}
    賈珍站在階磯上,因問:“管家在那裏?”底下站的小廝們見問,都一齊喝聲說:“叫管家!”登時林之孝一手整理著帽子跑了來,到賈珍跟前。賈珍道:“雖說這裏地方大,今兒不承望來這麼些人。你使的人,你就帶了往你的那院裏去;使不著的,打發到那院裏去。把小幺兒們多挑幾個在這二層門上同兩邊的角門上,伺候著要東西傳話。你可知道不知道,今兒小姐奶奶們都出來,一個閒人也到不了這裏。”林之孝忙答應“曉得”,又說了幾個“是”。賈珍道:“去罷。”又問: “怎麼不見蓉兒?”一聲未了,只見賈蓉從鐘樓裏跑了出來。賈珍道:“你瞧瞧他,我這裏也還沒敢說熱,他倒乘涼去了!”喝命家人啐他。那小廝們都知道賈珍素日的性子,違拗不得,有個小廝便上來向賈蓉臉上啐了一口。賈珍又道:“問著他!”那小廝便問賈蓉道:“爺還不怕熱,哥兒怎麼先乘涼去了?”賈蓉垂著手,一聲不敢說。那賈芸、賈萍、賈芹等聽見了,不但他們慌了,亦且連賈璜、賈㻞、賈瓊等也都忙了,一個一個從牆根下慢慢的溜上來。賈珍又向賈蓉道:“你站著作什麼?還不騎了馬跑到家裏,告訴你娘母子去!老太太同姑娘們都來了,叫他們快來伺候。”賈蓉聽說,忙跑了出來,一疊聲要馬,一面抱怨道:“早都不知作什麼的,這會子尋趁我。”一面又罵小子:“捆著手呢?馬也拉不來。”待要打發小子去,又恐後來對出來,說不得親自走一趟,騎馬去了,不在話下。
\end{parag}


\begin{parag}
    且說賈珍方要抽身進去,只見張道士站在旁邊陪笑說道:“論理我不比別人,應該裏頭伺候。只因天氣炎熱,衆位千金都出來了,法官不敢擅入,請爺的示下。恐老太太問,或要隨喜那裏,我只在這裏伺候罷了。”賈珍知道這張道士雖然是當日榮國府國公的替身,曾經先皇御口親呼爲“大幻仙人”,如今現掌“道錄司” 印,又是當今封爲“終了真人”,現今王公藩鎮都稱他爲“神仙”,所以不敢輕慢。二則他又常往兩個府裏去,凡夫人小姐都是見的。今見他如此說,便笑道:“咱們自己,你又說起這話來。再多說,我把你這鬍子還撏了呢!還不跟我進來。”那張道士呵呵大笑,跟了賈珍進來。
\end{parag}


\begin{parag}
    賈珍到賈母跟前,控身陪笑說:“這張爺爺進來請安。”賈母聽了,忙道:“攙他來。”賈珍忙去攙了過來。那張道士先哈哈笑道:“無量壽佛!老祖宗一向福壽安康?衆位奶奶小姐納福?一向沒到府裏請安,老太太氣色越發好了。”賈母笑道:“老神仙,你好?”張道士笑道:“託老太太萬福萬壽,小道也還康健。別的倒罷,只記掛著哥兒,一向身上好?前日四月二十六日,我這裏做遮天大王的聖誕,人也來的少,東西也很乾淨,我說請哥兒來逛逛,怎麼說不在家?”賈母說道: “果真不在家。”一面回頭叫寶玉。誰知寶玉解手去了纔來,忙上前問:“張爺爺好?”張道士忙抱住問了好,又向賈母笑道:“哥兒越發發福了。”賈母道:“他外頭好,裏頭弱。又搭著他老子逼著他念書,生生的把個孩子逼出病來了。”張道士道:“前日我在好幾處看見哥兒寫的字,作的詩,都好的了不得,怎麼老爺還抱怨說哥兒不大喜歡唸書呢?依小道看來,也就罷了。”又嘆道:“我看見哥兒的這個形容身段,言談舉動,怎麼就同當日國公爺一個稿子!”說著兩眼流下淚來。賈母聽說,也由不得滿臉淚痕,說道:“正是呢,我養這些兒子孫子,也沒一個像他爺爺的,就只這玉兒像他爺爺。”
\end{parag}


\begin{parag}
    那張道士又向賈珍道:“當日國公爺的模樣兒,爺們一輩的不用說,自然沒趕上,大約連大老爺、二老爺也記不清楚了。”說畢呵呵又一大笑,道:“前日在一個人家看見一位小姐,今年十五歲了,生的倒也好個模樣兒。我想著哥兒也該尋親事了。若論這個小姐模樣兒,聰明智慧,根基家當,倒也配的過。但不知老太太怎麼樣,小道也不敢造次。等請了老太太的示下,纔敢向人去說。”賈母道:“上回有和尚說了,這孩子命裏不該早娶,等再大一大兒再定罷。你可如今打聽著,不管他根基富貴,只要模樣配的上就好,來告訴我。便是那家子窮,不過給他幾兩銀子罷了。只是模樣性格兒難得好的。”
\end{parag}


\begin{parag}
    說畢,只見鳳姐兒笑道:“張爺爺,我們丫頭的寄名符兒你也不換去。前兒虧你還有那麼大臉,打發人和我要鵝黃緞子去!要不給你,又恐怕你那老臉上過不去。”張道士呵呵大笑道:“你瞧,我眼花了,也沒看見奶奶在這裏,也沒道多謝。符早已有了,前日原要送去的,不指望娘娘來作好事,就混忘了,還在佛前鎮著。待我取來。”說著跑到大殿上去,一時拿了一個茶盤,搭著大紅蟒緞經袱子,托出符來。大姐兒的奶子接了符。張道士方欲抱過大姐兒來,只見鳳姐笑道:“你就手裏拿出來罷了,又用個盤子託著。” 張道士道:“手裏不乾不淨的,怎麼拿?用盤子潔淨些。”鳳姐兒笑道:“你只顧拿出盤子來,倒唬我一跳。我不說你是爲送符,倒象是和我們化佈施來了。”衆人聽說,鬨然一笑,連賈珍也掌不住笑了。賈母回頭道:“猴兒猴兒,你不怕下割舌頭地獄?”鳳姐兒笑道:“我們爺兒們不相干。他怎麼常常的說我該積陰騭,遲了就短命呢!”
\end{parag}


\begin{parag}
    張道士也笑道:“我拿出盤子來一舉兩用,卻不爲化佈施,倒要將哥兒的這玉請了下來,托出去給那些遠來的道友並徒子徒孫們見識見識。”賈母道:“既這們著,你老人家老天拔地的跑什麼,就帶他去瞧了,叫他進來,豈不省事?”張道士道:“老太太不知道,看著小道是八十多歲的人,託老太太的福倒也健壯;二則外面的人多,氣味難聞,況是個暑熱的天,哥兒受不慣,倘或哥兒受了腌臢氣味,倒值多了。”賈母聽說,便命寶玉摘下通靈玉來,放在盤內。那張道士兢兢業業的用蟒袱子墊著,捧了出去。
\end{parag}


\begin{parag}
    這裏賈母與衆人各處遊玩了一回,方去上樓。只見賈珍回說:“張爺爺送了玉來了。”剛說著,只見張道士捧了盤子,走到跟前笑道:“衆人託小道的福,見了哥兒的玉,實在可罕。都沒什麼敬賀之物,這是他們各人傳道的法器,都願意爲敬賀之禮。哥兒便不希罕,只留著在房裏頑耍賞人罷。”賈母聽說,向盤內看時,只見也有金璜,也有玉玦,或有事事如意,或有歲歲平安,皆是珠穿寶貫,玉琢金鏤,共有三五十件。因說道:“你也胡鬧。他們出家人是那裏來的,何必這樣,這不能收。”張道士笑道:“這是他們一點敬心,小道也不能阻擋。老太太若不留下,豈不叫他們看著小道微薄,不象是門下出身了。”賈母聽如此說,方命人接了。寶玉笑道:“老太太,張爺爺既這麼說,又推辭不得,我要這個也無用,不如叫小子們捧了這個,跟著我出去散給窮人罷。”賈母笑道:“這倒說的是。”張道士又忙攔道:“哥兒雖要行好,但這些東西雖說不甚希奇,到底也是幾件器皿。若給了乞丐,一則與他們無益,二則反倒遭塌了這些東西。要舍給窮人,何不就散錢與他們。”寶玉聽說,便命收下,等晚間拿錢施捨罷了。說畢,張道士方退出去。
\end{parag}


\begin{parag}
    這裏賈母與衆人上了樓,在正面樓上歸坐。鳳姐等佔了東樓。衆丫頭等在西樓,輪流伺候。賈珍一時來回:“神前拈了戲,頭一本《白蛇記》。”賈母問: “《白蛇記》是什麼故事?”賈珍道:“是漢高祖斬蛇方起首的故事。第二本是《滿牀笏》。”賈母笑道:“這倒是第二本上?也罷了。神佛要這樣,也只得罷了。”又問第三本,賈珍道:“第三本是《南柯夢》。”賈母聽了便不言語。賈珍退了下來,至外邊預備著申表、焚錢糧、開戲,不在話下。
\end{parag}


\begin{parag}
    且說寶玉在樓上,坐在賈母旁邊,因叫個小丫頭子捧著方纔那一盤子賀物,將自己的玉帶上,用手翻弄尋撥,一件一件的挑與賈母看。賈母因看見有個赤金點翠的麒麟,便伸手拿了起來,笑道:“這件東西好像我看見誰家的孩子也帶著這麼一個的。”寶釵笑道:“史大妹妹有一個,比這個小些。”賈母道:“是雲兒有這個。”寶玉道:“他這麼往我們家去住著,我也沒看見。”探春笑道:“寶姐姐有心,不管什麼他都記得。”林黛玉冷笑道:“他在別的上還有限,惟有這些人帶的東西上越發留心。”寶釵聽說,便回頭裝沒聽見。寶玉聽見史湘雲有這件東西,自己便將那麒麟忙拿起來揣在懷裏。一面心裏又想到怕人看見他聽見史湘雲有了,他就留這件,因此手裏揣著,卻拿眼睛瞟人。只見衆人都倒不大理論,惟有林黛玉瞅著他點頭兒,似有讚歎之意。寶玉不覺心裏沒好意思起來,又掏了出來,向黛玉笑道:“這個東西倒好頑,我替你留著,到了家穿上你帶。”林黛玉將頭一扭,說道:“我不希罕。”寶玉笑道:“你果然不希罕,我少不得就拿著。”說著又揣了起來。
\end{parag}


\begin{parag}
    剛要說話,只見賈珍、賈蓉的妻子婆媳兩個來了,彼此見過,賈母方說:“你們又來做什麼,我不過沒事來逛逛。”一句話沒說了,只見人報:“馮將軍家有人來了。”原來馮紫英家聽見賈府在廟裏打醮,連忙預備了豬羊香燭茶銀之類的東西送禮。鳳姐兒聽了,忙趕過正樓來,拍手笑道:“噯呀!我就不防這個。只說咱們娘兒們來閒逛逛,人家只當咱們大擺齋壇的來送禮。都是老太太鬧的。這又不得不預備賞封兒。”剛說了,只見馮家的兩個管家娘子上樓來了。馮家兩個未去,接著趙侍郎也有禮來了。於是接二連三,都聽見賈府打醮,女眷都在廟裏,凡一應遠親近友,世家相與都來送禮。賈母才後悔起來,說:“又不是什麼正經齋事,我們不過閒逛逛,就想不到這禮上,沒的驚動了人。”因此雖看了一天戲,至下午便回來了,次日便懶怠去。鳳姐又說:“打牆也是動土,已經驚動了人,今兒樂得還去逛逛。”那賈母因昨日張道士提起寶玉說親的事來,誰知寶玉一日心中不自在,回家來生氣,嗔著張道士與他說了親,口口聲聲說從今以後不再見張道士了,別人也並不知爲什麼原故;二則林黛玉昨日回家又中了暑:因此二事,賈母便執意不去了。鳳姐見不去,自己帶了人去,也不在話下。
\end{parag}


\begin{parag}
    且說寶玉因見林黛玉又病了,心裏放不下,飯也懶去喫,不時來問。林黛玉又怕他有個好歹,因說道:“你只管看你的戲去,在家裏作什麼?”寶玉因昨日張道士提親,心中大不受用,今聽見林黛玉如此說,心裏因想道:“別人不知道我的心還可恕,連他也奚落起我來。”因此心中更比往日的煩惱加了百倍。若是別人跟前,斷不能動這肝火,只是林黛玉說了這話,倒比往日別人說這話不同,由不得立刻沉下臉來,說道:“我白認得了你。罷了,罷了!”林黛玉聽說,便冷笑了兩聲:“我也知道白認得了我,那裏像人家有什麼配的上呢。”寶玉聽了,便向前來直問到臉上:“你這麼說,是安心咒我天誅地滅?”林黛玉一時解不過這個話來。寶玉又道:“昨兒還爲這個賭了幾回咒,今兒你到底又準我一句。我便天誅地滅,你又有什麼益處?”林黛玉一聞此言,方想起上日的話來。今日原是自己說錯了,又是著急,又是羞愧,便顫顫兢兢的說道:“我要安心咒你,我也天誅地滅。何苦來!我知道,昨日張道士說親,你怕阻了你的好姻緣,你心裏生氣,來拿我煞性子。”
\end{parag}


\begin{parag}
    原來那寶玉自幼生成有一種下流癡病,況從幼時和黛玉耳鬢廝磨,心情相對;及如今稍明時事,又看了那些邪書僻傳,凡遠親近友之家所見的那些閨英闈秀,皆未有稍及林黛玉者,所以早存了一段心事,只不好說出來,故每每或喜或怒,變盡法子暗中試探。那林黛玉偏生也是個有些癡病的,也每用假情試探。因你也將真心真意瞞了起來,只用假意,我也將真心真意瞞了起來,只用假意,如此兩假相逢,終有一真。其間瑣瑣碎碎,難保不有口角之爭。即如此刻,寶玉的心內想的是: “別人不知我的心,還有可恕,難道你就不想我的心裏眼裏只有你!你不能爲我煩惱,反來以這話奚落堵我。可見我心裏一時一刻白有你,你竟心裏沒我。”心裏這意思,只是口裏說不出來。那林黛玉心裏想著:“你心裏自然有我,雖有‘金玉相對’之說,你豈是重這邪說不重我的?我便時常提這‘金玉’,你只管瞭然自若無聞的,方見得是待我重,而毫無此心了。如何我只一提‘金玉’的事,你就著急,可知你心裏時時有‘金玉’,見我一提,你又怕我多心,故意著急,安心哄我。”
\end{parag}


\begin{parag}
    看來兩個人原本是一個心,但都多生了枝葉,反弄成兩個心了。那寶玉心中又想著:“我不管怎麼樣都好,只要你隨意,我便立刻因你死了也情願。你知也罷,不知也罷,只由我的心,可見你方和我近,不和我遠。”那林黛玉心裏又想著:“你只管你,你好我自好,你何必爲我而自失。殊不知你失我自失。可見是你不叫我近你,有意叫我遠你了。”如此看來,卻都是求近之心,反弄成疏遠之。如此之話,皆他二人素習所存私心,也難備述。
\end{parag}


\begin{parag}
    如今只述他們外面的形容。那寶玉又聽見他說“好姻緣”三個字,越發逆了己意,心裏幹噎,口裏說不出話來,便賭氣向頸上抓下通靈寶玉,咬牙恨命往地下一摔,道:“什麼撈什骨子,我砸了你完事!”偏生那玉堅硬非常,摔了一下,竟文風沒動。寶玉見沒摔碎,便回身找東西來砸。林黛玉見他如此,早已哭起來,說道:“何苦來,你摔砸那啞吧物件。有砸他的,不如來砸我。”二人鬧著,紫鵑雪雁等忙來解勸。後來見寶玉下死力砸玉,忙上來奪,又奪不下來,見比往日鬧的大了,少不得去叫襲人。襲人忙趕了來,才奪了下來。寶玉冷笑道:“我砸我的東西,與你們什麼相干!”
\end{parag}


\begin{parag}
    襲人見他臉都氣黃了,眼眉都變了,從來沒氣的這樣,便拉著他的手,笑道:“你同妹妹拌嘴,不犯著砸他,倘或砸壞了,叫他心裏臉上怎麼過的去?”林黛玉一行哭著,一行聽了這話說到自己心坎兒上來,可見寶玉連襲人不如,越發傷心大哭起來。心裏一煩惱,方纔喫的香薷飲解暑湯便承受不住,“哇”的一聲都吐了出來。紫鵑忙上來用手帕子接住,登時一口一口的把一塊手帕子吐溼。雪雁忙上來捶。紫鵑道:“雖然生氣,姑娘到底也該保重著些。才吃了藥好些,這會子因和寶二爺拌嘴,又吐出來。倘或犯了病,寶二爺怎麼過的去呢?”寶玉聽了這話說到自己心坎兒上來,可見黛玉不如一紫鵑。又見林黛玉臉紅頭脹,一行啼哭,一行氣湊,一行是淚,一行是汗,不勝怯弱。寶玉見了這般,又自己後悔方纔不該同他較證,這會子他這樣光景,我又替不了他。心裏想著,也由不的滴下淚來了。襲人見他兩個哭,由不得守著寶玉也心酸起來,又摸著寶玉的手冰涼,待要勸寶玉不哭罷,一則又恐寶玉有什麼委曲悶在心裏,二則又恐薄了林黛玉。不如大家一哭,就丟開手了,因此也流下淚來。紫鵑一面收拾了吐的藥,一面拿扇子替林黛玉輕輕的扇著,見三個人都鴉雀無聲,各人哭各人的,也由不得傷心起來,也拿手帕子擦淚。四個人都無言對泣。
\end{parag}


\begin{parag}
    一時,襲人勉強笑向寶玉道:“你不看別的,你看看這玉上穿的穗子,也不該同林姑娘拌嘴。”林黛玉聽了,也不顧病,趕來奪過去,順手抓起一把剪子來要剪。襲人紫鵑剛要奪,已經剪了幾段。林黛玉哭道:“我也是白效力。他也不希罕,自有別人替他再穿好的去。”襲人忙接了玉道:“何苦來,這是我纔多嘴的不是了。”寶玉向林黛玉道:“你只管剪,我橫豎不帶他,也沒什麼。”
\end{parag}


\begin{parag}
    只顧裏頭鬧,誰知那些老婆子們見林黛玉大哭大吐,寶玉又砸玉,不知道要鬧到什麼田地,倘或連累了他們,便一齊往前頭回賈母王夫人知道,好不幹連了他們。那賈母王夫人見他們忙忙的作一件正經事來告訴,也都不知有了什麼大禍,便一齊進園來瞧他兄妹。急的襲人抱怨紫鵑爲什麼驚動了老太太、太太,紫鵑又只當是襲人去告訴的,也抱怨襲人。那賈母,王夫人進來,見寶玉也無言,林黛玉也無話,問起來又沒爲什麼事,便將這禍移到襲人紫鵑兩個人身上,說:“爲什麼你們不小心伏侍,這會子鬧起來都不管了!”因此將他二人連罵帶說教訓了一頓。二人都沒話,只得聽著。還是賈母帶出寶玉去了,方纔平服。
\end{parag}


\begin{parag}
    過了一日,至初三日,乃是薛蟠生日,家裏擺酒唱戲,來請賈府諸人。寶玉因得罪了林黛玉,二人總未見面,心中正自後悔,無精打采的,那裏還有心腸去看戲,因而推病不去。林黛玉不過前日中了些暑溽之氣,本無甚大病,聽見他不去,心裏想:“他是好喫酒看戲的,今日反不去,自然是因爲昨兒氣著了。再不然,他見我不去,他也沒心腸去。只是昨兒千不該萬不該剪了那玉上的穗子。管定他再不帶了,還得我穿了他才帶。”因而心中十分後悔。
\end{parag}


\begin{parag}
    那賈母見他兩個都生了氣,只說趁今兒那邊看戲,他兩個見了也就完了,不想又都不去。老人家急的抱怨說:“我這老冤家是那世裏的孽障,偏生遇見了這麼兩個不省事的小冤家,沒有一天不叫我操心。真是俗語說的,‘不是冤家不聚頭’。幾時我閉了這眼,斷了這口氣,憑著這兩個冤家鬧上天去,我眼不見心不煩,也就罷了。偏又不咽這口氣。”自己抱怨著也哭了。這話傳入寶林二人耳內。原來他二人竟是從未聽見過“不是冤家不聚頭”的這句俗語,如今忽然得了這句話,好似參禪的一般,都低頭細嚼此話的滋味,都不覺潸然泣下。雖不曾會面,然一個在瀟湘館臨風灑淚,一個在怡紅院對月長吁,卻不是人居兩地,情發一心!
\end{parag}


\begin{parag}
    襲人因勸寶玉道:“千萬不是,都是你的不是。往日家裏小廝們和他們的姊妹拌嘴,或是兩口子分爭,你聽見了,你還罵小廝們蠢,不能體貼女孩兒們的心。今兒你也這麼著了。明兒初五,大節下,你們兩個再這們仇人似的,老太太越發要生氣,一定弄的大家不安生。依我勸,你正經下個氣,陪個不是,大家還是照常一樣,這麼也好,那麼也好。”那寶玉聽見了不知依與不依,要知端詳,且聽下回分解。
\end{parag}


\begin{parag}
    \begin{note}蒙回末總評:一片哭聲,總因情重;金玉無言,何可爲證?\end{note}
\end{parag}

