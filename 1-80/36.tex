\chap{三十六}{繡鴛鴦夢兆絳芸軒 識分定情悟梨香院}

\begin{parag}
    \begin{note}庚、已卯回前批:絳芸軒夢兆是金針暗度法,夾寫月錢是爲襲人漸入金屋地步(庚:漸入金屋“步位”);梨香院是明寫大家蓄戲不免姦淫之陋,可不慎哉,慎哉!蒙本回末批同。\end{note}
\end{parag}


\begin{parag}
    \begin{note}蒙回前總:造物何當作主張,任人稟受福修長。畫薔亦自非容易,解得臣忠子也良。\end{note}
\end{parag}


\begin{parag}
    話說賈母自王夫人處回來,見寶玉一日好似一日,心中自是歡喜。因怕將來賈政又叫他,遂命人將賈政的親隨小廝頭兒喚來,吩咐他“以後倘有會人待客諸樣的事,你老爺要叫寶玉,你不用上來傳話,就回他說我說了:一則打重了,得着實將養幾個月才走得;二則他的‘星宿不利’,祭了星不見外人,過了八月才許出二門。”那小廝頭兒聽了,領命而去。賈母又命李嬤嬤、襲人等來,將此話說與寶玉,使他放心。那寶玉本就懶與士大夫、諸男人接談,又最厭‘峨冠禮服’,‘賀吊往還’等事,今日得了這句話,越發得了意,不但將親戚、朋友一概杜絕了,而且連家庭中晨昏定省亦發都隨他的便了,日日只在園中游臥,不過每日一清早到賈母、王夫人處走走就回來了,卻每每甘心爲諸丫嬛充役,竟也得十分閒消日月。或如寶釵輩有時見機導勸,反生起氣來,只說:“好好的一個清淨潔白女兒,也學的釣名沽譽,入了國賊祿鬼之流。這總是前人無故生事,立言豎辭,原爲導後世的鬚眉濁物。不想我生不幸,亦且瓊閨繡閣中亦染此風,真真有負天地鍾靈毓秀之德!”\begin{note}蒙側:寶玉何等心思,作者何等意見,此文何等筆墨!\end{note}因此禍延古人,除四書外,竟將別的書焚了。衆人見他如此瘋顛,也都不向他說這些正經話了。獨有林黛玉自幼不曾勸他去立身揚名等語,所以深敬黛玉。
\end{parag}


\begin{parag}
    閒言少述。如今且說王鳳姐自見金釧死後,忽見幾家僕人常來孝敬他些東西,\begin{note}蒙側:爲當塗人一笑。\end{note}又不時的來請安奉承,自己倒生了疑惑,不知何意。這日又見人來孝敬他東西,因晚間無人時笑向平兒道:“這幾家人不大管我的事,爲什麼忽然這麼和我貼近?”平兒冷笑道:“奶奶連這個都想不起來了?我猜他們的女兒都必是太太房裏的丫頭,如今太太房裏有四個大的,一個月一兩銀的分例下,剩的都是一個月幾百錢。如今金釧兒死了,必定他們要弄這兩銀子的巧宗兒呢。”鳳姐聽了,笑道:“是了,是了!倒是你提醒了。我看這些人也太不識足,錢也賺夠了,苦事情又侵不着,弄個丫頭搪塞着身子也就罷了,又還想這個。也罷了!他們幾家的錢,容易也不能花到我跟前,這是他們自尋的!送什麼來,我就收什麼!橫豎我有主意。”\begin{note}蒙側:難見高論!而其心思則不可問矣。任事者戒之。\end{note}鳳姐兒安下這個心,所以自管遷延着,等那些人把東西送足了,然後剩空方回王夫人。
\end{parag}


\begin{parag}
    這日午間,薛姨媽母女兩個與林黛玉等正在王夫人房裏大家喫東西呢,鳳姐兒得便回王夫人道:“自從玉釧兒姐姐死了,太太跟前少着一個人。太太或看準了那個丫頭好,就吩咐,下月好發放月錢的。”王夫人聽了,想了一想,道:“依我說,什麼是例必定四個、五個的,夠使就罷了,竟可以免了罷。”鳳姐笑道:“論理,太太說的也是。這原是舊例,別人屋裏還有兩個呢,太太倒不按例了。況且省下一兩銀子也有限。”王夫人聽了,又想一想,道:“也罷,這個分例只管關了來,不用補人,就把這一兩銀子給他妹妹玉釧兒罷。他姐姐伏侍了我一場,沒個好結果,剩下他妹妹跟着我,喫個雙分子也不爲過於了。”鳳姐答應着,回頭找玉釧兒,笑道:“大喜,大喜。”玉釧兒過來磕了頭。王夫人問道:“正要問你,如今趙姨娘、周姨娘的月例多少?”鳳姐道:“那是定例,每人二兩。趙姨娘有環兄弟的二兩,共是四兩,另外四串錢。”王夫人道:“可都按數給他們?”鳳姐見問的奇怪,忙道:“怎麼不按數給!”王夫人道:“前兒我恍惚聽見有人抱怨,說短了一吊錢,是什麼原故?”鳳姐忙笑道:“姨娘們的丫頭,月例原是人各一吊。從舊年他們外頭商議的,姨娘們每位的丫頭分例減半,人各五百錢,每位兩個丫頭,所以短了一吊錢。這也抱怨不着我,我倒樂得給他們呢,他們外頭又扣着,難道我添上不成?這個事我不過是接手兒,怎麼來,怎麼去,由不得我作主。我倒說了兩三回,仍舊添上這兩分的。他們說只有這個項數,叫我也難再說了。如今我手裏每月,連日子都不錯,給他們呢。先時在外頭關,那個月不打饑荒,何曾順順溜溜的得過一遭兒?”\begin{note}蒙側:能事能言。\end{note}王夫人聽說,也就罷了,半日又問:“老太太屋裏幾個一兩的?”鳳姐道:“八個。如今只有七個,那一個是襲人。”王夫人道:“這就是了。你寶兄弟也幷沒有一兩的丫頭,襲人還算是老太太房裏的人。”鳳姐笑道:“襲人原是老太太的人,不過給了寶兄弟使。他這一兩銀子還在老太太的丫頭分例上領。如今說因爲襲人是寶玉的人,裁了這一兩銀子,斷然使不得。若說再添一個人給老太太,這個還可以裁他的。若不裁他的,須得環兄弟屋裏也添上一個才公道均勻了。就是晴雯麝月等七個大丫頭,每月人各月錢一吊,佳蕙等八個小丫頭,每月人各月錢五百,還是老太太的話,別人如何惱得氣得呢。”薛姨娘笑道:“只聽鳳丫頭的嘴,倒象倒了核桃車子的,只聽他的帳也清楚,理也公道。”鳳姐笑道:“姑媽,難道我說錯了不成?”薛姨媽笑道:“說的何嘗錯,只是你慢些說豈不省力。”鳳姐纔要笑,忙又忍住了,聽王夫人示下。王夫人想了半日,向鳳姐兒道:“明兒挑一個好丫頭送去老太太使,補襲人,把襲人的一分裁了。把我每月的月例二十兩銀子裏,拿出二兩銀子一吊錢來給襲人。\begin{note}蒙側:寫盡慈母苦心。\end{note}以後凡事有趙姨娘、周姨娘的,也有襲人的,只是襲人的這一分都從我的分例上勻出來,不必動官中的就是了。”鳳姐一一的答應了,笑推薛姨媽道:“姑媽聽見了,我素日說的話如何?今兒果然應了我的話。”薛姨媽道:“早就該如此。模樣兒自然不用說的,他的那一種行事大方,說話見人,和氣裏頭帶着剛硬要強,這個實在難得。”王夫人含淚說道:“你們那裏知道襲人那孩子的好處?\begin{note}己、庚、有正、蒙,雙夾:“孩子”二字愈見親熱,故後文連呼二聲“我的兒”。\end{note}比我的寶玉強十倍!\begin{note}己、庚、有正、蒙,雙夾:忽加“我的寶玉”四字,愈令人墮淚,加“我的”二字者,是的顯襲人是“彼的”。然彼的何如此好,我的何如此不好?又氣又恨,寶玉罪有萬重矣。作者有多少眼淚寫此一句,觀者又不知有多少眼淚也。(庚“是彼的”作“是被的”。有正“恨”作“愧”)\end{note}寶玉果然是有造化的,能夠得他長長遠遠的伏侍他一輩子,也就罷了。”\begin{note}己、庚、有正、蒙,雙夾:真好文字,此批得出者。(有正“此批”作“寫”)\end{note}鳳姐道:“既這麼樣,就開了臉,明放他在屋裏豈不好?”王夫人道:“那就不好了,一則都年輕,二則老爺也不許,三則那寶玉見襲人是個丫頭,縱有放縱的事,倒能聽他的勸,如今作了跟前人,那襲人該勸的也不敢十分勸了。\begin{note}蒙側:苦心!作子弟的讀此等文章能不墜淚?\end{note}如今且渾著,等再過二、三年再說說。”
\end{parag}


\begin{parag}
    說畢半日,鳳姐見無話,便轉身出來。剛至廊簷上,只見有幾個執事的媳婦子正等他回事呢,見他出來,都笑道:“奶奶今兒回什麼事,這半天?可是要熱着了。”鳳姐把袖子挽了幾挽,跐着那角門的門檻子,\begin{note}蒙側:能事得意之人如畫。\end{note}笑道:“這裏過門風倒涼快,吹一吹再走。”又告訴衆人道:“你們說我回了這半日的話,太太把二百年頭裏的事都想起來問我,難道我不說罷。”又冷笑道:“我從今以後倒要幹幾樣克毒事了。抱怨給太太聽,我也不怕。糊塗油蒙了心,爛了舌頭,不得好死的下作東西,別作孃的春夢!明兒一裹腦子扣的日子還有呢。\begin{note}蒙側:的真活現。\end{note}如今裁了丫頭的錢,就抱怨了咱們。也不想一想是奴幾\begin{note}蒙本將“是奴己”改爲“自己”\end{note},也配使兩三個丫頭!”一面罵,一面方走了,自去挑人回賈母話去,不在話下。
\end{parag}


\begin{parag}
    卻說王夫人等這裏喫畢西瓜,又說了一回閒話,各自方散去。寶釵與黛玉等回至園中,寶釵因約黛玉往藕香榭去,黛玉回說立刻要洗澡,便各自散了。寶釵獨自行來,順路進了怡紅院,意欲尋寶玉談講,以解午倦。不想一入院來,鴉雀無聞,一幷連兩隻仙鶴在芭蕉下都睡著了。寶釵便順着遊廊來至房中,只見外間牀上橫三豎四,都是丫頭們睡覺。轉過十錦槅子,來至寶玉的房內。寶玉在牀上睡著了,襲人坐在身旁,手裏做針線,旁邊放著一柄白犀麈。寶釵咲近前來,悄悄的笑道:“你也過於小心了,這個屋裏那裏還有蒼蠅蚊子,還拿蠅帚子趕什麼?”襲人不防,猛抬頭見是寶釵,忙放下針線,起身悄悄笑道:“姑娘來了,我倒也不防,唬了一跳。\begin{note}蒙側:問情問景,隨便拈來,便是佳文佳語。\end{note}姑娘不知道,雖然沒有蒼蠅蚊子,誰知有一種小蟲子,從這紗眼裏鑽進來,人也看不見,只睡著了,咬一口,就象螞蟻夾的。”寶釵道:“怨不得,這屋子後頭又近水,又都是香花兒,這屋子裏頭又香。這種蟲子都是花心裏長的,聞香就撲。”說著,一面又瞧他手裏的針線,原來是個白綾紅裹的兜肚,上面扎着鴛鴦戲蓮的花樣,紅蓮綠葉,五色鴛鴦。寶釵道:“噯喲,好鮮亮活計!這是誰的,也值的費這麼大工夫?”襲人向牀上努嘴兒。\begin{note}蒙側:妙形景。\end{note}寶釵笑道:“這麼大了,還帶這個?”襲人笑道:“他原是不帶,所以特特的做的好了,叫他看見由不得不帶。如今天氣熱,睡覺都不留神,哄他帶上了,便是夜裏總蓋不嚴些兒,也就罷了。你說這一個就用了工夫,還沒看見他身上現帶的那一個呢!”寶釵笑道:“也虧你奈煩。”襲人道:“今兒做的工夫大了,脖子仾的怪酸的。”\begin{note}蒙側:隨便寫來,有神有理,生出下文多少故事。\end{note}又笑道:“好姑娘,你略坐一坐,我出去走走就來。”說着便走了。寶釵只顧看着活計,便不留心,一蹲身,剛剛的也坐在襲人方纔坐的所在,因又見那活計實在可愛,不由的拿起針來,替他代刺。
\end{parag}


\begin{parag}
    不想林黛玉因遇見史湘雲約他來與襲人道喜,二人來至院中,見靜悄悄的,湘雲便轉身先到廂房裏去找襲人。林黛玉卻來至窗外,隔着紗窗往裏一看,只見寶玉穿著銀紅紗衫子,隨便睡着在牀上,寶釵坐在身旁做針線,旁邊放着蠅帚子,林黛玉見了這個景兒,連忙把身子一藏,手握着嘴不敢笑出來,招手兒叫湘雲。湘雲一見他這般景況,只當有什麼新聞,忙也來一看,也要笑時,忽然想起寶釵素日待他厚道,便忙掩住口。知道林黛玉不讓人,怕他言語之中取笑,便忙拉過他來道:“走罷。我想起襲人來,他說午間要到池子裏去洗衣裳,想必去了,咱們那裏找他去。”林黛玉心下明白,冷笑了兩聲,只得隨他走了。\begin{note}蒙側:觸眼偏生礙,多心偏是癡。萬魔隨事起,何日是完時。\end{note}
\end{parag}


\begin{parag}
    這裏寶釵只剛做了兩三個花瓣,忽見寶玉在夢中喊罵說:“和尚道士的話如何信得?什麼是金玉姻緣,我偏說是木石姻緣!”薛寶釵聽了這話,不覺怔了。\begin{note}蒙側:請問:此“怔了”是囈語之故,還是囈語之意不妥之故?猜猜。\end{note}忽見襲人走過來,笑道:“還沒有醒呢。”寶釵搖頭。襲人又笑道:“我才碰見林姑娘、史大姑娘,他們可有進來?”寶釵道:“沒見他們進來。”因向襲人笑道:“他們沒告訴你什麼話?”襲人笑道:“左不過是他們那些玩話,有什麼正經說的。”寶釵笑道:“他們說的可不是玩話,我正要告訴你呢,你又忙忙的出去了。”
\end{parag}


\begin{parag}
    一句話未完,只見鳳姐兒打發人來叫襲人。寶釵笑道:“就是爲那話了。”襲人只得喚起兩個丫嬛來,一同寶釵出怡紅院,自往鳳姐這裏來。果然是告訴他這話,又叫他與王夫人叩頭,且不必去見賈母,倒把襲人不好意思的。見過王夫人急忙回來,寶玉已醒了,問起原故,襲人且含糊答應,至夜間人靜,襲人方告訴。\begin{note}蒙側:夜深人靜時,不減長生殿風味。何等告法?何等聽法?人生不遇此等景況,實辜負此一生!\end{note}寶玉喜不自禁,又向他笑道:“我可看你回家去不去了!那一回往家裏走了一趟,回來就說你哥哥要贖你,又說在這裏沒着落,終久算什麼,說了那麼些無情無義的生分話唬我。\begin{note}己、庚、有正、蒙,雙夾:“唬”字妙!爾果條明決男子,何得畏女子唬哉?(有正、蒙批:“條”作“系”;有正:“何得畏”作“何得”)\end{note}從今以後,我可看誰來敢叫你去。”襲人聽了,便冷笑道:“你倒別這麼說。從此以後我是太太的人了,我要走連你也不必告訴,只回了太太就走。”寶玉笑道:“就便算我不好,你回了太太竟去了,教別人聽見說我不好,你去了你也沒意思。”襲人笑道:“有什麼沒意思,難道作了強盜賊,我也跟着罷。再不然,還有一個死呢:人活百歲,橫豎要死,這一口氣不在,聽不見,看不見,就罷了!”\begin{note}蒙側:自古至今,大凡大英雄大豪傑,忠臣孝子,至其真極,不過一死,嗚呼哀哉!\end{note}寶玉聽見這話,便忙握他的嘴,說道:“罷,罷,罷,不用說這些話了。”襲人深知寶玉性情古怪,聽見奉承吉利話,又厭虛而不實,聽了這些盡情實話,又生悲感,便悔自己說冒撞了,連忙笑着用話截開,只揀那寶玉素喜談者問之。先問他春風秋月,再談及粉淡脂瑩,然後談到女兒如何好,又談到女兒死,襲人忙掩住口。寶玉談至濃快時,見他不說了,便笑道:“人誰不死,只要死的好。那些個鬚眉濁物,只知道文死諫,武死戰,這二死是大丈夫死名死節。竟何如不死的好:必定有昏君,他方諫,他只顧邀名,猛拚一死,將來棄君於何地!必定有刀兵,他方戰,猛拚一死,他只顧圖汗馬之名,將來棄國於何地!所以這皆非正死。”\begin{note}庚眉:玉兄此論大覺痛快人心。綺園。\end{note}襲人道:“忠臣良將,出於不得已他才死。”寶玉道:“那武將不過仗血氣之勇,踈謀少略,他自己無能,送了性命,這難道也是不得已!那文官更不可比武官了,他念兩句書污在心裏,若朝廷少有疵瑕,他就胡談亂勸,只顧他邀忠烈之名,濁氣一湧,即時棄死,這難道也是不得已!還要知道,那朝廷是受命於天,他不聖不仁,那天地斷不把這萬幾重任與他了。可知那些死的都是沽名,幷不知大義。\begin{note}蒙側:此一段議論文武之死,真真確確的非凡常可能道者。庚眉:死時當知大義,千古不磨之論。綺園。\end{note}比如我此時若果有造化,該死於時的,如今趁你們在,我就死了,再能夠你們哭我的眼淚流成大河,把我的屍首漂起來,送到那鴉雀不到的幽僻之處,隨風化了,自此再不要託生爲人,就是我死的得時了。”襲人忽見說出這些瘋話來,忙說困了,不理他。那寶玉方閤眼睡著,至次日也就丟開了。
\end{parag}


\begin{parag}
    一日,寶玉因各處遊的煩膩,便想起《牡丹亭》曲來,自己看了兩遍,猶不愜懷,因聞得梨香院的十二個女孩子中有小旦齡官最是唱的好,因著意出角門來找時,只見寶官玉官都在院內,見寶玉來了,都笑讓坐。寶玉因問:“齡官獨在那裏?”衆人都告訴他說:“在他房裏呢。”寶玉忙至他房內,只見齡官獨自倒在枕上,見他進來,文風不動。\begin{note}蒙側:另有風味。\end{note}寶玉素習與別的女孩子頑慣了的,只當齡官也同別人一樣,因進前來身旁坐下,又陪笑央他起來唱“嫋晴絲”一套。不想齡官見他坐下,忙抬身起來躲避,正色說道:“嗓子啞了。前兒娘娘傳進我們去,我還沒有唱呢。”寶玉見他坐正了,再一細看,原來就是那日薔薇花下劃“薔”字那一個。又見如此景況,從來未經過這番,被人棄厭自己,便訕訕的紅了臉,只得出來了。寶官等不解何故,因問其所以。寶玉便說了,遂出來。\begin{note}蒙側:非齡官不能如此做事,非寶玉不能如此忍。其文冷中濃具意蘊而。誠有“富貴不能移,威武不能屈”之意。\end{note}寶官便說道:“只略等一等,薔二爺來了叫他唱,是必唱的。”寶玉聽了,心下納悶,因問:“薔哥兒那去了?”寶官道:“纔出去了,一定還是齡官要什麼,他去變弄去了。”
\end{parag}


\begin{parag}
    寶玉聽了,以爲奇特,少站片時,果見賈薔從外頭來了,手裏又提着個雀兒籠子,上面扎着個小戲臺,幷一個雀兒,興頭頭往裏走着找齡官。見了寶玉,只得站住。寶玉問他:“是個什麼雀兒,會啣旗串戲臺?”賈薔笑道:“是個玉頂金豆。”寶玉道:“多少錢買的?”賈薔道:“一兩八錢銀子。”一面說,一面讓寶玉坐,自己往齡官房裏來。寶玉此刻把聽曲子的心都沒了,且要看他和齡官是怎樣。只見賈薔進去笑道:“你起來,瞧這個頑意兒。”齡官起身問是什麼,賈薔道:“買了雀兒你頑,省得天天悶悶的無個開心。我先頑着你看。”說着,便拿些穀子哄的那個雀兒果然在戲臺上亂串,啣鬼臉旗幟。衆女孩子都笑道“有趣”,獨齡官冷笑了兩聲,賭氣仍睡去了。賈薔還只管陪笑,問他好不好。齡官道:“你們家把好好的人弄了來,關在這牢坑裏學這個勞什子還不算,你這會子又弄個雀兒來,也偏生幹這個。你分明是弄了他來打趣形容我們,還問我好不好。”賈薔聽了,不覺慌起來,連忙賭身立誓。又道:“今兒我那裏的脂油蒙了心!費一二兩銀子買他來,原說解悶,就沒有想到這上頭。罷,罷,放了生,免免你的災病。”\begin{note}蒙側:此一番文章爲畫薔而來薔之畫爲不謬矣。\end{note}說着,果然將雀兒放了,一頓把將籠子拆了。齡官還說:“那雀兒雖不如人,他也有個老雀兒在窩裏,你拿了他來弄這個勞什子也忍得!今兒我咳嗽出兩口血來,太太叫大夫來瞧,不說替我細問問,你且弄這個來取笑。偏生我這沒人管沒人理的,又偏病。”說着又哭起來。賈薔忙道:“昨兒晚上我問了大夫,他說不相干。他說喫兩劑藥,後兒再瞧。誰知今兒又吐了。這會子請他去。”說着,便要請去。齡官又叫“站住,這會子大毒日頭地下,你賭氣子去請了來,我也不瞧!”賈薔聽如此說,只得又站住。寶玉見了這般景況,不覺癡了,這才領會了劃“薔”深意。自己站不住,便抽身走了。賈薔一心都在齡官身上,也不顧送,倒是別的女孩子送了出來。
\end{parag}


\begin{parag}
    那寶玉一心裁奪盤算,癡癡的回至怡紅院中,正值林黛玉和襲人坐着說話兒呢。寶玉一進來,就和襲人長嘆,說道:“我昨晚上的話竟說錯了,怪道老爺說我是‘管窺蠡測’。昨夜說你們的眼淚單葬我,這就錯了。我竟不能全得了。從此後只是各人各得眼淚罷了。”\begin{note}蒙側:這樣悟了,纔是真悟。\end{note}襲人昨夜不過是些頑話,已經忘了,不想寶玉今又提起來,便笑道:“你可真真有些瘋了。”寶玉默默不對,自此深悟人生情緣,各有分定,只是每每暗傷“不知將來葬我灑淚者爲誰?”此皆寶玉心中所懷,也不可十分妄擬。
\end{parag}


\begin{parag}
    且說林黛玉當下見了寶玉如此形像,便知是又從那裏着了魔來,也不便多問,因向他說道:“我纔在舅母跟前的,明兒是薛姨媽的生日,叫我順便來問你出去不出去?你打發人,前頭說一聲去。”寶玉道:“上回連大老爺的生日我也沒去,這會子我又去,倘或碰見了人呢?我一概都不去。這麼怪熱的,又穿衣裳,我不去,姨媽也未必惱。”襲人忙道:“這是什麼話?他比不得大老爺。這裏又住的近,又是親戚,你不去,豈不叫他思量!你怕熱,只清早起到那裏磕個頭,喫鍾茶再來,豈不好看?”寶玉未說話,黛玉便先笑道:“你看着人家趕蚊子的分上,也該去走走。”寶玉不解,忙問:“怎麼趕蚊子?”襲人便將昨日睡覺無人作伴,寶姑娘坐了一坐的話說了出來。寶玉聽了,忙說:“不該。我怎麼睡著了,褻瀆了他。”一面又說:“明日必去!”
\end{parag}


\begin{parag}
    正說着,忽見史湘雲穿的齊齊整整,走來辭說家裏打發人來接他。寶玉、林黛玉聽說,忙站起來讓坐。史湘雲也不坐,寶、林兩個只得送他至前面。那史湘雲只是眼淚汪汪的,見有他家人在跟前,又不敢十分委曲。少時薛寶釵趕來,愈覺繾綣難捨。還是寶釵心內明白,他家人若回去告訴了他嬸孃,待他家去恐受氣,因此倒催他走了。衆人送至二門前,寶玉還要往外送,\begin{note}己、庚、有正、蒙,雙夾:每逢此時就忘卻嚴父,可知前雲“爲你們死也情願”不假。\end{note}倒是湘雲攔住了。一時,回身又叫寶玉到跟前,悄悄的囑道:“便是老太太想不起我來,你時常提着打發人接我去。”寶玉連連答應了。眼看着他上車去了,大家方纔進來。要知端的,且聽下回分解正是。
\end{parag}


\begin{parag}
    \begin{note}蒙回末總:絳雲軒夢兆是金針暗渡法,夾寫月錢是爲襲人漸入金屋地步,梨香院是明寫大家蓄戲不免姦淫之陋可。慎哉慎哉!\end{note}
\end{parag}

