\chap{八十三}{省宮闈賈元妃染恙 鬧閨閫薛寶釵吞聲}



\begin{parag}
    話說探春湘雲纔要走時,忽聽外面一個人嚷道:“你這不成人的小蹄子!你是個什麼東西,來這園子裏頭混攪!”黛玉聽了,大叫一聲道:“這裏住不得了。”一手指着窗外,兩眼反插上去。原來黛玉住在大觀園中,雖靠着賈母疼愛,然在別人身上,凡事終是寸步留心。聽見窗外老婆子這樣罵着,在別人呢,一句是貼不上的,竟象專罵着自己的。自思一個千金小姐,只因沒了爹孃,不知何人指使這老婆子來這般辱罵,那裏委屈得來,因此肝腸崩裂,哭暈去了。紫鵑只是哭叫:“姑娘怎麼樣了,快醒轉來罷。”探春也叫了一回。半晌,黛玉回過這口氣,還說不出話來,那隻手仍向窗外指着。
\end{parag}


\begin{parag}
    探春會意,開門出去,看見老婆子手中拿着柺棍趕着一個不乾不淨的毛丫頭道:“我是爲照管這園中的花果樹木來到這裏,你作什麼來了!等我家去打你一個知道。”這丫頭扭着頭,把一個指頭探在嘴裏,瞅着老婆子笑。探春罵道:“你們這些人如今越發沒了王法了,這裏是你罵人的地方兒嗎!”老婆子見是探春,連忙陪着笑臉兒說道:“剛纔是我的外孫女兒,看見我來了他就跟了來。我怕他鬧,所以才吆喝他回去,那裏敢在這裏罵人呢。”探春道:“不用多說了,快給我都出去。這裏林姑娘身上不大好,還不快去麼。”老婆子答應了幾個“是”,說着一扭身去了。那丫頭也就跑了。
\end{parag}


\begin{parag}
    探春回來,看見湘雲拉着黛玉的手只管哭,紫鵑一手抱着黛玉,一手給黛玉揉胸口,黛玉的眼睛方漸漸的轉過來了。探春笑道:“想是聽見老婆子的話,你疑了心了麼?”黛玉只搖搖頭兒。探春道:“他是罵他外孫女兒,我纔剛也聽見了。這種東西說話再沒有一點道理的,他們懂得什麼避諱。”黛玉聽了點點頭兒,拉着探春的手道:“妹妹……。”叫了一聲,又不言語了。探春又道:“你別心煩。我來看你是姊妹們應該的,你又少人伏侍。只要你安心肯吃藥,心上把喜歡事兒想想,能夠一天一天的硬朗起來,大家依舊結社做詩,豈不好呢。”湘雲道:“可是三姐姐說的,那麼着不樂?”黛玉哽咽道:“你們只顧要我喜歡,可憐我那裏趕得上這日子,只怕不能夠了!”探春道:“你這話說的太過了。誰沒個病兒災兒的,那裏就想到這裏來了。你好生歇歇兒罷,我們到老太太那邊,回來再看你。你要什麼東西,只管叫紫鵑告訴我。”黛玉流淚道:“好妹妹,你到老太太那裏只說我請安,身上略有點不好,不是什麼大病,也不用老太太煩心的。”探春答應道:“我知道,你只管養着罷。”說着,才同湘雲出去了。
\end{parag}


\begin{parag}
    這裏紫鵑扶着黛玉躺在牀上,地下諸事,自有雪雁照料,自己只守着旁邊,看着黛玉,又是心酸,又不敢哭泣。那黛玉閉着眼躺了半晌,那裏睡得着?覺得園裏頭平日只見寂寞,如今躺在牀上,偏聽得風聲,蟲鳴聲,鳥語聲,人走的腳步聲,又象遠遠的孩子們啼哭聲,一陣一陣的聒噪的煩躁起來,因叫紫鵑放下帳子來。雪雁捧了一碗燕窩湯遞與紫鵑,紫鵑隔着帳子輕輕問道:“姑娘喝一口湯罷?”黛玉微微應了一聲。紫鵑復將湯遞給雪雁,自己上來攙扶黛玉坐起,然後接過湯來,擱在脣邊試了一試,一手摟着黛玉肩臂,一手端着湯送到脣邊。黛玉微微睜眼喝了兩三口,便搖搖頭兒不喝了。紫鵑仍將碗遞給雪雁,輕輕扶黛玉睡下。
\end{parag}


\begin{parag}
    靜了一時,略覺安頓。只聽窗外悄悄問道:“紫鵑妹妹在家麼?”雪雁連忙出來,見是襲人,因悄悄說道:“姐姐屋裏坐着。”襲人也便悄悄問道:“姑娘怎麼着?”一面走,一面雪雁告訴夜間及方纔之事。襲人聽了這話,也唬怔了,因說道:“怪道剛纔翠縷到我們那邊,說你們姑娘病了,唬的寶二爺連忙打發我來看看是怎麼樣。”正說着,只見紫鵑從裏間掀起簾子望外看,見襲人,點頭兒叫他。襲人輕輕走過來問道:“姑娘睡着了嗎?”紫鵑點點頭兒,問道:“姐姐才聽見說了?”襲人也點點頭兒,蹙着眉道:“終久怎麼樣好呢!那一位昨夜也把我唬了個半死兒。”紫鵑忙問怎麼了,襲人道:“昨日晚上睡覺還是好好兒的,誰知半夜裏一迭連聲的嚷起心疼來,嘴裏胡說白道,只說好象刀子割了去的似的。直鬧到打亮梆子以後纔好些了。你說唬人不唬人。今日不能上學,還要請大夫來吃藥呢。”正說着,只聽黛玉在帳子裏又咳嗽起來。紫鵑連忙過來捧痰盒兒接痰。黛玉微微睜眼問道:“你和誰說話呢?”紫鵑道:“襲人姐姐來瞧姑娘來了。”說着,襲人已走到牀前。黛玉命紫鵑扶起,一手指着牀邊,讓襲人坐下。襲人側身坐了,連忙陪着笑勸道:“姑娘倒還是躺着罷。”黛玉道:“不妨,你們快別這樣大驚小怪的。剛纔是說誰半夜裏心疼起來?”襲人道:是寶二爺偶然魘住了,不是認真怎麼樣。”黛玉會意,知道是襲人怕自己又懸心的原故,又感激,又傷心。因趁勢問道:“既是魘住了,不聽見他還說什麼?”襲人道:“也沒說什麼。”黛玉點點頭兒,遲了半日,嘆了一聲,才說道:“你們別告訴寶二爺說我不好,看耽擱了他的工夫,又叫老爺生氣。”襲人答應了,又勸道:“姑娘還是躺躺歇歇罷。”黛玉點頭,命紫鵑扶着歪下。襲人不免坐在旁邊,又寬慰了幾句,然後告辭,回到怡紅院,只說黛玉身上略覺不受用,也沒什麼大病。寶玉才放了心。
\end{parag}


\begin{parag}
    且說探春湘雲出了瀟湘館,一路往賈母這邊來。探春因囑咐湘雲道:“妹妹,回來見了老太太,別象剛纔那樣冒冒失失的了。”湘雲點頭笑道:“知道了,我頭裏是叫他唬的忘了神了。”說着,已到賈母那邊。探春因提起黛玉的病來。賈母聽了自是心煩,因說道:“偏是這兩個玉兒多病多災的。林丫頭一來二去的大了,他這個身子也要緊。我看那孩子太是個心細。”衆人也不敢答言。賈母便向鴛鴦道:“你告訴他們,明兒大夫來瞧了寶玉,就叫他到林姑娘那屋裏去。”鴛鴦答應着,出來告訴了婆子們,婆子們自去傳話。這裏探春湘雲就跟着賈母吃了晚飯,然後同回園中去。不提。到了次日,大夫來了,瞧了寶玉,不過說飲食不調,着了點兒風邪,沒大要緊,疏散疏散就好了。這裏王夫人鳳姐等一面遣人拿了方子回賈母,一面使人到瀟湘館告訴說大夫就過來。紫鵑答應了,連忙給黛玉蓋好被窩,放下帳子。雪雁趕着收拾房裏的東西。一時賈璉陪着大夫進來了,便說道:“這位老爺是常來的,姑娘們不用迴避。”老婆子打起簾子,賈璉讓着進入房中坐下。賈璉道”紫鵑姐姐,你先把姑娘的病勢向王老爺說說。”王大夫道:“且慢說。等我診了脈,聽我說了看是對不對,若有不合的地方,姑娘們再告訴我。”紫鵑便向帳中扶出黛玉的一隻手來,擱在迎手上。紫鵑又把鐲子連袖子輕輕的摟起,不叫壓住了脈息。那王大夫診了好一回兒,又換那隻手也診了,便同賈璉出來,到外間屋裏坐下,說道:“六脈皆弦,因平日鬱結所致。”說着,紫鵑也出來站在裏間門口。那王大夫便向紫鵑道:“這病時常應得頭暈,減飲食,多夢,每到五更,必醒個幾次。即日間聽見不幹自己的事,也必要動氣,且多疑多懼。不知者疑爲性情乖誕,其實因肝陰虧損,心氣衰耗,都是這個病在那裏作怪。不知是否?”紫鵑點點頭兒,向賈璉道:“說的很是。”王太醫道:“既這樣就是了。”說畢起身,同賈璉往外書房去開方子。小廝們早已預備下一張梅紅單帖,王太醫吃了茶,因提筆先寫道:
\end{parag}


\begin{qute2sp}
    六脈弦遲,素由積鬱。左寸無力,心氣已衰。關脈獨洪,肝邪偏旺。木氣不能疏達,勢必上侵脾土,飲食無味,甚至勝所不勝,肺金定受其殃。氣不流精,凝而爲痰,血隨氣湧,自然咳吐。理宜疏肝保肺,涵養心脾。雖有補劑,未可驟施。姑擬黑逍遙以開其先,複用歸肺固金以繼其後。不揣固陋,俟高明裁服。
\end{qute2sp}


\begin{parag}
    又將七味藥與引子寫了。賈璉拿來看時,問道:“血勢上衝,柴胡使得麼?”王大夫笑道:“二爺但知柴胡是升提之品,爲吐衄所忌。豈知用鱉血拌炒,非柴胡不足宣少陽甲膽之氣。以鱉血制之,使其不致升提,且能培養肝陰,制遏邪火。所以《內經》說:‘通因通用,塞因塞用。’柴胡用鱉血拌炒,正是‘假周勃以安劉’的法子。”賈璉點頭道:“原來是這麼着,這就是了。”王夫人又道:“先請服兩劑,再加減或再換方子罷。我還有一點小事,不能久坐,容日再來請安。”說着,賈璉送了出來,說道:“舍弟的藥就是那麼着了?”王大夫道:“寶二爺倒沒什麼大病,大約再喫一劑就好了。”說着,上車而去。
\end{parag}


\begin{parag}
    這裏賈璉一面叫人抓藥。一面回到房中告訴鳳姐黛玉的病原與大夫用的藥,述了一遍。只見周瑞家的走來回了幾件沒要緊的事,賈璉聽到一半,便說道:“你回二奶奶罷,我還有事呢。”說着就走了。周瑞家的回完了這件事,又說道:“我方纔到林姑娘那邊,看他那個病,竟是不好呢。臉上一點血色也沒有,摸了摸身上,只剩得一把骨頭。問問他,也沒有話說,只是淌眼淚。回來紫鵑告訴我說:‘姑娘現在病着,要什麼自己又不肯要,我打算要問二奶奶那裏支用一兩個月的月錢。如今吃藥雖是公中的,零用也得幾個錢。’我答應了他,替他來回奶奶。”鳳姐低了半日頭,說道:“竟這麼着罷:我送他幾兩銀子使罷,也不用告訴林姑娘。這月錢卻是不好支的,一個人開了例,要是都支起來,那如何使得呢。你不記得趙姨娘和三姑娘拌嘴了,也無非爲的是月錢。況且近來你也知道,出去的多,進來的少,總繞不過彎兒來。不知道的,還說我打算的不好,更有那一種嚼舌根的,說我搬運到孃家去了。周嫂子,你倒是那裏經手的人,這個自然還知道些。”周瑞家的道:“真正委屈死人!這樣大門頭兒,除了奶奶這樣心計兒當家罷了。別說是女人當不來,就是三頭六臂的男人,還撐不住呢。還說這些個混賬話。”說着,又笑了一聲,道:“奶奶還沒聽見呢,外頭的人還更糊塗呢。前兒周瑞回家來,說起外頭的人打諒着咱們府裏不知怎麼樣有錢呢。也有說‘賈府裏的銀庫幾間,金庫幾間,使的傢伙都是金子鑲了玉石嵌了的。’也有說‘姑娘做了王妃,自然皇上家的東西分的了一半子給孃家。前兒貴妃娘娘省親回來,我們還親見他帶了幾車金銀回來,所以家裏收拾擺設的水晶宮似的。那日在廟裏還願,花了幾萬銀子,只算得牛身上拔了一根毛罷咧。’有人還說‘他門前的獅子只怕還是玉石的呢。園子裏還有金麒麟,叫人偷了一個去,如今剩下一個了。家裏的奶奶姑娘不用說,就是屋裏使喚的姑娘們,也是一點兒不動,喝酒下棋,彈琴畫畫,橫豎有伏侍的人呢。單管穿羅罩紗,喫的戴的,都是人家不認得的。那些哥兒姐兒們更不用說了,要天上的月亮,也有人去拿下來給他頑。’還有歌兒呢,說是‘寧國府,榮國府,金銀財寶如糞土。喫不窮,穿不窮,算來……’”說到這裏,猛然嚥住。原來那時歌兒說道是”算來總是一場空”。這周瑞家的說溜了嘴,說到這裏,忽然想起這話不好,因嚥住了。鳳姐兒聽了,已明白必是句不好的話了。也不便追問,因說道:“那都沒要緊。只是這金麒麟的話從何而來?”周瑞家的笑道:“就是那廟裏的老道士送給寶二爺的小金麒麟兒。後來丟了幾天,虧了史姑娘撿着還了他,外頭就造出這個謠言來了。奶奶說這些人可笑不可笑?”鳳姐道:“這些話倒不是可笑,倒是可怕的。咱們一日難似一日,外面還是這麼講究。俗語兒說的,‘人怕出名豬怕壯’,況且又是個虛名兒,終久還不知怎麼樣呢。”周瑞家的道:“奶奶慮的也是。只是滿城裏茶坊酒鋪兒以及各衚衕兒都是這樣說,並且不是一年了,那裏握的住衆人的嘴。”鳳姐點點頭兒,因叫平兒稱了幾兩銀子,遞給周瑞家的,道:“你先拿去交給紫鵑,只說我給他添補買東西的。若要官中的,只管要去,別提這月錢的話。他也是個伶透人,自然明白我的話。我得了空兒,就去瞧姑娘去。”周瑞家的接了銀子,答應着自去。不提。
\end{parag}


\begin{parag}
    且說賈璉走到外面,只見一個小廝迎上來回道:“大老爺叫二爺說話呢。”賈璉急忙過來,見了賈赦。賈赦道:“方纔風聞宮裏頭傳了一個太醫院御醫,兩個吏目去看病,想來不是宮女兒下人了。這幾天娘娘宮裏有什麼信兒沒有?”賈璉道:“沒有。”賈赦道:“你去問問二老爺和你珍大哥。不然,還該叫人去到太醫院裏打聽打聽纔是。”賈璉答應了,一面吩咐人往太醫院去,一面連忙去見賈政賈珍。賈政聽了這話,因問道:“是那裏來的風聲?”賈璉道:“是大老爺才說的。”賈政道:“你索性和你珍大哥到裏頭打聽打聽。”賈璉道:“我已經打發人往太醫院打聽去了。”一面說着,一面退出來,去找賈珍。只見賈珍迎面來了,賈璉忙告訴賈珍。賈珍道:“我正爲也聽見這話,來回大老爺二老爺去的。”於是兩個人同着來見賈政。賈政道:“如系元妃,少不得終有信的。”說着,賈赦也過來了。到了晌午,打聽的人尚未回來。門上人進來,回說:“有兩個內相在外要見二位老爺呢。”賈赦道:“請進來。”門上的人領了老公進來。賈赦賈政迎至二門外,先請了娘娘的安,一面同着進來,走至廳上讓了坐。老公道:“前日這裏貴妃娘娘有些欠安。昨日奉過旨意,宣召親丁四人進裏頭探問。許各帶丫頭一人,餘皆不用。親丁男人只許在宮門外遞個職名,請安聽信,不得擅入。準於明日辰巳時進去,申酉時出來。”賈政賈赦等站着聽了旨意,復又坐下,讓老公喫茶畢,老公辭了出去。
\end{parag}


\begin{parag}
    賈赦賈政送出大門,回來先稟賈母。賈母道:“親丁四人,自然是我和你們兩位太太了。那一個人呢?”衆人也不敢答言,賈母想了一想,道:“必得是鳳姐兒,他諸事有照應。你們爺兒們各自商量去罷。”賈赦賈政答應了出來,因派了賈璉賈蓉看家外,凡文字輩至草字輩一應都去。遂吩咐家人預備四乘綠轎,十餘輛大車,明兒黎明伺候。家人答應去了。賈赦賈政又進去回明老太太,辰巳時進去,申酉時出來,今日早些歇歇,明日好早些起來收拾進宮。賈母道:“我知道,你們去罷。”赦政等退出。這裏邢夫人王夫人,鳳姐兒也都說了一會子元妃的病,又說了些閒話,才各自散了。
\end{parag}


\begin{parag}
    次日黎明,各間屋子丫頭們將燈火俱已點齊,太太們各梳洗畢,爺們亦各整頓好了。一到卯初,林之孝和賴大進來,至二門口回道:“轎車俱已齊備,在門外伺候着呢。”不一時,賈赦邢夫人也過來了。大家用了早飯。鳳姐先扶老太太出來,衆人圍隨,各帶使女一人,緩緩前行。又命李貴等二人先騎馬去外宮門接應,自己家眷隨後。文字輩至草字輩各自登車騎馬,跟着衆家人,一齊去了。賈璉賈蓉在家中看家。
\end{parag}


\begin{parag}
    且說賈家的車輛轎馬俱在外西垣門口歇下等着。一回兒,有兩個內監出來說:“賈府省親的太太奶奶們,着令入宮探問,爺們俱着令內宮門外請安,不得入見。”門上人叫快進去。賈府中四乘轎子跟着小內監前行,賈家爺們在轎後步行跟着,令衆家人在外等候。走近宮門口,只見幾個老公在門上坐着,見他們來了,便站起來說道:“賈府爺們至此。”賈赦賈政便捱次立定。轎子抬至宮門口,便都出了轎。早有幾個小內監引路,賈母等各有丫頭扶着步行。走至元妃寢宮,只見奎壁輝煌,琉璃照耀。又有兩個小宮女兒傳諭道:“只用請安,一概儀注都免。”賈母等謝了恩,來至牀前請安畢,元妃都賜了坐。賈母等又告了坐。元妃便向賈母道:“近日身上可好?”賈母扶着小丫頭,顫顫巍巍站起來,答應道:“託娘娘洪福,起居尚健。”元妃又向邢夫人王夫人問了好,邢王二夫人站着回了話。元妃又問鳳姐家中過的日子若何,鳳姐站起來回奏道:“尚可支持。”元妃道:“這幾年來難爲你操心。”鳳姐正要站起來回奏,只見一個宮女傳進許多職名,請娘娘龍目。元妃看時,就是賈赦賈政等若干人。那元妃看了職名,眼圈兒一紅,止不住流下淚來。宮女兒遞過絹子,元妃一面拭淚,一面傳諭道:“今日稍安,令他們外面暫歇。”賈母等站起來,又謝了恩。元妃含淚道:“父女弟兄,反不如小家子得以常常親近。”賈母等都忍着淚道:“娘娘不用悲傷,家中已託着娘娘的福多了。”元妃又問:“寶玉近來若何?”賈母道:“近來頗肯唸書。因他父親逼得嚴緊,如今文字也都做上來了。”元妃道:“這樣纔好。”遂命外宮賜宴,便有兩個宮女兒,四個小太監引了到一座宮裏,已擺得齊整,各按坐次坐了。不必細述。一時喫完了飯,賈母帶着他婆媳三人謝過宴,又耽擱了一回。看看已近酉初,不敢羈留,俱各辭了出來。元妃命宮女兒引道,送至內宮門,門外仍是四個小太監送出。賈母等依舊坐着轎子出來,賈赦接着,大夥兒一齊回去。到家又要安排明後日進宮,仍令照應齊集。不題。
\end{parag}


\begin{parag}
    且說薛家夏金桂趕了薛蟠出去,日間拌嘴沒有對頭,秋菱又住在寶釵那邊去了,只剩得寶蟾一人同住。既給與薛蟠作妾,寶蟾的意氣又不比從前了。金桂看去更是一個對頭,自己也後悔不來。一日,吃了幾杯悶酒,躺在炕上,便要借那寶蟾做個醒酒湯兒,因問着寶蟾道:“大爺前日出門,到底是到那裏去?你自然是知道的了。”寶蟾道:“我那裏知道。他在奶奶跟前還不說,誰知道他那些事!”金桂冷笑道:“如今還有什麼奶奶太太的,都是你們的世界了。別人是惹不得的,有人護庇着,我也不敢去虎頭上捉蝨子。你還是我的丫頭,問你一句話,你就和我摔臉子,說塞話。你既這麼有勢力,爲什麼不把我勒死了,你和秋菱不拘誰做了奶奶,那不清淨了麼!偏我又不死,礙着你們的道兒。”寶蟾聽了這話,那裏受得住,便眼睛直直的瞅着金桂道:“奶奶這些閒話只好說給別人聽去!我並沒和奶奶說什麼。奶奶不敢惹人家,何苦來拿着我們小軟兒出氣呢。正經的,奶奶又裝聽不見,‘沒事人一大堆’了。”說着,便哭天哭地起來。金桂越發性起,便爬下炕來,要打寶蟾。寶蟾也是夏家的風氣,半點兒不讓。金桂將桌椅杯盞,盡行打翻,那寶蟾只管喊冤叫屈,那裏理會他半點兒。豈知薛姨媽在寶釵房中聽見如此吵嚷,叫香菱:“你去瞧瞧,且勸勸他。”寶釵道:“使不得,媽媽別叫他去。他去了豈能勸他,那更是火上澆了油了。”薛姨媽道:“既這麼樣,我自己過去。”寶釵道:“依我說媽媽也不用去,由着他們鬧去罷。這也是沒法兒的事了。”薛姨媽道:“這那裏還了得!”說着,自己扶了丫頭,往金桂這邊來。寶釵只得也跟着過去,又囑咐香菱道:“你在這裏罷。”
\end{parag}


\begin{parag}
    母女同至金桂房門口,聽見裏頭正還嚷哭不止。薛姨媽道:“你們是怎麼着,又這樣家翻宅亂起來,這還象個人家兒嗎!矮牆淺屋的,難道都不怕親戚們聽見笑話了麼。”金桂屋裏接聲道:“我倒怕人笑話呢!只是這裏掃帚顛倒豎,也沒有主子,也沒有奴才,也沒有妻,沒有妾,是個混賬世界了。我們夏家門子裏沒見過這樣規矩,實在受不得你們家這樣委屈了!”寶釵道:“大嫂子,媽媽因聽見鬧得慌,纔過來的。就是問的急了些,沒有分清‘奶奶’‘寶蟾’兩字,也沒有什麼。如今且先把事情說開,大家和和氣氣的過日子,也省的媽媽天天爲咱們操心。”那薛姨媽道:“是啊,先把事情說開了,你再問我的不是還不遲呢。”金桂道:“好姑娘,好姑娘,你是個大賢大德的。你日後必定有個好人家,好女婿,決不象我這樣守活寡,舉眼無親,叫人家騎上頭來欺負我的。我是個沒心眼兒的人,只求姑娘我說話別往死裏挑撿,我從小兒到如今,沒有爹孃教導。再者我們屋裏老婆漢子大女人小女人的事,姑娘也管不得!”寶釵聽了這話,又是羞,又是氣,見他母親這樣光景,又是疼不過。因忍了氣說道:“大嫂子,我勸你少說句兒罷。誰挑撿你?又是誰欺負你?不要說是嫂子,就是秋菱,我也從來沒有加他一點聲氣兒的。”金桂聽了這幾句話,更加拍着炕沿大哭起來,說:“我那裏比得秋菱,連他腳底下的泥我還跟不上呢!他是來久了的,知道姑娘的心事,又會獻勤兒,我是新來的,又不會獻勤兒,如何拿我比他。何苦來,天下有幾個都是貴妃的命,行點好兒罷!別修的象我嫁個糊塗行子守活寡,那就是活活兒的現了眼了!”薛姨媽聽到這裏,萬分氣不過,便站起身來道:“不是我護着自己的女孩兒,他句句勸你,你卻句句慪他。你有什麼過不去,不要尋他,勒死我倒也是希松的。”寶釵忙勸道:“媽媽,你老人家不用動氣。咱們既來勸他,自己生氣,倒多了層氣。不如且出去,等嫂子歇歇兒再說。”因吩咐寶蟾道:“你可別再多嘴了。”跟了薛姨媽出得房來。
\end{parag}


\begin{parag}
    走過院子裏,只見賈母身邊的丫頭同着秋菱迎面走來。薛姨媽道:“你從那裏來,老太太身上可安?”那丫頭道:“老太太身上好,叫來請姨太太安,還謝謝前兒的荔枝,還給琴姑娘道喜。”寶釵道:“你多早晚來的?”那丫頭道:“來了好一會子了。”薛姨媽料他知道,紅着臉說道:“這如今我們家裏鬧得也不象個過日子的人家了,叫你們那邊聽見笑話。”丫頭道:“姨太太說那裏的話,誰家沒個碟大碗小磕着碰着的呢。那是姨太太多心罷咧。”說着,跟了回到薛姨媽房中,略坐了一回就去了。寶釵正囑咐香菱些話,只聽薛姨媽忽然叫道:“左肋疼痛的很。”說着,便向炕上躺下。唬得寶釵香菱二人手足無措。要知後事如何,下回分解。
\end{parag}