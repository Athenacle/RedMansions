\chap{四十}{史太君两宴大观园 金鸳鸯三宣牙牌令}


\begin{parag}
    \begin{note}蒙回前总:两宴不觉已深秋,惜春只如画春游。可怜富贵谁能保,只有恩情得到头。\end{note}
\end{parag}


\begin{parag}
    话说宝玉听了,忙进来看时,只见琥珀站在屏风跟前说:“快去吧,立等你说话呢。”宝玉来至上房,只见贾母正和王夫人众姊妹商议给史湘云还席。宝玉因说道:“我有个主意。既没有外客,吃的东西也别定了样数,谁素日爱吃的拣样儿做几样。也不要按桌席,每人跟前摆一张高几,各人爱吃的东西一两样,再一个什锦攒心盒子,自斟壶,岂不别致。”贾母听了,说“很是”,忙命传与厨房:“明日就拣我们爱吃的东西作了,按著人数,再装了盒子来。早饭也摆在园里吃。”商议之间早又掌灯,一夕无话。
\end{parag}


\begin{parag}
    次日清早起来,可喜这日天气清朗。李纨侵晨先起,看著老婆子丫头们扫那些落叶,\begin{note}蒙双夹:八月尽的光景。\end{note}并擦抹桌椅,预备茶酒器皿。只见丰儿带了刘姥姥板儿进来,说“大奶奶倒忙的紧。”李纨笑道:“我说你昨儿去不成,只忙著要去。”刘姥姥笑道:“老太太留下我,叫我也热闹一天去。”丰儿拿了几把大小钥匙,说道:“我们奶奶说了,外头的高几恐不够使,不如开了楼把那收著的拿下来使一天罢。奶奶原该亲自来的,因和太太说话呢,请大奶奶开了,带著人搬罢。”李氏便令素云接了钥匙,又令婆子出去把二门上的小厮叫几个来。李氏站在大观楼下往上看,令人上去开了缀锦阁,一张一张往下抬。小厮老婆子丫头齐动手,抬了二十多张下来。李纨道:“好生著,别慌慌张张鬼赶来似的,仔细碰了牙子。”又回头向刘姥姥笑道:“姥姥,你也上去瞧瞧。”刘姥姥听说,巴不得一声儿,便拉了板儿登梯上去进里面,只见乌压压的堆著些围屏、桌椅、大小花灯之类,虽不大认得,只见五彩炫耀,各有奇妙。念了几声佛,便下来了。然后锁上门,一齐才下来。李纨道:“恐怕老太太高兴,越性把舡上划子、篙桨、遮阳幔子都搬了下来预备著。”众人答应,复又开了,色色的搬了下来。令小厮传驾娘们到舡坞里撑出两只船来。
\end{parag}


\begin{parag}
    正乱著安排,只见贾母已带了一群人进来了。李纨忙迎上去,笑道:“老太太高兴,倒进来了。我只当还没梳头呢,才撷了菊花要送去。”一面说,一面碧月早捧过一个大荷叶式的翡翠盘子来,里面盛著各色的折枝菊花。贾母便拣了一朵大红的簪于鬓上。因回头看见了刘姥姥,忙笑道:“过来带花儿。”一语未完,凤姐便拉过刘姥姥,笑道:“让我打扮你。”说著,将一盘子花横三竖四的插了一头。贾母和众人笑的了不得。刘姥姥笑道:“我这头也不知修了什么福,今儿这样体面起来。”众人笑道:“你还不拔下来摔到他脸上呢,把你打扮的成了个老妖精了。”刘姥姥笑道:“我虽老了,年轻时也风流,爱个花儿粉儿的,今儿老风流才好。”
\end{parag}


\begin{parag}
    说笑之间,已来至沁芳亭子上。丫鬟们抱了一个大锦褥子来,铺在栏杆榻板上。贾母倚柱坐下,命刘姥姥也坐在旁边,因问他:“这园子好不好?”刘姥姥念佛说道:“我们乡下人到了年下,都上城来买画儿贴。时常闲了,大家都说,怎么得也到画儿上去逛逛。想著那个画儿也不过是假的,那里有这个真地方呢。谁知我今儿进这园里一瞧,竟比那画儿还强十倍。怎么得有人也照著这个园子画一张,我带了家去,给他们见见,死了也得好处。”贾母听说,便指著惜春笑道:“你瞧我这个小孙女儿,他就会画。等明儿叫他画一张如何?”刘姥姥听了,喜的忙跑过来,拉著惜春说道:“我的姑娘,你这么大年纪儿,又这么个好模样,还有这个能干,别是神仙托生的罢。”
\end{parag}


\begin{parag}
    贾母少歇一回,自然领著刘姥姥都见识见识。先到了潇湘馆。一进门,只见两边竹夹路,土地下苍苔布满,中间羊肠一条石子漫的路。刘姥姥让出路来贾母众人走,自己却赾土地。琥珀拉著他说道:“姥姥,你上来走,仔细苍苔滑了。”刘姥姥道:“不相干的,我们走熟了的,姑娘们只管走罢。可惜你们的那绣鞋,别沾脏了。”他只顾上头和人说话,不防底下果跴滑了,具一跤跌倒。众人拍手都哈哈的笑起来。贾母笑骂道:“小蹄子们,还不搀起来,只站著笑。”说话时,刘姥姥已爬了起来,自己也笑了,说道:“才说嘴就打了嘴。”贾母问他:“可扭了腰了不曾?叫丫头们捶一捶。”刘姥姥道:“那里说的我这么娇嫩了。那一天不跌两下子,都要捶起来,还了得呢。”紫鹃早打起湘帘,贾母等进来坐下。林黛玉亲自用小茶盘捧了一盖碗茶来奉与贾母。王夫人道:“我们不吃茶,姑娘不用倒了。”林黛玉听说,便命丫头把自己窗下常坐的一张椅子挪到下首,请王夫人坐了。刘姥姥因见窗下案上设著笔砚,又见书架上磊著满满的书,刘姥姥道:“这必定是那位哥儿的书房了。”贾母笑指黛玉道:“这是我这外孙女儿的屋子。”刘姥姥留神打量了黛玉一番,方笑道:“这那象个小姐的绣房,竟比那上等的书房还好。”贾母因问:“宝玉怎么不见?”众丫头们答说:“在池子里舡上呢。”贾母道:“谁又预备下舡了?”李纨忙回说:“才开楼拿几,我恐怕老太太高兴,就预备下了。”贾母听了方欲说话时,有人回说: “姨太太来了。” 贾母等刚站起来,只见薛姨妈早进来了,一面归坐,笑道:“今儿老太太高兴,这早晚就来了。”贾母笑道:“我才说来迟了的要罚他,不想姨太太就来迟了。”
\end{parag}


\begin{parag}
    说笑一会,贾母因见窗上纱的颜色旧了,便和王夫人说道:“这个纱新糊上好看,过了后来就不翠了。这个院子里头又没有个桃杏树,这竹子已是绿的,再拿这绿纱糊上反不配。我记得咱们先有四五样颜色糊窗的纱呢,明儿给他把这窗上的换了。”凤姐儿忙道:“昨儿我开库房,看见大板箱里还有好些匹银红蝉翼纱,也有各样折枝花样的,也有流云万福花样的,也有百蝶穿花花样的,颜色又鲜,纱又轻软,我竟没见过这样的。拿了两匹出来,作两床绵纱被,想来一定是好的。”贾母听了笑道:“呸,人人都说你没有不经过不见过,连这个纱还不认得呢,明儿还说嘴。”薛姨妈等都笑说:“凭他怎么经过见过,如何敢比老太太呢。老太太何不教导了他,我们也听听。”凤姐儿也笑说:“好祖宗,教给我罢。”贾母笑向薛姨妈众人道:“那个纱,比你们的年纪还大呢。怪不得他认作蝉翼纱,原也有些象,不知道的,都认作蝉翼纱。正经名字叫作‘软烟罗’。”凤姐儿道:“这个名儿也好听。只是我这么大了,纱罗也见过几百样,从没听见过这个名色。”贾母笑道: “你能够活了多大,见过几样没处放的东西,就说嘴来了。那个软烟罗只有四样颜色:一样雨过天晴,一样秋香色,一样松绿的,一样就是银红的。若是做了帐子,糊了窗屉,远远的看著,就似烟雾一样,所以叫作‘软烟罗’,那银红的又叫作‘霞影纱’。如今上用的府纱也没有这样软厚轻密的了。”薛姨妈笑道:“别说凤丫头没见,连我也没听见过。”凤姐儿一面说,早命人取了一匹来了。贾母说:“可不是这个!先时原不过是糊窗屉,后来我们拿这个作被作帐子,试试也竟好。明儿就找出几匹来,拿银红的替他糊窗子。”凤姐答应著。众人都看了,称赞不已。刘姥姥也觑著眼看个不了,念佛说道:“我们想他作衣裳也不能,拿著糊窗子,岂不可惜?”贾母道:“倒是做衣裳不好看。”凤姐忙把自己身上穿的一件大红绵纱袄子襟儿拉了出来,向贾母薛姨妈道:“看我的这袄儿。”贾母薛姨妈都说:“这也是上好的了,这是如今的上用内造的,竟比不上这个。”凤姐儿道:“这个薄片子,还说是上用内造呢,竟连官用的也比不上了。” 贾母道:“再找一找,只怕还有青的。若有时都拿出来,送这刘亲家两匹,做一个帐子我挂,下剩的添上里子,做些夹背心子给丫头们穿,白收著霉坏了。”凤姐忙答应了,仍令人送去。贾母起身笑道:“这屋里窄,再往别处逛去。”刘姥姥念佛道:“人人都说大家子住大房。昨儿见了老太太正房,配上大箱大柜大桌子大床,果然威武。那柜子比我们那一间房子还大还高。怪道后院子里有个梯子。我想并不上房晒东西,预备个梯子作什么?后来我想起来,定是为开顶柜收放东西,非离了那梯子,怎么得上去呢。如今又见了这小屋子,更比大的越发齐整了。满屋里的东西都只好看,都不知叫什么,我越看越舍不得离了这里。”凤姐道:“还有好的呢,我都带你去瞧瞧。”说著一径离了潇湘馆。
\end{parag}


\begin{parag}
    远远望见池中一群人在那里撑舡。贾母道:“他们既预备下船,咱们就坐。”一面说著,便向紫菱洲蓼溆一带走来。未至池前,只见几个婆子手里都捧著一色捏丝戗金五彩大盒子走来。凤姐忙问王夫人早饭在那里摆。王夫人道:“问老太太在那里,就在那里罢了。”贾母听说,便回头说:“你三妹妹那里就好。你就带了人摆去,我们从这里坐了舡去。”凤姐听说,便回身同了探春、李纨、鸳鸯、琥珀带著端饭的人等,抄著近路到了秋爽斋,就在晓翠堂上调开桌案。鸳鸯笑道:“天天咱们说外头老爷们吃酒吃饭都有一个篾片相公,拿他取笑儿。咱们今儿也得了一个女篾片了。”李纨是个厚道人,听了不解。凤姐儿却知是说的是刘姥姥了,也笑说道:“咱们今儿就拿他取个笑儿。”二人便如此这般的商议。李纨笑劝道:“你们一点好事也不做,又不是个小孩儿,还这么淘气,仔细老太太说。”鸳鸯笑道: “很不与你相干,有我呢。”
\end{parag}


\begin{parag}
    正说著,只见贾母等来了,各自随便坐下。先著丫鬟端过两盘茶来,大家吃毕。凤姐手里拿著西洋布手巾,裹著一把乌木三镶银箸,敁敠人位,按席摆下。贾母因说:“把那一张小楠木桌子抬过来,让刘亲家近我这边坐著。”众人听说,忙抬了过来。凤姐一面递眼色与鸳鸯,鸳鸯便拉了刘姥姥出去,那牡嘱咐了刘姥姥一席话,又说:“这是我们家的规矩,若错了我们就笑话呢。”调停已毕,然后归坐。薛姨妈是吃过饭来的,不吃,只坐在一边吃茶。\begin{note}庚双夹:妙!若只管写薛姨妈来则吃饭,则成何义理?\end{note}贾母带著宝玉、湘云、黛玉、宝钗一桌,王夫人带著迎春姊妹三个人一桌,刘姥姥傍著贾母一桌。贾母素日吃饭,皆有小丫鬟在旁边,拿著漱盂麈尾巾帕之物。如今鸳鸯是不当这差的了,今日鸳鸯偏接过麈尾来拂著。丫鬟们知道他要撮弄刘姥姥,便躲开让他。鸳鸯一面侍立,一面悄向刘姥姥说道:“别忘了。”刘姥姥道:“姑娘放心。”那刘姥姥入了坐,拿起箸来,沉甸甸的不伏手。原是凤姐和鸳鸯商议定了,单拿一双老年四楞象牙镶金的筷子与刘姥姥。刘姥姥见了,说道:“这叉爬子比俺那里铁掀还沉,那里犟的过他。”说的众人都笑起来。
\end{parag}


\begin{parag}
    只见一个媳妇端了一个盒子站在当地,一个丫鬟上来揭去盒盖,里面盛著两碗菜。李纨端了一碗放在贾母桌上。凤姐儿偏拣了一碗鸽子蛋放在刘姥姥桌上。贾母这边说声“请”,刘姥姥便站起身来,高声说道:“老刘,老刘,食量大似牛,吃一个老母猪不抬头。”自己却鼓著腮不语。众人先是发怔,后来一听,上上下下都哈哈的大笑起来。史湘云撑不住,一口饭都喷了出来;林黛玉笑岔了气,伏著桌子嗳哟;宝玉早滚到贾母怀里,贾母笑的搂著宝玉叫“心肝”;王夫人笑的用手指著凤姐儿,只说不出话来;薛姨妈也撑不住,口里茶喷了探春一裙子;探春手里的饭碗都合在迎春身上;惜春离了坐位,拉著他奶母叫揉一揉肠子。地下的无一个不弯腰屈背,也有躲出去蹲著笑去的,也有忍著笑上来替他姊妹换衣裳的,独有凤姐鸳鸯二人撑著,还只管让刘姥姥。刘姥姥拿起箸来,只觉不听使,又说道:“这里的鸡儿也俊,下的这蛋也小巧,怪俊的。我且肏攮一个。”众人方住了笑,听见这话又笑起来。贾母笑的眼泪出来,琥珀在后捶著。贾母笑道:“这定是凤丫头促狭鬼儿闹的,快别信他的话了。”那刘姥姥正夸鸡蛋小巧,要肏攮一个,凤姐儿笑道:“一两银子一个呢,你快尝尝罢,那冷了就不好吃了。”刘姥姥便伸箸子要夹,那里夹的起来,满碗里闹了一阵好的,好容易撮起一个来,才伸著脖子要吃,偏又滑下来滚在地下,忙放下箸子要亲自去捡,早有地下的人捡了出去了。刘姥姥叹道: “一两银子,也没听见响声儿就没了。”众人已没心吃饭,都看著他笑。贾母又说:“这会子又把那个筷子拿了出来,又不请客摆大筵席。都是凤丫头支使的,还不换了呢。”地下的人原不曾预备这牙箸,本是凤姐和鸳鸯拿了来的,听如此说,忙收了过去,也照样换上一双乌木镶银的。刘姥姥道:“去了金的,又是银的,到底不及俺们那个伏手。”凤姐儿道:“菜里若有毒,这银子下去了就试的出来。”刘姥姥道:“这个菜里若有毒,俺们那菜都成了砒霜了。那怕毒死了也要吃尽了。” 贾母见他如此有趣,吃的又香甜,把自己的也都端过来与他吃。又命一个老嬷嬷来,将各样的菜给板儿夹在碗上。
\end{parag}


\begin{parag}
    一时吃毕,贾母等都往探春卧室中去说闲话。这里收拾过残桌,又放了一桌。刘姥姥看著李纨与凤姐儿对坐著吃饭,叹道:“别的罢了,我只爱你们家这行事。怪道说‘礼出大家’。”凤姐儿忙笑道:“你可别多心,才刚不过大家取笑儿。”一言未了,鸳鸯也进来笑道:“姥姥别恼,我给你老人家赔个不是。”刘姥姥笑道:“姑娘说那里话,咱们哄著老太太开个心儿,可有什么恼的!你先嘱咐我,我就明白了,不过大家取个笑儿。我要心里恼,也就不说了。”鸳鸯便骂人“为什么不倒茶给姥姥吃?”刘姥姥忙道:“刚才那个嫂子倒了茶来,我吃过了。姑娘也该用饭了。”凤姐儿便拉鸳鸯:“你坐下和我们吃了罢,省的回来又闹。”鸳鸯便坐下了。婆子们添上碗箸来,三人吃毕。刘姥姥笑道:“我看你们这些人都只吃这一点儿就完了,亏你们也不饿。怪只道风儿都吹的倒。” 鸳鸯便问:“今儿剩的菜不少,都那去了?”婆子们道:“都还没散呢,在这里等著一齐散与他们吃。”鸳鸯道:“他们吃不了这些,挑两碗给二奶奶屋里平丫头送去。”凤姐儿道:“他早吃了饭了,不用给他。”鸳鸯道:“他不吃了,喂你们的猫。”婆子听了,忙拣了两样拿盒子送去。鸳鸯道:“素云那去了?”李纨道: “他们都在这里一处吃,又找他作什么。”鸳鸯道:“这就罢了。”凤姐儿道:“袭人不在这里,你倒是叫人送两样给他去。”鸳鸯听说,便命人也送两样去后,鸳鸯又问婆子们:“回来吃酒的攒盒可装上了?”婆子道:“想必还得一会子。”鸳鸯道:“催著些儿。”婆子应喏了。
\end{parag}


\begin{parag}
    凤姐儿等来至探春房中,只见他娘儿们正说笑。探春素喜阔朗,这三间屋子并不曾隔断。当地放著一张花梨大理石大案,案上磊著各种名人法帖,并数十方宝砚,各色笔筒,笔海内插的笔如树林一般。那一边设著斗大的一个汝窑花囊,插著满满的一囊水晶球儿的白菊。西墙上当中挂著一大幅米襄阳《烟雨图》,左右挂著一副对联,乃是颜鲁公墨迹,其词云:
\end{parag}


\begin{poem}
    \begin{pl}烟霞闲骨格,泉石野生涯。\end{pl}
\end{poem}


\begin{parag}
    案上设著大鼎。左边紫檀架上放著一个大观窑的大盘,盘内盛著数十个娇黄玲珑大佛手。右边洋漆架上悬著一个白玉比目磬,旁边挂著小锤。那板儿略熟了些,便要摘那锤子要击,丫鬟们忙拦住他。他又要佛手吃,探春拣了一个与他说:“顽罢,吃不得的。”东边便设著卧榻,拔步床上悬著葱绿双绣卉草虫的纱帐。板儿又跑过来看,说:“这是蝈蝈,这是蚂蚱。”刘姥姥忙打了他一巴掌,骂道:“下作黄子,没干没净的乱闹。倒叫你进来瞧瞧,就上脸了。”打的板儿哭起来,众人忙劝解方罢。贾母因隔著纱窗往后院内看了一回,说道:“后廊檐下的梧桐也好了,就只细些。”正说话,忽一阵风过,隐隐听得鼓乐之声。贾母问“是谁家娶亲呢?这里临街倒近。”王夫人等笑回道:“街上的那里听的见,这是咱们的那十几个女孩子们演习吹打呢。”贾母便笑道:“既是他们演,何不叫他们进来演习。他们也逛一逛,咱们可又乐了。”凤姐听说,忙命人出去叫来,又一面吩咐摆下条桌,铺上红毡子。贾母道:“就铺排在藕香榭的水亭子上,藉著水音更好听。回来咱们就在缀锦阁底下吃酒,又宽阔,又听的近。”众人都说那里好。贾母向薛姨妈笑道:“咱们走罢。他们姊妹们都不大喜欢人来坐著,怕脏了屋子。咱们别没眼色,正经坐一回子船喝酒去。”说著大家起身便走。探春笑道:“这是那里的话,求著老太太姨太太来坐坐还不能呢。”贾母笑道:“我的这三丫头却好,只有两个玉儿可恶。回来吃醉了,咱们偏往他们屋里闹去。”
\end{parag}


\begin{parag}
    说著,众人都笑了,一齐出来。走不多远,已到了 叶渚。 姑苏选来的几个驾娘早把两只棠舫撑来,众人扶了贾、王夫人、薛姨妈、刘姥姥、鸳鸯、玉钏儿上了这一只,落后李纨也跟上去。凤姐儿也上去,立在舡头上,也要撑舡。贾母在舱内道:“这不是顽的,虽不是河里,也有好深的。你快不给我进来。”凤姐儿笑道:“怕什么!老祖宗只管放心。”说著便一篙点开。到了池当中,舡小人多,凤姐只觉乱晃,忙把篙子递与驾娘,方蹲下了。然后迎春姊妹等并宝玉上了那只,随后跟来。其余老嬷嬷散众丫鬟俱沿河随行。宝玉道:“这些破荷叶可恨,怎么还不叫人来拔去。”宝钗笑道:“今年这几日,何曾饶了这园子闲了,天天逛,那里还有叫人来收拾的工夫。”林黛玉道:“我最不喜欢李义山的诗,只喜他这一句‘留得残荷听雨声’。偏你们又不留著残荷了。”宝玉道:“果然好句,以后咱们就别叫人拔去了。”说著已到了花溆的萝港之下,觉得阴森透骨,两滩上衰草残菱,更助秋情。更助秋情。
\end{parag}


\begin{parag}
    贾母因见岸上的清厦旷朗,便问“这是你薛姑娘的屋子不是?”众人道:“是。”贾母忙命拢岸,顺著云步石梯上去,一同进了蘅芜苑,只觉异香扑鼻。那些奇草仙藤愈冷愈苍翠,都结了实,似珊瑚豆子一般,累垂可爱。及进了房屋,雪洞一般,一色玩器全无,案上只有一个土定瓶中供著数枝菊花,并两部书,茶奁茶杯而已。床上只吊著青纱帐幔,衾褥也十分朴素。贾母叹道:“这孩子太老实了。你没有陈设,何妨和你姨娘要些。我也不理论,也没想到,你们的东西自然在家里没带了来。”说著,命鸳鸯去取些古董来,又嗔著凤姐儿:“不送些玩器来与你妹妹,这样小器。”王夫人凤姐儿等都笑回说:“他自己不要的。我们原送了来,他都退回去了。”薛姨妈也笑说:“他在家里也不大弄这些东西的。”贾母摇头道:“使不得。虽然他省事,倘或来一个亲戚,看著不象;二则年轻的姑娘们,房里这样素净,也忌讳。我们这老婆子,越发该住马圈去了。你们听那些书上戏上说的小姐们的绣房,精致的还了得呢。他们姊妹们虽不敢比那些小姐们,也不要很离了格儿。有现成的东西,为什么不摆?若很爱素净,少几样倒使得。我最会收拾屋子的,如今老了,没有这些闲心了。他们姊妹们也还学著收拾的好,只怕俗气,有好东西也摆坏了。我看他们还不俗。如今让我替你收拾,包管又大方又素净。我的梯己两件,收到如今,没给宝玉看见过,若经了他的眼,也没了。”说著叫过鸳鸯来,亲吩咐道:“你把那石头盆景儿和那架纱桌屏,还有个墨烟冻石鼎,这三样摆在这案上就够了。再把那水墨字画白绫帐子拿来,把这帐子也换了。”鸳鸯答应著,笑道: “这些东西都搁在东楼上的不知那个箱子里,还得慢慢找去,明儿再拿去也罢了。”贾母道:“明日后日都使得,只别忘了。”说著,坐了一回方出来,一径来至锦阁下。文官等上来请过安,因问“演习何曲”。贾母道:“只拣你们生的演习几套罢。”文官等下来,往藕香榭去不提。
\end{parag}


\begin{parag}
    这里凤姐儿已带著人摆设整齐,上面左右两张榻,榻上都铺著锦裀蓉簟,每一榻前有两张雕漆几,也有海棠式的,也有梅花式的,也有荷叶式的,也有葵花式的,也有方的,也有圆的,其式不一。一个上面放著炉瓶,一分攒盒,一个上面空设著,预备放人所喜食物。上面二榻四几,是贾母薛姨妈;下面一椅两几,是王夫人的,余者都是一椅一几。东边是刘姥姥,刘姥姥之下便是王夫人。西边便是史湘云,第二便是宝钗,第三便是黛玉,第四迎春、探春、惜春挨次下去,宝玉在末。李纨凤姐二人之几设于三层槛内,二层纱厨之外。攒盒式样,亦随几之式样。每人一把乌银洋錾自斟壶,一个十锦珐琅杯。
\end{parag}


\begin{parag}
    大家坐定,贾母先笑道:“咱们先吃两杯,今日也行一令才有意思。”薛姨妈等笑道:“老太太自然有好酒令,我们如何会呢,安心要我们醉了。我们都多吃两杯就有了。”贾母笑道:“姨太太今儿也过谦起来,想是厌我老了。”薛姨妈笑道:“不是谦,只怕行不上来倒是笑话了。”王夫人忙笑道:“便说不上来,就便多吃一杯酒,醉了睡觉去,还有谁笑话咱们不成。”薛姨妈点头笑道:“依令。老太太到底吃一杯令酒才是。”贾母笑道:“这个自然。”说著便吃了一杯。
\end{parag}


\begin{parag}
    凤姐儿忙走至当地,笑道:“既行令,还叫鸳鸯姐姐来行更好。”众人都知贾母所行之令必得鸳鸯提著,故听了这话,都说:“很是。”凤姐儿便拉了鸳鸯过来。王夫人笑道:“既在令内,没有站著的理。”回头命小丫头子:“端一张椅子,放在你二位奶奶的席上。”鸳鸯也半推半就,谢了坐,便坐下,也吃了一钟酒,笑道:“酒令大如军令,不论尊卑,惟我是主。违了我的话,是要受罚的。”王夫人等都笑道:“一定如此,快些说来。”鸳鸯未开口,刘姥姥便下了席,摆手道: “别这样捉弄人家,我家去了。”众人都笑道:“这却使不得。”鸳鸯喝令小丫头子们:“拉上席去!”小丫头子们也笑著,果然拉入席中。刘姥姥只叫:“饶了我罢!”鸳鸯道:“再多言的罚一壶。”刘姥姥方住了声。鸳鸯道:“如今我说骨牌副儿,从老太太起,顺领说下去,至刘姥姥止。比如我说一副儿,将这三张牌拆开,先说头一张,次说第二张,再说第三张,说完了,合成这一副儿的名字。无论诗词歌赋,成语俗话,比上一句,都要叶韵。错了的罚一杯。”众人笑道:“这个令好,就说出来。”鸳鸯道:“有了一副了。左边是张‘天’。”贾母道:“头上有青天。”众人道:“好。”鸳鸯道:“当中是个‘五与六’。”贾母道:“六桥梅花香彻骨。”鸳鸯道:“剩得一张‘六与幺 ’。”贾母道:“一轮红日出云霄。”鸳鸯道:“凑成便是个‘蓬头鬼’。”贾母道:“这鬼抱住钟馗腿。”说完,大家笑说:“极妙。”贾母饮了一杯。鸳鸯又道:“有了一副。左边是个‘大长五’。”薛姨妈道:“梅花朵朵风前舞。”鸳鸯道:“右边还是个‘大五长’。”薛姨妈道:“十月梅花岭上香。”鸳鸯道:“当中‘二五’是杂七。”薛姨妈道:“织女牛郎会七夕。”鸳鸯道:“凑成‘二郎游五岳’。”薛姨妈道:“世人不及神仙乐。”说完,大家称赏,饮了酒。鸳鸯又道:“有了一副。左边‘长幺’两点明。”湘云道:“双悬日月照乾坤。”鸳鸯道:“右边‘长幺’两点明。”湘云道:“闲花落地听无声。”鸳鸯道:“中间还得 ‘幺四’来。”湘云道:“日边红杏倚云栽。”鸳鸯道:“凑成‘樱桃九熟’。”湘云道:“御园却被鸟衔出。”说完饮了一杯。鸳鸯道:“有了一副。左边是‘长三’。”宝钗道:“双双燕子语梁间。” 鸳鸯道:“右边是‘三长’。”宝钗道:“水荇牵风翠带长。”鸳鸯道:“当中‘三六’九点在。”宝钗道:“三山半落青天外。”鸳鸯道:“凑成‘铁锁练孤舟 ’。”宝钗道:“处处风波处处愁。”说完饮毕。鸳鸯又道:“左边一个‘天’。”黛玉道:“良辰美景奈何天。”宝钗听了,回头看著他。黛玉只顾怕罚,也不理论。鸳鸯道:“中间‘锦屏’颜色俏。”黛玉道:“纱窗也没有红娘报。”鸳鸯道:“剩了‘二六’八点齐。”黛玉道:“双瞻玉座引朝仪。”鸳鸯道:“凑成‘篮子’好采花。”黛玉道:“仙杖香挑芍药花。”说完,饮了一口。鸳鸯道:“左边‘四五’成花九。”迎春道:“桃花带雨浓。”众人道:“该罚!错了韵,而且又不象。”迎春笑著饮了一口。原是凤姐儿和鸳鸯都要听刘姥姥的笑话,故意都令说错,都罚了。至王夫人,鸳鸯代说了个,下便该刘姥姥。刘姥姥道:“我们庄家人闲了,也常会几个人弄这个,但不如说的这么好听。少不得我也试一试。”众人都笑道:“容易说的。你只管说,不相干。”鸳鸯笑道:“左边‘四四’是个人。” 刘姥姥听了,想了半日,说道:“是个庄家人罢。”众人哄堂笑了。贾母笑道:“说的好,就是这样说。”刘姥姥也笑道:“我们庄家人,不过是现成的本色,众位别笑。”鸳鸯道:“中间‘三四’绿配红。”刘姥姥道:“大火烧了毛毛虫。”众人笑道:“这是有的,还说你的本色。”鸳鸯道:“右边‘幺四’真好看。”刘姥姥道:“一个萝卜一头蒜。”众人又笑了。鸳鸯笑道:“凑成便是一枝花。”刘姥姥两只手比著,说道:“花儿落了结个大倭瓜。”众人大笑起来。只听外面乱嚷 ——
\end{parag}


\begin{parag}
    \begin{note}蒙回末总:寓贫贱辈低首豪门,凌辱不计,诚可悲乎!此故作者以警贫穷。而富室贵家亦当于其间著意。\end{note}
\end{parag}

