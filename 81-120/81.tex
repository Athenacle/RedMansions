\chap{八十一}{佔旺相四美釣游魚 奉嚴詞兩番入家塾}



\begin{parag}
    且說迎春歸去之後,邢夫人象沒有這事,倒是王夫人撫養了一場,卻甚實傷感,在房中自己嘆息了一回。只見寶玉走來請安,看見王夫人臉上似有淚痕,也不敢坐,只在旁邊站着。王夫人叫他坐下,寶玉才捱上炕來,就在王夫人身旁坐了。王夫人見他呆呆的瞅着,似有欲言不言的光景,便道:“你又爲什麼這樣呆呆的?”寶玉道:“並不爲什麼,只是昨兒聽見二姐姐這種光景,我實在替他受不得。雖不敢告訴老太太,卻這兩夜只是睡不着。我想咱們這樣人家的姑娘,那裏受得這樣的委屈。況且二姐姐是個最懦弱的人,向來不會和人拌嘴,偏偏兒的遇見這樣沒人心的東西,竟一點兒不知道女人的苦處。”說着,幾乎滴下淚來。王夫人道:“這也是沒法兒的事。俗語說的,‘嫁出去的女孩兒潑出去的水’,叫我能怎麼樣呢。”寶玉道:“我昨兒夜裏倒想了一個主意:咱們索性回明瞭老太太,把二姐姐接回來,還叫他紫菱洲住着,仍舊我們姐妹弟兄們一塊兒喫,一塊兒頑,省得受孫家那混賬行子的氣。等他來接,咱們硬不叫他去。由他接一百回,咱們留一百回,只說是老太太的主意。這個豈不好呢!”王夫人聽了,又好笑,又好惱,說道:“你又發了呆氣了,混說的是什麼!大凡做了女孩兒,終久是要出門子的,嫁到人家去,孃家那裏顧得,也只好看他自己的命運,碰得好就好,碰得不好也就沒法兒。你難道沒聽見人說‘嫁雞隨雞,嫁狗隨狗’,那裏個個都象你大姐姐做娘娘呢。況且你二姐姐是新媳婦,孫姑爺也還是年輕的人,各人有各人的脾氣,新來乍到,自然要有些扭別的。過幾年大家摸着脾氣兒,生兒長女以後,那就好了。你斷斷不許在老太太跟前說起半個字,我知道了是不依你的。快去幹你的去罷,不要在這裏混說。”說得寶玉也不敢作聲,坐了一回,無精打彩的出來了。憋着一肚子悶氣,無處可泄,走到園中,一徑往瀟湘館來。
\end{parag}


\begin{parag}
    剛進了門,便放聲大哭起來。黛玉正在梳洗才畢,見寶玉這個光景,倒嚇了一跳,問:“是怎麼了?和誰慪了氣了?”連問幾聲。寶玉低着頭,伏在桌子上,嗚嗚咽咽,哭的說不出話來。黛玉便在椅子上怔怔的瞅着他,一會子問道:“到底是別人和你慪了氣了,還是我得罪了你呢?”寶玉搖手道:“都不是,都不是。”黛玉道:“那麼着爲什麼這麼傷起心來?”寶玉道:“我只想着咱們大家越早些死的越好,活着真真沒有趣兒!”黛玉聽了這話,更覺驚訝,道:“這是什麼話,你真正發了瘋了不成!”寶玉道:“也並不是我發瘋,我告訴你,你也不能不傷心。前兒二姐姐回來的樣子和那些話,你也都聽見看見了。我想人到了大的時候,爲什麼要嫁?嫁出去受人家這般苦楚!還記得咱們初結‘海棠社’的時候,大家吟詩做東道,那時候何等熱鬧。如今寶姐姐家去了,連香菱也不能過來,二姐姐又出了門子了,幾個知心知意的人都不在一處,弄得這樣光景。我原打算去告訴老太太接二姐姐回來,誰知太太不依,倒說我呆,混說,我又不敢言語。這不多幾時,你瞧瞧,園中光景,已經大變了。若再過幾年,又不知怎麼樣了。故此越想不由人不心裏難受起來。”黛玉聽了這番言語,把頭漸漸的低了下去,身子漸漸的退至炕上,一言不發,嘆了口氣,便向裏躺下去了。
\end{parag}


\begin{parag}
    紫鵑剛拿進茶來,見他兩個這樣,正在納悶。只見襲人來了,進來看見寶玉,便道:“二爺在這裏呢麼,老太太那裏叫呢。我估量着二爺就是在這裏。”黛玉聽見是襲人,便欠身起來讓坐。黛玉的兩個眼圈兒已經哭的通紅了。寶玉看見道:“妹妹,我剛纔說的不過是些呆話,你也不用傷心。你要想我的話時,身子更要保重纔好。你歇歇兒罷,老太太那邊叫我,我看看去就來。”說着,往外走了。襲人悄問黛玉道:“你兩個人又爲什麼?”黛玉道:“他爲他二姐姐傷心,我是剛纔眼睛發癢揉的,並不爲什麼。”襲人也不言語,忙跟了寶玉出來,各自散了。寶玉來到賈母那邊,賈母卻已經歇晌,只得回到怡紅院。到了午後,寶玉睡了中覺起來,甚覺無聊,隨手拿了一本書看。襲人見他看書,忙去沏茶伺候。誰知寶玉拿的那本書卻是《古樂府》,隨手翻來,正看見曹孟德“對酒當歌,人生幾何”一首,不覺刺心。因放下這一本,又拿一本看時,卻是晉文,翻了幾頁,忽然把書掩上,託着腮,只管癡癡的坐着。襲人倒了茶來,見他這般光景便道:“你爲什麼又不看了?”寶玉也不答言,接過茶來喝了一口,便放下了。襲人一時摸不着頭腦,也只管站在旁邊呆呆的看着他。忽見寶玉站起來,嘴裏咕咕噥噥的說道:“好一個‘放浪形骸之外’!”襲人聽了,又好笑,又不敢問他,只得勸道:“你若不愛看這些書,不如還到園裏逛逛,也省得悶出毛病來。”那寶玉只管口中答應,只管出着神往外走了。
\end{parag}


\begin{parag}
    一時走到沁芳亭,但見蕭疏景象,人去房空。又來至蘅蕪院,更是香草依然,門窗掩閉。轉過藕香榭來,遠遠的只見幾個人在蓼漵一帶欄杆上靠着,有幾個小丫頭蹲在地下找東西。寶玉輕輕的走在假山背後聽着。只聽一個說道:“看他洑上來不洑上來。”好似李紋的語音。一個笑道:“好,下去了。我知道他不上來的。”這個卻是探春的聲音。一個又道:“是了,姐姐你別動,只管等着。他橫豎上來。”一個又說:“上來了。”這兩個是李綺邢岫煙的聲兒。寶玉忍不住,拾了一塊小磚頭兒,往那水裏一撂,咕咚一聲,四個人都嚇了一跳,驚訝道:“這是誰這麼促狹?唬了我們一跳。”寶玉笑着從山子後直跳出來,笑道:“你們好樂啊,怎麼不叫我一聲兒?”探春道:“我就知道再不是別人,必是二哥哥這樣淘氣。沒什麼說的,你好好兒的賠我們的魚罷。剛纔一個魚上來,剛剛兒的要釣着,叫你唬跑了。”寶玉笑道:“你們在這裏頑竟不找我,我還要罰你們呢。”大家笑了一回。寶玉道:“咱們大家今兒釣魚佔佔誰的運氣好。看誰釣得着就是他今年的運氣好,釣不着就是他今年運氣不好。咱們誰先釣?”探春便讓李紋,李紋不肯。探春笑道:“這樣就是我先釣。”回頭向寶玉說道:“二哥哥,你再趕走了我的魚,我可不依了。”寶玉道:“頭裏原是我要唬你們頑,這會子你只管釣罷。”探春把絲繩拋下,沒十來句話的工夫,就有一個楊葉竄兒吞着鉤子把漂兒墜下去,探春把竿一挑,往地下一撩,卻活迸的。侍書在滿地上亂抓,兩手捧着,擱在小磁壇內清水養着。探春把釣竿遞與李紋。李紋也把釣竿垂下,但覺絲兒一動,忙挑起來,卻是個空鉤子。又垂下去,半晌鉤絲一動,又挑起來,還是空鉤子。李紋把那鉤子拿上來一瞧,原來往裏鉤了。李紋笑道:“怪不得釣不着。”忙叫素雲把鉤子敲好了,換上新蟲子,上邊貼好了葦片兒。垂下去一會兒,見葦片直沉下去,急忙提起來,倒是一個二寸長的鯽瓜兒。李紋笑着道:“寶哥哥釣罷。”寶玉道:“索性三妹妹和邢妹妹釣了我再釣。”岫煙卻不答言。只見李綺道:“寶哥哥先釣罷。”說着水面上起了一個泡兒。探春道:“不必盡着讓了。你看那魚都在三妹妹那邊呢,還是三妹妹快着釣罷。”李綺笑着接了釣竿兒,果然沉下去就釣了一個。然後岫煙也釣着了一個,隨將竿子仍舊遞給探春,探春才遞與寶玉。寶玉道:“我是要做姜太公的。”便走下石磯,坐在池邊釣起來,豈知那水裏的魚看見人影兒,都躲到別處去了。寶玉掄着釣竿等了半天,那釣絲兒動也不動。剛有一個魚兒在水邊吐沫,寶玉把竿子一幌,又唬走了。急的寶玉道:“我最是個性兒急的人,他偏性兒慢,這可怎麼樣呢。好魚兒,快來罷!你也成全成全我呢。”說得四人都笑了。一言未了,只見釣絲微微一動。寶玉喜得滿懷,用力往上一兜,把釣竿往石上一碰,折作兩段,絲也振斷了,鉤子也不知往那裏去了。衆人越發笑起來。探春道:“再沒見象你這樣鹵人。”正說着,只見麝月慌慌張張的跑來說:“二爺,老太太醒了,叫你快去呢。”五個人都唬了一跳。探春便問麝月道:“老太太叫二爺什麼事?”麝月道:“我也不知道。就只聽見說是什麼鬧破了,叫寶玉來問,還要叫璉二奶奶一塊兒查問呢。”嚇得寶玉發了一回呆,說道:“不知又是那個丫頭遭了瘟了。”探春道:“不知什麼事,二哥哥你快去,有什麼信兒,先叫麝月來告訴我們一聲兒。”說着,便同李紋李綺岫煙走了。
\end{parag}


\begin{parag}
    寶玉走到賈母房中,只見王夫人陪着賈母摸牌。寶玉看見無事,才把心放下了一半。賈母見他進來,便問道:“你前年那一次大病的時候,後來虧了一個瘋和尚和個瘸道士治好了的。那會子病裏,你覺得是怎麼樣?”寶玉想了一回,道:“我記得得病的時候兒,好好的站着,倒象背地裏有人把我攔頭一棍,疼的眼睛前頭漆黑,看見滿屋子裏都是些青面獠牙,拿刀舉棒的惡鬼。躺在炕上,覺得腦袋上加了幾個腦箍似的。以後便疼的任什麼不知道了。到好的時候,又記得堂屋裏一片金光直照到我房裏來,那些鬼都跑着躲避,便不見了。我的頭也不疼了,心上也就清楚了。”賈母告訴王夫人道:“這個樣兒也就差不多了。”
\end{parag}


\begin{parag}
    說着鳳姐也進來了,見了賈母,又回身見過了王夫人,說道:“老祖宗要問我什麼?”賈母道:“你前年害了邪病,你還記得怎麼樣?”鳳姐兒笑道:“我也不很記得了。但覺自己身子不由自主,倒象有些鬼怪拉拉扯扯要我殺人才好,有什麼,拿什麼,見什麼,殺什麼。自己原覺很乏,只是不能住手。”賈母道:“好的時候還記得麼?”鳳姐道:“好的時候好象空中有人說了幾句話似的,卻不記得說什麼來着。”賈母道:“這麼看起來竟是他了。他姐兒兩個病中的光景和才說的一樣。這老東西竟這樣壞心,寶玉枉認了他做乾媽。倒是這個和尚道人,阿彌陀佛,纔是救寶玉性命的,只是沒有報答他。”鳳姐道:“怎麼老太太想起我們的病來呢?”賈母道:“你問你太太去,我懶待說。”王夫人道:“纔剛老爺進來說起寶玉的乾媽竟是個混賬東西,邪魔外道的。如今鬧破了,被錦衣府拿住送入刑部監,要問死罪的了,前幾天被人告發的。那個人叫做什麼潘三保,有一所房子賣與斜對過當鋪裏。這房子加了幾倍價錢,潘三保還要加,當鋪裏那裏還肯。潘三保便買囑了這老東西,因他常到當鋪裏去,那當鋪里人的內眷都與他好的。他就使了個法兒,叫人家的內人便得了邪病,家翻宅亂起來。他又去說這個病他能治,就用些神馬紙錢燒獻了,果然見效。他又向人家內眷們要了十幾兩銀子。豈知老佛爺有眼,應該敗露了。這一天急要回去,掉了一個絹包兒。當鋪里人撿起來一看,裏頭有許多紙人,還有四丸子很香的香。正詫異着呢,那老東西倒回來找這絹包兒。這裏的人就把他拿住,身邊一搜,搜出一個匣子,裏面有象牙刻的一男一女,不穿衣服,光着身子的兩個魔王,還有七根硃紅繡花針。立時送到錦衣府去,問出許多官員家大戶太太姑娘們的隱情事來。所以知會了營裏,把他家中一抄,抄出好些泥塑的煞神,幾匣子鬧香。炕背後空屋子裏掛着一盞七星燈,燈下有幾個草人,有頭上戴着腦箍的,有胸前穿着釘子的,有項上拴着鎖子的。櫃子裏無數紙人兒,底下幾篇小賬,上面記着某家驗過,應找銀若干。得人家油錢香分也不計其數。鳳姐道:“咱們的病,一準是他。我記得咱們病後,那老妖精向趙姨娘處來過幾次,要向趙姨娘討銀子,見了我,便臉上變貌變色,兩眼黧雞似的。我當初還猜疑了幾遍,總不知什麼原故。如今說起來,卻原來都是有因的。但只我在這裏當家,自然惹人恨怨,怪不得人治我。寶玉可和人有什麼仇呢,忍得下這樣毒手。”賈母道:“焉知不因我疼寶玉不疼環兒,竟給你們種了毒了呢。”王夫人道:“這老貨已經問了罪,決不好叫他來對證。沒有對證,趙姨娘那裏肯認賬。事情又大,鬧出來,外面也不雅,等他自作自受,少不得要自己敗露的。”賈母道:“你這話說的也是,這樣事,沒有對證,也難作準。只是佛爺菩薩看的真,他們姐兒兩個,如今又比誰不濟了呢。罷了,過去的事,鳳哥兒也不必提了。今日你和你太太都在我這邊吃了晚飯再過去罷。”遂叫鴛鴦琥珀等傳飯。鳳姐趕忙笑道:“怎麼老祖宗倒操起心來!”王夫人也笑了。只見外頭幾個媳婦伺候。鳳姐連忙告訴小丫頭子傳飯:“我和太太都跟着老太太喫。”正說着,只見玉釧兒走來對王夫人道:“老爺要找一件什麼東西,請太太伺候了老太太的飯完了自己去找一找呢。”賈母道:“你去罷,保不住你老爺有要緊的事。”王夫人答應着,便留下鳳姐兒伺候,自己退了出來。
\end{parag}


\begin{parag}
    回至房中,和賈政說了些閒話,把東西找了出來。賈政便問道:“迎兒已經回去了,他在孫家怎麼樣?”王夫人道:“迎丫頭一肚子眼淚,說孫姑爺兇橫的了不得。”因把迎春的話述了一遍。賈政嘆道:“我原知不是對頭,無奈大老爺已說定了,教我也沒法。不過迎丫頭受些委屈罷了。”王夫人道:“這還是新媳婦,只指望他以後好了好。”說着,嗤的一笑。賈政道:“笑什麼?”王夫人道:“我笑寶玉,今兒早起特特的到這屋裏來,說的都是些孩子話。”賈政道:“他說什麼?”王夫人把寶玉的言語笑述了一遍。賈政也忍不住的笑,因又說道:“你提寶玉,我正想起一件事來。這小孩子天天放在園裏,也不是事。生女兒不得濟,還是別人家的人,生兒若不濟事,關係非淺。前日倒有人和我提起一位先生來,學問人品都是極好的,也是南邊人。但我想南邊先生性情最是和平,咱們城裏的小孩,個個踢天弄井,鬼聰明倒是有的,可以搪塞就搪塞過去了,膽子又大,先生再要不肯給沒臉,一日哄哥兒似的,沒的白耽誤了。所以老輩子不肯請外頭的先生,只在本家擇出有年紀再有點學問的請來掌家塾。如今儒大太爺雖學問也只中平,但還彈壓的住這些小孩子們,不至以顢頇了事。我想寶玉閒着總不好,不如仍舊叫他家塾中讀書去罷了。”王夫人道:“老爺說的很是。自從老爺外任去了,他又常病,竟耽擱了好幾年。如今且在家學裏溫習溫習,也是好的。”賈政點頭,又說些閒話,不題。
\end{parag}


\begin{parag}
    且說寶玉次日起來,梳洗已畢,早有小廝們傳進話來說:“老爺叫二爺說話。”寶玉忙整理了衣服,來至賈政書房中,請了安站着。賈政道:“你近來作些什麼功課?雖有幾篇字,也算不得什麼。我看你近來的光景,越發比頭幾年散蕩了,況且每每聽見你推病不肯唸書。如今可大好了,我還聽見你天天在園子裏和姊妹們頑頑笑笑,甚至和那些丫頭們混鬧,把自己的正經事,總丟在腦袋後頭。就是做得幾句詩詞,也並不怎麼樣,有什麼稀罕處!比如應試選舉,到底以文章爲主,你這上頭倒沒有一點兒工夫。我可囑咐你:自今日起,再不許做詩做對的了,單要習學八股文章。限你一年,若毫無長進,你也不用唸書了,我也不願有你這樣的兒子了。”遂叫李貴來,說:“明兒一早,傳焙茗跟了寶玉去收拾應唸的書籍,一齊拿過來我看看,親自送他到家學裏去。”喝命寶玉:“去罷!明日起早來見我。”寶玉聽了,半日竟無一言可答,因回到怡紅院來。
\end{parag}


\begin{parag}
    襲人正在着急聽信,見說取書,倒也歡喜。獨是寶玉要人即刻送信與賈母,欲叫攔阻。賈母得信,便命人叫寶玉來,告訴他說:“只管放心先去,別叫你老子生氣。有什麼難爲你,有我呢。”寶玉沒法,只得回來囑咐了丫頭們:“明日早早叫我,老爺要等着送我到家學裏去呢。”襲人等答應了,同麝月兩個倒替着醒了一夜。
\end{parag}


\begin{parag}
    次日一早,襲人便叫醒寶玉,梳洗了,換了衣服,打發小丫頭子傳了焙茗在二門上伺候,拿着書籍等物。襲人又催了兩遍,寶玉只得出來過賈政書房中來,先打聽“老爺過來了沒有?”書房中小廝答應:“方纔一位清客相公請老爺回話,裏邊說梳洗呢,命清客相公出去候着去了。”寶玉聽了,心裏稍稍安頓,連忙到賈政這邊來。恰好賈政着人來叫,寶玉便跟着進去。賈政不免又囑咐幾句話,帶了寶玉上了車,焙茗拿着書籍,一直到家塾中來。
\end{parag}


\begin{parag}
    早有人先搶一步回代儒說:“老爺來了。”代儒站起身來,賈政早已走入,向代儒請了安。代儒拉着手問了好,又問:“老太太近日安麼?”寶玉過來也請了安。賈政站着,請代儒坐了,然後坐下。賈政道:“我今日自己送他來,因要求託一番。這孩子年紀也不小了,到底要學個成人的舉業,纔是終身立身成名之事。如今他在家中只是和些孩子們混鬧,雖懂得幾句詩詞,也是胡謅亂道的,就是好了,也不過是風雲月露,與一生的正事毫無關涉。”代儒道:“我看他相貌也還體面,靈性也還去得,爲什麼不念書,只是心野貪頑。詩詞一道,不是學不得的,只要發達了以後,再學還不遲呢。”賈政道:“原是如此。目今只求叫他讀書,講書,作文章。倘或不聽教訓,還求太爺認真的管教管教他,纔不至有名無實的白耽誤了他的一世。”說畢,站起來又作了一個揖,然後說了些閒話,才辭了出去。代儒送至門首,說:“老太太前替我問好請安罷。”賈政答應着,自己上車去了。
\end{parag}


\begin{parag}
    代儒回身進來,看見寶玉在西南角靠窗戶擺着一張花梨小桌,右邊堆下兩套舊書,薄薄兒的一本文章,叫焙茗將紙墨筆硯都擱在抽屜裏藏着。代儒道:“寶玉,我聽見說你前兒有病,如今可大好了?”寶玉站起來道:“大好了。”代儒道:“如今論起來,你可也該用功了。你父親望你成人懇切的很。你且把從前念過的書,打頭兒理一遍。每日早起理書,飯後寫字,晌午講書,念幾遍文章就是了。”寶玉答應了個“是”,回身坐下時,不免四面一看。見昔時金榮輩不見了幾個,又添了幾個小學生,都是些粗俗異常的。忽然想起秦鍾來,如今沒有一個做得伴說句知心話兒的,心上悽然不樂,卻不敢作聲,只是悶着看書。代儒告訴寶玉道:“今日頭一天,早些放你家去罷。明日要講書了。但是你又不是很愚夯的,明日我倒要你先講一兩章書我聽,試試你近來的工課何如,我才曉得你到怎麼個分兒上頭。”說得寶玉心中亂跳。欲知明日聽解何如,且聽下回分解。
\end{parag}