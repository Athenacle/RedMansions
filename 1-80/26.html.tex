\chap{二十六}{蜂腰桥设言传心事 潇湘馆春困发幽情}
\begin{parag}
    \begin{note}蒙回前词:一个时才得传消息,一个是旧喜化作新歌。真真假假二事堪疑,哭向花林月底。\end{note}
\end{parag}


\begin{parag}
    话说宝玉养过了三十三天之后,不但身体强壮,亦且连脸上疮痕平服,仍回大观园内去。这也不在话下。
\end{parag}


\begin{parag}
    且说近日宝玉病的时节,贾芸带著家下小厮坐更看守,昼夜在这里,那红玉同众丫鬟也在这里守著宝玉,彼此相见多日,都渐渐混熟了。那红玉见贾芸手里拿的手帕子,倒象是自己从前掉的,待要问他,又不好问的。不料那和尚道士来过,用不著一切男人,贾芸仍种树去了。这件事待要放下,心内又放不下,待要问去,又怕人猜疑,正是犹豫不决神魂不定之际,忽听窗外问道:“姐姐在屋里没有?”\begin{note}甲戌侧:岔开正文,却是为正文作引。\end{note}\begin{note}庚辰侧:你看他偏不写正文,偏有许多闲文,却是补遗。\end{note}红玉闻听,在窗眼内望外一看,原来是本院的个小丫头名叫佳蕙的,因答说:“在家里,你进来罢。”佳蕙听了跑进来,就坐在床上,笑道:“我好造化!才刚在院子里洗东西,宝玉叫往林姑娘那里送茶叶,\begin{note}甲戌侧:交代井井有法。\end{note}\begin{note}庚辰侧:前文有言。\end{note}花大姐姐交给我送去。可巧老太太那里给林姑娘送钱来,\begin{note}庚辰侧:是补写否?\end{note}正分给他们的丫头们呢。\begin{note}甲戌侧:潇湘常事出自别院婢口中,反觉新鲜。\end{note}见我去了,林姑娘就抓了两把给我,也不知多少。你替我收著。”便把手帕子打开,把钱倒了出来,红玉替他一五一十的数了收起。\begin{note}庚辰眉:此等细事是旧族大家闺中常情,今特为暴发钱奴写来作鉴。一笑。壬午夏,雨窗。\end{note}
\end{parag}


\begin{parag}
    佳蕙道:“你这一程子心里到底觉怎么样?依我说,你竟家去住两日,请一个大夫来瞧瞧,吃两剂药就好了。”红玉道:“那里的话,好好的,家去作什么!” 佳蕙道:“我想起来了,林姑娘生的弱,时常他吃药,\begin{note}庚辰侧:是补写否?\end{note}你就和他要些来吃,也是一样。”\begin{note}甲戌侧:闲言中叙出黛玉之弱。草蛇灰线。\end{note}红玉道:“胡说!\begin{note}庚辰侧:如闻。\end{note}药也是混吃的。”佳蕙道:“你这也不是个长法儿,又懒吃懒喝的,终久怎么样?”\begin{note}庚辰侧:从旁人眼中口中出,妙极!\end{note}红玉道:“怕什么,还不如早些儿死了倒干净!”\begin{note}甲戌侧:此句令人气噎,总在无可奈何上来。\end{note}佳蕙道:“好好的,怎么说这些话?”红玉道: “你那里知道我心里的事!”
\end{parag}


\begin{parag}
    佳蕙点头想了一会,道:“可也怨不得,这个地方难站。就象昨儿老太太因宝玉病了这些日子,\begin{note}庚辰侧:是补文否?\end{note}说跟著伏侍的这些人都辛苦了,如今身上好了,各处还完了愿,\begin{note}庚辰侧:是补写否?\end{note}叫把跟著的人都按著等儿赏他们。\begin{note}庚辰侧:是补写否?\end{note}我们算年纪小,上不去,我也不抱怨;像你怎么也不算在里头?\begin{note}庚辰侧:道著心病。\end{note}我心里就不服。袭人那怕他得十分儿,也不恼他,原该的。说良心话,谁还敢比他呢?\begin{note}庚辰侧:确论公论,方见袭卿身份。\end{note}别说他素日殷勤小心,便是不殷勤小心,也拼不得。可气晴雯、绮霰他们这几个,都算在上等里去,仗著老子娘的脸面,众人倒捧著他去。你说可气不可气?”红玉道:“也不犯著气他们。俗语说的好,‘千里搭长棚,没有个不散的筵席’,\begin{note}甲戌侧:此时写出此等言语,令人堕泪。\end{note}谁守谁一辈子呢?不过三年五载,各人干各人的去了。那时谁还管谁呢?”这两句话不觉感动了佳蕙的心肠,\begin{note}庚辰侧:不但佳蕙,批书者亦泪下矣。\end{note}由不得眼睛红了,又不好意思好端端的哭,只得勉强笑道:“你这话说的却是。昨儿宝玉还说,\begin{note}庚辰侧:还是补文。\end{note}明儿怎么样收拾房子,怎么样做衣裳,倒象有几百年的熬煎。”\begin{note}甲戌侧:却是小女儿口中无味之谈,实是写宝玉不如一环婢。\end{note}\begin{note}甲戌眉:红玉一腔委屈怨愤,系身在怡红不能遂志,看官勿错认为芸儿害相思也。己卯冬。\end{note}\begin{note}甲戌眉:“狱神庙”红玉、茜雪一大回文字惜迷失无稿。[庚眉多八字:叹叹!丁亥夏。畸笏叟。]\end{note}
\end{parag}


\begin{parag}
    红玉听了冷笑了两声,方要说话,\begin{note}甲戌侧:文字又一顿。\end{note}只见一个未留头的小丫头子走进来,手里拿著些花样子并两张纸,说道:“这是两个样子,叫你描出来呢。”说著向红玉掷下,回身就跑了。红玉向外问道:“倒是谁的?也等不得说完就跑,谁蒸下馒头等著你,怕冷了不成!”那小丫头在窗外只说得一声: “是绮大姐姐的。”\begin{note}甲戌侧:是不合式之言、擢心语。\end{note}抬起脚来咕咚咕咚又跑了。\begin{note}甲戌侧:活现,活现之文。\end{note}红玉便赌气把那样子掷在一边,\begin{note}庚辰侧:何如?\end{note}向抽屉内找笔,找了半天都是秃了的,因说道:“前儿一枝新笔,\begin{note}庚辰侧:是补文否?\end{note}放在那里了?怎么一时想不起来。”\begin{note}庚辰侧:既在矮檐下,怎敢不低头?\end{note}一面说著,一面出神,\begin{note}甲戌侧:总是画境。\end{note}想了一会方笑道:“是了,前儿晚上莺儿拿了去了。”\begin{note}庚辰侧:还是补文。\end{note}便向佳蕙道:“你替我取了来。”佳蕙道:“花大姐姐还等著我替他抬箱子呢,你自己取去罢。”红玉道:“他等著你,你还坐著闲打牙儿?\begin{note}庚辰侧:袭人身份。\end{note}我不叫你取去,他也不等著你了。坏透了的小蹄子!”说著,自己便出房来,出了怡红院,一径往宝钗院内来。\begin{note}庚辰侧:曲折再四,方逼出正文来。\end{note}
\end{parag}


\begin{parag}
    刚至沁芳亭畔,只见宝玉的奶娘李嬷嬷从那边走来。\begin{note}甲戌侧:奇文,真令人不得机关。\end{note}红玉立住笑问道:“李奶奶,你老人家那去了?怎打这里来?”李嬷嬷站住将手一拍道:“你说说,好好的又看上了\begin{note}甲戌侧:囫囵不解语。\end{note}那个种树的什么云哥儿雨哥儿的,\begin{note}甲戌侧:奇文神文。\end{note}这会子逼著我叫了他来。明儿叫上房里听见,可又是不好。”\begin{note}甲戌侧:更不解。\end{note}红玉笑道:“你老人家当真的就依了他去叫了?”\begin{note}甲戌侧:是遂心语。\end{note}李嬷嬷道:“可怎么样呢?”\begin{note}甲戌侧:妙!的是老妪口气。\end{note}红玉笑道:“那一个要是知道好歹,\begin{note}甲戌侧:更不解。\end{note}就回不进来才是。”\begin{note}甲戌双夹:是私心语,神妙!\end{note}李嬷嬷道:“他又不痴,为什么不进来?”红玉道:“既是进来,你老人家该同他一齐来,回来叫他一个人乱碰,可是不好呢。”\begin{note}甲戌双夹:总是私心语,要直问又不敢,只用这等语慢慢的套出。有神理。\end{note}李嬷嬷道:“我有那样工夫和他走?不过告诉了他,回来打发个小丫头子或是老婆子,带进他来就完了。” 说著,拄著拐杖一径去了。红玉听说,便站著出神,且不去取笔。\begin{note}甲戌双夹:总是不言神情,另出花样。\end{note}
\end{parag}


\begin{parag}
    一时,只见一个小丫头子跑来,见红玉站在那里,便问道:“林姐姐,你在这里作什么呢?”红玉抬头见是小丫头子坠儿。\begin{note}甲戌双夹:坠儿者,赘也。人生天地间已是赘疣,况又生许多冤情孽债。叹叹!\end{note}红玉道:“那去?”坠儿道:“叫我带进芸二爷来。”\begin{note}庚辰侧:等的是这句话。\end{note}说著一径跑了。这里红玉刚走至蜂腰桥门前,只见那边坠儿引著贾芸来了。\begin{note}甲戌双夹:妙!不说红玉不走,亦不说走,只说“刚走到”三字,可知红玉有私心矣。若说出必定不走必定走,则文字死板,且亦棱角过露,非写女儿之笔也。\end{note}那贾芸一面走,一面拿眼把红玉一溜;那红玉只装著和坠儿说话,也把眼去一溜贾芸:四目恰相对时,红玉不觉脸红了,\begin{note}甲戌双夹:看官至此,须掩卷细想上三十回中篇篇句句点“红”字处,可与此处想如何?\end{note}一扭身往蘅芜苑去了。不在话下。
\end{parag}


\begin{parag}
    这里贾芸随著坠儿,逶迤来至怡红院中。坠儿先进去回明了,然后方领贾芸进去。贾芸看时,只见院内略略有几点山石,种著芭蕉,那边有两只仙鹤在松树下剔翎。一溜回廊上吊著各色笼子,各色仙禽异鸟。上面小小五间抱厦,一色雕镂新鲜花样隔扇,上面悬著一个匾额,四个大字,题道是“怡红快绿”。贾芸想道:“怪道叫‘怡红院’,原来匾上是恁样四个字。”\begin{note}甲戌双夹:伤哉,转眼便红稀绿瘦矣。叹叹!\end{note}正想著,只听里面隔著纱窗子笑说道:\begin{note}甲戌侧:此文若张僧繇点睛之龙,破壁飞矣,焉得不拍案叫绝!\end{note}“快进来罢。我怎么就忘了你两三个月!”贾芸听得是宝玉的声音,连忙进入房内。抬头一看,只见金碧辉煌,\begin{note}甲戌侧:器皿叠叠。\end{note}\begin{note}庚辰侧:不能细览之文。\end{note}文章闪灼,\begin{note}甲戌侧:陈设垒垒。\end{note}\begin{note}庚辰侧:不得细玩之文。\end{note}却看不见宝玉在那里。\begin{note}甲戌侧:武夷九曲之文。\end{note}一回头,只见左边立著一架大穿衣镜,从镜后转出两个一般大的十五六岁的丫头来说:“请二爷里头屋里坐。”贾芸连正眼也不敢看,连忙答应了。又进一道碧纱厨,只见小小一张填漆床上,悬著大红销金撒花帐子。宝玉穿著家常衣服,靸著鞋,倚在床上拿著本书,\begin{note}甲戌侧:这是等芸哥看,故作款式。若果真看书,在隔纱窗子说话时已经放下了。玉兄若见此批,必云:老货,他处处不放松我,可恨可恨!回思将余比作钗、颦等,乃一知己,余何幸也!一笑。\end{note}看见他进来,将书掷下,早堆著笑立起身来。\begin{note}庚辰侧:小叔身段。\end{note}贾芸忙上前请了安。宝玉让坐,便在下面一张椅子上坐了。宝玉笑道:“只从那个月见了你,我叫你往书房里来,谁知接接连连许多事情,就把你忘了。”贾芸笑道:“总是我没福,偏偏又遇著叔叔身上欠安。叔叔如今可大安了?”宝玉道:“大好了。我倒听见说你辛苦了好几天。”贾芸道:“辛苦也是该当的。叔叔大安了,也是我们一家子的造化。”\begin{note}甲戌侧:不伦不理,迎合字样,口气逼肖,可笑可叹!\end{note}\begin{note}庚辰侧:谁一家子?可发一大笑。\end{note}
\end{parag}


\begin{parag}
    说著,只见有个丫鬟端了茶来与他。那贾芸口里和宝玉说著话,眼睛却溜瞅那丫鬟:\begin{note}甲戌侧:前写不敢正眼,今又如此写,是用茶来,有心人故留此神,于接茶时站起,方不突然。庚辰侧:此句是认人,非前溜红玉之文。\end{note}细挑身材,容长脸面,穿著银红袄儿,青缎背心,白绫细折裙。──不是别个,却是袭人。\begin{note}甲戌侧:《水浒》文法用的恰,当是芸哥眼中也。\end{note}那贾芸自从宝玉病了几天,他在里头混了两日,他却把那有名人口认记了一半。\begin{note}甲戌侧:一路总是贾芸是个有心人,一丝不乱。\end{note}他也知道袭人在宝玉房中比别个不同,\begin{note}庚辰侧:如何?可知余前批不谬。\end{note}今见他端了茶来,宝玉又在旁边坐著,便忙站起来笑道: “姐姐怎么替我倒起茶来。我来到叔叔这里,又不是客,让我自己倒罢。”\begin{note}甲戌双夹:总写贾芸乖觉,一丝不乱。\end{note}宝玉道:“你只管坐著罢。丫头们跟前也是这样。”贾芸笑道:“虽如此说,叔叔房里姐姐们,我怎么敢放肆呢?”\begin{note}甲戌侧:红玉何以使得?\end{note}一面说,一面坐下吃茶。
\end{parag}


\begin{parag}
    那宝玉便和他说些没要紧的散话。\begin{note}甲戌双夹:妙极是极!况宝玉又有何正紧\begin{subnote}注:蒙本此处作“经”\end{subnote}可说的!\end{note}又说道谁家的戏子好,谁家的花园好,又告诉他谁家的丫头标致,谁家的酒席丰盛,又是谁家有奇货,又是谁家有异物。\begin{note}甲戌双夹:几个“谁家”,自北静王公侯驸马诸大家包括尽矣,写尽纨绔口角。\end{note}\begin{note}庚辰侧:脂砚斋再笔:对芸兄原无可说之话。\end{note}那贾芸口里只得顺著他说,说了一会,见宝玉有些懒懒的了,便起身告辞。宝玉也不甚留,只说:“你明儿闲了,只管来。”仍命小丫头子坠儿送他出去。
\end{parag}


\begin{parag}
    出了怡红院,贾芸见四顾无人,便把脚慢慢停著些走,口里一长一短和坠儿说话,先问他“几岁了?名字叫什么?你父母在那一行上?在宝叔房内几年了?\begin{note}甲戌侧:渐渐入港。\end{note}一个月多少钱?共总宝叔房内有几个女孩子?”那坠儿见问,便一桩桩的都告诉他了。贾芸又道:“才刚那个与你说话的,他可是叫小红?” 坠儿笑道:“他倒叫小红。你问他作什么?”贾芸道:“方才他问你什么手帕子,我倒拣了一块。”坠儿听了笑道:“他问了我好几遍,可有看见他的帕子。我有那么大工夫管这些事!今儿他又问我,他说我替他找著了,他还谢我呢。\begin{note}庚辰侧:“传”字正文,此处方露。\end{note}才在蘅芜苑门口说的,二爷也听见了,不是我撒谎。好二爷,你既拣了,给我罢。我看他拿什么谢我。”
\end{parag}


\begin{parag}
    原来上月贾芸进来种树之时,便拣了一块罗帕,便知是所在园内的人失落的,但不知是那一个人的,故不敢造次。今听见红玉问坠儿,便知是红玉的,心内不胜喜幸。又见坠儿追索,心中早得了主意,便向袖内将自己的一块取了出来,向坠儿笑道:“我给是给你,你若得了他的谢礼,不许瞒著我。”坠儿满口里答应了,接了手帕子,送出贾芸,回来找红玉,不在话下。\begin{note}甲戌双夹:至此一顿,狡猾之甚!原非书中正文之人,写来间色耳。\end{note}
\end{parag}


\begin{parag}
    如今且说宝玉打发了贾芸去后,意思懒懒的歪在床上,似有朦胧之态。袭人便走上来,坐在床沿上推他,说道:“怎么又要睡觉?闷的很,你出去逛逛不是?” 宝玉见说,便拉他的手笑道:“我要去,只是舍不得你。”袭人笑道:“快起来罢!”\begin{note}甲戌侧:不答得妙!\end{note}\begin{note}庚辰侧:不答上文,妙极!\end{note}一面说,一面拉了宝玉起来。宝玉道:“可往那去呢?怪腻腻烦烦的。”\begin{note}庚辰侧:玉兄最得意之文,起笔却如此写。\end{note}袭人道:“你出去了就好了。只管这么葳蕤 ,越发心里烦腻。”
\end{parag}


\begin{parag}
    宝玉无精打采的,只得依他。晃出了房门,在回廊上调弄了一回雀儿;出至院外,顺著沁芳溪看了一回金鱼。只见那边山坡上两只小鹿箭也似的跑来,宝玉不解其意,\begin{note}甲戌侧:余亦不解。\end{note}正自纳闷,只见贾兰在后面拿著一张小弓追了下来。\begin{note}甲戌侧:前文。\end{note}\begin{note}庚辰侧:此等文可是人能意料的?\end{note}一见宝玉在前面,便站住了,笑道:“二叔叔在家里呢,我只当出门去了。”宝玉道:“你又淘气了。好好的射他作什么?”贾兰笑道:“这会子不念书,闲著作什么?所以演习演习骑射。”\begin{note}甲戌侧:奇文奇语,默思之方意会。为玉兄之毫无一正事,只知安富尊荣而写。\end{note}\begin{note}庚辰侧:答得何其堂皇正大,何其坦然之至!\end{note}宝玉道: “把牙栽了,那时才不演呢。”
\end{parag}


\begin{parag}
    说著,顺著脚一径来至一个院门前,\begin{note}庚辰侧:像无意。\end{note}只见凤尾森森,龙吟细细。\begin{note}甲戌双夹:与后文“落叶萧萧,寒烟漠漠”一对,可伤可叹!\end{note}\begin{note}庚批:原无意。\end{note}举目望门上一看,只见匾上写著“潇湘馆”三字。\begin{note}甲戌侧:无一丝心机,反似初至者,故接有忘形忘情话来。\end{note}\begin{note}庚辰侧:三字如此出,足见真出无意。\end{note}宝玉信步走入,只见湘帘垂地,悄无人声。走至窗前,觉得一缕幽香从碧纱窗中暗暗透出。\begin{note}甲戌侧:写得出,写得出。\end{note}宝玉便将脸贴在纱窗上,往里看时,耳内忽听得\begin{note}甲戌双夹:未曾看见先听见,有神理。\end{note}细细的长叹了一声道:“‘每日家情思睡昏昏’。”\begin{note}甲戌侧:用情忘情神化之文。\end{note}\begin{note}庚辰眉:先用“凤尾森森,龙吟细细”八字,“一缕幽香自纱窗中暗暗透出”,“细细的长叹一声”等句,方引出“每日家情思昏睡睡”仙音妙音来,非纯化功夫之笔不能,可见行文之难。\end{note}宝玉听了,不觉心内痒将起来,再看时,只见黛玉在床上伸懒腰。\begin{note}甲戌侧:有神理,真真画出。\end{note}宝玉在窗外笑道:“为甚么 ‘每日家情思睡昏昏’?”一面说,一面掀帘子进来了。\begin{note}庚辰眉:二玉这回文字,作者亦在无意上写来,所谓“信手拈来无不是”也。\end{note}
\end{parag}


\begin{parag}
    林黛玉自觉忘情,不觉红了脸,拿袖子遮了脸,翻身向里装睡著了。宝玉才走上来要搬他的身子,只见黛玉的奶娘并两个婆子却跟了进来\begin{note}甲戌侧:一丝不漏,且避若干嚼蜡之文。\end{note}说:“妹妹睡觉呢,等醒了再请来。”刚说著,黛玉便翻身坐了起来,笑道:“谁睡觉呢。”\begin{note}甲戌侧:妙极!可知黛玉是怕宝玉去也。\end{note}那两三个婆子见黛玉起来,便笑道:“我们只当姑娘睡著了。”说著,便叫紫鹃说:“姑娘醒了,进来伺侯。”一面说,一面都去了。
\end{parag}


\begin{parag}
    黛玉坐在床上,一面抬手整理鬓发,一面笑向宝玉道:“人家睡觉,你进来作什么?”宝玉见他星眼微饧,香腮带赤,不觉神魂早荡,一歪身坐在椅子上,笑道:“你才说什么?”黛玉道:“我没说什么。”宝玉笑道:“给你个榧子吃!我都听见了。”
\end{parag}


\begin{parag}
    二人正说话,只见紫鹃进来。宝玉笑道:“紫鹃,把你们的好茶倒碗我吃。”紫鹃道:“那里是好的呢?要好的,只是等袭人来。”黛玉道:“别理他,你先给我舀水去罢。”紫鹃笑道:“他是客,自然先倒了茶来再舀水去。”说著倒茶去了。宝玉笑道:“好丫头,‘若共你多情小姐同鸳帐,怎舍得叠被铺床?’”\begin{note}甲戌侧:真正无意忘情。\end{note}\begin{note}庚辰侧:真正无意忘情冲口而出之语。\end{note}\begin{note}庚辰眉:方才见芸哥所拿之书一定是《西厢记》,不然如何忘情之此?\end{note}林黛玉登时撂下脸来,\begin{note}甲戌侧:我也要恼。\end{note}说道:“二哥哥,你说什么?”宝玉笑道:“我何尝说什么。”黛玉便哭道:“如今新兴的,外头听了村话来,也说给我听;看了混帐书,也来拿我取笑儿。我成了爷们解闷的。”一面哭著,一面下床来往外就走。宝玉不知要怎样,心下慌了,忙赶上来,“好妹妹,我一时该死,你别告诉去。我再要敢,嘴上就长个疔,烂了舌头。”
\end{parag}


\begin{parag}
    正说著,只见袭人走来说道:“快回去穿衣服,老爷叫你呢。”\begin{note}庚辰眉:若无如此文字收拾二玉,写颦无非至再哭恸哭,玉只以赔尽小心软求漫恳,二人一笑而止。且书内若此亦多多矣,未免有犯雷同之病。故用险句结住,使二玉心中不得不将现事抛却,各怀一惊心意,再作下文。壬午孟夏,雨窗。畸笏。\end{note}宝玉听了,不觉打了个焦雷的一般,\begin{note}甲戌侧:不止玉兄一惊,即阿颦亦不免一吓,作者只顾写来收拾二玉之文,忘却颦儿也。想作者亦似宝玉道《西厢》之句,忘情而出也。\end{note}也顾不得别的,疾忙回来穿衣服。出园来,只见焙茗在二门前等著,宝玉便问道:“是作什么?”焙茗道:“爷快出来罢,横竖是见去的,到那里就知道了。”一面说,一面催著宝玉。
\end{parag}


\begin{parag}
    转过大厅,宝玉心里还自狐疑,只听墙角边一阵呵呵大笑,回头看时,见是薛蟠拍著手跳了出来,笑道:\begin{note}甲戌侧:如此戏弄,非呆兄无人。欲释二玉,非此戏弄不能立解,勿得泛泛看过。不知作者胸中有多少丘壑。\end{note}\begin{note}庚辰侧:非呆兄行不出此等戏弄,但作者有多少丘壑在胸中,写来酷肖。\end{note}“要不说姨夫叫你,你那里出来的这么快。”焙茗也笑著跪下了。宝玉怔了半天,方解过来了,是薛蟠哄他出来。薛蟠连忙打恭作揖陪不是,\begin{note}庚辰侧:酷肖。\end{note}又求“不要难为了小子,都是我逼他去的”。宝玉也无法了,只好笑问道:“你哄我也罢了,怎么说我父亲呢?我告诉姨娘去,评评这个理,可使得么?”薛蟠忙道:“好兄弟,我原为求你快些出来,就忘了忌讳这句话。改日你也哄我,说我的父亲就完了。”\begin{note}甲戌侧:写粗豪无心人毕肖。\end{note}\begin{note}庚辰侧:真真乱话。\end{note}宝玉道:“嗳,嗳,越发该死了。”又向焙茗道:“反叛肏的,还跪著作什么!”焙茗连忙叩头起来。薛蟠道:“要不是我也不敢惊动,只因明儿五月初三日是我的生日,谁知古董行的程日兴,他不知那里寻了来的这么粗这么长粉脆的鲜藕,\begin{note}庚辰侧:如见如闻。\end{note}这么大的大西瓜,这么长一尾新鲜的鲟鱼,这么大的一个暹罗国进贡的灵柏香熏的暹猪。你说,他这四样礼可难得不难得?那鱼,猪不过贵而难得,这藕和瓜亏他怎么种出来的。我连忙孝敬了母亲,赶著给你们老太太、姨父、姨母送了些去。如今留了些,我要自己吃,恐怕折福,\begin{note}甲戌侧:呆兄亦有此语,批书人至此诵《往生咒》至恒河沙数也。\end{note}左思右想,除我之外,惟有你还配吃,\begin{note}甲戌侧:此语令人哭不得笑不得,亦真心语也。\end{note}所以特请你来。可巧唱曲儿的小么儿又才来了,我同你乐一天何如?”
\end{parag}


\begin{parag}
    一面说,一面来至他书房里。只见詹光、程日兴、胡斯来、单聘仁等并唱曲儿的都在这里,见他进来,请安的,问好的,都彼此见过了。吃了茶,薛蟠即命人摆酒来。说犹未了,众小厮七手八脚摆了半天,\begin{note}庚辰侧:又一个写法。\end{note}方才停当归坐。宝玉果见瓜藕新异,因笑道:“我的寿礼还未送来,倒先扰了。”薛蟠道:“可是呢,明儿你送我什么?”\begin{note}庚辰侧:逼真酷肖。\end{note}宝玉道:“我可有什么可送的?若论银钱吃穿等类的东西,\begin{note}甲戌侧:谁说的出?经过者方说得出。叹叹!\end{note}究竟还不是我的,惟有我写一张字,画一张画,才算是我的。”
\end{parag}


\begin{parag}
    薛蟠笑道:“你提画儿,我才想起来。昨儿我看人家一张春宫,\begin{note}庚辰侧:阿呆兄所见之画也!\end{note}画的著实好。上面还有许多的字,也没细看,只看落的款,是‘庚黄’\begin{note}甲戌侧:奇文,奇文!\end{note}画的。真真的好的了不得!”宝玉听说,心下猜疑道:“古今字画也都见过些,那里有个‘庚黄’?”想了半天,不觉笑将起来,命人取过笔来,在手心里写了两个字,又问薛蟠道:“你看真了是‘庚黄’?”薛蟠道:“怎么看不真!”\begin{note}甲戌眉:闲事顺笔,骂死不学之纨绔。叹叹!\end{note}\begin{note}庚辰眉:闲事顺笔将骂死不学之纨绔。壬午雨窗。畸笏。\end{note}宝玉将手一撒,与他看道:“别是这两字罢?其实与‘庚黄’相去不远。”众人都看时,原来是“唐寅”两个字,都笑道:“想必是这两字,大爷一时眼花了也未可知。”薛蟠只觉没意思,\begin{note}庚辰侧:实心人。\end{note}笑道:“谁知他‘糖银’‘果银’的。”
\end{parag}


\begin{parag}
    正说著,小厮来回:“冯大爷来了。”宝玉便知是神武将军冯唐之子冯紫英来了。薛蟠等一齐都叫:“快请。”说犹未了,只见冯紫英一路说笑,\begin{note}庚辰侧:如见如闻。\end{note}已进来了。\begin{note}甲戌侧:一派英气如在纸上,特为金闺润色也。\end{note}众人忙起席让坐。冯紫英笑道:“好呀!也不出门了,在家里高乐罢。”\begin{note}如见其人于纸上。\end{note}宝玉薛蟠都笑道:“一向少会,老世伯身上康健?”紫英答道:“家父倒也托庇康健。近来家母偶著了些风寒,不好了两天。”\begin{note}庚辰眉:紫英豪侠小文三段,是为金闺间色之文,壬午雨窗。\end{note}\begin{note}庚辰眉:写倪二、紫英、湘莲、玉菡侠文,皆各得传真写照之笔。丁亥夏。畸笏叟。\end{note}\begin{note}庚辰眉:惜“卫若兰射圃”文字无稿。叹叹!丁亥夏。笏叟。\end{note}薛蟠见他面上有些青伤,便笑道:“这脸上又和谁挥拳的?挂了幌子了。”冯紫英笑道:“从那一遭把仇都尉的儿子打伤了,我就记了再不怄气,如何又挥拳?这个脸上,是前日打围,在铁网山教兔鹘捎一翅膀。”\begin{note}庚辰侧:如何著想?新奇字样。\end{note}宝玉道:“几时的话?”紫英道:“三月二十八日去的,前儿也就回来了。”宝玉道:“怪道前儿初三四儿,我在沈世兄家赴席不见你呢。我要问,不知怎么就忘了。单你去了,还是老世伯也去了?”紫英道:“可不是家父去,我没法儿,去罢了。难道我闲疯了,咱们几个人吃酒听唱的不乐,寻那个苦恼去?这一次,大不幸之中又大幸。”\begin{note}甲戌侧:似又伏一大事样,英侠人累累如是,令人猜摹。\end{note}
\end{parag}


\begin{parag}
    薛蟠众人见他吃完了茶,都说道:“且入席,有话慢慢的说。”\begin{note}庚辰侧:□文再述。\end{note}冯紫英听说,便立起身来说道:“论理,我该陪饮几杯才是,只是今儿有一件大大要紧的事,回去还要见家父面回,实不敢领。”薛蟠宝玉众人那里肯依,死拉著不放。冯紫英笑道:“这又奇了。\begin{note}庚辰侧:如闻如见。\end{note}你我这些年,那回儿有这个道理的?果然不能遵命。若必定叫我领,拿大杯来,\begin{note}庚辰侧:写豪爽人如此。\end{note}我领两杯就是了。”众人听说,只得罢了,薛蟠执壶,宝玉把盏,斟了两大海。那冯紫英站著,一气而尽。\begin{note}甲戌侧:令人快活煞。\end{note}\begin{note}庚辰侧:爽快人如此,令人羡煞。\end{note}宝玉道:“你到底把这个‘不幸之幸’说完了再走。”冯紫英笑道:“今儿说的也不尽兴。我为这个,还要特治一东,请你们去细谈一谈;二则还有所恳之处。”说著执手就走。薛蟠道:“越发说的人热剌剌的丢不下。多早晚才请我们,告诉了。也免的人犹疑。”\begin{note}甲戌侧:实心人如此,丝毫行迹俱无,令人痛快煞。\end{note}冯紫英道:“多则十日,少则八天。”一面说,一面出门上马去了。众人回来,依席又饮了一回方散。\begin{note}甲戌侧:收拾得好。\end{note}
\end{parag}


\begin{parag}
    宝玉回至园中,袭人正记挂著他去见贾政,\begin{note}甲戌侧:生员切己之事,时刻难忘。\end{note}不知是祸是福,\begin{note}庚辰侧:下文伏线。\end{note}只见宝玉醉醺醺的回来,问其原故,宝玉一一向他说了。袭人道:“人家牵肠挂肚的等著,你且高乐去,也到底打发人来给个信儿。”宝玉道:“我何尝不要送信儿,只因冯世兄来了,就混忘了。”
\end{parag}


\begin{parag}
    正说,只见宝钗走进来笑道:“偏了我们新鲜东西了。”宝玉笑道:“姐姐家的东西,自然先偏了我们了。”宝钗摇头笑道:“昨儿哥哥倒特特的请我吃,我不吃他,叫他留著请人送人罢。我知道我命小福薄,不配吃那个。”\begin{note}甲戌侧:暗对呆兄言宝玉配吃语。\end{note}说著,丫鬟倒了茶来,吃茶说闲话儿,不在话下。
\end{parag}


\begin{parag}
    却说那林黛玉听见贾政叫了宝玉去了,一日不回来,心中也替他忧虑。\begin{note}甲戌侧:本是切己事。\end{note}至晚饭后,闻听宝玉来了,心里要找他问问是怎么样了。\begin{note}甲戌侧:呆兄此席,的是合和筵也。一笑。\end{note}\begin{note}庚辰侧:这席东道是和事酒不是?\end{note}一步步行来,见宝钗进宝玉的院内去了,\begin{note}甲戌侧:《石头记》最好看处是此等章法。\end{note}自己也便随后走了来。刚到了沁芳桥,只见各色水禽都在池中浴水,也认不出名色来,但见一个个文彩炫耀,好看异常,因而站住看了一会。\begin{note}庚辰侧:避难法。\end{note}再往怡红院来,只见院门关著,黛玉便以手扣门。
\end{parag}


\begin{parag}
    谁知晴雯和碧痕正拌了嘴,没好气,忽见宝钗来了,那晴雯正把气移在宝钗身上,\begin{note}庚辰眉:晴雯迁怒是常事耳,写钗、颦二卿身上,与踢袭人之文,令人与何处设想著笔?丁亥夏。畸笏叟。\end{note}正在院内抱怨说:“有事没事跑了来坐著,\begin{note}甲戌侧:犯宝钗如此写法。\end{note}叫我们三更半夜的不得睡觉!”\begin{note}甲戌侧:指明人则暗写。\end{note}忽听又有人叫门,晴雯越发动了气,也并不问是谁,\begin{note}甲戌侧:犯黛玉如此写明。\end{note}便说道:“都睡下了,明儿再来罢!”\begin{note}甲戌侧:不知人则明写。\end{note}林黛玉素知丫头们的情性,他们彼此顽耍惯了,恐怕院内的丫头没听真是他的声音,只当是别的丫头们来了,所以不开门,因而又高声说道:“是我,还不开么?”晴雯偏生还没听出来,\begin{note}甲戌侧:想黛玉高声亦不过你我平常说话一样耳,况晴雯素昔浮躁多气之人,如何辨得出?此刻须得批书人唱“大江东去”的喉咙,嚷著“是我林黛玉叫门”方可。又想若开了门,如何有后面很多好字样好文章,看官者意为是否?\end{note}便使性子说道:“凭你是谁,二爷吩咐的,一概不许放人进来呢!”林黛玉听了,不觉气怔在门外,待要高声问他,逗起气来,自己又回思一番:“虽说是舅母家如同自己家一样,到底是客边。\begin{note}甲戌侧:寄食者著眼,况颦儿何等人乎?\end{note}如今父母双亡,无依无靠,现在他家依栖。如今认真淘气,也觉没趣。”一面想,一面又滚下泪珠来。正是回去不是,站著不是。正没主意,只听里面一阵笑语之声,细听一听,竟是宝玉、宝钗二人。林黛玉心中益发动了气,左思右想,忽然想起了早起的事来:“必竟是宝玉恼我要告他的原故。但只我何尝告你了,你也打听打听,就恼我到这步田地。你今儿不叫我进来,难道明儿就不见面了!”越想越伤感,也不顾苍苔露冷,花径风寒,独立墙角边花阴之下,悲悲戚戚呜咽起来。\begin{note}甲戌侧:可怜杀!可疼杀!余亦泪下。\end{note}
\end{parag}


\begin{parag}
    原来这林黛玉秉绝代姿容,具希世俊美,不期这一哭,那附近柳枝花朵上的宿鸟栖鸦一闻此声,俱忒楞楞飞起远避,不忍再听。真是:
\end{parag}


\begin{poem}
    \begin{pl}花魂默默无情绪,鸟梦痴痴何处惊。\end{pl}\begin{note}甲戌侧:沉鱼落雁,闭月羞花,原来是哭出来的。一笑。\end{note}
\end{poem}


\begin{parag}
    因有一首诗道:
\end{parag}


\begin{poem}
    \begin{pl}颦儿才貌世应希,独抱幽芳出绣闺;\end{pl}

    \begin{pl}呜咽一声犹未了,落花满地鸟惊飞。\end{pl}
\end{poem}


\begin{parag}
    那林黛玉正自啼哭,忽听“吱喽”一声,院门开处,不知是那一个出来。要知端的,且听下回分解。\begin{note}甲戌侧:每阅此本,掩卷者十有八九,不忍下阅看完,想作者此时泪下如豆矣。\end{note}
\end{parag}


\begin{parag}
    \begin{note}甲戌:此回乃颦儿正文,故借小红许多曲折琐碎之笔作引。\end{note}
\end{parag}


\begin{parag}
    \begin{note}甲戌:怡红院见贾芸,宝玉心内似有如无,贾芸眼中应接不暇。\end{note}
\end{parag}


\begin{parag}
    \begin{note}甲戌:“凤尾森森,龙吟细细”八字,“一缕幽香从碧纱窗中暗暗透出”,又“细细的长叹一声”等句方引出“每日家情思睡昏昏”仙音妙音,俱纯化工夫之笔。\end{note}
\end{parag}


\begin{parag}
    \begin{note}甲戌:二玉这回文字,作者亦在无意上写来,所谓“信手拈来无不是”也。\end{note}
\end{parag}


\begin{parag}
    \begin{note}甲戌:收拾二玉文字,写颦无非哭玉、再哭、恸哭,玉只以陪事小心软求慢恳,二人一笑而止。且书内若此亦多多矣,未免有犯雷同之病。故险语结住,使二玉心中不得不将现事抛却,各怀以惊心意,再作下文。\end{note}
\end{parag}


\begin{parag}
    \begin{note}甲戌:前回倪二、紫英、湘莲、玉菡四样侠文皆得传真写照之笔,惜“卫若兰射圃”文字迷失无稿,叹叹!\end{note}
\end{parag}


\begin{parag}
    \begin{note}甲戌:晴雯迁怒系常事耳,写于钗、颦二卿身上与踢袭人、打平儿之文,令人于何处设想著笔。\end{note}
\end{parag}


\begin{parag}
    \begin{note}甲戌:黛玉望怡红之泣,是“每日家情思睡昏昏”上来。\end{note}
\end{parag}


\begin{parag}
    \begin{note}蒙回后总评:喜相逢,三生注定;遗手帕,月老红丝。幸得人语说连理,又忽见他枝并蒂。难猜未解细追思,罔多疑,空向花枝哭月底。\end{note}
\end{parag}

