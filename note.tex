\chapter*{凡例}
\addcontentsline{toc}{chapter}{凡例}

\begin{qute2sp}
    \large
    \begin{parag}
        《红楼梦》旨意。是书题名极多,一曰《红楼梦》,是总其全部之名也。又曰《风月宝鉴》,是戒妄动风月之情。又曰《石头记》,是自譬石头所记之事也。此三名则书中曾已点睛矣。如宝玉做梦,梦中有曲,名曰《红楼梦》十二支,此则《红楼梦》之点睛。又如贾瑞病,跛道人持一镜来,上面即錾风月宝鉴四字,此则《风月宝鉴》之点睛。又如道人亲见石上大书一篇故事,则系石头所记之往来,此则《石头记》之点睛处。然此书又名曰《金陵十二钗》,审其名,则必系金陵十二女子也。然通部细搜检去,上中下女子岂止十二人哉?若云其中自有十二个,则又未尝指明白系某某,极至“红楼梦”一回中,亦曾翻出金陵十二钗之簿籍,又有十二支曲可考。
    \end{parag}

    \begin{parag}
        书中凡写长安,在文人笔墨之间,则从古之称;凡愚夫妇儿女子家常口角,则曰中京,是不欲著迹于方向也。盖天子之邦,亦当以中为尊,特避其东南西北四字样也。
    \end{parag}

    \begin{parag}
        此书只是著意于闺中,故叙闺中之事切,略涉于外事者则简,不得谓其不均也。
    \end{parag}

    \begin{parag}
        此书不敢干涉朝廷,凡有不得不用朝政者只略用一笔带出,盖实不敢以写儿女之笔墨唐突朝廷之上也。又不得谓其不备。
    \end{parag}
\end{qute2sp}

\cleardoublepage
\clearpage