\chap{一百零六}{王熙鳳致禍抱羞慚 賈太君禱天消禍患}



\begin{parag}
    話說賈政聞知賈母危急,即忙進去看視。見賈母驚嚇氣逆,王夫人鴛鴦等喚醒回來,即用疏氣安神的丸藥服了,漸漸的好些,只是傷心落淚。賈政在旁勸慰,總說是“兒子們不肖,招了禍來累老太太受驚。若老太太寬慰些,兒子們尚可在外料理;若是老太太有什麼不自在,兒子們的罪孽更重了。”賈母道:“我活了八十多歲,自作女孩兒起到你父親手裏,都託着祖宗的福,從沒有聽見過那些事。如今到老了,見你們倘或受罪,叫我心裏過得去麼!倒不如合上眼隨你們去罷了。”說着,又哭。
\end{parag}


\begin{parag}
    賈政此時着急異常,又聽外面說:“請老爺,內廷有信。”賈政急忙出來,見是北靜王府長史,一見面便說“大喜。”賈政謝了,請長史坐下,“請問王爺有何諭旨?”那長史道:“我們王爺同西平郡王進內復奏,將大人的懼怕的心,感激天恩之話都代奏了。主上甚是憫恤,並念及貴妃溘逝未久,不忍加罪,着加恩仍在工部員外上行走。所封家產,惟將賈赦的入官,餘俱給還。並傳旨令盡心供職。惟抄出借券令我們王爺查覈,如有違禁重利的一概照例入官,其在定例生息的同房地文書盡行給還。賈璉着革去職銜,免罪釋放。”賈政聽畢即起身叩謝天恩,又拜謝王爺恩典。“先請長史大人代爲稟謝,明晨到闕謝恩,併到府裏磕頭。”那長史去了。少停,傳出旨來。承辦官遵旨一一查清,入官者入官,給還者給還,將賈璉放出,所有賈赦名下男婦人等造冊入官。
\end{parag}


\begin{parag}
    可憐賈璉屋內東西除將按例放出的文書發給外,其餘雖未盡入官的,早被查抄的人盡行搶去,所存者只有傢伙對象。賈璉始則懼罪,後蒙釋放已是大幸,及想起歷年積聚的東西並鳳姐的體己不下七八萬金,一朝而盡,怎得不痛。且他父親現禁在錦衣府,鳳姐病在垂危,一時悲痛。又見賈政含淚叫他,問道:“我因官事在身,不大理家,故叫你們夫婦總理家事。你父親所爲固難勸諫,那重利盤剝究竟是誰幹的?況且非咱們這樣人家所爲。如今入了官,在銀錢是不打緊的,這種聲名出去還了得嗎!”賈璉跪下說道:“侄兒辦家事,並不敢存一點私心。所有出入的賬目,自有賴大,吳新登,戴良等登記,老爺只管叫他們來查問。現在這幾年,庫內的銀子出多入少,雖沒貼補在內,已在各處做了好些空頭,求老爺問太太就知道了。這些放出去的帳,連侄兒也不知道那裏的銀子,要問周瑞旺兒才知道。”賈政道:“據你說來,連你自己屋裏的事還不知道,那些家中上下的事更不知道了。我這回也不來查問你,現今你無事的人,你父親的事和你珍大哥的事還不快去打聽打聽。”賈璉一心委屈,含着眼淚答應了出去。賈政嘆氣連連的想道:“我祖父勤勞王事,立下功勳,得了兩個世職,如今兩房犯事都革去了。我瞧這些子侄沒一個長進的。老天啊,老天啊!我賈家何至一敗如此!我雖蒙聖恩格外垂慈,給還家產,那兩處食用自應歸併一處,叫我一人那裏支撐的住。方纔璉兒所說更加詫異,說不但庫上無銀,而且尚有虧空,這幾年竟是虛名在外。只恨我自己爲什麼糊塗若此。倘或我珠兒在世,尚有膀臂,寶玉雖大,更是無用之物。”想到那裏,不覺淚滿衣襟。又想:“老太太偌大年紀,兒子們並沒有自能奉養一日,反累他嚇得死去活來。種種罪孽,叫我委之何人!”正在獨自悲切,只見家人稟報各親友進來看候。賈政一一道謝,說起:“家門不幸,是我不能管教子侄,所以至此。”有的說:“我久知令兄赦大老爺行事不妥,那邊珍哥更加驕縱。若說因官事錯誤得個不是,於心無愧,如今自己鬧出的,倒帶累了二老爺。”有的說:“人家鬧的也多,也沒見御史參奏,不是珍老大得罪朋友,何至如此。”有的說:“也不怪御史,我們聽見說是府上的家人同幾個泥腿在外頭哄嚷出來的。御史恐參奏不實,所以誆了這裏的人去才說出來的。我想府上待下人最寬的,爲什麼還有這事。”有的說:“大凡奴才們是一個養活不得的。今兒在這裏都是好親友我纔敢說,就是尊駕在外任,我保不得——你是不愛錢的,——那外頭的風聲也不好,都是奴才們鬧的。你該堤防些。如今雖說沒有動你的家,倘或再遇着主上疑心起來,好些不便呢。”賈政聽說,心下着忙道:“衆位聽見我的風聲怎樣?”衆人道:“我們雖沒聽見實據,只聞外面人說你在糧道任上怎麼叫門上家人要錢。”賈政聽了,便說道:“我是對得天的,從不敢起這要錢的念頭。只是奴才在外招搖撞騙,鬧出事來我就喫不住了。”衆人道:“如今怕也無益,只好將現在的管家們都嚴嚴的查一查,若有抗主的奴才,查出來嚴嚴的辦一辦。”賈政聽了點頭。便見門上進來回稟說:“孫姑爺那邊打發人來說,自己有事不能來,着人來瞧瞧。說大老爺該他一種銀子,要在二老爺身上還的。”賈政心內憂悶,只說:“知道了。”衆人都冷笑道:“人說令親孫紹祖混賬,真有些。如今丈人抄了家,不但不來瞧看幫補照應,倒趕忙的來要銀子,真真不在理上。”賈政道:“如今且不必說他。那頭親事原是家兄配錯的,我的侄女兒的罪已經受夠了,如今又招我來。”正說着,只見薛蝌進來說道:“我打聽錦衣府趙堂官必要照御史參的辦去,只怕大老爺和珍大爺喫不住。”衆人都道:“二老爺,還得是你出去求求王爺,怎麼挽回挽回纔好。不然這兩家就完了。”賈政答應致謝,衆人都散。
\end{parag}


\begin{parag}
    那時天已點燈時候,賈政進去請賈母的安,見賈母略略好些。回到自己房中,埋怨賈璉夫婦不知好歹,如今鬧出放賬取利的事情,大家不好。方見鳳姐所爲,心裏很不受用。鳳姐現在病重,知他所有什物盡被抄搶一光,心內鬱結,一時未便埋怨,暫且隱忍不言。一夜無話。次早賈政進內謝恩,併到北靜王府西平王府兩處叩謝,求兩位王爺照應他哥哥侄兒。兩位應許。賈政又在同寅相好處託情。
\end{parag}


\begin{parag}
    且說賈璉打聽得父兄之事不很妥,無法可施,只得回到家中。平兒守着鳳姐哭泣,秋桐在耳房中抱怨鳳姐。賈璉走近旁邊,見鳳姐奄奄一息,就有多少怨言,一時也說不出來。平兒哭道:“如今事已如此,東西已去不能復來。奶奶這樣,還得再請個大夫調治調治纔好。”賈璉啐道:“我的性命還不保,我還管他麼!”鳳姐聽見,睜眼一瞧,雖不言語,那眼淚流個不盡,見賈璉出去,便與平兒道:“你別不達事務了,到了這樣田地,你還顧我做什麼。我巴不得今兒就死纔好。只要你能夠眼裏有我,我死之後,你扶養大了巧姐兒,我在陰司裏也感激你的。”平兒聽了,放聲大哭。鳳姐道:“你也是聰明人。他們雖沒有來說我,他必抱怨我。雖說事是外頭鬧的,我若不貪財,如今也沒有我的事,不但是枉費心計,掙了一輩子的強,如今落在人後頭。我只恨用人不當,恍惚聽得那邊珍大爺的事說是強佔良民妻子爲妾,不從逼死,有個姓張的在裏頭,你想想還有誰,若是這件事審出來,咱們二爺是脫不了的,我那時怎樣見人。我要實時就死,又耽不起吞金服毒的。你到還要請大夫,可不是你爲顧我反倒害了我了麼。”平兒愈聽愈慘,想來實在難處,恐鳳姐自尋短見,只得緊緊守着。幸賈母不知底細,因近日身子好些,又見賈政無事,寶玉寶釵在旁天天不離左右,略覺放心。素來最疼鳳姐,便叫鴛鴦“將我體己東西拿些給鳳丫頭,再拿些銀錢交給平兒,好好的伏侍好了鳳丫頭,我再慢慢的分派。”又命王夫人照看了邢夫人。又加了寧國府第入官,所有財產房地等並家奴等俱造冊收盡,這裏賈母命人將車接了尤氏婆媳等過來。可憐赫赫寧府只剩得他們婆媳兩個並佩鳳偕鸞二人,連一個下人沒有。賈母指出房子一所居住,就在惜春所住的間壁。又派了婆子四人丫頭兩個伏侍。一應飯食起居在大廚房內分送,衣裙什物又是賈母送去,零星需用亦在賬房內開銷,俱照榮府每人月例之數。那賈赦賈珍賈蓉在錦衣府使用,賬房內實在無項可支。如今鳳姐一無所有,賈璉況又多債務滿身,賈政不知家務,只說已經託人,自有照應。賈璉無計可施,想到那親戚裏頭薛姨媽家已敗,王子騰已死,餘者親戚雖有,俱是不能照應,只得暗暗差人下屯將地畝暫賣了數千金作爲監中使費。賈璉如此一行,那些家奴見主家勢敗,也便趁此弄鬼,並將東莊租稅也就指名借用些。此是後話,暫且不提。
\end{parag}


\begin{parag}
    且說賈母見祖宗世職革去,現在子孫在監質審,邢夫人尤氏等日夜啼哭,鳳姐病在垂危,雖有寶玉寶釵在側,只可解勸,不能分憂,所以日夜不寧,思前想後,眼淚不幹。一日傍晚,叫寶玉回去,自己扎掙坐起,叫鴛鴦等各處佛堂上香,又命自己院內焚起斗香,用拐拄着出到院中。琥珀知是老太太拜佛,鋪下大紅短氈拜墊。賈母上香跪下磕了好些頭,唸了一回佛,含淚祝告天地道:“皇天菩薩在上,我賈門史氏,虔誠禱告,求菩薩慈悲。我賈門數世以來,不敢行兇霸道。我幫夫助子,雖不能爲善,亦不敢作惡。必是後輩兒孫驕侈暴佚,暴殄天物,以致閤府抄檢。現在兒孫監禁,自然凶多吉少,皆由我一人罪孽,不教兒孫,所以至此。我今即求皇天保佑:在監逢凶化吉,有病的早早安身。總有閤家罪孽,情願一人承當,只求饒恕兒孫。若皇天見憐,念我虔誠,早早賜我一死,寬免兒孫之罪。”默默說到此,不禁傷心,嗚嗚咽咽的哭泣起來。鴛鴦珍珠一面解勸,一面扶進房去。只見王夫人帶了寶玉寶釵過來請晚安,見賈母悲傷,三人也大哭起來。寶釵更有一層苦楚:想哥哥也在外監,將來要處決,不知可減緩否,翁姑雖然無事,眼見家業蕭條,寶玉依然瘋傻,毫無志氣。想到後來終身,更比賈母王夫人哭得更痛。寶玉見寶釵如此大慟,他亦有一番悲慼。想的是老太太年老不得安,老爺太太見此光景不免悲傷,衆姐妹風流雲散,一日少似一日。追想在園中吟詩起社,何等熱鬧,自從林妹妹一死,我鬱悶到今,又有寶姐姐過來,未便時常悲切。見他憂兄思母,日夜難得笑容,今見他悲哀欲絕,心裏更加不忍,竟嚎啕大哭。鴛鴦,彩雲,鶯兒,襲人見他們如此,也各有所思,便也嗚咽起來。餘者丫頭們看得傷心,也便陪哭,竟無人解慰。滿屋中哭聲驚天動地,將外頭上夜婆子嚇慌,急報於賈政知道。那賈政正在書房納悶,聽見賈母的人來報,心中着忙,飛奔進內。遠遠聽得哭聲甚衆,打諒老太太不好,急得魂魄俱喪,疾忙進來,只見坐着悲啼,神魂方定。說是“老太太傷心,你們該勸解,怎麼的齊打夥兒哭起來了。”衆人聽得賈政聲氣,急忙止哭,大家對面發怔。賈政上前安慰了老太太,又說了衆人幾句。各自心想道:“我們原恐老太太悲傷,故來勸解,怎麼忘情大家痛哭起來。”正自不解,只見老婆子帶了史侯家的兩個女人進來,請了賈母的安,又向衆人請安畢,便說:“我們家老爺,太太,姑娘打發我來,說聽見府裏的事原沒有什麼大事,不過一時受驚。恐怕老爺太太煩惱,叫我們過來告訴一聲,說這裏二老爺是不怕的了。我們姑娘本要自己來的,因不多幾日就要出閣,所以不能來了。”賈母聽了,不便道謝,說:“你回去給我問好。這是我們的家運合該如此。承你老爺太太惦記,過一日再來奉謝。你家姑娘出閣,想來你們姑爺是不用說的了。他們的家計如何?”兩個女人回道:“家計倒不怎麼着,只是姑爺長的很好,爲人又和平。我們見過好幾次,看來與這裏寶二爺差不多,還聽得說才情學問都好的。”賈母聽了,喜歡道:“咱們都是南邊人,雖在這裏住久了,那些大規矩還是從南方禮兒,所以新姑爺我們都沒見過。我前兒還想起我孃家的人來,最疼的就是你們家姑娘,一年三百六十天,在我跟前的日子倒有二百多天,混得這麼大了。我原想給他說個好女婿,又爲他叔叔不在家,我又不便作主。他既造化配了個好姑爺,我也放心。月裏出閣我原想過來喫杯喜酒的,不料我家鬧出這樣事來,我的心就象在熱鍋裏熬的似的,那裏能夠再到你們家去。你回去說我問好,我們這裏的人都說請安問好。你替另告訴你家姑娘,不要將我放在心裏。我是八十多歲的人了,就死也算不得沒福的了。只願他過了門,兩口子和順,百年到老,我便安心了。”說着,不覺掉下淚來。那女人道:“老太太也不必傷心。姑娘過了門,等回了九,少不得同姑爺過來請老太太的安,那時老太太見了才喜歡呢。”賈母點頭。那女人出去。別人都不理論,只有寶玉聽了發了一回怔,心裏想道:“如今一天一天的都過不得了。爲什麼人家養了女兒到大了必要出嫁,一出了嫁就改變。史妹妹這樣一個人又被他叔叔硬壓着配人了,他將來見了我必是又不理我了。我想一個人到了這個沒人理的分兒,還活着做什麼。”想到那裏,又是傷心。見賈母此時才安,又不敢哭泣,只是悶悶的。
\end{parag}


\begin{parag}
    一時賈政不放心,又進來瞧瞧老太太,見是好些,便出來傳了賴大,叫他將閤府裏管事家人的花名冊子拿來,一齊點了一點,除去賈赦入官的人,尚有三十餘家,共男女二百十二名。賈政叫現在府內當差的男人共二十一名進來,問起歷年居家用度,共有若干進來,該用若干出去。那管總的家人將近來支用簿子呈上。賈政看時,所入不敷所出,又加連年宮裏花用,帳上有在外浮借的也不少。再查東省地租,近年所交不及祖上一半,如今用度比祖上更加十倍。賈政不看則已,看了急得跺腳道:“這了不得!我打量雖是璉兒管事,在家自有把持,豈知好幾年頭裏已就寅年用了卯年的,還是這樣裝好看,竟把世職俸祿當作不打緊的事情,爲什麼不敗呢!我如今要就省儉起來,已是遲了。”想到那裏,揹着手踱來踱去,竟無方法。衆人知賈政不知理家,也是白操心着急,便說道:“老爺也不用焦心,這是家家這樣的。若是統總算起來,連王爺家還不夠。不過是裝着門面,過到那裏就到那裏。如今老爺到底得了主上的恩典,纔有這點子家產,若是一併入了官,老爺就不用過了不成。”賈政嗔道:“放屁!你們這班奴才最沒有良心的,仗着主子好的時候任意開銷,到弄光了,走的走,跑的跑,還顧主子的死活嗎!如今你們道是沒有查封是好,那知道外頭的名聲。大本兒都保不住,還擱得住你們在外頭支架子說大話誆人騙人,到鬧出事來望主子身上一推就完了。如今大老爺與珍大爺的事,說是咱們家人鮑二在外傳播的,我看這人口冊上並沒有鮑二,這是怎麼說?”衆人回道:“這鮑二是不在冊檔上的。先前在寧府冊上,爲二爺見他老實,把他們兩口子叫過來了。及至他女人死了,他又回寧府去。後來老爺衙門有事,老太太們爺們往陵上去,珍大爺替理家事帶過來的,以後也就去了。老爺數年不管家事,那裏知道這些事來。老爺打量冊上沒有名字的就只有這個人,不知一個人手下親戚們也有,奴才還有奴才呢。”賈政道:“這還了得!”想去一時不能清理,只得喝退衆人,早打了主意在心裏了,且聽賈赦等事審得怎樣再定。
\end{parag}


\begin{parag}
    一日正在書房籌算,只見一人飛奔進來說:“請老爺快進內廷問話。”賈政聽了心下着忙,只得進去。未知兇吉,下回分解。
\end{parag}
