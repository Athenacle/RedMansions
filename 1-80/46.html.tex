\chap{四十六}{尴尬人难免尴尬事 鸳鸯女誓绝鸳鸯偶}
\begin{parag}
    \begin{note}庚辰:此回亦有本而笔,非泛泛之笔也。\end{note}
\end{parag}


\begin{parag}
    \begin{note}庚辰:只看他题纲用“尴尬”二字于邢夫人,可知包藏含蓄文字之中莫能量也。\end{note}
\end{parag}


\begin{parag}
    \begin{note}蒙回前总:裹脚与缠头,欲觅终身伴。顾影自为怜,静住深深院。好事不称心,恶语将人慢。誓死守香闺,远却扬花片。\end{note}
\end{parag}


\begin{parag}
    话说林黛玉直到四更将阑,方渐渐的睡去,暂且无话。如今且说凤姐儿因见邢夫人叫他,不知何事,忙另穿戴了一番,坐车过来。邢夫人将房内人遣出,悄向凤姐儿道:“叫你来不为别事,有一件为难的事,老爷托我,我不得主意,先和你商议。老爷因看上了老太太的鸳鸯,要他在房里,叫我和老太太讨去。我想这倒平常有的事,只是怕老太太不给,你可有法子?”凤姐儿听了,忙道:“依我说,竟别碰这个钉子去。老太太离了鸳鸯,饭也吃不下去的,那里就舍得了?况且平日说起闲话来,老太太常说,老爷如今上了年纪,作什么左一个小老婆右一个小老婆放在屋里,没的耽误了人家。放著身子不保养,官儿也不好生作去,成日家和小老婆喝酒。太太听这话,很喜欢老爷呢?这会子回避还恐回避不及,倒拿草棍儿戳老虎的鼻子眼儿去了!太太别恼,我是不敢去的。明放著不中用,而且反招出没意思来。老爷如今上了年纪,行事不妥,太太该劝才是。比不得年轻,作这些事无碍。如今兄弟、侄儿、儿子、孙子一大群,还这么闹起来,怎样见人呢?”邢夫人冷笑道: “大家子三房四妾的也多,偏咱们就使不得?我劝了也未必依。就是老太太心爱的丫头,这么胡子苍白了又作了官的一个大儿子,要了作房里人,也未必好驳回的。我叫了你来,不过商议商议,你先派上了一篇不是。也有叫你去的理?自然是我说去。你倒说我不劝,你还不知道那性子的,劝不成,先和我恼了。”
\end{parag}


\begin{parag}
    凤姐儿知道邢夫人禀性愚犟(注:蒙本此作“拙”),只知承顺贾赦以自保,次则婪取财货为自得,家下一应大小事务,俱由贾赦摆布。凡出入银钱事务,一经他手,便克啬异常,以贾赦浪费为名,“须得我就中俭省,方可偿补”,儿女奴仆,一人不靠,一言不听的。如今又听邢夫人如此的话,便知他又弄左性,劝了不中用,连忙陪笑说道:“太太这话说的极是。我能活了多大,知道什么轻重?想来父母跟前,别说一个丫头,就是那么大的活宝贝,不给老爷给谁?背地里的话那里信得?我竟是个呆子。琏二爷或有日得了不是,老爷太太恨的那样,恨不得立刻拿来一下子打死;及至见了面,也罢了,依旧拿著老爷太太心爱的东西赏他。如今老太太待老爷,自然也是那样了。依我说,老太太今儿喜欢,要讨今儿就讨去。我先过去哄著老太太发笑,等太太过去了,我搭讪著走开,把屋子里的人我也带开,太太好和老太太说的。给了更好,不给也没妨碍,众人也不知道。”邢夫人见他这般说,便又喜欢起来,又告诉他道:“我的主意先不和老太太要。老太太要说不给,这事便死了。我心里想著先悄悄的和鸳鸯说。他虽害臊,我细细的告诉了他,他自然不言语,就妥了。那时再和老太太说,老太太虽不依,搁不住他愿意,常言‘人去不中留’,自然这就妥了。”凤儿姐笑道:“到底是太太有智谋,这是千妥万妥的。别说是鸳鸯,凭他是谁,那一个不想巴高望上,不想出头的?这半个主子不做,倒愿意做个丫头,将来配个小子就完了。”邢夫人笑道:“正是这个话了。别说鸳鸯,就是那些执事的大丫头,谁不愿意这样呢。你先过去,别露一点风声,我吃了晚饭就过来。”
\end{parag}


\begin{parag}
    凤姐儿暗想:“鸳鸯素习是个可恶的,虽如此说,保不严他就愿意。我先过去了,太太后过去,若他依了便没话说;倘或不依,太太是多疑的人,只怕就疑我走了风声,使他拿腔作势的。那时太太又见了应了我的话,羞恼变成怒,拿我出起气来,倒没意思。不如同著一齐过去了,他依也罢,不依也罢,就疑不到我身上了。”想毕,因笑道:“方才临来,舅母那边送了两笼子鹌鹑,我吩咐他们炸了,原要赶太太晚饭上送过来的。我才进大门时,见小子们抬车,说太太的车拔了缝,拿去收拾去了。不如这会子坐了我的车一齐过去倒好。”邢夫人听了,便命人来换衣服。凤姐忙著伏侍了一回,娘儿两个坐车过来。凤姐儿又说道:“太太过老太太那里去,我若跟了去,老太太若问起我过去作什么的,倒不好。不如太太先去,我脱了衣裳再来。”
\end{parag}


\begin{parag}
    邢夫人听了有理,便自往贾母处,和贾母说了一回闲话,便出来假托往王夫人房里去,从后门出去,打鸳鸯的卧房前过。只见鸳鸯正然坐在那里做针线,见了邢夫人,忙站起来。邢夫人笑道:“做什么呢?我瞧瞧,你扎的花儿越发好了。”一面说,一面便接他手内的针线瞧了一瞧,只管赞好。放下针线,又浑身打量。只见他穿著半新的藕合色的绫袄,青缎掐牙背心,下面水绿裙子。蜂腰削背,鸭蛋脸面,乌油头发,高高的鼻子,两边腮上微微的几点雀斑。鸳鸯见这般看他,自己倒不好意思起来,心里便觉诧异,因笑问道:“太太,这会子不早不晚的,过来做什么?”邢夫人使个眼色儿,跟的人退出。邢夫人便坐下,拉著鸳鸯的手笑道:“我特来给你道喜来了。”鸳鸯听了,心中已猜著三分,不觉红了脸,低了头不发一言。听邢夫人道:“你知道你老爷跟前竟没有个可靠的人,\begin{note}庚辰双夹:说得得体。我正想开口一句不知如何说,如此则妙极是极,如闻如见。\end{note}心里再要买一个,又怕那些人牙子家出来的不干不净,也不知道毛病儿,买了来家,三日两日,又要肏鬼吊猴的。因满府里要挑一个家生女儿收了,又没个好的:不是模样儿不好,就是性子不好,有了这个好处,没了那个好处。因此冷眼选了半年,这些女孩子里头,就只你是个尖儿,模样儿,行事作人,温柔可靠,一概是齐全的。意思要和老太太讨了你去,收在屋里。你比不得外头新买的,你这一进去了,进门就开了脸,就封你姨娘,又体面,又尊贵。你又是个要强的人,俗语说的,‘金子终得金子换’,谁知竟被老爷看重了你。如今这一来,你可遂了素日志大心高的愿了,也堵一堵那些嫌你的人的嘴。跟了我回老太太去!”说著拉了他的手就要走。鸳鸯红了脸,夺手不行。邢夫人知他害臊,因又说道:“这有什么臊处?你又不用说话,只跟著我就是了。”鸳鸯只低了头不动身。邢夫人见他这般,便又说道:“难道你不愿意不成?若果然不愿意,可真是个傻丫头了。放著主子奶奶不作,倒愿意作丫头!三年二年,不过配上个小子,还是奴才。你跟了我们去,你知道我的性子又好,又不是那不容人的人。老爷待你们又好。过一年半载,生下个一男半女,你就和我并肩了。家里的人你要使唤谁,谁还不动?现成主子不做去,错过这个机会,后悔就迟了。”鸳鸯只管低了头,仍是不语。邢夫人又道:“你这么个响快人,怎么又这样积粘起来?有什么不称心之处,只管说与我,我管你遂心如意就是了。”鸳鸯仍不语。邢夫人又笑道:“想必你有老子娘,你自己不肯说话,怕臊。你等他们问你,这也是理。让我问他们去,叫他们来问你,有话只管告诉他们。”说毕,便往凤姐儿房中来。
\end{parag}


\begin{parag}
    凤姐儿早换了衣服,因房内无人,便将此话告诉了平儿。平儿也摇头笑道:“据我看,此事未必妥。平常我们背著人说起话来,听他那主意,未必是肯的。也只说著瞧罢了。”凤姐儿道:“太太必来这屋里商议。依了还可,若不依,白讨个臊,当著你们,岂不脸上不好看。你说给他们炸鹌鹑,再有什么配几样,预备吃饭。你且别处逛逛去,估量著去了再来。”平儿听说,照样传给婆子们,便逍遥自在的往园子里来。
\end{parag}


\begin{parag}
    这里鸳鸯见邢夫人去了,必在凤姐儿房里商议去了,必定有人来问他的,不如躲了这里,\begin{note}庚辰双夹:终不免女儿气,不知躲在哪里方无人来罗唣,写得可怜可爱。\end{note}因找了琥珀说道:“老太太要问我,只说我病了,没吃早饭,往园子里逛逛就来。”琥珀答应了。鸳鸯也往园子里来,各处游玩,不想正遇见平儿。平儿因见无人,便笑道:“新姨娘来了!”鸳鸯听了,便红了脸,说道:“怪道你们串通一气来算计我!等著我和你主子闹去就是了。”平儿听了,自悔失言,便拉他到枫树底下,\begin{note}庚辰双夹:随笔带出妙景,正愁园中草木黄落,不想看此一句便恍如置身于千霞万锦绛雪红霜之中矣。\end{note}坐在一块石上,越性把方才凤姐过去回来所有的形景言词始末原由告诉与他。鸳鸯红了脸,向平儿冷笑道:“这是咱们好,比如袭人、琥珀、素云、紫鹃、彩霞、玉钏儿、麝月、翠墨,跟了史姑娘去的翠缕,死了的可人和金钏,去了的茜雪,\begin{note}庚辰双夹:余按此一算,亦是十二钗,真镜中花、水中月、云中豹、林中之鸟、穴中之鼠、无数可考、无人可指、有迹可追、有形可据、九曲八折、远响近影、迷离烟灼、纵横隐现、千奇百怪、眩目移神、现千手千眼大游戏法也。脂砚斋。\end{note}连上你我,这十来个人,从小儿什么话儿不说?什么事儿不作?这如今因都大了,各自干各自的去了,\begin{note}庚辰双夹:此语已可伤,犹未各自干各自去,后日更有各自之处也,知之乎!\end{note}然我心里仍是照旧,有话有事,并不瞒你们。这话我且放在你心里,且别和二奶奶说:别说大老爷要我做小老婆,就是太太这会子死了,他三媒六聘的娶我去作大老婆,我也不能去。”
\end{parag}


\begin{parag}
    平儿方欲笑答,只听山石背后哈哈的笑道:“好个没脸的丫头,亏你不怕牙碜。”二人听了不免吃了一惊,忙起身向山石背后找寻,不是别个,却是袭人笑著走了出来问:“什么事情?告诉我。”说著,三人坐在石上。平儿又把方才的话说与袭人听道:“真真这话论理不该我们说,这个大老爷太好色了,略平头正脸的,他就不放手了。”平儿道:“你既不愿意,我教你个法子,不用费事就完了。”鸳鸯道:“什么法子?你说来我听。”平儿笑道:“你只和老太太说,就说已经给了琏二爷了,大老爷就不好要了。”鸳鸯啐道:“什么东西!你还说呢!前儿你主子不是这么混说的?谁知应到今儿了!”袭人笑道:“他们两个都不愿意,我就和老太太说,叫老太太说把你已经许了宝玉了,大老爷也就死了心了。”鸳鸯又是气,又是臊,又是急,因骂道:“两个蹄子不得好死的!人家有为难的事,拿著你们当正经人,告诉你们与我排解排解,你们倒替换著取笑儿。你们自为都有了结果了,将来都是做姨娘的。据我看,天下的事未必都遂心如意。你们且收著些儿,别忒乐过了头儿!”二人见他急了,忙陪笑央告道:“好姐姐,别多心,咱们从小儿都是亲姊妹一般,不过无人处偶然取个笑儿。你的主意告诉我们知道,也好放心。”鸳鸯道:“什么主意!我只不去就完了。”平儿摇头道:“你不去未必得干休。大老爷的性子你是知道的。虽然你是老太太房里的人,此刻不敢把你怎么样,将来难道你跟老太太一辈子不成?也要出去的。那时落了他的手,倒不好了。”鸳鸯冷笑道:“老太太在一日,我一日不离这里;若是老太太归西去了,他横竖还有三年的孝呢,没个娘才死了他先纳小老婆的!等过三年,知道又是怎么个光景,那时再说。纵到了至急为难,我剪了头发作姑子去;不然,还有一死。一辈子不嫁男人,又怎么样?乐得干净呢!”平儿袭人笑道:“真这蹄子没了脸,越发信口儿都说出来了。”鸳鸯道:“事到如此,臊一会怎么样!你们不信,慢慢的看著就是了。太太才说了,找我老子娘去。我看他南京找去!”平儿道:“你的父母都在南京看房子,没上来,终久也寻的著。现在还有你哥哥嫂子在这里。可惜你是这里的的家生女儿,不如我们两个人是单在这里。”鸳鸯道:“家生女儿怎么样?‘牛不吃水强按头’?我不愿意,难道杀我的老子娘不成?”
\end{parag}


\begin{parag}
    正说著,只见他嫂子从那边走来。袭人道:“当时找不著你的爹娘,一定和你嫂子说了。”鸳鸯道:“这个娼妇专管是个‘九国贩骆驼的’,听了这话,他有个不奉承去的!”说话之间,已来到跟前。他嫂子笑道:“那里没找到,姑娘跑了这里来!你跟了我来,我和你说话。”平儿袭人都忙让坐。他嫂子说:“姑娘们请坐,我找我们姑娘说句话。” 袭人平儿都装不知道,笑道:“什么话这样忙?我们这里猜谜儿赢手批子打呢,等猜了这个再去。”鸳鸯道:“什么话?你说罢。”他嫂子笑道:“你跟我来,到那里我告诉你,横竖有好话儿。”鸳鸯道:“可是大太太和你说的那话?”他嫂子笑道:“姑娘既知道,还奈何我!快来,我细细的告诉你可是天大的喜事。”鸳鸯听说,立起身来,照他嫂子脸上下死劲啐了一口,指著他骂道:“你快夹著屄嘴离了这里,好多著呢!什么‘好话’!宋徽宗的鹰,赵子昂的马,都是好画儿。什么 ‘喜事’!状元痘儿灌的浆儿又满是喜事。怪道成日家羡慕人家女儿作了小老婆了,一家子都仗著他横行霸道的,一家子都成了小老婆了!看的眼热了,也把我送在火坑里去。我若得脸呢,你们外头横行霸道,自己就封自己是舅爷了。我若不得脸败了时,你们把忘八脖子一缩,生死由我。”一面说,一面哭,平儿袭人拦著劝。他嫂子脸上下不来,因说道:“愿意不愿意,你也好说,不犯著牵三挂四的。俗语说,‘当著矮人,别说矮话’。姑奶奶骂我,我不敢还言;这二位姑娘并没惹著你,小老婆长小老婆短,大家脸上怎么过得去?”袭人平儿忙道:“你倒别这么说,他也并不是说我们,你倒别牵三挂四的。你听见那位太太、太爷们封我们做小老婆?况且我们两个也没有爹娘哥哥兄弟在这门子里仗著我们横行霸道的。他骂的人自有他骂的,我们犯不著多心。”鸳鸯道:“他见我骂了他,他臊了,没的盖脸,又拿话挑唆你们两个,幸亏你们两个明白。原是我急了,也没分别出来,他就挑出这个空儿来。”他嫂子自觉没趣,赌气去了。
\end{parag}


\begin{parag}
    鸳鸯气得还骂,平儿袭人劝他一回,方才罢了。平儿因问袭人道:“你在那里藏著做甚么的?我们竟没看见你。”袭人道:“我因为往四姑娘房里瞧我们宝二爷去的,谁知迟了一步,说是来家里来了。我疑惑怎么不遇见呢,想要往林姑娘家里找去,又遇见他的人说也没去。我这里正疑惑是出园子去了,可巧你从那里来了,我一闪,你也没看见。后来他又来了。我从这树后头走到山子石后,我却见你两个说话来了,谁知你们四个眼睛没见我。”
\end{parag}


\begin{parag}
    一语未了,又听身后笑道:“四个眼睛没见你?你们六个眼睛竟没见我!”三人唬了一跳,回身一看,不是别个,正是宝玉走来。\begin{note}庚辰双夹:通部情案皆必从石兄挂号,然各有各稿,穿插神妙。\end{note}袭人先笑道:“叫我好找,你那里来?”宝玉笑道:“我从四妹妹那里出来,迎头看见你来了,我就知道是找我去的,我就藏了起来哄你。看你低著头过去了,进了院子就出来了,逢人就问。我在那里好笑,只等你到了跟前唬你一跳的,后来见你也藏藏躲躲的,我就知道也是要哄人了。我探头往前看了一看,却是他两个,所以我就绕到你身后。你出去,我就躲在你躲的那里了。”平儿笑道:“咱们再往后找找去,只怕还找出两个人来也未可知。”宝玉笑道:“这可再没了。”鸳鸯已知话俱被宝玉听了,只伏在石头上装睡。宝玉推他笑道:“这石头上冷,咱们回房里去睡,岂不好?”说著拉起鸳鸯来,又忙让平儿来家坐吃茶。平儿和袭人都劝鸳鸯走,鸳鸯方立起身来,四人竟往怡红院来。宝玉将方才的话俱已听见,心中自然不快,只默默的歪在床上,任他三人在外间说笑。
\end{parag}


\begin{parag}
    那边邢夫人因问凤姐儿鸳鸯的父母,凤姐因回说:“他爹的名字叫金彩,\begin{note}庚辰双夹:姓金名彩,由“鸳鸯”二字化出,因文而生文也。\end{note}两口子都在南京看房子,从不大上京。他哥哥金文翔,\begin{note}庚辰双夹:更妙!\end{note}现在是老太太那边的买办。他嫂子也是老太太那边浆洗的头儿。”\begin{note}庚辰双夹:只鸳鸯一家写得荣府中人各有各职,如目已睹。\end{note}邢夫人便令人叫了他嫂子金文翔媳妇来,细细说与他。金家媳妇自是喜欢,兴兴头头找鸳鸯,只望一说必妥,不想被鸳鸯抢白一顿,又被袭人平儿说了几句,羞恼回来,便对邢夫人说:“不中用,他倒骂了我一场。”因凤姐儿在旁,不敢提平儿,只说:“袭人也帮著他抢白我,也说了许多不知好歹的话,回不得主子的。太太和老爷商议再买罢。谅那小蹄子也没有这么大福,我们也没有这么大造化。”邢夫人听了,因说道:“又与袭人什么相干?他们如何知道的?”又问:“还有谁在跟前?”金家的道:“还有平姑娘。”凤姐儿忙道:“你不该拿嘴巴子打他回来?我一出了门,他就逛去了;回家来连一个影儿也摸不著他!他必定也帮著说什么呢!”金家的道:“平姑娘没在跟前,远远的看著倒象是他,可也不真切,不过是我白忖度。”凤姐便命人去:“快打了他来,告诉他我来家了,太太也在这里,请他来帮个忙儿。”丰儿忙上来回道:“林姑娘打发了人下请字请了三四次,他才去了。奶奶一进门我就叫他去的。林姑娘说:‘告诉你奶奶,我烦他有事呢。’”凤姐儿听了方罢,故意的还说:“天天烦他,有些什么事!”
\end{parag}


\begin{parag}
    邢夫人无计,吃了饭回家,晚间告诉了贾赦。贾赦想了一想,即刻叫贾琏来说:“南京的房子还有人看著,不止一家,即刻叫上金彩来。”贾琏回道:“上次南京信来,金彩已经得了痰迷心窍,那边连棺材银子都赏了,不知如今是死是活,便是活著,人事不知,叫来也无用。他老婆子又是个聋子。”贾赦听了,喝了一声,又骂:“下流囚攮的,偏你这么知道,还不离了我这里!”唬得贾琏退出,一时又叫传金文翔。贾琏在外书房伺候著,又不敢家去,又不敢见他父亲,只得听著。一时金文翔来了,小幺儿们直带入二门里去,隔了五六顿饭的工夫才出来去了。贾琏暂且不敢打听,隔了一会,又打听贾赦睡了,方才过来。至晚间凤姐儿告诉他,方才明白。
\end{parag}


\begin{parag}
    鸳鸯一夜没睡,至次日,他哥哥回贾母接他家去逛逛,贾母允了,命他出去。鸳鸯意欲不去,只怕贾母疑心,只得勉强出来。他哥哥只得将贾赦的话说与他,又许他怎么体面,又怎么当家作姨娘。鸳鸯只咬定牙不愿意。他哥哥无法,少不得去回复了贾赦。贾赦怒起来,因说道:“我这话告诉你,叫你女人向他说去,就说我的话:‘自古嫦娥爱少年’,他必定嫌我老了,大约他恋著少爷们,多半是看上了宝玉,只怕也有贾琏。果有此心,叫他早早歇了心,我要他不来,此后谁还敢收?此是一件。第二件,想著老太太疼他,将来自然往外聘作正头夫妻去。叫他细想,凭他嫁到谁家去,也难出我的手心。除非他死了,或是终身不嫁男人,我就伏了他!若不然时,叫他趁早回心转意,有多少好处。”贾赦说一句,金文翔应一声“是”。贾赦道:“你别哄我,我明儿还打发你太太过去问鸳鸯,你们说了,他不依,便没你们的不是。若问他,他再依了,仔细你的脑袋!”
\end{parag}


\begin{parag}
    金文翔忙应了又应,退出回家,也不等得告诉他女人转说,竟自已对面说了这话。把个鸳鸯气的无话可回,想了一想,便说道:“便愿意去,也须得你们带了我回声老太太去。”他哥嫂听了,只当回想过来,都喜之不胜。他嫂子即刻带了他上来见贾母。
\end{parag}


\begin{parag}
    可巧王夫人、薛姨妈、李纨、凤姐儿、宝钗等姊妹并外头的几个执事有头脸的媳妇,都在贾母跟前凑趣儿呢。鸳鸯喜之不尽,拉了他嫂子,到贾母跟前跪下,一行哭,一行说,把邢夫人怎么来说,园子里他嫂子又如何说,今儿他哥哥又如何说,“因为不依,方才大老爷越性说我恋著宝玉,不然要等著往外聘,我到天上,这一辈子也跳不出他的手心去,终久要报仇。我是横了心的,当著众人在这里,我这一辈子莫说是‘宝玉’,便是‘宝金’‘宝银’‘宝天王’‘宝皇帝’,横竖不嫁人就完了!就是老太太逼著我,我一刀子抹死了,也不能从命!若有造化,我死在老太太之先;若没造化,该讨吃的命,伏侍老太太归了西,我也不跟著我老子娘哥哥去,我或是寻死,或是剪了头发当尼姑去!若说我不是真心,暂且拿话来支吾,日后再图别的,天地鬼神,日头月亮照著嗓子,从嗓子里头长疔烂了出来,烂化成酱在这里!”原来他一进来时,便袖了一把剪子,一面说著,一面左手打开头发,右手便铰。众婆娘丫鬟忙来拉住,已剪下半绺来了。众人看时,幸而他的头发极多,铰的不透,连忙替他挽上。贾母听了,气的浑身乱战,口内只说:“我通共剩了这么一个可靠的人,他们还要来算计!”因见王夫人在旁,便向王夫人道:“你们原来都是哄我的!外头孝敬,暗地里盘算我。有好东西也来要,有好人也要,剩了这么个毛丫头,见我待他好了,你们自然气不过,弄开了他,好摆弄我!”王夫人忙站起来,不敢还一言。\begin{note}庚辰双夹:千奇百怪,王妇人亦有罪乎?老人家迁怒之言必应如此。\end{note}薛姨妈见连王夫人怪上,反不好劝的了。李纨一听见鸳鸯的话,早带了姊妹们出去。
\end{parag}


\begin{parag}
    探春有心的人,想王夫人虽有委曲,如何敢辩;薛姨妈也是亲姊妹,自然也不好辩的;宝钗也不便为姨母辩;李纨、凤姐、宝玉一概不敢辩;这正用著女孩儿之时,迎春老实,惜春小,因此窗外听了一听,便走进来陪笑向贾母道:“这事与太太什么相干?老太太想一想,也有大伯子要收屋里的人,小婶子如何知道?便知道,也推不知道。”犹未说完,贾母笑道:“可是我老糊涂了!姨太太别笑话我。你这个姐姐他极孝顺我,不象我那大太太一味怕老爷,婆婆跟前不过应景儿。可是委屈了他。”薛姨妈只答应“是”,又说:“老太太偏心,多疼小儿子媳妇,也是有的。”贾母道:“不偏心!”因又说道:“宝玉,我错怪了你娘,你怎么也不提我,看著你娘受委屈?”宝玉笑道:“我偏著娘说大爷大娘不成?通共一个不是,我娘在这里不认,却推谁去?我倒要认是我的不是,老太太又不信。”贾母笑道: “这也有理。你快给你娘跪下,你说太太别委屈了,老太太有年纪了,看著宝玉罢。”宝玉听了,忙走过去,便跪下要说;王夫人忙笑著拉他起来,说:“快起来,快起来,断乎使不得。终不成你替老太太给我赔不是不成?”宝玉听说,忙站起来。\begin{note}庚辰双夹:宝玉亦有罪了。\end{note}贾母又笑道:“凤姐儿也不提我。”\begin{note}庚辰双夹:阿凤也有了罪。奇奇怪怪之文,所谓《石头记》不是作出来的。\end{note}凤姐儿笑道:“我倒不派老太太的不是,老太太倒寻上我了?”贾母听了,与众人都笑道:“这可奇了!倒要听听这不是。”凤姐儿道:“谁教老太太会调理人,调理的水葱儿似的,怎么怨得人要?我幸亏是孙子媳妇,若是孙子,我早要了,还等到这会子呢。”贾母笑道:“这倒是我的不是了?”凤姐儿笑道:“自然是老太太的不是了。”贾母笑道:“这样,我也不要了,你带了去罢!”凤姐儿道:“等著修了这辈子,来生托生男人,我再要罢。”贾母笑道:“你带了去,给琏儿放在屋里,看你那没脸的公公还要不要了!”凤姐儿道:“琏儿不配,就只配我和平儿这一对烧糊了的卷子和他混罢。”说的众人都笑起来了。丫鬟回说:“大太太来了。”王夫人忙迎了出去。要知端的
\end{parag}


\begin{parag}
    \begin{note}蒙回末总:鸳鸯女从热闹中别具一副肠胃,不轻许人一事,是官途中药石仙方。\end{note}
\end{parag}

