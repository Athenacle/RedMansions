\chap{八十二}{老學究講義警頑心 病瀟湘癡魂驚惡夢}



\begin{parag}
    話說寶玉下學回來,見了賈母。賈母笑道:“好了,如今野馬上了籠頭了。去罷,見見你老爺,回來散散兒去罷。”寶玉答應着,去見賈政。賈政道:“這早晚就下了學了麼?師父給你定了工課沒有?”寶玉道:“定了。早起理書,飯後寫字,晌午講書念文章。”賈政聽了,點點頭兒,因道:“去罷,還到老太太那邊陪着坐坐去。你也該學些人功道理,別一味的貪頑。晚上早些睡,天天上學早些起來。你聽見了?”寶玉連忙答應幾個“是”,退出來,忙忙又去見王夫人,又到賈母那邊打了個照面兒。
\end{parag}


\begin{parag}
    趕着出來,恨不得一走就走到瀟湘館纔好。剛進門口,便拍着手笑道:“我依舊回來了!”猛可裏倒唬了黛玉一跳。紫鵑打起簾子,寶玉進來坐下。黛玉道:“我恍惚聽見你念書去了。這麼早就回來了?”寶玉道:“噯呀,了不得!我今兒不是被老爺叫了唸書去了麼,心上倒象沒有和你們見面的日子了。好容易熬了一天,這會子瞧見你們,竟如死而復生的一樣,真真古人說‘一日三秋,這話再不錯的。”黛玉道:“你上頭去過了沒有?”寶玉道:“都去過了。”黛玉道:“別處呢?”寶玉道:“沒有。”黛玉道:“你也該瞧瞧他們去。”寶玉道:“我這會子懶待動了,只和妹妹坐着說一會子話兒罷。老爺還叫早睡早起,只好明兒再瞧他們去了。”黛玉道:“你坐坐兒,可是正該歇歇兒去了。”寶玉道:“我那裏是乏,只是悶得慌。這會子咱們坐着才把悶散了,你又催起我來。”黛玉微微的一笑,因叫紫鵑:“把我的龍井茶給二爺沏一碗。二爺如今唸書了,比不的頭裏。”紫鵑笑着答應,去拿茶葉,叫小丫頭子沏茶。寶玉接着說道:“還提什麼唸書,我最厭這些道學話。更可笑的是八股文章,拿他誆功名混飯喫也罷了,還要說代聖賢立言。好些的,不過拿些經書湊搭湊搭還罷了,更有一種可笑的,肚子裏原沒有什麼,東拉西扯,弄的牛鬼蛇神,還自以爲博奧。這那裏是闡發聖賢的道理。目下老爺口口聲聲叫我學這個,我又不敢違拗,你這會子還提唸書呢。”黛玉道:“我們女孩兒家雖然不要這個,但小時跟着你們雨村先生唸書,也曾看過。內中也有近情近理的,也有清微淡遠的。那時候雖不大懂,也覺得好,不可一概抹倒。況且你要取功名,這個也清貴些。”寶玉聽到這裏,覺得不甚入耳,因想黛玉從來不是這樣人,怎麼也這樣勢欲燻心起來?又不敢在他跟前駁回,只在鼻子眼裏笑了一聲。正說着,忽聽外面兩個人說話,卻是秋紋和紫鵑。只聽秋紋道:“襲人姐姐叫我老太太那裏接去,誰知卻在這裏。”紫鵑道:“我們這裏才沏了茶,索性讓他喝了再去。”說着,二人一齊進來。寶玉和秋紋笑道:“我就過去,又勞動你來找。”秋紋未及答言,只見紫鵑道:“你快喝了茶去罷,人家都想了一天了。”秋紋啐道:“呸,好混賬丫頭!”說的大家都笑了。寶玉起身才辭了出來。黛玉送到屋門口兒,紫鵑在臺階下站着,寶玉出去,纔回房裏來。
\end{parag}


\begin{parag}
    卻說寶玉回到怡紅院中,進了屋子,只見襲人從裏間迎出來,便問:“回來了麼?”秋紋應道:“二爺早來了,在林姑娘那邊來着。”寶玉道:“今日有事沒有? ”襲人道:“事卻沒有。方纔太太叫鴛鴦姐姐來吩咐我們:如今老爺發狠叫你念書,如有丫鬟們再敢和你頑笑,都要照着晴雯司棋的例辦。我想,伏侍你一場,賺了這些言語,也沒什麼趣兒。”說着,便傷起心來。寶玉忙道:“好姐姐,你放心。我只好生唸書,太太再不說你們了。我今兒晚上還要看書,明日師父叫我講書呢。我要使喚,橫豎有麝月秋紋呢,你歇歇去罷。”襲人道:“你要真肯唸書,我們伏侍你也是歡喜的。”寶玉聽了,趕忙吃了晚飯,就叫點燈,把念過的”四書”翻出來。只是從何處看起?翻了一本,看去章章裏頭似乎明白,細按起來,卻不很明白。看着小注,又看講章,鬧到梆子下來了,自己想道:“我在詩詞上覺得很容易,在這個上頭竟沒頭腦。”便坐着呆呆的呆想。襲人道:“歇歇罷,做工夫也不在這一時的。”寶玉嘴裏只管胡亂答應。麝月襲人才伏侍他睡下,兩個才也睡了。及至睡醒一覺,聽得寶玉炕上還是翻來覆去。襲人道:“你還醒着呢麼?你倒別混想了,養養神明兒好唸書。”寶玉道:“我也是這樣想,只是睡不着。你來給我揭去一層被。”襲人道:“天氣不熱,別揭罷。”寶玉道:“我心裏煩躁的很。”自把被窩褪下來。襲人忙爬起來按住,把手去他頭上一摸,覺得微微有些發燒。襲人道:“你別動了,有些發燒了。”寶玉道:“可不是。”襲人道:“這是怎麼說呢!”寶玉道:“不怕,是我心煩的原故。你別吵嚷,省得老爺知道了,必說我裝病逃學,不然怎麼病的這樣巧。明兒好了,原到學裏去就完事了。”襲人也覺得可憐,說道:“我靠着你睡罷。”便和寶玉捶了一回脊樑,不知不覺大家都睡着了。直到紅日高升,方纔起來。寶玉道:“不好了,晚了!”急忙梳洗畢,問了安,就往學裏來了。代儒已經變着臉,說:“怪不得你老爺生氣,說你沒出息。第二天你就懶惰,這是什麼時候纔來!”寶玉把昨兒發燒的話說了一遍,方過去了,原舊唸書。到了下晚,代儒道:“寶玉,有一章書你來講講。”寶玉過來一看,卻是”後生可畏”章。寶玉心上說:“這還好,幸虧不是‘學’‘庸’。”問道:“怎麼講呢?”代儒道:“你把節旨句子細細兒講來。”寶玉把這章先朗朗的唸了一遍,說:“這章書是聖人勸勉後生,教他及時努力,不要弄到……”說到這裏,抬頭向代儒一瞧。代儒覺得了,笑了一笑道:“你只管說,講書是沒有什麼避忌的。《禮記》上說‘臨文不諱’,只管說,‘不要弄到’什麼?”寶玉道:“不要弄到老大無成。先將‘可畏’二字激發後生的志氣,後把‘不足畏’二字警惕後生的將來。”說罷,看着代儒。代儒道:“也還罷了。串講呢?”寶玉道:“聖人說,人生少時,心思才力,樣樣聰明能幹,實在是可怕的。那裏料得定他後來的日子不象我的今日。若是悠悠忽忽到了四十歲,又到五十歲,既不能夠發達,這種人雖是他後生時象個有用的,到了那個時候,這一輩子就沒有人怕他了。”代儒笑道:“你方纔節旨講的倒清楚,只是句子裏有些孩子氣。‘無聞’二字不是不能發達做官的話。‘聞’是實在自己能夠明理見道,就不做官也是有‘聞’了。不然,古聖賢有遁世不見知的,豈不是不做官的人,難道也是‘無聞’麼?‘不足畏’是使人料得定,方與‘焉知’的‘知’字對針,不是‘怕’的字眼。要從這裏看出,方能入細。你懂得不懂得?”寶玉道:“懂得了。”代儒道:“還有一章,你也講一講。”代儒往前揭了一篇,指給寶玉。寶玉看是”吾未見好德如好色者也。”寶玉覺得這一章卻有些刺心,便陪笑道:“這句話沒有什麼講頭。”代儒道:“胡說!譬如場中出了這個題目,也說沒有做頭麼?”寶玉不得已,講道:“是聖人看見人不肯好德,見了色便好的了不得。殊不想德是性中本有的東西,人偏都不肯好他。至於那個色呢,雖也是從先天中帶來,無人不好的。但是德乃天理,色是人慾,人那裏肯把天理好的象人慾似的。孔子雖是嘆息的話,又是望人迴轉來的意思。並且見得人就有好德的好得終是浮淺,直要象色一樣的好起來,那纔是真好呢。”代儒道:“這也講的罷了。我有句話問你:你既懂得聖人的話,爲什麼正犯着這兩件病?我雖不在家中,你們老爺也不曾告訴我,其實你的毛病我卻盡知的。做一個人,怎麼不望長進?你這會兒正是‘後生可畏’的時候,‘有聞’‘不足畏’全在你自己做去了。我如今限你一個月,把念過的舊書全要理清,再念一個月文章。以後我要出題目叫你作文章了。如若懈怠,我是斷乎不依的。自古道:‘成人不自在,自在不成人。’你好生記着我的話。”寶玉答應了,也只得天天按着功課幹去。不提。
\end{parag}


\begin{parag}
    且說寶玉上學之後,怡紅院中甚覺清淨閒暇。襲人倒可做些活計,拿着針線要繡個檳榔包兒,想着如今寶玉有了工課,丫頭們可也沒有饑荒了。早要如此,晴雯何至弄到沒有結果?兔死狐悲,不覺滴下淚來。忽又想到自己終身本不是寶玉的正配,原是偏房。寶玉的爲人,卻還拿得住,只怕娶了一個利害的,自己便是尤二姐香菱的後身。素來看着賈母王夫人光景及鳳姐兒往往露出話來,自然是黛玉無疑了。那黛玉就是個多心人。想到此際,臉紅心熱,拿着針不知戳到那裏去了,便把活計放下,走到黛玉處去探探他的口氣。
\end{parag}


\begin{parag}
    黛玉正在那裏看書,見是襲人,欠身讓坐。襲人也連忙迎上來問:“姑娘這幾天身子可大好了?”黛玉道:“那裏能夠,不過略硬朗些。你在家裏做什麼呢?”襲人道:“如今寶二爺上了學,房中一點事兒沒有,因此來瞧瞧姑娘,說說話兒。”說着,紫鵑拿茶來。襲人忙站起來道:“妹妹坐着罷。”因又笑道:“我前兒聽見秋紋說,妹妹背地裏說我們什麼來着。”紫鵑也笑道:“姐姐信他的話!我說寶二爺上了學,寶姑娘又隔斷了,連香菱也不過來,自然是悶的。”襲人道:“你還提香菱呢,這才苦呢,撞着這位太歲奶奶,難爲他怎麼過!”把手伸着兩個指頭道:“說起來,比他還利害,連外頭的臉面都不顧了。”黛玉接着道:“他也夠受了,尤二姑娘怎麼死了。”襲人道:“可不是。想來都是一個人,不過名分裏頭差些,何苦這樣毒?外面名聲也不好聽。”黛玉從不聞襲人背地裏說人,今聽此話有因,便說道:“這也難說。但凡家庭之事,不是東風壓了西風,就是西風壓了東風。”襲人道:“做了旁邊人,心裏先怯了,那裏倒敢去欺負人呢。”
\end{parag}


\begin{parag}
    說着,只見一個婆子在院裏問道:“這裏是林姑娘的屋子麼?”那位姐姐在這裏呢?”雪雁出來一看,模模糊糊認得是薛姨媽那邊的人,便問道:“作什麼?”婆子道:“我們姑娘打發來給這裏林姑娘送東西的。”雪雁道:“略等等兒。”雪雁進來回了黛玉,黛玉便叫領他進來。那婆子進來請了安,且不說送什麼,只是覷着眼瞧黛玉,看的黛玉臉上倒不好意思起來,因問道:“寶姑娘叫你來送什麼?”婆子方笑着回道:“我們姑娘叫給姑娘送了一瓶兒蜜餞荔枝來。”回頭又瞧見襲人,便問道:“這位姑娘不是寶二爺屋裏的花姑娘麼?”襲人笑道:“媽媽怎麼認得我?”婆子笑道:“我們只在太太屋裏看屋子,不大跟太太姑娘出門,所以姑娘們都不大認得。姑娘們碰着到我們那邊去,我們都模糊記得。”說着,將一個瓶兒遞給雪雁,又回頭看看黛玉,因笑着向襲人道:“怨不得我們太太說這林姑娘和你們寶二爺是一對兒,原來真是天仙似的。”襲人見他說話造次,連忙岔道:“媽媽,你乏了,坐坐喫茶罷。”那婆子笑嘻嘻的道:“我們那裏忙呢,都張羅琴姑娘的事呢。姑娘還有兩瓶荔枝,叫給寶二爺送去。”說着,顫顫巍巍告辭出去。黛玉雖惱這婆子方纔冒撞,但因是寶釵使來的,也不好怎麼樣他。等他出了屋門,才說一聲道:“給你們姑娘道費心。”那老婆子還只管嘴裏咕咕噥噥的說:“這樣好模樣兒,除了寶玉,什麼人擎受的起。”黛玉只裝沒聽見。襲人笑道:“怎麼人到了老來,就是混說白道的,叫人聽着又生氣,又好笑。”一時雪雁拿過瓶子來與黛玉看。黛玉道:“我懶待喫,拿了擱起去罷。”又說了一回話,襲人才去了。
\end{parag}


\begin{parag}
    一時晚妝將卸,黛玉進了套間,猛抬頭看見了荔枝瓶,不禁想起日間老婆子的一番混話,甚是刺心。當此黃昏人靜,千愁萬緒,堆上心來。想起自己身上不牢,年紀又大了。看寶玉的光景,心裏雖沒別人,但是老太太舅母又不見有半點意思。深恨父母在時,何不早定了這頭婚姻。又轉念一想道:“倘若父母在時,別處定了婚姻,怎能夠似寶玉這般人才心地,不如此時尚有可圖。”心內一上一下,輾轉纏綿,竟象轆轤一般。嘆了一回氣,掉了幾點淚,無情無緒,和衣倒下。
\end{parag}


\begin{parag}
    不知不覺,只見小丫頭走來說道:“外面雨村賈老爺請姑娘。”黛玉道:“我雖跟他讀過書,卻不比男學生,要見我作什麼?況且他和舅舅往來,從未提起,我也不便見的。”因叫小丫頭:“回覆‘身上有病不能出來’,與我請安道謝就是了。”小丫頭道:“只怕要與姑娘道喜,南京還有人來接。”說着,又見鳳姐同邢夫人,王夫人,寶釵等都來笑道:“我們一來道喜,二來送行。”黛玉慌道:“你們說什麼話?”鳳姐道:“你還裝什麼呆。你難道不知道林姑爺升了湖北的糧道,娶了一位繼母,十分合心合意。如今想着你撂在這裏,不成事體,因託了賈雨村作媒,將你許了你繼母的什麼親戚,還說是續絃,所以着人到這裏來接你回去。大約一到家中就要過去的,都是你繼母作主。怕的是道兒上沒有照應,還叫你璉二哥哥送去。”說得黛玉一身冷汗。黛玉又恍惚父親果在那裏做官的樣子,心上急着硬說道:“沒有的事,都是鳳姐姐混鬧。”只見邢夫人向王夫人使個眼色兒,”他還不信呢,咱們走罷。”黛玉含着淚道:“二位舅母坐坐去。”衆人不言語,都冷笑而去。黛玉此時心中乾急,又說不出來,哽哽咽咽。恍惚又是和賈母在一處的似的,心中想道:“此事惟求老太太,或還可救。”於是兩腿跪下去,抱着賈母的腰說道:“老太太救我!我南邊是死也不去的!況且有了繼母,又不是我的親孃。我是情願跟着老太太一塊兒的。”但見老太太待著臉兒笑道:“這個不干我事。”黛玉哭道:“老太太,這是什麼事呢。”老太太道:“續絃也好,倒多一副妝奩。”黛玉哭道:“我若在老太太跟前,決不使這裏分外的閒錢,只求老太太救我。”賈母道:“不中用了。做了女人,終是要出嫁的,你孩子家,不知道,在此地終非了局。”黛玉道:“我在這裏情願自己做個奴婢過活,自做自喫,也是願意。只求老太太作主。”老太太總不言語。黛玉抱着賈母的腰哭道:“老太太,你向來最是慈悲的,又最疼我的,到了緊急的時候怎麼全不管!不要說我是你的外孫女兒,是隔了一層了,我的娘是你的親生女兒,看我娘分上,也該護庇些。”說着,撞在懷裏痛哭,聽見賈母道:“鴛鴦,你來送姑娘出去歇歇。我倒被他鬧乏了。”黛玉情知不是路了,求去無用,不如尋個自盡,站起來往外就走。深痛自己沒有親孃,便是外祖母與舅母姊妹們,平時何等待的好,可見都是假的。又一想:“今日怎麼獨不見寶玉?或見一面,看他還有法兒?”便見寶玉站在面前,笑嘻嘻地說:“妹妹大喜呀。”黛玉聽了這一句話,越發急了,也顧不得什麼了,把寶玉緊緊拉住說:“好,寶玉,我今日才知道你是個無情無義的人了。”寶玉道:“我怎麼無情無義?你既有了人家兒,咱們各自幹各自的了。”黛玉越聽越氣,越沒了主意,只得拉着寶玉哭道:“好哥哥,你叫我跟了誰去?”寶玉道:“你要不去,就在這裏住着。你原是許了我的,所以你纔到我們這裏來。我待你是怎麼樣的,你也想想。”黛玉恍惚又象果曾許過寶玉的,心內忽又轉悲作喜,問寶玉道:“我是死活打定主意的了。你到底叫我去不去?”寶玉道:“我說叫你住下。你不信我的話,你就瞧瞧我的心。”說着,就拿着一把小刀子往胸口上一劃,只見鮮血直流。黛玉嚇得魂飛魄散,忙用手握着寶玉的心窩,哭道:“你怎麼做出這個事來,你先來殺了我罷!”寶玉道:“不怕,我拿我的心給你瞧。”還把手在劃開的地方兒亂抓。黛玉又顫又哭,又怕人撞破,抱住寶玉痛哭。寶玉道:“不好了,我的心沒有了,活不得了。”說着,眼睛往上一翻,咕咚就倒了。黛玉拼命放聲大哭。只聽見紫鵑叫道:“姑娘,姑娘,怎麼魘住了?快醒醒兒脫了衣服睡罷。”黛玉一翻身,卻原來是一場惡夢。
\end{parag}


\begin{parag}
    喉間猶是哽咽,心上還是亂跳,枕頭上已經溼透,肩背身心,但覺冰冷。想了一回,“父親死得久了,與寶玉尚未放定,這是從那裏說起?”又想夢中光景,無倚無靠,再真把寶玉死了,那可怎麼樣好!一時痛定思痛,神魂俱亂。又哭了一回,遍身微微的出了一點兒汗,扎掙起來,把外罩大襖脫了,叫紫鵑蓋好了被窩,又躺下去。翻來覆去,那裏睡得着。只聽得外面淅淅颯颯,又象風聲,又象雨聲。又停了一會子,又聽得遠遠的吆呼聲兒,卻是紫鵑已在那裏睡着,鼻息出入之聲。自己扎掙着爬起來,圍着被坐了一會。覺得窗縫裏透進一縷涼風來,吹得寒毛直豎,便又躺下。正要朦朧睡去,聽得竹枝上不知有多少家雀兒的聲兒,啾啾唧唧,叫個不住。那窗上的紙,隔着屜子,漸漸的透進清光來。
\end{parag}


\begin{parag}
    黛玉此時已醒得雙眸炯炯,一回兒咳嗽起來,連紫鵑都咳嗽醒了。紫鵑道:“姑娘,你還沒睡着麼?又咳嗽起來了,想是着了風了。這會兒窗戶紙發清了,也待好亮起來了。歇歇兒罷,養養神,別盡着想長想短的了。”黛玉道:“我何嘗不要睡,只是睡不着。你睡你的罷。”說了又嗽起來。紫鵑見黛玉這般光景,心中也自傷感,睡不着了。聽見黛玉又嗽,連忙起來,捧着痰盒。這時天已亮了。黛玉道:“你不睡了麼?”紫鵑笑道:“天都亮了,還睡什麼呢。”黛玉道:“既這樣,你就把痰盒兒換了罷。”紫鵑答應着,忙出來換了一個痰盒兒,將手裏的這個盒兒放在桌上,開了套間門出來,仍舊帶上門,放下撒花軟簾,出來叫醒雪雁。開了屋門去倒那盒子時,只見滿盒子痰,痰中好些血星,唬了紫鵑一跳,不覺失聲道:“噯喲,這還了得!”黛玉里面接着問是什麼,紫鵑自知失言,連忙改說道:“手裏一滑,幾乎撂了痰盒子。”黛玉道:“不是盒子裏的痰有了什麼?”紫鵑道:“沒有什麼。”說着這句話時,心中一酸,那眼淚直流下來,聲兒早已岔了。黛玉因爲喉間有些甜腥,早自疑惑,方纔聽見紫鵑在外邊詫異,這會子又聽見紫鵑說話聲音帶着悲慘的光景,心中覺了八九分,便叫紫鵑:“進來罷,外頭看涼着。”紫鵑答應了一聲,這一聲更比頭裏悽慘,竟是鼻中酸楚之音。黛玉聽了,涼了半截。看紫鵑推門進來時,尚拿手帕拭眼。黛玉道:“大清早起,好好的爲什麼哭?”紫鵑勉強笑道:“誰哭來,早起起來眼睛裏有些不舒服。姑娘今夜大概比往常醒的時候更大罷,我聽見咳嗽了大半夜。”黛玉道:“可不是,越要睡,越睡不着。”紫鵑道:“姑娘身上不大好,依我說,還得自己開解着些。身子是根本,俗語說的,‘留得青山在,依舊有柴燒。’況這裏自老太太,太太起,那個不疼姑娘。”只這一句話,又勾起黛玉的夢來。覺得心頭一撞,眼中一黑,神色俱變,紫鵑連忙端着痰盒,雪雁捶着脊樑,半日才吐出一口痰來。痰中一縷紫血,簌簌亂跳。紫鵑雪雁臉都唬黃了。兩個旁邊守着,黛玉便昏昏躺下。紫鵑看着不好,連忙努嘴叫雪雁叫人去。
\end{parag}


\begin{parag}
    雪雁纔出屋門,只見翠縷翠墨兩個人笑嘻嘻的走來。翠縷便道:“林姑娘怎麼這早晚還不出門?我們姑娘和三姑娘都在四姑娘屋裏講究四姑娘畫的那張園子景兒呢。”雪雁連忙擺手兒,翠縷翠墨二人倒都嚇了一跳,說:“這是什麼原故?”雪雁將方纔的事,一一告訴他二人。二人都吐了吐舌頭兒說:“這可不是頑的!你們怎麼不告訴老太太去?這還了得!你們怎麼這麼糊塗。”雪雁道:“我這裏纔要去,你們就來了。”正說着,只聽紫鵑叫道:“誰在外頭說話?姑娘問呢。”三個人連忙一齊進來。翠縷翠墨見黛玉蓋着被躺在牀上,見了他二人便說道:“誰告訴你們了?你們這樣大驚小怪的。”翠墨道:“我們姑娘和雲姑娘才都在四姑娘屋裏講究四姑娘畫的那張園子圖兒,叫我們來請姑娘來,不知姑娘身上又欠安了。”黛玉道:“也不是什麼大病,不過覺得身子略軟些,躺躺兒就起來了。你們回去告訴三姑娘和雲姑娘,飯後若無事,倒是請他們來這裏坐坐罷。寶二爺沒到你們那邊去?”二人答道:“沒有。”翠墨又道:“寶二爺這兩天上了學了,老爺天天要查功課,那裏還能象從前那麼亂跑呢。”黛玉聽了,默然不言。二人又略站了一回,都悄悄的退出來了。
\end{parag}


\begin{parag}
    且說探春湘雲正在惜春那邊論評惜春所畫大觀園圖,說這個多一點,那個少一點,這個太疏,那個太密。大家又議着題詩,着人去請黛玉商議。正說着,忽見翠縷翠墨二人回來,神色匆忙。湘雲便先問道:“林姑娘怎麼不來?”翠縷道:“林姑娘昨日夜裏又犯了病了,咳嗽了一夜。我們聽見雪雁說,吐了一盒子痰血。”探春聽了詫異道:“這話真麼?”翠縷道:“怎麼不真。”翠墨道:“我們剛纔進去去瞧了瞧,顏色不成顏色,說話兒的氣力兒都微了。”湘雲道:“不好的這麼着,怎麼還能說話呢。”探春道:“怎麼你這麼糊塗,不能說話不是已經……”說到這裏卻嚥住了。惜春道:“林姐姐那樣一個聰明人,我看他總有些瞧不破,一點半點兒都要認起真來。天下事那裏有多少真的呢。”探春道:“既這麼着,咱們都過去看看。倘若病的利害,咱們好過去告訴大嫂子回老太太,傳大夫進來瞧瞧,也得個主意。”湘雲道:“正是這樣。”惜春道:“姐姐們先去,我回來再過去。”於是探春湘雲扶了小丫頭,都到瀟湘館來。進入房中,黛玉見他二人,不免又傷心起來。因又轉念想起夢中,連老太太尚且如此,何況他們。況且我不請他們,他們還不來呢。心裏雖是如此,臉上卻礙不過去,只得勉強令紫鵑扶起,口中讓坐。探春湘雲都坐在牀沿上,一頭一個。看了黛玉這般光景,也自傷感。探春便道:“姐姐怎麼身上又不舒服了?”黛玉道:“也沒什麼要緊,只是身子軟得很。”紫鵑在黛玉身後偷偷的用手指那痰盒兒。湘雲到底年輕,性情又兼直爽,伸手便把痰盒拿起來看。不看則已,看了唬的驚疑不止,說:“這是姐姐吐的?這還了得!”初時黛玉昏昏沉沉,吐了也沒細看,此時見湘雲這麼說,回頭看時,自己早已灰了一半。探春見湘雲冒失,連忙解說道:“這不過是肺火上炎,帶出一半點來,也是常事。偏是雲丫頭,不拘什麼,就這樣蠍蠍螫螫的!”湘雲紅了臉,自悔失言。探春見黛玉精神短少,似有煩倦之意,連忙起身說道:“姐姐靜靜的養養神罷,我們回來再瞧你。”黛玉道:“累你兩位惦着。”探春又囑咐紫鵑好生留神伏侍姑娘,紫鵑答應着。探春纔要走,只聽外面一個人嚷起來。未知是誰,下回分解。
\end{parag}