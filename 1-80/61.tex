\chap{六十一}{投鼠忌器寶玉情贓 判冤決獄平兒情權}


\begin{parag}
    那柳家的笑道:“好猴兒崽子,你親嬸子找野老兒去了,你豈不多得一個叔叔,有什麼疑的!別討我把你頭上的榪子蓋似的幾根屄毛撏下來!還不開門讓我進去呢。”這小廝且不開門,且拉著笑說:“好嬸子,你這一進去,好歹偷些杏子出來賞我喫。我這裏老等。你若忘了時,日後半夜三更打酒買油的,我不給你老人家開門,也不答應你,隨你幹叫去。”柳氏啐道:“發了昏的,今年不比往年,把這些東西都分給了衆奶奶了。一個個的不象抓破了臉的,人打樹底下一過,兩眼就象那黧雞似的,還動他的果子!昨兒我從李子樹下一走,偏有一個蜜蜂兒往臉上一過,我一招手兒,偏你那好舅母就看見了。他離的遠看不真,只當我摘李子呢,就屄聲浪嗓喊起來,說又是‘還沒供佛呢’,又是‘老太太、太太不在家還沒進鮮呢,等進了上頭,嫂子們都有分的’,倒象誰害了饞癆等李子出汗呢。叫我也沒好話說,搶白了他一頓。可是你舅母姨娘兩三個親戚都管著,怎不和他們要的,倒和我來要。這可是‘倉老鼠和老鴰去借糧——守著的沒有,飛著的有’。”小廝笑道:“哎喲喲,沒有罷了,說上這些閒話!我看你老以後就用不著我了?就便是姐姐有了好地方,將來更呼喚著的日子多,只要我們多答應他些就有了。”柳氏聽了,笑道: “你這個小猴精,又搗鬼弔白的,你姐姐有什麼好地方了?”那小廝笑道:“別哄我了,早已知道了。單是你們有內牽,難道我們就沒有內牽不成?我雖在這裏聽哈,裏頭卻也有兩個姊妹成個體統的,什麼事瞞了我們!”
\end{parag}


\begin{parag}
    正說著,只聽門內又有老婆子向外叫:“小猴兒們,快傳你柳嬸子去罷,再不來可就誤了。”柳家的聽了,不顧和小廝說話,忙推門進去,笑說:“不必忙,我來了。”一面來至廚房,──雖有幾個同伴的人,他們都不敢自專,單等他來調停分派──一面問衆人:“五丫頭那去了?”衆人都說:“才往茶房裏找他們姊妹去了。”
\end{parag}


\begin{parag}
    柳家的聽了,便將茯苓霜擱起,且按著房頭分派菜饌。忽見迎春房裏小丫頭蓮花兒走來\begin{note}庚雙夾:總是寫春景將殘。\end{note}說:“司棋姐姐說了,要碗雞蛋,燉的嫩嫩的。”柳家的道:“就是這樣尊貴。不知怎的,今年這雞蛋短的很,十個錢一個還找不出來。昨兒上頭給親戚家送粥米去,四五個買辦出去,好容易才湊了二千個來。我那裏找去?你說給他,改日喫罷。”蓮花兒道:“前兒要喫豆腐,你弄了些餿的,叫他說了我一頓。今兒要雞蛋又沒有了。什麼好東西,我就不信連雞蛋都沒有了,別叫我翻出來。”一面說,一面真個走來,揭起菜箱一看,只見裏面果有十來個雞蛋,說道:“這不是?你就這麼利害!喫的是主子的,我們的分例,你爲什麼心疼?又不是你下的蛋,怕人吃了。”柳家的忙丟了手裏的活計,便上來說道:“你少滿嘴裏混唚!你娘才下蛋呢!通共留下這幾個,預備菜上的澆頭。姑娘們不要,還不肯做上去呢,預備接急的。你們吃了,倘或一聲要起來,沒有好的,連雞蛋都沒了。你們深宅大院,水來伸手,飯來張口,只知雞蛋是平常物件,那裏知道外頭買賣的行市呢。別說這個,有一年連草根子還沒了的日子還有呢。我勸他們,細米白飯,每日肥雞大鴨子,將就些兒也罷了。喫膩了膈,天天又鬧起故事來了。雞蛋、豆腐,又是什麼麪筋、醬蘿蔔炸兒,敢自倒換口味。只是我又不是答應你們的,一處要一樣,就是十來樣。我倒別伺候頭層主子,只預備你們二層主子了。”蓮花聽了,便紅了臉,喊道:“誰天天要你什麼來?你說上這兩車子話!叫你來,不是爲便宜卻爲什麼。前兒小燕來,說晴雯姐姐要喫蘆蒿,你怎麼忙的還問肉炒雞炒?小燕說:‘葷的因不好才另叫你炒個麪筋的,少擱油纔好。’你忙的倒說自己發昏,趕著洗手炒了,狗顛兒似的親捧了去。今兒反倒拿我作筏子,說我給衆人聽。”柳家的忙道:“阿彌陀佛!這些人眼見的。別說前兒一次,就從舊年一立廚房以來,凡各房裏偶然間不論姑娘姐兒們要添一樣半樣,誰不是先拿了錢來,另買另添。有的沒的,名聲好聽,說我單管姑娘廚房省事,又有剩頭兒,算起帳來,惹人噁心:連姑娘帶姐兒們四五十人,一日也只管要兩隻雞,兩隻鴨子,十來斤肉,一吊錢的菜蔬。你們算算,夠作什麼的?連本項兩頓飯還撐持不住,還擱的住這個點這樣,那個點那樣,買來的又不喫,又買別的去。既這樣,不如回了太太,多添些分例,也象大廚房裏預備老太太的飯,把天下所有的菜蔬用水牌寫了,天天轉著喫,喫到一個月現算倒好。連前兒三姑娘和寶姑娘偶然商議了要喫個油鹽炒枸杞芽兒來,現打發個姐兒拿著五百錢來給我,我倒笑起來了,說:‘二位姑娘就是大肚子彌勒佛,也吃不了五百錢的去。這三二十個錢的事,還預備的起。’趕著我送回錢去,到底不收,說賞我打酒喫,又說:‘如今廚房在裏頭,保不住屋裏的人不去叨登,一鹽一醬,那不是錢買的。你不給又不好,給了你又沒的賠。你拿著這個錢,全當還了他們素日叨登的東西窩兒。’這就是明白體下的姑娘,我們心裏只替他念佛。沒的趙姨奶奶聽了又氣不忿,又說太便宜了我,隔不了十天,也打發個小丫頭子來尋這樣尋那樣,我倒好笑起來。你們竟成了例,不是這個,就是那個,我那裏有這些賠的。”
\end{parag}


\begin{parag}
    正亂時,只見司棋又打發人來催蓮花兒,說他:“死在這裏了,怎麼就不回去?”蓮花兒賭氣回來,便添了一篇話,告訴了司棋。司棋聽了,不免心頭起火。此刻伺候迎春飯罷,帶了小丫頭們走來,見了許多人正喫飯,見他來的勢頭不好,都忙起身陪笑讓坐。司棋便喝命小丫頭子動手,“凡箱櫃所有的菜蔬,只管丟出來餵狗,大家賺不成。”小丫頭子們巴不得一聲,七手八腳搶上去,一頓亂翻亂擲的。衆人一面拉勸,一面央告司棋說:“姑娘別誤聽了小孩子的話。柳嫂子有八個頭,也不敢得罪姑娘,說雞蛋難買是真。我們才也說他不知好歹,憑是什麼東西,也少不得變法兒去。他已經悟過來了,連忙蒸上了。姑娘不信瞧那火上。”
\end{parag}


\begin{parag}
    司棋被衆人一頓好言,方將氣勸的漸平。小丫頭們也沒得摔完東西,便拉開了。司棋連說帶罵,鬧了一回,方被衆人勸去。柳家的只好摔碗丟盤自己咕嘟了一回,蒸了一碗蛋令人送去。司棋全潑了地下了。那人回來也不敢說,恐又生事。
\end{parag}


\begin{parag}
    柳家的打發他女兒喝了一回湯,吃了半碗粥,又將茯苓霜一節說了。五兒聽罷,便心下要分些贈芳官,遂用紙另包了一半,趁黃昏人稀之時,自己花遮柳隱的來找芳官。且喜無人盤問。一徑到了怡紅院門前,不好進去,只在一簇玫瑰花前站立,遠遠的望著。有一盞茶時,可巧小燕出來,忙上前叫住。小燕不知是那一個,至跟前方看真切,因問作什麼。五兒笑道:“你叫出芳官來,我和他說話。”小燕悄笑道:“姐姐太性急了,橫豎等十來日就來了,只管找他做什麼。方纔使了他往前頭去了,你且等他一等。不然,有什麼話告訴我,等我告訴他。恐怕你等不得,只怕關園門了。”五兒便將茯苓霜遞與了小燕,又說這是茯苓霜,如何喫,如何補益,“我得了些送他的,轉煩你遞與他就是了。”說畢,作辭回來。
\end{parag}


\begin{parag}
    正走蓼漵一帶,忽見迎頭林之孝家的帶著幾個婆子走來,五兒藏躲不及,只得上來問好。林之孝家的問道:“我聽見你病了,怎麼跑到這裏來?”五兒陪笑道: “因這兩日好些,跟我媽進來散散悶。才因我媽使我到怡紅院送傢伙去。”林之孝家的說道:“這話岔了。方纔我見你媽出來我才關門。既是你媽使了你去,他如何不告訴我說你在這裏呢,竟出去讓我關門,是何主意?可知是你扯謊。”五兒聽了,沒話回答,只說:“原是我媽一早教我取去的,我忘了,捱到這時我纔想起來了。只怕我媽錯當我先出去了,所以沒和大娘說得。”
\end{parag}


\begin{parag}
    林之孝家的聽他辭鈍色虛,又因近日玉釧兒說那邊正房內失落了東西,幾個丫頭對賴,沒主兒,心下便起了疑。可巧小蟬、蓮花兒並幾個媳婦子走來,見了這事,便說道:“林奶奶倒要審審他。這兩日他往這裏頭跑的不象,鬼鬼唧唧的,不知幹些什麼事。”小蟬又道:“正是。昨兒玉釧姐姐說,太太耳房裏的櫃子開了,少了好些零碎東西。璉二奶奶打發平姑娘和玉釧姐姐要些玫瑰露,誰知也少了一罐子。若不是尋露,還不知道呢。”蓮花兒笑道:“這話我沒聽見,今兒我倒看見一個露瓶子。”林之孝家的正因這些事沒主兒,每日鳳姐使平兒催逼他,一聽此言,忙問在那裏。蓮花兒便說:“在他們廚房裏呢。”林之孝家的聽了,忙命打了燈籠,帶著衆人來尋。五兒急的便說:“那原是寶二爺屋裏的芳官給我的。”林之孝家的便說:“不管你方官圓官,現有了贓證,我只呈報了,憑你主子前辯去。”一面說,一面進入廚房,蓮花兒帶著,取出露瓶。恐還有偷的別物,又細細搜了一遍,又得了一包茯苓霜,一併拿了,帶了五兒,來回李紈與探春。
\end{parag}


\begin{parag}
    那時李紈正因蘭哥兒病了,不理事務,只命去見探春。探春已歸房。人回進去,丫鬟們都在院內納涼,探春在內盥沐,只有待書回進去。半日,出來說:“姑娘知道了,叫你們找平兒回二奶奶去。”林之孝家的只得領出來。到鳳姐兒那邊,先找著了平兒,平兒進去回了鳳姐。鳳姐方纔歇下,聽見此事,便吩咐:“將他娘打四十板子,攆出去,永不許進二門。把五兒打四十板子,立刻交給莊子上,或賣或配人。”平兒聽了,出來依言吩咐了林之孝家的。五兒唬的哭哭啼啼,給平兒跪著,細訴芳官之事。平兒道:“這也不難,等明日問了芳官便知真假。但這茯苓霜前日人送了來,還等老太太、太太回來看了纔敢打動,這不該偷了去。”五兒見問,忙又將他舅舅送的一節說了出來。平兒聽了,笑道:“這樣說,你竟是個平白無辜之人,拿你來頂缸。此時天晚,奶奶才進了藥歇下,不便爲這點子小事去絮叨。如今且將他交給上夜的人看守一夜,等明兒我回了奶奶,再做道理。”林之孝家的不敢違拗,只得帶了出來交與上夜的媳婦們看守,自便去了。
\end{parag}


\begin{parag}
    這裏五兒被人軟禁起來,一步不敢多走。又兼衆媳婦也有勸他說,不該做這沒行止之事;也有報怨說,正經更還坐不上來,又弄個賊來給我們看,倘或眼不見尋了死,逃走了,都是我們不是。於是又有素日一干與柳家不睦的人,見了這般,十分趁願,都來奚落嘲戲他。這五兒心內又氣又委屈,竟無處可訴;且本來怯弱有病,這一夜思茶無茶,思水無水,思睡無衾枕,嗚嗚咽咽直哭了一夜。
\end{parag}


\begin{parag}
    誰知和他母女不和的那些人,巴不得一時攆出他們去,惟恐次日有變,大家先起了個清早,都悄悄的來買轉平兒,一面送些東西,一面又奉承他辦事簡斷,一面又講述他母親素日許多不好。平兒一一的都應著,打發他們去了,卻悄悄的來訪襲人,問他可果真芳官給他露了。襲人便說:“露卻是給芳官,芳官轉給何人我卻不知。”襲人於是又問芳官,芳官聽了,唬天跳地,忙應是自己送他的。芳官便又告訴了寶玉,寶玉也慌了,說:“露雖有了,若勾起茯苓霜來,他自然也實供。若聽見了是他舅舅門上得的,他舅舅又有了不是,豈不是人家的好意,反被咱們陷害了。”因忙和平兒計議:“露的事雖完,然這霜也是有不是的。好姐姐,你叫他說也是芳官給他的就完了。”平兒笑道:“雖如此,只是他昨晚已經同人說是他舅舅給的了,如何又說你給的?況且那邊所丟的露也是無主兒,如今有贓證的白放了,又去找誰?誰還肯認?衆人也未必心服。”晴雯走來笑道:“太太那邊的露再無別人,分明是彩雲偷了給環哥兒去了。你們可瞎亂說。”平兒笑道:“誰不知是這個原故,但今玉釧兒急的哭,悄悄問著他,他應了,玉釧也罷了,大家也就混著不問了。難道我們好意兜攬這事不成!可恨彩雲不但不應,他還擠玉釧兒,說他偷了去了。兩個人窩裏發炮,先吵的閤府皆知,我們如何裝沒事人。少不得要查的。殊不知告失盜的就是賊,又沒贓證,怎麼說他。”寶玉道:“也罷,這件事我也應起來,就說是我唬他們頑的,悄悄的偷太太了的來了。兩件事都完了。”襲人道:“也倒是件陰騭事,保全人的賊名兒。只是太太聽見你又說你小孩子氣,不知好歹了。”平兒笑道:“這也倒是小事。如今便從趙姨娘屋裏起了贓來也容易,我只怕又傷著一個好人的體面。別人都別管,這一個人豈不又生氣。我可憐的是他,不肯爲了打老鼠傷了玉瓶。”說著,把三個指頭一伸。襲人等聽說,便知他說的是探春。大家都忙說:“可是這話。竟是我們這裏應了起來的爲是。”平兒又笑道:“也須得把彩雲和玉釧兒兩個業障叫了來,問準了他方好。不然他們得了益,不說爲這個,倒象我沒了本事問不出來,煩出這裏來完事,他們以後越發偷的偷,不管的不管了。”襲人等笑道:“正是,也要你留下地步。
\end{parag}


\begin{parag}
    平兒便命人叫了他兩個來,說道:“不用慌,賊已有了。”玉釧兒先問賊在那裏,平兒道:“現在二奶奶屋裏,你問他什麼應什麼。我心裏明知不是他偷的,可憐他害怕都承認。這裏寶二爺不過意,要替他認一半。我待要說出來,但只是這做賊的素日又是和我好的一個姊妹,窩主卻是平常,裏面又傷著一個好人的體面,因此爲難,少不得央求寶二爺應了,大家無事。如今反要問你們兩個,還是怎樣?若從此以後大家小心存體面,這便求寶二爺應了;若不然,我就回了二奶奶,別冤屈了好人。”彩雲聽了,不覺紅了臉,一時羞惡之心感發,便說道:“姐姐放心,也別冤了好人,也別帶累了無辜之人傷體面。偷東西原是趙姨奶奶央告我再三,我拿了些與環哥是情真。連太太在家我們還拿過,各人去送人,也是常事。我原說嚷過兩天就罷了。如今既冤屈了好人,我心也不忍。姐姐竟帶了我回奶奶去,我一概應了完事。”衆人聽了這話,一個個都詫異,他竟這樣有肝膽。寶玉忙笑道:“彩雲姐姐果然是個正經人。如今也不用你應,我只說是我悄悄的偷的唬你們頑,如今鬧出事來,我原該承認。只求姐姐們以後省些事,大家就好了。”彩雲道:“我乾的事爲什麼叫你應,死活我該去受。”平兒襲人忙道:“不是這樣說,你一應了,未免又叨登出趙姨奶奶來,那時三姑娘聽了,豈不生氣。竟不如寶二爺應了,大家無事,且除這幾個人皆不得知道這事,何等的乾淨。但只以後千萬大家小心些就是了。要拿什麼,好歹奈到太太到家,那怕連這房子給了人,我們就沒幹繫了。”彩雲聽了,低頭想了一想,方依允。
\end{parag}


\begin{parag}
    開是大家商議妥貼,平兒帶了他兩個並芳官往前邊來,至上夜房中叫了五錢,將茯苓霜一節也悄悄的教他說系芳官所贈,五兒感謝不盡。平兒帶他們來至自己這邊,已見林之孝家的帶領了幾個媳婦,押解著柳家的等夠多時。林之孝家的又向平兒說:“今兒一早押了他來,恐園裏沒人伺候姑娘們的飯,我暫且將秦顯的女人派了去伺候。姑娘一併回明奶奶,他倒乾淨謹慎,以後就派他常伺候罷。”平兒道:“秦顯的女人是誰?我不大相熟。”林之孝家的道:“他是園裏南角子上夜的白日裏沒什麼事,所以姑娘不大相識。高高孤拐,大大的眼睛,最乾淨爽利的。”玉釧兒道:“是了。姐姐,你怎麼忘了?他是跟二姑娘的司棋的嬸孃。司棋的父母雖是大老爺那邊的人,他這叔叔卻是咱們這邊的。”平兒聽了,方想起來,笑道:“哦,你早說是他,我就明白了。”又笑道:“也太派急了些。如今這事八下里水落石出了,連前兒太太屋裏丟的也有了主兒。是寶玉那日過來和這兩個業障要什麼的,偏這兩個業障慪他頑,說太太不在家不敢拿。寶玉便瞅他兩個不提防的時節,自己進去拿了些什麼出來。這兩個業障不知道,不唬慌了。如今寶玉聽見帶累了別人,方細細的告訴了我,拿出東西來我瞧,一件不差。那茯苓霜是寶玉外頭得了的,也曾賞過許多人,不獨園內人有,連媽媽子們討了出去給親戚們喫,又轉送人,襲人了曾給過芳官之流的人。他們私情各相來往,也是常事。前兒那兩簍還擺在議事廳上,好好的原封沒動,怎麼就混賴起人來。等我回了奶奶再說。”說畢,抽身進了臥房,將此事照前言回了鳳姐兒一遍。
\end{parag}


\begin{parag}
    鳳姐兒道:“雖如此說,但寶玉爲人不管青紅皁白愛兜攬事情。別人再求求他去,他又擱不住人兩句好話,給他個炭簍子戴上,什麼事他不應承。咱們若信了,將來若大事也如此,如何治人。還要細細的追求才是。依我的主意,把太太屋裏的丫頭都拿來,雖不便擅加拷打,只叫他們墊著磁瓦子跪在太陽地下,茶飯也別給喫。一日不說跪一日,便是鐵打的,一日也管招了。又道是‘蒼蠅不抱無縫的蛋’。雖然這柳家的沒偷,到底有些影兒,人才說他。雖不加賊刑,也革出不用。朝廷家原有掛誤的,倒也不算委屈了他。”平兒道:“何苦來操這心!‘得放手時須放手’,什麼大不了的事,樂得不施恩呢。依我說,縱在這屋裏操上一百分的心,終久咱們是那邊屋裏去的。沒的結些小人仇恨,使人含怨。況且自己又三災八難的,好容易懷了一個哥兒,到了六七個月還掉了,焉知不是素日操勞太過,氣惱傷著的。如今乘早兒見一半不見一半的,也倒罷了。”一席話,說的鳳姐兒倒笑了,說道:“憑你這小蹄子發放去罷。我才精爽些了,沒的淘氣。”平兒笑道:“這不是正經!”說畢,轉身出來,一一發放。要知端的,且聽下回分解。
\end{parag}
