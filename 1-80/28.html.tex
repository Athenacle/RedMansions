\chap{二十八}{蒋玉菡情赠茜香罗 薛宝钗羞笼红麝串}


\begin{parag}
    \begin{note}庚:茜香罗、红麝串写于一回,盖琪官虽系优人,后回与袭人供奉玉兄宝卿得同终始者,非泛泛之文也。自“闻曲”回以后,回回写药方,是白描颦儿添病也。\end{note}
\end{parag}


\begin{parag}
    话说林黛玉只因昨夜晴雯不开门一事,错疑在宝玉身上。至次日又可巧遇见饯花之期,正是一腔无明正未发泄,又勾起伤春愁思,因把些残花落瓣去掩埋,由不得感花伤己,哭了几声,便随口念了几句。不想宝玉在山坡上听见,先不过点头感叹;次后听到“侬今葬花人笑痴,他年葬侬知是谁”,“一朝春尽红颜老,花落人亡两不知”等句,不觉恸倒山坡之上,怀里兜的落花撒了一地。试想林黛玉的花颜月貌,将来亦到无可寻觅之时,宁不心碎肠断!既黛玉终归无可寻觅之时,推之于他人,如宝钗、香菱、袭人等,亦可到无可寻觅之时矣。宝钗等终归无可寻觅之时,则自己又安在哉?且自身尚不知何在何往,则斯处、斯园、斯花、斯柳,又不知当属谁姓矣!因此一而二,二而三,反复推求了去,\begin{note}庚侧:百转千回矣。\end{note}真不知此时此际欲为何等蠢物,杳无所知,逃大造,出尘网,使可解释这段悲伤。\begin{note}甲侧:非大善知识,说不出这句话来。\end{note}\begin{note}甲眉:不言炼句炼字辞藻工拙,只想景想情想事想理,反复推求悲伤感慨,乃玉兄一生之天性。真颦儿之知己,玉兄外实无一人。想昨阻批《葬花吟》之客,嫡是玉兄之化身无疑。余几作点金为铁之人,笨甚笨甚!\end{note}正是:花影不离身左右,鸟声只在耳东西。\begin{note}甲侧:二句作禅语参。\end{note}\begin{note}甲眉:一大篇《葬花吟》却如此收拾,真好机思笔仗,令人焉的不叫绝称奇!\end{note}
\end{parag}


\begin{parag}
    那林黛玉正自伤感,忽听山坡上也有悲声,心下想道:“人人都笑我有些痴病,难道还有一个痴子不成?”\begin{note}甲侧:岂敢岂敢。\end{note}想著,抬头一看,见是宝玉。林黛玉看见,便道:“啐!我道是谁,原来是这个狠心短命的……”刚说到“短命”二字,又把口掩住,\begin{note}甲侧:“情情”,不忍道出“的”字来。\end{note}长叹了一声,\begin{note}庚侧:不忍也。\end{note}自己抽身便走了。
\end{parag}


\begin{parag}
    这里宝玉悲恸了一回,忽然抬头不见了黛玉,便知黛玉看见他躲开了,自己也觉无味,抖抖土起来,下山寻归旧路,\begin{note}甲侧:折得好,誓不写开门见山文字。\end{note}往怡红院来。可巧\begin{note}庚侧:哄人字眼。\end{note}看见林黛玉在前头走,连忙赶上去,说道:“你且站住。我知你不理我,我只说一句话,从今后撂开手。”\begin{note}甲侧:非此三字难留莲步,玉兄之机变如此。\end{note}林黛玉回头看见是宝玉,待要不理他,听他说“只说一句话,从此撂开手”,这话里有文章,少不得站住说道: “有一句话,请说来。”宝玉笑道:“两句话,说了你听不听?”\begin{note}甲侧:相离尚远,用此句补空,好近阿颦。\end{note}黛玉听说,回头就走。\begin{note}庚侧:走得是。\end{note}宝玉在身后面叹道:“既有今日,何必当初!”\begin{note}甲侧:自言自语,真是一句话。\end{note}林黛玉听见这话,由不得站住,回头道:“当初怎么样?今日怎么样?”宝玉叹道:\begin{note}甲侧:以下乃答言,非一句话也。\end{note}“当初姑娘来了,那不是我陪著顽笑?\begin{note}甲侧:我阿颦之恼,玉兄实摸不著,不得不将自幼之苦心实事一诉,方可明心以白今日之故,勿作闲文看。\end{note}凭我心爱的,姑娘要,就拿去;我爱吃的,听见姑娘也爱吃,连忙干干净净收著等姑娘吃。一桌子吃饭,一床上睡觉。丫头们想不到的,我怕姑娘生气,我替丫头们想到了。我心里想著:姊妹们从小儿长大,亲也罢,热也罢,和气到了儿,才见得比人好。\begin{note}庚侧:要紧语。\end{note}如今谁承望姑娘人大心大,\begin{note}庚侧:反派不是。\end{note}不把我放在眼睛里,倒把外四路的什么宝姐姐\begin{note}庚侧:心事。\end{note}凤姐姐\begin{note}甲侧:用此人瞒看官也,瞒颦儿也。心动阿颦在此数句也。一节颇似说辞,玉兄口中却是衷肠话。\end{note}的放在心坎儿上,倒把我三日不理四日不见的。我又没个亲兄弟亲姊妹。──虽然有两个,你难道不知道是和我隔母的?我也和你似的独出,只怕同我的心一样。谁知我是白操了这个心,弄的有\begin{note}寃\end{note}无处诉!”说著不觉滴下眼泪来。\begin{note}甲侧:玉兄泪非容易有的。\end{note}
\end{parag}


\begin{parag}
    黛玉耳内听了这话,眼内见了这形景,心内不觉灰了大半,也不觉滴下泪来,低头不语。宝玉见他这般形景,遂又说道:“我也知道我如今不好了,但只凭著怎么不好,万不敢在妹妹跟前有错处。\begin{note}庚侧:有是语。\end{note}便有一二分错处,你倒是或教导我,戒我下次,\begin{note}庚侧:可怜语。\end{note}或骂我两句,打我两下,我都不灰心。谁知你总不理我,\begin{note}庚侧:实难为情。\end{note}叫我摸不著头脑,少魂失魄,不知怎么样才好。\begin{note}庚侧:真有是事。\end{note}就便死了,也是个屈死鬼,任凭高僧高道忏悔也不能超生,\begin{note}庚侧:又瞒看官及批书人。\end{note}还得你申明了缘故,我才得托生呢!”
\end{parag}


\begin{parag}
    黛玉听了这个话,不觉将昨晚的事都忘在九霄云外了,\begin{note}甲侧:“情情”本来面目也。\end{note}\begin{note}庚侧:“情情”衷肠。\end{note}便说道:“你既这么说,昨儿为什么我去了,你不叫丫头开门?”\begin{note}庚侧:正文,该问。\end{note}宝玉诧异道:“这话从那里说起?\begin{note}庚侧:实实不知。\end{note}我要是这么样,立刻就死了!”\begin{note}甲侧:急了。\end{note}林黛玉啐道:\begin{note}庚侧:如闻。\end{note}“大清早起死呀活的,也不忌讳。你说有呢就有,没有就没有,起什么誓呢。”宝玉道:“实在没有见你去。就是宝姐姐坐了一坐,\begin{note}庚侧:不要兄言,彼已亲睹。\end{note}就出来了。”林黛玉想了一想,笑道:“是了。想必是你的丫头们懒待动,丧声歪气的也是有的。”宝玉道:“想必是这个原故。等我回去问了是谁,教训教训他们就好了。”\begin{note}庚侧:玉兄口气毕真。\end{note}黛玉道:“你的那些姑娘们\begin{note}庚侧:不快活之称。\end{note}也该教训教训,\begin{note}庚侧:照样的妙!\end{note}只是我论理不该说。今儿得罪了我的事小,倘或明儿宝姑娘来,\begin{note}庚侧:也还一句,的是心坎上人。\end{note}什么贝姑娘来,也得罪了,事情岂不大了。”\begin{note}甲侧:至此心事全无矣。\end{note}说著抿著嘴笑。宝玉听了,又是咬牙,又是笑。
\end{parag}


\begin{parag}
    二人正说话,只见丫头来请吃饭,\begin{note}甲侧:收拾得干净。\end{note}遂都往前头来了。王夫人见了林黛玉,因问道:“大姑娘,你吃那鲍太医的药可好些?”\begin{note}庚侧:是新换了的口气。\end{note}林黛玉道:“也不过这么著。老太太还叫我吃王大夫的药呢。”\begin{note}庚侧:何如?\end{note}宝玉道:“太太不知道,林妹妹是内症,先天生的弱,所以禁不住一点风寒,不过吃两剂煎药就好了,散了风寒,还是吃丸药\begin{note}甲侧:引下文。\end{note}的好。”王夫人道:“前儿大夫说了个丸药的名字,我也忘了。”宝玉道:“我知道那些丸药,不过叫他吃什么人参养荣丸。”王夫人道:“不是。”宝玉又道:“八珍益母丸?左归?右归?再不,就是麦味地黄丸。”王夫人道:“都不是。我只记得有个‘金刚’两个字的。”\begin{note}甲侧:奇文奇语。\end{note}宝玉扎手笑道:\begin{note}甲侧:慈母前放肆了。\end{note}\begin{note}庚眉:此写玉兄,亦是释却心中一夜半日要事,故大大一泄。己冬夜。\end{note}“从来没听见有个什么‘金刚丸’。若有了‘金刚丸’,自然有‘菩萨散’了!”\begin{note}甲侧:宝玉因黛玉事完,一心无挂碍,故不知不觉手之舞之,足之蹈之。\end{note}说的满屋里人都笑了。宝钗抿嘴笑道:“想是天王补心丹。”\begin{note}甲侧:慧心人自应知之。\end{note}王夫人笑道:“是这个名儿。如今我也糊涂了。”宝玉道:“太太倒不糊涂,都是叫‘金刚’‘菩萨’支使糊涂了。”\begin{note}甲侧:是语甚对,余幼时所闻之语合符,哀哉伤哉!\end{note}王夫人道:“扯你娘的臊!又欠你老子捶你了。”\begin{note}庚侧:伏线。\end{note}宝玉笑道:“我老子再不为这个捶我的。”\begin{note}甲侧:此语亦不假。\end{note}
\end{parag}


\begin{parag}
    王夫人又道:“既有这个名儿,明儿就叫人买些来吃。”\begin{note}庚眉:写药案是暗度颦卿病势渐加之笔,非泛泛闲文也。丁亥夏。笏叟。\end{note}宝玉笑道:“这些都不中用的。太太给我三百六十两银子,我替妹妹配一料丸药,包管一料不完就好了。”王夫人道:“放屁!什么药就这么贵?”宝玉笑道:“当真的呢,我这个方子比别的不同。那个药名儿也古怪,一时也说不清。只讲那头胎紫河车,\begin{note}庚侧:只闻名。\end{note}人形带叶参,三百六十两还不够。龟大何首乌,\begin{note}庚侧:听也不曾听过。\end{note}千年松根茯苓胆,\begin{note}庚眉:写得不犯冷香丸方子。前“玉生香”回中颦云“他有金你有玉;他有冷香你岂不该有暖香?” 是宝玉无药可配矣。今颦儿之剂若许材料皆系滋补热性之药,兼有许多奇物,而尚未拟名,何不竟以“暖香”名之?以代补宝玉之不足,岂不三人一体矣。己冬夜。\end{note}诸如此类的药都不算为奇,\begin{note}庚侧:还有奇的。\end{note}只在群药里算。那为君的药,说起来唬人一跳。前儿薛大哥哥求了我一二年,我才给了他这方子。他拿了方子去又寻了二三年,花了有上千的银子,才配成了。太太不信,只问宝姐姐。”宝钗听说,笑著摇手儿说:“我不知道,也没听见。你别叫姨娘问我。”王夫人笑道:“到底是宝丫头,好孩子,不撒谎。”宝玉站在当地,听见如此说,一回身把手一拍,说道:“我说的倒是真话呢,倒说我撒谎。”口里说著,忽一回身,只见林黛玉坐在宝钗身后抿著嘴笑,用手指头在脸上画著羞他。\begin{note}庚侧:好看煞,在颦儿必有之。\end{note}
\end{parag}


\begin{parag}
    凤姐因在里间屋里看著人放桌子,\begin{note}庚侧:且不接宝玉文字,妙!\end{note}听如此说,便走来笑道:“宝兄弟不是撒谎,这倒是有的。上日薛大哥亲自和我来寻珍珠,我问他作什么,他说配药。他还抱怨说,不配也罢了,如今那里知道这么费事。我问他什么药,他说是宝兄弟的方子,说了多少药,我也没工夫听。他说不然我也买几颗珍珠了,只是定要头上带过的,所以来和我寻。他说:‘妹妹就没散的,花儿上也得,掐下来,过后儿我拣好的再给妹妹穿了来。’我没法儿,把两枝珠花儿现拆了给他。还要了一块三尺上用大红纱去,乳钵乳了隔面子呢。”凤姐说一句,那宝玉念一句佛,说:“太阳在屋子里呢!”凤姐说完了,宝玉又道:“太太想,这不过是将就呢。正经按那方子,这珍珠宝石定要在古坟里的,有那古时富贵人家装裹的头面,拿了来才好。如今那里为这个去刨坟掘墓,所以只是活人带过的,也可以使得。”王夫人道:“阿弥陀佛,不当家花花的!就是坟里有这个,人家死了几百年,这会子翻尸盗骨的,作了药也不灵!”\begin{note}甲侧:不止阿凤圆谎,今作者亦为圆谎了,看此数句则知矣。\end{note}
\end{parag}


\begin{parag}
    宝玉向林黛玉说道:“你听见了没有,难道二姐姐也跟著我撒谎不成?”脸望著黛玉说话,却拿眼睛瞟著宝钗。黛玉便拉王夫人道:“舅母听听,宝姐姐不替他圆谎,他支吾著我。”王夫人也道:“宝玉很会欺负你妹妹。”宝玉笑道:“太太不知道这原故。宝姐姐先在家里住著,那薛大哥哥的事,他也不知道,何况如今在里头住著呢,自然是越发不知道了。\begin{note}庚侧:分析得是,不敢正犯。\end{note}林妹妹才在背后羞我,打谅我撒谎呢。”
\end{parag}


\begin{parag}
    正说著,只见贾母房里的丫头找宝玉林黛玉去吃饭。林黛玉也不叫宝玉,便起身拉了那丫头就走。那丫头说等著宝玉一块儿走。林黛玉道:“他不吃饭了,咱们走。我先走了。”说著便出去了。宝玉道:“我今儿还跟著太太吃罢。”王夫人道:“罢,罢,我今儿吃斋,你正经吃你的去罢。”宝玉道:“我也跟著吃斋。”说著便叫那丫头“去罢”,自己先跑到桌子上坐了。王夫人向宝钗等笑道:“你们只管吃你们的,由他去罢。”宝钗因笑道:“你正经去罢。吃不吃,陪著林姑娘走一趟,他心里打紧的不自在呢。”宝玉道:“理他呢,过一会子就好了。”\begin{note}庚侧:后文方知。\end{note}
\end{parag}


\begin{parag}
    一时吃过饭,宝玉一则怕贾母记挂,二则也记挂著林黛玉,忙忙的要茶漱口。探春惜春都笑道:“二哥哥,你成日家忙些什么?\begin{note}甲侧:冷眼人自然了了。\end{note}吃饭吃茶也是这么忙碌碌的。”宝钗笑道:“你叫他快吃了瞧林妹妹去罢,叫他在这里胡羼些什么。”宝玉吃了茶,便出来,一直往西院来。可巧走到凤姐儿院门前,只见凤姐蹬著门槛子拿耳挖子剔牙,\begin{note}庚侧:也才吃了饭。\end{note}看著十来个小厮们挪花盆呢。\begin{note}庚侧:是阿凤身段。\end{note}见宝玉来了,笑道:“你来的好。进来,进来,替我写几个字儿。”宝玉只得跟了进来。到了屋里,凤姐命人取过笔砚纸来,向宝玉道:“大红妆缎四十匹,蟒缎四十匹,上用纱各色一百匹,金项圈四个。”宝玉道:“这算什么?又不是帐,又不是礼物,怎么个写法?”凤姐儿道:“你只管写上,横竖我自己明白就罢了。”\begin{note}庚侧:有是语,有是事。\end{note}宝玉听说只得写了。凤姐一面收起,一面笑道:“还有句话告诉你,不知你依不依?你屋里有个丫头叫红玉,我合你说说,要叫了来使唤,总也没说,今儿见你才想起来。”\begin{note}甲侧:字眼。\end{note}宝玉道:“我屋里的人也多的很,姐姐喜欢谁,只管叫了来,何必问我。”\begin{note}甲侧:红玉接杯倒茶,自纱屉内觅至回廊下,再见此处如些写来,可知玉兄除颦外,俱是行云流水。\end{note}凤姐笑道:“既这么著,我就叫人带他去了。”\begin{note}甲侧:又了却怡红一冤孽,一叹!\end{note}宝玉道:“只管带去。”说著便要走。\begin{note}甲侧:忙极!\end{note}凤姐儿道:“你回来,我还有一句话呢。”宝玉道:“老太太叫我呢,\begin{note}甲侧:非也,林妹妹叫我呢。一笑。\end{note}有话等我回来罢。”说著便来至贾母这边,只见都已吃完饭了。贾母因问他:“跟著你娘吃了什么好的?”宝玉笑道:“也没什么好的,我倒多吃了一碗饭。”\begin{note}甲侧:安慰祖母之心也。\end{note}因问:“林妹妹在那里?”\begin{note}甲侧:何如?余言不谬。\end{note}贾母道:“里头屋里呢。”
\end{parag}


\begin{parag}
    宝玉进来,只见地下一个丫头吹熨斗,炕上两个丫头打粉线,黛玉弯著腰拿著剪子裁什么呢。宝玉走进来笑道:“哦,这是作什么呢?才吃了饭,这么空著头,一会子又头疼了。”黛玉并不理,只管裁他的。有一个丫头说道:“那块绸子角儿还不好呢,再熨他一熨。”黛玉便把剪子一撂,说道:“理他呢,过一会子就好了。”\begin{note}甲侧:有意无意,暗合针对,无怪玉兄纳闷。\end{note}宝玉听了,只是纳闷。只见宝钗探春等也来了,和贾母说了一回话。宝钗也进来问:“林妹妹作什么呢?”因见林黛玉裁剪,因笑道:“妹妹越发能干了,连裁剪都会了。”黛玉笑道:“这也不过是撒谎哄人罢了。”宝钗笑道:“我告诉你个笑话儿,才刚为那个药,我说了个不知道,宝兄弟心里不受用了。”林黛玉道:“理他呢,过会子就好了。”\begin{note}甲眉:连重二次前言,是颦、宝气味暗合,勿认做有小人过言也。\end{note}宝玉向宝钗道:“老太太要抹骨牌,正没人呢,你抹骨牌去罢。”宝钗听说,便笑道:“我是为抹骨牌才来了?”说著便走了。林黛玉道:“你倒是去罢,这里有老虎,看吃了你!”说著又裁。宝玉见他不理,只得还陪笑说道:“你也出去逛逛再裁不迟。”林黛玉总不理。宝玉便问丫头们:“这是谁叫裁的?”林黛玉见问丫头们,便说道:“凭他谁叫我裁,也不管二爷的事!”宝玉方欲说话,只见有人进来回说“外头有人请”。宝玉听了,忙撤身出来。黛玉向外头说道:\begin{note}甲侧:仍丢不下,叹叹!\end{note}“阿弥陀佛!赶你回来,我死了也罢了。”\begin{note}甲侧:何苦来?余不忍听。\end{note}
\end{parag}


\begin{parag}
    宝玉出来,到外面,只见焙茗说道:“冯大爷家请。”宝玉听了,知道是昨日的话,便说:“要衣裳去。”自己便往书房里来。焙茗一直到了二门前等人,\begin{note}甲侧:此门请出玉兄来,故信步又至书房,文人弄墨,虚点缀也。\end{note}只见一个老婆子出来了,焙茗上去说道:“宝二爷在书房里等出门的衣裳,你老人家进去带个信儿。”那婆子说:“你妈的屄!\begin{note}庚侧:活现活跳。\end{note}倒好,宝二爷如今在园子里住著,\begin{note}甲侧:与夜间叫人对看。\end{note}跟他的人都在园子里,你又跑了这里来带信儿!”焙茗听了,笑道:“骂的是,我也糊涂了。”说著一径往东边二门前来。可巧门上小厮在甬路底下踢球,焙茗将原故说了。小厮跑了进去,半日抱了一个包袱出来,递与焙茗。回到书房里,宝玉换了,命人备马,只带著焙茗、锄药、双瑞、双寿四个小厮去了。
\end{parag}


\begin{parag}
    一径到了冯紫英家门口,有人报与了冯紫英,出来迎接进去。只见薛蟠早已在那里久候,还有许多唱曲儿的小厮并唱小旦的蒋玉菡、锦香院的妓女云儿。大家都见过了,然后吃茶。宝玉擎茶笑道:“前儿所言幸与不幸之事,我昼悬夜想,今日一闻呼唤即至。”冯紫英笑道:“你们令表兄弟倒都心实。前日不过是我的设辞,诚心请你们一饮,恐又推托,故说下这句话。\begin{note}甲眉:若真有一事,则不成《石头记》文字矣。作者的三昧在兹,批书人得书中三昧亦在兹。壬午孟夏。\end{note}今日一邀即至,谁知都信真了。”说毕大家一笑,然后摆上酒来,依次坐定。冯紫英先命唱曲儿的小厮过来让酒,然后命云儿也来敬。
\end{parag}


\begin{parag}
    那薛蟠三杯下肚,不觉忘了情,拉著云儿的手笑道:“你把那梯己新样儿的曲子唱个我听,我吃一坛如何?”云儿听说,只得拿起琵琶来,唱道:
\end{parag}


\begin{parag}
    两个寃家,都难丢下,想著你来又记挂著他。两个人形容俊俏,都难描画。想昨宵幽期私订在荼蘼架,一个偷情,一个寻拿,拿住了三曹对案,我也无回话。\begin{note}甲侧:此唱一曲为直刺宝玉。\end{note}
\end{parag}


\begin{parag}
    唱毕笑道:“你喝一坛子罢了。”薛蟠听说,笑道:“不值一坛,再唱好的来。”
\end{parag}


\begin{parag}
    宝玉笑道:“听我说来:如此滥饮,易醉而无味。我先喝一大海,\begin{note}庚眉:大海饮酒,西堂产九台灵芝日也,批书至此,宁不悲乎?壬午重阳日。\end{note}发一新令,有不遵者,连罚十大海,逐出席外与人斟酒。”\begin{note}甲侧:谁曾经过?叹叹!西堂故事。\end{note}冯紫英蒋玉菡等都道:“有理,有理。”宝玉拿起海来一气饮干,说道:“如今要说悲、愁、喜、乐四字,却要说出女儿来,还要注明这四字原故。说完了,饮门杯。酒面要唱一个新鲜时样曲子;酒底要席上生风一样东西,或古诗、旧对、《四书》、《五经》、成语。”薛蟠未等说完,先站起来拦道:“我不来,别算我。\begin{note}甲侧:爽人爽语。\end{note}这竟是捉弄我呢!”\begin{note}庚侧:岂敢?\end{note}云儿也站起来,推他坐下,笑道:“怕什么?这还亏你天天吃酒呢,难道你连我也不如!我回来还说呢。说是了,罢;不是了,不过罚上几杯,那里就醉死了。你如今一乱令,倒喝十大海,下去斟酒不成?”\begin{note}庚侧:有理。\end{note}众人都拍手道妙。薛蟠听说无法,只得坐了。听宝玉说道:
\end{parag}


\begin{poem}
    \begin{pl} 女儿悲,青春已大守空闺。\end{pl}

    \begin{pl} 女儿愁,悔教夫婿觅封侯。\end{pl}

    \begin{pl} 女儿喜,对镜晨妆颜色美。\end{pl}

    \begin{pl} 女儿乐,秋千架上春衫薄。\end{pl}
\end{poem}


\begin{parag}
    众人听了,都道:“说得有理。”薛蟠独扬著脸摇头说:“不好,该罚!”众人问:“如何该罚?”薛蟠道:“他说的我通不懂,怎么不该罚?”云儿便拧他一把,笑道:“你悄悄的想你的罢。回来说不出,又该罚了。”于是拿琵琶听宝玉唱道:
\end{parag}


\begin{poem}
    \begin{pl}滴不尽相思血泪抛红豆,\end{pl}

    \begin{pl}睡不稳纱窗风雨黄昏后,\end{pl}

    \begin{pl}忘不了新愁与旧愁,\end{pl}

    \begin{pl}咽不下玉粒金莼噎满喉,\end{pl}

    \begin{pl}照不见菱花镜里形容瘦。\end{pl}

    \begin{pl}展不开的眉头,挨不明的更漏。\end{pl}

    \begin{pl}呀!恰便似遮不住的青山隐隐,流不断的绿水悠悠。\end{pl}
\end{poem}


\begin{parag}
    唱完,大家齐声喝彩,独薛蟠说无板。宝玉饮了门杯,便拈起一片梨来,说道:“雨打梨花深闭门。”完了令。
\end{parag}


\begin{parag}
    下该冯紫英,说道:
\end{parag}


\begin{poem}
    \begin{pl}女儿悲,儿夫染病在垂危。\end{pl}

    \begin{pl}女儿愁,大风吹倒梳妆楼。\end{pl}

    \begin{pl}女儿喜,头胎养了双生子。\end{pl}

    \begin{pl}女儿乐,私向花园掏蟋蟀。\end{pl}\begin{note}甲侧:紫英口中应当如是。\end{note}
\end{poem}


\begin{parag}
    说毕,端起酒来,唱道:
\end{parag}


\begin{qute2sp}
    \begin{poem}
        \begin{pl}你是个可人,你是个多情,你是个刁钻古怪鬼灵精,你是个神仙也不灵。我说的话儿你全不信,只叫你去背地里细打听,才知道我疼你不疼!\end{pl}
    \end{poem}
\end{qute2sp}


\begin{parag}
    唱完,饮了门杯,说道:“鸡声茅店月。”令完,下该云儿。
\end{parag}


\begin{parag}
    云儿便说道:“女儿悲,将来终身指靠谁?”\begin{note}甲侧:道著了。\end{note}薛蟠叹道:“我的儿,有你薛大爷在,你怕什么!”众人都道:“别混他,别混他!”云儿又道:“女儿愁,妈妈打骂何时休!”薛蟠道:“前儿我见了你妈,还吩咐他不叫他打你呢。”众人都道:“再多言者罚酒十杯。”薛蟠连忙自己打了一个嘴巴子,说道:“没耳性,再不许说了。”云儿又道:“女儿喜,情郎不舍还家里。女儿乐,住了箫管弄弦索。”说完,便唱道:
\end{parag}

\begin{qute2sp}
    \begin{poem}
        \begin{pl}豆蔻开花三月三,一个虫儿往里钻。钻了半日不得进去,爬到花儿上打秋千。肉儿小心肝,我不开了你怎么钻?\end{pl}
        \begin{note}甲侧:双关,妙!\end{note}
    \end{poem}
\end{qute2sp}


\begin{parag}
    唱毕,饮了门杯,说道:“桃之夭夭。”令完了,下该薛蟠。
\end{parag}


\begin{parag}
    薛蟠道:“我可要说了:女儿悲──”说了半日,不见说底下的。冯紫英笑道:“悲什么?快说来。”薛蟠登时急的眼睛铃铛一般,瞪了半日,才说道:“女儿悲──”又咳嗽了两声,\begin{note}甲侧:受过此急者,大都不止呆兄一人耳。\end{note}说道:“女儿悲,嫁了个男人是乌龟。”众人听了都大笑起来。\begin{note}甲眉:此段与《金瓶梅》内西门庆、应伯爵在李桂姐家饮酒一回对看,未知孰家生动活泼?\end{note}薛蟠道:“笑什么,难道我说的不是?一个女儿嫁了汉子,要当忘八,他怎么不伤心呢?”众人笑的弯腰说道:“你说的很是,快说底下的。”薛蟠瞪了一瞪眼,又说道:“女儿愁──”说了这句,又不言语了。众人道:“怎么愁?”薛蟠道:“绣房撺出个大马猴。”众人呵呵笑道:“该罚,该罚!这句更不通,先还可恕。”\begin{note}甲侧:不愁,一笑。\end{note}说著便要筛酒。宝玉笑道:“押韵就好。”薛蟠道: “令官都准了,你们闹什么?”众人听说,方才罢了。云儿笑道:“下两句越发难说了,我替你说罢。”薛蟠道:“胡说!当真我就没好的了!听我说罢:女儿喜,洞房花烛朝慵起。”众人听了,都诧异道:“这句何其太韵?”薛蟠又道:“女儿乐,一根往里戳。”\begin{note}甲侧:有前韵句,故有是句。\end{note}众人听了,都扭著脸说道:“该死,该死该死,该死!快唱了罢。”薛蟠便唱道:“一个蚊子哼哼哼。”众人都怔了,说“这是个什么曲儿?”薛蟠还唱道:“两个苍蝇嗡嗡嗡。”众人都道:“罢,罢,罢!”薛蟠道:“爱听不听!这是新鲜曲儿,叫作哼哼韵。你们要懒待听,边酒底都免了,我就不唱。\begin{note}甲侧:何尝呆?\end{note}”众人都道:“免了罢,免了罢,倒别耽误了别人家。”
\end{parag}


\begin{parag}
    于是蒋玉菡说道:
\end{parag}


\begin{poem}
    \begin{pl}女儿悲,丈夫一去不回归。\end{pl}

    \begin{pl}女儿愁,无钱去打桂花油。\end{pl}

    \begin{pl}女儿喜,灯花\end{pl}\begin{note}甲侧:佳谶也。\end{note}\begin{pl}并头结双蕊。\end{pl}

    \begin{pl}女儿乐,夫唱妇随真和合。\end{pl}

\end{poem}


\begin{parag}
    说毕,唱道:
\end{parag}

\begin{qute2sp}
    \begin{poem}
        \begin{pl}可喜你天生成百媚娇,恰便似活神仙离碧霄。度青春,年正小;配鸾凤,真也著。呀!看天河正高,听谯楼鼓敲,剔银灯同入鸳帏悄。\end{pl}
    \end{poem}
\end{qute2sp}


\begin{parag}
    唱毕,饮了门杯,笑道:“这诗词上我倒有限。幸而昨日见了一副对子,可巧\begin{note}甲侧:真巧!\end{note}只记得这句,幸而席上还有这件东西。”\begin{note}甲侧:瞒过众人。\end{note}说毕,便干了酒,拿起一朵木樨来,念道:“花气袭人知昼暖。”
\end{parag}


\begin{parag}
    众人倒都依了,完令。薛蟠又跳了起来,喧嚷道:“了不得,了不得!该罚,该罚!这席上又没有宝贝,\begin{note}甲侧:奇谈。\end{note}你怎么念起宝贝来?”蒋玉菡怔了,说道:“何曾有宝贝?”薛蟠道:“你还赖呢!你再念来。”蒋玉菡只得又念了一遍。薛蟠道:“袭人可不是宝贝是什么!你们不信,只问他。”说毕,指著宝玉。宝玉没好意思起来,说:“薛大哥,你该罚多少?”薛蟠道:“该罚,该罚!”说著拿起酒来,一饮而尽。冯紫英与蒋玉菡等不知原故,云儿便告诉了出来。\begin{note}甲侧:用云儿细说,的是章法。\end{note}\begin{note}庚眉:云儿知怡红细事,可想玉兄之风情月意也。壬午重阳。\end{note}蒋玉菡忙起身陪罪。众人都道:“不知者不作罪。”
\end{parag}


\begin{parag}
    少刻,宝玉出席解手,蒋玉菡便随了出来。二人站在廊檐下,蒋玉菡又陪不是。宝玉见他妩媚温柔,心中十分留恋,便紧紧的搭著他的手,叫他:“闲了往我们那里去。还有一句话借问,也是你们贵班中,有一个叫琪官的,他在那里?如今名驰天下,我独无缘一见。”蒋玉菡笑道:“就是我的小名儿。”宝玉听说,不觉欣然跌足笑道:“有幸,有幸!果然名不虚传。今儿初会,便怎么样呢?”想了一想,向袖中取出扇子,将一个玉诀扇坠解下来,递与琪官,道:“微物不堪,略表今日之谊。”琪官接了,笑道:“无功受禄,何以克当!也罢,我这里得了一件奇物,今日早起方系上,还是簇新的,聊可表我一点亲热之意。”说毕撩衣,将系小衣儿一条大红汗巾子解了下来,递与宝玉,道:“这汗巾子是茜香国女国王所贡之物,夏天系著,肌肤生香,不生汗渍。昨日北静王给我的,今日才上身。若是别人,我断不肯相赠。二爷请把自己系的解下来,给我系著。”宝玉听说,喜不自禁,连忙接了,将自己一条松花汗巾解了下来,递与琪官。\begin{note}甲侧:红绿牵巾是这样用法。一笑。\end{note}二人方束好,只见一声大叫:“我可拿住了!”只见薛蟠跳了出来,拉著二人道:“放著酒不吃,两个人逃席出来干什么?快拿出来我瞧瞧。”二人都道:“没有什么。”薛蟠那里肯依,还是冯紫英出来才解开了。于是复又归坐饮酒,至晚方散。
\end{parag}


\begin{parag}
    宝玉回至园中,宽衣吃茶。袭人见扇子上的坠儿没了,便问他:“往那里去了?”宝玉道:“马上丢了。”\begin{note}庚侧:随口谎言。\end{note}睡觉时只见腰里一条血点似的大红汗巾子,袭人便猜了八九分,因说道:“你有了好的系裤子,把我那条还我罢。”宝玉听说,方想起那条汗巾子原是袭人的,不该给人才是,心里后悔,口里说不出来,只得笑道:“我赔你一条罢。”袭人听了,点头叹道:“我就知道又干这些事!也不该拿著我的东西给那起混帐人去。也难为你,心里没个算计儿。” 再要说几句,又恐怄上他的酒来,少不得也睡了,一宿无话。
\end{parag}


\begin{parag}
    至次日天明,方才醒了,只见宝玉笑道:“夜里失了盗也不晓得,你瞧瞧裤子上。”袭人低头一看,只见昨日宝玉系的那条汗巾子系在自己腰里呢,便知是宝玉夜间换了,忙一顿把解下来,说道:“我不希罕这行子,趁早儿拿了去!”宝玉见他如此,只得委婉解劝了一回。袭人无法,只得系在腰里。过后宝玉出去,终久解下来掷在个空箱子里,自己又换了一条系著。
\end{parag}


\begin{parag}
    宝玉并未理论,因问起昨日可有什么事情。袭人便回说:“二奶奶打发人叫了红玉去了。他原要等你来的,我想什么要紧,我就作了主,打发他去了。”宝玉道:“很是。我已知道了,不必等我罢了。”袭人又道:“昨儿贵妃打发夏太监出来,送了一百二十两银子,叫在清虚观初一到初三打三天平安醮,唱戏献供,叫珍大爷领著众位爷们跪香拜佛呢。还有端午儿的节礼也赏了。”说著命小丫头子来,将昨日所赐之物取了出来,只见上等宫扇两柄,红麝香珠二串,凤尾罗二端,芙蓉簟一领。宝玉见了,喜不自胜,问“别人的也都是这个?”袭人道:“老太太的多著一个香如意,一个玛瑙枕。太太、老爷、姨太太的只多著一个如意。你的同宝姑娘的一样。\begin{note}甲侧:金姑玉郎是这样写法。\end{note}林姑娘同二姑娘、三姑娘、四姑娘只单有扇子同数珠儿,别人都没了。大奶奶、二奶奶他两个是每人两匹纱,两匹罗,两个香袋,两个锭子药。”宝玉听了,笑道:“这是怎么个原故?怎么林姑娘的倒不同我的一样,倒是宝姐姐的同我一样!别是传错了罢?”袭人道:“昨儿拿出来,都是一份一份的写著签子,怎么就错了!你的是在老太太屋里的,我去拿了来了。老太太说了,明儿叫你一个五更天进去谢恩呢。”宝玉道:“自然要走一趟。”说著便叫紫绡来:“拿了这个到林姑娘那里去,就说是昨儿我得的,爱什么留下什么。”紫绡答应了,拿了去,不一时回来说:“林姑娘说了,昨儿也得了,二爷留著罢。”
\end{parag}


\begin{parag}
    宝玉听说,便命人收了。刚洗了脸出来,要往贾母那里请安去,只见林黛玉顶头来了。宝玉赶上去笑道:“我的东西叫你拣,你怎么不拣?”林黛玉昨日所恼宝玉的心事早又丢开,又顾今日的事了,因说道:“我没这么大福禁受,比不得宝姑娘,什么金什么玉的,我们不过是草木之人!”\begin{note}甲侧:自道本是绛珠草也。\end{note}宝玉听他提出“金玉”二字来,不觉心动疑猜,便说道:“除了别人说什么金什么玉,我心里要有这个想头,天诛地灭,万世不得人身!”林黛玉听他这话,便知他心里动了疑,忙又笑道:“好没意思,白白的说什么誓?管你什么金什么玉的呢!”宝玉道:“我心里的事也难对你说,日后自然明白。除了老太太、老爷、太太这三个人,第四个就是妹妹了。要有第五个人,我也说个誓。”林黛玉道:“你也不用说誓,我很知道你心里有‘妹妹’,但只是见了‘姐姐’,就把‘妹妹’ 忘了。” 宝玉道:“那是你多心,我再不的。”林黛玉道:“昨儿宝丫头不替你圆谎,为什么问著我呢?那要是我,你又不知怎么样了。”
\end{parag}


\begin{parag}
    正说著,只见宝钗从那边来了,二人便走开了。宝钗分明看见,只装看不见,低著头过去了,到了王夫人那里,坐了一回,然后到了贾母这边,只见宝玉在这里呢。\begin{note}甲侧:宝钗往王夫人处去,故宝玉先在贾母处,一丝不乱。\end{note}薛宝钗因往日母亲对王夫人等曾提过“金锁是个和尚给的,等日后有玉的方可结为婚姻”等语,\begin{note}甲侧:此处表明以后二宝文章,宜换眼看。\end{note}所以总远著宝玉。\begin{note}甲眉:峰峦全露,又用烟云截断,好文字。\end{note}昨儿见元春所赐的东西,独他与宝玉一样,心里越发没意思起来。幸亏宝玉被一个林黛玉缠绵住了,心心念念只记挂著林黛玉,并不理论这事。此刻忽见宝玉笑问道:“宝姐姐,我瞧瞧你的红麝串子?”可巧宝钗左腕上笼著一串,见宝玉问他,少不得褪了下来。宝钗生的肌肤丰泽,容易褪不下来。宝玉在旁看著雪白一段酥臂,不觉动了羡慕之心,暗暗想道: “这个膀子要长在林妹妹身上,或者还得摸一摸,偏生长在他身上。”正是恨没福得摸,忽然想起“金玉”一事来,再看看宝钗形容,只见脸若银盆,眼似水杏,唇不点而红,眉不画而翠,\begin{note}甲侧:太白所谓“清水出芙蓉”。\end{note}比林黛玉另具一种妩媚风流,不觉就呆了,\begin{note}甲侧:忘情,非呆也。\end{note}宝钗褪了串子来递与他也忘了接。宝钗见他怔了,自己倒不好意思的,丢下串子,回身才要走,只见林黛玉蹬著门槛子,嘴里咬著手帕子笑呢。宝钗道:“你又禁不得风吹,怎么又站在那风口里?”林黛玉笑道:“何曾不是在屋里的。只因听见天上一声叫唤,出来瞧了瞧,原来是个呆雁。”薛宝钗道:“呆雁在那里呢?我也瞧一瞧。”林黛玉道: “我才出来,他就‘忒儿’一声飞了。”口里说著,将手里的帕子一甩,向宝玉脸上甩来。宝玉不防,正打在眼上,“嗳哟”了一声。要知端的,且听下回分解。
\end{parag}


\begin{parag}
    \begin{note}甲:茜香罗、红麝串写于一回,盖琪官虽系优人,后回与袭人供奉玉兄宝卿得同终始者,非泛泛之文也。自“闻曲”回以后,回回写药方,是白描颦儿添病也。前“玉生香”回中颦云“他有金你有玉;他有冷香你岂不该有暖香?”是宝玉无药可配矣。今颦儿之剂若许材料皆系滋补热性之药,兼有许多奇物,而尚未拟名,何不竟以“暖香”名之?以代补宝玉之不足,岂不三人一体矣。宝玉忘情,露于宝钗,是后回累累忘情之引。茜香罗暗系于袭人腰中,系伏线之文。\end{note}
\end{parag}


\begin{parag}
    \begin{note}蒙回后总评:世间最苦是疑情,不遇知音休应声。盟誓已成了,莫迟误今生。\end{note}
\end{parag}
