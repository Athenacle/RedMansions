\chap{三十八}{林潇湘魁夺菊花诗 薛蘅芜讽和螃蟹咏}
\begin{parag}
    \begin{note}蒙回前总批:美人用别号,亦新奇花样,且韵且雅,呼去觉满口生香。起社出自探春意,作者已伏下兴利除弊之文也。此回从放笔写诗写词作札,看他诗复诗,词复词,札复札,总不相犯。\end{note}
\end{parag}


\begin{parag}
    \begin{note}蒙回前总批:湘云,诗客也,前回写之其今才起社,后用不即不离闲人数语数折,仍归社中。何巧活之笔如此?\end{note}
\end{parag}


\begin{parag}
    \begin{note}庚辰:题曰“菊花诗”、“螃蟹咏”,伪自太君前阿凤若许诙谐中不失体、鸳鸯平儿宠婢中多少放肆之迎合取乐写来,似难入题,却轻轻用弄水戏鱼之看花等游玩事及王夫人云“这里风大”一句收住入题,并无纤毫牵强,此重作轻抹法也。妙极!好看煞!\end{note}
\end{parag}


\begin{parag}
    话说宝钗湘云二人计议已妥,一宿无话。湘云次日便请贾母等赏桂花。贾母等都说道:“是他有兴头,须要扰他这雅兴。”\begin{note}庚辰双行夹批:若在世俗小家,则云:“你是客,在我们舍下,怎么反扰你的呢?”一何可笑。\end{note}至午,果然贾母带了王夫人凤姐兼请薛姨妈等进园来。贾母因问:“那一处好?”\begin{note}庚辰双行夹批:必如此问方好。\end{note}王夫人道:“凭老太太爱在哪一处,就在哪一处。”\begin{note}庚辰双行夹批:必是王夫人如此答方妙。\end{note}凤姐道:“藕香榭已经摆下了,那山坡下两颗桂花开的又好,河里的水又碧清,坐在河当中亭子上岂不敞亮,看著水眼也清亮。”\begin{note}庚辰双行夹批:智者乐水,岂具然乎?\end{note}贾母听了,说:“这话很是。”说著,就引了众人往藕香榭来。原来这藕香榭盖在池中,四面有窗,左右有曲廊可通,亦是跨水接岸,后面又有曲折竹桥暗接。众人上了竹桥,凤姐忙上来搀著贾母,口里说:“老祖宗只管迈大步走,不相干的,这竹子桥规矩 咯吱咯喳的。”\begin{note}庚辰双行夹批:如见其势,如临其上,非走过者形容不到。\end{note}
\end{parag}


\begin{parag}
    一时进入榭中,只见栏杆外另放著两张竹案,一个上面设著杯箸酒具,一个上头设著茶筅茶盂各色茶具。那边有两三个丫头煽风炉煮茶,这一边另外几个丫头也煽风炉烫酒呢。贾母喜的忙问:“这茶想的到,且是地方,东西都干净。”湘云笑道:“这是宝姐姐帮著我预备的。”贾母道:“我说这个孩子细致,凡事想的妥当。”一面说,一面又看见柱上挂的黑漆嵌蚌的对子,命人念。湘云念道:
\end{parag}


\begin{poem}
    \begin{pl}芙蓉影破归兰桨,菱藕香深写竹桥。\end{pl}
    \begin{note}庚辰双行夹批:妙极!此处忽又补出一处不入贾政“试才”一回,皆错综其事,不作一直笔也。\end{note}
\end{poem}

\begin{parag}
    贾母听了,又抬头看匾,因回头向薛姨妈道:“我先小时,家里也有这么一个亭子,叫做什么‘枕霞阁’。我那时也只象他们这么大年纪,同姊妹们天天顽去。那日谁知我失了脚掉下去,几乎没淹死,好容易救了上来,到底被那木钉把头碰破了。如今这鬓角上那指头顶大一块窝儿就是那残破了。众人都怕经了水,又怕冒了风,都说活不得了,谁知竟好了。”凤姐不等人说,先笑道:“那时要活不得,如今这大福可叫谁享呢!可知老祖宗从小儿的福寿就不小,神差鬼使碰出那个窝儿来,好盛福寿的。寿星老儿头上原是一个窝儿,因为万福万寿盛满了,所以倒凸高出些来了。”未及说完,贾母与众人都笑软了。\begin{note}庚辰双行夹批:看他忽用贾母数语,闲闲又补出此书之前似已有一部《十二钗》的一般,令人遥忆不能一见,余则将欲补出 枕霞阁中十二钗来,岂不又添一部新书?\end{note}贾母笑道:“这猴儿惯的了不得了,只管拿我取笑起来,恨的我撕你那油嘴。”凤姐笑道:“回来吃螃蟹,恐积了冷在心里,讨老祖宗笑一笑开开心,一高兴多吃两个就无妨了。”贾母笑道:“明儿叫你日夜跟著我,我倒常笑笑觉的开心,不许回家去。”王夫人笑道:“老太太因为喜欢他,才惯的他这样,还这样说,他明儿越发无礼了。”贾母笑道:“我喜欢他这样,况且他又不是那不知高低的孩子。家常没人,娘儿们原该这样。横竖礼体不错就罢,没的倒叫他从神儿似的作什么。”\begin{note}庚辰双行夹批:近之暴发专讲理法竟不知礼法,此似无礼而礼法井井,所谓“整瓶不动半瓶摇”,又曰“习惯成自然”,真不谬也。\end{note}
\end{parag}


\begin{parag}
    说著,一齐进入亭子,献过茶,凤姐忙著搭桌子,要杯箸。上面一桌,贾母、薛姨妈、宝钗、黛玉、宝玉;东边一桌,史湘云、王夫人、迎、探、惜;西边靠门一桌,李纨和凤姐的,虚设坐位,二人皆不敢坐,只在贾母王夫人两桌上伺候。凤姐吩咐:“螃蟹不可多拿来,仍旧放在蒸笼里,拿十个来,吃了再拿。”一面又要水洗了手,站在贾母跟前剥蟹肉,头次让薛姨妈。薛姨妈道:“我自己掰著吃香甜,不用人让。”凤姐便奉与贾母。二次的便与宝玉,又说:“把酒烫的滚热的拿来。”又命小丫头们去取了菊花叶儿桂花蕊熏的绿豆面子来,预备著洗手。史湘云陪著吃了一个,就下座来让人,又出至外头,令人盛两盘子与赵姨娘周姨娘送去。又见凤姐走来道:“你不惯张罗,你吃你的去。我先替你张罗,等散了我再吃。”湘云不肯,又令人在那边廊上摆了两桌,让鸳鸯、琥珀、彩霞、彩云、平儿去坐。鸳鸯因向凤姐笑道:“二奶奶在这里伺候,我们可吃去了。”凤姐儿道:“你们只管去,都交给我就是了。”说著,史湘云仍入了席。凤姐和李纨也胡乱应个景儿。凤姐仍是下来张罗,一时出至廊上,鸳鸯等正吃的高兴,见他来了,鸳鸯等站起来道:“奶奶又出来作什么?让我们也受用一会子。”凤姐笑道:“鸳鸯小蹄子越发坏了,我替你当差,倒不领情,还抱怨我。还不快斟一钟酒来我喝呢。”鸳鸯笑著忙斟了一杯酒,送至凤姐唇边,凤姐一扬脖子吃了。平儿早剔了一壳黄子送来,凤姐道:“多倒些姜醋。”一面也吃了,笑道:“你们坐著吃罢,我可去了。”鸳鸯笑道:“好没脸,吃我们的东西。”凤姐儿笑道:“你和我少作怪。你知道你琏二爷爱上了你,要和老太太讨了你做小老婆呢。”鸳鸯道:“啐,这也是作奶奶说出来的话!我不拿腥手抹你一脸算不得。”说著赶来就要抹。凤姐儿央道:“好姐姐,饶我这一遭儿罢。”琥珀笑道:“鸳丫头要去了,平丫头还饶他?你们看看他,没有吃了两个螃蟹,倒喝了一碟子醋,他也算不会揽酸了。”平儿手里正掰了个满黄的螃蟹,听如此奚落他,便拿著螃蟹照著琥珀脸上抹来,口内笑骂“我把你这嚼舌根的小蹄子!”琥珀也笑著往旁边一躲,平儿使空了,往前一撞,正恰恰的抹在凤姐儿腮上。凤姐儿正和鸳鸯嘲笑,不防唬了一跳,嗳哟了一声。众人撑不住都哈哈的大笑起来。凤姐也禁不住笑骂道:“死娼妇!吃离了眼了,混抹你娘的。” 平儿忙赶过来替他擦了,亲自去端水。鸳鸯道:“阿弥陀佛!这是个报应。”贾母那边听见,一叠声问:“见了什么这样乐,告诉我们也笑笑。”鸳鸯等忙高声笑回道:“二奶奶来抢螃蟹吃,平儿恼了,抹了他主子一脸的螃蟹黄子。主子奴才打架呢。”贾母和王夫人等听了也笑起来。贾母笑道:“你们看他可怜见的,把那小腿子脐子给他点子吃也就完了。”鸳鸯等笑著答应了,高声又说道:“这满桌子的腿子,二奶奶只管吃就是了。”凤姐洗了脸走来,又伏侍贾母等吃了一回。黛玉独不敢多吃,只吃了一点儿夹子肉就下来了。
\end{parag}


\begin{parag}
    贾母一时不吃了,大家方散,都洗了手,也有看花的,也有弄水看鱼的,游玩了一回。王夫人因回贾母说:“这里风大,才又吃了螃蟹,老太太还是回房去歇歇罢了。若高兴,明日再来逛逛。”贾母听了,笑道:“正是呢。我怕你们高兴,我走了又怕扫了你们的兴。既这么说,咱们就都去吧。”回头又龈湘云:“别让你宝哥哥林姐姐多吃了。”湘云答应著。又嘱咐湘云宝钗二人说:“你两个也别多吃。那东西虽好吃,不是什么好的,吃多了肚子疼。”二人忙应著送出园外,仍旧回来,令将残席收拾了另摆。宝玉道:“也不用摆,咱们且作诗。把那大团圆桌就放在当中,酒菜都放著。也不必拘定坐位,有爱吃的大家去吃,散坐岂不便宜。”宝钗道:“这话极是。”湘云道:“虽如此说,还有别人。”因又命另摆一桌,拣了热螃蟹来,请袭人、紫鹃、司棋、侍书、入画、莺儿、翠墨等一处共坐。山坡桂树底下铺下两条花毡,命答应的婆子并小丫头等也都坐了,只管随意吃喝,等使唤再来。
\end{parag}


\begin{parag}
    湘云便取了诗题,用针绾在墙上。众人看了,都说:“新奇固新奇,只怕作不出来。”湘云又把不限韵的原故说了一番。宝玉道:“这才是正理,我也最不喜限韵。”林黛玉因不大吃酒,又不吃螃蟹,自令人掇了一个绣墩倚栏坐著,拿著钓竿钓鱼。宝钗手里拿著一枝桂花玩了一回,俯在窗槛上掐了桂蕊掷向水面,引的游鱼浮上来唼喋。湘云出一回神,又让一回袭人等,又招呼山坡下的众人只管放量吃。探春和李纨惜春立在垂柳阴中看鸥鹭。迎春又独在花阴下拿著花针穿茉莉花。\begin{note}庚辰双行夹批:看他各人各式,亦如画家有孤耸独出则有攒三聚五,疏疏密密,直是一幅《百美图》。\end{note}宝玉又看了一回黛玉钓鱼,一回又俯在宝钗旁边说笑两句,一回又看袭人等吃螃蟹,自己也陪他饮两口酒。袭人又剥一壳肉给他吃。黛玉放下钓竿,走至座间,拿起那乌银梅花自斟壶来,\begin{note}庚辰双行夹批:写壶非写壶,正写黛玉。\end{note}拣了一个小小的海棠冻石蕉叶杯。\begin{note}庚辰双行夹批:妙杯!非写杯,正写黛玉。“拣”字有神理,盖黛玉不善饮,此任性也。\end{note}丫鬟看见,知他要饮酒,忙著走上来斟。黛玉道:“你们只管吃去,让我自斟,这才有趣儿。”说著便斟了半盏,看时却是黄酒,因说道:“我吃了一点子螃蟹,觉得心口微微的疼,须得热热的喝口烧酒。”宝玉忙道:“有烧酒。”便令将那合欢花浸的酒烫一壶来。\begin{note}庚辰双行夹批:伤哉!作者犹记矮□舫前以合欢花酿酒乎?屈指二十年矣。\end{note}黛玉也只吃了一口便放下了。宝钗也走过来,另拿了一只杯来,也饮了一口,便蘸笔至墙上把头一个《忆菊》勾了,底下又赘了一个“蘅”字。\begin{note}庚辰双行夹批:妙极韵极!\end{note}宝玉忙道:“好姐姐,第二个我已经有了四句了,你让我作罢。”宝钗笑道:“我好容易有了一首,你就忙的这样。”黛玉也不说话,接过笔来把第八个《问菊》勾了,接著把第十一个《菊梦》也勾了,也赘一个“潇”字。\begin{note}庚辰双行夹批:这两个妙题料定黛玉必喜,岂让人作去哉?\end{note}宝玉也拿起笔来,将第二个《访菊》也勾了,也赘上一个“绛”字。探春走来看看道:“竟没有人作《簪菊》,让我作这《簪菊》。”又指著宝玉笑道:“才宣过总不许带出闺阁字样来,你可要留神。”说著,只见史湘云走来,将第四第五《对菊》《供菊》一连两个都勾了,也赘上一个“湘”字。探春道:“你也该起个号。”湘云笑道:“我们家里如今虽有几处轩馆,我又不住著,借了来也没趣。”\begin{note}庚辰双行夹批:今之不读书暴发户偏爱起一别号。一笑。\end{note}宝钗笑道:“方才老太太说,你们家也有这个水亭叫‘枕霞阁’,难道不是你的。如今虽没了,你到底是旧主人。”众人都道有理,宝玉不待湘云动手,便代将“湘”字抹了,改了一个“霞”字。又有顿饭工夫,十二题已全,各自誊出来,都交与迎春,另拿了一张雪浪笺过来,一并誊录出来,某人作的底下赘明某人的号。李纨等从头看起:
\end{parag}
\begin{poem}

    \begin{pl}

        忆菊 蘅芜君    \end{pl}
    \begin{note}庚辰双行夹批:真用此号,妙极!\end{note}

    \begin{pl}

        怅望西风抱闷思,蓼红苇白断肠时。
    \end{pl}
    \begin{pl}

        空篱旧圃秋无迹,瘦月清霜梦有知。
    \end{pl}
    \begin{pl}

        念念心随归雁远,寥寥坐听晚砧痴。
    \end{pl}
    \begin{pl}

        谁怜为我黄花病,慰语重阳会有期。
    \end{pl}

    \emptypl

    \begin{pl}

        访菊 怡红公子
    \end{pl}
    \begin{pl}

        闲趁霜晴试一游,酒杯药盏莫淹留。
    \end{pl}
    \begin{pl}

        霜前月下谁家种,槛外篱边何处秋。
    \end{pl}
    \begin{pl}

        蜡屐远来情得得,冷吟不尽兴悠悠。
    \end{pl}
    \begin{pl}

        黄花若解怜诗客,休负今朝挂杖头。
    \end{pl}

    \emptypl
    \begin{pl}

        种菊 怡红公子
    \end{pl}
    \begin{pl}

        携锄秋圃自移来,篱畔庭前故故栽。
    \end{pl}
    \begin{pl}

        昨夜不期经雨活,今朝犹喜带霜开。
    \end{pl}
    \begin{pl}

        冷吟秋色诗千首,醉酹寒香酒一杯。
    \end{pl}
    \begin{pl}

        泉溉泥封勤护惜,好知井迳绝尘埃。
    \end{pl}

    \emptypl
    \begin{pl}

        对菊 枕霞旧友
    \end{pl}
    \begin{pl}

        别圃移来贵比金,一丛浅淡一丛深。
    \end{pl}
    \begin{pl}

        萧疏篱畔科头坐,清冷香中抱膝吟。
    \end{pl}
    \begin{pl}

        数去更无君傲世,看来惟有我知音。
    \end{pl}
    \begin{pl}

        秋光荏苒休辜负,相对原宜惜寸阴。
    \end{pl}

    \emptypl
    \begin{pl}

        供菊 枕霞旧友
    \end{pl}
    \begin{pl}

        弹琴酌酒喜堪俦,几案婷婷点缀幽。
    \end{pl}
    \begin{pl}

        隔座香分三径露,抛书人对一枝秋。
    \end{pl}
    \begin{pl}

        霜清纸帐来新梦,圃冷斜阳忆旧游。
    \end{pl}
    \begin{pl}

        傲世也因同气味,春风桃李未淹留。
    \end{pl}
    \emptypl
    \begin{pl}

        咏菊 潇湘妃子
    \end{pl}
    \begin{pl}

        无赖诗魔昏晓侵,绕篱欹石自沉音。
    \end{pl}
    \begin{pl}

        毫端蕴秀临霜写,口齿噙香对月吟。
    \end{pl}
    \begin{pl}

        满纸自怜题素怨,片言谁解诉秋心。
    \end{pl}
    \begin{pl}

        一从陶令平章后,千古高风说到今。
    \end{pl}

    \emptypl
    \begin{pl}

        画菊 蘅芜君
    \end{pl}
    \begin{pl}

        诗余戏笔不知狂,岂是丹青费较量。
    \end{pl}
    \begin{pl}

        聚叶泼成千点墨,攒花染出几霜痕。
    \end{pl}
    \begin{pl}

        淡浓神会风前影,跳脱秋生腕底香。
    \end{pl}
    \begin{pl}

        莫认东篱闲采掇,粘屏聊以慰重阳。
    \end{pl}

    \emptypl
    \begin{pl}

        问菊 潇湘妃子
    \end{pl}
    \begin{pl}

        欲讯秋情众莫知,喃喃负手叩东篱。
    \end{pl}
    \begin{pl}

        孤标傲世偕谁隐,一样花开为底迟?
    \end{pl}
    \begin{pl}

        圃露庭霜何寂寞,雁归蛩病可相思?
    \end{pl}
    \begin{pl}

        休言举世无谈者,解语何妨片语时。
    \end{pl}

    \emptypl
    \begin{pl}

        簪菊 蕉下客
    \end{pl}
    \begin{pl}

        瓶供篱栽日日忙,折来休认镜中妆。
    \end{pl}
    \begin{pl}

        长安公子因花癖,彭泽先生是酒狂。
    \end{pl}
    \begin{pl}

        短鬓冷沾三径露,葛巾香染九秋霜。
    \end{pl}
    \begin{pl}

        高情不入时人眼,拍手凭他笑路旁。
    \end{pl}

    \emptypl
    \begin{pl}

        菊影 枕霞旧友
    \end{pl}
    \begin{pl}

        秋光叠叠复重重,潜度偷移三径中。
    \end{pl}
    \begin{pl}

        窗隔疏灯描远近,篱筛破月锁玲珑。
    \end{pl}
    \begin{pl}

        寒芳留照魂应驻,霜印传神梦也空。
    \end{pl}
    \begin{pl}

        珍重暗香休踏碎,凭谁醉眼认朦胧。
    \end{pl}

    \emptypl
    \begin{pl}

        菊梦 潇湘妃子
    \end{pl}
    \begin{pl}

        篱畔秋酣一觉清,和云伴月不分明。
    \end{pl}
    \begin{pl}

        登仙非慕庄生蝶,忆旧还寻陶令盟。
    \end{pl}
    \begin{pl}

        睡去依依随雁断,惊回故故恼蛩鸣。
    \end{pl}
    \begin{pl}

        醒时幽怨同谁诉,衰草寒烟无限情。
    \end{pl}

    \emptypl
    \begin{pl}

        残菊 蕉下客
    \end{pl}
    \begin{pl}

        露凝霜重渐倾欹,宴赏才过小雪时。
    \end{pl}
    \begin{pl}

        蒂有余香金淡泊,枝无全叶翠离披。
    \end{pl}
    \begin{pl}

        半床落月蛩声病,万里寒云雁阵迟。
    \end{pl}
    \begin{pl}

        明岁秋风知再会,暂时分手莫相思。
    \end{pl}
\end{poem}

\begin{parag}
    众人看一首,赞一首,彼此称扬不已。李纨笑道:“等我从公评来。通篇看来,各有各人的警句。今日公评:《咏菊》第一,《问菊》第二,《菊梦》第三,题目新,诗也新,立意更新,恼不得要推潇湘妃子为魁了;然后《簪菊》《对菊》《供菊》《画菊》《忆菊》次之。”宝玉听说,喜的拍手叫“极是,极公道。”黛玉道:“我那首也不好,到底伤于纤巧些。”李纨道:“巧的却好,不露堆砌生硬。”黛玉道:“据我看来,头一句好的是‘圃冷斜阳忆旧游’,这句背面傅粉。‘抛书人对一枝秋’已经妙绝,将供菊说完,没处再说,故翻回来想到未折未供之先,意思深透。”李纨笑道:“固如此说,你的‘口齿噙香’句也敌的过了。”探春又道:“到底要算蘅芜君沉著,‘秋无迹’、‘梦有知’,把个忆字竟烘染出来了。”宝钗笑道:“你的‘短鬓冷沾’、‘葛巾香染’,也就把簪菊形容的一个缝儿也没了。”湘云道:“‘偕谁隐’、‘为底迟’,真个把个菊花问的无言可对。”李纨笑道:“你的‘科头坐’、‘抱膝吟’,竟一时也不能别开,菊花有知,也必腻烦了。”说的大家都笑了。宝玉笑道:“我又落第。难道‘谁家种’、‘何处秋’、‘蜡屐远来’、‘冷吟不尽’,都不是访,‘昨夜雨’、‘今朝霜’,都不是种不成?但恨敌不上‘口齿噙香对月吟’、‘清冷香中抱膝吟’、‘短鬓’、‘葛巾’、‘金淡泊’、‘翠离披’、‘秋无迹’、‘梦有知’这几句罢了。”\begin{note}庚辰双行夹批:总写宝玉不及,妙极!\end{note}又道:“明儿闲了,我一个人作出十二首来。”李纨道:“你的也好,只是不及这几句新巧就是了。”
\end{parag}


\begin{parag}
    大家又评了一回,复又要了热蟹来,就在大圆桌子上吃了一回。宝玉笑道:“今日持螯赏桂,亦不可无诗。\begin{note}庚辰双行夹批:全是他忙,全是他不及。妙极!\end{note}我已吟成,谁还敢作呢?”说著,便忙洗了手提笔写出。\begin{note}庚辰双行夹批:且莫看诗,只看他偏于如许一大回诗后又写一回诗,岂世人想得到的?\end{note}众人看道:
\end{parag}


\begin{poem}
    \begin{pl}

        持螯更喜桂阴凉,泼醋擂姜兴欲狂。
    \end{pl}
    \begin{pl}

        饕餮王孙应有酒,横行公子却无肠。
    \end{pl}
    \begin{pl}

        脐间积冷馋忘忌,指上沾腥洗尚香。
    \end{pl}
    \begin{pl}

        原为世人美口腹,坡仙曾笑一生忙。
    \end{pl}

\end{poem}

\begin{parag}
    黛玉笑道:“这样的诗,要一百首也有。”\begin{note}庚辰双行夹批:看他这一说。\end{note}宝玉笑道:“你这会子才力已尽,不说不能作了,还贬人家。”黛玉听了,并不答言,也不思索,提起笔来一挥,已有了一首。众人看道:
\end{parag}


\begin{poem}
    \begin{pl}

        铁甲长戈死未忘,堆盘色相喜先尝。
    \end{pl}
    \begin{pl}

        螯封嫩玉双双满,壳凸红脂块块香。
    \end{pl}
    \begin{pl}

        多肉更怜卿八足,助情谁劝我千觞。
    \end{pl}
    \begin{pl}

        对斯佳品酬佳节,桂拂清风菊带霜。
    \end{pl}

\end{poem}

\begin{parag}
    宝玉看了正喝彩,黛玉便一把撕了,令人烧去,因笑道:“我的不及你的,我烧了他。你那个很好,比方才的菊花诗还好,你留著他给人看。”宝钗接著笑道:“我也勉强了一首,未必好,写出来取笑儿罢。”说著也写了出来。大家看时,写道是:
\end{parag}


\begin{poem}
    \begin{pl}

        桂霭桐阴坐举觞,长安涎口盼重阳。
    \end{pl}
    \begin{pl}

        眼前道路无经纬,皮里春秋空黑黄。
    \end{pl}
\end{poem}

\begin{parag}
    看到这里,众人不禁叫绝。宝玉道:“写得痛快!我的诗也该烧了。”又看底下道:
\end{parag}


\begin{poem}
    \begin{pl}

        酒未敌腥还用菊,性防积冷定须姜。
    \end{pl}
    \begin{pl}

        于今落釜成何益,月浦空余禾黍香。
    \end{pl}
\end{poem}

\begin{parag}
    众人看毕,都说这是食螃蟹绝唱,这些小题目,原要寓大意才算是大才,只是讽刺世人太毒了些。说著,只见平儿复进园来。不知作什么,且听下回分解。
\end{parag}


\begin{parag}
    \begin{note}蒙回末总批:请看此回中,闺中女儿能作此等豪情韵事,且笔下各能自画其性情,毫不乖舛。作者之锦绣口,无庸赘续。其用意之深,奖励之勤,都此文者亦不得轻忽戒之也。\end{note}
\end{parag}
