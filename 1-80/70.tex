\chap{七十}{林黛玉重建桃花社 史湘雲偶填柳絮詞}


\begin{parag}
    \begin{note}蒙回前總評:空將佛事圖相報,以觸飄風散豔花。一片精神傳好句,題成讖語任呼嗟!\end{note}
\end{parag}


\begin{parag}
    話說賈璉自在梨香院伴宿七日夜,天天僧道不斷做佛事。賈母喚了他去,吩咐不許送往家廟中。賈璉無法,只得又和時覺說了,就在尤三姐之上點了一個穴,破土埋葬。那日送殯,只不過族中人與王信夫婦,尤氏婆媳而已。鳳姐一應不管,只憑他自去辦理。因又年近歲逼,諸務蝟集不算外,又有林之孝開了一個人名單子來,共有八個二十五歲的單身小廝應該娶妻成房,等裏面有該放的丫頭們好求指配。鳳姐看了,先來問賈母和王夫人。大家商議,雖有幾個應該發配的,奈各人皆有原故:第一個鴛鴦發誓不去。自那日之後,一向未和寶玉說話,也不盛妝濃飾。衆人見他志堅,也不好相強。第二個琥珀,又有病,這次不能了。彩雲因近日和賈環分崩,也染了無醫之症。只有鳳姐兒和李紈房中粗使的大丫鬟出去了,其餘年紀未足。令他們外頭自娶去了。
\end{parag}


\begin{parag}
    原來這一向因鳳姐病了,李紈探春料理家務不得閒暇,接著過年過節,出來許多雜事,竟將詩社擱起。如今仲春天氣,雖得了工夫,爭奈寶玉因冷遁了柳湘蓮,劍刎了尤小妹,金逝了尤二姐,氣病了柳五兒,連連接接,閒愁胡恨,一重不了一重添。弄得情色若癡,語言常亂,似染怔忡之疾。慌的襲人等又不敢回賈母,只百般逗他頑笑。
\end{parag}


\begin{parag}
    這日清晨方醒,只聽外間房內咭咭呱呱笑聲不斷。襲人因笑說:“你快出去解救,晴雯和麝月兩個人按住溫都裏那膈肢呢。”寶玉聽了,忙披上灰鼠襖子出來一瞧,只見他三人被褥尚未疊起,大衣也未穿。那晴雯只穿蔥綠院綢小襖,紅小衣紅睡鞋,披著頭髮,騎在雄奴身上。麝月是紅綾抹胸,披著一身舊衣,在那裏抓雄奴的肋肢。雄奴卻仰在炕上,穿著撒花緊身兒,紅褲綠襪,兩腳亂蹬,笑的喘不過氣來。寶玉忙上前笑說:“兩個大的欺負一個小的,等我助力。”說著,也上牀來膈肢晴雯。晴雯觸癢,笑的忙丟下雄奴,和寶玉對抓。雄奴趁勢又將晴雯按倒,向他肋下抓動。襲人笑說:“仔細凍著了。”看他四人裹在一處倒好笑。
\end{parag}


\begin{parag}
    忽有李紈打發碧月來說:“昨兒晚上奶奶在這裏把塊手帕子忘了,不知可在這裏?”小燕說:“有,有,有,我在地下拾了起來,不知是那一位的,才洗了出來晾著,還未乾呢。”碧月見他四人亂滾,因笑道:“倒是這裏熱鬧,大清早起就咭咭呱呱的頑到一處。”寶玉笑道:“你們那裏人也不少,怎麼不頑?”碧月道: “我們奶奶不頑,把兩個姨娘和琴姑娘也賓住了。如今琴姑娘又跟了老太太前頭去了,更寂寞了。兩個姨娘今年過了,到明年冬天都去了,又更寂寞呢。你瞧寶姑娘那裏,出去了一個香菱,就冷清了多少,把個雲姑娘落了單。”
\end{parag}


\begin{parag}
    正說著,只見湘雲又打發了翠縷來說:“請二爺快出去瞧好詩。”寶玉聽了,忙問:“那裏的好詩?”翠縷笑道:“姑娘們都在沁芳亭上,你去了便知。”寶玉聽了,忙梳洗了出來,果見黛玉,寶釵,湘雲,寶琴,探春都在那裏,手裏拿著一篇詩看。見他來時,都笑說:“這會子還不起來,咱們的詩社散了一年,也沒有人作興。如今正是初春時節,萬物更新,正該鼓舞另立起來纔好。”湘雲笑道:“一起詩社時是秋天,就不應發達。如今卻好萬物逢春,皆主生盛。況這首桃花詩又好,就把海棠社改作桃花社。”\begin{note}庚雙夾:起時是後有名,此是先有名。\end{note}寶玉聽著,點頭說:“很好。”且忙著要詩看。衆人都又說:“咱們此時就訪稻香老農去,大家議定好起的。”說著,一齊起來,都往稻香村來。寶玉一壁走,一壁看那紙上寫著《桃花行》一篇,曰:
\end{parag}


\begin{poem}
    \begin{pl}桃花簾外東風軟,桃花簾內晨妝懶。\end{pl}

    \begin{pl}簾外桃花簾內人,人與桃花隔不遠。\end{pl}

    \begin{pl}東風有意揭簾櫳,花欲窺人簾不卷。\end{pl}

    \begin{pl}桃花簾外開仍舊,簾中人比桃花瘦。\end{pl}

    \begin{pl}花解憐人花也愁,隔簾消息風吹透。\end{pl}

    \begin{pl}風透湘簾花滿庭,庭前春色倍傷情。\end{pl}

    \begin{pl}閒苔院落門空掩,斜日欄杆人自憑。\end{pl}

    \begin{pl}憑欄人向東風泣,茜裙偷傍桃花立。\end{pl}

    \begin{pl}桃花桃葉亂紛紛,花綻新紅葉凝碧。\end{pl}

    \begin{pl}霧裹煙封一萬株,烘樓照壁紅模糊。\end{pl}

    \begin{pl}天機燒破鴛鴦錦,春酣欲醒移珊枕。\end{pl}

    \begin{pl}侍女金盆進水來,香泉影蘸胭脂冷。\end{pl}

    \begin{pl}胭脂鮮豔何相類,花之顏色人之淚;\end{pl}

    \begin{pl}若將人淚比桃花,淚自長流花自媚。\end{pl}

    \begin{pl}淚眼觀花淚易幹,淚乾春盡花憔悴。\end{pl}

    \begin{pl}憔悴花遮憔悴人,花飛人倦易黃昏。\end{pl}

    \begin{pl}一聲杜宇春歸盡,寂寞簾櫳空月痕。\end{pl}


\end{poem}


\begin{parag}
    寶玉看了並不稱讚,卻滾下淚來。便知出自黛玉,因此落下淚來,又怕衆人看見,又忙自己擦了。因問:“你們怎麼得來?”寶琴笑道:“你猜是誰做的?”寶玉笑道:“自然是瀟湘子稿。”寶琴笑道:“現是我作的呢。”寶玉笑道:“我不信。這聲調口氣,迥乎不像蘅蕪之體,所以不信。”寶釵笑道:“所以你不通。難道杜工部首首隻作‘叢菊兩開他日淚’之句不成!一般的也有‘紅綻雨肥梅’‘水荇牽風翠帶長’之媚語。”寶玉笑道:“固然如此說。但我知道姐姐斷不許妹妹有此傷悼語句,妹妹雖有此才,是斷不肯作的。比不得林妹妹曾經離喪,作此哀音。”衆人聽說,都笑了。
\end{parag}


\begin{parag}
    已至稻香村中,將詩與李紈看了,自不必說稱賞不已。說起詩社,大家議定:“明日乃三月初二日,就起社,便改'海棠社'爲'桃花社',林黛玉就爲社主。明日飯後,齊集瀟湘館”。因又大家擬題,黛玉便說:“大家就要桃花詩一百韻。”寶釵道:“使不得。從來桃花詩最多,縱作了必落套,比不得你這一首古風。須得再擬。”正說著,人回:“舅太太來了。姑娘出去請安。”因此大家都往前頭來見王子騰的夫人,陪著說話。喫飯畢,又陪入園中來,各處遊頑一遍。至晚飯後掌燈方去。
\end{parag}


\begin{parag}
    次日乃是探春的壽日,元春早打發了兩個小太監送了幾件頑器。閤家皆有壽儀,自不必說。飯後,探春換了禮服,各處行禮。黛玉笑向衆人道:“我這一社開的又不巧了,偏忘了這兩日是他的生日。雖不擺酒唱戲的,少不得都要陪他在老太太,太太跟前頑笑一日,如何能得閒空兒。”因此改至初五。
\end{parag}


\begin{parag}
    這日衆姊妹皆在房中侍早膳畢,便有賈政書信到了。寶玉請安,將請賈母的安稟拆開念與賈母聽,上面不過是請安的話,說六月中準進京等語。其餘家信事務之帖,自有賈璉和王夫人開讀。衆人聽說六七月回京,都喜之不盡。偏生近日王子騰之女許與保寧侯之子爲妻,擇日於五月初十日過門,鳳姐兒又忙著張羅,常三五日不在家。這日王子騰的夫人又來接鳳姐兒,一併請衆甥男甥女閒樂一日。賈母和王夫人命寶玉,探春,林黛玉,寶釵四人同鳳姐去。衆人不敢違拗,只得回房去另妝飾了起來。五人作辭,去了一日,掌燈方回。寶玉進入怡紅院,歇了半刻,襲人便乘機見景勸他收一收心,閒時把書理一理預備著。寶玉屈指算一算說:“還早呢。”襲人道:“書是第一件,字是第二件。到那時你縱有了書,你的字寫的在那裏呢?”寶玉笑道:“我時常也有寫的好些,難道都沒收著?”襲人道:“何曾沒收著。你昨兒不在家,我就拿出來共算,數了一數,纔有五六十篇。這三四年的工夫,難道只有這幾張字不成。依我說,從明日起,把別的心全收了起來,天天快臨幾張字補上。雖不能按日都有,也要大概看得過去。”寶玉聽了,忙的自己又親檢了一遍,實在搪塞不去,便說:“明日爲始,一天寫一百字纔好。”說話時大家安下。至次日起來梳洗了,便在窗下研墨,恭楷臨帖。賈母因不見他,只當病了,忙使人來問。寶玉方去請安,便說寫字之故,先將早起清晨的工夫盡了出來,再作別的,因此出來遲了。賈母聽了,便十分歡喜,吩咐他:“以後只管寫字唸書,不用出來也使得。你去回你太太知道。”寶玉聽說,便往王夫人房中來說明。王夫人便說:“臨陣磨槍,也不中用。有這會子著急,天天寫寫念念,有多少完不了的。這一趕,又趕出病來才罷。”寶玉回說不妨事。這裏賈母也說怕急出病來。探春寶釵等都笑說:“老太太不用急。書雖替他不得,字卻替得的。我們每人每日臨一篇給他,搪塞過這一步就完了。一則老爺到家不生氣,二則他也急不出病來。”賈母聽說,喜之不盡。
\end{parag}


\begin{parag}
    原來林黛玉聞得賈政回家,必問寶玉的功課,寶玉肯分心,恐臨期吃了虧。因此自己只裝作不耐煩,把詩社便不起,也不以外事去勾引他。探春寶釵二人每日也臨一篇楷書字與寶玉,寶玉自己每日也加工,或寫二百三百不拘。至三月下旬,便將字又集湊出許多來。這日正算,再得五十篇,也就混的過了。誰知紫鵑走來,送了一卷東西與寶玉,拆開看時,卻是一色老油竹紙上臨的鐘王蠅頭小楷,字跡且與自己十分相似。喜的寶玉和紫鵑作了一個揖,又親自來道謝。史湘雲寶琴二人亦皆臨了幾篇相送。湊成雖不足功課,亦足搪塞了。寶玉放了心,於是將所應讀之書,又溫理過幾遍。正是天天用功,可巧近海一帶海嘯,又遭蹋了幾處生民。地方官題本奏聞,奉旨就著賈政順路查看賬濟回來。如此算去,至冬底方回。寶玉聽了,便把書字又擱過一邊,仍是照舊遊蕩。
\end{parag}


\begin{parag}
    時值暮春之際,史湘雲無聊,因見柳花飄舞,便偶成一小令,調寄《如夢令》,其詞曰:
\end{parag}


\begin{poem}
    \begin{pl}豈是繡絨殘吐,\end{pl}

    \begin{pl}捲起半簾香霧,\end{pl}

    \begin{pl}纖手自拈來,\end{pl}

    \begin{pl}空使鵑啼燕妒。\end{pl}

    \begin{pl}且住,且住,\end{pl}

    \begin{pl}莫使春光別去。\end{pl}


\end{poem}


\begin{parag}
    自己作了,心中得意,便用一條紙兒寫好,與寶釵看了,又來找黛玉。黛玉看畢,笑道:“好,也新鮮有趣。我卻不能。”湘雲笑道:“咱們這幾社總沒有填詞。你明日何不起社填詞,改個樣兒,豈不新鮮些。”黛玉聽了,偶然興動,便說:“這話說的極是。我如今便請他們去。”說著,一面吩咐預備了幾色果點之類,一面就打發人分頭去請衆人。這裏他二人便擬了柳絮之題,又限出幾個調來,寫了綰在壁上。
\end{parag}


\begin{parag}
    衆人來看時,以柳絮爲題,限各色小調。又都看了史湘雲的,稱賞了一回。寶玉笑道:“這詞上我們平常,少不得也要胡謅起來。”於是大家拈鬮,寶釵便拈得了《臨江仙》,寶琴拈得《西江月》,探春拈得了《南柯子》,黛玉拈得了《唐多令》,寶玉拈得了《蝶戀花》。紫鵑炷了一支夢甜香,\begin{note}庚雙夾:重建,故又寫香。\end{note}大家思索起來。一時黛玉有了,寫完。接著寶琴寶釵都有了。他三人寫完,互相看時,寶釵便笑道:“我先瞧完了你們的,再看我的。”探春笑道:“噯呀,今兒這香怎麼這樣快,已剩了三分了。我纔有了半首。”因又問寶玉可有了。寶玉雖作了些,只是自己嫌不好,又都抹了,要另作,回頭看香,已將燼了。李紈笑道:“這算輸了。蕉丫頭的半首且寫出來。”探春聽說,忙寫了出來。衆人看時,\begin{note}庚雙夾:卻是先看沒作完的,總是又變一格也。\end{note}上面卻只半首《南柯子》,寫道是:
\end{parag}


\begin{poem}
    \begin{pl}空掛纖纖縷,徒垂絡絡絲,\end{pl}

    \begin{pl}也難綰系也難羈,一任東西南北各分離。\end{pl}
\end{poem}


\begin{parag}
    李紈笑道:“這也卻好作,何不續上?”寶玉見香沒了,情願認負,不肯勉強塞責,將筆擱下,來瞧這半首。見沒完時,反倒動了興開了機,乃提筆續道是:
\end{parag}


\begin{poem}
    \begin{pl}落去君休惜,飛來我自知。\end{pl}

    \begin{pl}鶯愁蝶倦晚芳時,縱是明春再見隔年期。\end{pl}

\end{poem}


\begin{parag}
    衆人笑道:“正經你份內的又不能,這卻偏有了。縱然好,也不算得。”說著,看黛玉的《唐多令》:
\end{parag}


\begin{poem}
    \begin{pl}粉墮百花州,香殘燕子樓。\end{pl}

    \begin{pl}一團團逐對成毬。\end{pl}

    \begin{pl}飄泊亦如人命薄,空繾綣,說風流。\end{pl}

    \begin{pl}草木也知愁,韶華竟白頭!\end{pl}

    \begin{pl}嘆今生誰舍誰收?\end{pl}

    \begin{pl}嫁與東風春不管,\end{pl}

    \begin{pl}憑爾去,忍淹留。\end{pl}

\end{poem}


\begin{parag}
    衆人看了,俱點頭感嘆,說:“太作悲了,好是固然好的。”因又看寶琴的是《西江月》:
\end{parag}


\begin{poem}
    \begin{pl}漢苑零星有限,隋堤點綴無窮。\end{pl}

    \begin{pl}三春事業付東風,明月梅花一夢。\end{pl}

    \begin{pl}幾處落紅庭院,誰家香雪簾櫳?\end{pl}

    \begin{pl}江南江北一般同,偏是離人恨重!\end{pl}

\end{poem}


\begin{parag}
    衆人都笑說:“到底是他的聲調壯。‘幾處’‘誰家’兩句最妙。”寶釵笑道:“終不免過於喪敗。我想,柳絮原是一件輕薄無根無絆的東西,然依我的主意,偏要把他說好了,纔不落套。所以我謅了一首來,未必合你們的意思。”衆人笑道:“不要太謙。我們且賞鑑,自然是好的。”因看這一首,《臨江仙》道是:
\end{parag}


\begin{poem}
    \begin{pl}白玉堂前春解舞,東風捲得均勻。\end{pl}
\end{poem}


\begin{parag}
    湘雲先笑道:“好一個‘東風捲得均勻’!這一句就出人之上了。”又看底下道:
\end{parag}


\begin{poem}
    \begin{pl}蜂團蝶陣亂紛紛。\end{pl}

    \begin{pl}幾曾隨逝水,豈必委芳塵。\end{pl}

    \begin{pl}萬縷千絲終不改,任他隨聚隨分。\end{pl}

    \begin{pl}韶華休笑本無根,\end{pl}

    \begin{pl}好風頻借力,送我上青雲。\end{pl}


\end{poem}


\begin{parag}
    衆人拍案叫絕,都說:“果然翻得好,氣力自然,是這首爲尊。纏綿悲慼,讓瀟湘妃子,情致嫵媚,卻是枕霞,小薛與蕉客今日落第,要受罰的。”寶琴笑道:“我們自然受罰,但不知付白卷子的又怎麼罰?”李紈道:“不要忙,這定要重重罰他。下次爲例。”
\end{parag}


\begin{parag}
    一語未了,只聽窗外竹子上一聲響,恰似窗屜子倒了一般,衆人唬了一跳。丫鬟們出去瞧時,簾外丫鬟嚷道:“一個大蝴蝶風箏掛在竹梢上了。”衆丫鬟笑道: “好一個齊整風箏!不知是誰家放斷了繩,拿下他來。”寶玉等聽了,也都出來看時,寶玉笑道:“我認得這風箏。這是大老爺那院裏嬌紅姑娘放的,拿下來給他送過去罷。”紫鵑笑道:“難道天下沒有一樣的風箏,單他有這個不成?我不管,我且拿起來。”探春道:“紫鵑也學小氣了。你們一般的也有,這會子拾人走了的,也不怕忌諱。”黛玉笑道:“可是呢,知道是誰放晦氣的,快掉出去罷。把咱們的拿出來,咱們也放晦氣。”紫鵑聽了,趕著命小丫頭們將這風箏送出與園門上值日的婆子去了,倘有人來找,好與他們去的。
\end{parag}


\begin{parag}
    這裏小丫頭們聽見放風箏,巴不得七手八腳都忙著拿出個美人風箏來。也有搬高凳去的,也有捆剪子股的,也有撥籰子的。寶釵等都立在院門前,命丫頭們在院外敞地下放去。寶琴笑道:“你這個不大好看,不如三姐姐的那一個軟翅子大鳳凰好。”寶釵笑道:“果然。” 因回頭向翠墨笑道:“你把你們的拿來也放放。”翠墨笑嘻嘻的果然也取去了。寶玉又興頭起來,也打發個小丫頭子家去,說:“把昨兒賴大娘送我的那個大魚取來。”小丫頭子去了半天,空手回來,笑道:“晴姑娘昨兒放走了。”寶玉道:“我還沒放一遭兒呢。”探春笑道:“橫豎是給你放晦氣罷了。”寶玉道:“也罷。再把那個大螃蟹拿來罷。”丫頭去了,同了幾個人扛了一個美人並籰子來,說道:“襲姑娘說,昨兒把螃蟹給了三爺了。這一個是林大娘才送來的,放這一個罷。” 寶玉細看了一回,只見這美人做的十分精緻。心中歡喜,便命叫放起來。此時探春的也取了來,翠墨帶著幾個小丫頭子們在那邊山坡上已放了起來。寶琴也命人將自己的一個大紅蝙蝠也取來。寶釵也高興,也取了一個來,卻是一連七個大雁的,都放起來。獨有寶玉的美人放不起去。寶玉說丫頭們不會放,自己放了半天,只起房高便落下來了。急的寶玉頭上出汗,衆人又笑。寶玉恨的擲在地下,指著風箏道:“若不是個美人,我一頓腳跺個稀爛。”黛玉笑道:“那是頂線不好,拿出去另使人打了頂線就好了。”寶玉一面使人拿去打頂線,一面又取一個來放。大家都仰面而看,天上這幾個風箏都起在半空中去了。
\end{parag}


\begin{parag}
    一時丫鬟們又拿了許多各式各樣的送飯的來,頑了一回。紫鵑笑道:“這一回的勁大,姑娘來放罷。”黛玉聽說,用手帕墊著手,頓了一頓,果然風緊力大,接過籰子來,隨著風箏的勢將籰子一鬆,只聽一陣豁刺刺響,登時籰子線盡。黛玉因讓衆人來放。衆人都笑道:“各人都有,你先請罷。”黛玉笑道:“這一放雖有趣,只是不忍。”李紈道:“放風箏圖的是這一樂,所以又說放晦氣,你更該多放些,把你這病根兒都帶了去就好了。”紫鵑笑道:“我們姑娘越發小氣了。那一年不放幾個子,今忽然又心疼了。姑娘不放,等我放。”說著便向雪雁手中接過一把西洋小銀剪子來,齊籰子根下寸絲不留,咯登一聲鉸斷,笑道:“這一去把病根兒可都帶了去了。”那風箏飄飄搖搖,只管往後退了去,一時只有雞蛋大小,展眼只剩了一點黑星,再展眼便不見了。衆人皆仰面睃眼說:“有趣,有趣。”寶玉道: “可惜不知落在那裏去了。若落在有人煙處,被小孩子得了還好,若落在荒郊野外無人煙處,我替他寂寞。想起來把我這個放去,教他兩個作伴兒罷。”於是也用剪子剪斷,照先放去。探春正要剪自己的鳳凰,見天上也有一個鳳凰,因道:“這也不知是誰家的。”衆人皆笑說:“且別剪你的,看他倒象要來絞的樣兒。”說著,只見那鳳凰漸逼近來,遂與這鳳凰絞在一處。衆人方要往下收線,那一家也要收線,正不開交,又見一個門扇大的玲瓏喜字帶響鞭,在半天如鐘鳴一般,也逼近來。衆人笑道:“這一個也來絞了。且別收,讓他三個絞在一處倒有趣呢。”說著,那喜字果然與這兩個鳳凰絞在一處。三下齊收亂頓,誰知線都斷了,那三個風箏飄飄搖搖都去了。衆人拍手鬨然一笑,說:“倒有趣,可不知那喜字是誰家的,忒促狹了些。”黛玉說:“我的風箏也放去了,我也乏了,我也要歇歇去了。”寶釵說: “且等我們放了去,大家好散。”說著,看姊妹都放去了,大家方散。黛玉回房歪著養乏。要知端的,下回便見。
\end{parag}


\begin{parag}
    \begin{note}蒙回後總評:文於雪天聯詩篇,一樣機軸兩機筆墨。前文以聯句起,以燈謎結,以作畫爲中間橫風吹斷,此文以填詞起,以風箏結,以寫字爲中間橫風吹斷,是一樣機軸;前文敘聯句詳,此文敘填詞略,是兩樣筆墨,前文之敘作畫略,此文之敘寫字詳,是兩樣筆墨。前文敘燈謎,敘猜燈謎,此文敘風箏,敘放風箏,是一樣機軸;前文敘七律在聯句後,此文敘古歌在填詞前,是兩樣筆墨。前文敘黛玉替寶玉寫詩,此文敘寶玉替探春續詞,是一樣機軸,前文賦詩後有一首詩,此文填詞前有一首詞,是兩樣筆墨。噫!參伍之變,錯綜其數,此固難爲粗心這道也!\end{note}
\end{parag}
