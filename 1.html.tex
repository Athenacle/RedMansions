\chap{一}{甄士隐梦幻识通灵 贾雨村风尘怀闺秀}

\begin{parag}
    \begin{note} \begin{subnote} 按:此段批语混入正文。\end{subnote}
        甲戌、庚、蒙批:此书开卷第一回也,作者自云:因曾历过一番梦幻之后,故将真事隐去,而借通灵之说,撰此《石头记》一书也。故曰“甄士隐梦幻识通灵”。但书中所记何事,又因何而撰是书哉?自又云:今风尘碌碌,一事无成,忽念及当日所有之女子,一一细推了去,觉其行止见识,皆出于我之上。何我堂堂之须眉,曾不若彼裙钗哉! 蒙侧:何非梦幻,何不通灵?作者托言,原当有自。受气清浊,本无男女之别。实愧则有余,悔又无益之大无可奈何之日也!当此时,则自欲将已往所赖,上赖天恩,下承祖德,锦衣纨绔之时、饫甘餍美肥之日,背父母教育之恩,负师兄规训之德,已至今日一事无成、半生潦倒之罪, 蒙侧:明告看者。编述一记,以告普天下人。我之罪固不能免,然闺阁中本自历历有人,万不可因我之不肖,自护其短,则一并使其泯灭也。 蒙侧:因为传他,并可传我。虽今日之茆椽蓬牖,瓦灶绳床,其风晨月夕,阶柳庭花,亦未有伤于我之襟怀笔墨者。虽我未学,下笔无文,何为不用假语村言,敷演出一段故事来,以悦人之耳目哉。故曰“风尘怀闺秀”,乃是第一回题纲正义也。开卷即云“风尘怀闺秀”,则知作者本意原为记述当日闺友闺情,并非怨世骂时之书矣。虽一时有涉于世态,然亦不得不叙者,但非其本旨耳,阅者切记之。诗曰:
    \end{note}

\end{parag}


\begin{poem}
    \begin{pl}\begin{note} 浮生着甚苦奔忙?盛席华筵终散场。\end{note}\end{pl}

    \begin{pl}\begin{note} 悲喜千般同幻渺,古今一梦尽荒唐。\end{note}\end{pl}

    \begin{pl}\begin{note} 谩言红袖啼痕重,更有情痴抱恨长。\end{note}\end{pl}

    \begin{pl}\begin{note} 字字看来皆是血,十年辛苦不寻常!\end{note}\end{pl}
\end{poem}


\begin{parag}
    \begin{note}杨、庚、觉、舒回前:此回中凡用梦用幻等字,是提醒阅者眼目,亦是此书立意本旨。\end{note}

\end{parag}


\begin{parag}
    列位看官,你道此书从何而来?说起根由虽近荒唐
    \begin{note}甲侧:自占地步。自首荒唐。妙!\end{note},细谙则深有趣味。待在下将此来历注明,方使闻者了然不惑。

\end{parag}


\begin{parag}
    原来,当年女娲氏炼石补天之时\begin{note}甲侧:补天济世,勿认真,用常言。\end{note},于大荒山\begin{note}甲侧:荒唐也。\end{note}无稽崖\begin{note}甲侧:无稽也。\end{note}炼成高经十二丈\begin{note}甲侧:总应十二钗。\end{note}、方经二十四丈\begin{note}甲侧:照应副十二钗。\end{note}顽石三万六千五百零一块。娲皇氏只用了三万六千五百块\begin{note}甲侧:合周天之数。\end{note},只单单的剩下了一块未用\begin{note}甲侧:剩了这一块便生出这许多故事。使当日虽不以此补天,就该去补地之坑陷,使地平坦,而不得有此一部鬼话。蒙侧:数足,偏遗我。“不堪入选”句中透出心眼。\end{note},便弃在此山青埂峰下\begin{note}甲眉:妙!自谓落堕情根,故无补天之用。\end{note}。谁知此石自经煅炼之后,灵性已通\begin{note}甲侧:煅炼后,性方通。甚哉!人生不能学也。\end{note},因见众石俱得补天,独自己无材不堪入选,遂自怨自嗟,日夜悲号惭愧。

\end{parag}


\begin{parag}
    一日,正当嗟悼之际,俄见一僧一道远远而来,生得气骨不凡,丰神迥异,\begin{note}蒙双:这是真像,非幻像也。靖眉:作者自己形容。\end{note}说说笑笑来至峰下,坐于石边高谈快论。先是说些云山雾海,神僊玄幻之事,后便说到红尘中荣华富贵。此石听了,不觉打动凡心,也想要到人间去享一享这荣华富贵,但自恨粗蠢,不得已,便口吐人言,\begin{note}甲侧:竟有人问口生于何处,其无心肝,可笑可恨之极!\end{note}向那僧道说道:“大师!弟子蠢物\begin{note}甲侧:岂敢岂敢。\end{note},不能见礼了。适闻二位谈那人世间荣耀繁华,心切慕之。弟子质虽粗蠢\begin{note}甲侧:岂敢岂敢。\end{note},性却稍通,况见二师仙形道体,定非凡品,必有补天济世之材,利物济人之德。如蒙发一点慈心,携带弟子得入红尘,在那富贵场中、温柔乡里受享几年,自当永佩洪恩,万劫不忘也。”二仙师听毕,齐憨笑道:“善哉,善哉!那红尘中有却有些乐事,但不能永远依恃。况又有‘美中不足,好事多魔’八个字紧相连属。瞬息间则又乐极悲生,人非物换。究竟是到头一梦,万境归空。\begin{note}甲侧:四句乃一部之总纲。\end{note}倒不如不去的好。”这石凡心已炽,那里听得进这话去,乃复苦求再四。二仙知不可强制,乃叹道:“此亦静极思动,无中生有之数也。既如此,我们便携你去受享受享,只是到不得意时,切莫后悔。”石道:“自然,自然。”那僧又道:“若说你性灵,却又如此质蠢,并更无奇贵之处,如此也只好踮脚而已。\begin{note}甲侧:煅炼过尚与人踮脚,不学者又当如何?\end{note}也罢,我如今大施佛法助你助,待劫终之日,复还本质,以了此案。你道好否?\begin{note}甲侧:妙!佛法亦须偿还,况世人之债乎?近之赖债者来看此句。所谓游戏笔墨也。\end{note}”石头听了,感谢不尽。那僧便念咒书符,大展幻\begin{note}甲侧:明点幻字。好!\end{note}术,将一块大石,登时变成一块鲜明莹洁的美玉,且又缩成扇坠大小的可佩可拿。\begin{note}甲侧:奇诡险怪之文,有如髯苏《石钟》《赤壁》用幻处。\end{note}那僧托于掌上,笑道:“形体倒也是个宝物了,\begin{note}甲侧:自愧之语。蒙双:世上人原自据看得见处为凭。\end{note}还只没有实在的好处\begin{note}甲侧,蒙、戚、觉双:妙极!今之金玉其外、败絮其中者,见此大不欢喜。\end{note},得再镌上数字,使人一见便知是奇物方妙。\begin{note}甲侧:世上原宜假,不宜真也。谚云:“一日卖了三千假,三日卖不出一个真。”信哉!\end{note}然后好携你到那昌明隆盛之邦\begin{note}甲侧:伏长安大都。\end{note},诗礼簪缨之族\begin{note}甲侧:伏荣国府。\end{note},花锦繁华之地\begin{note}甲侧:伏大观园。\end{note},温柔富贵之乡\begin{note}甲侧:伏紫芸轩。\end{note}去安身乐业。\begin{note}甲侧:何不再添一句云:“择个绝世情痴作主人”?甲眉:昔子房后谒黄石公,惟见一石。子房当时恨不能随此石去。余亦恨不能随此石去也。聊供阅者一笑。\end{note}”石头听了,喜不能禁,乃问:“不知赐了弟子那几件奇处,\begin{note}甲侧:可知若果有奇贵之处,自己亦不知者;若自以奇贵而居,究竟是无真奇贵之人。\end{note}又不知携了弟子到何处?望乞明示,使弟子不惑。”那僧笑道:“你且莫问,日后自然明白的。”说着,便袖了那石,同那道人飘然而去,竟不知投奔何方何舍去了。

\end{parag}


\begin{parag}
    后来,不知又过了几世几劫。因有个空空道人访道求仙,忽从这大荒山无稽崖青埂峰下经过,忽见一大块石上字迹分明,编述历历。空空道人乃从头一看,原来就是无材补天,幻形入世,\begin{note}甲侧:八字便是作者一生惭恨。\end{note}蒙茫茫大士、渺渺真人携入红尘,历尽一番离合悲欢、炎凉世态的一段故事。后面又有一首偈云:
\end{parag}


\begin{poem}
    \begin{pl} 无材可与补苍天,\end{pl}\begin{note}甲侧:书之本旨。\end{note}

    \begin{pl} 枉入红尘若许年!\end{pl}\begin{note}甲侧:惭愧之言,呜咽如闻。\end{note}

    \begin{pl} 此系身前身后事,\end{pl}

    \begin{pl} 倩谁寄去作神传?\end{pl}
\end{poem}


\begin{parag}
    诗后便是此石堕落之乡,投胎之处,亲自经历的一段陈迹故事。其中家庭闺阁琐事,以及闲情诗词到还全备,或\begin{note}甲侧:或字谦得好。\end{note}可适情解闷,然朝代年纪、地舆邦国,\begin{note}甲侧:若用此套者,胸中必无好文字,手中断无新笔墨。\end{note}却反失落无考。\begin{note}甲侧:据余说,却大有考证。蒙侧:妙在无考。\end{note}
\end{parag}


\begin{parag}
    空空道人遂向石头说道:“石兄,你这一段故事,据你自己说有些趣味,故编写在此,意欲问世传奇。据我看来,第一件,无朝代年纪可考,\begin{note}甲戌侧:先驳得妙。\end{note}第二件,幷无大贤大忠理朝廷治风俗的善政,\begin{note}甲戌侧:将世人欲驳之腐言预先代人驳尽。妙!\end{note}其中只不过几个异样女子,或情或痴,或小才微善,亦无班姑蔡女之德能。我纵抄去,恐世人不爱看呢。”
\end{parag}


\begin{parag}
    石头笑答道:“我师何太痴耶!若云无朝代可考,今我师竟假借汉唐等年纪添缀,又有何难?\begin{note}甲戌侧:所以答得好。\end{note}但我想,历来野史,皆蹈一辙,莫如我这不借此套者,反倒新奇别致,不过只取其事体情理罢了,又何必拘拘于朝代年纪哉!再者,市井俗人喜看理治之书者甚少,爱适趣闲文者特多。历来野史,或讪谤君相,或贬人妻女,\begin{note}甲戌侧:先批其大端。\end{note}奸淫凶恶,不可胜数。更有一种风月笔墨,其淫秽污臭,涂毒笔墨,坏人子弟,又不可胜数。至若佳人才子等书,则又千部共出一套,且其中终不能不涉于淫滥,以致满纸潘安子建、西子文君,不过作者要写出自己的那两首情诗艳赋来,故假拟出男女二人名姓,又必旁出一小人其间拨乱,\begin{note}蒙侧:放笔以情趣世人,幷评倒多少传奇。文气淋漓,字句切实。\end{note}亦如剧中之小丑然。且嬛婢开口即者也之乎,非文即理。故逐一看去,悉皆自相矛盾,大不近情理之话。竟不如我半世亲睹亲闻的这几个女子,虽不敢说强似前代书中所有之人,但事迹原委,亦可以消愁破闷,也有几首歪诗熟话,可以喷饭供酒。至若离合悲欢,兴衰际遇,则又追踪蹑迹,不敢稍加穿凿,徒为供人之目而反失其真传者。\begin{note}甲戌眉:事则实事,然亦叙得有间架、有曲折、有顺逆、有映带、有隐有见、有正有闰,以致草蛇灰线、空谷传声、一击两鸣、明修栈道、暗渡 陈仓、云龙雾雨、两山对峙、烘云托月、背面敷粉、千皴万染诸奇书中之秘法,亦不复少。余亦于逐回中搜剔刮剖明白注释以待高明,再批示误谬。甲戌眉:开卷一篇立意真,打破历来小说窠臼 。阅其笔则是《庄子》《离骚》之亚。甲戌眉:斯亦太过。\end{note}今之人,贫者日为衣食所累,富者又怀不足之心,纵然一时稍闲,又有贪淫恋色、好货寻愁之事,那里去有工夫看那理治之书?所以我这一段故事,也不愿世人称奇道妙,也不定要世人喜悦检读,\begin{note}甲戌侧:转得更好。\end{note}只愿他们当那醉淫饱卧之时,或避世去愁之际,把此一玩,岂不省了些寿命筋力?就比那谋虚逐妄,却也省了口舌是非之害,腿脚奔忙之苦。再者,亦令世人换新眼目,不比那些胡牵乱扯,忽离忽遇,满纸才人淑女、子建文君、红娘小玉等通共熟套之旧稿。我师意为何如?”\begin{note}甲戌侧:余代空空道人答曰:“不独破愁醒盹,且有大益。”\end{note}
\end{parag}


\begin{parag}
    空空道人听如此说,思忖半晌,将《石头记》\begin{note}甲戌侧:本名。\end{note}再检阅一遍,\begin{note}甲戌侧:这空空道人也太小心了,想亦世之一腐儒耳。\end{note}因见上面虽有些指奸责佞贬恶诛邪之语,\begin{note}甲戌侧:亦断不可少。\end{note}亦非伤时骂世之旨,\begin{note}甲戌侧:要紧句。\end{note}及至君仁臣良父慈子孝,凡伦常所关之处,皆是称功颂德,眷眷无穷,实非别书之可比。虽其中大旨谈情,亦不过实录其事,又非假拟妄称,\begin{note}甲戌侧:要紧句。\end{note}一味淫邀艶约、私订偷盟之可比。因毫不干涉时世,\begin{note}甲戌侧:要紧句。\end{note}方从头至尾抄录回来,问世传奇。从此空空道人因空见色,由色生情,传情入色,自色悟空,遂易名为情僧,改《石头记》为《情僧录》。至吴玉峰题曰《红楼梦》。东鲁孔梅溪则题曰《风月宝鉴》。\begin{note}甲戌眉:雪芹旧有《风月宝鉴》之书,乃其弟棠村序也。今棠村已逝,余睹新怀旧,故仍因之。\end{note}后因曹雪芹于悼红轩中披阅十载,增删五次,纂成目录,分出章回,则题曰《金陵十二钗》。\begin{note}甲戌眉:若云雪芹披阅增删,然后开卷至此,这一篇楔子又系谁撰?足见作者之笔狡猾之甚。后文如此处者不少。这正是作者用画家烟云模糊处,观者万不可被作者瞒蔽了去,方是巨眼。\end{note}幷题一绝云:
\end{parag}


\begin{poem}
    \begin{pl}满纸荒唐言,一把辛酸泪!\end{pl}

    \begin{pl}都云作者痴,谁解其中味?\end{pl}
    \begin{note}甲戌双夹:此是第一首标题诗。甲戌眉:能解者方有辛酸之泪,哭成此书。壬午除夕,书未成,芹为泪尽而逝。余常哭芹,泪亦待尽。每思觅青埂峰再问石兄,奈不遇癞头和尚何!怅怅!今而后惟愿造化主再出一芹一脂,是书何幸,余二人亦大快遂心于九泉矣。甲午八日泪笔。\end{note}
\end{poem}


\begin{parag}
    至脂砚斋甲戌抄阅再评,仍用《石头记》。出则既明,且看石上是何故事。按那石上书云:\begin{note}甲戌侧:以下系石上所记之文。\end{note}
\end{parag}


\begin{parag}
    当日地陷东南,这东南一隅有处曰姑苏,\begin{note}甲戌侧:是金陵。\end{note}有城曰阊门者,最是红尘中一二等富贵风流之地。\begin{note}甲戌侧:妙极!是石头口气,惜米颠不遇此石。\end{note}这阊门外有个十里\begin{note}甲戌侧:开口先云势利,是伏甄、封二姓之事。\end{note}街,街内有个仁清\begin{note}甲戌侧:又言人情,总为士隐火后伏笔。\end{note}巷,巷内有个古庙,因地方窄狭,\begin{note}甲戌侧:世路宽平者甚少。亦凿。\end{note}人皆呼作葫芦\begin{note}甲戌侧:糊涂也,故假语从此具焉。\end{note}庙。\begin{note}蒙侧:画的虽不依样,却是葫芦。\end{note}庙旁住著一家乡宦,\begin{note}甲戌侧:不出荣国大族,先写乡宦小家,从小至大,是此书章法。\end{note}姓甄,\begin{note}甲戌眉:真。后之甄宝玉亦借此音,后不注。\end{note}名费,\begin{note}甲戌侧:废。\end{note}字士隐。\begin{note}甲戌侧:托言将真事隐去也。\end{note}嫡妻封\begin{note}甲戌侧:风。因风俗来。\end{note}氏,情性贤淑,深明礼义。\begin{note}甲戌侧:八字正是写日后之香菱,见其根源不凡。\end{note}家中虽不甚富贵,然本地便也推他为望族了。\begin{note}甲戌侧:本地推为望族,宁、荣则天下推为望族,叙事有层落。\end{note}因这甄士隐禀性恬淡,不以功名为念,\begin{note}甲戌侧:自是羲皇上人,便可作是书之朝代年纪矣。总写香菱根基,原与正十二钗无异。蒙侧:伏笔。\end{note}每日只以观花修竹,酌酒吟诗为乐,倒是神仙一流人品。只是一件不足:如今年已半百,膝下无儿,\begin{note}甲戌侧:所谓“美中不足”也。\end{note}只有一女,乳名英莲,\begin{note}甲戌侧:设云“应怜”也。\end{note}年方三岁。
\end{parag}


\begin{parag}
    一日,炎夏永昼。\begin{note}甲戌侧:热日无多。\end{note}士隐于书房闲坐,至手倦抛书,伏几少憩,不觉朦胧睡去。梦至一处,不辨是何地方。忽见那厢来了一僧一道,\begin{note}甲戌侧:是方从青埂峰袖石而来也,接得无痕。\end{note}且行且谈。
\end{parag}


\begin{parag}
    只听道人问道:“你携了这蠢物,意欲何往?”那僧笑道:“你放心,如今现有一段风流公案正该了结,这一干风流冤家,尚未投胎入世。趁此机会,就将此蠢物夹带于中,使他去经历经历。”那道人道:“原来近日风流冤孽又将造劫历世去不成?\begin{note}蒙侧:苦恼是“造劫历世”,又不能不“造劫历世”,悲夫!\end{note}但不知落于何方何处?”
\end{parag}


\begin{parag}
    那僧笑道:“此事说来好笑,竟是千古未闻的罕事。只因西方灵河岸上三生石畔,\begin{note}甲戌侧:妙!所谓“三生石上旧精魂”也。甲戌眉:全用幻。情之至,莫如此。今采来压卷,其后可知。\end{note}有绛\begin{note}甲戌侧:点“红”字。\end{note}珠\begin{note}甲戌侧:细思“绛珠”二字岂非血泪乎。\end{note}草一株,时有赤瑕\begin{note}甲戌侧:点“红”字“玉”字二。甲戌眉:按“瑕”字本注:“玉小赤也,又玉有病也。”以此命名恰极。\end{note}宫神瑛\begin{note}甲戌侧:单点“玉”字二。\end{note}侍者,日以甘露灌溉,这绛珠草便得久延岁月。后来既受天地精华,复得雨露滋养,遂得脱却草胎木质,得换人形,仅修成个女体,终日游于离恨天外,饥则食蜜青果为膳,渴则饮灌愁海水为汤。\begin{note}甲戌侧:饮食之名奇甚,出身履历更奇甚,写黛玉来历自与别个不同。\end{note}只因尚未酬报灌溉之德,故其五内便郁结著一段缠绵不尽之意。\begin{note}甲戌侧:妙极!恩怨不清,西方尚如此,况世之人乎?趣甚警甚!甲戌眉:以顽石草木为偶,实历尽风月波澜,尝遍情缘滋味,至无可如何,始结此木石因果,以泄胸中悒郁。古人之“一花一石如有意,不语不笑能留人”,此之谓也。蒙侧:点题处,清雅。\end{note}恰近日这神瑛侍者凡心偶炽,\begin{note}甲戌侧:总悔轻举妄动之意。\end{note}乘此昌明太平朝世,意欲下凡造历幻\begin{note}甲戌侧:点“幻”字。\end{note}缘,已在警幻\begin{note}甲戌侧:又出一警幻,皆大关键处。\end{note}仙子案前挂了号。警幻亦曾问及灌溉之情未偿,趁此倒可了结的。那绛珠仙子道:“他是甘露之惠,我幷无此水可还。他既下世为人,我也去下世为人,但把我一生所有的眼泪还他,也偿还得过他了。”\begin{note}甲戌侧:观者至此请掩卷思想,历来小说中可曾有此句?千古未闻之奇文。甲戌眉:知眼泪还债,大都作者一人耳。余亦知此意,但不能说得出。蒙侧:恩情山海债,唯有泪堪还。\end{note}因此一事,就勾出多少风流冤家来,\begin{note}甲戌侧:余不及一人者,盖全部之主惟二玉二人也。\end{note}陪他们去了结此案。”
\end{parag}


\begin{parag}
    那道人道:“果是罕闻,实未闻有还泪之说。\begin{note}蒙侧:作想得奇!\end{note}想来这一段故事,比历来风月事故更加琐碎细腻了。”那僧道:“历来几个风流人物,不过传其大概以及诗词篇章而已,至家庭闺阁中一饮一食,总未述记。再者,大半风月故事,不过偷香窃玉、暗约私奔而已,幷不曾将儿女之真情发泄一二。\begin{note}蒙侧:所以别致。\end{note}想这一干人入世,其情痴色鬼,贤愚不肖者,悉与前人传述不同矣。”
\end{parag}


\begin{parag}
    那道人道:“趁此何不你我也去下世度脱\begin{note}蒙侧:“度脱”,请问是幻不是幻?\end{note}几个,岂不是一场功德?”那僧道:“正合吾意,你且同我到警幻仙子宫中,将蠢物交割清楚,待这一干风流孽鬼下世已完,你我再去。\begin{note}蒙侧:幻中幻,何不可幻?情中情,谁又无情?不觉僧道亦入幻中矣。\end{note}如今虽已有一半落尘,然犹未全集。”\begin{note}甲戌侧:若从头逐个写去,成何文字?《石头记》得力处在此。丁亥春。\end{note}
\end{parag}


\begin{parag}
    道人道:“既如此,便随你去来。”
\end{parag}


\begin{parag}
    却说甄士隐俱听得明白,但不知所云蠢物系何东西。遂不禁上前施礼,笑问道:“二仙师请了。”那僧道也忙答礼相问。士隐因说道:“适闻仙师所谈因果,实人世罕闻者。但弟子愚浊,不能洞悉明白,若蒙大开痴顽,备细一闻,弟子则洗耳谛听,稍能警省,亦可免沉伦之苦。”二仙笑道:“此乃玄机不可预泄者。到那时不要忘了我二人,便可跳出火坑矣。”士隐听了,不便再问。因笑道:“玄机不可预泄,但适云‘蠢物’,不知为何,或可一见否?”那僧道:“若问此物,倒有一面之缘。”说著,取出递与士隐。士隐接了看时,原来是块鲜明美玉,上面字迹分明,镌著“通灵宝玉”四字,\begin{note}甲戌侧:凡三四次始出明玉形,隐屈之至。\end{note}后面还有几行小字。正欲细看时,那僧便说已到幻境,\begin{note}甲戌侧:又点“幻”字,云书已入幻境矣。蒙侧:幻中言幻,何等法门。\end{note}便强从手中夺了去,与道人竟过一大石牌坊,上书四个大字,乃是“太虚幻境”。\begin{note}甲戌侧:四字可思。\end{note}两边又有一幅对联,道是:\begin{note}蒙双夹:无极太极之轮转,色空之相生,四季之随行,皆不过如此。\end{note}
\end{parag}


\begin{poem}
    \begin{pl}假作真时真亦假,无为有处有还无。\end{pl}
    \begin{note}甲夹:叠用真假有无字,妙!\end{note}
\end{poem}


\begin{parag}
    士隐意欲也跟了过去,方举步时,忽听一声霹雳,有若山崩地陷。士隐大叫一声,定睛一看,\begin{note}蒙侧:真是大警觉大转身。\end{note}只见烈日炎炎,芭蕉冉冉,\begin{note}甲戌侧:醒得无痕,不落旧套。\end{note}所梦之事便忘了对半。\begin{note}甲戌侧:妙极!若记得,便是俗笔了。\end{note}
\end{parag}


\begin{parag}
    又见奶母正抱了英莲走来。士隐见女儿越发生得粉妆玉琢,乖觉可喜,便伸手接来,抱在怀内,斗他顽耍一回,又带至街前,看那过会的热闹。方欲进来时,只见从那边来了一僧一道,\begin{note}甲戌侧:所谓“万境都如梦境看”也。\end{note}那僧则癞头跣脚,那道则跛足蓬头,\begin{note}甲戌侧:此则是幻像。\end{note}疯疯癫癫,挥霍谈笑而至。及至到了他门前,看见士隐抱著英莲,那僧便大哭起来,\begin{note}甲戌侧:奇怪!所谓情僧也。\end{note}又向士隐道:“施主,你把这有命无运,累及爹娘之物,抱在怀内作甚?”\begin{note}甲戌眉:八个字屈死多少英雄?屈死多少忠臣孝子?屈死多少仁人志士?屈死多少词客骚人?今又被作者将此一把眼泪洒与闺阁之中,见得裙钗尚遭逢此数,况天下之男子乎?看他所写开卷之第一个女子便用此二语以定终身,则知托言寓意之旨,谁谓独寄兴于一“情”字耶!武侯之三分,武穆之二帝,二贤之恨,及今不尽,况今之草芥乎?家国君父事有大小之殊,其理其运其数则略无差异。知运知数者则必谅而后叹也。\end{note}士隐听了,知是疯话,也不去睬他。那僧还说:“舍我罢,舍我罢!”士隐不耐烦,便抱女儿撤身要进去,\begin{note}蒙侧:如果舍出,则不成幻境矣。行文至此,又不得不有此一语。\end{note}那僧乃指著他大笑,口内念了四句言词道:
\end{parag}


\begin{poem}
    \begin{pl}惯养娇生笑你痴,\end{pl}\begin{note}甲戌侧:为天下父母痴心一哭。\end{note}

    \begin{pl}菱花空对雪澌澌。\end{pl}\begin{note}甲戌侧:生不遇时。遇又非偶。\end{note}

    \begin{pl}好防佳节元宵后,\end{pl}\begin{note}甲戌侧:前后一样,不直云前而云后,是讳知者。\end{note}

    \begin{pl}便是烟消火灭时!\end{pl}\begin{note}甲戌侧:伏后文。\end{note}
\end{poem}


\begin{parag}
    士隐听得明白,心下犹豫,意欲问他们来历。只听道人说道:“你我不必同行,就此分手,各干营生去罢。三劫后,\begin{note}甲戌眉:佛以世谓“劫”,凡三十年为一世。三劫者,想以九十春光寓言也。\end{note}我在北邙山等你,会齐了同往太虚幻境销号。”那僧道:“妙妙妙!”说毕,二人一去,再不见个踪影了。士隐心中此时自忖:这两个人必有来历,该试一问,如今悔却晚也。
\end{parag}


\begin{parag}
    这士隐正痴想,忽见隔壁\begin{note}甲戌侧:“隔壁”二字极细极险,记清。\end{note}葫芦庙内寄居的一个穷儒,姓贾名化,\begin{note}甲戌侧:假话。妙!\end{note}表字时飞,\begin{note}甲戌侧:实非。妙!\end{note}别号雨村\begin{note}甲戌侧:雨村者,村言粗语也。言以村粗之言演出一段假话也。\end{note}者走了出来。这贾雨村原系胡州\begin{note}甲戌侧:胡诌也。\end{note}人氏,也是诗书仕宦之族,因他生于末世,\begin{note}甲戌侧:又写一末世男子。\end{note}父母祖宗根基已尽,人口衰丧,只剩得他一身一口,在家乡无益。\begin{note}蒙侧:形容落破诗书子弟,逼真。\end{note}因进京求取功名,再整基业。自前岁来此,又淹蹇住了,暂寄庙中安身,每日卖字作文为生,\begin{note}蒙侧:“庙中安身”、“卖字为生”,想是过午不食的了。\end{note}故士隐常与他交接。\begin{note}甲戌侧:又夹写士隐实是翰林文苑,非守钱虏也,直灌入“慕雅女雅集苦吟诗”一回。\end{note}当下雨村见了士隐,忙施礼陪笑道:“老先生倚门伫望,敢是街市上有甚新闻否?”士隐笑道:“非也,适因小女啼哭,引他出来作耍,正是无聊之甚,兄来得正妙,请入小斋一谈,彼此皆可消此永昼。”说著,便令人送女儿进去,自与雨村携手来至书房中。小童献茶。方谈得三五句话,忽家人飞报:“严\begin{note}甲戌侧:“炎”也。炎既来,火将至矣。\end{note}老爷来拜。”士隐慌的忙起身谢罪道:“恕诳驾之罪,略坐,弟即来陪。”雨村忙起身亦让道:“老先生请便。晚生乃常造之客,稍候何妨。”\begin{note}蒙侧:世态人情,如闻其声。\end{note}说著,士隐已出前厅去了。
\end{parag}


\begin{parag}
    这里雨村且翻弄书籍解闷。忽听得窗外有女子嗽声,雨村遂起身往窗外一看,原来是一个丫嬛,在那里撷花,生得仪容不俗,眉目清明,\begin{note}甲戌侧:八字足矣。\end{note}虽无十分姿色,却亦有动人之处。\begin{note}甲戌眉:更好。这便是真正情理之文。可笑近之小说中满纸“羞花闭月”等字。这是雨村目中,又不与后之人相似。\end{note}雨村不觉看的呆了。\begin{note}甲戌侧:今古穷酸色心最重。\end{note}那甄家丫嬛撷了花,方欲走时,猛抬头见窗内有人,敝巾旧服,虽是贫窘,然生得腰圆背厚,面阔口方,更兼剑眉星眼,直鼻权腮。\begin{note}甲戌侧:是莽操遗容。甲戌眉:最可笑世之小说中,凡写奸人则用“鼠耳鹰腮”等语。\end{note}这丫嬛忙转身回避,心下乃想:“这人生的这样雄壮,却又这样褴褛,想他定是我家主人常说的什么贾雨村了,每有意帮助周济,只是没甚机会。我家幷无这样贫窘亲友,想定是此人无疑了。怪道又说他必非久困之人。”如此想来,不免又回头两次。\begin{note}甲戌眉:这方是女儿心中意中正文。又最恨近之小说中满纸红拂紫烟。蒙侧:如此忖度,岂得为无情?\end{note}雨村见他回了头,便自为这女子心中有意于他,\begin{note}甲戌侧:今古穷酸皆会替女妇心中取中自己。\end{note}便狂喜不尽,自为此女子必是个巨眼英雄,风尘中之知己也。\begin{note}蒙侧:在此处已把种点出。\end{note}一时小童进来,雨村打听得前面留饭,不可久待,遂从夹道中自便出门去了。士隐待客既散,知雨村自便,也不去再邀。
\end{parag}


\begin{parag}
    一日,早又中秋佳节。士隐家宴已毕,乃又另具一席于书房,却自己步月至庙中来邀雨村。\begin{note}甲戌侧:写士隐爱才好客。\end{note}原来雨村自那日见了甄家之婢曾回顾他两次,自为是个知己,便时刻放在心上。\begin{note}蒙侧:也是不得不留心。不独因好色,多半感知音。\end{note}今又正值中秋,不免对月有怀,因而口占五言一律云:\begin{note}甲戌双夹:这是第一首诗。后文香奁闺情皆不落空。余谓雪芹撰此书,中亦有传诗之意。\end{note}
\end{parag}


\begin{poem}
    \begin{pl}未卜三生愿,频添一段愁。\end{pl}

    \begin{pl}闷来时敛额,行去几回头。\end{pl}

    \begin{pl}自顾风前影,谁堪月下俦?\end{pl}

    \begin{pl}蟾光如有意,先上玉人楼。\end{pl}
\end{poem}


\begin{parag}
    雨村吟罢,因又思及平生抱负,苦未逢时,乃又搔首对天长叹,复高吟一联曰:
\end{parag}


\begin{poem}
    \begin{pl}玉在匮中求善价,钗于奁内待时飞。\begin{note}甲戌侧:表过黛玉则紧接上宝钗。甲夹:前用二玉合传,今用二宝合传,自是书中正眼。蒙侧:偏有些脂气。\end{note}\end{pl}\end{poem}


\begin{parag}
    恰值士隐走来听见,笑道:“雨村兄真抱负不浅也!”雨村忙笑道:“不过偶吟前人之句,何敢狂诞至此。”因问:“老先生何兴至此?”士隐笑道:“今夜中秋,俗谓‘团圆之节’,想尊兄旅寄僧房,不无寂寥之感,故特具小酌,邀兄到敝斋一饮,不知可纳芹意否?”雨村听了,幷不推辞,\begin{note}蒙侧:“不推辞”语便不入估(俗)矣。\end{note}便笑道:“既蒙厚爱,何敢拂此盛情。”\begin{note}甲戌侧:写雨村豁达,气象不俗。\end{note}说著,便同士隐复过这边书院中来。
\end{parag}


\begin{parag}
    须臾茶毕,早已设下杯盘,那美酒佳肴自不必说。二人归坐,先是款斟漫饮,次渐谈至兴浓,不觉飞觥限斝起来。当时街坊上家家箫管,户户弦歌,当头一轮明月,飞彩凝辉,二人愈添豪兴,酒到杯干。雨村此时已有七八分酒意,狂兴不禁,乃对月寓怀,口号一绝云:
\end{parag}


\begin{poem}
    \begin{pl}时逢三五便团圆,\end{pl}\begin{note}甲戌侧:是将发之机。\end{note}

    \begin{pl}满把晴光护玉栏。\end{pl}\begin{note}甲戌侧:奸雄心事,不觉露出。\end{note}

    \begin{pl}天上一轮才捧出,\end{pl}

    \begin{pl}人间万姓仰头看。\end{pl}\begin{note}甲戌眉:这首诗非本旨,不过欲出雨村,不得不有者。用中秋诗起,用中秋诗收,又用起诗社于秋日。所叹者三春也,却用三秋作关键。\end{note}
\end{poem}


\begin{parag}
    士隐听了,大叫:“妙哉!吾每谓兄必非久居人下者,今所吟之句,飞腾之兆已见,不日可接履于云霓之上矣。可贺,可贺!”\begin{note}蒙侧:伏笔,作□言语。妙!\end{note}乃亲斟一斗为贺。\begin{note}甲戌侧:这个“斗”字莫作升斗之斗看,可笑。\end{note}雨村因干过,叹道:“非晚生酒后狂言,若论时尚之学,\begin{note}甲戌侧:四字新而含蓄最广,若必指明,则又落套矣。\end{note}晚生也或可去充数沽名,只是目今行囊路费一概无措,神京路远,非赖卖字撰文即能到者。”士隐不待说完,便道:“兄何不早言。愚每有此心,但每遇兄时,兄幷未谈及,愚故未敢唐突。今既及此,愚虽不才,‘义利’二字却还识得。\begin{note}蒙侧:“义利”二字,时人故自不识。\end{note}且喜明岁正当大比,兄宜作速入都,春闱一战,方不负兄之所学也。其盘费余事,弟自代为处置,亦不枉兄之谬识矣!”当下即命小童进去,速封五十两白银,幷两套冬衣。\begin{note}甲戌眉:写士隐如此豪爽,又无一些粘皮带骨之气相,愧杀近之读书假道学矣。\end{note}又云:“十九日乃黄道之期,兄可即买舟西上,待雄飞高举,明冬再晤,岂非大快之事耶!”雨村收了银衣,不过略谢一语,幷不介意,仍是吃酒谈笑。\begin{note}甲戌侧:写雨村真是个英雄。蒙侧:托大处,即遇此等人,又不得太琐细。\end{note}那天已交了三更,二人方散。
\end{parag}


\begin{parag}
    士隐送雨村去后,回房一觉,直至红日三竿方醒。\begin{note}甲戌侧:是宿酒。\end{note}因思昨夜之事,意欲再写两封荐书与雨村带至神都,使雨村投谒个仕宦之家为寄足之地。\begin{note}甲戌侧:又周到如此。\end{note}因使人过去请时,那家人去了回来说:“和尚说,贾爷今日五鼓已进京去了,也曾留下话与和尚转达老爷,说:‘读书人不在黄道黑道,总以事理为要,不及面辞了。’”\begin{note}甲戌侧:写雨村真令人爽快。\end{note}士隐听了,也只得罢了。
\end{parag}


\begin{parag}
    真是闲处光阴易过,倏忽又是元霄佳节矣。士隐命家人霍启\begin{note}甲戌侧:妙!祸起也。此因事而命名。\end{note}抱了英莲去看社火花灯,半夜中,霍启因要小解,便将英莲放在一家门槛上坐著。待他小解完了来抱时,那有英莲的踪影?急得霍启直寻了半夜,至天明不见,那霍启也就不敢回来见主人,便逃往他乡去了。那士隐夫妇,见女儿一夜不归,便知有些不妥,再使几人去寻找,回来皆云连音响皆无。夫妻二人,半世只生此女,一旦失落,岂不思想,因此昼夜啼哭,几乎不曾寻死。\begin{note}甲戌眉:喝醒天下父母之痴心。蒙侧:天下作子弟的,看了想去。\end{note}看看的一月,士隐先就得了一病,当时封氏孺人也因思女构疾,日日请医疗治。
\end{parag}


\begin{parag}
    不想这日三月十五,葫芦庙中炸供,那些和尚不加小心,致使油锅火逸,便烧著窗纸。此方人家多用竹篱木壁者,\begin{note}甲戌侧:土俗人风。蒙侧:交竹滑溜婉转。\end{note}大抵也因劫数,于是接二连三,牵五挂四,将一条街烧得如火焰山一般。\begin{note}甲戌眉:写出南直召祸之实病。\end{note}彼时虽有军民来救,那火已成了势,如何救得下?直烧了一夜,方渐渐的熄去,也不知烧了几家。只可怜甄家在隔壁,早已烧成一片瓦砾场了。只有他夫妇幷几个家人的性命不曾伤了。急得士隐惟跌足长叹而已。只得与妻子商议,且到田庄上去安身。偏值近年水旱不收,鼠盗蜂起,无非抢田夺地,鼠窃狗偷,民不安生,因此官兵剿捕,难以安身。士隐只得将田庄都折变了,便携了妻子与两个丫嬛投他岳丈家去。
\end{parag}


\begin{parag}
    他岳丈名唤封肃,\begin{note}蒙双夹:风俗。\end{note}本贯大如州人氏,\begin{note}甲戌眉:托言大概如此之风俗也。\end{note}虽是务农,家中都还殷实。今见女婿这等狼狈而来,心中便有些不乐。\begin{note}甲戌侧:所以大概之人情如是,风俗如是也。蒙侧:大都不过如此。\end{note}幸而\begin{note}蒙侧:若非“幸而”,则有不留之意。\end{note}士隐还有折变田地的银子未曾用完,拿出来托他随分就价薄置些须房地,为后日衣食之计。那封肃便半哄半赚,些须与他些薄田朽屋。士隐乃读书之人,不惯生理稼穑等事,勉强支持了一二年,越觉穷了下去。封肃每见面时,便说些现成话,且人前人后又怨他们不善过活,只一味好吃懒作\begin{note}甲戌侧:此等人何多之极。\end{note}等语。士隐知投人不著,心中未免悔恨,再兼上年惊唬,急忿怨痛,已有积伤,暮年之人,贫病交攻,竟渐渐的露出那下世的光景来。\begin{note}蒙侧:几几乎。世人则不能止于几几乎,可悲!观至此,不……\end{note}
\end{parag}


\begin{parag}
    可巧这日,拄了拐杖挣挫到街前散散心时,忽见那边来了一个跛足道人,疯癫落脱,麻屣鹑衣,口内念著几句言词,道是:
\end{parag}


\begin{poem}
    \begin{pl}世人都晓神仙好,惟有功名忘不了;\end{pl}

    \begin{pl}古今将相在何方?荒冢一堆草没了!\end{pl}

    \begin{pl}世人都晓神仙好,只有金银忘不了;\end{pl}

    \begin{pl}终朝只恨聚无多,及到多时眼闭了。\end{pl}

    \begin{pl}世人都晓神仙好,只有姣妻忘不了;\end{pl}
    \begin{note}蒙双夹:要写情要写幻境,偏先写出一篇奇人奇境来。\end{note}

    \begin{pl}夫妻日日说恩情,夫死又随人去了。\end{pl}

    \begin{pl}世人都晓神仙好,只有儿孙忘不了;\end{pl}

    \begin{pl}痴心父母古来多,孝顺子孙谁见了!\end{pl}
\end{poem}


\begin{parag}
    士隐听了,便迎上来道:“你满口说些什么?只听见些‘好’‘了’‘好’‘了’。那道人笑道:“你若果听见‘好’‘了’二字,还算你明白。可知世上万般,好便是了,了便是好。若不了,便不好,若要好,须是了。我这歌儿,便名《好了歌》”士隐本是有宿慧的,一闻此言,心中早已彻悟。因笑道:“且住!待我将你这《好了歌》解注出来何如?”道人笑道:“你解,你解。”士隐乃说道:
\end{parag}


\begin{poem}
    \begin{pl}陋室空堂,当年笏满床,\end{pl}\begin{note}甲戌侧:宁、荣未有之先。\end{note}

    \begin{pl}衰草枯杨,曾为歌舞场。\end{pl}\begin{note}甲戌侧:宁、荣既败之后。\end{note}

    \begin{pl}蛛丝儿结满雕梁,\end{pl}\begin{note}甲戌侧:潇湘馆、紫芸轩等处。\end{note}

    \begin{pl}绿纱今又糊在蓬窗上。\end{pl}\begin{note}甲戌侧:雨村等一干新荣暴发之家。甲戌眉:先说场面,忽新忽败,忽丽忽朽,已见得反复不了。\end{note}

    \begin{pl}说什么脂正浓,粉正香,如何两鬓又成霜?\end{pl}\begin{note}甲戌侧:宝钗、湘云一干人。\end{note}

    \begin{pl}昨日黄土陇头堆白骨,\end{pl}\begin{note}甲戌侧:黛玉、晴雯一干人。\end{note}

    \begin{pl}今宵红灯帐底卧鸳鸯。\end{pl}\begin{note}甲戌眉:一段妻妾迎新送死,倏恩倏爱,倏痛倏悲,缠绵不了。\end{note}

    \begin{pl}金满箱,银满箱,\end{pl}\begin{note}甲戌侧:熙凤一干人。\end{note}

    \begin{pl}展眼乞丐人皆谤。\end{pl}\begin{note}甲戌侧:甄玉、贾玉一干人。\end{note}

    \begin{pl}正叹他人命不长,那知自已归来丧!\end{pl}\begin{note}甲戌眉:一段石火光阴,悲喜不了。风露草霜,富贵嗜欲,贪婪不了。\end{note}

    \begin{pl}训有方,保不定日后\end{pl}\begin{note}甲戌侧:言父母死后之日。\end{note}作强梁。\begin{note}甲戌侧:柳湘莲一干人。\end{note}

    \begin{pl}择膏粱,谁承望流落在烟花巷!\end{pl}\begin{note}甲戌眉:一段儿女死后无凭,生前空为筹划计算,痴心不了。\end{note}

    \begin{pl}因嫌纱帽小,致使锁枷杠,\end{pl}\begin{note}甲戌侧:贾赦、雨村一干人。\end{note}

    \begin{pl}昨怜破袄寒,今嫌紫蟒长。\end{pl}\begin{note}甲戌侧:贾兰、贾菌一干人。甲戌眉:一段功名升黜无时,强夺苦争,喜惧不了。\end{note}
    \begin{pl}乱烘烘你方唱罢我登场,\end{pl}\begin{note}甲戌侧:总收。甲戌眉:总收古今亿兆痴人,共历幻场,此幻事扰扰纷纷,无日可了。\end{note}

    \begin{pl}反认他乡是故乡。\end{pl}\begin{note}甲戌侧:太虚幻境青埂峰一幷结住。\end{note}

    \begin{pl}甚荒唐,到头来都是为他人作嫁衣裳!\end{pl}\begin{note}甲戌侧:语虽旧句,用于此妥极是极。苟能如此,便能了得。甲戌眉:此等歌谣原不宜太雅,恐其不能通俗,故只此便妙极。其说得痛切处,又非一味俗语可到。蒙双夹:谁不解得世事如此,有龙象力者方能放得下。\end{note}
\end{poem}


\begin{parag}
    那疯跛道人听了,拍掌笑道:“解得切,解得切!”士隐便笑一声“走罢!”\begin{note}甲戌侧:如闻如见。甲戌眉:“走罢”二字真悬崖撒手,若个能行?蒙侧:一转念间登彼岸。靖眉:“走罢”二字,如见如闻,真悬崖撒手。非过来人,若个能行?\end{note}将道人肩上褡裢抢了过来背著,竟不回家,同了疯道人飘飘而去。
\end{parag}


\begin{parag}
    当下烘动街坊,众人当作一件新闻传说。封氏闻得此信,哭个死去活来,只得与父亲商议,遣人各处访寻,那讨音信?无奈何,少不得依靠著他父母度日。幸而身边还有两个旧日的丫嬛伏侍,主仆三人,日夜作些针线发卖,帮著父亲用度。那封肃虽然日日报怨,也无可奈何了。
\end{parag}


\begin{parag}
    这日,那甄家大丫嬛在门前买线,忽听得街上喝道之声,众人都说新太爷到任。丫嬛于是隐在门内看时,只见军牢快手,一对一对的过去,俄而大轿抬著一个乌帽猩袍的官府过去。\begin{note}甲戌侧:雨村别来无恙否?可贺可贺。甲戌眉:所谓“乱哄哄,你方唱罢我登场”是也。\end{note}丫嬛倒发了个怔,自思这官好面善,倒象在那里见过的。于是进入房中,也就丢过不在心上。\begin{note}甲戌侧:是无儿女之情,故有夫人之分。蒙侧:起初到底有心乎?无心乎?\end{note}至晚间,正待歇息之时,忽听一片声打的门响,许多人乱嚷,说:“本府太爷差人来传人问话。”\begin{note}蒙侧:不忘情的先写出头一位来了。\end{note}封肃听了,唬得目瞪口呆,不知有何祸事。
\end{parag}


\begin{parag}
    \begin{note}蒙:出口神奇,幻中不幻。文势跳跃,情里生情。借幻说法,而幻中更自多情,因情捉笔,而情里偏成痴幻。试问君家识得否,色空空色两无干。\end{note}
\end{parag}

