\chap{六十六}{情小妹恥情歸地府 冷二郎一冷入空門}

\begin{parag}
    \begin{note}蒙回前總評:餘嘆世人不識“情”字,常把“淫”字當作“情”字。殊不知淫裏有情,情裏無淫,淫必傷情,情必戒淫,情斷處淫生,淫斷處情生。三姐項上一橫,是絕情,乃是正情;湘蓮萬根皆消,是無情,乃是至情。生爲情人,死爲情鬼。故結句曰“來自情天,去自情海”,豈非一篇至情文字?再看他書,則全是“淫”不是 “情”了。\end{note}
\end{parag}


\begin{parag}
    話說鮑二家的打他一下子,笑道:“原有些真的,叫你又編了這混話,越發沒了捆兒。你倒不象跟二爺的人,這些混話倒象是寶玉那邊的了。”\begin{note}庚雙夾:好極之文,將茗煙等已全寫出,可謂一擊兩鳴法,不寫之寫也。\end{note}尤二姐纔要又問,忽見尤三姐笑問道:“可是你們家那寶玉,除了上學,他作些什麼?”\begin{note}庚雙夾:拍案叫絕!此處方問,是何文情!\end{note}興兒笑道:“姨娘別問他,說起來姨娘也未必信。他長了這麼大,獨他沒有上過正經學堂。我們家從祖宗直到二爺,誰不是寒窗十載,偏他不喜讀書。老太太的寶貝,老爺先還管,如今也不敢管了。成天家瘋瘋顛顛的,說的話人也不懂,乾的事人也不知。外頭人人看著好清俊模樣兒,心裏自然是聰明的,誰知是外清而內濁,見了人,一句話也沒有。所有的好處,雖沒上過學,倒難爲他認得幾個字。每日也不習文,也不學武,又怕見人,只愛在丫頭羣裏鬧。再者也沒剛柔,有時見了我們,喜歡時沒上沒下,大家亂頑一陣;不喜歡各自走了,他也不理人。我們坐著臥著,見了他也不理,他也不責備。因此沒人怕他,只管隨便,都過的去。”尤三姐笑道:“主子寬了,你們又這樣;嚴了,又抱怨。可知難纏。”\begin{note}庚雙夾:情語情文至語。\end{note}尤二姐道:“我們看他倒好,原來這樣。可惜了一個好胎子。”尤三姐道:“姐姐信他胡說,咱們也不是見一面兩面的,行事言談喫喝,原有些女兒氣,那是隻在裏頭慣了的。若說糊塗,那些兒糊塗?姐姐記得,穿孝時咱們同在一處,那日正是和尚們進來繞棺,咱們都在那裏站著,他只站在頭裏擋著人。人說他不知禮,又沒眼色。過後他沒悄悄的告訴咱們說:‘姐姐不知道,我並不是沒眼色。想和尚們髒,恐怕氣味燻了姐姐們。’接著他喫茶,姐姐又要茶,那個老婆子就拿了他的碗倒。他趕忙說:‘我喫髒了的,另洗了再拿來。’這兩件上,我冷眼看去,原來他在女孩子們前不管怎樣都過的去,只不大合外人的式,所以他們不知道。”尤二姐聽說,笑道:“依你說,你兩個已是情投意合了。竟把你許了他,豈不好?”三姐見有興兒,不便說話,只低頭嗑瓜子。興兒笑道:“若論模樣兒行事爲人,倒是一對好的。只是他已有了,只未露形。將來準是林姑娘定了的。因林姑娘多病,二則都還小,故尚未及此。再過三二年,老太太便一開言,那是再無不準的了。”大家正說話,只見隆兒又來了,說:“老爺有事,是件機密大事,要遣二爺往平安州去。不過三五日就起身,來回也得半月工夫。今日不能來了。請老奶奶早和二姨定了那事,明日爺來,好作定奪。”說著,帶了興兒回去了。
\end{parag}


\begin{parag}
    這裏尤二姐命掩了門早睡,盤問他妹子一夜。至次日午後,賈璉方來了。尤二姐因勸他說:“既有正事,何必忙忙又來,千萬別爲我誤事。”賈璉道:“也沒甚事,只是偏偏的又出來了一件遠差。出了月就起身,得半月工夫纔來。”尤二姐道:“既如此,你只管放心前去,這裏一應不用你記掛。三妹子他從不會朝更暮改的。他已說了改悔,必是改悔的。他已擇定了人,你只要依他就是了。”賈璉問是誰,尤二姐笑道:“這人此刻不在這裏,不知多早纔來,也難爲他眼力。自己說了,這人一年不來,他等一年;十年不來,等十年;若這人死了再不來了,他情願剃了頭當姑子去,喫長齋唸佛,以了今生。”賈璉問:“到底是誰,這樣動他的心?”二姐笑道:“說來話長。五年前我們老孃家裏做生日,媽和我們到那裏給老孃拜壽。他家請了一起串客,裏頭有個作小生的叫作柳湘蓮,\begin{note}庚雙夾:千奇百怪之文何至於此!\end{note}他看上了,如今要是他才嫁。舊年我們聞得柳湘蓮惹了一個禍逃走了,不知可有來了不曾?”賈璉聽了道:“怪道呢!我說是個什麼樣人,原來是他!果然眼力不錯。你不知道這柳二郎,那樣一個標緻人,最是冷面冷心的,差不多的人,都無情無義。他最和寶玉合的來。去年因打了薛呆子,他不好意思見我們的,不知那裏去了一向。後來聽見有人說來了,不知是真是假。一問寶玉的小子們就知道了。倘或不來,他萍蹤浪跡,知道幾纔來,豈不白耽擱了?”尤二姐道:“我們這三丫頭說的出來,乾的出來,他怎樣說,只依他便了。”
\end{parag}


\begin{parag}
    二人正說之間,只見尤三姐走來說道:“姐夫,你只放心。我們不是那心口兩樣人,說什麼是什麼。若有了姓柳的來,我便嫁他。從今日起,我喫齋唸佛,只伏侍母親,等他來了,嫁了他去,若一百年不來,我自己修行去了。”說著,將一根玉簪,擊作兩段,“一句不真,就如這簪子!”說著,回房去了,真個竟非禮不動,非禮不言起來。賈璉無了法,只得和二姐商議了一回家務,復回家與鳳姐商議起身之事。一面著人問茗煙,茗煙說:“竟不知道。大約未來;若來了,必是我知道的。”一面又問他的街坊,也說未來。賈璉只得回覆了二姐。至起身之日已近,前兩天便說起身,卻先往二姐這邊來住兩夜,從這裏再悄悄長行。果見小妹竟又換了一個人,又見二姐持家勤慎,自是不消記掛。
\end{parag}


\begin{parag}
    是日一早出城,就奔平安州大道,曉行夜住,渴飲飢餐。方走了三日,那日正走之間,頂頭來了一羣馱子,內中一夥,主僕十來騎馬,走的近來一看,不是別人,竟是薛蟠和柳湘蓮來了。賈璉深爲奇怪,\begin{note}庚雙夾:餘亦爲怪。\end{note}忙伸馬迎了上來,大家一齊相見,說些別後寒溫,大家便入酒店歇下,敘談敘談。賈璉因笑說:“鬧過之後,我們忙著請你兩個和解,誰知柳兄蹤跡全無。怎麼你兩個今日倒在一處了?”薛蟠笑道:“天下竟有這樣奇事。我同夥計販了貨物,自春天起身,往回裏走,一路平安。誰知前日到了平安州界,遇一夥強盜,已將東西劫去。不想柳二弟從那邊來了,方把賊人趕散,奪回貨物,還救了我們的性命。我謝他又不受,所以我們結拜了生死弟兄,如今一路進京。從此後我們是親弟親兄一般。到前面岔口上分路,他就分路往南二百里有他一個姑媽,他去望候望候。我先進京去安置了我的事,然後給他尋一所宅子,尋一門好親事,大家過起來。”賈璉聽了道:“原來如此,倒教我們懸了幾日心。”因又聽道尋親,又忙說道:“我正有一門好親事堪配二弟。”說著,便將自己娶尤氏,如今又要發嫁小姨一節說了出來,只不說尤三姐自擇之語。又囑薛蟠且不可告訴家裏,等生了兒子,自然是知道的。薛蟠聽了大喜,說:“早該如此,這都是舍表妹之過。”湘蓮忙笑說:“你又忘情了,還不住口。”薛蟠忙止住不語,便說:“既是這等,這門親事定要做的。”湘蓮道:“我本有願,定要一個絕色的女子。如今既是貴昆仲高誼,顧不得許多了,任憑裁奪,我無不從命。”賈璉笑道:“如今口說無憑,等柳兄一見,便知我這內娣的品貌是古今有一無二的了。”湘蓮聽了大喜,說:“既如此說,等弟探過姑娘,不過月中就進京的,那時再定如何?”賈璉笑道:“你我一言爲定,只是我信不過柳兄。你乃是萍蹤浪跡,倘然淹滯不歸,豈不誤了人家。須得留一定禮。”湘蓮道:“大丈夫豈有失信之理。小弟素系寒貧,況且客中,何能有定禮。”薛蟠道:“我這裏現成,就備一分二哥帶去。”賈璉笑道:“也不用金帛之禮,須是柳兄親身自有之物,不論物之貴賤,不過我帶去取信耳。”湘蓮道:“既如此說,弟無別物,此劍防身,不能解下。囊中尚有一把鴛鴦劍,乃吾家傳代之寶,弟也不敢擅用,只隨身收藏而已。賈兄請拿去爲定。弟縱系水流花落之性,然亦斷不捨此劍者。”說畢,大家又飲了幾杯,方各自上馬,作別起程。正是:將軍不下馬,各自奔前程。
\end{parag}


\begin{parag}
    且說賈璉一日到了平安州,見了節度,完了公事。因又囑他十月前後務要還來一次,賈璉領命。次日連忙取路回家,先到尤二姐處探望。誰知賈璉出門之後,尤二姐操持家務十分謹肅,每日關門合戶,一點外事不聞。他小妹子果是個斬釘截鐵之人,每日侍奉母姊之餘,只安分守已,隨分過活。雖是夜晚間孤衾獨枕,不慣寂寞,奈一心丟了衆人,只念柳湘蓮早早回來完了終身大事。這日賈璉進門,見了這般景況,喜之不盡,深念二姐之德。大家敘些寒溫之後,賈璉便將路上相遇湘蓮一事說了出來,又將鴛鴦劍取出,遞與三姐。三姐看時,上面龍吞夔護,珠寶晶瑩,將靶一掣,裏面卻是兩把合體的。一把上面鏨著一“鴛”字,一把上面鏨著一“鴦”字,冷颼颼,明亮亮,如兩痕秋水一般。三姐喜出望外,連忙收了,掛在自己繡房牀上,每日望著劍,自笑終身有靠。賈璉住了兩天,回去復了父命,回家合宅相見。那時鳳姐已大愈,出來理事行走了。賈璉又將此事告訴了賈珍。賈珍因近日又遇了新友,將這事丟過,不在心上,任憑賈璉裁奪,只怕賈璉獨力不加,少不得又給了他三十兩銀子。賈璉拿來交與二姐預備妝奩。
\end{parag}


\begin{parag}
    誰知八月內湘蓮方進了京,先來拜見薛姨媽,又遇見薛蝌,方知薛蟠不慣風霜,不服水土,一進京時便病倒在家,請醫調治。聽見湘蓮來了,請入臥室相見。薛姨媽也不念舊事,只感新恩,母子們十分稱謝。又說起親事一節,凡一應東西皆已妥當,只等擇日。柳湘蓮也感激不盡。
\end{parag}


\begin{parag}
    次日又來見寶玉,二人相會,如魚得水。湘蓮因問賈璉偷娶二房之事,寶玉笑道:“我聽見茗煙一干人說,我卻未見,我也不敢多管。我又聽見茗煙說,璉二哥哥著實問你,不知有何話說?”湘蓮就將路上所有之事一概告訴寶玉,寶玉笑道:“大喜,大喜!難得這個標緻人,果然是個古今絕色,堪配你之爲人。”湘蓮道: “既是這樣,他那裏少了人物,如何只想到我。況且我又素日不甚和他厚,也關切不至此。路上工夫忙忙的就那樣再三要來定,難道女家反趕著男家不成。我自己疑惑起來,後悔不該留下這劍作定。所以後來想起你來,可以細細問個底裏纔好。”寶玉道:“你原是個精細人,如何既許了定禮又疑惑起來?你原說只要一個絕色的,如今既得了個絕色便罷了,何必再疑?”湘蓮道:“你既不知他娶,如何又知是絕色?”寶玉道:“他是珍大嫂子的繼母帶來的兩位小姨。我在那裏和他們混了一個月,怎麼不知?真真一對尤物,他又姓尤。”湘蓮聽了,跌足道:“這事不好,斷乎做不得了。你們東府裏除了那兩個石頭獅子乾淨,只怕連貓兒狗兒都不乾淨。我不做這剩忘八。”\begin{note}庚雙夾:奇極之文!趣極之文!《金瓶梅》中有云“把忘八的臉打綠了”,已奇之至,此雲“剩忘八”,豈不更奇!\end{note}寶玉聽說,紅了臉。湘蓮自慚失言,連忙作揖說:“我該死胡說。\begin{note}庚雙夾:忽用湘蓮提東府之事罵及寶玉,可是人想得到的?所謂“一個人不曾放過”。\end{note}你好歹告訴我,他品行如何?”寶玉笑道:“你既深知,又來問我作甚麼?連我也未必乾淨了。”湘蓮笑道:“原是我自己一時忘情,好歹別多心。”寶玉笑道:“何必再提,這倒是有心了。”湘蓮作揖告辭出來,若去找薛蟠,一則他現臥病,二則他又浮躁,不如去索回定禮。主意已定,便一徑來找賈璉。
\end{parag}


\begin{parag}
    賈璉正在新房中,聞得湘蓮來了,喜之不禁,忙迎了出來,讓到內室與尤老相見。湘蓮只作揖稱老伯母,自稱晚生,賈璉聽了詫異。喫茶之間,湘蓮便說:“客中偶然忙促,誰知家姑母於四月間訂了弟婦,使弟無言可回。若從了老兄背了姑母,似非合理。若系金帛之訂,弟不敢索取,但此劍系祖父所遺,請仍賜回爲幸。” 賈璉聽了,便不自在,還說:“定者,定也。原怕反悔所以爲定。豈有婚姻之事,出入隨意的?還要斟酌。”湘蓮笑道:“雖如此說,弟願領責領罰,然此事斷不敢從命。”賈璉還要饒舌,湘蓮便起身說:“請兄外坐一敘,此處不便。”那尤三姐在房明明聽見。好容易等了他來,今忽見反悔,便知他在賈府中得了消息,自然是嫌自己淫奔無恥之流,不屑爲妻。今若容他出去和賈璉說退親,料那賈璉必無法可處,自己豈不無趣。一聽賈璉要同他出去,連忙摘下劍來,將一股雌鋒隱在肘內,出來便說:“你們不必出去再議,還你的定禮。”一面淚如雨下,左手將劍並鞘送與湘蓮,右手回肘只往項上一橫。可憐“揉碎桃花紅滿地,玉山傾倒再難扶”,芳靈蕙性,渺渺冥冥,不知那邊去了。當下唬得衆人急救不迭。尤老一面嚎哭,一面又罵湘蓮。賈璉忙揪住湘蓮,命人捆了送官。尤二姐忙止淚反勸賈璉:“你太多事,人家並沒威逼他死,是他自尋短見。你便送他到官,又有何益,反覺生事出醜。不如放他去罷,豈不省事。”賈璉此時也沒了主意,便放了手命湘蓮快去。湘蓮反不動身,泣道:“我並不知是這等剛烈賢妻,可敬,可敬。”湘蓮反扶屍大哭一場。等買了棺木,眼見入殮,又俯棺大哭一場,方告辭而去。
\end{parag}


\begin{parag}
    出門無所之,昏昏默默,自想方纔之事。原來尤三姐這樣標緻,又這等剛烈,自悔不及。正走之間,只見薛蟠的小廝尋他家去,那湘蓮只管出神。那小廝帶他到新房之中,十分齊整。忽聽環珮叮噹,尤三姐從外而入,一手捧著鴛鴦劍,一手捧著一卷冊子,向柳湘蓮泣道:“妾癡情待君五年矣,不期君果冷心冷面,妾以死報此癡情。妾今奉警幻之命,前往太虛幻境修注案中所有一干情鬼。妾不忍一別,故來一會,從此再不能相見矣。”說著便走。湘蓮不捨,忙欲上來拉住問時,那尤三姐便說:“來自情天,去由情地。前生誤被情惑,今既恥情而覺,與君兩無干涉。”說畢,一陣香風,無蹤無影去了。
\end{parag}


\begin{parag}
    湘蓮警覺,似夢非夢,睜眼看時,那裏有薛家小童,也非新室,竟是一座破廟,旁邊坐著一個跏腿道士捕蝨。湘蓮便起身稽首相問:“此係何方?仙師仙名法號?”道士笑道:“連我也不知道此係何方,我係何人,不過暫來歇足而已。”柳湘蓮聽了,不覺冷然如寒冰侵骨,掣出那股雄劍,將萬根煩惱絲一揮而盡,便隨那道士,不知往那裏去了。後回便見
\end{parag}


\begin{parag}
    \begin{note}蒙回後總評:尤三姐失身時,濃妝豔抹凌辱羣兇;擇夫後,唸佛喫齋敬奉老母;能辨寶玉能識湘蓮,活是紅拂文君一流人物。\end{note}
\end{parag}


\begin{parag}
    \begin{note}蒙回後總評:鴛鴦劍能斬鴛鴦,鴛鴦人能破鴛鴦。豈有此理?鴛鴦劍夢裏不會殺姦婦,鴛鴦人白日偏要助淫夫。焉有此情?真天地間不測的怪事!\end{note}
\end{parag}
