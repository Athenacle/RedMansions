\chap{五十二}{俏平兒情掩蝦鬚鐲 勇晴雯病補雀金裘}


\begin{parag}
    \begin{note}蒙回前總:寫黛玉弱症的是弱症,寫晴雯時症的是時症,寫湘雲性快的是快性,寫晴雯性傲的是傲性。彼何人斯?而有肖物手段。\end{note}
\end{parag}


\begin{parag}
    賈母道:“正是這話了。上次我要說這話,我見你們的大事多,如今又添出這些事來,你們固然不敢抱怨,未免想著我只顧疼這些小孫子孫女兒們,就不體貼你們這當家人了。你既這麼說出來,更好了。”因此時薛姨媽李嬸都在座,邢夫人及尤氏婆媳也都過來請安,還未過去,賈母向王夫人等說道:“今兒我才說這話,素日我不說,一則怕逞了鳳丫頭的臉,二則衆人不伏。今日你們都在這裏,都是經過妯娌姑嫂的,還有他這樣想的到的沒有?”薛姨媽、李嬸、尤氏等齊笑說:“真個少有。別人不過是禮上面子情兒,實在他是真疼小叔子小姑子。就是老太太跟前,也是真孝順。”賈母點頭嘆道:“我雖疼他,我又怕他太伶俐也不是好事。”鳳姐兒忙笑道:“這話老祖宗說差了。世人都說太伶俐聰明,怕活不長。世人都說得,人人都信,獨老祖宗不當說,不當信。老祖宗只有伶俐聰明過我十倍的,怎麼如今這樣福壽雙全的?只怕我明兒還勝老祖宗一倍呢!我活一千歲後,等老祖宗歸了西,我才死呢。”賈母笑道:“衆人都死了,單剩下咱們兩個老妖精,有什麼意思。”說的衆人都笑了。
\end{parag}


\begin{parag}
    寶玉因記掛著晴雯襲人等事,便先回園裏來。到房中,藥香滿屋,一人不見,只見晴雯獨臥於炕上,臉面燒的飛紅,又摸了一摸,只覺燙手。忙又向爐上將手烘暖,伸進被去摸了一摸身上,也是火燒。因說道:“別人去了也罷,麝月秋紋也這樣無情,各自去了?”晴雯道:“秋紋是我攆了他去喫飯的,麝月是方纔平兒來找他出去了。兩人鬼鬼祟祟的,不知說什麼。必是說我病了不出去。”寶玉道:“平兒不是那樣人。況且他並不知你病特來瞧你,想來一定是找麝月來說話,偶然見你病了,隨口說特瞧你的病,這也是人情乖覺取和的常事。便不出去,有不是,與他何干?你們素日又好,斷不肯爲這無干的事傷和氣。”晴雯道:“這話也是,只是疑他爲什麼忽然間瞞起我來。”\begin{note}庚雙夾:寶玉一篇推情度理之談以射正事,不知何如。\end{note}寶玉笑道:“讓我從後門出去,到那窗根下聽聽說些什麼,來告訴你。”說著,果然從後門出去,至窗下潛聽。
\end{parag}


\begin{parag}
    只聞麝月悄問道:“你怎麼就得了的?”\begin{note}庚雙夾:妙!這纔有神理,是平兒說過一半了。若此時從寶玉口中從頭說起一原一故,直是二人特等寶玉來聽方說起也。\end{note}平兒道:“那日洗手時不見了,二奶奶就不許吵嚷,出了園子,即刻就傳給園裏各處的媽媽們小心查訪。我們只疑惑邢姑娘的丫頭,本來又窮,只怕小孩子家沒見過,拿了起來也是有的。再不料定是你們這裏的。幸而二奶奶沒有在屋裏,你們這裏的宋媽媽去了,拿著這支鐲子,說是小丫頭子墜兒偷起來的,被他看見,來回二奶奶的。\begin{note}庚雙夾:妙極!紅玉既有歸結,墜兒豈可不表哉?可知“奸賊”二字是相連的。故“情”字原非正道,墜兒原不情也,不過一愚人耳,可以傳奸即可以爲盜。二次小竊皆出於寶玉房中,亦大有深意在焉。\end{note}我趕著忙接了鐲子,想了一想:寶玉是偏在你們身上留心用意、爭勝要強的,那一年有一個良兒偷玉,剛冷了一二年間,還有人提起來趁願,這會子又跑出一個偷金子的來了。而且更偷到街坊家去了。偏是他這樣,偏是他的人打嘴。所以我倒忙叮嚀宋媽,千萬別告訴寶玉,只當沒有這事,別和一個人提起。第二件,老太太、太太聽了也生氣。三則襲人和你們也不好看。所以我回二奶奶,只說:‘我往大奶奶那裏去的,誰知鐲子褪了口,丟在草根底下,雪深了沒看見。今兒雪化盡了,黃澄澄的映著日頭,還在那裏呢,我就揀了起來。’二奶奶也就信了,所以我來告訴你們。你們以後防著他些,別使喚他到別處去。等襲人回來,你們商議著,變個法子打發出去就完了。”麝月道:“這小娼婦也見過些東西,怎麼這麼眼皮子淺。”平兒道:“究竟這鐲子能多少重,原是二奶奶說的,這叫做‘蝦鬚鐲’,倒是這顆珠子還罷了。晴雯那蹄子是塊爆炭,要告訴了他,他是忍不住的。一時氣了,或打或罵,依舊嚷出來不好,所以單告訴你留心就是了。”說著便作辭而去。
\end{parag}


\begin{parag}
    寶玉聽了,又喜又氣又嘆。喜的是平兒竟能體貼自己;氣的是墜兒小竊;嘆的是墜兒那樣一個伶俐人,作出這醜事來。因而回至房中,把平兒之話一長一短告訴了晴雯。又說:“他說你是個要強的,如今病著,聽了這話越發要添病,等好了再告訴你。”晴雯聽了,果然氣的蛾眉倒蹙,鳳眼圓睜,即時就叫墜兒。寶玉忙勸道:“你這一喊出來,豈不辜負了平兒待你我之心了。不如領他這個情,過後打發他就完了。”晴雯道:“雖如此說,只是這口氣如何忍得!”寶玉道:“這有什麼氣的?你只養病就是了。”
\end{parag}


\begin{parag}
    晴雯服了藥,至晚間又服二和,夜間雖有些汗,還未見效,仍是發燒,頭疼鼻塞聲重。次日,王太醫又來診視,另加減湯劑。雖然稍減了燒,仍是頭疼。寶玉便命麝月:“取鼻菸來,給他嗅些,痛打幾個嚏噴,就通了關竅。”麝月果真去取了一個金鑲雙扣金星玻璃的一個扁盒來,遞與寶玉。寶玉便揭翻盒扇,裏面有西洋琺琅的黃髮赤身女子,兩肋又有肉翅,裏面盛著些真正汪恰洋菸。\begin{note}庚雙夾:“汪恰”,西洋一等寶煙也。\end{note}晴雯只顧看畫兒,寶玉道:“嗅些,走了氣就不好了。”晴雯聽說,忙用指甲挑了些嗅入鼻中,不怎樣。便又多多挑了些嗅入。忽覺鼻中一股酸辣透入囟門,接連打了五六個嚏噴,眼淚鼻涕登時齊流。\begin{note}庚雙夾:寫得出。\end{note}晴雯忙收了盒子,笑道:“了不得,好辣!快拿紙來。”早有小丫頭子遞過一搭子細紙,晴雯便一張一張的拿來醒鼻子。寶玉笑問:“如何?”晴雯笑道:“果覺通快些,只是太陽還疼。”寶玉笑道:“越性盡用西洋藥治一治,只怕就好了。”說著,便命麝月:“和二奶奶要去,就說我說了:姐姐那裏常有那西洋貼頭疼的膏子藥,叫做‘依弗哪’,找尋一點兒。”麝月答應了,去了半日,果拿了半節來。便去找了一塊紅緞子角兒,鉸了兩塊指頂大的圓式,將那藥烤和了,用簪挺攤上。晴雯自拿著一面靶鏡,貼在兩太陽上。麝月笑道:“病的蓬頭鬼一樣,如今貼了這個,倒俏皮了。二奶奶貼慣了,倒不大顯。”說畢,又向寶玉道: “二奶奶說了:明日是舅老爺生日,太太說了叫你去呢。明兒穿什麼衣裳?今兒晚上好打點齊備了,省得明兒早起費手。”寶玉道:“什麼順手就是什麼罷了。一年鬧生日也鬧不清。”說著,便起身出房,往惜春房中去看畫。
\end{parag}


\begin{parag}
    剛到院門外邊,忽見寶琴的小丫鬟名小螺者從那邊過去,寶玉忙趕上問:“那去?”小螺笑道:“我們二位姑娘都在林姑娘房裏呢,我如今也往那裏去。”寶玉聽了,轉步也便同他往瀟湘館來。不但寶釵姊妹在此,且連邢岫煙也在那裏,四人圍坐在熏籠上敘家常。紫鵑倒坐在暖閣裏,臨窗作針黹。一見他來,都笑說:“又來了一個!可沒了你的坐處了。”寶玉笑道:“好一副‘冬閨集豔圖’!可惜我遲來了一步。橫豎這屋子比各屋子暖,這椅子上坐著並不冷。”說著,便坐在黛玉常坐的搭著灰鼠椅搭一張椅上。因見暖閣之中有一玉石條盆,裏面攢三聚五栽著一盆單瓣水仙,點著宣石,便極口贊:“好花!這屋子越發暖,這花香的越清香。昨日未見。”黛玉因說道:“這是你家的大總管賴大嬸子送薛二姑娘的,兩盆臘梅、兩盆水仙。他送了我一盆水仙,他送了蕉丫頭一盆臘梅。我原不要的,又恐辜負了他的心。你若要,我轉送你如何?”寶玉道:“我屋裏卻有兩盆,只是不及這個。琴妹妹送你的,如何又轉送人,這個斷使不得。”黛玉道:“我一日藥吊子不離火,我竟是藥培著呢,那裏還擱的住花香來燻?越發弱了。況且這屋子裏一股藥香,反把這花香攪壞了。不如你抬了去,這花也清淨了,沒雜味來攪他。”寶玉笑道: “我屋裏今兒也有病人煎藥呢,你怎麼知道的?”黛玉笑道:“這話奇了,我原是無心的話,誰知你屋裏的事?你不早來聽說古記,這會子來了,自驚自怪的。”
\end{parag}


\begin{parag}
    寶玉笑道:“咱們明兒下一社又有了題目了,就詠水仙臘梅。”黛玉聽了,笑道:“罷,罷!我再不敢作詩了,作一回,罰一回,沒的怪羞的。”說著,便兩手握起臉來。寶玉笑道:“何苦來!又奚落我作什麼。我還不怕臊呢,你倒握起臉來了。”寶釵因笑道:“下次我邀一社,四個詩題,四個詞題。每人四首詩,四闋詞。頭一個詩題《詠〈太極圖〉》,限一先的韻,五言律,要把一先的韻都用盡了,一個不許剩。” 寶琴笑道:“這一說,可知是姐姐不是真心起社了,這分明難人。若論起來,也強扭的出來,不過顛來倒去弄些《易經》上的話生填,究竟有何趣味。我八歲時節,跟我父親到西海沿子上買洋貨,誰知有個真真國的女孩子,才十五歲,那臉面就和那西洋畫上的美人一樣,也披著黃頭髮,打著聯垂,滿頭帶的都是珊瑚、貓兒眼、祖母綠這些寶石;身上穿著金絲織的鎖子甲洋錦襖袖;帶著倭刀,也是鑲金嵌寶的,實在畫兒上的也沒他好看。有人說他通中國的詩書,會講五經,能作詩填詞,因此我父親央煩了一位通事官,煩他寫了一張字,就寫的是他作的詩。”衆人都稱奇道異。寶玉忙笑道:“好妹妹,你拿出來我瞧瞧。”寶琴笑道:“在南京收著呢,此時那裏去取來?”寶玉聽了,大失所望,便說:“沒福得見這世面。”黛玉笑拉寶琴道:“你別哄我們。我知道你這一來,你的這些東西未必放在家裏,自然都是要帶了來的,這會子又扯謊說沒帶來。他們雖信,我是不信的。”寶琴便紅了臉,低頭微笑不語。寶釵笑道:“偏這個顰兒慣說這些白話,把你就伶俐的。”黛玉道:“若帶了來,就給我們見識見識也罷了。”寶釵笑道:“箱子籠子一大堆還沒理清,知道在那個裏頭呢!等過日收拾清了,找出來大家再看就是了。”又向寶琴道:“你若記得,何不念念我們聽聽?”寶琴方答道:“記得是首五言律,外國的女子也就難爲他了。”寶釵道:“你且別唸,等把雲兒叫了來,也叫他聽聽。”說著,便叫小螺來吩咐道:“你到我那裏去,就說我們這裏有一個外國美人來了,作的好詩,請你這‘詩瘋子’來瞧去,再把我們‘詩呆子’也帶來。”小螺笑著去了。
\end{parag}


\begin{parag}
    半日,只聽湘雲笑問:“那一個外國美人來了?”一頭說,一頭果和香菱來了。衆人笑道:“人未見形,先已聞聲。”寶琴等忙讓坐,遂把方纔的話重敘了一遍。湘雲笑道:“快念來聽聽。”寶琴因念道:
\end{parag}


\begin{poem}
    \begin{pl}昨夜朱樓夢,今宵水國吟。\end{pl}

    \begin{pl}島雲蒸大海,嵐氣接叢林。\end{pl}

    \begin{pl}月本無今古,情緣自淺深。\end{pl}

    \begin{pl}漢南春歷歷,焉得不關心。\end{pl}


\end{poem}


\begin{parag}
    衆人聽了,都道:“難爲他!竟比我們中國人還強。”一語未了,只見麝月走來說:“太太打發人來告訴二爺,明兒一早往舅舅那裏去,就說太太身上不大好,不得親自來。”寶玉忙站起來答應道:“是。”因問寶釵寶琴可去。寶釵道:“我們不去。昨兒單送了禮去了。”大家說了一回方散。
\end{parag}


\begin{parag}
    寶玉因讓諸姊妹先行,自己落後。黛玉便又叫住他問道:“襲人到底多早晚回來。”寶玉道:“自然等送了殯纔來呢。”黛玉還有話說,又不曾出口,出了一回神,便說道:“你去罷。” 寶玉也覺心裏有許多話,只是口裏不知要說什麼,想了一想,也笑道:“明日再說罷。”一面下了階磯,低頭正欲邁步,復又忙回身問道:“如今的夜越發長了,你一夜咳嗽幾遍?醒幾次?”\begin{note}庚雙夾:此皆好笑之極,無味扯淡之極,回思則瀝血滴髓之至情至神也。豈別部偷寒送暖私奔暗約一味淫情浪態之小說可比哉?\end{note}黛玉道:“昨兒夜裏好了,只嗽兩遍,卻只睡了四更一個更次,就再不能睡了。”寶玉又笑道:“正是有句要緊的話,這會子纔想起來。”一面說,一面便捱過身來,悄悄道:“我想寶姐姐送你的燕窩──”一語未了,只見趙姨娘走了進來瞧黛玉,問:“姑娘這兩天好?”黛玉便知他是從探春處來,從門前過,順路的人情。黛玉忙陪笑讓坐,說:“難得姨娘想著,怪冷的,親自走來。”又忙命倒茶,一面又使眼色與寶玉。寶玉會意,便走了出來。
\end{parag}


\begin{parag}
    正值喫晚飯時,見了王夫人,王夫人又囑咐他早去。寶玉回來,看晴雯吃了藥。此夕寶玉便不命晴雯挪出暖閣來,自己便在晴雯外邊。又命將熏籠抬至暖閣前,麝月便在薰籠上。一宿無話。
\end{parag}


\begin{parag}
    至次日,天未明時,晴雯便叫醒麝月道:“你也該醒了,只是睡不夠!你出去叫人給他預備茶水,我叫醒他就是了。”麝月忙披衣起來道:“咱們叫起他來,穿好衣裳,抬過這火箱去,再叫他們進來。老嬤嬤們已經說過,不叫他在這屋裏,怕過了病氣。如今他們見咱們擠在一處,又該嘮叨了。”晴雯道:“我也是這麼說呢。”二人才叫時,寶玉已醒了,忙起身披衣。麝月先叫進小丫頭子來,收拾妥當了,才命秋紋檀雲等進來,一同伏侍寶玉梳洗畢。麝月道:“天又陰陰的,只怕有雪,穿那一套氈的罷。”寶玉點頭,即時換了衣裳。小丫頭便用小茶盤捧了一蓋碗建蓮紅棗兒湯來,寶玉喝了兩口。麝月又捧過一小碟法制紫薑來,寶玉噙了一塊。又囑咐了晴雯一回,便往賈母處來。
\end{parag}


\begin{parag}
    賈母猶未起來,知道寶玉出門,便開了房門,命寶玉進去。寶玉見賈母身後寶琴面向裏也睡未醒。賈母見寶玉身上穿著荔色哆羅呢的天馬箭袖,大紅猩猩氈盤金彩繡石青妝緞沿邊的排穗褂子。賈母道:“下雪呢麼?”寶玉道:“天陰著,還沒下呢!”賈母便命鴛鴦來:“把昨兒那一件烏雲豹的氅衣給他罷。”鴛鴦答應了,走去果取了一件來。寶玉看時,金翠輝煌,碧彩閃灼,又不似寶琴所披之鳧靨裘。只聽賈母笑道:“這叫作‘雀金呢 ’,這是哦囉斯國拿孔雀毛拈了線織的。前兒把那一件野鴨子的給了你小妹妹,\begin{note}庚雙夾:“小”字妙!蓋王夫人之末女也。\end{note}這件給你罷。”寶玉磕了一個頭,便披在身上。賈母笑道:“你先給你娘瞧瞧去再去。”寶玉答應了,便出來,只見鴛鴦站在地下揉眼睛。因自那日鴛鴦發誓決絕之後,他總不和寶玉講話。寶玉正自日夜不安,此時見他又要回避,寶玉便上來笑道:“好姐姐,你瞧瞧,我穿著這個好不好。”鴛鴦一摔手,便進賈母房中來了。寶玉只得到了王夫人房中,與王夫人看了,然後又回至園中,與晴雯麝月看過後,至賈母房中回說:“太太看了,只說可惜了的,叫我仔細穿,別遭踏了他。”賈母道:“就剩下了這一件,你遭踏了也再沒了。這會子特給你做這個也是沒有的事。”說著又囑咐他:“不許多喫酒,早些回來。”寶玉應了幾個“是”。
\end{parag}


\begin{parag}
    老嬤嬤跟至廳上,只見寶玉的奶兄李貴和王榮、張若錦、趙亦華、錢啓、周瑞六個人,帶著茗煙、伴鶴、鋤藥、掃紅四個小廝,背著衣包,抱著坐褥,籠著一匹雕鞍彩轡的白馬,早已伺候多時了。老嬤嬤又吩咐了他六人些話,六個人忙答應了幾個“是”,忙捧鞭墜鐙。寶玉慢慢的上了馬,李貴和王榮籠著嚼環,錢啓周瑞二人在前引導,張若錦、趙亦華在兩邊緊貼寶玉後身。寶玉在馬上笑道:“周哥,錢哥,咱們打這角門走罷,省得到了老爺的書房門口又下來。”周瑞側身笑道:“老爺不在家,書房天天鎖著的,爺可以不用下來罷了。”寶玉笑道:“雖鎖著,也要下來的。”錢啓李貴等都笑道:“爺說的是。便託懶不下來,倘或遇見賴大爺林二爺,雖不好說爺,也勸兩句。有的不是,都派在我們身上,又說我們不教爺禮了。”周瑞錢啓便一直出角門來。
\end{parag}


\begin{parag}
    正說話時,頂頭果見賴大進來。寶玉忙籠住馬,意欲下來。賴大忙上來抱住腿。寶玉便在鐙上站起來,笑攜他的手,說了幾句話。接著又見一個小廝帶著二三十個拿掃帚簸箕的人進來,見了寶玉,都順牆垂手立住,獨那爲首的小廝打千兒,請了一個安。寶玉不識名姓,只微笑點了點頭兒。馬已過去,\begin{note}庚雙夾:總爲後文伏線。\end{note}那人方帶人去了。於是出了角門,門外又有李貴等六人的小廝並幾個馬伕,早預備下十來匹馬專候。一出了角門,李貴等都各上了馬,前引傍圍的一陣煙去了,不在話下。
\end{parag}


\begin{parag}
    這裏晴雯吃了藥,仍不見病退,急的亂罵大夫,說:“只會騙人的錢,一劑好藥也不給人喫。”\begin{note}庚雙夾:奇文。真嬌憨女兒之語也。\end{note}麝月笑勸他道: “你太性急了,俗語說:‘病來如山倒,病去如抽絲。’又不是老君的仙丹,那有這樣靈藥!你只靜養幾天,自然好了。你越急越著手。”晴雯又罵小丫頭子們: “那裏鑽沙去了!瞅我病了,都大膽子走了。明兒我好了,一個一個的才揭你們的皮呢!”唬的小丫頭子篆兒忙進來問:“姑娘作什麼?”\begin{note}庚雙夾:此“姑娘”亦“姑姑”“娘娘”之稱,亦如賈璉處小廝呼平兒,皆南北互用一語也。脂硯。\end{note}晴雯道:“別人都死絕了,就剩了你不成?”說著,只見墜兒也蹭了進來。晴雯道:“你瞧瞧這小蹄子,不問他還不來呢。這裏又放月錢了,又散果子了,你該跑在頭裏了。你往前些,我不是老虎吃了你!”墜兒只得前湊。晴雯便冷不防欠身一把將他的手抓住,\begin{note}庚雙夾:是病臥之時。\end{note}向枕邊取了一丈青,向他手上亂戳,口內罵道:“要這爪子作什麼?拈不得針,拿不動線,只會偷嘴喫。眼皮子又淺,爪子又輕,打嘴現世的,不如戳爛了!”墜兒疼的亂哭亂喊。麝月忙拉開墜兒,按晴雯睡下,笑道:“纔出了汗,又作死。等你好了,要打多少打不的?這會子鬧什麼!”晴雯便命人叫宋嬤嬤進來,說道:“寶二爺才告訴了我,叫我告訴你們,墜兒很懶,寶二爺當面使他,他撥嘴兒不動,連襲人使他,他背後罵他。今兒務必打發他出去,明兒寶二爺親自回太太就是了。”宋嬤嬤聽了,心下便知鐲子事發,因笑道:“雖如此說,也等花姑娘回來知道了,再打發他。”晴雯道:“寶二爺今兒千叮嚀萬囑咐的,什麼‘花姑娘’‘草姑娘’,我們自然有道理。你只依我的話,快叫他家的人來領他出去。”麝月道:“這也罷了。早也去,晚也去,帶了去早清淨一日。”
\end{parag}


\begin{parag}
    宋嬤嬤聽了,只得出去喚了他母親來,打點了他的東西,又來見晴雯等,說道:“姑娘們怎麼了,你侄女兒不好,\begin{note}庚雙夾:“侄女”二字妙,餘前注不謬。\end{note}你們教導他,怎麼攆出去?也到底給我們留個臉兒。”晴雯道:“你這話只等寶玉來問他,與我們無干。”那媳婦冷笑道:“我有膽子問他去!他那一件事不是聽姑娘們的調停?他縱依了,姑娘們不依,也未必中用。比如方纔說話,雖是背地裏,姑娘就直叫他的名字。在姑娘們就使得,在我們就成了野人了。”晴雯聽說,一發急紅了臉,說道:“我叫了他的名字了,你在老太太跟前告我去,說我撒野,也攆出我去。”麝月忙道:“嫂子,你只管帶了人出去,有話再說。這個地方豈有你叫喊講禮的?你見誰和我們講過禮?別說嫂子你,就是賴奶奶林大娘,也得擔待我們三分。便是叫名字,從小兒直到如今,都是老太太吩咐過的,你們也知道的,恐怕難養活,巴巴的寫了他的小名兒,各處貼著叫萬人叫去,爲的是好養活。連挑水挑糞花子都叫得,何況我們!連昨兒林大娘叫了一聲‘爺’,老太太還說他呢,此是一件。二則,我們這些人常回老太太的話去,可不叫著名字回話,難道也稱‘爺’?那一日不把寶玉兩個字念二百遍,偏嫂子又來挑這個了!過一日嫂子閒了,在老太太、太太跟前,聽聽我們當著面兒叫他就知道了。嫂子原也不得在老太太、太太跟前當些體統差事,成年家只在三門外頭混,怪不得不知我們裏頭的規矩。這裏不是嫂子久站的,再一會,不用我們說話,就有人來問你了。有什麼分證話,且帶了他去,你回了林大娘,叫他來找二爺說話。家裏上千的人,你也跑來,我也跑來,我們認人問姓,還認不清呢!”說著,便叫小丫頭子:“拿了擦地的布來擦地!”那媳婦聽了,無言可對,亦不敢久立,賭氣帶了墜兒就走。宋媽媽忙道:“怪道你這嫂子不知規矩,你女兒在這屋裏一場,臨去時,也給姑娘們磕個頭。沒有別的謝禮,──便有謝禮,他們也不希罕,──不過磕個頭,盡了心。怎麼說走就走?”墜兒聽了,只得翻身進來,給他兩個磕了兩個頭,又找秋紋等。他們也不睬他。那媳婦嗐聲嘆氣,口不敢言,抱恨而去。
\end{parag}


\begin{parag}
    晴雯方纔又閃了風,著了氣,反覺更不好了,翻騰至掌燈,剛安靜了些。只見寶玉回來,進門就嗐聲跺腳。麝月忙問原故,寶玉道:“今兒老太太喜喜歡歡的給了這個褂子,誰知不防後襟子上燒了一塊,幸而天晚了,老太太、太太都不理論。”一面說,一面脫下來。麝月瞧時,果見有指頂大的燒眼,說:“這必定是手爐裏的火迸上了。這不值什麼,趕著叫人悄悄的拿出去,叫個能幹織補匠人織上就是了。”說著便用包袱包了,交與一個媽媽送出去。說:“趕天亮就有才好。千萬別給老太太、太太知道。”婆子去了半日,仍舊拿回來,說:“不但能幹織補匠人,就連裁縫繡匠並作女工的問了,都不認得這是什麼,都不敢攬。”麝月道:“這怎麼樣呢!明兒不穿也罷了。”寶玉道:“明兒是正日子,老太太、太太說了,還叫穿這個去呢。偏頭一日燒了,豈不掃興。”晴雯聽了半日,忍不住翻身說道:“拿來我瞧瞧罷。沒個福氣穿就罷了。這會子又著急。”寶玉笑道:“這話倒說的是。”說著,便遞與晴雯,又移過燈來,細看了一會。晴雯道:“這是孔雀金線織的,如今咱們也拿孔雀金線就象界線似的界密了,只怕還可混得過去。”麝月笑道:“孔雀線現成的,但這裏除了你,還有誰會界線?”晴雯道:“說不得,我掙命罷了。”寶玉忙道:“這如何使得!纔好了些,如何做得活。”晴雯道:“不用你蠍蠍螫螫的,我自知道。”一面說,一面坐起來,挽了一挽頭髮,披了衣裳,只覺頭重身輕,滿眼金星亂迸,實實撐不住。若不做,又怕寶玉著急,少不得恨命咬牙挨著。便命麝月只幫著拈線。晴雯先拿了一根比一比,笑道:“這雖不很象,若補上,也不很顯。”寶玉道:“這就很好,那裏又找囉嘶國的裁縫去。”晴雯先將裏子拆開,用茶杯口大的一個竹弓釘牢在背面,再將破口四邊用金刀刮的散鬆鬆的,然後用針紉了兩條,分出經緯,亦如界線之法,先界出地子後,依本衣之紋來回織補。補兩針,又看看,織補兩針,又端詳端詳。無奈頭暈眼黑,氣喘神虛,補不上三五針,伏在枕上歇一會。寶玉在旁,一時又問:“喫些滾水不喫?”一時又命:“歇一歇。”一時又拿一件灰鼠斗篷替他披在背上,一時又命拿個拐枕與他靠著。急的晴雯央道:“小祖宗!你只管睡罷。再熬上半夜,明兒把眼睛摳摟了,怎麼處!”寶玉見他著急,只得胡亂睡下,仍睡不著。一時只聽自鳴鐘已敲了四下,\begin{note}庚雙夾:按“四下”乃寅正初刻,“寅”此樣寫法,避諱也。\end{note}剛剛補完;又用小牙刷慢慢的剔出絨毛來。麝月道:“這就很好,若不留心,再看不出的。”寶玉忙要了瞧瞧,說道:“真真一樣了。”晴雯已嗽了幾陣,好容易補完了,說了一聲:“補雖補了,到底不象,我也再不能了!”噯喲了一聲,便身不由主倒下。要知端的,且聽下回分解。
\end{parag}
