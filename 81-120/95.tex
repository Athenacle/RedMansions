\chap{九十五}{因訛成實元妃薨逝 以假混真寶玉瘋顛}



\begin{parag}
    話說焙茗在門口和小丫頭子說寶玉的玉有了,那小丫頭急忙回來告訴寶玉。衆人聽了,都推着寶玉出去問他,衆人在廊下聽着。寶玉也覺放心,便走到門口問道:“你那裏得了?快拿來。”焙茗道:“拿是拿不來的,還得託人做保去呢。”寶玉道:“你快說是怎麼得的,我好叫人取去。”焙茗道:“我在外頭知道林爺爺去測字,我就跟了去。我聽見說在當鋪裏找,我沒等他說完,便跑到幾個當鋪裏去。我比給他們瞧,有一家便說有。我說給我罷,那鋪子裏要票子。我說當多少錢,他說三百錢的也有,五百錢的也有。前兒有一個人拿這麼一塊玉當了三百錢去,今兒又有人也拿了一塊玉當了五百錢去。”寶玉不等說完,便道:“你快拿三百五百錢去取了來,我們挑着看是不是。”裏頭襲人便啐道:“二爺不用理他。我小時候兒聽見我哥哥常說,有些人賣那些小玉兒,沒錢用便去當。想來是家家當鋪裏有的。”衆人正在聽得詫異,被襲人一說,想了一想,倒大家笑起來,說:“快叫二爺進來罷,不用理那糊塗東西了。他說的那些玉,想來不是正經東西。”
\end{parag}


\begin{parag}
    寶玉正笑着,只見岫煙來了。原來岫煙走到櫳翠庵見了妙玉,不及閒話,便求妙玉扶乩。妙玉冷笑幾聲,說道:“我與姑娘來往,爲的是姑娘不是勢利場中的人。今日怎麼聽了那裏的謠言,過來纏我。況且我並不曉得什麼叫扶乩。”說着,將要不理。岫煙懊悔此來,知他脾氣是這麼着的,“一時我已說出,不好白回去,又不好與他質證他會扶乩的話。”只得陪着笑將襲人等性命關係的話說了一遍,見妙玉略有活動,便起身拜了幾拜。妙玉嘆道:“何必爲人作嫁。但是我進京以來,素無人知,今日你來破例,恐將來纏繞不休。”岫煙道:“我也一時不忍,知你必是慈悲的。便是將來他人求你,願不願在你,誰敢相強。”妙玉笑了一笑,叫道婆焚香,在箱子裏找出沙盤乩架,書了符,命岫煙行禮,祝告畢,起來同妙玉扶着乩。不多時,只見那仙乩疾書道:
\end{parag}


\begin{qute2sp}
    噫!來無跡,去無蹤,青埂峯下倚古松。欲追尋,山萬重,入我門來一笑逢。
\end{qute2sp}


\begin{parag}
    書畢,停了乩。岫煙便問請是何仙,妙玉道:“請的是拐仙。”岫煙錄了出來,請教妙玉解識。妙玉道:“這個可不能,連我也不懂。你快拿去,他們的聰明人多着哩。”岫煙只得回來。進入院中,各人都問怎麼樣了。岫煙不及細說,便將所錄乩語遞與李紈。衆姊妹及寶玉爭看,都解的是:“一時要找是找不着的,然而丟是丟不了的,不知幾時不找便出來了。但是青埂峯不知在那裏?”李紈道:“這是仙機隱語。咱們家裏那裏跑出青埂峯來,必是誰怕查出,撂在有松樹的山子石底下,也未可定。獨是‘入我門來’這句,到底是入誰的門呢?”黛玉道:“不知請的是誰!”岫煙道:“拐仙。”探春道:“若是仙家的門,便難入了。”
\end{parag}


\begin{parag}
    襲人心裏着忙,便捕風捉影的混找,沒一塊石底下不找到,只是沒有。回到院中,寶玉也不問有無,只管傻笑。麝月着急道:“小祖宗!你到底是那裏丟的,說明了,我們就是受罪也在明處啊。”寶玉笑道:“我說外頭丟的,你們又不依。你如今問我,我知道麼!”李紈探春道:“今兒從早起鬧起,已到三更來的天了。你瞧林妹妹已經掌不住,各自去了。我們也該歇歇兒了,明兒再鬧罷。”說着,大家散去。寶玉即便睡下。可憐襲人等哭一回,想一回,一夜無眠。暫且不提。
\end{parag}


\begin{parag}
    且說黛玉先自回去,想起金石的舊話來,反自喜歡,心裏說道:“和尚道士的話真個信不得。果真金玉有緣,寶玉如何能把這玉丟了呢。或者因我之事,拆散他們的金玉,也未可知。”想了半天,更覺安心,把這一天的勞乏竟不理會,重新倒看起書來。紫鵑倒覺身倦,連催黛玉睡下。黛玉雖躺下,又想到海棠花上,說“這塊玉原是胎裏帶來的,非比尋常之物,來去自有關係。若是這花主好事呢,不該失了這玉呀?看來此花開的不祥,莫非他有不吉之事?”不覺又傷起心來。又轉想到喜事上頭,此花又似應開,此玉又似應失,如此一悲一喜,直想到五更,方睡着。
\end{parag}


\begin{parag}
    次日,王夫人等早派人到當鋪裏去查問,鳳姐暗中設法找尋。一連鬧了幾天,總無下落。還喜賈母賈政未知。襲人等每日提心吊膽,寶玉也好幾天不上學,只是怔怔的,不言不語,沒心沒緒的。王夫人只知他因失玉而起,也不大着意。那日正在納悶,忽見賈璉進來請安,嘻嘻的笑道:“今日聽得軍機賈雨村打發人來告訴二老爺說,舅太爺升了內閣大學士,奉旨來京,已定明年正月二十日宣麻。有三百里的文書去了,想舅太爺晝夜趲行,半個多月就要到了。侄兒特來回太太知道。”王夫人聽說,便歡喜非常。正想孃家人少,薛姨媽家又衰敗了,兄弟又在外任,照應不着。今日忽聽兄弟拜相回京,王家榮耀,將來寶玉都有倚靠,便把失玉的心又略放開些了。天天專望兄弟來京。忽一天,賈政進來,滿臉淚痕,喘吁吁的說道:“你快去稟知老太太,即刻進宮。不用多人的,是你伏侍進去。因娘娘忽得暴病,現在太監在外立等,他說太醫院已經奏明痰厥,不能醫治。”王夫人聽說,便大哭起來。賈政道:“這不是哭的時候,快快去請老太太,說得寬緩些,不要嚇壞了老人家。”賈政說着,出來吩咐家人伺候。王夫人收了淚,去請賈母,只說元妃有病,進去請安。賈母唸佛道:“怎麼又病了!前番嚇的我了不得,後來又打聽錯了。這回情願再錯了也罷。”王夫人一面回答,一面催鴛鴦等開箱取衣飾穿戴起來。王夫人趕着回到自己房中,也穿戴好了,過來伺候。一時出廳上轎進宮。不題。
\end{parag}


\begin{parag}
    且說元春自選了鳳藻宮後,聖眷隆重,身體發福,未免舉動費力。每日起居勞乏,時發痰疾。因前日侍宴回宮,偶沾寒氣,勾起舊病。不料此回甚屬利害,竟至痰氣壅塞,四肢厥冷。一面奏明,即召太醫調治。豈知湯藥不進,連用通關之劑,並不見效。內官憂慮,奏請預辦後事。所以傳旨命賈氏椒房進見。賈母王夫人遵旨進宮,見元妃痰塞口涎,不能言語,見了賈母,只有悲泣之狀,卻少眼淚。賈母進前請安,奏些寬慰的話。少時賈政等職名遞進,宮嬪傳奏,元妃目不能顧,漸漸臉色改變。內宮太監即要奏聞,恐派各妃看視,椒房姻戚未便久羈,請在外宮伺候。賈母王夫人怎忍便離,無奈國家制度,只得下來,又不敢啼哭,惟有心內悲感。朝門內官員有信。不多時,只見太監出來,立傳欽天監。賈母便知不好,尚未敢動。稍刻,小太監傳諭出來說:“賈娘娘薨逝。”是年甲寅年十二月十八日立春,元妃薨日是十二月十九日,已交卯年寅月,存年四十三歲。賈母含悲起身,只得出宮上轎回家。賈政等亦已得信,一路悲慼。到家中,邢夫人,李紈,鳳姐,寶玉等出廳分東西迎着賈母請了安,並賈政王夫人請安,大家哭泣。不題。
\end{parag}


\begin{parag}
    次日早起,凡有品級的,按貴妃喪禮,進內請安哭臨。賈政又是工部,雖按照儀注辦理,未免堂上又要周旋他些,同事又要請教他,所以兩頭更忙,非比從前太后與周妃的喪事了。但元妃並無所出,惟諡曰”賢淑貴妃”。此是王家制度,不必多贅。只講賈府中男女天天進宮,忙的了不得。幸喜鳳姐兒近日身子好些,還得出來照應家事,又要預備王子騰進京接風賀喜。鳳姐胞兄王仁知道叔叔入了內閣,仍帶家眷來京。鳳姐心裏喜歡,便有些心病,有這些孃家的人,也便撂開,所以身子倒覺比前好了些。王夫人看見鳳姐照舊辦事,又把擔子卸了一半,又眼見兄弟來京,諸事放心,倒覺安靜些。獨有寶玉原是無職之人,又不念書,代儒學裏知他家裏有事,也不來管他,賈政正忙,自然沒有空兒查他。想來寶玉趁此機會,竟可與姊妹們天天暢樂,不料他自失了玉後,終日懶怠走動,說話也糊塗了。並賈母等出門回來,有人叫他去請安,便去,沒人叫他,他也不動。襲人等懷着鬼胎,又不敢去招惹他,恐他生氣。每天茶飯,端到面前便喫,不來也不要。襲人看這光景不象是有氣,竟象是有病的。襲人偷着空兒到瀟湘館告訴紫鵑,說是“二爺這麼着,求姑娘給他開導開導。”紫鵑雖即告訴黛玉,只因黛玉想着親事上頭一定是自己了,如今見了他,反覺不好意思:“若是他來呢,原是小時在一處的,也難不理他,若說我去找他,斷斷使不得。”所以黛玉不肯過來。襲人又背地裏去告訴探春。那知探春心裏明明知道海棠開得怪異,“寶玉”失的更奇,接連着元妃姐姐薨逝,諒家道不祥,日日愁悶,那有心腸去勸寶玉。況兄妹們男女有別,只好過來一兩次。寶玉又終是懶懶的,所以也不大常來。
\end{parag}


\begin{parag}
    寶釵也知失玉。因薛姨媽那日應了寶玉的親事,回去便告訴了寶釵。薛姨媽還說:“雖是你姨媽說了,我還沒有應準,說等你哥哥回來再定。你願意不願意?”寶釵反正色的對母親道:“媽媽這話說錯了。女孩兒家的事情是父母做主的。如今我父親沒了,媽媽應該做主的,再不然問哥哥。怎麼問起我來?”所以薛姨媽更愛惜他,說他雖是從小嬌養慣的,卻也生來的貞靜,因此在他面前,反不提起寶玉了。寶釵自從聽此一說,把”寶玉”兩個字自然更不提起了。如今雖然聽見失了玉,心裏也甚驚疑,倒不好問,只得聽旁人說去,竟象不與自己相干的。只有薛姨媽打發丫頭過來了好幾次問信。因他自己的兒子薛蟠的事焦心,只等哥哥進京便好爲他出脫罪名,又知元妃已薨,雖然賈府忙亂,卻得鳳姐好了,出來理家,也把賈家的事撂開了。只苦了襲人,雖然在寶玉跟前低聲下氣的伏侍勸慰,寶玉竟是不懂,襲人只有暗暗的着急而已。
\end{parag}


\begin{parag}
    過了幾日,元妃停靈寢廟,賈母等送殯去了幾天。豈知寶玉一日呆似一日,也不發燒,也不疼痛,只是喫不象喫,睡不象睡,甚至說話都無頭緒。那襲人麝月等一發慌了,回過鳳姐幾次。鳳姐不時過來,起先道是找不着玉生氣,如今看他失魂落魄的樣子,只有日日請醫調治。煎藥吃了好幾劑,只有添病的,沒有減病的。及至問他那裏不舒服,寶玉也不說出來。直至元妃事畢,賈母惦記寶玉,親自到園看視。王夫人也隨過來。襲人等忙叫寶玉接去請安。寶玉雖說是病,每日原起來行動,今日叫他接賈母去,他依然仍是請安,惟是襲人在旁扶着指教。賈母看了,便道:“我的兒,我打諒你怎麼病着,故此過來瞧你。今你依舊的模樣兒,我的心放了好些。”王夫人也自然是寬心的。但寶玉並不回答,只管嘻嘻的笑。賈母等進屋坐下,問他的話,襲人教一句,他說一句,大不似往常,直是一個傻子似的。賈母愈看愈疑,便說:“我才進來看時,不見有什麼病,如今細細一瞧,這病果然不輕,竟是神魂失散的樣子。到底因什麼起的呢?”王夫人知事難瞞,又瞧瞧襲人怪可憐的樣子,只得便依着寶玉先前的話,將那往南安王府裏去聽戲時丟了這塊玉的話,悄悄的告訴了一遍。心裏也彷徨的很,生恐賈母着急,並說:“現在着人在四下裏找尋,求籤問卦,都說在當鋪裏找,少不得找着的。”賈母聽了,急得站起來,眼淚直流,說道:“這件玉如何是丟得的!你們忒不懂事了,難道老爺也是撂開手的不成!”王夫人知賈母生氣,叫襲人等跪下,自己斂容低首回說:“媳婦恐老太太着急老爺生氣,都沒敢回。”賈母咳道:“這是寶玉的命根子。因丟了,所以他是這麼失魂喪魄的。還了得!況是這玉滿城裏都知道,誰撿了去便叫你們找出來麼!叫人快快請老爺,我與他說。”那時嚇得王夫人襲人等俱哀告道:“老太太這一生氣,回來老爺更了不得了。現在寶玉病着,交給我們盡命的找來就是了。”賈母道:“你們怕老爺生氣,有我呢。”便叫麝月傳人去請,不一時傳進話來,說:“老爺謝客去了。”賈母道:“不用他也使得。你們便說我說的話,暫且也不用責罰下人,我便叫璉兒來寫出賞格,懸在前日經過的地方,便說有人撿得送來者,情願送銀一萬兩,如有知人撿得送信找得者,送銀五千兩。如真有了,不可吝惜銀子。這麼一找,少不得就找出來了。若是靠着咱們家幾個人找,就找一輩子,也不能得。”王夫人也不敢直言。賈母傳話告訴賈璉,叫他速辦去了。賈母便叫人:“將寶玉動用之物都搬到我那裏去,只派襲人秋紋跟過來,餘者仍留園內看屋子。”寶玉聽了,終不言語,只是傻笑。
\end{parag}


\begin{parag}
    賈母便攜了寶玉起身,襲人等攙扶出園。回到自己房中,叫王夫人坐下,看人收拾裏間屋內安置,便對王夫人道:“你知道我的意思麼?我爲的園裏人少,怡紅院裏的花樹忽萎忽開,有些奇怪。頭裏仗着一塊玉能除邪祟,如今此玉丟了,生恐邪氣易侵,故我帶他過來一塊兒住着。這幾天也不用叫他出去,大夫來就在這裏瞧。”王夫人聽說,便接口道:“老太太想的自然是。如今寶玉同着老太太住了,老太太福氣大,不論什麼都壓住了。”賈母道:“什麼福氣,不過我屋裏乾淨些,經卷也多,都可以念念定定心神。你問寶玉好不好?”那寶玉見問,只是笑。襲人叫他說“好”,寶玉也就說“好”。王夫人見了這般光景,未免落淚,在賈母這裏,不敢出聲。賈母知王夫人着急,便說道:“你回去罷,這裏有我調停他。晚上老爺回來,告訴他不必見我,不許言語就是了。”王夫人去後,賈母叫鴛鴦找些安神定魄的藥,按方吃了。不題。
\end{parag}


\begin{parag}
    且說賈政當晚回家,在車內聽見道兒上人說道:“人要發財也容易的很。”那個問道:“怎麼見得?”這個人又道:“今日聽見榮府裏丟了什麼哥兒的玉了,貼着招帖兒,上頭寫着玉的大小式樣顏色,說有人撿了送去,就給一萬兩銀子,送信的還給五千呢。”賈政雖未聽得如此真切,心裏詫異,急忙趕回,便叫門上的人問起那事來。門上的人稟道:“奴才頭裏也不知道,今兒晌午璉二爺傳出老太太的話,叫人去貼帖兒,才知道的。”賈政便嘆氣道:“家道該衰,偏生養這麼一個孽障!才養他的時候滿街的謠言,隔了十幾年略好了些,這會子又大張曉諭的找玉,成何道理!”說着,忙走進裏頭去問王夫人。王夫人便一五一十的告訴。賈政知是老太太的主意,又不敢違拗,只抱怨王夫人幾句。又走出來,叫瞞着老太太,背地裏揭了這個帖兒下來。豈知早有那些遊手好閒的人揭了去了。
\end{parag}


\begin{parag}
    過了些時,竟有人到榮府門上,口稱送玉來。家內人們聽見,喜歡的了不得,便說:“拿來,我給你回去。”那人便懷內掏出賞格來,指給門上人瞧,”這不是你府上的帖子麼,寫明送玉來的給銀一萬兩。二太爺,你們這會子瞧我窮,回來我得了銀子,就是個財主了。別這麼待理不理的。”門上聽他話頭來得硬,說道:“你到底略給我瞧一瞧,我好給你回去。”那人初倒不肯,後來聽人說得有理,便掏出那玉,託在掌中一揚說:“這是不是?”衆家人原是在外服役,只知有玉,也不常見,今日纔看見這玉的模樣兒了。急忙跑到裏頭,搶頭報似的。那日賈政賈赦出門,只有賈璉在家。衆人回明,賈璉還細問真不真。門上人口稱:“親眼見過,只是不給奴才,要見主子,一手交銀,一手交玉。”賈璉卻也喜歡,忙去稟知王夫人,即便回明賈母。把個襲人樂得合掌唸佛。賈母並不改口,一迭連聲:“快叫璉兒請那人到書房內坐下,將玉取來一看,即便送銀。”賈璉依言,請那人進來當客待他,用好言道謝:“要借這玉送到裏頭,本人見了,謝銀分釐不短。”那人只得將一個紅綢子包兒送過去。賈璉打開一看,可不是那一塊晶瑩美玉嗎。賈璉素昔原不理論,今日倒要看看,看了半日,上面的字也彷彿認得出來,什麼”除邪祟”等字。賈璉看了,喜之不勝,便叫家人伺候,忙忙的送與賈母王夫人認去。
\end{parag}


\begin{parag}
    這會子驚動了閤家的人,都等着爭看。鳳姐見賈璉進來,便劈手奪去,不敢先看,送到賈母手裏。賈璉笑道:“你這麼一點兒事還不叫我獻功呢。”賈母打開看時,只見那玉比先前昏暗了好些。一面擦摸,鴛鴦拿上眼鏡兒來,戴着一瞧,說:“奇怪,這塊玉倒是的,怎麼把頭裏的寶色都沒了呢?”王夫人看了一會子,也認不出,便叫鳳姐過來看。鳳姐看了道:“象倒象,只是顏色不大對。不如叫寶兄弟自己一看就知道了。”襲人在旁也看着未必是那一塊,只是盼得的心盛,也不敢說出不象來。鳳姐於是從賈母手中接過來,同着襲人拿來給寶玉瞧。這時寶玉正睡着才醒。鳳姐告訴道:“你的玉有了。”寶玉睡眼朦朧,接在手裏也沒瞧,便往地上一撂道:“你們又來哄我了。”說着只是冷笑。鳳姐連忙拾起來,道:“這也奇了,怎麼你沒瞧就知道呢。”寶玉也不答言,只管笑。王夫人也進屋裏來了,見他這樣,便道:“這不用說了。他那玉原是胎裏帶來的一種古怪東西,自然他有道理。想來這個必是人見了帖兒照樣做的。”大家此時恍然大悟。賈璉在外間屋裏聽見這話,便說道:“既不是,快拿來給我問問他去,人家這樣事,他敢來鬼混。”賈母喝住道:“璉兒,拿了去給他,叫他去罷。那也是窮極了的人沒法兒了,所以見我們家有這樣事,他便想着賺幾個錢也是有的。如今白白的花了錢弄了這個東西,又叫咱們認出來了。依着我不要難爲他,把這玉還他,說不是我們的,賞給他幾兩銀子。外頭的人知道了,才肯有信兒就送來呢。若是難爲了這一個人,就有真的,人家也不敢拿來了。”賈璉答應出去。那人還等着呢,半日不見人來,正在那裏心裏發虛,只見賈璉氣忿走出來了。未知何如,下回分解。
\end{parag}