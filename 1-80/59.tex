\chap{五十九}{柳葉渚邊嗔鶯吒燕 絳雲軒裏召將飛符}


\begin{parag}
    話說寶玉聽說賈母等回來,隨多添了一件衣服,拄杖前邊來,都見過了。賈母等因每日辛苦,都要早些歇息,一宿無話,次日五鼓,又往朝中去。
\end{parag}


\begin{parag}
    離送靈日不遠,鴛鴦、琥珀、翡翠、玻璃四人都忙著打點賈母之物,玉釧、彩雲、彩霞等皆打疊王夫人之物,當面查點與跟隨的管事媳婦們。跟隨的一共大小六個丫鬟,十個老婆子媳婦子,男人不算。連日收拾馱轎器械。鴛鴦與玉釧兒皆不隨去,只看屋子。一面先幾日預發帳幔鋪陳之物,先有四五個媳婦並幾個男人領了出來,坐了幾輛車繞道先至下處,鋪陳安插等候。
\end{parag}


\begin{parag}
    臨日,賈母帶著蓉妻坐一乘馱轎,王夫人在後亦坐一乘馱轎,賈珍騎馬率了衆家丁護衛。又有幾輛大車與婆子丫鬟等坐,並放些隨換的衣包等件。是日薛姨媽尤氏率領諸人直送至大門外方回。賈璉恐路上不便,一面打發了他父母起身趕上賈母王夫人馱轎,自己也隨後帶領家丁押後跟來。
\end{parag}


\begin{parag}
    榮府內賴大添派人丁上夜,將兩處廳院都關了,一應出入人等,皆走西邊小角門。日落時,便命關了儀門,不放人出入。園中前後東西角門亦皆關鎖,只留王夫人大房之後常系他姊妹出入之門,東邊通薛姨媽的角門,這兩門因在內院,不必關鎖。裏面鴛鴦和玉釧兒也各將上房關了,自領丫鬟婆子下房去安歇。每日林之孝之妻進來,帶領十來個婆子上夜,穿堂內又添了許多小廝們坐更打梆子,已安插得十分妥當。
\end{parag}


\begin{parag}
    一日清曉,寶釵春困已醒,搴帷下榻,微覺輕寒,啓戶視之,見園中土潤苔青,原來五更時落了幾點微雨。於是喚起湘雲等人來,一面梳洗,湘雲因說兩腮作癢,恐又犯了杏癍癬,因問寶釵要些薔薇硝來。寶釵道:“前兒剩的都給了妹子。”因說:“顰兒配了許多,我正要和他要些,因今年竟沒發癢,就忘了。”因命鶯兒去取些來。鶯兒應了纔去時,蕊官便說:“我同你去,順便瞧瞧藕官。”說著,一徑同鶯兒出了蘅蕪苑。
\end{parag}


\begin{parag}
    二人你言我語,一面行走,一面說笑,不覺到了柳葉渚,順著柳堤走來。因見柳葉才吐淺碧,絲若垂金,鶯兒便笑道:“你會拿著柳條子編東西不會?”蕊官笑道:“編什麼東西?”鶯兒道:“什麼編不得?頑的使的都可。等我摘些下來,帶著這葉子編個花籃兒,採了各色花放在裏頭,纔是好頑呢。”說著,且不去取硝,且伸手挽翠披金,採了許多的嫩條,命蕊官拿著。他卻一行走一行編花籃,隨路見花便採一二枝,編出一個玲瓏過樑的籃子。枝上自有本來翠葉滿布,將花放上,卻也別致有趣。喜的蕊官笑道:“姐姐,給了我罷。”鶯兒道:“這一個咱們送林姑娘,回來咱們再多采些,編幾個大家頑。”說著,來至瀟湘館中。
\end{parag}


\begin{parag}
    黛玉也正晨妝,見了籃子,便笑說:“這個新鮮花籃是誰編的?”鶯兒笑說:“我編了送姑娘頑的。”黛玉接了笑道:“怪道人贊你的手巧,這頑意兒卻也別緻。”一面瞧了,一面便命紫鵑掛在那裏。鶯兒又問候了薛姨媽,方和黛玉要硝。黛玉忙命紫鵑包了一包,遞與鶯兒。黛玉又道:“我好了,今日要出去逛逛。你回去說與姐姐,不用過來問候媽了,也不敢勞他來瞧我,梳了頭同媽都往你那裏去,連飯也端了那裏去喫,大家熱鬧些。”
\end{parag}


\begin{parag}
    鶯兒答應了出來,便到紫鵑房中找蕊官。只見藕官與蕊官二人正說得高興,不能相舍,因說:“姑娘也去呢,藕官先同我們去等著豈不好?”紫鵑聽如此說,便也說道:“這話倒是,他這裏淘氣的也可厭。”一面說,一面便將黛玉的匙箸用一塊洋巾包了,交與藕官道:“你先帶了這個去,也算一趟差了。”
\end{parag}


\begin{parag}
    藕官接了,笑嘻嘻同他二人出來,一徑順著柳堤走來。鶯兒便又採些柳條,越性坐在山石上編起來,又命蕊官先送了硝去再來。他二人只顧愛看他編,那裏捨得去。鶯兒只顧催說:“你們再不去,我也不編了。”藕官便說:“我同你去了,再快回來。”二人方去了。
\end{parag}


\begin{parag}
    這裏鶯兒正編,只見何婆的小女春燕走來,笑問:“姐姐織什麼呢?”正說著,蕊藕二人也到了。春燕便向藕官道:“前兒你到底燒什麼紙?被我姨媽看見了,要告你沒告成,倒被寶玉賴了他一大些不是,氣的他一五一十告訴我媽。你們在外頭這二三年積了些什麼仇恨,如今還不解開?”藕官冷笑道:“有什麼仇恨?他們不知足,反怨我們了。在外頭這兩年,別的東西不算,只算我們的米菜,不知賺了多少家去,閤家子吃不了,還有每日買東買西賺的錢。在外逢我們使他們一使兒,就怨天怨地的。你說說可有良心?”春燕笑道:“他是我的姨媽,也不好向著外人反說他的。怨不得寶玉說:‘女孩兒未出嫁,是顆無價之寶珠;出了嫁,不知怎麼就變出許多的不好的毛病來,雖是顆珠子,卻沒有光彩寶色,是顆死珠了;再老了,更變的不是珠子,竟是魚眼睛了。分明一個人,怎麼變出三樣來?’這話雖是混話,倒也有些不差。別人不知道,只說我媽和姨媽,他老姊妹兩個,如今越老了越把錢看的真了。先時老姐兒兩個在家抱怨沒個差使,沒個進益,幸虧有了這園子,把我挑進來,可巧把我分到怡紅院。家裏省了我一個人的費用不算外,每月還有四五百錢的餘剩,這也還說不夠。後來老姊妹二人都派到梨香院去照看他們,藕官認了我姨媽,芳官認了我媽,這幾年著實寬裕了。如今挪進來也算撒開手了,還只無厭。你說好笑不好笑?我姨媽剛和藕官吵了,接著我媽爲洗頭就和芳官吵。芳官連要洗頭也不給他洗。昨日得月錢,推不去了,買了東西先叫我洗。我想了一想:我自有錢,就沒錢要洗時,不管襲人、晴雯、麝月,那一個跟前和他們說一聲,也都容易,何必借這個光兒?好沒意思。所以我不洗。他又叫我妹妹小鳩兒洗了,才叫芳官,果然就吵起來。接著又要給寶玉吹湯,你說可笑死了人?我見他一進來,我就告訴那些規矩。他只不信,只要強做知道的,足的討個沒趣兒。幸虧園裏的人多,沒人分記的清楚誰是誰的親故。若有人記得,只有我們一家人吵,什麼意思呢?你這會子又跑來弄這個。這一帶地上的東西都是我姑娘管著,一得了這地方,比得了永遠基業還利害,每日早起晚睡,自己辛苦了還不算,每日逼著我們來照看,生恐有人糟踏,又怕誤了我的差使。如今進來了,老姑嫂兩個照看得謹謹慎慎,一根草也不許人動。你還掐這些花兒,又折他的嫩樹,他們即刻就來,仔細他們抱怨。”鶯兒道:“別人亂折亂掐使不得,獨我使得。自從分了地基之後,每日裏各房皆有分例,喫的不用算,單管花草頑意兒。誰管什麼,每日誰就把各房裏姑娘丫頭戴的,必要各色送些折枝的去,還有插瓶的。惟有我們說了:‘一概不用送,等要什麼再和你們要。’究竟沒有要過一次。我今便掐些,他們也不好意思說的。”
\end{parag}


\begin{parag}
    一語未了,他姑娘果然拄了拐走來。鶯兒春燕等忙讓坐。那婆子見採了許多嫩柳,又見藕官等都採了許多鮮花,心內便不受用;看著鶯兒編,又不好說什麼,便說春燕道:“我叫你來照看照看,你就貪住頑不去了。倘或叫起你來,你又說我使你了,拿我做隱身符兒你來樂。”春燕道:“你老又使我,又怕,這會子反說我。難道把我劈做八瓣子不成?”鶯兒笑道:“姑媽,你別信小燕的話。這都是他摘下來的,煩我給他編,我攆他,他不去。”春燕笑道:“你可少頑兒,你只顧頑兒,老人家就認真了。”那婆子本是愚頑之輩,兼之年近昏耄,惟利是命,一概情面不管,正心疼肝斷,無計可施,聽鶯兒如此說,便以老賣老,拿起拄杖來向春燕身上擊上幾下,罵道:“小蹄子,我說著你,你還和我強嘴兒呢。你媽恨的牙根癢癢,要撕你的肉喫呢。你還來和我強梆子似的。”打的春燕又愧又急,哭道:“鶯兒姐姐頑話,你老就認真打我。我媽爲什麼恨我?我又沒燒胡了洗臉水,有什麼不是!”鶯兒本是頑話,忽見婆子認真動了氣,忙上去拉住,笑道:“我纔是頑話,你老人家打他,我豈不愧?”那婆子道:“姑娘,你別管我們的事,難道爲姑娘在這裏,不許我管孩子不成?”鶯兒聽見這般蠢話,便賭氣紅了臉,撒了手冷笑道:“你老人家要管,那一刻管不得,偏我說了一句頑話就管他了。我看你老管去!”說著,便坐下,仍編柳籃子。
\end{parag}


\begin{parag}
    偏又有春燕的娘出來找他,喊道:“你不來舀水,在那裏做什麼呢?”那婆子便接聲兒道:“你來瞧瞧,你的女兒連我也不服了!在那裏排揎我呢。”那婆子一面走過來說:“姑奶奶,又怎麼了?我們丫頭眼裏沒娘罷了,連姑媽也沒了不成?”鶯兒見他娘來了,只得又說原故。他姑娘那裏容人說話,便將石上的花柳與他娘瞧道:“你瞧瞧,你女兒這麼大孩子頑的。他先領著人糟踏我,我怎麼說人?”他娘也正爲芳官之氣未平,又恨春燕不遂他的心,便走上來打耳刮子,罵道:“小娼婦,你能上去了幾年?你也跟那起輕狂浪小婦學,怎麼就管不得你們了?乾的我管不得,你是我屄裏掉出來的,難道也不敢管你不成!既是你們這起蹄子到的去的地方我到不去,你就該死在那裏伺候,又跑出來浪漢。”一面又抓起柳條子來,直送到他臉上,問道:“這叫作什麼?這編的是你孃的屄!”鶯兒忙道:“那是我們編的,你老別指桑罵槐。”那婆子深妒襲人晴雯一干人,已知凡房中大些的丫鬟都比他們有些體統權勢,凡見了這一干人,心中又畏又讓,未免又氣又恨,亦且遷怒於衆,復又看見了藕官,又是他令姊的冤家,四處湊成一股怒氣。
\end{parag}


\begin{parag}
    那春燕啼哭著往怡紅院去了。他娘又恐問他爲何哭,怕他又說出自己打他,又要受晴雯等之氣,不免著起急來,又忙喊道:“你回來!我告訴你再去。”春燕那裏肯回來?急的他娘跑了去又拉他。他回頭看見,便也往前飛跑。他娘只顧趕他,不防腳下被青苔滑倒,引的鶯兒三個人反都笑了。鶯兒便賭氣將花柳皆擲於河中,自回房去。這裏把個婆子心疼的只念佛,又罵:“促狹小蹄子!糟踏了花兒,雷也是要打的。”自己且掐花與各房送去不提。
\end{parag}


\begin{parag}
    卻說春燕一直跑入院中,頂頭遇見襲人往黛玉處去問安。春燕便一把抱住襲人,說:“姑娘救我!我娘又打我呢。”襲人見他娘來了,不免生氣,便說道:“三日兩頭兒打了乾的打親的,還是買弄你女兒多,還是認真不知王法?”這婆子來了幾日,見襲人不言不語是好性的,便說道:“姑娘你不知道,別管我們閒事!都是你們縱的,這會子還管什麼?”說著,便又趕著打。襲人氣的轉身進來,見麝月正在海棠下晾手巾,聽得如此喊鬧,便說:“姐姐別管,看他怎樣。”一面使眼色與春燕,春燕會意,便直奔了寶玉去。衆人都笑說:“這可是沒有的事都鬧出來了。”麝月向婆子道:“你再略煞一煞氣兒,難道這些人的臉面,和你討一個情還討不下來不成?”那婆子見他女兒奔到寶玉身邊去,又見寶玉拉了春燕的手說:“別怕,有我呢。”春燕又一行哭,又一行說,把方纔鶯兒等事都說出來。寶玉越發急起來,說:“你只在這裏鬧也罷了,怎麼連親戚也都得罪起來?”麝月又向婆子及衆人道:“怨不得這嫂子說我們管不著他們的事,我們雖無知錯管了,如今請出一個管得著的人來管一管,嫂子就心服口服,也知道規矩了。”便回頭叫小丫頭子:“去把平兒給我們叫來!平兒不得閒就把林大娘叫了來。”那小丫頭應了就走。衆媳婦上來笑說:“嫂子,快求姑娘們叫回那孩子罷。平姑娘來了,可就不好了。”那婆子說道:“憑你那個平姑娘來也憑個理,沒有娘管女兒大家管著孃的。”衆人笑道:“你當是那個平姑娘?是二奶奶屋裏的平姑娘。他有情呢,你說兩句;他一翻臉,嫂子你吃不了兜著走!”
\end{parag}


\begin{parag}
    說話之間,只見小丫頭子回來說:“平姑娘正有事,問我作什麼,我告訴了他,他說:‘既這樣,且攆他出去,告訴了林大娘在角門外打他四十板子就是了。 ’”那婆子聽如此說,自不捨得出去,便又淚流滿面,央告襲人等說:“好容易我進來了,況且我是寡婦,家裏沒人,正好一心無掛的在裏頭伏侍姑娘們。姑娘們也便宜,我家裏也省些攪過。我這一去,又要去自己生火過活,將來不免又沒了過活。”襲人見他如此,早又心軟了,便說:“你既要在這裏,又不守規矩,又不聽說,又亂打人。那裏弄你這個不曉事的來,天天鬥口,也叫人笑話,失了體統。”晴雯道:“理他呢,打發去了是正經。誰和他去對嘴對舌的。”那婆子又央衆人道:“我雖錯了,姑娘們吩咐了,我以後改過。姑娘們那不是行好積德。”一面又央春燕道:“原是我爲打你起的,究竟沒打成你,我如今反受了罪?你也替我說說。”寶玉見如此可憐,只得留下,吩咐他不可再鬧。那婆子走來一一的謝過了下去。
\end{parag}


\begin{parag}
    只見平兒走來,問系何事。襲人等忙說:“已完了,不必再提。”平兒笑道:“‘得饒人處且饒人’,得省的將就省些事也罷了。能去了幾日,只聽各處大小人兒都作起反來了,一處不了又一處,叫我不知管那一處的是。”襲人笑道:“我只說我們這裏反了,原來還有幾處。” 平兒笑道:“這算什麼。正和珍大奶奶算呢,這三四日的工夫,一共大小出來了八九件了。你這裏是極小的,算不起數兒來,還有大的可氣可笑之事。”不知襲人問他果系何事,且聽下回分解。
\end{parag}
