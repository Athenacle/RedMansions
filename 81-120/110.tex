\chap{一百一十}{史太君壽終歸地府 王鳳姐力詘失人心}



\begin{parag}
    卻說賈母坐起說道:“我到你們家已經六十多年了。從年輕的時候到老來,福也享盡了。自你們老爺起,兒子孫子也都算是好的了。就是寶玉呢,我疼了他一場。”說到那裏,拿眼滿地下瞅着。王夫人便推寶玉走到牀前。賈母從被窩裏伸出手來拉着寶玉道:“我的兒,你要爭氣纔好!”寶玉嘴裏答應,心裏一酸,那眼淚便要流下來,又不敢哭,只得站着,聽賈母說道:“我想再見一個重孫子我就安心了。我的蘭兒在那裏呢?”李紈也推賈蘭上去。賈母放了寶玉,拉着賈蘭道:“你母親是要孝順的,將來你成了人,也叫你母親風光風光。鳳丫頭呢?”鳳姐本來站在賈母旁邊,趕忙走到眼前說:“在這裏呢。”賈母道:“我的兒,你是太聰明瞭,將來修修福罷。我也沒有修什麼,不過心實喫虧,那些喫齋唸佛的事我也不大幹,就是舊年叫人寫了些《金剛經》送送人,不知送完了沒有?”鳳姐道:“沒有呢。”賈母道:“早該施捨完了纔好。我們大老爺和珍兒是在外頭樂了,最可惡的是史丫頭沒良心,怎麼總不來瞧我。”鴛鴦等明知其故,都不言語。賈母又瞧了一瞧寶釵,嘆了口氣,只見臉上發紅。賈政知是迴光返照,即忙進上蔘湯。賈母的牙關已經緊了,合了一回眼,又睜着滿屋裏瞧了一瞧。王夫人寶釵上去輕輕扶着,邢夫人鳳姐等便忙穿衣,地下婆子們已將牀安設停當,鋪了被褥,聽見賈母喉間略一響動,臉變笑容,竟是去了,享年八十三歲。衆婆子疾忙停牀。於是賈政等在外一邊跪着,邢夫人等在內一邊跪着,一齊舉起哀來。外面家人各樣預備齊全,只聽裏頭信兒一傳出來,從榮府大門起至內宅門扇扇大開,一色淨白紙糊了,孝棚高起,大門前的牌樓立時豎起,上下人等登時成服。賈政報了丁憂。禮部奏聞,主上深仁厚澤,念及世代功勳,又系元妃祖母,賞銀一千兩,諭禮部主祭。家人們各處報喪。衆親友雖知賈家勢敗,今見聖恩隆重,都來探喪。擇了吉時成殮,停靈正寢。賈赦不在家,賈政爲長,寶玉,賈環,賈蘭是親孫,年紀又小,都應守靈。賈璉雖也是親孫,帶着賈蓉尚可分派家人辦事。雖請了些男女外親來照應,內裏邢王二夫人,李紈,鳳姐,寶釵等是應靈旁哭泣的,尤氏雖可照應,他賈珍外出依住榮府,一向總不上前,且又榮府的事不甚諳練。賈蓉的媳婦更不必說了。惜春年小,雖在這裏長的,他於家事全不知道。所以內裏竟無一人支持,只有鳳姐可以照管裏頭的事。況又賈璉在外作主,裏外他二人倒也相宜。
\end{parag}


\begin{parag}
    鳳姐先前仗着自己的才幹,原打量老太太死了他大有一番作用。邢王二夫人等本知他曾辦過秦氏的事,必是妥當,於是仍叫鳳姐總理裏頭的事。鳳姐本不應辭,自然應了,心想:“這裏的事本是我管的,那些家人更是我手下的人,太太和珍大嫂子的人本來難使喚些,如今他們都去了。銀項雖沒有了對牌,這種銀子是現成的。外頭的事又是他辦着。雖說我現今身子不好,想來也不致落褒貶,必是比寧府裏還得辦些。”心下已定,且待明日接了三,後日一早便叫周瑞家的傳出話去,將花名冊取上來。鳳姐一一的瞧了,統共只有男僕二十一人,女僕只有十九人,餘者俱是些丫頭,連各房算上,也不過三十多人,難以點派差使。心裏想道:“這回老太太的事倒沒有東府裏的人多。”又將莊上的弄出幾個,也不敷差遣。正在思算,只見一個小丫頭過來說:“鴛鴦姐姐請奶奶。”鳳姐只得過去。只見鴛鴦哭得淚人一般,一把拉着鳳姐兒說道:“二奶奶請坐,我給二奶奶磕個頭。雖說服中不行禮,這個頭是要磕的。”鴛鴦說着跪下。慌的鳳姐趕忙拉住,說道:“這是什麼禮,有話好好的說。”鴛鴦跪着,鳳姐便拉起來。鴛鴦說道:“老太太的事一應內外都是二爺和二奶奶辦,這種銀子是老太太留下的。老太太這一輩子也沒有糟踏過什麼銀錢,如今臨了這件大事,必得求二奶奶體體面面的辦一辦纔好。我方纔聽見老爺說什麼詩云子曰,我不懂,又說什麼‘喪與其易,甯戚’,我聽了不明白。我問寶二奶奶,說是老爺的意思老太太的喪事只要悲切纔是真孝,不必糜費圖好看的念頭。我想老太太這樣一個人,怎麼不該體面些!我雖是奴才丫頭,敢說什麼,只是老太太疼二奶奶和我這一場,臨死了還不叫他風光風光!我想二奶奶是能辦大事的,故此我請二奶奶來求作個主。我生是跟老太太的人,老太太死了我也是跟老太太的,若是瞧不見老太太的事怎麼辦,將來怎麼見老太太呢!”鳳姐聽了這話來的古怪,便說:“你放心,要體面是不難的。況且老爺雖說要省,那勢派也錯不得。便拿這項銀子都花在老太太身上,也是該當的。”鴛鴦道:“老太太的遺言說,所有剩下的東西是給我們的,二奶奶倘或用着不夠,只管拿這個去折變補上。就是老爺說什麼,我也不好違老太太的遺言。那日老太太分派的時候不是老爺在這裏聽見的麼。”鳳姐道:“你素來最明白的,怎麼這會子那樣的着急起來了。”鴛鴦道:“不是我着急,爲的是大太太是不管事的,老爺是怕招搖的,若是二奶奶心裏也是老爺的想頭,說抄過家的人家喪事還是這麼好,將來又要抄起來,也就不顧起老太太來,怎麼處!在我呢是個丫頭,好歹礙不着,到底是這裏的聲名。”鳳姐道:“我知道了,你只管放心,有我呢!”鴛鴦千恩萬謝的託了鳳姐。
\end{parag}


\begin{parag}
    那鳳姐出來想道:“鴛鴦這東西好古怪,不知打了什麼主意,論理老太太身上本該體面些。噯,不要管他,且按着咱們家先前的樣子辦去。”於是叫了旺兒家的來把話傳出去請二爺進來。不多時,賈璉進來,說道:“怎麼找我?你在裏頭照應着些就是了。橫豎作主是咱們二老爺,他說怎麼着咱們就怎麼着。”鳳姐道:“你也說起這個話來了,可不是鴛鴦說的話應驗了麼。”賈璉道:“什麼鴛鴦的話?”鳳姐便將鴛鴦請進去的話述了一遍。賈璉道:“他們的話算什麼。纔剛二老爺叫我去,說老太太的事固要認真辦理,但是知道的呢,說是老太太自己結果自己,不知道的只說咱們都隱匿起來了,如今很寬裕。老太太的這種銀子用不了誰還要麼,仍舊該用在老太太身上。老太太是在南邊的墳地雖有,陰宅卻沒有。老太太的柩是要歸到南邊去的,留這銀子在祖墳上蓋起些房屋來,再餘下的置買幾頃祭田。咱們回去也好,就是不回去,也叫這些貧窮族中住着,也好按時按節早晚上香,時常祭掃祭掃。你想這些話可不是正經主意?據你這個話,難道都花了罷?”鳳姐道:“銀子發出來了沒有?”賈璉道:“誰見過銀子!我聽見咱們太太聽見了二老爺的話,極力的竄掇二太太和二老爺,說這是好主意。叫我怎麼着!現在外頭棚槓上要支幾百銀子,這會子還沒有發出來。我要去,他們都說有,先叫外頭辦了回來再算。你想這些奴才們有錢的早溜了,按着冊子叫去,有的說告病,有的說下莊子去了。走不動的有幾個,只有賺錢的能耐,還有賠錢的本事麼!”鳳姐聽了,呆了半天,說道:“這還辦什麼!”正說着,見來了一個丫頭說:“大太太的話問二奶奶,今兒第三天了,裏頭還很亂,供了飯還叫親戚們等着嗎?叫了半天,來了菜,短了飯,這是什麼辦事的道理!”鳳姐急忙進去,吆喝人來伺候,胡弄着將早飯打發了。偏偏那日人來的多,裏頭的人都死眉瞪眼的。鳳姐只得在那裏照料了一會子,又惦記着派人,趕着出來叫了旺兒家的傳齊了家人女人們,一一分派了。衆人都答應着不動。鳳姐道:“什麼時候,還不供飯!”衆人道:“傳飯是容易的,只要將裏頭的東西發出來,我們纔好照管去。”鳳姐道:“糊塗東西,派定了你們少不得有的。”衆人只得勉強應着。鳳姐即往上房取發應用之物,要去請示邢王二夫人,見人多難說,看那時候已經日漸平西了,只得找了鴛鴦,說要老太太存的這一分傢伙。鴛鴦道:“你還問我呢,那一年二爺當了贖了來了麼!”鳳姐道:“不用銀的金的,只要這一分平常使的。”鴛鴦道:“大太太珍大奶奶屋裏使的是那裏來的!”鳳姐一想不差,轉身就走,只得到王夫人那邊找了玉釧彩雲,纔拿了一分出來,急忙叫彩明登帳,發與衆人收管。
\end{parag}


\begin{parag}
    鴛鴦見鳳姐這樣慌張,又不好叫他回來,心想:“他頭裏作事何等爽利周到,如今怎麼掣肘的這個樣兒。我看這兩三天連一點頭腦都沒有,不是老太太白疼了他了嗎!”那裏知邢夫人一聽賈政的話,正合着將來家計艱難的心,巴不得留一點子作個收局。況且老太太的事原是長房作主,賈赦雖不在家,賈政又是拘泥的人,有件事便說請大奶奶的主意。邢夫人素知鳳姐手腳大,賈璉的鬧鬼,所以死拿住不放鬆。鴛鴦只道已將這項銀兩交了出去了,故見鳳姐掣肘如此,便疑爲不肯用心,便在賈母靈前嘮嘮叨叨哭個不了。邢夫人等聽了話中有話,不想到自己不令鳳姐便宜行事,反說鳳丫頭果然有些不用心。王夫人到了晚上叫了鳳姐過來說:“咱們家雖說不濟,外頭的體面是要的。這兩三日人來人往,我瞧着那些人都照應不到,想是你沒有吩咐。還得你替我們操點心兒纔好。”鳳姐聽了,呆了一會,要將銀兩不湊手的話說出,但是銀錢是外頭管的,王夫人說的是照應不到,鳳姐也不敢辨,只好不言語。邢夫人在旁說道:“論理該是我們做媳婦的操心,本不是孫子媳婦的事。但是我們動不得身,所以託你的,你是打不得撒手的。”鳳姐紫漲了臉,正要回說,只聽外頭鼓樂一奏,是燒黃昏紙的時候了,大家舉起哀來,又不得說,鳳姐原想回來再說,王夫人催他出去料理,說道:“這裏有我們的,你快快兒的去料理明兒的事罷。”
\end{parag}


\begin{parag}
    鳳姐不敢再言,只得含悲忍泣的出來,又叫人傳齊了衆人,又吩咐了一會,說:“大娘嬸子們可憐我罷!我上頭捱了好些說,爲的是你們不齊截,叫人笑話。明兒你們豁出些辛苦來罷。”那些人回道:“奶奶辦事不是今兒個一遭兒了,我們敢違拗嗎。只是這回的事上頭過於累贅。只說打發這頓飯罷,有的在這裏喫,有的要在家裏喫,請了那位太太,又是那位奶奶不來。諸如此類,那得齊全。還求奶奶勸勸那些姑娘們不要挑飭就好了。”鳳姐道:“頭一層是老太太的丫頭們是難纏的,太太們的也難說話,叫我說誰去呢。”衆人道:“從前奶奶在東府裏還是署事,要打要罵,怎麼這樣鋒利,誰敢不依。如今這些姑娘們都壓不住了?”鳳姐嘆道:“東府裏的事雖說託辦的,太太雖在那裏,不好意思說什麼。如今是自己的事情,又是公中的,人人說得話。再者外頭的銀錢也叫不靈,即如棚裏要一件東西,傳了出來總不見拿進來。這叫我什麼法兒呢。”衆人道:“二爺在外頭倒怕不應付麼?”鳳姐道:“還提那個,他也是那裏爲難。第一件銀錢不在他手裏,要一件得回一件,那裏湊手。”衆人道:“老太太這項銀子不在二爺手裏嗎?”鳳姐道:“你們回來問管事的便知道了。”衆人道:“怨不得我們聽見外頭男人抱怨說:‘這麼件大事,咱們一點摸不着,淨當苦差!’叫人怎麼能齊心呢?”鳳姐道:“如今不用說了,眼面前的事大家留些神罷。倘或鬧的上頭有了什麼說的,我和你們不依的。”衆人道:“奶奶要怎麼樣他們敢抱怨嗎,只是上頭一人一個主意,我們實在難周到的。”鳳姐聽了沒法,只得央說道:“好大娘們!明兒且幫我一天,等我把姑娘們鬧明白了再說罷咧。”衆人聽命而去。
\end{parag}


\begin{parag}
    鳳姐一肚子的委屈,愈想愈氣,直到天亮又得上去。要把各處的人整理整理,又恐邢夫人生氣,要和王夫人說,怎奈邢夫人挑唆。這些丫頭們見邢夫人等不助着鳳姐的威風,更加作踐起他來。幸得平兒替鳳姐排解,說是“二奶奶巴不得要好,只是老爺太太們吩咐了外頭,不許糜費,所以我們二奶奶不能應付到了。”說過幾次才得安靜些。雖說僧經道懺,上祭掛帳,絡繹不絕,終是銀錢吝嗇,誰肯踊躍,不過草草了事。連日王妃誥命也來得不少,鳳姐也不能上去照應,只好在底下張羅,叫了那個,走了這個,發一回急,央及一會,胡弄過了一起,又打發一起。別說鴛鴦等看去不象樣,連鳳姐自己心裏也過不去了。
\end{parag}


\begin{parag}
    邢夫人雖說是冢婦,仗着“悲慼爲孝”四個字,倒也都不理會。王夫人落得跟了邢夫人行事,餘者更不必說了。獨有李紈瞧出鳳姐的苦處,也不敢替他說話,只自嘆道:“俗話說的,‘牡丹雖好,全仗綠葉扶持’,太太們不虧了鳳丫頭,那些人還幫着嗎!若是三姑娘在家還好,如今只有他幾個自己的人瞎張羅,面前背後的也抱怨說是一個錢摸不着,臉面也不能剩一點兒。老爺是一味的盡孝,庶務上頭不大明白,這樣的一件大事,不撒散幾個錢就辦的開了嗎!可憐鳳丫頭鬧了幾年,不想在老太太的事上,只怕保不住臉了。”於是抽空兒叫了他的人來吩咐道:“你們別看着人家的樣兒,也糟踏起璉二奶奶來。別打量什麼穿孝守靈就算了大事了,不過混過幾天就是了。看見那些人張羅不開,便插個手兒也未爲不可,這也是公事,大家都該出力的。”那些素服李紈的人都答應着說:“大奶奶說得很是。我們也不敢那麼着,只聽見鴛鴦姐姐們的口話兒好象怪璉二奶奶的似的。”李紈道:“就是鴛鴦我也告訴過他,我說璉二奶奶並不是在老太太的事上不用心,只是銀子錢都不在他手裏,叫他巧媳婦還作的上沒米的粥來嗎?如今鴛鴦也知道了,所以他不怪他了。只是鴛鴦的樣子竟是不象從前了,這也奇怪,那時候有老太太疼他倒沒有作過什麼威福,如今老太太死了,沒有了仗腰子的了,我看他倒有些氣質不大好了。我先前替他愁,這會子幸喜大老爺不在家才躲過去了,不然他有什麼法兒。”
\end{parag}


\begin{parag}
    說着,只見賈蘭走來說:“媽媽睡罷,一天到晚人來客去的也乏了,歇歇罷。我這幾天總沒有摸摸書本兒,今兒爺爺叫我家裏睡,我喜歡的很,要理個一兩本書纔好。別等脫了孝再都忘了。”李紈道:“好孩子,看書呢自然是好的。今兒且歇歇罷,等老太太送了殯再看罷。”賈蘭道:“媽媽要睡,我也就睡在被窩裏頭想想也罷了。”衆人聽了都誇道:“好哥兒,怎麼這點年紀得了空兒就想到書上!不象寶二爺娶了親的人還是那麼孩子氣,這幾日跟着老爺跪着,瞧他很不受用,巴不得老爺一動身就跑過來找二奶奶,不知唧唧咕咕的說些什麼,甚至弄的二奶奶都不理他了。他又去找琴姑娘,琴姑娘也遠避他。邢姑娘也不很同他說話。倒是咱們本家的什麼喜姑娘咧四姑娘咧,哥哥長哥哥短的和他親蜜。我們看那寶二爺除了和奶奶姑娘們混混,只怕他心裏也沒有別的事,白過費了老太太的心,疼了他這麼大,那裏及蘭哥兒一零兒呢。大奶奶,你將來是不愁的了。”李紈道:“就好也還小,只怕到他大了,咱們家還不知怎麼樣了呢!環哥兒你們瞧着怎麼樣?”衆人道:“這一個更不象樣兒了!兩個眼睛倒象個活猴兒似的,東溜溜,西看看,雖在那裏嚎喪,見了奶奶姑娘們來了,他在孝幔子裏頭淨偷着眼兒瞧人呢。”李紈道:“他的年紀其實也不小了。前日聽見說還要給他說親呢,如今又得等着了。噯,還有一件事,——咱們家這些人,我看來也是說不清的,且不必說閒話,——後日送殯各房的車輛是怎麼樣了?”衆人道:“璉二奶奶這幾天鬧的象失魂落魄的樣兒了,也沒見傳出去。昨兒聽見我的男人說,璉二爺派了薔二爺料理,說是咱們家的車也不夠,趕車的也少,要到親戚家去借去呢。”李紈笑道:“車也都是借得的麼?”衆人道:“奶奶說笑話兒了,車怎麼借不得?只是那一日所有的親戚都用車,只怕難借,想來還得僱呢。”李紈道:“底下人的只得僱,上頭白車也有僱的麼?”衆人道:“現在大太太東府裏的大奶奶小蓉奶奶都沒有車了,不僱那裏來的呢?”李紈聽了嘆息道:“先前見有咱們家兒的太太奶奶們坐了僱的車來咱們都笑話,如今輪到自己頭上了。你明兒去告訴你的男人,我們的車馬早早兒的預備好了,省得擠。”衆人答應了出去。不題。
\end{parag}


\begin{parag}
    且說史湘雲因他女婿病着,賈母死後只來的一次,屈指算是後日送殯,不能不去。又見他女婿的病已成癆症,暫且不妨,只得坐夜前一日過來。想起賈母素日疼他,又想到自己命苦,剛配了一個才貌雙全的男人,性情又好,偏偏的得了冤孽症候,不過捱日子罷了。於是更加悲痛,直哭了半夜。鴛鴦等再三勸慰不止。寶玉瞅着也不勝悲傷,又不好上前去勸,見他淡妝素服,不敷脂粉,更比未出嫁的時候猶勝幾分。轉念又看寶琴等淡素裝飾,自有一種天生丰韻。獨有寶釵渾身孝服,那知道比尋常穿顏色時更有一番雅緻。心裏想道:“所以千紅萬紫終讓梅花爲魁,殊不知並非爲梅花開的早,竟是‘潔白清香’四字是不可及的了。但只這時候若有林妹妹也是這樣打扮,又不知怎樣的丰韻了!”想到這裏,不覺的心酸起來,那淚珠便直滾滾的下來了,趁着賈母的事,不妨放聲大哭。衆人正勸湘雲不止,外間又添出一個哭的來了。大家只道是想着賈母疼他的好處,所以傷悲,豈知他們兩個人各自有各自的心事。這場大哭,不禁滿屋的人無不下淚。還是薛姨媽李嬸孃等勸住。
\end{parag}


\begin{parag}
    明日是坐夜之期,更加熱鬧。鳳姐這日竟支撐不住,也無方法,只得用盡心力,甚至咽喉嚷破敷衍過了半日。到了下半天,人客更多了,事情也更繁了,瞻前不能顧後。正在着急,只見一個小丫頭跑來說:“二奶奶在這裏呢,怪不得大太太說,裏頭人多照應不過來,二奶奶是躲着受用去了。”鳳姐聽了這話,一口氣撞上來,往下一咽,眼淚直流,只覺得眼前一黑,嗓子裏一甜,便噴出鮮紅的血來,身子站不住,就蹲倒在地。幸虧平兒急忙過來扶住。只見鳳姐的血吐個不住。未知性命如何,下回分解。
\end{parag}