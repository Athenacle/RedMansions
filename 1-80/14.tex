\chap{一十四}{林如海捐館揚州城 賈寶玉路謁北靜王}

\begin{parag}
    \begin{note}甲:鳳姐用彩明,因自識字不多,且彩明系未冠之童。\end{note}
\end{parag}


\begin{parag}
    \begin{note}甲:寫鳳姐之珍貴,寫鳳姐之英氣,寫鳳姐之聲勢,寫鳳姐之心機,寫鳳姐之驕大。\end{note}
\end{parag}


\begin{parag}
    \begin{note}甲:昭兒回,並非林文、璉文,是黛玉正文。\end{note}
\end{parag}


\begin{parag}
    \begin{note}甲:牛,醜也。清,屬水,子也。柳拆卯字。彪拆虎字,寅字寓焉。陳即辰。翼火爲蛇;巳字寓焉。馬,午也。魁拆鬼,鬼,金羊,未字寓焉。侯、猴同音,申也。曉鳴,雞也,酉字寓焉。石即豕,亥字寓焉。其祖曰守業,即守夜也,犬字寓焉。此所謂十二支寓焉。\end{note}
\end{parag}


\begin{parag}
    \begin{note}甲:路謁北靜王,是寶玉正文。\end{note}
\end{parag}


\begin{parag}
    \begin{note}蒙:家書一紙千金重,勾引難防囑下人。任你無雙肝膽烈,多情奮起自眉顰。\end{note}
\end{parag}


\begin{parag}
    話說寧國府中都總管來升聞得裏面委請了鳳姐,因傳齊了同事人等說道:“如今請了西府裏璉二奶奶管理內事,倘或他來支取東西,或是說話,我們須要比往日小心些。每日大家早來晚散,寧可辛苦這一個月,過後再歇著,不要把老臉丟了。\begin{note}庚側:此是都總管的話頭。\end{note}那是個有名的烈貨,臉酸心硬,一時惱了,不認人的。”衆人都道:“有理。”又有一個笑道:“論理,我們裏面也須得他來整理整理,\begin{note}庚側:伏線在二十板之誤差婦人。\end{note}都忒不像了。”正說著,只見來旺媳婦拿了對牌來領取呈文京榜紙札,票上批著數目。衆人連忙讓坐倒茶,一面命人按數取紙來抱著,同來旺媳婦一路來至儀門口,方交與來旺媳婦自己抱進去了。
\end{parag}


\begin{parag}
    鳳姐即命彩明釘造簿冊。\begin{note}甲眉:寧府如此大家,阿鳳如此身份,豈有使貼身丫頭與家裏男人答話交事之理呢?此作者忽略之處。\end{note}\begin{note}庚眉:彩明系未冠小童,阿鳳便於出入使令者。老兄並未前後看明,是男是女,亂加批駁。可笑。\end{note}\begin{note}庚眉:且明寫阿鳳不識字之故。壬午春。\end{note}即時傳來升媳婦,兼要家口花名冊來查看,又限於明日一早傳齊家人媳婦進來聽差等語。大概點了一點數目單冊,\begin{note}甲側:已有成見。\end{note}問了來升媳婦幾句話,便坐車回家。一宿無話。
\end{parag}


\begin{parag}
    至次日,卯正二刻便過來了。那寧國府中婆娘媳婦聞得到齊,只見鳳姐正與來升媳婦分派,衆人不敢擅入,只在窗外聽覷。\begin{note}甲側:傳神之筆。\end{note}只聽鳳姐與來升媳婦道:“既託了我,我就說不得要討你們嫌了。\begin{note}甲側:先站地步。\end{note}我可比不得你們奶奶好性兒,由著你們去,再不要說你們‘這府裏原是這樣’的話,\begin{note}甲側:此話聽熟了。一嘆!\end{note}\begin{note}蒙側:“不要說”,“原是這樣的話”,破盡痼弊根底。\end{note}如今可要依著我行,\begin{note}甲側:婉轉得妙!\end{note}錯我半點兒,管不得誰是有臉的,誰是沒臉的,一例現清白處治。”說著,便吩咐彩明念花名冊,按名一個一個的喚進來看視。\begin{note}庚側:量才而用之意。\end{note}
\end{parag}


\begin{parag}
    一時看完,便又吩咐道:“這二十個分作兩班,一班十個,每日在裏頭單管人客來往倒茶,別的事不用他們管。這二十個也分作兩班,每日單管本家親戚茶飯,別的事也不用他們管。這四十個人也分作兩班,單在靈前上香添油,掛幔守靈,供茶供飯,隨起舉哀,別的事也不與他們相干。這四個人單在內茶房收管杯碟茶器,若少一件,便叫他四個人描賠。這四個人單管酒飯器皿,少一件,也是他四個人描賠。這八個人單管監收祭禮。這八個人單管各處燈油、蠟燭、紙札,我總支了來,交與你八個,然後按我的定數再往各處去分派。這三十個每日輪流各處上夜,照管門戶,監察火燭,打掃地方。這下剩的按著房屋分開,某人守某處,某處所有桌椅古董起,至於痰盒撣帚,一草一苗,或丟或壞,就和守這處的人算帳描賠。來升家的每日攬總查看,或有偷懶的,賭錢喫酒的,打架拌嘴的,立刻來回我。你有徇情,經我查出,三四輩子的老臉就顧不成了。如今都有定規,以後那一行亂了,只和那一行說話。素日跟我的人,隨身自有鐘錶,不論大小事,我是皆有一定的時辰。橫豎你們上房裏也有時辰鍾。卯正二刻我來點卯,巳正喫早飯,凡有領牌回事的,只在午初刻,戍初燒過黃昏紙,我親到各處查一遍,回來上夜的交明鑰匙。第二日仍是卯正二刻過來。說不得咱們大家辛苦這幾日,\begin{note}甲側:是協理口氣,好聽之至!\end{note}\begin{note}庚側:所謂先禮後兵是也。\end{note}事完了,你們家大爺自然賞你們。”\begin{note}庚側:滑賊,好收煞。\end{note}
\end{parag}


\begin{parag}
    說罷,又吩咐按數發與茶葉、油燭、雞毛撣子、笤帚等物。一面又搬取傢伙:桌圍、椅搭、坐褥、氈席、痰盒、腳踏之類。一面交發,一面提筆登記,某人管某處,某人領某物,開得十分清楚。衆人領了去,也都有了投奔,不似先時只揀便宜的做,剩下的苦差沒個招攬。各房中也不能趁亂失迷東西。便是人來客往,也都安靜了,不比先前一個正擺茶,又去端飯,正陪舉哀,又顧接客。如這些無頭緒,荒亂、推託、偷閒、竊取等弊,次日一概都蠲了。
\end{parag}


\begin{parag}
    鳳姐兒見自己威重令行,心中十分得意。因見尤氏犯病,賈珍又過於悲哀,不大進飲食,自己每日從那府中煎了各樣細粥,精緻小菜,命人送來勸食。\begin{note}庚眉:寫鳳之心機。\end{note}賈珍也另外吩咐每日送上等菜到抱廈內,單與鳳姐喫。\begin{note}庚眉:寫鳳之珍貴。\end{note}那鳳姐不畏勤勞,\begin{note}蒙雙夾:不畏勤勞者,一則任專而易辦,一則技癢而莫遏。士爲知己者死。不過勤勞,有何可畏?\end{note}天天於卯正二刻就過來點卯理事,\begin{note}庚眉:寫鳳之英勇。\end{note}獨在抱廈內起坐,不與衆妯娌合羣,便有堂客來往,也不迎會。\begin{note}庚眉:寫鳳之驕大。如此寫得可嘆可笑。\end{note}
\end{parag}


\begin{parag}
    這日乃五七正五日上,那應佛僧正開方破獄,傳燈照亡,參閻君,拘都鬼,延請地藏王,開金橋,引幢幡;那道士們正伏章申表,朝三清,叩玉帝;禪僧們行香,放焰口,拜水懺;又有十三衆尼僧,搭繡衣,趿紅鞋,在靈前默誦接引諸咒,十分熱鬧。那鳳姐必知今日人客不少,在家中歇宿一夜,至寅正,平兒便請起來梳洗。及收拾完備,更衣盥手,吃了幾口奶子糖粳粥,漱口已畢,已是卯正二刻了。來旺媳婦率領諸人伺候已久。鳳姐出至廳前,上了車,前面打了一對明角燈,大書 “榮國府”三個大字,款款來至寧府。大門上門燈朗掛,兩邊一色戳燈,照如白晝,白汪汪穿孝僕從兩邊侍立。請車至正門上,小廝等退去,衆媳婦上來揭起車簾。鳳姐下了車,一手扶著豐兒,兩個媳婦執著手把燈罩,簇擁著鳳姐進來。寧府諸媳婦迎來請安接待。鳳姐緩緩走入會芳園中登仙閣靈前,一見了棺材,那眼淚恰似斷線之珠,滾將下來。院中許多小廝垂手伺候燒紙。鳳姐吩咐得一聲:“供茶燒紙。”只聽一棒鑼嗚,諸樂齊奏,早有人端過一張大圈椅來,放在靈前,鳳姐坐了,放聲大哭。\begin{note}庚側:誰家行事,寧不墮淚?\end{note}於是裏外男女上下,見鳳姐出聲,都忙忙接聲嚎哭。
\end{parag}


\begin{parag}
    一時賈珍尤氏遣人來勸,鳳姐方纔止住。來旺媳婦獻茶漱口畢,鳳姐方起身,別過族中諸人,自入抱廈內來,按名查點,各項人數都已到齊,只有迎送親客上的一人未到。\begin{note}庚側:須得如此,方見文章妙用。餘前批非謬。\end{note}即命傳到。那人已張惶愧懼。鳳姐冷笑\begin{note}甲側:凡鳳姐惱時,偏偏用“笑”字,是章法。\end{note}道:“我說是誰誤了,原來是你!\begin{note}庚側:四字有神,是有名姓上等人口氣。\end{note}你原比他們有體面,所以纔不聽我的話。”那人道:“小的天天來的早,只有今兒,醒了覺得早些,因又睡迷了,來遲了一步,求奶奶饒過這次。”正說著,只見榮府中的王興媳婦來了,\begin{note}甲側:慣起波瀾,慣能忙中寫閒,又慣用曲筆,又慣綜錯,真妙!\end{note}\begin{note}庚側:偏用這等閒文間住。\end{note}在前探頭。
\end{parag}


\begin{parag}
    鳳姐且不發放這人,\begin{note}庚側:的是鳳姐作派。\end{note}卻先問:“王興媳婦作什麼?”王興媳婦巴不得先問他完了事,連忙進去說:“領牌取線,打車轎上網絡。”\begin{note}庚側:是喪事中用物,閒閒寫卻。\end{note}說著,將個帖兒遞上去。鳳姐命彩明念道:“大轎兩頂,小轎四頂,車四輛,共用大小絡子若干根,用珠兒線若干斤。”鳳姐聽了,數目相合,便命彩明登記,取榮國府對牌擲下。王興家的去了。
\end{parag}


\begin{parag}
    鳳姐方欲說話時,見榮國府的四個執事人進來,都是要支領東西領牌來的。鳳姐命彩明要了帖念過,聽了一共四件,指兩件說道:“這兩件開銷錯了,再算清了來取。”\begin{note}庚側:好看煞,這等文字。\end{note}說著擲下帖子來。那二人掃興而去。
\end{parag}


\begin{parag}
    鳳姐因見張材家的在旁,\begin{note}庚側:又一頓挫。\end{note}因問:“你有什麼事?”張材家的忙取帖兒回說:“就是方纔車轎圍作成,領取裁縫工銀若干兩。”鳳姐聽了,便收了帖子,命彩明登記。待王興家的交過牌,得了買辦的回押相符,然後方與張材家的去領。一面又命念那一個,是爲寶玉外書房完竣,支買紙料糊裱。\begin{note}庚側:卻從閒中,又引出一件關係文字來。\end{note}鳳姐聽了,即命收帖兒登記,待張材家的繳清,又發與這人去了。
\end{parag}


\begin{parag}
    鳳姐便說道:“明兒他也睡迷了,後兒我也睡迷了,\begin{note}甲側:接上文,一點痕跡俱無,且是仍與方纔諸人說話神色口角。庚側:接的緊,且無痕跡,是山斷雲連法也。\end{note}將來都沒了人了。本來要饒你,只是我頭一次寬了,下次人就難管,不如現開發的好。”登時放下臉來,喝令:“帶出去,打二十板子!”一面又擲下寧國府對牌:“出去說與來升,革他一月銀米!”衆人聽說,又見鳳姐眉立,\begin{note}庚側:二字如神。\end{note}知是惱了,不敢怠慢,拖人的出去拖人,執牌傳諭的忙去傳諭。那人身不由己,已拖出去捱了二十大板,還要進來叩謝。鳳姐道:“明日再有誤的,打四十,後日的六十,有捱打的,只管誤!”說著,吩咐:“散了罷。”窗外衆人聽說,方各自執事去了。彼時寧國榮國兩處執事領牌交牌的,人來人往不絕,那抱愧被打之人含羞去了,\begin{note}甲側:又伏下文,非獨爲阿鳳之威勢費此一段筆墨。\end{note}這才知道鳳姐利害。衆人不敢偷閒,自此兢兢業業,\begin{note}庚側:收拾得好。\end{note}執事保全。不在話下。
\end{parag}


\begin{parag}
    如今且說寶玉\begin{note}庚側:忙中閒筆。\end{note}因見今日人衆,恐秦鍾受了委曲,因默與他商議,要同他往鳳姐處來坐。秦鍾道:“他的事多,況且不喜人去,咱們去了,他豈不煩膩。”\begin{note}甲側:純是體貼人情。\end{note}寶玉道:“他怎好膩我們,不相干,只管跟我來。”說著,便拉了秦鍾,直至抱廈。鳳姐才喫飯,見他們來了,便笑道:“好長腿子,快上來罷。”寶玉道:“我們偏了。”\begin{note}庚側:家常戲言,畢肖之至!\end{note}鳳姐道:“在這邊外頭喫的,還是那邊喫的?”寶玉道:“這邊同那些渾人\begin{note}甲側:奇稱。試問誰是清人?\end{note}喫什麼!原是那邊,我們兩個同老太太吃了來的。”一面歸坐。
\end{parag}


\begin{parag}
    鳳姐喫畢,就有寧國府中的一個媳婦來領牌,爲支取香燈事。鳳姐笑道:“我算著你們今兒該來支取,總不見來,想是忘了。這會子到底來取,要忘了,自然是你們包出來,都便宜了我。”那媳婦笑道:“何嘗不是忘了,\begin{note}甲側:此婦亦善迎合。庚側:下人迎合湊趣,畢真。\end{note}方纔想起來,再遲一步,也領不成了!”說罷,領牌而去。
\end{parag}


\begin{parag}
    一時登記交牌。秦鍾因笑道:“你們兩府裏都是這牌,倘或別人私弄一個,支了銀子跑了,怎樣?”\begin{note}庚側:小人語。\end{note}鳳姐笑道:“依你說,都沒王法了。”寶玉道:“怎麼咱們家沒人領牌子做東西?”\begin{note}庚側:寫不理家務公子之語。\end{note}鳳姐道:“人家來領的時候,你還做夢呢。\begin{note}庚側:言甚是也。\end{note}我且問你,你們這夜書多早晚才念呢?”\begin{note}庚側:補前文之未到。\end{note}寶玉道:“巴不得這如今就唸纔好,他們只是不快給收拾出書房來,這也無法。”鳳姐笑道: “你請我一請,包管就快了。”寶玉道:“你要快也不中用。他們該作到那裏的,自然就有了。”鳳姐笑道:“便是他們作,也得要東西,擱不住我不給對牌是難的。”寶玉聽說,便猴\begin{note}庚側:詩中知有煉字一法,不期於《石頭記》中多得其妙。\end{note}向鳳姐身上立刻要牌,說:“好姐姐,給出牌子來,叫他們要東西去。” 鳳姐道:“我乏的身子上生疼,還擱的住揉搓。你放心罷,今兒才領了紙裱糊去了,他們該要的還等叫呢,可不傻了?”寶玉不信,鳳姐便叫彩明查冊子與寶玉看了。
\end{parag}


\begin{parag}
    正鬧著,人回:“蘇州去的人昭兒來了。”\begin{note}甲側:接得好!\end{note}鳳姐急命喚進來。昭兒打千兒請安。鳳姐便問:“回來做什麼的?”昭兒道:“二爺打發回來的。林姑老爺是九月初三日巳時沒的。\begin{note}甲眉:顰兒方可長居榮府之文。\end{note}二爺帶了林姑娘\begin{note}庚側:暗寫黛玉。\end{note}同送林姑老爺靈到蘇州,大約趕年底就回來。二爺打發小的來報個信請安,討老太太示下,還瞧瞧奶奶家裏好,叫把大毛服帶幾件去。”鳳姐道:“你見過別人了沒有?”昭兒道:“都見過了。”說畢,連忙退去。鳳姐向寶玉笑道:“你林妹妹可在咱們家住長了。”\begin{note}庚側:此係無意中之有意,妙!\end{note}寶玉道:“了不得,想來這幾日他不知哭的怎樣呢!”說著,蹙眉長嘆。
\end{parag}


\begin{parag}
    鳳姐見昭兒回來,因當著人未及細問賈璉,心中自是記掛,待要回去,爭奈事情繁雜,一時去了,恐有延遲失誤,惹人笑話。少不得耐到晚上回來,復令昭兒進來,細問一路平安信息。連夜打點大毛衣服,和平兒親自檢點包裹,再細細追想所需\begin{note}蒙側:“追想所需”四字,寫盡能事者之所以爲能事者之底蘊。\end{note}何物,一併包藏交付昭兒。又細細吩咐昭兒“在外好生小心伏侍,不要惹你二爺生氣;時時勸他少喫酒,別勾引他認得渾賬老婆,\begin{note}甲側:切心事耶?\end{note}”“回來打折你的腿”\begin{note}甲側:此一句最要緊。\end{note}等語。趕亂完了,天已四更將盡,總睡下又走了困,\begin{note}庚側:此爲病源伏線。後文方不突然。\end{note}不覺又是天明雞唱,忙梳洗過寧府中來。
\end{parag}


\begin{parag}
    那賈珍因見發引日近,親自坐車,帶了陰陽司吏,往鐵檻寺來踏看寄靈所在。又一一囑咐住持色空,好生領備新鮮陳設,多請名僧,以備接靈使用。色空忙看晚齋。賈珍也無心茶飯,因天晚不得進城,就在淨室胡亂歇了一夜。次日早,便進城來料理出殯之事,一面又派人先往鐵檻寺,連夜另外修飾停靈之處,並廚茶等項接靈人口坐落。
\end{parag}


\begin{parag}
    裏面鳳姐見日期有限,也預先逐細分派料理,一面又派榮府中車轎人從跟王夫人送殯,又顧自己送殯去佔下處。目今正值繕國公誥命亡故,王邢二夫人又去打祭送殯;西安郡王妃華誕,送壽禮;鎮國公誥命生了長男,預備賀禮;又有胞兄王仁連家眷回南,一面寫家信稟叩父母並帶往之物;又有迎春染病,每日請醫服藥,看醫生啓帖、症源、藥案等事,亦難盡述。又兼發引在邇,因此忙的鳳姐茶飯也沒工夫喫得,坐臥不得清淨。\begin{note}庚眉:總得好。\end{note}剛到了寧府,榮府的人又跟到寧府;既回到榮府,寧府的人又找到榮府。鳳姐見如此,心中倒十分歡喜,並不偷安推託,恐落人褒貶,因此日夜不暇,籌理得十分的整肅。於是合族上下無不稱歎者。
\end{parag}


\begin{parag}
    這日伴宿之夕,裏面兩班小戲並耍百戲的與親朋堂客伴宿,尤氏猶臥內於室,一應張羅款待,獨是鳳姐一人周全承應。合族中雖有許多妯娌,但或有羞口的,或有羞腳的,或有不慣見人的,也有懼貴怯官的,種種之類,俱不及鳳姐舉止舒徐,言語慷慨,珍貴寬大;因此也不把衆人放在眼裏,揮霍指示,任其所爲,目若無人。\begin{note}甲側:寫秦氏之喪,卻只爲鳳姐一人。\end{note}一夜中燈明火彩,客送官迎,那百般熱鬧,自不用說的。至天明,吉時已到,一般六十四名青衣請靈,前面銘旌上大書“奉天洪建兆年不易之朝\begin{note}庚眉:“兆年不易之朝,永治太平之國”,奇甚妙甚!\end{note}誥封一等寧國公冢孫婦防護內廷紫禁道御前侍衛龍禁尉享強壽賈門秦氏恭人之靈柩”。一應執事陳設,皆系現趕著新做出來的,一色光豔奪目。寶珠自行未嫁女之禮外,摔喪駕靈,十分哀苦。
\end{parag}


\begin{parag}
    那時官客送殯的,有鎮國公牛清之孫現襲一等伯牛繼宗,理國公柳彪之孫現襲一等子柳芳,齊國公陳翼之孫世襲三品威鎮將軍陳瑞文,治國公馬魁之孫世襲三品威遠將軍馬尚,修國公侯明之孫世襲一等子侯孝康;繕國公誥命亡故,其孫石光珠守孝不曾來得。\begin{note}庚眉:牛,醜也。清,屬水,子也。柳拆卯字。彪拆虎字,寅字寓焉。陳即辰。翼火爲蛇;巳字寓焉。馬,午也。魁拆鬼,鬼,金羊,未字寓焉。侯、猴同音,申也。曉鳴,雞也,酉字寓焉。石即豕,亥字寓焉。其祖曰守業,即守夜也,犬字寓焉。此所謂十二支寓焉。\end{note}這六家與榮寧二家,當日所稱“八公”的便是。餘者更有南安郡王之孫,西寧郡王之孫,忠靖侯史鼎,平原侯之孫世襲二等男蔣子寧,定城侯之孫世襲二等男兼京營遊擊謝鯨,襄陽侯之孫世襲二等男戚建輝,景田侯之孫五城兵馬司裘良。餘者錦鄉侯公子韓奇,神威將軍公子馮紫英,衛若蘭等諸王孫公子,不可枚數。堂客算來亦有十來頂大轎,三四十小轎,連家下大小轎車輛,不下百十餘乘。連前面各色執事、陳設、百耍,浩浩蕩蕩,一帶擺出三四里遠來。
\end{parag}


\begin{parag}
    走不多時,路旁綵棚高搭,設席張筵,和音奏樂,俱是各家路祭:第一座是王府東平王府祭棚,第二座是南安郡王祭棚,第三座是西寧郡王,第四座是北靜郡王的。原來這四王,當日惟北靜王功高,及今子孫猶襲王爵。現今北靜王水溶年未弱冠,生得形容秀美,性情謙和。近聞寧國公冢孫媳告殂,因想當日彼此祖父相與之情,同難同榮,未以異姓相視,因此不以王位自居,上日也曾探喪上祭,如今又設路祭,命麾下的各官在此伺候。自己五更入朝,公事一畢,便換了素服,坐大轎鳴鑼張傘而來,至棚前落轎。手下各官兩旁擁侍,軍民人衆不得往還。
\end{parag}


\begin{parag}
    一時只見府大殯浩浩蕩蕩、壓地銀山一般從北而至。\begin{note}庚眉:數字道盡聲勢。壬午春。畸笏老人。\end{note}早有寧府開路傳事人看見,連忙回去報與賈珍。賈珍急命前面駐紮,同賈赦賈政三人連忙迎來,以國禮相見。水溶在轎內欠身含笑答禮,仍以世交稱呼接待,並不妄自尊大。賈珍道:“犬婦之喪,累蒙郡駕下臨,廕生輩何以克當。” 水溶笑道:“世交之誼,何出此言。”遂回頭命長府官主祭代奠。賈赦等一旁還禮畢,復身又來謝恩。
\end{parag}


\begin{parag}
    水溶十分謙遜,因問賈政道:“那一位是銜玉而誕者?\begin{note}庚眉:忙中閒筆,點綴玉兄,方不是正文中之正人。作者良苦。壬午春。畸笏。\end{note}幾次要見一見,都爲雜冗所阻,想今日是來的,何不請來一會?”賈政聽說,忙回去,急命寶玉脫去孝服,領他前來。那寶玉素日就曾聽得父兄親友人等說閒話時,贊水溶是個賢王,\begin{note}蒙側:寶玉見北靜王,是爲後文伏線。\end{note}且生得才貌雙全,風流瀟灑,每不以官俗國體所縛。每思相會,只是父親拘束嚴密,無由得會,今日反來叫他,自是喜歡。一面走,一面早瞥見那水溶坐在轎內,好個儀表人才。不知近看時又是怎樣,且聽下回分解。
\end{parag}


\begin{parag}
    \begin{note}庚:此回將大家喪事詳細剔盡,如見其氣概,如聞其聲音,絲毫不錯,作者不負大家後裔。\end{note}
\end{parag}


\begin{parag}
    \begin{note}寫秦死之盛,賈珍之奢,實是卻寫得一個鳳姐。\end{note}
\end{parag}


\begin{parag}
    \begin{note}蒙:大抵事之不理,法之不行,多因偏於愛惡,幽柔不斷。請看鳳姐無私,猶能整齊喪事。況丈夫輩受職於廟堂之上,倘能奉公守法,一毫不苟,承上率下,何安不行?\end{note}
\end{parag}