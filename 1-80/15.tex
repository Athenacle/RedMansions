\chap{一十五}{王鳳姐弄權鐵檻寺 秦鯨卿得趣饅頭庵}


\begin{parag}
    \begin{note}甲:寶玉謁北靜王辭對神色,方露出本來面目,迥非在閨閣中之形景。\end{note}
\end{parag}


\begin{parag}
    \begin{note}甲:北靜王問玉上字果驗否,政老對以未曾試過,是隱卻多少捕風捉影閒文。\end{note}
\end{parag}


\begin{parag}
    \begin{note}甲:北靜王論聰明伶俐,又年幼時爲溺愛所累,亦大得病源之語。\end{note}
\end{parag}


\begin{parag}
    \begin{note}甲:鳳姐中火,寫紡線村姑,是寶玉閒花野景一得情趣。\end{note}
\end{parag}


\begin{parag}
    \begin{note}甲:鳳姐另住,明明系秦、玉、智能幽事,卻是爲淨虛鑽營鳳姐大大一件事作引。\end{note}
\end{parag}


\begin{parag}
    \begin{note}甲:秦、智幽情,忽寫寶、秦事雲:“不知算何賬目,未見真切,不曾記得,此係疑案,不敢纂創。”是不落套中,且省卻多少累贅筆墨。昔安南國使有題一丈紅句雲:“五尺牆頭遮不得,留將一半與人看。”\end{note}
\end{parag}


\begin{parag}
    \begin{note}蒙:欲顯錚錚不避嫌,英雄每入小人緣。鯨卿些子風流事,膽落魂銷已可憐。\end{note}
\end{parag}


\begin{parag}
    話說寶玉舉目見北靜王水溶頭上戴著潔白簪纓銀翅王帽,穿著江牙海水五爪坐龍白蟒袍,系著碧玉紅鞓帶,面如美玉,目似明星,真好秀麗人物。寶玉忙搶上來參見,水溶連忙從轎內伸出手來挽住。見寶玉戴著束髮銀冠,勒著雙龍出海抹額,穿著白蟒箭袖,圍著攢珠銀帶,面若春花,目如點漆。\begin{note}甲側:又換此一句,如見其形。\end{note}水溶笑道:“名不虛傳,果然如‘寶’似‘玉’。”\begin{note}靖本眉:傷心筆。\end{note}因問:“銜的那寶貝在那裏?”寶玉見問,連忙從衣內取了遞與過去。水溶細細的看了,又唸了那上頭的字,因問:“果靈驗否?”賈政忙道:“雖如此說,只是未曾試過。”水溶一面極口稱奇道異,一面理好彩絛,親自與寶玉帶上,\begin{note}甲側:鍾愛之至。\end{note}又攜手問寶玉幾歲,讀何書。寶玉一一的答應。
\end{parag}


\begin{parag}
    水溶見他語言清楚,談吐有致,\begin{note}庚眉:八字道盡玉兄,如此等方是玉兄正文寫照。壬午春。\end{note}一面又向賈政笑道:“令郎真乃龍駒鳳雛,非小王在世翁前唐突,將來‘雛鳳清於老鳳聲’,\begin{note}甲側:妙極!開口便是西昆體,寶玉聞之,寧不刮目哉?\end{note}未可量也。”賈政忙陪笑道:“犬子豈敢謬承金獎。賴藩郡餘禎 ,果如是言,亦廕生輩之幸矣。”\begin{note}庚側:謙的得體。\end{note}水溶又道:“只是一件,令郎如是資質,想老太夫人、夫人輩自然鍾愛極矣;但吾輩後生,甚不宜鍾溺,鍾溺則未免荒失學業。昔小王曾蹈此轍,想令郎亦未必不如是也。若令郎在家難以用功,不妨常到寒第。小王雖不才,卻多蒙海上衆名士凡至都者,未有不另垂青,是以寒第高人頗聚。令郎常去談會談會,則學問可以日進矣。”賈政忙躬身答應。
\end{parag}


\begin{parag}
    水溶又將腕上一串念珠卸了下來,遞與寶玉道:“今日初會,傖促竟無敬賀之物,此係前日聖上親賜鶺鴒香念珠一串,權爲賀敬之禮。”寶玉連忙接了,回身奉與賈政。\begin{note}庚側:轉出沒調教。\end{note}賈政與寶玉一齊謝過。於是賈赦、賈珍等一齊上來請回輿,水溶道:“逝者已登仙界,非碌碌你我塵寰中之人也。小王雖上叩天恩,虛邀郡襲,豈可越仙輀而進也?”賈赦等見執意不從,只得告辭謝恩回來,命手下掩樂停音,滔滔然將殯過完,\begin{note}庚側:有層次,好看煞。\end{note}方讓水溶回輿去了。不在話下。
\end{parag}


\begin{parag}
    且說寧府送殯,一路熱鬧非常。剛至城門前,又有賈赦、賈政、賈珍等諸同僚屬下各家祭棚接祭,一一的謝過,然後出城,竟奔鐵檻寺大路行來。彼時賈珍帶賈蓉來到諸長輩前讓坐轎上馬,因而賈赦一輩的各自上了車轎,賈珍一輩的也將要上馬。鳳姐兒因記掛著寶玉,\begin{note}甲側:千百件忙事內不漏一絲。\end{note}\begin{note}庚側:細心人自應如是。\end{note}怕他在郊外縱性逞強,不服家人的話,賈政管不著這些小事,惟恐有個失閃,難見賈母,因此便命小廝來喚他。寶玉只得來到他車前。鳳姐笑道:“好兄弟,你是個尊貴人,女孩兒一樣的人品,\begin{note}甲側:非此一句寶玉必不依,阿鳳真好才情。\end{note}別學他們猴在馬上。下來,咱們姐兒兩個坐車,豈不好?”寶玉聽說,忙下了馬,爬入鳳姐車上,二人說笑前進。
\end{parag}


\begin{parag}
    不一時,只見從那邊兩騎馬壓地飛來,\begin{note}庚側:有氣有聲,有形有影。\end{note}離鳳姐車不遠,一齊躥下來,扶車回說:“這裏有下處,奶奶請歇更衣。”鳳姐急命請邢夫人王夫人的示下,\begin{note}庚側:有次序。\end{note}那人回來說:“太太們說不用歇了,叫奶奶自便罷。”鳳姐聽了,便命歇了再走。衆小廝聽了,一帶轅馬,岔出人羣,往北飛走。寶玉在車內急命請秦相公。那時秦鍾正騎馬隨著他父親的轎,忽見寶玉的小廝跑來請他去打尖。秦鍾看時,只見鳳姐兒的車往北而去,後面拉著寶玉的馬,搭著鞍籠,便知寶玉同鳳姐坐車,自己也便帶馬趕上來,同入一莊門內。早有家人將衆莊漢攆盡。那莊農人家無多房舍,婆娘們無處迴避,只得由他們去了。那些村姑莊婦見了鳳姐、寶玉、秦鐘的人品衣服,禮數款段,豈有不愛看的?
\end{parag}


\begin{parag}
    一時鳳姐進入茅堂,因命寶玉等先出去頑頑。寶玉等會意,因同秦鍾出來,帶著小廝們各處遊頑。凡莊農動用之物,皆不曾見過。\begin{note}庚側:真,畢真!\end{note}寶玉一見了鍬、钁、鋤、犁等物,皆以爲奇,不知何項所使,其名爲何。\begin{note}甲側:凡膏粱子弟齊來著眼。\end{note}小廝在旁一一的告訴了名色,說明原委。\begin{note}甲側:也蓋因未見之故也。\end{note}寶玉聽了,因點頭嘆道:“怪道古人詩上說:‘誰知盤中餐,粒粒皆辛苦。’正爲此也。”\begin{note}甲側:聰明人自是一喝即悟。\end{note}\begin{note}庚眉:寫玉兄正文總於此等處,作者良苦。壬午季春。\end{note}一面說,一面又至一間房屋前,只見炕上有個紡車,寶玉又問小廝們:“這又是什麼?”小廝們又告訴他原委。寶玉聽說,便上來擰轉作耍,自爲有趣。只見一個約有十七八歲的村莊丫頭跑了來亂嚷:“別動壞了!”\begin{note}庚側:天生地設之文。\end{note}衆小廝忙斷喝攔阻,寶玉忙丟開手,陪笑說道:\begin{note}庚眉:一“忙”字,二“陪笑”字,寫玉兄是在女兒分上。壬午季春。\end{note}“我因爲沒見過這個,所以試他一試。”那丫頭道:“你們那裏會弄這個,站開了,\begin{note}甲側:如聞其聲,見其形。\end{note}\begin{note}庚側:三字如聞。\end{note}\begin{note}蒙側:這丫頭是技癢,是多情,是自己生活恐至損壞?寶玉此時一片心神,另有主張。\end{note}我紡與你瞧。”秦鍾暗拉寶玉笑道:“此卿大有意趣。”\begin{note}庚側:忙中閒筆;卻伏下文。\end{note}寶玉一把推開,笑道:“該死的!\begin{note}甲側:的是寶玉生性之言。\end{note}再胡說,我就打了!”\begin{note}庚側:玉兄身分本心如此。\end{note}說著,只見那丫頭紡起線來。寶玉正要說話時,\begin{note}庚眉:若說話,便不是《石頭記》中文字也。\end{note}只聽那邊老婆子叫道:“二丫頭,快過來!”那丫頭聽見,丟下紡車,一徑去了。
\end{parag}


\begin{parag}
    寶玉悵然無趣。\begin{note}甲側:處處點“情”,又伏下一段後文。\end{note}只見鳳姐兒打發人來叫他兩個進去。鳳姐洗了手,換衣服抖灰,問他們換不換。寶玉不換,只得罷了。家下僕婦們將帶著行路的茶壺茶杯、十錦屜盒、各樣小食端來,鳳姐等喫過茶,待他們收拾完備,便起身上車。外面旺兒預備下賞封,賞了那本村主人,莊婦等來叩賞。鳳姐並不在意,寶玉卻留心看時,內中並沒有二丫頭。\begin{note}庚側:妙在不見。\end{note}一時上了車,出來走不多遠,只見迎頭二丫頭懷裏抱著他小兄弟,\begin{note}庚側:妙在此時方見,錯綜之妙如此!\end{note}同著幾個小女孩子說笑而來。寶玉恨不得下車跟了他去,料是衆人不依的,少不得以目相送,爭奈車輕馬快,\begin{note}甲側:四字有文章。人生離聚亦未嘗不如此也。\end{note}一時展眼無蹤。
\end{parag}


\begin{parag}
    走不多時,仍又跟上大殯了。早有前面法鼓金鐃,幢幡寶蓋:鐵檻寺接靈衆僧齊至。少時到入寺中,另演佛事,重設香壇。安靈於內殿偏室之中,寶珠安於裏寢室相伴。外面賈珍款待一應親友,也有擾飯的,也有不喫飯而辭的,一應謝過乏,從公侯伯子男一起一起的散去,至未末時分方纔散盡了。裏面的堂客皆是鳳姐張羅接待,先從顯官誥命散起,也到晌午大錯時方散盡了。只有幾個親戚是至近的,等做過三日安靈道場方去。那時邢、王二夫人知鳳姐必不能來家,也便就要進城。王夫人要帶寶玉去,寶玉乍到郊外,那裏肯回去,只要跟鳳姐住著。王夫人無法,只得交與鳳姐便回來了。
\end{parag}


\begin{parag}
    原來這鐵檻寺原是寧榮二公當日修造,現今還是有香火地畝佈施,以備京中老了人口,在此便宜寄放。其中陰陽兩宅俱已預備妥貼,\begin{note}甲雙夾:大凡創業之人,無有不爲子孫深謀至細。奈後輩仗一時之榮顯,猶爲不足,另生枝葉,雖華麗過先,奈不常保,亦足可嘆,爭及先人之常保其樸哉!近世浮華子弟齊來著眼。\end{note}好爲送靈人口寄居。\begin{note}甲側:祖宗爲子孫之心細到如此!\end{note}\begin{note}庚眉:《石頭記》總於沒要緊處閒三二筆,寫正文筋骨。看官當用巨眼,不爲被瞞過方好。壬午季春。\end{note}不想如今後輩人口繁盛,其中貧富不一,或性情參商,\begin{note}甲雙夾:所謂“源遠水則濁,枝繁果則稀”。餘爲天下癡心祖宗爲子孫謀千年業者痛哭。\end{note}有那家業艱難安分的,\begin{note}甲側:妙在艱難就安分,富貴則不安分矣。\end{note}便住在這裏了;有那尚排場有錢勢的,只說這裏不方便,一定另外或村莊或尼庵尋個下處,爲事畢宴退之所。\begin{note}甲側:真真辜負祖宗體貼子孫之心。\end{note}即今秦氏之喪,族中諸人皆權在鐵檻寺下榻,獨有鳳姐嫌不方便,\begin{note}甲側:不用說,阿鳳自然不肯將就一刻的。\end{note}因而早遣人來和饅頭庵的姑子淨虛說了,騰出兩間房子來作下處。
\end{parag}


\begin{parag}
    原來這饅頭庵就是水月庵,因他廟裏做的饅頭好,就起了這個渾號,離鐵檻寺不遠。\begin{note}甲雙夾:前人詩云:“縱有千年鐵門限,終須一個土饅頭。”是此意。故“不遠”二字有文章。\end{note}當下和尚工課已完,奠過晚茶,賈珍便命賈蓉請鳳姐歇息。鳳姐見還有幾個妯娌們陪著女親,自己便辭了衆人,帶著寶玉、秦鍾往水月庵來。原來秦業年邁多病,\begin{note}甲側:伏一筆。\end{note}不能在此,只命秦鍾等待安靈罷了。那秦鍾便只跟著鳳姐、寶玉,一時到了水月庵,淨虛帶領智善、智能兩個徒弟出來迎接,大家見過。鳳姐等來至淨室更衣淨手畢,因見智能兒越發長高了,模樣兒越發出息了,因說道:“你們師徒怎麼這些日子也不往我們那裏去?”淨虛道:“可是這幾天都沒工夫,因胡老爺府裏產了公子,太太送了十兩銀子來這裏,叫請幾位師父念三日《血盆經》,忙的沒個空兒,就沒來請奶奶的安。”\begin{note}甲側:虛陪一個胡姓,妙!言是胡塗人之所爲也。\end{note}
\end{parag}


\begin{parag}
    不言老尼陪著鳳姐。且說秦鍾、寶玉二人正在殿上頑耍,因見智能過來,寶玉笑道:“能兒來了。”秦鍾道:“理那東西作什麼?”寶玉笑道:“你別弄鬼,那一日在老太太屋裏,一個人沒有,你摟著他作什麼呢?這會子還哄我。”\begin{note}甲側:補出前文未到處,細思秦鍾近日在榮府所爲可知矣。\end{note}秦鍾笑道:“這可是沒有的話。”寶玉笑道:“有沒有也不管你,你只叫他倒碗茶來我喫,就丟開手。”秦鍾笑道:“這又奇了,你叫他倒去,還怕他不倒?何必要我說呢。”寶玉道: “我叫他倒的是無情意的,不及你叫他倒的是有情意的。”\begin{note}甲側:總作如是等奇語。\end{note}秦鍾只得說道:“能兒,倒碗茶來給我。”那智能兒自幼在榮府走動,無人不識,因常與寶玉秦鍾頑笑。他如今大了,漸知風月,便看上了秦鍾人物風流,那秦鍾也極愛他妍媚,二人雖未上手,卻已情投意合了。\begin{note}甲側:不愛寶玉,卻愛案鍾,亦是各有情孽。\end{note}今智能見了秦鍾,心眼俱開,走去倒了茶來。秦鍾笑說:“給我。”\begin{note}甲側:如聞其聲。\end{note}寶玉叫:“給我!”智能兒抿著嘴笑道:“一碗茶也爭,我難道手裏有蜜!”\begin{note}甲側:一語畢肖,如聞其語,觀者已自酥倒,不知作者從何著想。\end{note}寶玉先搶得了,喫著,方要問話,只見智善來叫智能去擺茶碟子,一時來請他兩個去喫茶果點心。他兩個那裏喫這些東西?坐一坐仍出來頑耍。
\end{parag}


\begin{parag}
    鳳姐也略坐片時,便回至淨室歇息,老尼相送。此時衆婆娘媳婦見無事,都陸續散了,自去歇息,跟前不過幾個心腹常服侍小婢,老尼便趁機說道:“我下有一事,要到府裏求太太,先請奶奶一個示下。”鳳姐因問何事。老尼道:“阿彌陀佛!\begin{note}甲側:開口稱佛,畢肖。可嘆可笑!\end{note}只因當日我先在長安縣內善才庵\begin{note}甲側:“才”字妙。\end{note}內出家的時節,那時有個施主姓張,是大財主。他有個女兒小名金哥,\begin{note}甲側:俱從“財”一字上發出。\end{note}那年都往我廟裏來進香,不想遇見了長安府府太爺的小舅子李衙內。那李衙內一心看上,要娶金哥,打發人來求親,不想金哥已受了原任守備的公子的聘定。張家若退親,又怕守備不依,因此說已有了人家。誰知李公子執意不依,定要娶他女兒。張家正無計策,兩處爲難。不想守備家聽了此信,也不管青紅皁白,便來作踐辱罵,說一個女孩兒許幾家,偏不許退定禮,就打官司告狀起來。\begin{note}甲雙夾:守備一聞便問,斷無此理。此必是張家懼府尹之勢,必先退定禮,守備方不從,或有之。此時老尼,只欲與張家完事,故將此言遮飾,以便退親,受張家之賄也。\end{note}那張家急了,\begin{note}甲雙夾:如何便急了,話無頭緒,可知張家理缺。此係作者巧摹老尼無頭緒之語,莫認作者無頭緒,正是神處奇處。摹一人,一人必到紙上活現。\end{note}只得著人上京來尋門路,賭氣偏要退定禮。\begin{note}甲側:如何?的是張家要與府尹攀親!\end{note}我想如今長安節度雲老爺與府上最契,可以求太太與老爺說聲,打發一封書去,求雲老爺和那守備說聲,不怕那守備不依。若是肯行,張家連傾家孝順,也都情願。”\begin{note}甲雙夾:壞極,妙極!若與府尹攀了親,何惜張財不能再得?小人之心如此,良民遭害如此!\end{note}
\end{parag}


\begin{parag}
    鳳姐聽了笑道:“這事倒不大,\begin{note}甲側:五字是阿鳳心跡!\end{note}只是太太再不管這樣的事。”老尼道:“太太不管,奶奶也可以主張了。”鳳姐聽說笑道:“我也不等銀子使,也不做這樣的事。”\begin{note}庚側:口是心非,如聞已見。\end{note}淨虛聽了,打去妄想,半晌嘆\begin{note}庚側:一嘆轉出多少至惡不畏之文來。\end{note}道:“雖如此說,張家已知我來求府裏,如今不管這事,張家不知道沒工夫管這事,不希罕他的謝禮,倒像府裏連這點子手段也沒有的一般。”\begin{note}庚眉:閨閣營謀說事,往往被此等語惑了。\end{note}
\end{parag}


\begin{parag}
    鳳姐聽了這話,便發了興頭,說道:“你是素日知道我的,從來不信什麼是陰司地獄報應的,\begin{note}庚側:批書人深知卿有是心,嘆嘆!\end{note}憑是什麼事,我說要行就行。你叫他拿三千銀子來,我就替他出這口氣。”老尼聽說,喜不自禁,忙說:“有!有!這個不難。”鳳姐又道:“我比不得他們扯篷拉縴的圖銀子。\begin{note}庚側:欺人太甚。\end{note}這三千銀子,不過是給打發說去的小廝作盤纏,使他賺幾個辛苦錢,我一個錢也不要他的。\begin{note}庚眉:對如是之奸尼,阿鳳不得不如是語。\end{note}便是三萬兩,我此刻也拿的出來。”\begin{note}甲側:阿鳳欺人如此。\end{note}老尼連忙答應,又說道:“既如此,奶奶明日就開恩也罷了。”鳳姐道:“你瞧瞧我忙的,那一處少了我?既應了你,自然快快的了結。”老尼道:“這點子事,別人的跟前就忙的不知怎麼樣,若是奶奶的跟前,再添上些也不夠奶奶一發揮的。\begin{note}蒙側:“若是奶奶”等語,陷害殺無窮英明豪烈者。譽而不喜,毀而不怒,或可逃此等術法。\end{note}只是俗語說的‘能者多勞’,太太因大小事見奶奶妥貼,越發都推給奶奶了,奶奶也要保重金體纔是。”一路話奉承的鳳姐越發受用,也不顧勞乏,更攀談起來。\begin{note}甲側:總寫阿鳳聰明中的癡人。\end{note}
\end{parag}


\begin{parag}
    誰想秦鍾趁黑無人,來尋智能。剛至後面房中,只見智能獨在房中洗茶碗,秦鍾跑來便摟著親嘴。智能兒急的跺腳說:“這算什麼!再這麼我就叫喚。”秦鍾求道:“好人,我已急死了。你今兒再不依,我就死在這裏。”智能道:“你想怎樣?除非我出了這牢坑,離了這些人,才依你。”秦鍾道:“這也容易,只是遠水救不得近渴。”說著,一口吹了燈,滿屋漆黑,將智能抱到炕上,就雲雨起來。\begin{note}庚側:小風波事,亦在人意外。誰知爲小秦伏線,大有根處。\end{note}\begin{note}庚眉:實表姦淫,尼庵之事如此。壬午季春。\end{note}\begin{note}庚批:又寫秦鍾智能事,尼庵之事如此。壬午季春。畸笏。\end{note}那智能百般的掙挫不起,又不好叫的,\begin{note}庚側:還是不肯叫。\end{note}少不得依他了。正在得趣,只見一人進來,將他二人按住,也不則聲。二人不知是誰,唬的不敢動一動。只聽那人嗤的一聲,掌不住笑了,\begin{note}庚側:請掩卷細思此刻形景,真可噴飯。歷來風月文字可有如此趣味者?\end{note}二人聽聲方知是寶玉。秦鍾連忙起來,抱怨道:“這算什麼?”寶玉笑道:“你倒不依,咱們就喊起來。”羞的智能趁黑地跑了。\begin{note}庚眉:若歷寫完,則不是《石頭記》文字了,壬午季春。\end{note}寶玉拉了秦鍾出來道:“你可還和我強?”\begin{note}蒙側:請問此等光景,是強是順?一片兒女之態,自與凡常不同。細極,妙極!\end{note}秦鍾笑道:“好人,\begin{note}庚側:前以二字稱智能,今又稱玉兄,看官細思。\end{note}你只別嚷的衆人知道,你要怎樣我都依你。”寶玉笑道:“這會子也不用說,等一會睡下,再細細的算帳。”一時寬衣要安歇的時節,鳳姐在裏間,秦鍾寶玉在外間,滿地下皆是家下婆子,打鋪坐更。鳳姐因怕通靈玉失落,便等寶玉睡下,命人拿來塞在自己枕邊。寶玉不知與秦鍾算何帳目,未見真切,未曾記得,此係疑案,不敢纂創。\begin{note}甲雙夾:忽又作如此評斷,似自相矛盾,卻是最妙之文。若不如此隱去,則又有何妙文可寫哉?這方是世人意料不到之大奇筆。若通部中萬萬件細微之事懼備,《石頭記》真亦太覺死板矣。故特因此二三件隱事,指石之未見真切,淡淡隱去,越覺得雲煙渺茫之中,無限丘壑在焉。\end{note}
\end{parag}


\begin{parag}
    一宿無話,至次日一早,便有賈母王夫人打發了人來看寶玉,又命多穿兩件衣服,無事寧可回去。寶玉那裏肯回去,又有秦鍾戀著智能,調唆寶玉求鳳姐再住一天。鳳姐想了一想:\begin{note}甲側:一想便有許多的好處。真好阿鳳!\end{note}凡喪儀大事雖妥,還有一半點小事未曾安插,可以指此再住一日,豈不又在賈珍跟前送了滿情;二則又可以完淨虛那事;三則順了寶玉的心,賈母聽見,豈不歡喜?因有此三益,\begin{note}甲側:世人只雲一舉兩得,獨阿鳳一舉更添一。\end{note}便向寶玉道:“我的事都完了,你要在這裏逛,少不得索性辛苦一日罷了,明兒可是定要走的了。”寶玉聽說,千姐姐萬姐姐的央求:“只住一日,明兒回去的。”於是又住了一夜。
\end{parag}


\begin{parag}
    鳳姐便命悄悄將昨日老尼之事,說與來旺兒。來旺兒心中俱已明白,急忙進城找著主文的相公,假託賈璉所囑,修書一封,\begin{note}甲側:不細。\end{note}連夜往長安縣來,不過百里路程,兩日工夫俱已妥協。那節度使名喚雲光,久受賈府之情,這點小事,豈有不允之理,給了回書,旺兒回來。且不在話下。\begin{note}甲側:一語過下。\end{note}
\end{parag}


\begin{parag}
    卻說鳳姐等又過了一日,次日方別了老尼,著他三日後往府裏去討信。\begin{note}甲側:過至下回。\end{note}那秦鍾與智能百般不忍分離,背地裏多少幽期密約,俱不用細述,只得含恨而別。鳳姐又到鐵檻寺中照望一番。寶珠執意不肯回家,賈珍只得派婦女相伴。後回再見。
\end{parag}


\begin{parag}
    \begin{note}蒙:請看作者寫勢利之情,亦必因激動;寫兒女之情,偏生含蓄不吐,可謂細針密縫。其述說一段,言語形跡無不逼真,聖手神文,敢不薰沐拜讀?\end{note}
\end{parag}