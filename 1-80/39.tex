\chap{三十九}{村老嫗荒談承色笑 癡情子實意覓蹤跡}

\begin{parag}
    \begin{note}蒙回前總:只爲貧寒不揀行,當家趨入且逢迎。豈知著意無名利,便是三才最上層。\end{note}
\end{parag}


\begin{parag}
    話說衆人見平兒來了,都說:“你們奶奶作什麼呢,怎麼不來了?”平兒笑道:“他那裏得空兒來。因爲說沒有好生喫得,又不得來,所以叫我來問還有沒有,叫我要幾個拿了家去喫罷。”湘雲道:“有,多著呢。”忙令人拿了十個極大的。平兒道:“多拿幾個團臍的。”衆人又拉平兒坐,平兒不肯。李紈拉著他笑道: “偏要你坐。”拉著他身邊坐下,端了一杯酒送到他嘴邊。平兒忙喝了一口就要走。李紈道:“偏不許你去。顯見得只有鳳丫頭,就不聽我的話了。”說著又命嬤嬤們:“先送了盒子去,就說我留下平兒了。”那婆子一時拿了盒子回來說:“二奶奶說,叫奶奶和姑娘們別笑話要嘴喫。這個盒子裏是方纔舅太太那裏送來的菱粉糕和雞油卷兒,給奶奶姑娘們喫的。”又向平兒道:“說使你來你就貪住頑不去了。勸你少喝一杯兒罷。”平兒笑道:“多喝了又把我怎麼樣?”一面說,一面只管喝,又喫螃蟹。李紈攬著他笑道:“可惜這麼個好體面模樣兒,命卻平常,只落得屋裏使喚。不知道的人,誰不拿你當作奶奶太太看。”
\end{parag}


\begin{parag}
    平兒一面和寶釵湘雲等喫喝,一面回頭笑道:“奶奶,別隻摸的我怪癢的。”李氏道:“噯喲!這硬的是什麼?”平兒道:“鑰匙。”李氏道:“什麼鑰匙?要緊梯己東西怕人偷了去,卻帶在身上。我成日家和人說笑,有個唐僧取經,就有個白馬來馱他;劉智遠打天下,就有個瓜精來送盔甲;有個鳳丫頭,就有個你。你就是你奶奶的一把總鑰匙,還要這鑰匙作什麼。”平兒笑道:“奶奶吃了酒,又拿了我來打趣著取笑兒了。”寶釵笑道:“這倒是真話。我們沒事評論起人來,你們這幾個都是百個裏頭挑不出一個來,妙在各人有各人的好處。”李紈道:“大小都有個天理。比如老太太屋裏,要沒那個鴛鴦如何使得。從太太起,那一個敢駁老太太的回,現在他敢駁回。偏老太太只聽他一個人的話。老太太那些穿戴的,別人不記得,他都記得,要不是他經管著,不知叫人誆騙了多少去呢。那孩子心也公道,雖然這樣,倒常替人說好話兒,還倒不依勢欺人的。”惜春笑道:“老太太昨兒還說呢,他比我們還強呢。”平兒道:“那原是個好的,我們那裏比的上他。”寶玉道:“太太屋裏的彩霞,是個老實人。”探春道:“可不是,外頭老實,心裏有數兒。太太是那麼佛爺似的,事情上不留心,他都知道。凡百一應事都是他提著太太行。連老爺在家出外去的一應大小事,他都知道。太太忘了,他背地裏告訴太太。”李紈道:“那也罷了。”指著寶玉道:“這一個小爺屋裏要不是襲人,你們度量到個什麼田地!鳳丫頭就是楚霸王,也得這兩隻膀子好舉千斤鼎。他不是這丫頭,就得這麼周到了!”平兒笑道:“先時陪了四個丫頭,死的死,去的去,只剩下我一個孤鬼了。”李紈道:“你倒是有造化的。鳳丫頭也是有造化的。想當初你珠大爺在日,何曾也沒兩個人。你們看我還是那容不下人的?天天只見他兩個不自在。所以你珠大爺一沒了,趁年輕我都打發了。若有一個守得住,我倒有個膀臂。”說著滴下淚來。衆人都道:“又何必傷心,不如散了倒好。”說著便都洗了手,大家約往賈母王夫人處問安。
\end{parag}


\begin{parag}
    衆婆子丫頭打掃亭子,收拾杯盤。襲人和平兒同往前去,讓平兒到房裏坐坐,再喝一杯茶。平兒說:“不喝茶了,再來吧。”說著便要出去。襲人又叫住問道: “這個月的月錢,連老太太和太太還沒放呢,是爲什麼?”平兒見問,忙轉身至襲人跟前,見方近無人,才悄悄說道:“你快別問,橫豎再遲幾天就放了。”襲人笑道:“這是爲什麼,唬得你這樣?”平兒悄悄告訴他道:“這個月的月錢,我們奶奶早已支了,放給人使呢。等別處的利錢收了來,湊齊了才放呢。因爲是你,我才告訴你,你可不許告訴一個人去。”襲人道:“難道他還短錢使,還沒個足厭?何苦還操這心。”平兒笑道:“何曾不是呢。這幾年拿著這一項銀子,翻出有幾百來了。他的公費月例又使不著,十兩八兩零碎攢了放出去,只他這梯己利錢,一年不到,上千的銀子呢。”襲人笑道:“拿著我們的錢,你們主子奴才賺利錢,哄的我們呆呆的等著。”平兒道:“你又說沒良心的話。你難道還少錢使?”襲人道:“我雖不少,只是我也沒地方使去,就只預備我們那一個。”平兒道:“你倘若有要緊的事用錢使時,我那裏還有幾兩銀子,你先拿來使,明兒我扣下你的就是了。”襲人道:“此時也用不著,怕一時要用起來不夠了,我打發人去取就是了。”
\end{parag}


\begin{parag}
    平兒答應著,一徑出了園門,來至家內,只見鳳姐兒不在房裏。忽見上回來打抽豐的那劉姥姥和板兒又來了,坐在那邊屋裏,還有張材家的周瑞家的陪著,又有兩三個丫頭在地下倒口袋裏的棗子倭瓜並些野菜。衆人見他進來,都忙站起來了。\begin{note}庚雙夾:妙文!上回是先見平兒後見鳳姐,此則先見鳳姐後見平兒也。何錯綜巧妙得情得理之至耶?\end{note}劉姥姥因上次來過,知道平兒的身分,忙跳下地來問“姑娘好”,又說:“家裏都問好。早要來請姑奶奶的安看姑娘來的,因爲莊家忙。好容易今年多打了兩石糧食,瓜果菜蔬也豐盛。這是頭一起摘下來的,並沒敢賣呢,留的尖兒孝敬姑奶奶姑娘們嚐嚐。姑娘們天天山珍海味的也喫膩了,這個喫個野意兒,也算是我們的窮心。” 平兒忙道:“多謝費心。”又讓坐,自己也坐了。又讓“張嬸子周大娘坐”,又令小丫頭子倒茶去。周瑞張材兩家的因笑道:“姑娘今兒臉上有些春色,眼圈兒都紅了。”平兒笑道:“可不是。我原是不喫的,大奶奶和姑娘們只是拉著死灌,不得已喝了兩盅,臉就紅了。”張材家的笑道:“我倒想著要喫呢,又沒人讓我。明兒再有人請姑娘,可帶了我去罷。”說著大家都笑了。周瑞家的道:“早起我就看見那螃蟹了,一斤只好秤兩個三個。這麼三大簍,想是有七八十斤呢。”周瑞家的道:“若是上上下下只怕還不夠。”平兒道:“那裏夠,不過都是有名兒的喫兩個子。那些散衆的,也有摸得著的,也有摸不著的。”劉姥姥道:“這樣螃蟹,今年就值五分一斤。十斤五錢,五五二兩五,三五一十五,再搭上酒菜,一共倒有二十多兩銀子。阿彌陀佛!這一頓的錢夠我們莊家人過一年了。”平兒因問:“想是見過奶奶了?”\begin{note}庚雙夾:寫平兒伶俐如此。\end{note}劉姥姥道:“見過了,叫我們等著呢。”說著又往窗外看天氣,\begin{note}庚雙夾:是八月中當開窗時,細緻之甚。\end{note}說道:“天好早晚了,我們也去罷,別出不去城纔是饑荒呢。”周瑞家的道:“這話倒是,我替你瞧瞧去。”說著一徑去了,半日方來,笑道:“可是你老的福來了,竟投了這兩個人的緣了。”平兒等問怎麼樣,周瑞家的笑道:“二奶奶在老太太的跟前呢。我原是悄悄的告訴二奶奶,‘劉姥姥要家去呢,怕晚了趕不出城去。’二奶奶說:‘大遠的,難爲他扛了那些沉東西來,晚了就住一夜明兒再去。’這可不是投上二奶奶的緣了。這也罷了,偏生老太太又聽見了,問劉姥姥是誰。二奶奶便回明白了。老太太說:‘我正想個積古的老人家說話兒,請了來我見一見。’這可不是想不到天上緣分了。”說著,催劉姥姥下來前去。劉姥姥道:“我這生像兒怎好見的。好嫂子,你就說我去了罷。”平兒忙道:“你快去罷,不相干的。我們老太太最是惜老憐貧的,比不得那個狂三詐四的那些人。想是你怯上,我和周大娘送你去。”說著,同周瑞家的引了劉姥姥往賈母這邊來。
\end{parag}


\begin{parag}
    二門口該班的小廝們見了平兒出來,都站起來了,又有兩個跑上來,趕著平兒叫“姑娘”。\begin{note}庚雙夾:想這一個“姑娘”非下稱上之“姑娘”也,按北俗以姑母曰“姑姑”,南俗曰“娘娘”,此“姑娘”定是“姑姑”“娘娘”之稱。每見大家風俗多有小童稱少主妾曰“姑姑”“娘娘”者。按此書中若干人說話語氣及動用前照飲食諸項,皆東南西北互相兼用,此“姑娘”之稱亦南北相兼而用無疑矣。\end{note}平兒問:“又說什麼?”那小廝笑道:“這會子也好早晚了,我媽病了,等著我去請大夫。好姑娘,我討半日假可使的?”平兒道:“你們倒好,都商議定了,一天一個告假,又不回奶奶,只和我胡纏。前兒住兒去了,二爺偏生叫他,叫不著,我應起來了,還說我作了情。你今兒又來了。”\begin{note}庚雙夾:分明幾回沒寫到賈璉,今忽閒中一語便補得賈璉這邊天天熱鬧,令人卻如看見聽見一般。所謂不寫之寫也。劉姥姥眼中耳中又一番識面,奇妙之甚!\end{note}周瑞家的道:“當真的他媽病了,姑娘也替他應著,放了他罷。”平兒道:“明兒一早來。聽著,我還要使你呢,再睡的日頭曬著屁股再來!你這一去,帶個信兒給旺兒,就說奶奶的話,問著他那剩的利錢。明兒若不交了來,奶奶也不要了,就越性送他使罷。”\begin{note}庚雙夾:交代過襲人的話,看他如此說,真比鳳姐又甚一層。李紈之語不謬也。不知阿鳳何等福得此一人。\end{note}那小廝歡天喜地答應去了。
\end{parag}


\begin{parag}
    平兒等來至賈母房中,彼時大觀園中姊妹們都在賈母前承奉。\begin{note}庚雙夾:妙極!連寶玉一併類入姊妹隊中了。\end{note}劉姥姥進去,只見滿屋裏珠圍翠繞,花枝招展,並不知都系何人。只見一張榻上歪著一位老婆婆,身後坐著一個紗羅裹的美人一般的一個丫鬟在那裏捶腿,鳳姐兒站著正說笑。\begin{note}庚雙夾:奇奇怪怪文章。在劉姥姥眼中以爲阿鳳至尊至貴,普天下人獨該站著說,阿鳳獨坐纔是。如何今見阿鳳獨站哉?真妙文字。\end{note}劉姥姥便知是賈母了,忙上來陪著笑,福了幾福,口裏說:“請老壽星安。”\begin{note}庚雙夾:更妙!賈母之號何其多耶?在諸人口中則曰“老太太”,在阿鳳口中則曰“老祖宗”,在僧尼口中則曰“老菩薩”,在劉姥姥口中則曰“老壽星”者,卻似有數人,想去則皆賈母,難得如此各盡其妙,劉姥姥亦善應接。\end{note}賈母亦欠身問好,又命周瑞家的端過椅子來坐著。那板兒仍是怯人,不知問候。\begin{note}庚雙夾:“仍”字妙!蓋有上文故也。不知教訓者來看此句。\end{note}賈母道:“老親家,你今年多大年紀了?”劉姥姥忙立身答道:“我今年七十五了。”賈母向衆人道:“這麼大年紀了,還這麼健朗。比我大好幾歲呢。我要到這麼大年紀,還不知怎麼動不得呢。”劉姥姥笑道:“我們生來是受苦的人,老太太生來是享福的。若我們也這樣,那些莊家活也沒人作了。”賈母道:“眼睛牙齒都還好?”劉姥姥道:“都還好,就是今年左邊的槽牙活動了。”賈母道: “我老了,都不中用了,眼也花,耳也聾,記性也沒了。你們這些老親戚,我都不記得了。親戚們來了,我怕人笑我,我都不會,不過嚼的動的喫兩口,睡一覺,悶了時和這些孫子孫女兒頑笑一回就完了。”劉姥姥笑道:“這正是老太太的福了。我們想這麼著也不能。”賈母道:“什麼福,不過是個老廢物罷了。”說的大家都笑了。賈母又笑道:“我才聽見鳳哥兒說,你帶了好些瓜菜來,叫他快收拾去了,我正想個地裏現擷的瓜兒菜兒喫。外頭買的,不像你們田地裏的好喫。”劉姥姥笑道:“這是野意兒,不過喫個新鮮。依我們想魚肉喫,只是喫不起。”賈母又道:“今兒既認著了親,別空空兒的就去。不嫌我這裏,就住一兩天再去。我們也有個園子,園子裏頭也有果子,你明日也嚐嚐,帶些家去,你也算看親戚一趟。”鳳姐兒見賈母喜歡,也忙留道:“我們這裏雖不比你們的場院大,空屋子還有兩間。你住兩天罷,把你們那裏的新聞故事兒說些與我們老太太聽聽。”賈母笑道:“鳳丫頭別拿他取笑兒。他是鄉屯裏的人,老實,那裏擱的住你打趣他。”說著,又命人去先抓果子與板兒喫。板兒見人多了,又不敢喫。賈母又命拿些錢給他,叫小幺兒們帶他外頭頑去。劉姥姥吃了茶,便把些鄉村中所見所聞的事情說與賈母,賈母益發得了趣味。正說著,鳳姐兒便令人來請劉姥姥喫晚飯。賈母又將自己的菜揀了幾樣,命人送過去與劉姥姥喫。
\end{parag}


\begin{parag}
    鳳姐知道合了賈母的心,吃了飯便又打發過來。鴛鴦忙令老婆子帶了劉姥姥去洗了澡,自己挑了兩件隨常的衣服令給劉姥姥換上。\begin{note}庚雙夾:一段寫鴛鴦身份權勢心機,只寫賈母也。\end{note}那劉姥姥那裏見過這般行事,忙換了衣裳出來,坐在賈母榻前,又搜尋些話出來說。彼時寶玉姊妹們也都在這裏坐著,他們何曾聽見過這些話,自覺比那些瞽目先生說的書還好聽。那劉姥姥雖是個村野人,卻生來的有些見識,況且年紀老了,世情上經歷過的,見頭一個賈母高興,第二見這些哥兒姐兒們都愛聽,便沒了說的也編出些話來講。因說道:“我們村莊上種地種菜,每年每日,春夏秋冬,風裏雨裏,那有個坐著的空兒,天天都是在那地頭子上作歇馬涼亭,什麼奇奇怪怪的事不見呢。就象去年冬天,接連下了幾天雪,地下壓了三四尺深。我那日起的早,還沒出房門,只聽外頭柴草響。我想著必定是有人偷柴草來了。我爬著窗戶眼兒一瞧,卻不是我們村莊上的人。”賈母道:“必定是過路的客人們冷了,見現成的柴,抽些烤火去也是有的。”劉姥姥笑道:“也並不是客人,所以說來奇怪。老壽星當個什麼人?原來是一個十七八歲的極標緻的一個小姑娘,梳著溜油光的頭,穿著大紅襖兒,白綾裙子──”\begin{note}庚雙夾:劉姥姥的口氣如此。\end{note}剛說到這裏,忽聽外面人吵嚷起來,又說:“不相干的,別唬著老太太。”賈母等聽了,忙問怎麼了,丫鬟回說:“南院馬棚裏走了水,不相干,已經救下去了。”賈母最膽小的,聽了這個話,忙起身扶了人出至廊上來瞧,只見東南上火光猶亮。賈母唬的口內唸佛,忙命人去火神跟前燒香。王夫人等也忙都過來請安,又回說“已經下去了,老太太請進房去罷。”賈母足的看著火光息了方領衆人進來。\begin{note}庚雙夾:一段爲後回作引,然偏於寶玉愛聽時截住。\end{note}寶玉且忙著問劉姥姥:“那女孩兒大雪地作什麼抽柴草?倘或凍出病來呢?”賈母道:“都是才說抽柴草惹出火來了,你還問呢。別說這個了,再說別的罷。”寶玉聽說,心內雖不樂,也只得罷了。劉姥姥便又想了一篇,說道:“我們莊子東邊莊上,有個老奶奶子,今年九十多歲了。他天天喫齋唸佛,誰知就感動了觀音菩薩夜裏來託夢說: ‘你這樣虔心,原來你該絕後的,如今奏了玉皇,給你個孫子。’原來這老奶奶只有一個兒子,這兒子也只一個兒子,好容易養到十七八歲上死了,哭的什麼似的。後果然又養了一個,今年才十三四歲,生的雪團兒一般,聰明伶俐非常。可見這些神佛是有的。”這一夕話,實合了賈母王夫人的心事,連王夫人也都聽住了。
\end{parag}


\begin{parag}
    寶玉心中只記掛著抽柴的故事,因悶悶的心中籌劃。探春因問他:“昨日擾了史大妹妹,咱們回去商議著邀一社,又還了席,也請老太太賞菊花,何如?”寶玉笑道:“老太太說了,還要擺酒還史妹妹的席,叫咱們作陪呢。等著吃了老太太的,咱們再請不遲。”探春道:“越往前去越冷了,老太太未必高興。”寶玉道: “老太太又喜歡下雨下雪的。不如咱們等下頭場雪,請老太太賞雪豈不好?咱們雪下吟詩,也更有趣了。”林黛玉忙笑道:“咱們雪下吟詩?依我說,還不如弄一捆柴火,雪下抽柴,還更有趣兒呢。”說著,寶釵等都笑了。寶玉瞅了他一眼,也不答話。
\end{parag}


\begin{parag}
    一時散了,背地裏寶玉足的拉了劉姥姥,細問那女孩兒是誰。劉姥姥只得編了告訴他道:“那原是我們莊北沿地埂子上有一個小祠堂裏供的,不是神佛,當先有個什麼老爺。”說著又想名姓。寶玉道:“不拘什麼名姓,你不必想了,只說原故就是了。”劉姥姥道:“這老爺沒有兒子,只有一位小姐,名叫茗玉。小姐知書識字,老爺太太愛如珍寶。可惜這茗玉小姐生到十七歲,一病死了。”寶玉聽了,跌足嘆惜,又問後來怎麼樣。劉姥姥道:“因爲老爺太太思念不盡,便蓋了這祠堂,塑了這茗玉小姐的像,派了人燒香撥火。如今日久年深的,人也沒了,廟也爛了,那個像就成了精。”寶玉忙道:“不是成精,規矩這樣人是雖死不死的。”劉姥姥道:“阿彌陀佛!原來如此。不是哥兒說,我們都當他成精。他時常變了人出來各村莊店道上閒逛。我才說這抽柴火的就是他了。我們村莊上的人還商議著要打了這塑像平了廟呢。”寶玉忙道:“快別如此。若平了廟,罪過不小。”劉姥姥道:“幸虧哥兒告訴我,我明兒回去告訴他們就是了。”寶玉道:“我們老太太、太太都是善人,合家大小也都好善喜舍,最愛修廟塑神的。我明兒做一個疏頭,替你化些佈施,你就做香頭,攢了錢把這廟修蓋,再裝潢了泥像,每月給你香火錢燒香豈不好?”劉姥姥道:“若這樣,我託那小姐的福,也有幾個錢使了。”寶玉又問他地名莊名,來往遠近,坐落何方。劉姥姥便順口胡謅了出來。
\end{parag}


\begin{parag}
    寶玉信以爲真,回至房中,盤算了一夜。次日一早,便出來給了茗煙幾百錢,按著劉姥姥說的方向地名,著茗煙去先踏看明白,回來再做主意。那茗煙去後,寶玉左等也不來,右等也不來,急的熱鍋上的螞蟻一般。好容易等到日落,方見茗煙興興頭頭的回來。寶玉忙道:“可有廟了?”茗煙笑道:“爺聽的不明白,叫我好找。那地名座落不似爺說的一樣,所以找了一日,找到東北上田埂子上纔有一個破廟。”寶玉聽說,喜的眉開眼笑,忙說道:“劉姥姥有年紀的人,一時錯記了也是有的。你且說你見的。”茗煙道:“那廟門卻倒是朝南開,也是稀破的。我找的正沒好氣,一見這個,我說‘可好了’,連忙進去。一看泥胎,唬的我跑出來了,活似真的一般。”寶玉喜的笑道:“他能變化人了,自然有些生氣。”茗煙拍手道:“那裏有什麼女孩兒,竟是一位青臉紅髮的瘟神爺。”寶玉聽了,啐了一口,罵道:“真是一個無用的殺才!這點子事也幹不來。”茗煙道:“二爺又不知看了什麼書,或者聽了誰的混話,信真了,把這件沒頭腦的事派我去碰頭,怎麼說我沒用呢?”寶玉見他急了,忙撫慰他道:“你別急。改日閒了你再找去。若是他哄我們呢,自然沒了,若真是有的,你豈不也積了陰騭。我必重重的賞你。”正說著,只見二門上的小廝來說:“老太太房裏的姑娘們站在二門口找二爺呢。”
\end{parag}


\begin{parag}
    \begin{note}蒙回末總:此回第一寫勢利之好財,第二寫窮苦趨勢之求財。且文章不得雷同,先既有杜詩,而今不得不用套坡公之遺事,以振其餘響即此,以點染寶玉之癡。其文真如環轉,無端倪可指。\end{note}
\end{parag}

