\chap{一百一十六}{得通靈幻境悟仙緣 送慈柩故鄉全孝道}



\begin{parag}
    話說寶玉一聽麝月的話,身往後仰,復又死去,急得王夫人等哭叫不止。麝月自知失言致禍,此時王夫人等也不及說他。那麝月一面哭着,一面打定主意,心想:“若是寶玉一死,我便自盡跟了他去!”不言麝月心裏的事。且言王夫人等見叫不回來,趕着叫人出來找和尚救治。豈知賈政進內出去時,那和尚已不見了。賈政正在詫異,聽見裏頭又鬧,急忙進來。見寶玉又是先前的樣子,口關緊閉,脈息全無。用手在心窩中一摸,尚是溫熱。賈政只得急忙請醫灌藥救治。
\end{parag}


\begin{parag}
    那知那寶玉的魂魄早已出了竅了。你道死了不成?卻原來恍恍惚惚趕到前廳,見那送玉的和尚坐着,便施了禮。那知和尚站起身來,拉着寶玉就走。寶玉跟了和尚,覺得身輕如葉,飄飄搖搖,也沒出大門,不知從那裏走了出來。行了一程,到了個荒野地方,遠遠的望見一座牌樓,好象曾到過的。正要問那和尚時,只見恍恍惚惚來了一個女人。寶玉心裏想道:“這樣曠野地方,那得有如此的麗人,必是神仙下界了。”寶玉想着,走近前來細細一看,竟有些認得的,只是一時想不起來。見那女人和和尚打了一個照面就不見了。寶玉一想,竟是尤三姐的樣子,越發納悶:“怎麼他也在這裏?”又要問時,那和尚拉着寶玉過了那牌樓,只見牌上寫着“真如福地”四個大字,兩邊一幅對聯,乃是:
\end{parag}


\begin{poem}
    \begin{pl}
        假去真來真勝假,無原有是有非無。
    \end{pl}
\end{poem}


\begin{parag}
    轉過牌坊,便是一座宮門。門上橫書四個大字道“福善禍淫”。又有一副對子,大書雲:
\end{parag}


\begin{poem}
    \begin{pl}
        過去未來,莫謂智賢能打破,
    \end{pl}


    \begin{pl}
        前因後果,須知親近不相逢。
    \end{pl}

\end{poem}


\begin{parag}
    寶玉看了,心下想道:“原來如此。我倒要問問因果來去的事了。”這麼一想,只見鴛鴦站在那裏招手兒叫他。寶玉想道:“我走了半日,原不曾出園子,怎麼改了樣子了呢?”趕着要和鴛鴦說話,豈知一轉眼便不見了,心裏不免疑惑起來。走到鴛鴦站的地方兒,乃是一溜配殿,各處都有匾額。寶玉無心去看,只向鴛鴦立的所在奔去。見那一間配殿的門半掩半開,寶玉也不敢造次進去,心裏正要問那和尚一聲,回過頭來,和尚早已不見了。寶玉恍惚,見那殿宇巍峨,絕非大觀園景象。便立住腳,抬頭看那匾額上寫道:“引覺情癡”。兩邊寫的對聯道:
\end{parag}

\begin{poem}
    \begin{pl}
        喜笑悲哀都是假,貪求思慕總因癡。
    \end{pl}
\end{poem}


\begin{parag}
    寶玉看了,便點頭嘆息。想要進去找鴛鴦問他是什麼所在,細細想來甚是熟識,便仗着膽子推門進去。滿屋一瞧,並不見鴛鴦,裏頭只是黑漆漆的,心下害怕。正要退出,見有十數個大櫥,櫥門半掩。
\end{parag}


\begin{parag}
    寶玉忽然想起:“我少時做夢曾到過這個地方。如今能夠親身到此,也是大幸。”恍惚間,把找鴛鴦的念頭忘了。便壯着膽把上首的大櫥開了櫥門一瞧,見有好幾本冊子,心裏更覺喜歡,想道:“大凡人做夢,說是假的,豈知有這夢便有這事。我常說還要做這個夢再不能的,不料今兒被我找着了。但不知那冊子是那個見過的不是?”伸手在上頭取了一本,冊上寫着“金陵十二釵正冊”。寶玉拿着一想道:“我恍惚記得是那個,只恨記不得清楚。”便打開頭一頁看去,見上頭有畫,但是畫跡模糊,再瞧不出來。後面有幾行字跡也不清楚,尚可摹擬,便細細的看去,見有什麼“玉帶”,上頭有個好象“林”字,心裏想道:“不要是說林妹妹罷?”便認真看去,底下又有“金簪雪裏”四字,詫異道”怎麼又象他的名字呢。”復將前後四句合起來一念道:“也沒有什麼道理,只是暗藏着他兩個名字,並不爲奇。獨有那‘憐’字‘嘆’字不好。這是怎麼解?”想到那裏,又自啐道:“我是偷着看,若只管呆想起來,倘有人來,又看不成了。”遂往後看去,也無暇細玩那圖畫,只從頭看去。看到尾兒有幾句詞,什麼“相逢大夢歸”一句,便恍然大悟道:“是了,果然機關不爽,這必是元春姐姐了。若都是這樣明白,我要抄了去細玩起來,那些姊妹們的壽夭窮通沒有不知的了。我回去自不肯泄漏,只做一個未卜先知的人,也省了多少閒想。”又向各處一瞧,並沒有筆硯,又恐人來,只得忙着看去。只見圖上影影有一個放風箏的人兒,也無心去看。急急的將那十二首詩詞都看遍了。也有一看便知的,也有一想便得的,也有不大明白的,心下牢牢記着。一面嘆息,一面又取那《金陵又副冊》一看,看到“堪羨優伶有福,誰知公子無緣”先前不懂,見上面尚有花席的影子,便大驚痛哭起來。
\end{parag}


\begin{parag}
    待要往後再看,聽見有人說道:“你又發呆了!林妹妹請你呢。”好似鴛鴦的聲氣,回頭卻不見人。心中正自驚疑,忽鴛鴦在門外招手。寶玉一見,喜得趕出來。但見鴛鴦在前影影綽綽的走,只是趕不上。寶玉叫道:“好姐姐,等等我。”那鴛鴦並不理,只顧前走。寶玉無奈,盡力趕去,忽見別有一洞天,樓閣高聳,殿角玲瓏,且有好些宮女隱約其間。寶玉貪看景緻,竟將鴛鴦忘了。寶玉順步走入一座宮門,內有奇花異卉,都也認不明白。惟有白石花闌圍着一顆青草,葉頭上略有紅色,但不知是何名草,這樣矜貴。只見微風動處,那青草已搖擺不休,雖說是一枝小草,又無花朵,其嫵媚之態,不禁心動神怡,魂消魄喪。寶玉只管呆呆的看着,只聽見旁邊有一人說道:“你是那裏來的蠢物,在此窺探仙草!”寶玉聽了,吃了一驚,回頭看時,卻是一位仙女,便施禮道:“我找鴛鴦姐姐,誤入仙境,恕我冒昧之罪。請問神仙姐姐,這裏是何地方?怎麼我鴛鴦姐姐到此還說是林妹妹叫我?望乞明示。”那人道:“誰知你的姐姐妹妹,我是看管仙草的,不許凡人在此逗留。”寶玉欲待要出來,又捨不得,只得央告道:“神仙姐姐既是那管理仙草的,必然是花神姐姐了。但不知這草有何好處?”那仙女道:“你要知道這草,說起來話長着呢。那草本在靈河岸上,名曰絳珠草。因那時萎敗,幸得一個神瑛侍者日以甘露灌溉,得以長生。後來降凡歷劫,還報了灌溉之恩,今返歸真境。所以警幻仙子命我看管,不令蜂纏蝶戀。”寶玉聽了不解,一心疑定必是遇見了花神了,今日斷不可當面錯過,便問:“管這草的是神仙姐姐了。還有無數名花必有專管的,我也不敢煩問,只有看管芙蓉花的是那位神仙?”那仙女道:“我卻不知,除是我主人方曉。”寶玉便問道:“姐姐的主人是誰?”那仙女道:“我主人是瀟湘妃子。”寶玉聽道:“是了,你不知道這位妃子就是我的表妹林黛玉。”那仙女道:“胡說。此地乃上界神女之所,雖號爲瀟湘妃子,並不是娥皇女英之輩,何得與凡人有親。你少來混說,瞧着叫力士打你出去。”
\end{parag}


\begin{parag}
    寶玉聽了發怔,只覺自形穢濁,正要退出,又聽見有人趕來說道:“裏面叫請神瑛侍者。”那人道:“我奉命等了好些時,總不見有神瑛侍者過來,你叫我那裏請去。”那一個笑道:“才退去的不是麼?”那侍女慌忙趕出來說:“請神瑛侍者回來。”寶玉只道是問別人,又怕被人追趕,只得踉蹌而逃。正走時,只見一人手提寶劍迎面攔住說:“那裏走!”唬得寶玉驚慌無措,仗着膽抬頭一看卻不是別人,就是尤三姐。寶玉見了,略定些神,央告道:“姐姐怎麼你也來逼起我來了。”那人道:“你們兄弟沒有一個好人,敗人名節,破人婚姻。今兒你到這裏,是不饒你的了!”寶玉聽去話頭不好,正自着急,只聽後面有人叫道:“姐姐快快攔住,不要放他走了。”尤三姐道:“我奉妃子之命等侯已久,今兒見了,必定要一劍斬斷你的塵緣。”寶玉聽了益發着忙,又不懂這些話到底是什麼意思,只得回頭要跑。豈知身後說話的並非別人,卻是晴雯。寶玉一見,悲喜交集,便說:“我一個人走迷了道兒,遇見仇人,我要逃回,卻不見你們一人跟着我。如今好了,晴雯姐姐,快快的帶我回家去罷。”晴雯道:“侍者不必多疑,我非晴雯,我是奉妃子之命特來請你一會,並不難爲你。”寶玉滿腹狐疑,只得問道:“姐姐說是妃子叫我,那妃子究是何人?”晴雯道:“此時不必問,到了那裏自然知道。”寶玉沒法,只得跟着走。細看那人背後舉動恰是晴雯,那面目聲音是不錯的了,”怎麼他說不是?我此時心裏模糊。且別管他,到了那邊見了妃子,就有不是,那時再求他,到底女人的心腸是慈悲的,必是恕我冒失。”正想着,不多時到了一個所在。只見殿宇精緻,色彩輝煌,庭中一叢翠竹,戶外數本蒼松。廊檐下立着幾個侍女,都是宮妝打扮,見了寶玉進來,便悄悄的說道:“這就是神瑛侍者麼?”引着寶玉的說道:“就是。你快進去通報罷。”有一侍女笑着招手,寶玉便跟着進去。過了幾層房舍,見一正房,珠簾高掛。那侍女說:“站着候旨。”寶玉聽了,也不敢則聲,只得在外等着。那侍女進去不多時,出來說:“請侍者參見。”又有一人捲起珠簾。只見一女子,頭戴花冠,身穿繡服,端坐在內。寶玉略一抬頭,見是黛玉的形容,便不禁的說道:“妹妹在這裏!叫我好想。”那簾外的侍女悄吒道:“這侍者無禮,快快出去。”說猶未了,又見一個侍兒將珠簾放下。寶玉此時欲待進去又不敢,要走又不捨,待要問明,見那些侍女並不認得,又被驅逐,無奈出來。心想要問晴雯,回頭四顧,並不見有晴雯。心下狐疑,只得怏怏出來,又無人引着,正欲找原路而去,卻又找不出舊路了。正在爲難,見鳳姐站在一所房檐下招手。寶玉看見喜歡道:“可好了,原來回到自己家裏了。我怎麼一時迷亂如此。”急奔前來說:“姐姐在這裏麼,我被這些人捉弄到這個分兒。林妹妹又不肯見我,不知何原故。”說着,走到鳳姐站的地方,細看起來並不是鳳姐,原來卻是賈蓉的前妻秦氏。寶玉只得立住腳要問“鳳姐姐在那裏”,那秦氏也不答言,竟自往屋裏去了。寶玉恍恍惚惚的又不敢跟進去,只得呆呆的站着,嘆道:“我今兒得了什麼不是,衆人都不理我。”便痛哭起來。見有幾個黃巾力士執鞭趕來,說是”何處男人敢闖入我們這天仙福地來,快走出去!”寶玉聽得,不敢言語。正要尋路出來,遠遠望見一羣女子說笑前來。寶玉看時,又象有迎春等一干人走來,心裏喜歡,叫道:“我迷住在這裏,你們快來救我!”正嚷着,後面力士趕來。寶玉急得往前亂跑,忽見那一羣女子都變作鬼怪形像,也來追撲。
\end{parag}


\begin{parag}
    寶玉正在情急,只見那送玉來的和尚手裏拿着一面鏡子一照,說道:“我奉元妃娘娘旨意,特來救你。”登時鬼怪全無仍是一片荒郊。寶玉拉着和尚說道:“我記得是你領我到這裏,你一時又不見了。看見了好些親人,只是都不理我,忽又變作鬼怪,到底是夢是真,望老師明白指示。”那和尚道:“你到這裏曾偷看什麼東西沒有?”寶玉一想道:“他既能帶我到天仙福地,自然也是神仙了,如何瞞得他。況且正要問個明白。”便道:“我倒見了好些冊子來着。”那和尚道:“可又來,你見了冊子還不解麼!世上的情緣都是那些魔障。只要把歷過的事情細細記着,將來我與你說明。”說着,把寶玉狠命的一推,說:“回去罷!”寶玉站不住腳,一交跌倒,口裏嚷道:“阿喲!”
\end{parag}


\begin{parag}
    王夫人等正在哭泣,聽見寶玉蘇來,連忙叫喚。寶玉睜眼看時,仍躺在炕上,見王夫人寶釵等哭的眼泡紅腫。定神一想,心裏說道:“是了,我是死去過來的。”遂把神魂所歷的事呆呆的細想,幸喜多還記得,便哈哈的笑道:“是了,是了。”王夫人只道舊病復發,便好延醫調治,即命丫頭婆子快去告訴賈政,說是”寶玉回過來了,頭裏原是心迷住了,如今說出話來,不用備辦後事了。”賈政聽了,即忙進來看視,果見寶玉蘇來,便道:“沒福的癡兒你要唬死誰麼!”說着,眼淚也不知不覺流下來了。又嘆了幾口氣,仍出去叫人請醫生診脈服藥。這裏麝月正思自盡,見寶玉一過來,也放了心。只見王夫人叫人端了桂圓湯叫他喝了幾口,漸漸的定了神。王夫人等放心,也沒有說麝月,只叫人仍把那玉交給寶釵給他帶上,”想起那和尚來,這玉不知那裏找來的,也是古怪。怎麼一時要銀一時又不見了,莫非是神仙不成?”寶釵道:“說起那和尚來的蹤跡去的影響,那玉並不是找來的。頭裏丟的時候,必是那和尚取去的。”王夫人道:“玉在家裏怎麼能取的了去?”寶釵道:“既可送來,就可取去。”襲人麝月道:“那年丟了玉,林大爺測了個字,後來二奶奶過了門,我還告訴過二奶奶,說測的那字是什麼‘賞’字。二奶奶還記得麼?”寶釵想道:“是了。你們說測的是當鋪裏找去,如今才明白了,竟是個和尚的‘尚’字在上頭,可不是和尚取了去的麼。”王夫人道:“那和尚本來古怪。那年寶玉病的時候,那和尚來說是我們家有寶貝可解,說的就是這塊玉了。他既知道,自然這塊玉到底有些來歷。況且你女婿養下來就嘴裏含着的。古往今來,你們聽見過這麼第二個麼。只是不知終久這塊玉到底是怎麼着,就連咱們這一個也還不知是怎麼着。病也是這塊玉,好也是這塊玉,生也是這塊玉——”說到這裏忽然住了,不免又流下淚來。寶玉聽了,心裏卻也明白,更想死去的事愈加有因,只不言語,心裏細細的記憶。那時惜春便說道:“那年失玉,還請妙玉請過仙,說是‘青埂峯下倚古松’,還有什麼‘入我門來一笑逢’的話,想起來‘入我門’三字大有講究。佛教的法門最大,只怕二哥不能入得去。”寶玉聽了,又冷笑幾聲。寶釵聽了,不覺的把眉頭兒肐揪着發起怔來。尤氏道:“偏你一說又是佛門了。你出家的念頭還沒有歇麼?”惜春笑道:“不瞞嫂子說,我早已斷了葷了。”王夫人道:“好孩子,阿彌陀佛,這個念頭是起不得的。”惜春聽了,也不言語。寶玉想“青燈古佛前”的詩句,不禁連嘆幾聲。忽又想起一牀蓆一枝花的詩句來,拿眼睛看着襲人,不覺又流下淚來。衆人都見他忽笑忽悲,也不解是何意,只道是他的舊病。豈知寶玉觸處機來,竟能把偷看冊上詩句俱牢牢記住了,只是不說出來,心中早有一個成見在那裏了。暫且不題。
\end{parag}


\begin{parag}
    且說衆人見寶玉死去復生,神氣清爽,又加連日服藥,一天好似一天,漸漸的復原起來。便是賈政見寶玉已好,現在丁憂無事,想起賈赦不知幾時遇赦,老太太的靈柩久停寺內,終不放心,欲要扶柩回南安葬,便叫了賈璉來商議。賈璉便道:“老爺想得極是,如今趁着丁憂幹了一件大事更好。將來老爺起了服,生恐又不能遂意了。但是我父親不在家,侄兒呢又不敢僭越。老爺的主意很好,只是這件事也得好幾千銀子。衙門裏緝贓那是再緝不出來的。”賈政道:“我的主意是定了,只爲大爺不在家,叫你來商議商議怎麼個辦法。你是不能出門的。現在這裏沒有人,我爲是好幾口材都要帶回去的,一個怎麼樣的照應呢,想起把蓉哥兒帶了去。況且有他媳婦的棺材也在裏頭。還有你林妹妹的,那是老太太的遺言說跟着老太太一塊兒回去的。我想這一項銀子只好在那裏挪借幾千,也就夠了。”賈璉道:“如今的人情過於淡薄。老爺呢,又丁憂,我們老爺呢,又在外頭,一時借是借不出來的了。只好拿房地文書出去押去。”賈政道:“住的房子是官蓋的,那裏動得。”賈璉道:“住房是不能動的。外頭還有幾所可以出脫的,等老爺起復後再贖也使得。將來我父親回來了,倘能也再起用,也好贖的。只是老爺這麼大年紀,辛苦這一場,侄兒們心裏實不安。”賈政道:“老太太的事,是應該的。只要你在家謹慎些,把持定了纔好。”賈璉道:“老爺這倒只管放心,侄兒雖糊塗,斷不敢不認真辦理的。況且老爺回南少不得多帶些人去,所留下的人也有限了,這點子費用還可以過的來。就是老爺路上短少些,必經過賴尚榮的地方,可也叫他出點力兒。”賈政道:“自己的老人家的事,叫人家幫什麼。”賈璉答應了“是”,便退出來打算銀錢。
\end{parag}


\begin{parag}
    賈政便告訴了王夫人,叫他管了家,自己便擇了發引長行的日子,就要起身。寶玉此時身體復元,賈環賈蘭倒認真唸書,賈政都交付給賈璉,叫他管教,“今年是大比的年頭。環兒是有服的,不能入場,蘭兒是孫子,服滿了也可以考的,務必叫寶玉同着侄兒考去。能夠中一個舉人,也好贖一贖咱們的罪名。”賈璉等唯唯應命。賈政又吩咐了在家的人,說了好些話,才別了宗祠,便在城外唸了幾天經,就發引下船,帶了林之孝等而去。也沒有驚動親友,惟有自家男女送了一程回來。
\end{parag}


\begin{parag}
    寶玉因賈政命他赴考,王夫人便不時催逼查考起他的工課來。那寶釵襲人時常勸勉,自不必說。那知寶玉病後雖精神日長,他的念頭一發更奇僻了,竟換了一種。不但厭棄功名仕進,竟把那兒女情緣也看淡了好些。只是衆人不大理會,寶玉也並不說出來。一日,恰遇紫鵑送了林黛玉的靈柩回來,悶坐自己屋裏啼哭,想道:“寶玉無情,見他林妹妹的靈柩回去並不傷心落淚,見我這樣痛哭也不來勸慰,反瞅着我笑。這樣負心的人,從前都是花言巧語來哄着我們!前夜虧我想得開,不然幾乎又上了他的當。只是一件叫人不解,如今我看他待襲人等也是冷冷兒的。二奶奶是本來不喜歡親熱的,麝月那些人就不抱怨他麼?我想女孩子們多半是癡心的,白操了那些時的心,看將來怎樣結局!”正想着,只見五兒走來瞧他,見紫鵑滿面淚痕,便說:“姐姐又想林姑娘了?想一個人聞名不如眼見,頭裏聽着寶二爺女孩子跟前是最好的,我母親再三的把我弄進來。豈知我進來了,盡心竭力的伏侍了幾次病,如今病好了,連一句好話也沒有剩出來,如今索性連眼兒也都不瞧了。”紫鵑聽他說的好笑,便噗嗤的一笑,啐道:“呸,你這小蹄子,你心裏要寶玉怎麼個樣兒待你纔好?女孩兒家也不害臊,連名公正氣的屋裏人瞧着他還沒事人一大堆呢,有功夫理你去!”因又笑着拿個指頭往臉上抹着問道:“你到底算寶玉的什麼人哪?”那五兒聽了,自知失言,便飛紅了臉。待要解說不是要寶玉怎麼看待,說他近來不憐下的話,只聽院門外亂嚷說:“外頭和尚又來了,要那一萬銀子呢。太太着急,叫璉二爺和他講去,偏偏璉二爺又不在家。那和尚在外頭說些瘋話,太太叫請二奶奶過去商量。”不知怎樣打發那和尚,下回分解。
\end{parag}