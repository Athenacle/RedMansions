\chap{一百零八}{強歡笑蘅蕪慶生辰 死纏綿瀟湘聞鬼哭}



\begin{parag}
    卻說賈政先前曾將房產並大觀園奏請入官,內廷不收,又無人居住,只好封鎖。因園子接連尤氏惜春住宅,太覺曠闊無人,遂將包勇罰看荒園。此時賈政理家,又奉了賈母之命將人口漸次減少,諸凡省儉,尚且不能支持。幸喜鳳姐爲賈母疼惜,王夫人等雖則不大喜歡,若說治家辦事尚能出力,所以將內事仍交鳳姐辦理。但近來因被抄以後,諸事運用不來,也是每形拮据。那些房頭上下人等原是寬裕慣的,如今較之往日,十去其七,怎能周到,不免怨言不絕。鳳姐也不敢推遲,扶病承歡賈母。過了些時,賈赦賈珍各到當差地方,恃有用度,暫且自安,寫書回家,都言安逸,家中不必掛念。於是賈母放心,邢夫人尤氏也略略寬懷。
\end{parag}


\begin{parag}
    一日,史湘雲出嫁回門,來賈母這邊請安。賈母提起他女婿甚好,史湘雲也將那裏過日平安的話說了,請老太太放心。又提起黛玉去世,不免大家淚落。賈母又想起迎春苦楚,越覺悲傷起來。史湘雲勸解一回,又到各家請安問好畢,仍到賈母房中安歇,言及”薛家這樣人家被薛大哥鬧的家破人亡。今年雖是緩決人犯,明年不知可能減等?”賈母道:“你還不知道呢,昨兒蟠兒媳婦死的不明白,幾乎又鬧出一場大事來。還幸虧老佛爺有眼,叫他帶來的丫頭自己供出來了,那夏奶奶纔沒的鬧了,自家攔住相驗。你姨媽這裏纔將皮裹肉的打發出去了。你說說,真真是六親同運!薛家是這樣了,姨太太守着薛蝌過日,爲這孩子有良心他說哥哥在監裏尚未結局,不肯娶親。你邢妹妹在大太太那邊也就很苦。琴姑娘爲他公公死了尚未滿服,梅家尚未娶去。二太太的孃家舅太爺一死,鳳丫頭的哥哥也不成人,那二舅太爺也是個小氣的,又是官項不清,也是打饑荒。甄家自從抄家以後別無信息。”湘雲道:“三姐姐去了曾有書字回家麼?”賈母道:“自從嫁了去,二老爺回來說,你三姐姐在海疆甚好。只是沒有書信,我也日夜惦記,爲着我們家連連的出些不好事,所以我也顧不來。如今四丫頭也沒有給他提親。環兒呢,誰有功夫提起他來。如今我們家的日子比你從前在這裏的時侯更苦些。只可憐你寶姐姐,自過了門,沒過一天安逸日子。你二哥哥還是這樣瘋瘋顛顛,這怎麼處呢!”湘雲道:“我從小兒在這裏長大的,這裏那些人的脾氣我都知道的。這一回來了,竟都改了樣子了。我打量我隔了好些時沒來,他們生疏我。我細想起來,竟不是的,就是見了我,瞧他們的意思原要象先前一樣的熱鬧,不知道怎麼,說說就傷心起來了。我所以坐坐就到老太太這裏來了。”賈母道:“如今這樣日子在我也罷了,你們年輕輕兒的人還了得!我正要想個法兒叫他們還熱鬧一天才好,只是打不起這個精神來。”湘雲道:“我想起來了,寶姐姐不是後兒的生日嗎,我多住一天,給他拜過壽,大家熱鬧一天。不知老太太怎麼樣?”賈母道:“我真正氣糊塗了。你不提我竟忘了,後日可不是他的生日!我明日拿出錢來,給他辦個生日。他沒有定親的時侯倒做過好幾次,如今他過了門,倒沒有做。寶玉這孩子頭裏很伶俐很淘氣,如今爲着家裏的事不好,把這孩子越發弄的話都沒有了。倒是珠兒媳婦還好,他有的時侯是這麼着,沒的時侯他也是這麼着,帶着蘭兒靜靜兒的過日子,倒難爲他。”湘雲道:“別人還不離,獨有璉二嫂子連模樣兒都改了,說話也不伶俐了。明日等我來引導他們,看他們怎麼樣。但是他們嘴裏不說,心裏要抱怨我,說我有了——”湘雲說到那裏,卻把臉飛紅了。賈母會意,道:“這怕什麼。原來姊妹們都是在一處樂慣了的,說說笑笑,再別要留這些心。大凡一個人,有也罷沒也罷,總要受得富貴耐得貧賤才好。你寶姐姐生來是個大方的人,頭裏他家這樣好,他也一點兒不驕傲,後來他家壞了事,他也是舒舒坦坦的。如今在我家裏,寶玉待他好,他也是那樣安頓,一時待他不好,不見他有什麼煩惱。我看這孩子倒是個有福氣的。你林姐姐那是個最小性兒又多心的,所以到底不長命。鳳丫頭也見過些事,很不該略見些風波就改了樣子,他若這樣沒見識,也就是小器了。後兒寶丫頭的生日,我替另拿出銀子來,熱熱鬧鬧給他做個生日,也叫他歡喜這一天。”湘雲答應道:“老太太說得很是。索性把那些姐妹們都請來了,大家敘一敘。”賈母道:“自然要請的。”一時高興道:“叫鴛鴦拿出一百銀子來交給外頭,叫他明日起預備兩天的酒飯。”鴛鴦領命,叫婆子交了出去。一宿無話。次日傳話出去,打發人去接迎春,又請了薛姨媽寶琴,叫帶了香菱來。又請李嬸孃。不多半日,李紋李綺都來了。寶釵本沒有知道,聽見老太太的丫頭來請,說:“薛姨太太來了,請二奶奶過去呢。”寶釵心裏喜歡,便是隨身衣服過去,要見他母親。只見他妹子寶琴並香菱都在這裏,又見李嬸孃等人也都來了。心想:“那些人必是知道我們家的事情完了,所以來問侯的。”便去問了李嬸孃好,見了賈母,然後與他母親說了幾句話,便與李家姐妹們問好。湘雲在旁說道:“太太們請都坐下,讓我們姐妹們給姐姐拜壽。”寶釵聽了倒呆了一呆,回來一想:“可不是明日是我的生日嗎!”便說:“妹妹們過來瞧老太太是該的,若說爲我的生日,是斷斷不敢的。”正推讓着,寶玉也來請薛姨媽李嬸孃的安。聽見寶釵自己推讓,他心裏本早打算過寶釵生日,因家中鬧得七顛八倒,也不敢在賈母處提起,今見湘雲等衆人要拜壽,便喜歡道:“明日纔是生日,我正要告訴老太太來。”湘雲笑道:“扯臊,老太太還等你告訴。你打量這些人爲什麼來?是老太太請的!”寶釵聽了,心下未信。只聽賈母合他母親道:“可憐寶丫頭做了一年新媳婦,家裏接二連三的有事,總沒有給他做過生日。今日我給他做個生日,請姨太太,太太們來大家說說話兒。”薛姨媽道:“老太太這些時心裏才安,他小人兒家還沒有孝敬老太太,倒要老太太操心。”湘雲道:“老太太最疼的孫子是二哥哥,難道二嫂子就不疼了麼!況且寶姐姐也配老太太給他做生日。”寶釵低頭不語。寶玉心裏想道:“我只說史妹妹出了閣是換了一個人了,我所以不敢親近他,他也不來理我。如今聽他的話,原是和先前一樣的。爲什麼我們那個過了門更覺得靦腆了,話都說不出來了呢?”正想着,小丫頭進來說:“二姑奶奶回來了。”隨後李紈鳳姐都進來,大家廝見一番。迎春提起他父親出門,說:“本要趕來見見,只是他攔着不許來,說是咱們家正是晦氣時侯,不要沾染在身上。我扭不過,沒有來,直哭了兩三天。”鳳姐道:“今兒爲什麼肯放你回來?”迎春道:“他又說咱們家二老爺又襲了職,還可以走走,不妨事的,所以才放我來。”說着,又哭起來。賈母道:“我原爲氣得慌,今日接你們來給孫子媳婦過生日,說說笑笑解個悶兒。你們又提起這些煩事來,又招起我的煩惱來了。”迎春等都不敢作聲了。鳳姐雖勉強說了幾句有興的話,終不似先前爽利,招人發笑。賈母心裏要寶釵喜歡,故意的嘔鳳姐兒說話。鳳姐也知賈母之意,便竭力張羅,說道:“今兒老太太喜歡些了。你看這些人好幾時沒有聚在一處,今兒齊全。”說着回過頭去,看見婆婆尤氏不在這裏,又縮住了口。賈母爲着“齊全”兩字,也想邢夫人等,叫人請去。邢夫人,尤氏惜春等聽見老太太叫,不敢不來,心內也十分不願意,想着家業零敗,偏又高興給寶釵做生日,到底老太太偏心,便來了也是無精打采的。賈母問起岫煙來,邢夫人假說病着不來。賈母會意,知薛姨媽在這裏有些不便,也不提了。
\end{parag}


\begin{parag}
    一時擺下果酒。賈母說:“也不送到外頭,今日只許咱們娘兒們樂一樂。”寶玉雖然娶過親的人,因賈母疼愛,仍在裏頭打混,但不與湘雲寶琴等同席,便在賈母身旁設着一個坐兒,他代寶釵輪流敬酒。賈母道:“如今且坐下大家喝酒,到挨晚兒再到各處行禮去。若如今行起來了,大家又鬧規矩,把我的興頭打回去就沒趣了。”寶釵便依言坐下。賈母又叫人來道:“咱們今兒索性灑脫些,各留一兩個人伺侯。我叫鴛鴦帶了彩雲,鶯兒,襲人,平兒等在後間去,也喝一鍾酒。”鴛鴦等說:“我們還沒有給二奶奶磕頭,怎麼就好喝酒去呢。”賈母道:“我說了,你們只管去,用的着你們再來。”鴛鴦等去了。這裏賈母才讓薛姨媽等喝酒,見他們都不是往常的樣子,賈母着急道:“你們到底是怎麼着?大家高興些纔好。”湘雲道:“我們又喫又喝,還要怎樣!”鳳姐道:“他們小的時侯兒都高興,如今都礙着臉不敢混說,所以老太太瞧着冷淨了。”
\end{parag}


\begin{parag}
    寶玉輕輕的告訴賈母道:“話是沒有什麼說的,再說就說到不好的上頭來了。不如老太太出個主意,叫他們行個令兒罷。”賈母側着耳朵聽了,笑道:“若是行令,又得叫鴛鴦去。”寶玉聽了,不待再說,就出席到後間去找鴛鴦,說:“老太太要行令,叫姐姐去呢。”鴛鴦道:“小爺,讓我們舒舒服服的喝一杯罷,何苦來又來攪什麼。”寶玉道:“當真老太太說,得叫你去呢,與我什麼相干。”鴛鴦沒法,說道:“你們只管喝,我去了就來。”便到賈母那邊。老太太道:“你來了,不是要行令嗎。”鴛鴦道:“聽見寶二爺說老太太叫,我敢不來嗎。不知老太太要行什麼令兒?”賈母道:“那文的怪悶的慌,武的又不好,你倒是想個新鮮頑意兒纔好。”鴛鴦想了想道:“如今姨太太有了年紀,不肯費心,倒不如拿出令盤骰子來,大家擲個曲牌名兒賭輸贏酒罷。”賈母道:“這也使得。”便命人取骰盆放在桌上。鴛鴦說:“如今用四個骰子擲去,擲不出名兒來的罰一杯,擲出名兒來,每人喝酒的杯數兒擲出來再定。”衆人聽了道:“這是容易的,我們都隨着。”鴛鴦便打點兒。衆人叫鴛鴦喝了一杯,就在他身上數起,恰是薛姨媽先擲。薛姨媽便擲了一下,卻是四個麼。鴛鴦道:“這是有名的,叫做‘商山四皓’。有年紀的喝一杯。”於是賈母,李嬸孃,邢王二夫人都該喝。賈母舉酒要喝,鴛鴦道:“這是姨太太擲的,還該姨太太說個曲牌名兒,下家兒接一句《千家詩》。說不出的罰一杯。”薛姨媽道:“你又來算計我了,我那裏說得上來。”賈母道:“不說到底寂寞,還是說一句的好。下家兒就是我了,若說不出來,我陪姨太太喝一鍾就是了。”薛姨媽便道:“我說個‘臨老入花叢’。”賈母點點頭兒道:“將謂偷閒學少年。”說完,骰盆過到李紋,便擲了兩個四兩個二。鴛鴦說:“也有名了,這叫作‘劉阮入天台’。”李紋便接着說了個”二士入桃源。”下手兒便是李紈,說道:“尋得桃源好避秦。”大家又喝了一口。骰盆又過到賈母跟前,便擲了兩個二兩個三。賈母道:“這要喝酒了?”鴛鴦道:“有名兒的,這是‘江燕引雛’。衆人都該喝一杯。”鳳姐道:“雛是雛,倒飛了好些了。”衆人瞅了他一眼,鳳姐便不言語。賈母道:“我說什麼呢,‘公領孫’罷。”下手是李綺,便說道:“閒看兒童捉柳花。”衆人都說好。寶玉巴不得要說,只是令盆輪不到,正想着,恰好到了跟前,便擲了一個二兩個三一個麼,便說道:“這是什麼?”鴛鴦笑道:“這是個‘臭’,先喝一杯再擲罷。”寶玉只得喝了又擲,這一擲擲了兩個三兩個四,鴛鴦道:“有了,這叫做‘張敞畫眉’。”寶玉明白打趣他,寶釵的臉也飛紅了。鳳姐不大懂得,還說:“二兄弟快說了,再找下家兒是誰。”寶玉明知難說,自認“罰了罷,我也沒下家。”過了令盆輪到李紈,便擲了一下兒。鴛鴦道:“大奶奶擲的是‘十二金釵’。”寶玉聽了,趕到李紈身旁看時,只見紅綠對開,便說:“這一個好看得很。”忽然想起十二釵的夢來,便呆呆的退到自己座上,心裏想,”這十二釵說是金陵的,怎麼家裏這些人如今七大八小的就剩了這幾個。”復又看看湘雲寶釵,雖說都在,只是不見了黛玉,一時按捺不住,眼淚便要下來。恐人看見,便說身上躁的很,脫脫衣服去,掛了籌出席去了。這史湘雲看見寶玉這般光景,打量寶玉擲不出好的,被別人擲了去,心裏不喜歡,便去了,又嫌那個令兒沒趣,便有些煩。只見李紈道:“我不說了,席間的人也不齊,不如罰我一杯。”賈母道:“這個令兒也不熱鬧,不如蠲了罷。讓鴛鴦擲一下,看擲出個什麼來。”小丫頭便把令盆放在鴛鴦跟前。鴛鴦依命便擲了兩個二一個五,那一個骰子在盆中只管轉,鴛鴦叫道:“不要五!”那骰子單單轉出一個五來。鴛鴦道:“了不得!我輸了。”賈母道:“這是不算什麼的嗎?”鴛鴦道:“名兒倒有,只是我說不上曲牌名來。”賈母道:“你說名兒,我給你謅。”鴛鴦道:“這是浪掃浮萍。”賈母道:“這也不難,我替你說個‘秋魚入菱窠’。”鴛鴦下手的就是湘雲,便道:“白萍吟盡楚江秋。”衆人都道:“這句很確。”賈母道:“這令完了。咱們喝兩杯喫飯罷。”回頭一看,見寶玉還沒進來,便問道:“寶玉那裏去了,還不來?”鴛鴦道:“換衣服去了。”賈母道:“誰跟了去的?”那鶯兒便上來回道:“我看見二爺出去,我叫襲人姐姐跟了去了。”賈母王夫人才放心。
\end{parag}


\begin{parag}
    等了一回,王夫人叫人去找來。小丫頭子到了新房,只見五兒在那裏插蠟。小丫頭便問:“寶二爺那裏去了?”五兒道:“在老太太那邊喝酒呢。”小丫頭道:“我在老太太那裏,太太叫我來找的。豈有在那裏倒叫我來找的理。”五兒道:“這就不知道了,你到別處找去罷。”小丫頭沒法,只得回來,遇見秋紋,便道:“你見二爺那裏去了?”秋紋道:“我也找他。太太們等他喫飯,這會子那裏去了呢?你快去回老太太去,不必說不在家,只說喝了酒不大受用不喫飯了,略躺一躺再來,請老太太們喫飯罷。”小丫頭依言回去告訴珍珠,珍珠依言回了賈母。賈母道:“他本來喫不多,不喫也罷了。叫他歇歇罷。告訴他今兒不必過來,有他媳婦在這裏。”珍珠便向小丫頭道:“你聽見了?”小丫頭答應着,不便說明,只得在別處轉了一轉,說告訴了。衆人也不理會,便喫畢飯,大家散坐說話。不題。
\end{parag}


\begin{parag}
    且說寶玉一時傷心,走了出來,正無主意,只見襲人趕來,問是怎麼了。寶玉道:“不怎麼,只是心裏煩得慌。何不趁他們喝酒咱們兩個到珍大奶奶那裏逛逛去。”襲人道:“珍大奶奶在這裏,去找誰?”寶玉道:“不找誰,瞧瞧他現在這裏住的房屋怎麼樣。”襲人只得跟着,一面走,一面說。走到尤氏那邊,又一個小門兒半開半掩,寶玉也不進去。只見看園門的兩個婆子坐在門坎上說話兒。寶玉問道:“這小門開着麼?”婆子道:“天天是不開的。今兒有人出來說,今日預備老太太要用園裏的果子,故開着門等着。”寶玉便慢慢的走到那邊,果見腰門半開,寶玉便走了進去。襲人忙拉住道:“不用去,園裏不乾淨,常沒有人去,不要撞見什麼。”寶玉仗着酒氣,說:“我不怕那些。”襲人苦苦的拉住不容他去。婆子們上來說道:“如今這園子安靜的了。自從那日道士拿了妖去,我們摘花兒,打果子一個人常走的。二爺要去,咱們都跟着,有這些人怕什麼。”寶玉喜歡,襲人也不便相強,只得跟着。
\end{parag}


\begin{parag}
    寶玉進得園來,只見滿目淒涼,那些花木枯萎,更有幾處亭館,彩色久經剝落,遠遠望見一叢修竹,倒還茂盛。寶玉一想,說:“我自病時出園住在後邊,一連幾個月不准我到這裏,瞬息荒涼。你看獨有那幾杆翠竹菁蔥,這不是瀟湘館麼!”襲人道:“你幾個月沒來,連方向都忘了。咱們只管說話,不覺將怡紅院走過了。”回過頭來用手指着道:“這纔是瀟湘館呢。”寶玉順着襲人的手一瞧,道:“可不是過了嗎!咱們回去瞧瞧。”襲人道:“天晚了,老太太必是等着喫飯,該回去了。”寶玉不言,找着舊路,竟往前走。
\end{parag}


\begin{parag}
    你道寶玉雖離了大觀園將及一載,豈遂忘了路徑?只因襲人恐他見了瀟湘館,想起黛玉又要傷心,所以用言混過。豈知寶玉只望裏走,天又晚,恐招了邪氣,故寶玉問他,只說已走過了,欲寶玉不去。不料寶玉的心惟在瀟湘館內。襲人見他往前急走,只得趕上,見寶玉站着,似有所見,如有所聞,便道:“你聽什麼?”寶玉道:“瀟湘館倒有人住着麼?”襲人道:“大約沒有人罷。”寶玉道:“我明明聽見有人在內啼哭,怎麼沒有人!”襲人道:“你是疑心。素常你到這裏,常聽見林姑娘傷心,所以如今還是那樣。”寶玉不信,還要聽去。婆子們趕上說道:“二爺快回去罷。天已晚了,別處我們還敢走走,只是這裏路又隱僻,又聽得人說這裏林姑娘死後常聽見有哭聲,所以人都不敢走的。”寶玉襲人聽說,都吃了一驚。寶玉道:“可不是。”說着,便滴下淚來,說:“林妹妹,林妹妹,好好兒的是我害了你了!你別怨我,只是父母作主,並不是我負心。”愈說愈痛,便大哭起來。襲人正在沒法,只見秋紋帶着些人趕來對襲人道:“你好大膽,怎麼領了二爺到這裏來!老太太,太太他們打發人各處都找到了,剛纔腰門上有人說是你同二爺到這裏來了,唬得老太太,太太們了不得,罵着我,叫我帶人趕來,還不快回去麼!”寶玉猶自痛哭。襲人也不顧他哭,兩個人拉着就走,一面替他拭眼淚,告訴他老太太着急。寶玉沒法,只得回來。
\end{parag}


\begin{parag}
    襲人知老太太不放心,將寶玉仍送到賈母那邊。衆人都等着未散。賈母便說:“襲人,我素常知你明白,才把寶玉交給你,怎麼今兒帶他園裏去!他的病纔好,倘或撞着什麼,又鬧起來,這便怎麼處?”襲人也不敢分辯,只得低頭不語。寶釵看寶玉顏色不好,心裏着實的喫驚。倒還是寶玉恐襲人受委屈,說道:“青天白日怕什麼。我因爲好些時沒到園裏逛逛,今兒趁着酒興走走。那裏就撞着什麼了呢!”鳳姐在園裏喫過大虧的,聽到那裏寒毛倒豎,說:“寶兄弟膽子忒大了。”湘雲道:“不是膽大,倒是心實。不知是會芙蓉神去了,還是尋什麼仙去了。”寶玉聽着,也不答言。獨有王夫人急的一言不發。賈母問道:“你到園裏可曾唬着麼?這回不用說了,以後要逛,到底多帶幾個人纔好。不然大家早散了。回去好好的睡一夜,明日一早過來,我還要找補,叫你們再樂一天呢。不要爲他又鬧出什麼原故來。”衆人聽說,辭了賈母出來。薛姨媽便到王夫人那裏住下。史湘雲仍在賈母房中。迎春便往惜春那裏去了。餘者各自回去。不題。獨有寶玉回到房中,噯聲嘆氣。寶釵明知其故,也不理他,只是怕他憂悶,勾出舊病來,便進裏間叫襲人來細問他寶玉到園怎麼的光景。未知襲人怎生回說,下回分解。
\end{parag}