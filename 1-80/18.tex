\chap{一十八}{林黛玉誤剪香囊袋 賈元春歸省慶元宵}

\begin{parag}
    \begin{note}戚本回前總評:一物珍藏見致情,豪華每向鬧中爭。黛林寶薛傳佳句,豪宴仙緣留趣名。爲剪荷包綰兩意,屈從優女結三生。可憐轉眼皆虛話,雲自飄飄月自明。\end{note}
\end{parag}


\begin{parag}
    至院外,就有跟賈政的幾個小廝上來攔腰抱住,都說:“今兒虧我們,老爺才喜歡,老太太打發人出來問了幾遍,都虧我們回說喜歡;\begin{note}庚側:下人口氣畢肖。\end{note}不然,若老太太叫你進去,就不得展才了。人人都說,你才那些詩比世人的都強。今兒得了這樣的彩頭,該賞我們了。”寶玉笑道:“每人一吊錢。”衆人道:“誰沒見那一吊錢!\begin{note}庚側:錢亦有沒用處。\end{note}把這荷包賞了罷。”說著,一個上來解荷包,那一個就解扇囊,不容分說,將寶玉所佩之物盡行解去。又道:“好生送上去罷。”一個抱了起來,幾個圍繞,送至賈母二門前。\begin{note}庚側:好收煞。\end{note}那時賈母已命人看了幾次。衆奶孃丫鬟跟上來,見過賈母,知道不曾難爲著他,心中自是喜歡。
\end{parag}


\begin{parag}
    少時襲人倒了茶來,見身邊佩物一件無存,\begin{note}庚側:襲人在玉兄一身無時不照察到。\end{note}因笑道:“帶的東西又是那起沒臉的東西們解了去了。”林黛玉聽說,走來瞧瞧,果然一件無存,因向寶玉道:“我給你的那個荷包也給他們了?\begin{note}庚側:又起樓閣。\end{note}你明兒再想我的東西,可不能夠了!”說畢,賭氣回房,將前日寶玉所煩他作的那個香袋兒,做了一半,賭氣拿過來就鉸。寶玉見他生氣,便知不妥,忙趕過來,早剪破了。寶玉已見過這香囊,雖尚未完,卻十分精巧,費了許多工夫,今見無故剪了,卻也可氣。因忙把衣領解了,從裏面紅襖襟上將黛玉所給的那荷包解了下來,遞與黛玉瞧道:“你瞧瞧,這是什麼!我那一回把你的東西給人了?”林黛玉見他如此珍重,帶在裏面,\begin{note}庚雙夾:按理論之,則是“天下本無事,庸人自擾之”。若以兒女之情論之,則事必有之,事必有之,理又系今古小說中不能道得寫得,談情者不能說出講出,情癡之至文也!\end{note}可知是怕人拿去之意,因此又自悔莽撞,未見皁白就剪了香袋,\begin{note}庚雙夾:情癡之至!若無此悔便是一庸俗小性之女子矣。\end{note}因此又愧又氣,低頭一言不發。寶玉道:“你也不用剪,我知道你是懶待給我東西。我連這荷包奉還,何如?”說著,擲向他懷中便走。\begin{note}庚雙夾:這確是難怪。\end{note}黛玉見如此,越發氣起來,聲嚥氣堵,又汪汪的滾下淚來,\begin{note}庚雙夾:怒之極正是情之極。\end{note}拿起荷包來又剪。寶玉見他如此,忙回身搶住,笑道:“好妹妹,饒了他罷!”\begin{note}庚雙夾:這方是寶玉。\end{note}黛玉將剪子一摔,拭淚說道:“你不用同我好一陣歹一陣的,要惱,就撂開手。這當了什麼!”說著,賭氣上牀,面向裏倒下拭淚。禁不住寶玉上來“妹妹”長“妹妹”短賠不是。
\end{parag}


\begin{parag}
    前面賈母一片聲找寶玉。衆奶孃丫鬟們忙回說:“在林姑娘房裏呢。”賈母聽說道:“好,好,好!讓他們姊妹們一處頑頑罷。才他老子拘了他這半天,讓他開心一會子罷。只別叫他們拌嘴,不許扭了他。”衆人答應著。黛玉被寶玉纏不過,只得起來道:“你的意思不叫我安生,我就離了你。”說著往外就走。寶玉笑道:“你到那裏,我跟到那裏。”一面仍拿起荷包來帶上。黛玉伸手搶道:“你說不要了,這會子又帶上,我也替你怪臊的!”說著,嗤的一聲笑了。寶玉道:“好妹妹,明日另替我作個香袋兒罷。”黛玉道:“那也只瞧我的高興罷了。”一面說,一面二人出房,到王夫人上房中去了,\begin{note}庚雙夾:一段點過二玉公案,斷不可少。\end{note}可巧寶釵亦在那裏。
\end{parag}


\begin{parag}
    此時王夫人那邊熱鬧非常。\begin{note}庚雙夾:四字特補近日千忙萬冗多少花團錦簇文字。\end{note}原來賈薔已從姑蘇採買了十二個女孩子,並聘了教習,以及行頭等事來了。那時薛姨媽另遷於東北上一所幽靜房舍居住,將梨香院早已騰挪出來,另行修理了,就令教習在此教演女戲。又另派家中舊有曾演學過歌唱的衆女人們,如今皆已皤然老嫗了,\begin{note}庚雙夾:又補出當日寧、榮在世之事,所謂此是末世之時也。\end{note}著他們帶領管理。就令賈薔總理其日用出入銀錢等事,以及諸凡大小所需之物料帳目。\begin{note}庚雙夾:補出女戲一段,又伏一案。\end{note}又有林之孝家的來回:“採訪聘買的十個小尼姑、小道姑都有了,連新作的二十分道袍也有了。外有一個帶髮修行的,本是蘇州人氏,祖上也是讀書仕宦之家。因生了這位姑娘自小多病,買了許多替身兒皆不中用,到底這位姑娘親自入了空門,方纔好了,所以帶髮修行,今年才十八歲,法名妙玉。\begin{note}庚雙夾:妙卿出現。至此細數十二釵,以賈家四豔再加薛林二冠有六,添秦可卿有七,熙鳳有八,李紈有九,今又加妙玉僅得十人矣。後有史湘雲與熙鳳之女巧姐兒者共十二人,雪芹題曰“金陵十二釵”是本宗《紅樓夢》十二曲之意。後寶琴、岫煙、李紋、李綺皆陪客也,《紅樓夢》中所謂副十二釵是也。又有又副冊三斷詞乃晴雯、襲人、香菱三人,餘未多及,想爲金釧、玉釧、鴛鴦、苗雲\begin{subnote}按:書中不見此人,想是彩雲?\end{subnote}、平兒等人無疑矣。觀者不待言可知,故不必多費筆墨。\end{note}\begin{note}庚眉:妙玉世外人也,故筆筆帶寫,妙極妥極!畸笏。\end{note}\begin{note}庚眉:是處引十二釵總未的確,皆系漫擬也。至回末警幻情榜方知正、副、再副及三四副芳諱。壬午季春。畸笏。\end{note}如今父母俱已亡故,身邊只有兩個老嬤嬤,一個小丫頭伏侍。文墨也極通,經文也不用學了,模樣兒又極好。因聽見長安都中有觀音遺蹟並貝葉遺文,去歲隨了師父上來,\begin{note}庚雙夾:因此方使妙卿入都。\end{note}現在西門外牟尼院住著。他師父極精演先天神數,於去冬圓寂了。妙玉本欲扶靈回鄉的,他師父臨寂遺言,說他‘衣食起居不宜回鄉,在此靜居,後來自有你的結果 ’。所以他竟未回鄉。”王夫人不等回完,便說:“既這樣,我們何不接了他來。”林之孝家的回道:“請他,他說:‘侯門公府,必以貴勢壓人,我再不去的。 ’”\begin{note}庚雙夾:補出妙卿身世不凡心性高潔。\end{note}王夫人道:“他既是官宦小姐,自然驕傲些,就下個帖子請他何妨。”林之孝家的答應了出去,命書啓相公寫請帖去請妙玉。次日遣人備車轎去接等後話,暫且擱過,此時不能表白。\begin{note}庚雙夾:補尼道一段,又伏一案。\end{note}\begin{note}己眉:“不能表白”後是第十八回的起頭。\end{note}
\end{parag}


\begin{parag}
    當下又人回,工程上等著糊東西的紗綾,請鳳姐去開樓揀紗綾;又有人來回,請鳳姐開庫,收金銀器皿。連王夫人並上房丫鬟等衆,皆一時不得閒的。寶釵便說:“咱們別在這裏礙手礙腳,找探丫頭去。”說著,同寶玉黛玉往迎春等房中來閒頑,無話。
\end{parag}


\begin{parag}
    王夫人等日日忙亂,直到十月將盡,幸皆全備:各處監管都交清帳目;各處古董文玩,皆已陳設齊備;採辦鳥雀的,自仙鶴、孔雀以及鹿、兔、雞、鵝等類,悉已買全,交於園中各處像景飼養;賈薔那邊也演出二十出雜戲來;小尼姑、道姑也都學會了念幾卷經咒。賈政方略心意寬暢,\begin{note}蒙雙夾:好極!可見智者心無一時癡怠!\end{note}又請賈母等進園,色色斟酌,點綴妥當,再無一些遺漏不當之處了。於是賈政方擇日題本。\begin{note}蒙雙夾:至此方完大觀園工程公案,觀者則爲大觀園費盡精神,餘則爲若筆墨卻只因一個葬花冢。\end{note}本上之日,奉硃批准奏:次年正月十五日上元之日,恩准貴妃省親。賈府領了此恩旨,益發晝夜不閒,年也不曾好生過的。\begin{note}庚雙夾:一語帶過。是以“歲首祭宗祀,元宵開夜宴”一回留在後文細寫。\end{note}
\end{parag}


\begin{parag}
    展眼元宵在邇,自正月初八日,就有太監出來先看方向:何處更衣,何處燕坐,何處受禮,何處開宴,何處退息。又有巡察地方總理關防太監等,帶了許多小太監出來,各處關防,擋圍幕,指示賈宅人員何處退,何處跪,何處進膳,何處啓事,種種儀注不一。外面又有工部官員並五城兵備道打掃街道,攆逐閒人。賈赦等督率匠人扎花燈煙火之類,至十四日,俱已停妥。這一夜,上下通不曾睡。
\end{parag}


\begin{parag}
    至十五日五鼓,自賈母等有爵者,俱各按品服大妝。園內各處,帳舞龍蟠,簾飛綵鳳,金銀煥彩,珠寶爭輝,\begin{note}庚雙夾:是元宵之夕,不寫燈月而燈光月色滿紙矣。\end{note}鼎焚百合之香,瓶插長春之蕊,\begin{note}庚雙夾:抵一篇大賦。\end{note}靜悄無人咳嗽。\begin{note}庚雙夾:有此句方足。\end{note}賈赦等在西街門外,賈母等在榮府大門外。街頭巷口,俱系圍幕擋嚴。正等的不耐煩,忽一太監坐大馬而來,\begin{note}庚雙夾:有是理。\end{note}賈母忙接入,問其消息。太監道:“早多著呢!未初刻用過晚膳,未正二刻還到寶靈宮拜佛,\begin{note}庚雙夾:暗貼王夫人,細。\end{note}酉初刻進太明宮領宴看燈方請旨,只怕戍初才起身呢。”鳳姐聽了道:\begin{note}庚側:自然當家人先說話。\end{note}“既是這麼著,老太太、太太且請回房,等是時候再來也不遲。”於是賈母等暫且自便,園中悉賴鳳姐照理。又命執事人帶領太監們去喫酒飯。
\end{parag}


\begin{parag}
    一時傳人一擔一擔的挑進蠟燭來,各處點燈。方點完時,忽聽外邊馬跑之聲。\begin{note}庚雙夾:靜極故聞之。細極。\end{note} 一時,有十來個太監都喘吁吁跑來拍手兒。\begin{note}庚雙夾:畫出內家風範。《石頭記》最難之處別書中摸不著。\end{note}這些太監會意,\begin{note}庚側:難得他寫的出,是經過之人也。\end{note}都知道是“來了,來了”,各按方向站住。賈赦領合族子侄在西街門外,賈母領合族女眷在大門外迎接。半日靜悄悄的。忽見一對紅衣太監騎馬緩緩的走來,\begin{note}庚雙夾:形容畢肖。\end{note}至西街門下了馬,將馬趕出圍幕之外,便垂手面西站住。\begin{note}庚雙夾:形容畢肖。\end{note}半日又是一對,亦是如此。少時便來了十來對,方聞得隱隱細樂之聲。一對對龍旌鳳翣,雉羽夔頭,又有銷金提爐焚著御香;然後一把曲柄七鳳金黃傘過來,便是冠袍帶履。又有值事太監捧著香珠、繡帕、漱盂、拂塵等類。一隊隊過完,後面方是八個太監抬著一頂金頂金黃繡鳳版輿,緩緩行來。賈母等連忙路旁跪下。 \begin{note}庚側:一絲不亂。\end{note}早飛跑過幾個太監來,扶起賈母、邢夫人、王夫人來。那版輿抬進大門、入儀門往東去,到一所院落門前,有執拂太監跪請下輿更衣。於是抬輿入門,太監等散去,只有昭容、彩嬪等引領元春下輿。只見院內各色花燈熌灼,\begin{note}庚側:元春月中。 \end{note}皆系紗綾紮成,精緻非常。上面有一匾燈,寫著“體仁沐德”四字。元春入室,更衣畢復出,上輿進園。只見園中香菸繚繞,花彩繽紛,處處燈光相映,時時細樂聲喧,說不盡這太平景象,富貴風流。——此時自己回想當初在大荒山中,青埂峯下,那等淒涼寂寞;若不虧癩僧、跛道二人攜來到此,又安能得見這般世面。本欲作一篇《燈月賦》、《省親頌》,以志今日之事,但又恐入了別書的俗套。按此時之景,即作一賦一讚,也不能形容得盡其妙;即不作賦贊,其豪華富麗,觀者諸公亦可想而知矣。所以倒是省了這工夫紙墨,且說正經的爲是。\begin{note}庚雙夾:自“此時”以下皆石頭之語,真是千奇百怪之文。\end{note}\begin{note}庚眉:如此繁華盛極花團錦簇之文忽用石兄自語截住,是何筆力!令人安得不拍案叫絕。試閱歷來諸小說中有如此章法乎?\end{note}
\end{parag}


\begin{parag}
    且說賈妃在轎內看此園內外如此豪華,因默默嘆息奢華過費。忽又見執拂太監跪請登舟。賈妃乃下輿。只見清流一帶,勢若游龍,兩邊石欄上,皆系水晶玻璃各色風燈,點的如銀光雪浪;上面柳杏諸樹雖無花葉,然皆用通草綢綾紙絹依勢作成,粘於枝上的,每一株懸燈數盞;更兼池中荷荇鳧鷺之屬,亦皆系螺蚌羽毛之類作就的。諸燈上下爭輝,真系玻璃世界,珠寶乾坤。船上亦系各種精緻盆景諸燈,珠簾繡幕,桂楫蘭橈,自不必說。已而入一石港,港上一面匾燈,明現著“蓼汀花漵” 四字。按此四字,並“有鳳來儀”等處,皆繫上回賈政偶然一試寶玉之課藝才情耳,何今日認真用此匾聯?況賈政世代詩書,來往諸客屏侍坐陪者,悉皆才技之流,豈無一名手題撰,竟用小兒一戲之辭苟且搪塞?\begin{note}庚眉:駁得好!\end{note}真似暴發新榮之家,濫使銀錢,一味抹油塗朱,畢則大書“前門綠柳垂金鎖,後戶青山列錦屏”之類,則以爲大雅可觀,豈《石頭記》中通部所表之寧榮賈府所爲哉!據此論之,竟大相矛盾了。\begin{note}庚雙夾:石兄自謙,妙!可代答雲“豈敢!”\end{note}將原委說明,大家方知。\begin{note}庚眉:《石頭記》慣用特犯不犯之筆,讀之真令人驚心駭目。\end{note}
\end{parag}


\begin{parag}
    當日這賈妃未入宮時,自幼亦系賈母教養。後來添了寶玉,賈妃乃長姊,寶玉爲弱弟,賈妃之心上念母年將邁,始得此弟,是以憐愛寶玉,與諸弟待之不同。且同隨賈母,刻未離。那寶玉未入學堂之先,三四歲時,已得賈妃手引口傳,教授了幾本書、數千字在腹內了。\begin{note}庚側:批書人領過此教,故批至此竟放聲大哭,俺先姊仙逝太早,不然餘何得爲廢人耶?\end{note}其名分雖系姊弟,其情狀有如母子。自入宮後,時時帶信出來與父母說:“千萬好生扶養,不嚴不能成器,過嚴恐生不虞,且致父母之憂。”眷念切愛之心,刻未能忘。前日賈政聞塾師背後贊寶玉偏才盡有,賈政未信,適巧遇園已落成,令其題撰,聊一試其情思之清濁。其所擬之匾聯雖非妙句,在幼童爲之,亦或可取。即另使名公大筆爲之,固不費難,然想來倒不如這本家風味有趣。\begin{note}庚側:轉得好。\end{note}更使賈妃見之,知系其愛弟所爲,亦或不負其素日切望之意。\begin{note}庚側:有是論。\end{note}\begin{note}庚雙夾:一駁一解,跌宕搖曳,且寫得父母兄弟體貼戀愛之情,淋漓痛切,真是天倫至情。\end{note}因有這段原委,故此竟用了寶玉所題之聯額。那日雖未曾題完,後來亦曾補擬。\begin{note}庚雙夾:一句補前文之不暇,啓後文之苗裔。至後文凹晶館黛玉口中又一補,所謂“一擊空谷,八方皆應”。\end{note}
\end{parag}


\begin{parag}
    閒文少敘,且說賈妃看了四字,笑道:“‘花漵’二字便妥,何必‘蓼汀’?”侍坐太監聽了,忙下小舟登岸,飛傳與賈政。賈政聽了,即忙移換。\begin{note}庚雙夾:每的周到可悅。\end{note}一時,舟臨內岸,復棄舟上輿,便見琳宮綽約,桂殿巍峨。石牌坊上明顯“天仙寶鏡”四字,\begin{note}庚雙夾:不得不用俗。\end{note}賈妃忙命換“省親別墅”四字。\begin{note}庚雙夾:妙!是特留此四字與彼自命。\end{note}於是進入行宮。但見庭燎燒空,\begin{note}庚雙夾:庭燎最俗。\end{note}香屑布地,火樹琪花,金窗玉檻。說不盡簾卷蝦鬚,毯鋪魚獺,鼎飄麝腦之香,屏列雉尾之扇。真是:
\end{parag}


\begin{poem}
    \begin{pl}金門玉戶神仙府,桂殿蘭宮妃子家。\end{pl}
\end{poem}


\begin{parag}
    賈妃乃問:“此殿何無匾額?”隨侍太監跪啓曰:“此係正殿,外臣未敢擅擬。”賈妃點頭不語。禮儀太監跪請升座受禮,兩陛樂起。禮儀太監二人引賈赦、賈政等於月臺下排班,殿上昭容傳諭曰:“免。”太監引賈赦等退出。又有太監引榮國太君及女眷等自東階升月臺上排班,\begin{note}庚雙夾:一絲不亂,精緻大方。有如歐陽公九九。\end{note}昭容再諭曰:“免。”於是引退。
\end{parag}


\begin{parag}
    茶已三獻,賈妃降座,樂止。退入側殿更衣,方備省親車駕出園。至賈母正室,欲行家禮,賈母等俱跪止不迭。賈妃滿眼垂淚,方彼此上前廝見,一手攙賈母,一手攙王夫人,三個人滿心裏皆有許多話,只是俱說不出,只管嗚咽對淚。\begin{note}庚雙夾:《石頭記》得力擅長全是此等地方。庚眉:非經歷過如何寫得出!壬午春。\end{note}邢夫人、李紈、王熙鳳、迎、探、惜三姊妹等,俱在旁圍繞,垂淚無言。半日,賈妃方忍悲強笑,安慰賈母、王夫人道:“當日既送我到那不得見人的去處,好容易今日回家娘兒們一會,不說說笑笑,反倒哭起來。一會子我去了,又不知多早晚纔來!”說到這句,不覺又哽咽起來。\begin{note}庚雙夾:追魂攝魄,《石頭記》傳神摸影全在此等地方,他書中不得有此見識。\end{note}邢夫人忙上來解勸。\begin{note}庚雙夾:說完不可,不先說不可,說之不痛不可,最難說者是此時賈妃口中之語。只如此一說,千貼萬妥,一字不可更改,一字不可增減,入情入神之至!\end{note}賈母等讓賈妃歸座,又逐次一一見過,又不免哭泣一番。然後東西兩府掌家執事人丁等在廳外行禮,及兩府掌家執事媳婦領丫鬟等行禮畢。賈妃因問:“薛姨媽、寶釵、黛玉因何不見?”王夫人啓曰:“外眷無職,未敢擅入。”\begin{note}庚雙夾:所謂詩書世家,守禮如此。偏是暴發,驕妄自大。\end{note}賈妃聽了,忙命快請。\begin{note}庚雙夾:又謙之如此,真是好界好人物。\end{note}一時薛姨媽等進來,欲行國禮,亦命免過,上前各敘闊別寒溫。又有賈妃原帶進宮去的丫鬟抱琴等\begin{note}庚雙夾:前所謂賈家四釵之鬟暗以琴棋書畫排行,至此始全。\end{note}上來叩見,賈母等連忙扶起,命人別室款待。執事太監及彩嬪、昭容各侍從人等,寧國府及賈赦那宅兩處自有人款待,只留三四個小太監答應。母女姊妹深敘些離別情景,\begin{note}庚雙夾:“深”字妙!\end{note}及家務私情。
\end{parag}


\begin{parag}
    又有賈政至簾外問安,賈妃垂簾行參拜等事。又隔簾含淚謂其父曰:“田舍之家,雖齏鹽布帛,終能聚天倫之樂;今雖富貴已極,骨肉各方,然終無意趣!”賈政亦含淚啓道:“臣,草莽寒門,鳩羣鴉屬之中,豈意得徵鳳鸞之瑞。\begin{note}庚側:此語猶在耳。\end{note}今貴人上錫天恩,下昭祖德,此皆山川日月之精奇、祖宗之遠德鍾於一人,幸及政夫婦。且今上啓天地生物之大德,垂古今未有之曠恩,雖肝腦塗地,臣子豈能得報於萬一!惟朝乾夕惕,忠於厥職外,願我君萬壽千秋,乃天下蒼生之同幸也。貴妃切勿以政夫婦殘年爲念,懣憤金懷,更祈自加珍愛。惟業業兢兢,勤慎恭肅以侍上,庶不負上體貼眷愛如此之隆恩也。”賈妃亦囑“只以國事爲重,暇時保養,切勿記念”等語。賈政又啓:“園中所有亭臺軒館,皆系寶玉所題;如果有一二稍可寓目者,請別賜名爲幸。”元妃聽了寶玉能題,便含笑說:“果進益了。”賈政退出。賈妃見寶、林二人亦發比別姊妹不同,真是姣花軟玉一般。因問:“寶玉爲何不進見?”\begin{note}庚雙夾:至此方出寶玉。\end{note}賈母乃啓:“無諭,外男不敢擅入。”元妃命快引進來。小太監出去引寶玉進來,先行國禮畢,元妃命他進前,攜手攔攬於懷內,又撫其頭頸,\begin{note}庚側:作書人將批書人哭壞了。\end{note}笑道:“比先竟長了好些……”一語未終,淚如雨下。\begin{note}庚雙夾:至此一句便補足前面許多文字。\end{note}
\end{parag}


\begin{parag}
    尤氏、鳳姐等上來啓道:“筵宴齊備,請貴妃遊幸。”元妃等起身,命寶玉導引,遂同諸人步至園門前。早見燈光火樹之中,諸般羅列非常。進園來先從“有鳳來儀”、“紅香綠玉”、“杏簾在望”、“蘅芷清芬”等處,登樓步閣,涉水緣山,百般眺覽徘徊。一處處鋪陳不一,一樁樁點綴新奇。賈妃極加獎贊,又勸:“以後不可太奢,此皆過分之極。”已而至正殿,諭免禮歸座,大開筵宴。賈母等在下相陪,尤氏、李紈、鳳姐等親捧羹把盞。
\end{parag}


\begin{parag}
    元妃乃命傳筆硯伺候,親搦湘管,擇其幾處最喜者賜名。按其書雲:
\end{parag}


\begin{qute}
    \begin{parag}
        “顧恩思義”匾額\newline
        天地啓宏慈,赤子蒼頭同感戴;\newline
        古今垂曠典,九州萬國被恩榮。\newline
        此一匾一聯書於正殿。\begin{note}庚雙夾:是貴妃口氣。\end{note}
    \end{parag}

    \begin{parag}
        “大觀園”園之名\newline
        “有鳳來儀”賜名曰“瀟湘館”。 \newline
        “紅香綠玉”改作“怡紅快綠”。即名曰“怡紅院”。\newline
        “蘅芷清芳”賜名曰“蘅蕪苑”。 \newline
        “杏簾在望”賜名曰“浣葛山莊”。 \newline
    \end{parag}
\end{qute}


\begin{parag}
    正樓曰“大觀樓”,東面飛樓曰“綴錦閣”,西面斜樓曰“含芳閣”;更有“蓼風軒”、“藕香榭”、\begin{note}庚雙夾:雅而新。\end{note}“紫菱洲”、“ 葉渚”等名;又有四字的匾額十數個,諸如“梨花春雨”、“桐剪秋風”、“荻蘆夜雪”等名,此時悉難全記。\begin{note}庚雙夾:故意留下秋爽齋、凸碧山堂、凹晶溪館、暖香塢等處爲後文另換眼目之地步。\end{note}又命舊有匾聯者俱不必摘去。於是先題一絕雲:
\end{parag}


\begin{poem}
    \begin{pl}銜山抱水建來精,\end{pl}

    \begin{pl}多少工夫築始成。\end{pl}

    \begin{pl}天上人間諸景備,\end{pl}

    \begin{pl}芳園應錫大觀名。\end{pl}
    \begin{note}庚雙夾:詩卻平平,蓋彼不長於此也,故只如此。\end{note}
\end{poem}


\begin{parag}
    寫畢,向諸姐妹笑道:“我素乏捷才,且不長於吟詠,妹輩素所深知。今夜聊以塞責,不負斯景而已。異日少暇,必補撰《大觀園記》並《省親頌》等文,以記今日之事。妹輩亦各題一匾一詩,隨才之長短,亦暫吟成,不可因我微才所縛。且喜寶玉竟知題詠,是我意外之想。此中‘瀟湘館’、‘蘅蕪院’二處,我所極愛,次之‘怡紅院’、‘浣葛山莊’,此四大處,必得別有章句題詠方妙。前所題之聯雖佳,如今再各賦五言律一首,使我當面試過,方不負我自幼教授之苦心。”寶玉只得答應了,下來自去構思。
\end{parag}


\begin{parag}
    迎、探、惜三人之中,要算探春又出於姊妹之上,然自忖亦難與薛林爭衡,\begin{note}庚雙夾:只一語便寫出寶黛二人,又寫出探卿知己知彼,伏下後文多少地步。\end{note}只得勉強隨衆塞責而已。李紈也勉強湊成一律。\begin{note}庚雙夾:不表薛林可知。\end{note}賈妃先挨次看姊妹們的,寫道是:
\end{parag}


\begin{poem}
    \begin{pl}曠性怡情匾額 迎春\end{pl}

    \begin{pl}園成景備特精奇,\end{pl}

    \begin{pl}奉命羞題額曠怡。\end{pl}

    \begin{pl}誰信世間有此景,\end{pl}

    \begin{pl}游來寧不暢神思?\end{pl}
\end{poem}


\begin{poem}
    \begin{pl}萬象爭輝匾額 探春\end{pl}

    \begin{pl}名園築出勢巍巍,\end{pl}

    \begin{pl}奉命何慚學淺微。\end{pl}

    \begin{pl}精妙一時言不出,\end{pl}

    \begin{pl}果然萬物有光輝。\end{pl}
\end{poem}


\begin{poem}
    \begin{pl}文章造化匾額 惜春\end{pl}

    \begin{pl}山水橫拖千里外,\end{pl}

    \begin{pl}樓臺高起五雲中。\end{pl}

    \begin{pl}園修日月光輝裏,\end{pl}

    \begin{pl}景奪文章造化功\end{pl}
    \begin{note}庚雙夾:更牽強。三首之中還算探卿略有作意,故後文寫出許多意外妙文。\end{note}
\end{poem}


\begin{poem}
    \begin{pl}文采風流匾額 李紈\end{pl}

    \begin{pl}秀水明山抱復回,\end{pl}

    \begin{pl}風流文采勝蓬萊。\end{pl}
    \begin{note}庚雙夾:超妙!\end{note}

    \begin{pl}綠裁歌扇迷芳草,\end{pl}

    \begin{pl}紅襯湘裙舞落梅。\end{pl}
    \begin{note}庚雙夾:湊成。\end{note}

    \begin{pl}珠玉自應傳盛世,\end{pl}

    \begin{pl}神仙何幸下瑤臺。\end{pl}

    \begin{pl}名園一自邀遊賞,\end{pl}

    \begin{pl}未許凡人到此來。\end{pl}
    \begin{note}庚雙夾:此四詩列於前正爲滃託下韻也。\end{note}

\end{poem}


\begin{poem}
    \begin{pl}凝暉鍾瑞匾額\end{pl}
    \begin{note}庚雙夾:便又含蓄。\end{note} \begin{pl}薛寶釵\end{pl}

    \begin{pl}芳園築向帝城西,\end{pl}

    \begin{pl}華日祥雲籠罩奇。\end{pl}

    \begin{pl}高柳喜遷鶯出谷,\end{pl}

    \begin{pl}修篁時待鳳來儀。\end{pl}
    \begin{note}庚雙夾:恰極!\end{note}

    \begin{pl}文風已着宸遊夕,\end{pl}

    \begin{pl}孝化應隆遍省時。\end{pl}

    \begin{pl}睿藻仙才盈彩筆,\end{pl}

    \begin{pl}自慚何敢再爲辭?\end{pl}
    \begin{note}庚雙夾:好詩!此不過頌聖應酬耳,未見長,以後漸知。\end{note}

\end{poem}


\begin{poem}
    \begin{pl}世外仙園匾額\end{pl}
    \begin{note}庚雙夾:落思便不與人同。\end{note}\begin{pl}林黛玉\end{pl}

    \begin{pl}名園築何處,\end{pl}

    \begin{pl}仙境別紅塵。\end{pl}

    \begin{pl}借得山川秀,\end{pl}

    \begin{pl}添來景物新。\end{pl}
    \begin{note}庚雙夾:所謂「信手拈來無不是」,阿顰自是一種心思。\end{note}

    \begin{pl}香融金谷酒,\end{pl}

    \begin{pl}花媚玉堂人。\end{pl}

    \begin{pl}何幸邀恩寵,\end{pl}

    \begin{pl}宮車過往頻?\end{pl}
    \begin{note}庚雙夾:末二首是應制詩。餘謂寶林二作未見長,何也?該後文別有驚人之句也。在寶卿有不屑爲此,在黛卿實不足一爲。\end{note}

\end{poem}


\begin{parag}
    賈妃看畢,稱賞一番,又笑道:“終是薛林二妹之作與衆不同,非愚姊妹可同列者。”原來林黛玉安心今夜大展奇才,將衆人壓倒,\begin{note}庚雙夾:這卻何必,然尤物方如此。\end{note}不想賈妃只命一匾一詠,倒不好違諭多作,只胡亂作一首五律應景罷了。\begin{note}庚雙夾:請看前詩,卻雲是胡亂應景。\end{note}
\end{parag}


\begin{parag}
    彼時寶玉尚未作完,只剛做了“瀟湘館”與“蘅蕪苑”二首,正作“怡紅院”一首,起草內有“綠玉春猶卷”一句。寶釵轉眼瞥見,便趁衆人不理論,急忙回身悄推他道:“他\begin{note}庚雙夾:此“他”字指賈妃。\end{note}因不喜‘紅香綠玉’四字,改了‘怡紅快綠’;你這會子偏用‘綠玉’二字,豈不是有意和他爭馳了?況且蕉葉之說也頗多,再想一個改了罷。”寶玉見寶釵如此說,便拭汗說道:\begin{note}庚雙夾:想見其構思之苦方是至情。最厭近之小說中滿紙“神童”“天分”等語。\end{note}“我這會子總想不起什麼典故出處來。”寶釵笑道:“你只把‘綠玉’的‘玉’字改作‘蠟’字就是了。”寶玉道:“‘綠蠟’\begin{note}庚側:好極!\end{note}可有出處?”寶釵見問,悄悄的咂嘴點頭\begin{note}庚側:媚極!韻極!\end{note}笑道:“虧你今夜不過如此,將來金殿對策,你大約連‘趙錢孫李’都忘了呢!\begin{note}庚雙夾:有得寶卿奚落,但就謂寶卿無情,只是較阿顰施之特正耳。\end{note}唐錢珝詠芭蕉詩頭一句‘冷燭無煙綠蠟幹’,你都忘了不成?”\begin{note}庚雙夾:此等處便是用硬證實處,最是大力量,但不知是何心思,是從何落思,穿插到此玲瓏錦繡地步。庚眉:如此章法又是不曾見過的。如此穿插安得不令人拍案叫絕!壬午季春。\end{note}寶玉聽了,不覺洞開心臆,笑道:“該死,該死!現成眼前之物偏倒想不起來了,真可謂‘一字師’了。從此後我只叫你師父,再不叫姐姐了。”寶釵亦悄悄的笑道:“還不快作上去,只管姐姐妹妹的。誰是你姐姐?那上頭穿黃袍的纔是你姐姐,你又認我這姐姐來了。”一面說笑,因說笑又怕他耽延工夫,遂抽身走開了。\begin{note}庚雙夾:一段忙中閒文,已是好看之極,出人意外。\end{note}寶玉只得續成,共有了三首。
\end{parag}


\begin{parag}
    此時林黛玉未得展其抱負,自是不快。因見寶玉獨作四律,大費神思,何不代他作兩首,也省他些精神不到之處。\begin{note}庚雙夾:寫黛玉之情思,待寶玉卻又如此,是與前文特犯不犯之處。庚眉:偏又寫一樣,是何心意構思而得?畸笏。\end{note}想著,便也走至寶玉案旁,悄問:“可都有了?”寶玉道:“纔有了三首,只少‘杏簾在望’一首了。”黛玉道:“既如此,你只抄錄前三首罷。趕你寫完那三首,我也替你作出這首了。”說畢,低頭一想,早已吟成一律,\begin{note}庚雙夾:瞧他寫阿顰只如此便妙極。\end{note}便寫在紙條上,搓成個糰子,擲在他跟前。\begin{note}庚眉:紙條送迭系童生祕訣,黛卿自何處學得?一笑。丁亥春。\end{note}寶玉打開一看,只覺此首比自己所作的三首高過十倍,真是喜出望外,\begin{note}庚雙夾:這等文字亦是觀書者望外之想。\end{note}遂忙恭楷呈上。賈妃看道:
\end{parag}


\begin{poem}
    \begin{pl}有鳳來儀 臣寶玉謹題\end{pl}

    \begin{pl}秀玉初成實,\end{pl}

    \begin{pl}堪宜待鳳凰。\end{pl}\begin{note}庚雙夾:起便拿得住。\end{note}

    \begin{pl}竿竿青欲滴,\end{pl}

    \begin{pl}個個綠生涼。\end{pl}

    \begin{pl}迸砌防階水,\end{pl}

    \begin{pl}穿簾礙鼎香。\end{pl}\begin{note}庚雙夾:妙句!古云:「竹密何妨水過?」,今偏翻案。\end{note}

    \begin{pl}莫搖清碎影,\end{pl}

    \begin{pl}好夢晝初長。\end{pl}

    \begin{pl}蘅芷清芬\end{pl}

    \begin{pl}蘅蕪滿淨苑,\end{pl}

    \begin{pl}蘿薜助芬芳。\end{pl}\begin{note}庚雙夾:「助」字妙!通部書所以皆善煉字。\end{note}

    \begin{pl}軟襯三春草,\end{pl}

    \begin{pl}柔拖一縷香。\end{pl}\begin{note}庚雙夾:刻畫入妙。\end{note}

    \begin{pl}輕煙迷曲徑,\end{pl}

    \begin{pl}冷翠滴迴廊。\end{pl}\begin{note}庚雙夾:甜脆滿頰。\end{note}

    \begin{pl}誰謂池塘曲,\end{pl}

    \begin{pl}謝家幽夢長。\end{pl}

    \begin{pl}怡紅快綠\end{pl}

    \begin{pl}深庭長日靜,\end{pl}

    \begin{pl}兩兩出嬋娟。\end{pl}\begin{note}庚雙夾:雙起雙敲,讀此首始信前雲「有蕉無棠不可,有棠無蕉更不可」等批非泛泛妄批駁他人,到自己身上則無能爲之論也。\end{note}

    \begin{pl}綠蠟\end{pl}\begin{note}庚雙夾:本是「玉」字,此尊寶卿改,似較「玉」字佳。\end{note}\begin{pl}春猶卷,\end{pl}\begin{note}庚雙夾:是蕉。\end{note}

    \begin{pl}紅妝夜未眠。\end{pl}\begin{note}庚雙夾:是海棠。\end{note}

    \begin{pl}憑欄垂絳袖,\end{pl}\begin{note}庚雙夾:是海棠之情。\end{note}

    \begin{pl}倚石護青煙。\end{pl}\begin{note}庚雙夾:是芭蕉之神。何得如此工恰自然?真是好詩,卻是好書。\end{note}

    \begin{pl}對立東風裏,\end{pl}\begin{note}庚雙夾:雙收。\end{note}

    \begin{pl}主人應解憐。\end{pl}\begin{note}庚雙夾:歸到主人方不落空。王梅隱雲:「詠物體又難雙承雙落,一味雙拿則不免牽強。」此首可謂詩題兩稱,極工、極切、極流利嫵媚。\end{note}
\end{poem}


\begin{poem}
    \begin{pl}杏簾在望\end{pl}

    \begin{pl}杏簾招客飲,\end{pl}

    \begin{pl}在望有山莊。\end{pl}\begin{note}庚雙夾:分題作一氣呵成,格調熟練,自是阿顰口氣。\end{note}

    \begin{pl}菱荇鵝兒水,\end{pl}

    \begin{pl}桑榆燕子梁。\end{pl}\begin{note}庚雙夾:阿顰之心臆才情原與人別,亦不是從讀書中得來。\end{note}

    \begin{pl}一畦春韭熟,\end{pl}

    \begin{pl}十里稻花香。\end{pl}

    \begin{pl}盛世無飢餒,\end{pl}

    \begin{pl}何須耕織忙。\end{pl}\begin{note}庚雙夾:以幻入幻,順水推舟,且不失應制,所以稱阿顰。\end{note}

\end{poem}


\begin{parag}
    賈妃看畢,喜之不盡,說:“果然進益了!”又指“杏簾”一首爲前三首之冠。遂將“浣葛山莊”改爲“稻香村”。\begin{note}庚雙夾:如此服善,妙!庚眉:仍用玉兄前擬“稻香村”,卻如此幻筆幻體,文章之格式至矣盡矣!壬午春。\end{note}又命探春另以彩箋謄錄出方纔一共十數首詩,出令太監傳與外廂。賈政等看了,都稱頌不已。賈政又進《歸省頌》。元妃又命以瓊酥金膾等物,賜與寶玉並賈蘭。\begin{note}庚雙夾:忙中點出賈蘭,一人不落。\end{note}此時賈蘭極幼,未達諸事,只不過隨母依叔行禮,故無別傳。賈環從年內染病未痊,自有閒處調養,故亦無傳。\begin{note}庚雙夾:補明方不遺失。\end{note}
\end{parag}


\begin{parag}
    那時賈薔帶領十二個女戲,在樓下正等的不耐煩,只見一太監飛來說:“作完了詩,快拿戲目來!”賈薔急將錦冊呈上,並十二個花名單子。少時,太監出來,只點了四齣戲:
\end{parag}


\begin{parag}
    第一齣《豪宴》;\begin{note}庚雙夾:《一捧雪》中伏賈家之敗。\end{note}
\end{parag}


\begin{parag}
    第二齣《乞巧》;\begin{note}庚雙夾:《長生殿》中伏元妃之死。\end{note}
\end{parag}


\begin{parag}
    第三齣《仙緣》;\begin{note}庚雙夾:《邯鄲夢》中伏甄寶玉送玉。\end{note}
\end{parag}


\begin{parag}
    第四齣《離魂》。\begin{note}庚雙夾:《牡丹亭》中伏黛玉死。所點之戲劇伏四事,乃通部書之大過節、大關鍵。\end{note}
\end{parag}


\begin{parag}
    賈薔忙張羅扮演起來。一個個歌欺裂石之音,舞有天魔之態。雖是妝演的形容,卻作盡悲歡情狀。\begin{note}庚雙夾:二句畢矣。\end{note}剛演完了,一太監執一金盤糕點之屬進來,問:“誰是齡官?”賈薔便知是賜齡官之物,喜的忙接了,\begin{note}庚雙夾:何喜之有?伏下後面許多文字只用一“喜”字。\end{note}命齡官叩頭。太監又道:“貴妃有諭,說:‘齡官極好,再作兩齣戲,不拘那兩出就是了。’”賈薔忙答應了,因命齡官做《遊園》、《驚夢》二出。齡官自爲此二出原非本角之戲,執意不作,定要作《相約》《相罵》二出。\begin{note}庚雙夾:《釵釧記》中總隱後文不盡風月等文。\end{note}\begin{note}庚雙夾:按近之俗語云:“寧養千軍,不養一戲。”蓋甚言優伶之不可養之意也。大抵一班之中此一人技業稍出衆,此一人則拿腔作勢、轄衆恃能種種可惡,使主人逐之不捨責之不可,雖欲不憐而實不能不憐,雖欲不愛而實不能不愛。餘歷梨園弟子廣矣,個個皆然,亦曾與慣養梨園諸世家兄弟談議及此,衆皆知其事而皆不能言。今閱《石頭記》至“原非本角之戲,執意不作”二語,便見其恃能壓衆、喬酸嬌妒,淋漓滿紙矣。復至“情悟梨香院”一回更將和盤托出,與餘三十年前目睹身親之人現形於紙上。使言《石頭記》之爲書,情之至極、言之至恰,然非領略過乃事、迷蹈過乃情,即觀此,茫然嚼蠟,亦不知其神妙也。\end{note}賈薔扭他不過,\begin{note}庚雙夾:如何反扭他不過?其中隱許多文字。\end{note}只得依他作了。賈妃甚喜,命“不可難爲了這女孩子,好生教習”,\begin{note}庚雙夾:可知尤物了。\end{note}額外賞了兩匹宮緞、兩個荷包並金銀錁子、食物之類。\begin{note}庚雙夾:有伏下一個尤物,一段新文。\end{note}然後撤筵,將未到之處復又遊頑。忽見山環佛寺,忙另盥手進去焚香拜佛,又題一匾雲:“苦海慈航”。\begin{note}庚雙夾:寫通部人事一篇熱文,卻如此冷收。\end{note}又額外加恩與一班幽尼女道。
\end{parag}


\begin{parag}
    少時,太監跪啓:“賜物俱齊,請驗等例。”乃呈上略節。賈妃從頭看了,俱甚妥協,即命照此遵行。太監聽了,下來一一發放。原來賈母的是金、玉如意各一柄,沉香拐拄一根,伽楠念珠一串,“富貴長春”宮緞四匹,“福壽綿長”宮綢四匹,紫金“筆錠如意”錁十錠,“吉慶有魚”銀錁十錠。邢夫人、王夫人二分,只減了如意、拐、珠四樣。賈敬、賈赦、賈政等,每分御製新書二部,寶墨二匣,金、銀爵各二支,表禮按前。寶釵、黛玉諸姊妹等,每人新書一部,寶硯一方,新樣格式金銀錁二對。寶玉亦同此。\begin{note}庚雙夾:此中忽夾上寶玉,可思。\end{note}賈蘭則是金銀項圈二個,金銀錁二對。尤氏、李紈、鳳姐等,皆金銀錁四錠,表禮四端。外表禮二十四端,清錢一百串,是賜與賈母、王夫人及諸姊妹房中奶孃衆丫鬟的。賈珍、賈璉、賈環、賈蓉等,皆是表禮一分,金錁一雙。其餘綵緞百端,金銀千兩,御酒華筵,是賜東西兩府凡園中管理工程、陳設、答應及司戲、掌燈諸人的。外有清錢五百串,是 統 役、優伶、百戲、雜行人丁的。
\end{parag}


\begin{parag}
    衆人謝恩已畢,執事太監啓道:“時已醜正三刻,請駕回鑾。”賈妃聽了,不由的滿眼又滾下淚來。卻又勉強堆笑,拉住賈母、王夫人的手,緊緊的不忍釋放,\begin{note}庚雙夾:使人鼻酸。\end{note}再四叮嚀:“不須記掛,好生自養。如今天恩浩蕩,一月許進內省視一次,見面是盡有的,何必傷慘。倘明歲天恩仍許歸省,萬不可如此奢華靡費了。”\begin{note}庚雙夾:妙極之讖,試看別書中專能故用一不祥之語爲讖?今偏不然,只有如此現成一語,便是不再之讖,只看他用一“倘”字便隱諱,自然之至。\end{note}賈母等已哭的哽噎難言。賈妃雖不忍別,怎奈皇家規範,違錯不得,只得忍心上輿去了。這裏諸人好容易將賈母、王夫人安慰解勸,攙扶出園去了。\begin{note}庚眉:一回離合悲歡夾寫之文,正如山陰道上令人應接不暇,尚有許多忙中閒、閒中忙小波瀾,一絲不漏,一筆不苟。\end{note}
\end{parag}


\begin{parag}
    \begin{note}蒙回末總:此回鋪陳,非身經歷,開巨眼,伸文筆,則必有所滯罣牽強,豈能如此觸處成趣,立後文之根,足本文之情者。且借象說法,學我佛開經,代天女散花,已成此奇文妙趣,惟不得與四才子書之作者同時討論臧否,爲可恨恨耳。\end{note}
\end{parag}