\chap{一十七}{会芳园试才题对额 贾宝玉机敏动诸宾}

\begin{parag}
    \begin{note}庚辰:此回宜分二回方妥。\end{note}\begin{subnote}按:己卯本與庚辰本第十七、十八回尚未分回。\end{subnote}
\end{parag}


\begin{parag}
    \begin{note}戚本:宝玉系诸艳之贯,故大观园对额必得玉兄题跋,且暂题灯匾联上,再请赐题,此千妥万当之章法。\end{note}
\end{parag}


\begin{parag}
    诗曰:豪华虽足羡,离别却难堪。博得虚名在,谁人识苦甘?\begin{note}庚辰侧批:好诗,全是讽刺。近之谚云:“又要马儿好,又要马儿不吃草。”真骂尽无厌贪痴之辈。\end{note}
\end{parag}


\begin{parag}
    话说秦钟既死,宝玉痛哭不已,李贵等好容易劝解半日方住,归时犹是凄恻哀痛。贾母帮了几十两银子,外又备奠仪,宝玉去吊纸。七日后便送殡掩埋了,别无记述。只有宝玉日日思慕感悼,然亦无可如何了。\begin{note}庚辰双行夹批:每于此等文后使用此语作结,是板定大章法,亦是此书大旨。\end{note}
\end{parag}


\begin{parag}
    又不知历过几日何时,\begin{note}庚辰侧批:惯用此等章法。\end{note}\begin{note}庚辰双行夹批:年表如此写,亦妙!\end{note}这日贾珍等来回贾政:“园内工程俱已告竣,大老爷已瞧过了,只等老爷瞧了,或有不妥之处,再行改造,好题匾额对联的。”贾政听了,沉思一回,说道:“这匾额对联倒是一件难事。论理该请贵妃赐题才是,然贵妃若不亲睹其景,大约亦必不肯妄拟;若直待贵妃游幸过再请题,偌大景致,若干亭榭,无字标题,也觉寥落无趣,任有花柳山水,也断不能生色。”众清客在旁笑答道:“老世翁所见极是。如今我们有个愚见:各处匾额对联断不可少,亦断不可定名。如今且按其景致,或两字、三字、四字,虚合其意,拟了出来,暂且做出灯匾联悬了。待贵妃游幸时,再请定名,岂不两全?”贾政等听了,都道:“所见不差。我们今日且看看去,只管题了,若妥当便用;不妥时,然后将雨村请来,令他再拟。”\begin{note}庚辰双行夹批:点雨村,照应前文。\end{note}众人笑道:“老爷今日一拟定佳,何必又待雨村。”贾政笑道:“你们不知,我自幼于花鸟山水题咏上就平平;\begin{note}庚辰侧批:是纱帽头口气。\end{note}如今上了年纪,且案牍纷烦,于这怡情悦性文章上更生疏了,纵拟了出来,不免迂腐古板,反不能使花柳园亭生色,似不妥协,反没意思。”\begin{note}庚辰眉批:政老情字如此写。壬午季春。畸笏。\end{note}众清客笑道:“这也无妨。我们大家看了公拟,各举其长,优则存之,劣则删也,未为不可。”贾政道:“此论极是。且喜今日天气和暖,大家去逛逛。”\begin{note}庚辰双行夹批:音光,字去声,出《谐声字笺》。\end{note}说著起身,引众人前往。
\end{parag}


\begin{parag}
    贾珍先去园中知会众人。可巧近日宝玉因思念秦钟,忧戚不尽,贾母常命人带他到园中来戏耍。\begin{note}庚辰侧批:现成榫楔,一丝不费力。若特唤出宝玉来,则成何文字?\end{note}此时亦才进去,忽见贾珍走来,向他笑道:“你还不出去,老爷就来了。”宝玉听了,带著奶娘小厮们,一溜烟就出园来。\begin{note}庚辰侧批:不肖子弟来看形容。余初看之,不觉怒焉,盖谓作者形容余幼年往事,因思彼亦自写其照,何独余哉?信笔书之,供诸大众同一发笑。\end{note}方转过弯,顶头贾政引众客来了,躲之不及,只得一边站了。贾政近日因闻得塾掌称赞宝玉专能对对联,虽不喜读书,偏倒有些歪才情似的,\begin{note}蒙侧批:如此顺写,笔间写来,然却是宝玉正传。\end{note}今日偶然撞见这机会,便命他跟来。\begin{note}庚辰双行夹批:如此偶然方妙,若特特唤来题额,真不成文矣。\end{note}宝玉只得随往,尚不知何意。
\end{parag}


\begin{parag}
    贾政刚至园门前,只见贾珍带领许多执事人来,一旁侍立。贾政道:“你且把园门都关上,我们先瞧了外面再进去。”\begin{note}庚辰双行夹批:是行家看法。\end{note}贾珍听说,命人将门关了。贾政先秉正看门。只见正门五间,上面桶瓦泥鳅脊;那门栏窗隔,皆是细雕新鲜花样,并无朱粉涂饰;一色水磨群墙,\begin{note}庚辰双行夹批:门雅,墙雅,不落俗套。\end{note}下面白石台矶,凿成西番草花样。左右一望,皆雪白粉墙,下面虎皮石,随势砌去,果然不落富丽俗套,自是欢喜。遂命开门,只见迎门一带翠嶂挡在前面。\begin{note}庚辰双行夹批:掩映好极。\end{note}众清客都道:“好山,好山!”贾政道:“非此一山,一进来园中所有之景悉入目中,则有何趣。”众人道:“极是。非胸中大有邱壑,焉想及此。”说著,往前一望,见白石崚嶒,\begin{note}庚辰双行夹批:想入其中,一时难辩方向。用“前”“后”“这边”“那边”等字,正是不辨东西。\end{note}或如鬼怪,或如猛兽,纵横拱立,上面苔藓成斑,藤萝掩映,\begin{note}庚辰双行夹批:曾用两处旧有之园所改,故如此写方可,细极。\end{note}其中微露羊肠小径,\begin{note}庚辰双行夹批:好景界,山子野精于此技。此是小径,非行车蔫通道,今贾政原欲游览其景,故指此等处写之。想其通路大道,自是堂堂冠冕气象,无庸细写者也。后于省亲之时已得知矣。\end{note}贾政道:“我们就从此小径游去,回来由那一边出去,方可遍览。”
\end{parag}


\begin{parag}
    说毕,命贾珍在前引导,自己扶了宝玉,逶迤进入山口。\begin{note}庚辰侧批:宝玉此刻已料定吉多凶少。\end{note}\begin{note}庚辰双行夹批:此回乃一部之纲绪,不得不细写,尤不可不细批注。盖后文十二钗书,出入来往之境,方不能错乱,观者亦如身临足到矣。今贾政虽进的是正门。却行的是僻路,按此一大园,羊肠鸟道不止几百十条,穿东度西,临山过水,万勿以今日贾政所行之径,考其方向基址。故正殿反于末后写之,足见未由大道而往,乃逶迤转折而经也。\end{note}抬头忽见山上有镜面白石一块,\begin{note}庚辰侧批:新奇。\end{note}正是迎面留题处。\begin{note}庚辰双行夹批:留题处便精,不必限定凿 镂银一色恶俗,赖及枣梨之力。\end{note}贾政回头笑道:“诸公请看,此处题以何名方妙?”众人听说,也有说该题“叠翠”二字,也有说该题“锦嶂”的,又有说“赛香炉”的,又有说“小终南”的,种种名色,不止几十个。原来众客心中早知贾政要试宝玉的功业进益何如,只将些俗套来敷衍。宝玉亦料定此意。\begin{note}庚辰双行夹批:补明好。\end{note}贾政听了,便回头命宝玉拟来。宝玉道:“尝闻古人有云:‘编新不如述旧,刻古终胜雕今。’\begin{note}庚辰双行夹批:未闻古人说此两句,却又似有者。\end{note}况此处并非主山正景,原无可题之处,不过是探景一进步耳。\begin{note}庚辰双行夹批:此论却是。\end{note}莫如直书‘曲径通幽处’这旧句旧诗在上,倒还大方气派。”众人听了,都赞道:“是极!二世兄天分高,才情远,不似我们读腐了书的。”贾政笑道:“不可谬奖。他年小,不过以一知充十知用,取笑罢了。再俟选拟。”
\end{parag}


\begin{parag}
    说著,进入石洞来,只见佳木笼葱,奇花熌灼,一带清流,从花木深处曲折泻于石隙之下。
    \begin{note}庚辰双行夹批:这水是人力引来做的。\end{note}
    再进数步,渐向北边,
    \begin{note}庚辰双行夹批:细极。后文所以云进贾母卧房后之角门,是诸钗日相来往之境也。后文又云,诸钗所居之处,只在西北一带,最近贾母卧室之后,皆从此“北”字而来。\end{note}
    平坦宽豁,两边飞楼插空,雕甍绣槛,皆隐于山坳树杪之间。俯而视之,则清溪泻雪,石磴穿云,
    \begin{note}庚辰双行夹批:前已写山至宽处,此则由低至高处,各景皆遍。\end{note}
    白石为栏,环抱池沿,石桥三港,兽面衔吐。桥上有亭。
    \begin{note}庚辰双行夹批:前已写山写石,今则写池写楼,各景皆遍。\end{note}
    贾政与诸人上了亭子,倚栏坐了,
    \begin{note}庚辰双行夹批:此亭大抵四通八达,为诸小径之咽喉要路。\end{note}
    因问:“诸公以何题此?”诸人都道:“当日欧阳公《醉翁亭记》有云:‘有亭翼然。’就名‘翼然’。”贾政笑道:“‘翼然’虽佳,但此亭压水而成,还须偏于水题方称。依我拙裁,欧阳公之‘泻出于两峰之间’,竟用他这一个‘泻’字。”有一客道:“是极,是极。竟是‘泻玉’二字妙。”贾政拈髯寻思,因抬头见宝玉侍侧,便笑命他也拟一个来。宝玉听说,连忙回道:“老爷方才所议已是。但是如今追究了去,似乎当日欧阳公题酿泉用一‘泻’字则妥,今日此泉若亦用‘泻’字,则觉不妥。况此处虽为省亲驻跸别墅,亦当入于应制之例,用此等字眼,亦觉粗陋不雅。求再拟较此蕴藉含蓄者。”贾政笑道:“诸公听此论若如?方才众人编新,你又说不如述古;如今我们述古,你又说粗陋不妥。你且说你的来我听。”宝玉道:“有用‘泻玉’二字,则莫若‘沁芳’
    \begin{note}庚辰侧批:真新雅。\end{note}
    二字,
    \begin{note}庚辰双行夹批:果然。\end{note}
    岂不新雅?”贾政拈髯点头不语。
    \begin{note}庚辰眉批:六字是严父大露悦容也。壬午春。\end{note}
    众人都忙迎合,赞宝玉才情不凡。贾政道:“匾上二字容易,再作一副七言对联来。”宝玉听说,立于亭上,四顾一望,便机上心来,乃念道:
\end{parag}
\begin{poem}
    \begin{pl}绕堤柳借三篙翠,\end{pl}
    \begin{note}庚辰双行夹批:要紧,贴切水字。\end{note}

    \begin{pl}隔岸花分一脉香。\end{pl}
    \begin{note}庚辰双行夹批:恰极,工极!绮靡秀媚,香奁正体。\end{note}
\end{poem}


\begin{parag}
    贾政听了,点头微笑。众人先称赞不已。
\end{parag}


\begin{parag}
    于是出亭过池,一山一石,一花一木,莫不著意观览。\begin{note}庚辰双行夹批:浑写两句,已见经行处愈远,更至北一路矣。\end{note}忽抬头看见前面一带粉垣,里面数楹修舍,有千百竿翠竹遮映。众人都道:“好个所在!”\begin{note}庚辰侧批:此方可为颦儿之居。\end{note}于是大家进入,只见入门便是曲折游廊,\begin{note}庚辰双行夹批:不犯超手游廊。\end{note}阶下石子漫成甬路。上面小小两三间房舍,一明两暗,里面都是合著地步打就的床几椅案。从里间房内又得一小门,出去则是后院,有大株梨花兼著芭蕉。又有两间小小退步。后院墙下忽开一隙,得泉一派,开沟仅尺许,灌入墙内,绕阶缘屋至前院,盘旋竹下而出。
\end{parag}


\begin{parag}
    贾政笑道:“这一处还罢了。\begin{note}庚辰侧批:一处。\end{note}若能月夜坐此窗下读书,不枉虚生一世。”说毕,看著宝玉,唬的宝玉忙垂了头。\begin{note}庚辰双行夹批:点一笔。\end{note}众客忙用话开释,\begin{note}庚辰双行夹批:客不可不有。\end{note}又说道:“此处的匾该题四个字。”贾政笑问:“那四字?”一个道是“淇水遗风。”贾政道:“俗。”\begin{note}庚辰双行夹批:余亦如此。\end{note}又一个是“睢园遗迹” 。贾政道:“也俗。”贾珍笑道:“还是宝兄弟拟一个来。”\begin{note}庚辰眉批:又换一章法。壬午春。\end{note}贾政道:“他未曾作,先要议论人家的好歹,可见就是个轻薄人。”\begin{note}庚辰侧批:知子者莫如父。\end{note}众客道:“议论的极是,其奈他何。”贾政道:“休如此纵了他。”因命他道:“今日任你狂为乱道,先设议论来,然后方许你作。\begin{note}庚辰双行夹批:又一格式,不然,不独死板,且亦大失严父素体。\end{note}\begin{note}庚辰眉批:于作诗文时虽政老亦有如此令旨,可知严父亦无可奈何也。不学纨绔来看。畸笏。\end{note}方才众人说的,可有使得的?”宝玉见问,答道:“都似不妥。”\begin{note}庚辰双行夹批:明知是故意要他搬驳议论,落得肆行施展。\end{note}贾政冷笑道:“怎么不妥?”宝玉道:“这是第一处行幸之处,必须颂圣方可。若用四字的匾,又有古人现成的,何必再作。”贾政道:“难道‘淇水’‘睢园’不是古人的?”宝玉道:“这太板腐了。莫若‘有凤来仪’四字。”\begin{note}庚辰双行夹批:果然,妙在双关暗合。\end{note}众人都哄然叫妙。贾政点头道:“畜生,畜生,可谓‘管窥蠡测’矣。”因命:“再题一联来。”宝玉便念道:
\end{parag}
\begin{poem}

    \begin{pl}宝鼎茶闲烟尚绿,\end{pl}
    \begin{note}庚辰双行夹批:“尚”字妙极!不必说竹,然恰恰是竹中精舍。\end{note}
    \begin{pl}幽窗棋罢指犹凉。\end{pl}
    \begin{note}庚辰双行夹批:“犹”字妙!“尚绿”、“犹凉”四字,便如置身于森森万竿之中。\end{note}
\end{poem}


\begin{parag}
    贾政摇头说道:“也未见长。”说毕,引众人出来。
\end{parag}


\begin{parag}
    方欲走时,忽又想起一事来,\begin{note}己卯侧批:不板。\end{note}因问贾珍道:“这些院落房宇并几案桌椅都算有了,\begin{note}庚辰侧批:此一顿少不得。\end{note}还有那些帐幔帘子并陈设玩器古董,可也都是一处一处合式配就的?”\begin{note}庚辰双行夹批:大篇长文不如此顿,则成何话说?\end{note}贾珍回道:“那陈设的东西早已添了许多,自然临期合式陈设。帐幔帘子,昨日听见琏兄弟说,还不全。那原是一起工程之时就画了各处的图样,量准尺寸,就打发人办去的。想必昨日得了一半。”\begin{note}庚辰双行夹批:补出近日忙冗,千头万绪景况。\end{note}贾政听了,便知此事不是贾珍的首尾,便令人去唤贾琏。
\end{parag}


\begin{parag}
    一时贾琏赶来。\begin{note}庚辰双行夹批:写出忙冗景况。\end{note}贾政问他共有几种,现今得了几种,尚欠几种。贾琏见问,忙向靴桶取靴掖内装的一个纸折略节来,\begin{note}庚辰双行夹批:细极!从头至尾,誓不作一笔逸安苟且之笔。\end{note}看了一看,回道:“妆蟒绣堆、\begin{note}庚辰双行夹批:一字一句。\end{note}刻丝弹墨\begin{note}庚辰双行夹批:二字一句。\end{note}并各色绸绫大小幔子一百二十架,昨日得了八十架,下欠四十架。帘子二百挂,昨日俱得了。外有猩猩毡帘二百挂,金丝藤红漆竹帘二百挂,墨漆竹帘二百挂,五彩线络盘花帘二百挂,每样得了一半,也不过秋天都全了。椅搭、桌围、床裙、桌套,每分一千二百件,也有了。”
\end{parag}


\begin{parag}
    一面走,一面说,\begin{note}庚辰双行夹批:是极!\end{note}倏尔青山斜阻。\begin{note}庚辰双行夹批:“斜”字细,不必拘定方向。诸钗所居之处,若稻香村、潇湘馆、怡红院、秋爽斋、蘅芜苑等,都相隔不远,究竟只在一隅。然处置得巧妙,使人见其千邱万壑,恍然不知所穷,所谓会心处不在乎远。大抵一山一水,一木一石,全在人之穿插布置耳。\end{note}转过山怀中,隐隐露出一带黄泥筑就墙,墙头上皆稻茎掩护。\begin{note}庚辰双行夹批:配的好!\end{note}有几百株杏花,如喷火蒸霞一般。里面数楹茅屋。外面却是桑、 榆、槿、柘,各色树稚新条,随其曲折,编就两溜青篱。篱外山坡之下,有一土井,旁有桔槔辘轳之属。下面分畦列亩,佳蔬菜花,漫然无际。\begin{note}庚辰双行夹批:阅至此,又笑别部小说中,一方个花园中,皆是牡丹亭、芍药圃、雕栏画拣、琼榭朱楼,略不差别。\end{note}
\end{parag}


\begin{parag}
    贾政笑道:“倒是此处有些道理。固然系人力穿凿,此时一见,未免勾引起我归农之意。\begin{note}庚辰双行夹批:极热中偏以冷笔点之,所以为妙。\end{note}我们且进去歇息歇息。”说毕,方欲进篱门去,忽见路旁有一石碣,亦为留题之备。\begin{note}庚辰侧批:真妙真新。\end{note}\begin{note}庚辰双行夹批:更恰当。若有悬额之处,或再用镜面石,岂复成文哉?忽想到“石碣”二字,又托出许多郊野气色来,一肚皮千邱万壑,只在这石碣上。\end{note}众人笑道:“更妙,更妙!此处若悬匾待题,则田舍家风一洗尽矣。立此一碣,又觉生色许多,非范石湖田家之咏不足以尽其妙。”\begin{note}庚辰侧批:赞得是,这个蔑翁有些意思。\end{note}\begin{note}庚辰双行夹批:客不可不养。\end{note}贾政道:“诸公请题。”众人道:“方才世兄有云,‘编新不如述旧’,此处古人已道尽矣,莫若直书‘杏花村’妙极。”贾政听了,笑向贾珍道:“正亏提醒了我。此处都妙极,只是还少一个酒幌,明日竟作一个,不必华丽,就依外面村庄的式样作来,用竹竿挑在树梢。”贾珍答应了,又回道:“此处竟还不可养别的雀鸟,只是买些鹅鸭鸡类,才都相称了。”贾政与众人都道:“更妙。”贾政又向众人道:“‘杏花村’固佳,只是犯了正名,村名直待请名方可。”众客都道:“是呀。如今虚的,便是什么字样好?”大家想著,宝玉却等不得了,\begin{note}庚辰双行夹批:又换一格方不板。\end{note}也不等贾政的命,\begin{note}庚辰双行夹批:忘情有理。\end{note}便说道:“旧诗云:‘红杏梢头挂酒旗。’如今莫若‘杏帘在望’\begin{note}庚辰双行夹批:妙在一“在”字。\end{note}四字。”众人都道:“好个‘在望’!又暗合‘杏花村’意。”宝玉冷笑道:\begin{note}庚辰双行夹批:忘情最妙。\end{note}“村名若用‘杏花’二字,则俗陋不堪了。又有古人诗云:‘柴门临水稻花香。’何不就用‘稻香村’的妙?”众人听了,亦发哄声拍手道:“妙!”贾政一声喝断:“无知的业障!\begin{note}庚辰眉批:爱之至,喜之至,故作此语。作者至此,宁不笑杀?壬午春。\end{note}你能知道几个古人,能记得几首熟诗,也敢在老先生前卖弄!你方才那些胡说的,不过是试你的清浊,取笑而已,你就认真了!”说著,引众人步入茆堂,里面纸窗木榻,富贵气象一洗皆尽。贾政心中自是喜欢,却瞅宝玉道:“此处如何?”众人见问,都忙悄悄的推宝玉,教他说好。宝玉不听人言,便应声道:“不及‘有凤来仪 ’多矣。”\begin{note}庚辰双行夹批:公然自定名,妙!\end{note}贾政听了道:“无知的蠢物!你只知朱楼画栋,恶赖富丽为佳,那里知道这清幽气象。终是不读书之过!”宝玉忙答道:“老爷教训的固是,但古人常云‘天然’二字,不知何意?”
\end{parag}


\begin{parag}
    众人见宝玉牛心,都怪他呆痴不改。今见问“天然”二字,众人忙道:“别的都明白,为何连‘天然’不知?‘天然’者,天之自然而有,非人力之所成也。”宝玉道:“却又来!此处置一田庄,分明见得人力穿凿扭捏而成。远无邻村,近不负郭,背山山无脉,临水水无源,高无隐寺之塔,下无通市之桥,峭然孤出,似非大观。争似先处有自然之理,得自然之气,虽种竹引泉,亦不伤于穿凿。古人云‘天然图画’四字,正畏非其地而强为其地,非其山而强为其山,虽百般精而终不相宜……”未及说完,贾政气的喝命:“叉出去!”刚出去,又喝命:“回来!”命再题一联:“若不通,一并打嘴!”\begin{note}庚辰眉批:所谓奈何他不得也,呵呵!畸笏。\end{note}宝玉只得念道:
\end{parag}


\begin{poem}
    \begin{pl} 新涨绿添浣葛处,\end{pl}
    \begin{note}庚辰双行夹批:采《诗》颂圣最恰当。\end{note}
    \begin{pl} 好云香护采芹人。\end{pl}
    \begin{note}庚辰双行夹批:采《风》采《雅》都恰当。然冠冕中又不失香奁格调。\end{note}
\end{poem}


\begin{parag}
    贾政听了,摇头说:“更不好。”一面引人出来,转过山坡,穿花度柳,抚石依泉,过了茶蘼架,再入木香棚,越牡丹亭,度芍药圃,入蔷薇院,出芭蕉 坞,盘旋曲折。\begin{note}庚辰双行夹批:略用套语一束,与前顿破格不板。\end{note}忽闻水声潺湲,泻出石洞,上则萝薜倒垂,下则落花浮荡。\begin{note}庚辰双行夹批:仍是沁芳溪矣,究竟基址不大,全是曲折掩映之巧可知。\end{note}众人都道:“好景,好景!”贾政道:“诸公题以何名?”众人道:“再不必拟了,恰恰乎是‘武陵源’三个字。”贾政笑道:“又落实了,而且陈旧。”众人笑道:“不然就用‘秦人旧舍’四字也罢了。”宝玉道:“这越发过露了。‘秦人旧舍’说避乱之意,如何使得?莫若‘蓼汀花溆’四字。”贾政听了,更批胡说。
\end{parag}


\begin{parag}
    于是要进港洞时,又想起有船无船。贾珍道:“采莲船共四只,座船一只,如今尚未造成。”贾政笑道:“可惜不得入了。”贾珍道:“从山上盘道亦可进去。” 说毕,在前导引,大家攀藤抚树过去。只见水上落花愈多,其水愈清,溶溶荡荡,曲折萦迂。池边两行垂柳,杂著桃杏,遮天蔽日,真无一些尘土。忽见柳阴中又露出一个折带朱栏板桥来,\begin{note}庚辰双行夹批:此处才见一朱粉字样,绿柳红桥,此等点缀亦不可少。后文写芦雪广\end{note}\begin{subnote}按:广,音眼。就山筑成之房屋。韩愈《陪杜侍御游湘西两寺》诗:“剖竹走泉源,开廊架崖广。”各本或作“庵”“庭”“庐”,皆非。今从庚辰本改。\end{subnote}\begin{note}则曰蜂腰板桥,都施之得宜,非一幅死稿也。\end{note}度过桥去,诸路可通,\begin{note}庚辰双行夹批:补四字,细极!不然,后文宝钗来往,则将日日爬山越岭矣。记清此处,则知后文宝玉所行常径,非此处也。\end{note}便见一所清凉瓦舍,一色水磨砖墙,清瓦花堵。那大主山所分之脉,\begin{note}庚辰双行夹批:两见大主山,稻香村又云怀中,不写主山,而主山处处映带连络不断可知矣。\end{note}皆穿墙而过。\begin{note}庚辰双行夹批:好想。\end{note}
\end{parag}


\begin{parag}
    贾政道:“此处这所房子,无味的很。”\begin{note}庚辰双行夹批:先故顿此一笔,使后文愈觉生色,未扬先抑之法。盖钗、颦对峙有甚难写者。\end{note}因而步入门时,忽迎面突出插天的大玲珑山石来,四面群绕各式石块,竟把里面所有房屋悉皆遮住,而且一株花木也无。\begin{note}庚辰双行夹批:更奇妙!\end{note}只见许多异草:或有牵藤的,或有引蔓的,或垂山巅,或穿石隙,甚至垂檐绕柱,萦砌盘阶,\begin{note}庚辰双行夹批:更妙?\end{note}或如翠带飘摇,或如金绳盘屈,或实若丹砂,或花如金桂,味芬气馥,非花香之可比。\begin{note}庚辰双行夹批:前三处皆还在人意之中,此一处则今古书中未见之工程也。连用几“或”字,是从昌黎《南山诗》中学得。\end{note}贾政不禁笑道:“有趣!\begin{note}庚辰双行夹批:前有“无味”二字,及云“有趣”二字,更觉生色,更觉重大。\end{note}只是不大认识。”有的说:“是薜荔藤萝。” 政道:“薜荔藤萝不得如此异香。”宝玉道:“果然不是。这些之中也有藤萝薜荔。那香的是杜若蘅芜,那一种大约是茝兰,这一种大约是清葛,那一种是金簦草,这一种是玉蕗藤,红的自然是紫芸,绿的定是青芷。\begin{note}庚辰双行夹批:金簦草,见《字汇》。玉蕗,见《楚辞》“菎蕗杂于黀蒸”。茝、葛、芸、芷,皆不必注,见者太多。此书中异物太多,有人生之未闻未见者,然实系所有之物,或名差理同者亦有之。\end{note}想来《离骚》《文选》等书上所有的那些异草,也有叫作什么藿蒳姜荨的,也有叫什么纶组紫绛的,还有石帆、水松、扶留等样,\begin{note}庚辰双行夹批:左太冲《吴都赋》。\end{note}又有叫作什么绿荑的,还有什么丹椒、蘼芜、风连。\begin{note}庚辰双行夹批:以上《蜀都赋》。\end{note}如今年深岁改,人不能识,故皆象形夺名,渐渐的唤差了,也是有的。”\begin{note}庚辰双行夹批:自实注一笔,妙!\end{note}未及说完,贾政喝道:“谁问你来!”\begin{note}庚辰双行夹批:又一样止法。\end{note}唬的宝玉倒退,不敢再说。
\end{parag}


\begin{parag}
    贾政因见两边俱是超手游廊,便顺著游廊步入。只见上面五间清厦连著卷棚,四面出廊,绿窗油壁,更比前几处清雅不同。贾政叹道:“此轩中煮茶操琴,亦不必再焚香矣。\begin{note}庚辰双行夹批:前二处,一曰“月下读书”,一曰“勾引起归农之意”,此则“操琴煮茶”,断语皆妙。\end{note}此造已出意外,诸公必有佳作新题以颜其额,方不负此。”众人笑道:“再莫若‘兰风蕙露’贴切了。”贾政道:“也只好用这四字。其联若何?”一人道:“我倒想了一对,大家批削改正。”念道是:
\end{parag}


\begin{poem}
    \begin{pl}    麝兰芳霭斜阳院,\end{pl}

    \begin{pl}    杜若香飘明月洲。\end{pl}
\end{poem}


\begin{parag}
    众人道:“妙则妙矣,只是‘斜阳’二字不妥。”那人道:“古人诗云:‘蘼芜满手泣斜晖’。”众人道:“颓丧,颓丧 ”又一人道:“我也有一联,诸公评阅评阅。”因念道:
\end{parag}
\begin{poem}
    \begin{pl}三径香风飘玉蕙,\end{pl}

    \begin{pl}一庭明月照金兰。\end{pl}
    \begin{note}庚辰双行夹批:此二联皆不过为钓宝玉之饵,不必认真批评。\end{note}
\end{poem}


\begin{parag}
    贾政拈髯沉吟,意欲也题一联。忽抬头见宝玉在旁不敢则声,因喝道:“怎么你应说话时又不说了?还要等人请教你不成!”宝玉听说,便回道:“此处并没有什么‘兰麝’、‘明月’、‘洲渚’之类,若要这样著迹说来,就题二百联也不能完。”贾政道:“谁按著你的头,叫你必定说这些字样呢?”宝玉道:“如此说,匾上则莫若‘蘅芷清芬’四字。对联则是:
\end{parag}
\begin{poem}
    \begin{pl}吟成豆蔻诗犹艳,\end{pl}
    \begin{pl}睡足荼蘼梦亦香。\end{pl}\begin{note}庚辰双行夹批:实佳。\end{note}
\end{poem}


\begin{parag}
    贾政笑道:“这是套的‘书成蕉叶文犹绿’,不足为奇。”众客道:“李太白‘凤凰台’之作,全套‘黄鹤楼’,\begin{note}庚辰侧批:这一位蔑翁更有意思。\end{note}只要套得妙。如今细评起来,方才这一联,竟比‘书成蕉叶’尤觉幽娴活泼。视‘书成’之句,竟似套此而来。”贾政笑说:“岂有此理!”
\end{parag}


\begin{parag}
    说著,大家出来。行不多远,则见崇阁巍峨,层楼高起,面面琳宫合抱,迢迢复道萦纡,青松拂檐,玉兰绕砌,金辉兽面,彩焕螭头。贾政道:“这是正殿了。\begin{note}庚辰双行夹批:想来此殿在园之正中。按园不是殿方之基,西北一带通贾母卧室后,可知西北一带是多宽出一带来的,诸钗始便于行也。\end{note}只是太富丽了些。”众人都道:“要如此方是。虽然贵妃崇尚节俭,天性恶繁悦朴,\begin{note}庚辰侧批:写出贾妃身分天性。\end{note}然今日之尊,礼仪如此,不为过也。”一面说,一面走,只见正面\begin{note}庚辰双行夹批:正面,细。\end{note}现出一座玉石牌坊来,上面龙蟠螭护,玲珑凿就。贾政道:“此处书以何文?”众人道:“必是‘蓬莱仙境’方妙。”贾政摇头不语。宝玉见了这个所在,心中忽有所动,寻思起来,倒像在那里曾见过的一般,却一时想不起那年那月日的事了。\begin{note}庚辰双行夹批:仍归于葫芦一梦之太虚玄境。\end{note}贾政又命他作题,宝玉只顾细思前景,全无心于此了。众人不知其意,只当他受了这半日的折磨,精神耗散,才尽辞穷了;再要考难逼迫,著了急,或生出事来,倒不便。遂忙都劝贾政:“罢,罢,明日再题罢了。”贾政心中也怕贾母不放心,\begin{note}庚辰双行夹批:一笔不漏。\end{note}遂冷笑道:“你这畜生,也竟有不能之时了。也罢,限你一日,明日若再不能,我定不饶。这是要紧之处,更要好生作来!”\begin{note}庚辰眉批:一路顺顺逆逆,已成千邱万壑之景,若不有此一段大江截住,直成一盆景矣。作者从何落笔著想!\end{note}
\end{parag}


\begin{parag}
    说著,引人出来,再一观望,原来自进门起,所行至此,才游了十之五六。\begin{note}庚辰双行夹批:总住,妙!伏下后文所补等处。若都入此回写完,不独太繁,使后文冷落,亦且非《石头记》之笔。\end{note}又值人来回,有雨村处遣人来回话。\begin{note}庚辰双行夹批:又一紧,故不能终局也。此处渐渐写雨村亲切,正为后文地步。伏脉千里,横云断岭法。\end{note}贾政笑道:“此数处不能游了。虽如此,到底从那一边出去,纵不能细观,也可稍览。”说著,引众客行来,至一大桥前,水如晶帘一般奔入。原来这桥便是通外河之闸,引泉而入者。\begin{note}庚辰双行夹批:写出水源,要紧之极!近之画家著意于山,若不讲水。又造园圃者,唯知弄莽憨顽石壅笨冢辄谓之景,皆不知水为先著。此园大概一描,处处未尝离水,盖又未写明水之从来,今终补出,精细之至!\end{note}贾政因问:“此闸何名?”宝玉道:“此乃沁芳泉之正源,就名‘沁芳闸’。”\begin{note}庚辰双行夹批:究竟只一脉,赖人力引导之功,园不易造,景非泛写。\end{note}贾政道:“胡说!偏不用‘沁芳’二字。”\begin{note}庚辰双行夹批:此以下皆系文终之余波,收的方不突。\end{note}
\end{parag}


\begin{parag}
    于是一路行来,或清堂茅舍,或堆石为垣,或编花为牖,或山下得幽尼佛寺,或林中藏女道丹房,或长廊曲洞,或方厦圆亭,贾政皆不及进去。\begin{note}庚辰双行夹批:伏下栊翠庵、芦雪广、凸碧山庄、凹晶溪馆、暖香坞等诸处,于后文一段一段补之,方得云龙作雨之势。\end{note}因说半日腿酸,未尝歇息,忽又见前面又露出一所院落来,\begin{note}庚辰眉批:问卿此居比大荒山若何?\end{note}贾政笑道:“到此可要进去歇息歇息了。”说著,一径引人绕著碧桃花,\begin{note}庚辰双行夹批:怡红院如此写来,用无意之笔,却是极精细文字。\end{note}穿过一层竹篱花障编就的月洞门,\begin{note}庚辰双行夹批:未写其居,先写其境。\end{note}俄见粉墙环护,绿柳周垂。\begin{note}庚辰双行夹批:与“万竿修竹”遥映。\end{note}贾政与众人进去,一入门,两边都是游廊相接。院中点衬几块山石,一边种著数本芭蕉;那一边乃是一颗西府海棠,其势若伞,绿垂碧缕,葩吐丹砂。众人赞道:“好花,好花!从来也见过许多海棠,那里有这样妙的。”贾政道:“这叫作‘女儿棠’,\begin{note}庚辰双行夹批:妙名。\end{note}乃是外国之种。俗传系出‘女儿国’中,\begin{note}庚辰旁批:出自政老口中,奇特之至!\end{note}云彼国此种最盛,亦荒唐不经之说罢了。”\begin{note}庚辰侧批:政老应如此语。\end{note}众人笑道:“然虽不经,如何此名传久了?”宝玉道:“大约骚人咏士,以花之色红晕若施脂,轻弱似扶病,\begin{note}庚辰双行夹批:体贴的切,故形容的妙。\end{note}\begin{note}庚辰眉批:十字若海棠有知,必深深谢之。\end{note}大近乎闺阁风度,所以以‘女儿’命名。想因被世间俗恶听了,他便以野史纂入为证,以俗传俗,以讹传讹,都认真了。”\begin{note}庚辰双行夹批:不独此花,近之谬传者不少,不能悉道,只借此花数语驳尽。\end{note}众人都摇身赞妙。
\end{parag}


\begin{parag}
    一面说话,一面都在廊外抱厦下打就的榻上坐了。\begin{note}庚辰双行夹批:至阶又至檐,不肯轻易写过。\end{note}贾政因问:“想几个什么新鲜字来题此?”一客道:“‘蕉鹤’二字最妙。”又一个道:“‘崇光泛彩’方妙。”贾政与众人都道:“好个‘崇光泛彩’!”宝玉也道:“妙极。”又叹:“只是可惜了。”众人问:“如何可惜?”宝玉道:“此处蕉棠两植,其意暗蓄‘红’‘绿’二字在内。若只说蕉,则棠无著落;若只说棠,蕉亦无著落。固有蕉无棠不可,有棠无蕉更不可。”贾政道:“依你如何?”宝玉道:“依我,题‘红香绿玉’四字,方两全其妙。”贾政摇头道:“不好,不好!”
\end{parag}


\begin{parag}
    说著,引人进入房内。只见这几间房内收拾的与别处不同,竟分不出间隔来的,\begin{note}庚辰侧批:特为青埂峰下凄凉与别处不同耳。庚辰双行夹批:新奇希见之式。\end{note}原来四面皆是雕空玲珑木板,或“流云百蝠”,或“岁寒三友”,或山水人物,或翎毛花卉,或集锦,或博古,\begin{note}庚辰双行夹批:花样周全之极!然必用下文者,正是作者无聊,撰出新异笔墨,使观者眼目一新。所谓集小说之大成,游戏笔墨,雕虫之技,无所不备,可谓善戏者矣。又供诸人同学一戏,洵为妙极。\end{note}或,\begin{note}前庚辰双行夹批:金玉篆文是可考正箓,今则从俗花样,真是醒睡魔。其中诗词雅谜以及各种风俗字文,一概不必究,只据此等处便是一绝。\end{note}各种花样,皆是名手雕镂,五彩销金嵌宝的。\begin{note}庚辰双行夹批:至此方见一朱彩之处,亦必如此式方可。可笑近之园庭,行动便以粉油从事。\end{note}一隔一隔,或有贮书处,或有设鼎处,或安置笔砚处,或供花设瓶、安放盆景处,其隔各式各样,或天圆地方,或葵花蕉叶,或连环半壁。真是花团锦簇,剔透玲珑。倏尔五色纱糊就,竟系小窗;倏尔彩绫轻覆,竟系幽户。\begin{note}庚辰双行夹批:精工之极!\end{note}且满墙满壁,皆系随依古董玩器之形抠成的槽子。诸如琴、剑、悬瓶、\begin{note}庚辰双行夹批:悬于壁上之瓶也。\end{note}桌屏之类,虽悬于壁,却都是与壁相平的。\begin{note}庚辰双行夹批:皆系人意想不到,日所未见之文,若云拟编虚想出来,焉能如此?一段极清极细,后文鸳鸯瓶、紫玛瑙碟、西洋酒令、自行船等文,不必细表。\end{note}众人都道:“好精致想头!难为怎么想来?”\begin{note}庚辰双行夹批:谁不如此赞?\end{note}
\end{parag}


\begin{parag}
    原来贾政等走了进来,未进两层,便都迷了旧路,左瞧也有门可通,右瞧又有窗暂隔,及到了跟前,又被一架书挡住。回头再走,又有窗纱明透,门径可行;及至门前,忽见迎面也进来了一群人,都与自己形相一样,——却是一架玻璃大镜相照。及转过镜去,\begin{note}庚辰侧批:石兄迷否?\end{note}益发见门子多了。\begin{note}庚辰侧批:所谓投投是道是也。\end{note}贾珍笑道:“老爷随我来。从这门出去,便是后院,从后院出去,倒比先近了。”说著,又转了两层纱厨锦隔,果得一门出去,\begin{note}庚辰侧批:此方便门也。\end{note}院中满架蔷薇、宝相。转过花障,则见清溪前阻。众人咤异:“这股水又是从何而来?”贾珍遥指道:“原从那闸起流至那洞口,从东北山坳里引到那村庄里,又开一道岔口,引到西南上,共总流到这里,仍旧合在一处,\begin{note}庚辰侧批:于怡红总一园之看,是书中大立意。\end{note}从那墙下出去。”众人听了,都道:“神妙之极!”说著,忽见大山阻路。众人都道:“迷了路了。”贾珍笑道:“随我来。”仍在前导引,众人随他,直由山脚边忽一转,便是平坦宽阔大路,\begin{note}庚辰侧批:众善归缘,自然有平坦大道。\end{note}豁然大门前见。\begin{note}庚辰双行夹批:可见前进来是小路径,此云忽一转,便是平坦宽阔之正甬路也,细极!\end{note}众人都道:“有趣,有趣,真搜神夺巧之至也!”于是大家出来。\begin{note}庚辰眉批:以上可当《大观园记》。\end{note}那宝玉一心只记挂著里边,又不见贾政吩咐,少不得跟到书房。贾政忽想起他来,方喝道:“你还不去?难道还逛不足!\begin{note}庚辰侧批:冤哉冤哉!\end{note}也不想逛了这半日,老太太必悬挂著。快进去,疼你也白疼了。”\begin{note}庚辰双行夹批:如此去法,大家严父风范,无家法者不知。\end{note}宝玉听说,方退了出来。
\end{parag}


\begin{parag}
    \begin{note}蒙回末总评:好将富贵回头看,总有文章如意难。零落机缘君记去,黄金万斗大观摊。\end{note}
\end{parag}

