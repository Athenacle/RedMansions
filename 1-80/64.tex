\chap{六十四}{幽淑女悲題五美吟 浪蕩子情遺九龍珮}


\begin{parag}
    \begin{note}蒙回前總:此一回緊接賈敬靈柩進城,原當鋪敘寧府喪儀之盛,然上回秦氏病故鳳姐理喪已描寫殆盡,若仍極力寫去,不過加倍熱鬧而已,故書中於迎靈送殯極忙亂處卻只閒閒數筆帶過。忽插入釵玉評詩、璉尤贈珮一段閒雅文字來,正所謂“急脈緩受”也。\end{note}
\end{parag}


\begin{parag}
    \begin{note}題曰:深閨有奇女,絕世空珠翠。情癡苦累多,未習顏憔悴。哀哉千秋魂,薄命無二致。 當松 間人,好醜非其類。\end{note}
\end{parag}



\begin{parag}
    話說賈蓉見家中諸事已妥,連忙趕至寺中,回明賈珍。於是連夜分派各項執事人役,並預備一切應用幡槓等物。擇於初四日卯時請靈柩進城,一面使人知會諸位親友。是日,喪儀昆耀,賓客如雲,自鐵檻寺至寧府,夾路看的何止數萬人。內中有嗟嘆的,也有羨慕的,又有一等半瓶醋的讀書人,說是“喪禮與其奢易莫若儉戚” 的,一路紛紛議論不一。至未申時方到,將靈柩停放在正堂之內。供奠舉哀已畢,親友漸次散回,只剩族中人分理迎賓送客等事。近親只有邢大舅相伴未去。賈珍賈蓉此時爲禮法所拘,不免在靈旁藉草枕塊,恨苦居喪。人散後,仍乘空尋他小姨子們廝混。寶玉亦每日在寧府穿孝,至晚人散,方回園裏。鳳姐身體未愈,雖不能時常在此,或遇開壇誦經親友上祭之日,亦扎掙過來,相幫尤氏料理。
\end{parag}


\begin{parag}
    一日,供畢早飯,因此時天氣尚長,賈珍等連日勞倦,不免在靈旁假寐。寶玉見無客至,遂欲回家看視黛玉,因先回至怡紅院中。進入門來,只見院中寂靜無人,有幾個老婆子與小丫頭們在迴廊下取便乘涼,也有睡臥的,也有坐著打盹的。寶玉也不去驚動。只有四兒看見,連忙上前來打簾子。將掀起時,只見芳官自內帶笑跑出,幾乎與寶玉撞個滿懷。一見寶玉,方含笑站住,說道:“你怎麼來了?你快與我攔住晴雯,他要打我呢。”一語未了,只聽得屋內嘻溜譁喇的亂響,不知是何物撒了一地。隨後晴雯趕來罵道:“我看你這小蹄子往那裏去,輸了不叫打。寶玉不在家,我看你有誰來救你。”寶玉連忙帶笑攔住,說道:“你妹子小,不知怎麼得罪了你,看我的分上,饒他罷。”晴雯也不想寶玉此時回來,乍一見,不覺好笑,遂笑說道:“芳官竟是個狐狸精變的,竟是會拘神遣將的符咒也沒有這樣快。”又笑道:“就是你真請了神來,我也不怕。”遂奪手仍要捉拿芳官。芳官早已藏在寶玉身後。寶玉遂一手拉了晴雯,一手攜了芳官,進入屋內。看時,只見西邊炕上麝月、秋紋、碧痕、紫綃等正在那裏抓子兒贏瓜子兒呢。卻是芳官輸與晴雯,芳官不肯叫打,跑了出去。晴雯因趕芳官,將懷內的子兒撒了一地。寶玉歡喜道:“如此長天,我不在家,正恐你們寂寞,吃了飯睡覺睡出病來,大家尋件事頑笑消遣甚好。”因不見襲人,又問道:“你襲人姐姐呢?”晴雯道:“襲人麼,越發道學了,獨自個在屋裏面壁呢。這好一會我沒進去,不知作什麼呢,一些聲氣也聽不見。你快瞧瞧去罷,或者此時參悟了,也未可定。”
\end{parag}


\begin{parag}
    寶玉聽說,一面笑,一面走至裏間。只見襲人坐在近窗牀上,手中拿著一根灰色絛子,正在那裏打結子呢。見寶玉進來,連忙站起來,笑道:“晴雯這東西編派我什麼呢。我因要趕著打完了這結子,沒工夫和他們瞎鬧,因哄他們道:‘你們頑去罷,趁著二爺不在家,我要在這裏靜坐一坐,養一養神。’他就編派了我這些混話,什麼‘面壁了’‘參禪了’的,等一會我不撕他那嘴。”寶玉笑著挨近襲人坐下,瞧他打結子,問道:“這麼長天,你也該歇息歇息,或和他們頑笑,要不,瞧瞧林妹妹去也好。怪熱的,打這個那裏使?”襲人道:“我見你帶的扇套還是那年東府裏蓉大奶奶的事情上作的。那個青東西除族中或親友家夏天有喪事方帶得著,一年遇著帶一兩遭,平常又不犯做。如今那府裏有事,這是要過去天天帶的,所以我趕著另作一個。等打完了結子,給你換下那舊的來。你雖然不講究這個,若叫老太太回來看見,又該說我們躲懶,連你的穿帶之物都不經心了。”寶玉笑道:“這真難爲你想的到。只是也不可過於趕,熱著了倒是大事。”說著,芳官早託了一杯涼水內新湃的茶來。因寶玉素昔秉賦柔脆,雖暑月不敢用冰,只以新汲井水將茶連壺浸在盆內,不時更換,取其涼意而已。寶玉就芳官手內吃了半盞,遂向襲人道: “我來時已吩咐了茗煙,若珍大哥那邊有要緊的客來時,叫他即刻送信;若無要緊的事,我就不過去了。” 說畢,遂出了房門,又回頭向碧痕等道:“如有事往林姑娘處來找我。”於是一徑往瀟湘館來看黛玉。
\end{parag}


\begin{parag}
    將過了沁芳橋,只見雪雁領著兩個老婆子,手中都拿著菱藕瓜果之類。寶玉忙問雪雁道:“你們姑娘從來不喫這些涼東西的,拿這些瓜果何用?不是要請那位姑娘奶奶麼?”雪雁笑道:“我告訴你,可不許你對姑娘說去。”寶玉點頭應允。雪雁便命兩個婆子:“先將瓜果送去交與紫鵑姐姐。他要問我,你就說我做什麼呢,就來。”那婆子答應著去了。雪雁方說道:“我們姑娘這兩日方覺身上好些了。今日飯後,三姑娘來會著要瞧二奶奶去,姑娘也沒去。又不知想起了甚麼來,自己傷感了一回,提筆寫了好些,不知是詩是詞。叫我傳瓜果去時,又聽叫紫鵑將屋內擺著的小琴桌上的陳設搬下來,將桌子挪在外間當地,又叫將那龍文鼒\begin{note}蒙雙夾:子之切,小鼎也。\end{note}放在桌上,等瓜果來時聽用。若說是請人呢,不犯先忙著把個爐擺出來。若說點香呢,我們姑娘素日屋內除擺新鮮花果木瓜之類,又不大喜薰衣服;就是點香,亦當點在常坐臥之處。難道是老婆子們把屋子燻臭了要拿香薰燻不成。究竟連我也不知何故。”說畢,便連忙的去了。
\end{parag}


\begin{parag}
    寶玉這裏不由的低頭心內細想道:“據雪雁說來,必有原故。若是同那一位姊妹們閒坐,亦不必如此先設饌具。或者是姑爹姑媽的忌辰,但我記得每年到此日期老太太都吩咐另外整理餚饌送去與林妹妹私祭,此時已過。大約必是七月因爲瓜果之節,家家都上秋祭的墳,林妹妹有感於心,所以在私室自己奠祭,取《禮記》 ‘春秋薦其時食’之意,也未可定。但我此刻走去,見他傷感,必極力勸解,,又怕他煩惱鬱結於心;若不去,又恐他過於傷感,無人勸止。兩件皆足致疾。莫若先到鳳姐姐處一看,在彼稍坐即回。如若見林妹妹傷感,再設法開解,既不至使其過悲,哀痛稍申,亦不至抑鬱致病。”想畢,遂出了園門,一徑到鳳姐處來。
\end{parag}


\begin{parag}
    正有許多執事婆子們回事畢,紛紛散出。鳳姐兒正倚著門和平兒說話呢。一見了寶玉,笑道:“你回來了麼。我才吩咐了林之孝家的。叫他使人告訴跟你的小廝,若沒什麼事趁便請你回來歇息歇息。再者那裏人多,你那裏禁得住那些氣味。不想恰好你倒來了。”寶玉笑道:“多謝姐姐記掛。我也因今日沒事,又見姐姐這兩日沒往那府裏去,不知身上可大愈否,所以回來看視看視。”鳳姐道:“左右也不過是這樣,三日好兩日不好的。老太太、太太不在家,這些大娘們,噯,那一個是安分的,每日不是打架,就拌嘴,連賭博偷盜的事情,都鬧出來了兩三件了。雖說有三姑娘幫著辦理,他又是個沒出閣的姑娘。也有叫他知道得的,也有往他說不得的事,也只好強扎掙著罷了。總不得心靜一會兒。別說想病好,求其不添,也就罷了。”寶玉道:“雖如此說,姐姐還要保重身體,少操些心纔是。”說畢,又說了些閒話,別了鳳姐,一直往園中走來。
\end{parag}


\begin{parag}
    進了瀟湘館院門看時,只見爐嫋殘煙,奠餘玉醴。紫鵑正看著人往裏搬桌子,收陳設呢。寶玉便知已經祭完了,走入屋內,只見黛玉面向裏歪著,病體懨懨,大有不勝之態。紫鵑連忙說道:“寶二爺來了。”黛玉方慢慢的起來,含笑讓坐。寶玉道:“妹妹這兩天可大好些了?氣色倒覺靜些,只是爲何又傷心了?”黛玉道:“可是你沒的說了,好好的我多早晚又傷心了?”寶玉笑道:“妹妹臉上現有淚痕,如何還哄我呢。只是我想妹妹素日本來多病,凡事當各自寬解,不可過作無益之悲。若作踐壞了身子,使我……”說到這裏,覺得以下的話有些難說,連忙嚥住。只因他雖說和黛玉一處長大,情投意合,又願同生死,卻只是心中領會,從來未曾當面說出。況兼黛玉心多,每每說話造次,得罪了他。今日原爲的是來勸解,不想把話又說造次了,接不下去,心中一急,又怕黛玉惱他。又想一想自己的心實在的是爲好,因而轉急爲悲,早已滾下淚來。黛玉起先原惱寶玉說話不論輕重,如今見此光景,心有所感,本來素昔愛哭,此時亦不免無言對泣。
\end{parag}


\begin{parag}
    卻說紫鵑端了茶來,打諒二人又爲何事角口,因說道:“姑娘才身上好些,寶二爺又來慪氣了,到底是怎麼樣?”寶玉一面拭淚笑道:“誰敢慪妹妹了。”一面搭訕著起來閒步。只見硯臺底下微露一紙角,不禁伸手拿起。黛玉忙要起身來奪,已被寶玉揣在懷內,笑央道:“好妹妹,賞我看看罷。”黛玉道:“不管什麼,來了就混翻。”一語未了,只見寶釵走來,笑道:“寶兄弟要看什麼?”寶玉因未見上面是何言詞,又不知黛玉心中如何,未敢造次回答,卻望著黛玉笑。黛玉一面讓寶釵坐,一面笑說道:“我曾見古史中有才色的女子,終身遭際令人可欣可羨可悲可嘆者甚多。今日飯後無事,因欲擇出數人,胡亂湊幾首詩以寄感慨,可巧探丫頭來會我瞧鳳姐姐去,我也身上懶懶的沒同他去。纔將做了五首,一時困倦起來,撂在那裏,不想二爺來了就瞧見了,其實給他看也倒沒有什麼,但只我嫌他是不是的寫給人看去。”寶玉忙道:“我多早晚給人看來呢。昨日那把扇子,原是我愛那幾首白海棠的詩,所以我自己用小楷寫了,不過爲的是在手中看著便易。我豈不知閨閣中詩詞字跡是輕易往外傳誦不得的。自從你說了,我總沒拿出園子去。”寶釵道:“林妹妹這慮的也是。你既寫在扇子上,偶然忘記了,拿在書房裏去被相公們看見了,豈有不問是誰做的呢。倘或傳揚開了,反爲不美。自古道‘女子無才便是德’,總以貞靜爲主,女工還是第二件。其餘詩詞,不過是閨中游戲,原可以會可以不會。咱們這樣人家的姑娘,倒不要這些才華的名譽。”因又笑向黛玉道:“拿出來給我看看無妨,只不叫寶兄弟拿出去就是了。” 黛玉笑道:“既如此說,連你也可以不必看了。”又指著寶玉笑道:“他早已搶了去了。”寶玉聽了,方自懷內取出,湊至寶釵身旁,一同細看。只見寫道:
\end{parag}


\begin{poem}
    \begin{pl}西施\end{pl}


    \begin{pl}一代傾城逐浪花,吳宮空自憶兒家。\end{pl}


    \begin{pl}效顰莫笑東村女,頭白溪邊尚浣紗。\end{pl}
    \emptypl

    \begin{pl}虞姬\end{pl}


    \begin{pl}腸斷烏騅夜嘯風,虞兮幽恨對重瞳。\end{pl}


    \begin{pl}黥彭甘受他年醢,飲劍何如楚帳中。\end{pl}
    \emptypl

    \begin{pl}明妃\end{pl}


    \begin{pl}絕豔驚人出漢宮,紅顏命薄古今同。\end{pl}


    \begin{pl}君王縱使輕顏色,予奪權何畀畫工?\end{pl}
    \emptypl

    \begin{pl}綠珠\end{pl}


    \begin{pl}瓦礫明珠一例拋,何曾石尉重嬌嬈。\end{pl}


    \begin{pl}都緣頑福前生造,更有同歸慰寂寥。\end{pl}
    \emptypl

    \begin{pl}紅拂\end{pl}


    \begin{pl}長揖雄談態自殊,美人巨眼識窮途。\end{pl}


    \begin{pl}尸居餘氣楊公幕,豈得羈縻女丈夫。\end{pl}


\end{poem}


\begin{parag}
    寶玉看了,讚不絕口,又說道:“妹妹這詩恰好只做了五首,何不就命曰《五美吟》。”於是不容分說,便提筆寫在後面。\begin{note}蒙雙夾:《五美吟》與後《十獨吟》對照。\end{note}寶釵亦說道:“做詩不論何題,只要善翻古人之意。若要隨人腳蹤走去,縱使字句精工,已落第二義,究竟算不得好詩,即如前人所詠昭君之詩甚多,有悲挽昭君的,有怨恨延壽的,又有譏漢帝不能使畫工圖貌賢臣而畫美人的。紛紛不一。後來王荊公復有‘意態由來畫不成,當時枉殺毛延壽’;永叔有‘耳目所見尚如此,萬里安能制夷狄’。二詩俱能各出己見,不與人同。今日林妹妹這五首詩,亦可謂命意新奇,別開生面了。”
\end{parag}


\begin{parag}
    仍欲往下說時,只見有人回道:“璉二爺回來了。適才外間傳說,往東府裏去了好一會了,想必就回來的。”寶玉聽了,連忙起身,迎至大門以內等待。恰好賈璉自外下馬進來。於是寶玉先迎著賈璉跪下,口中給賈母王夫人等請了安,又給賈璉請了安。二人攜手走了進來。只見李紈、鳳姐、寶釵、黛玉、迎、探、惜等早在中堂等候,一一相見已畢。因聽賈璉說道:“老太太明日一早到家,一路身體甚好。今日先打發了我來回家看視,明日五更,仍要出城迎接。”說畢,衆人又問了些路途的景況。因賈璉是遠歸,遂大家別過,讓賈璉回房歇息。一宿晚景,不必細述。
\end{parag}


\begin{parag}
    至次日飯時前後,果見賈母王夫人等到來。衆人接見已畢,略坐了一坐,吃了一杯茶,便領了王夫人等人過寧府中來。只聽見裏面哭聲震天,卻是賈赦賈璉送賈母到家即過這邊來了。當下賈母進入裏面,早有賈赦賈璉率領族中人哭著迎了出來。他父子一邊一個挽了賈母,走至靈前,又有賈珍賈蓉跪著撲入賈母懷中痛哭。賈母暮年人,見此光景,亦摟了珍蓉等痛哭不已。賈赦賈璉在旁苦勸,方略略止住。又轉至靈右,見了尤氏婆媳,不免又相持大痛一場。哭畢,衆人方上前一一請安問好。賈珍因賈母纔回家來,未得歇息,坐在此間,看著未免要傷心,遂再三求賈母回家;王夫人等亦再三相勸。賈母不得已,方回來了。果然年邁的人禁不住風霜傷感,至夜間便覺頭悶目酸,鼻塞聲重。連忙請了醫生來診脈下藥,足足的忙亂了半夜一日。幸而發散的快,未曾傳經,至三更天,些鬚髮了點汗,脈靜身涼,大家方放了心。至次日仍服藥調理。
\end{parag}


\begin{parag}
    又過了數日,乃賈敬送殯之期,賈母猶未大愈,遂留寶玉在家侍奉。鳳姐因未曾甚好,亦未去。其餘賈赦、賈璉、邢夫人、王夫人等率領家人僕婦,都送至鐵檻寺,至晚方回。賈珍尤氏並賈蓉仍在寺中守靈,等過百日後,方扶柩回籍。家中仍託尤老孃並二姐三姐照管。
\end{parag}


\begin{parag}
    卻說賈璉素日既聞尤氏姐妹之名,恨無緣得見。近因賈敬停靈在家,每日與二姐三姐相認已熟,不禁了垂涎之意。況知與賈珍賈蓉等素有聚麀之誚,因而乘機百般撩撥,眉目傳情。那三姐卻只是淡淡相對,只有二姐也十分有意。但只是眼目衆多,無從下手。賈璉又怕賈珍喫醋,不敢輕動,只好二人心領神會而已。此時出殯以後,賈珍家下人少,除尤老孃帶領二姐三姐並幾個粗使的丫鬟老婆子在正室居住外,其餘婢妾,都隨在寺中。外面僕婦,不過晚間巡更,日間看守門戶。白日無事,亦不進裏面去。所以賈璉便欲趁此下手。遂託相伴賈珍爲名,亦在寺中住宿,又時常藉著替賈珍料理家務,不時至寧府中來勾搭二姐。
\end{parag}


\begin{parag}
    一日,有小管家俞祿來回賈珍道:“前者所用棚槓孝布並請槓人青衣,共使銀一千一百十兩,除給銀五百兩外,仍欠六百零十兩。昨日兩外買賣人俱來催討,小的特來討爺的示下。”賈珍道:“你且向庫上領去就是了,這又何必來回我。”俞祿道:“昨日已曾上庫上去領,但只是老爺賓天以後,各處支領甚多,所剩還要預備百日道場及廟中用度,此時竟不能發給。所以小的今日特來回爺,或者爺內庫裏暫且發給,或者挪借何項,吩咐了小的好辦。”賈珍笑道:“你還當是先呢,有銀子放著不使。你無論那裏借了給他罷。”俞祿笑回道:“若說一二百,小的還可以挪借;這五六百,小的一時那裏辦得來。”賈珍想了一回,向賈蓉道:“你問你娘去,昨日出殯以後,有江南甄家送來打祭銀五百兩,未曾交到庫上去,你先要了來,給他去罷。”賈蓉答應了,連忙過這邊來回了尤氏,復轉來回他父親道:“昨日那項銀子已使了二百兩,下剩的三百兩令人送至家中交與老孃收了。”賈珍道:“既然如此,你就帶了他去,向你老孃要了出來交給他。再也瞧瞧家中有事無事,問你兩個姨娘好。下剩的俞祿先借了添上罷。”
\end{parag}


\begin{parag}
    賈蓉與俞祿答應了,方欲退出,只見賈璉走了進來。俞祿忙上前請了安。賈璉便問何事,賈珍一一告訴了。賈璉心中想道:“趁此機會正可至寧府尋二姐。”一面遂說道:“這有多大事,何必向人借去。昨日我方得了一項銀子還沒有使呢,莫若給他添上,豈不省事。”賈珍道:“如此甚好。你就吩咐了蓉兒,一併令他取去。”賈璉忙道:“這必得我親身取去。再我這幾日沒回家了,還要給老太太、老爺、太太們請請安去。到大哥那邊查查家人們有無生事,再也給親家太太請請安。”賈珍笑道:“只是又勞動你,我心裏倒不安。”賈璉也笑道:“自家兄弟,這有何妨呢。”賈珍又吩咐賈蓉道:“你跟了你叔叔去,也到那邊給老太太、老爺、太太們請安,說我和你娘都請安,打聽打聽老太太身上可大安了?還服藥呢沒有?”賈蓉一一答應了,跟隨賈璉出來,帶了幾個小廝,騎上馬一同進城。
\end{parag}


\begin{parag}
    在路叔侄閒話。賈璉有心,便提到尤二姐,因誇說如何標緻,如何做人好,舉止大方,言語溫柔,無一處不令人可敬可愛, “人人都說你嬸子好,據我看那裏及你二姨一零兒呢。”賈蓉揣知其意,便笑道:“叔叔既這麼愛他,我給叔叔作媒,說了做二房,何如?”賈璉笑道:“你這是頑話還是正經話?”賈蓉道:“我說的是當真的話。”賈璉又笑道:“敢自好呢。只是怕你嬸子不依,再也怕你老孃不願意。況且我聽見說你二姨兒已有了人家了。” 賈蓉道:“這都無妨。我二姨兒三姨兒都不是我老爺養的,原是我老孃帶了來的。聽見說,我老孃在那一家時,就把我二姨兒許給皇糧莊頭張家,指腹爲婚。後來張家遭了官司敗落了,我老孃又自那家嫁了出來,如今這十數年,兩家音信不通。我老孃時常報怨,要與他家退婚,我父親也要將二姨轉聘。只等有了好人家,不過令人找著張家,給他十幾兩銀子,寫上一張退婚的字兒。想張家窮極了的人,見了銀子,有什麼不依的。再他也知道咱們這樣的人家,也不怕他不依。又是叔叔這樣人說了做二房,我管保我老孃和我父親都願意。倒只是嬸子那裏卻難。”賈璉聽到這裏,心花都開了,那裏還有什麼話說,只是一味呆笑而已。賈蓉又想了一想,笑道:“叔叔若有膽量,依我的主意管保無妨,不過多花上幾個錢。”賈璉忙道:“有何主意,快些說來,我沒有不依的。”賈蓉道:“叔叔回家,一點聲色也別露,等我回明瞭我父親,向我老孃說妥,然後在咱們府後方近左右買上一所房子及應用傢伙,再撥兩窩子家人過去伏侍。擇了日子,人不知鬼不覺娶了過去,囑咐家人不許走漏風聲。嬸子在裏面住著,深宅大院,那裏就得知道了。叔叔兩下里住著,過個一年半載,即或鬧出來,不過捱上老爺一頓罵。叔叔只說嬸子總不生育,原是爲子嗣起見,所以私自在外面作成此事。就是嬸子,見生米做成熟飯,也只得罷了。再求一求老太太,沒有不完的事。”自古道“慾令智昏”,賈璉只顧貪圖二姐美色,聽了賈蓉一篇話,遂爲計出萬全,將現今身上有服,並停妻再娶,嚴父妒妻種種不妥之處,皆置之度外了。卻不知賈蓉亦非好意,素日因同他姨娘有情,只因賈珍在內,不能暢意。如今若是賈璉娶了,少不得在外居住,趁賈璉不在時,好去鬼混之意。賈璉那裏思想及此,遂向賈蓉致謝道:“好侄兒,你果然能夠說成了,我買兩個絕色的丫頭謝你。”說著,已至寧府門首。賈蓉說道:“叔叔進去,向我老孃要出銀子來,就交給俞祿罷。我先給老太太請安去。”賈璉含笑點頭道:“老太太跟前別說我和你一同來的。”賈蓉道:“知道。”又附耳向賈璉道:“今日要遇見二姨,可別性急了,鬧出事來,往後倒難辦了。”賈璉笑道:“少胡說,你快去罷。我在這裏等你。”於是賈蓉自去給賈母請安。
\end{parag}


\begin{parag}
    賈璉進入寧府,早有家人頭兒率領家人等請安,一路圍隨至廳上。賈璉一一的問了些話,不過塞責而已,便命家人散去,獨自往裏面走來。原來賈璉賈珍素日親密,又是弟兄,本無可避忌之人,自來是不等通報的。於是走至上房,早有廊下伺候的老婆子打起簾子,讓賈璉進去。賈璉進入房中一看,只見南邊炕上只有尤二姐帶著兩個丫鬟一處做活,卻不見尤老孃與三姐。賈璉忙上前問好想見。尤二姐含笑讓坐,便靠東邊排插兒坐下。賈璉仍將上首讓與二姐兒,說了幾句見面情兒,便笑問道:“親家太太和三妹妹那裏去了,怎麼不見?”尤二姐笑道:“纔有事往後頭去了,也就來的。”此時伺候的丫鬟因倒茶去,無人在跟前,賈璉不住的拿眼瞟著二姐。二姐低了頭,只含笑不理。賈璉又不敢造次動手動腳,因見二姐手中拿著一條拴著荷包的絹子擺弄,便搭訕著往腰裏摸了摸,說道:“檳榔荷包也忘記了帶了來,妹妹有檳榔,賞我一口喫。”二姐道:“檳榔倒有,就只是我的檳榔從來不給人喫。”賈璉便笑著欲近身來拿。二姐怕人看見不雅,便連忙一笑,撂了過來。賈璉接在手中,都倒了出來,揀了半塊喫剩下的撂在口中吃了,又將剩下的都揣了起來。剛要把荷包親身送過去,只見兩個丫鬟倒了茶來。賈璉一面接了茶喫茶,一面暗將自己帶的一個漢玉九龍珮解了下來,拴在手絹上,趁丫鬟回頭時,仍撂了過去。二姐亦不去拿,只裝看不見,坐著喫茶。只聽後面一陣簾子響,卻是尤老孃三姐帶著兩個小丫鬟自後面走來。賈璉送目與二姐,令其拾取,這尤二姐亦只是不理。賈璉不知二姐何意,甚是著急,只得迎上來與尤老孃三姐相見。一面又回頭看二姐時,只見二姐笑著,沒事人似的;再又看一看絹子,已不知那裏去了,賈璉方放了心。
\end{parag}


\begin{parag}
    於是大家歸坐後,敘了些閒話。賈璉說道:“大嫂子說,前日有一包銀子交給親家太太收起來了,今日因要還人,大哥令我來取。再也看看家裏有事無事。”尤老孃聽了,連忙使二姐拿鑰匙去取銀子。這裏賈璉又說道:“我也要給親家太太請請安,瞧瞧二位妹妹。親家太太臉面倒好,只是二位妹妹在我們家裏受委屈。”尤老孃笑道:“咱們都是至親骨肉,說那裏的話。在家裏也是住著,在這裏也是住著。不瞞二爺說,我們家裏自從先夫去世,家計也著實艱難了,全虧了這裏姑爺幫助。如今姑爺家裏有了這樣大事,我們不能別的出力,白看一看家,還有什麼委屈了的呢。”正說著,二姐已取了銀子來,交與尤老孃。尤老孃便遞與賈璉。賈璉叫一個小丫頭叫了一個老婆子來,吩咐他道:“你把這個交給俞祿,叫他拿過那邊去等我。”老婆子答應了出去。
\end{parag}


\begin{parag}
    只聽得院內是賈蓉的聲音說話。須臾進來,給他老孃姨娘請了安,又向賈璉笑道:“纔剛老爺還問叔叔呢,說是有什麼事情要使喚。原要使人到廟裏去叫,我回老爺說叔叔就來。老爺還吩咐我,路上遇著叔叔叫快去呢。”賈璉聽了,忙要起身,又聽賈蓉和他老孃說道:“那一次我和老太太說的,我父親要給二姨說的姨父,就和我這叔叔的面貌身量差不多兒。老太太說好不好?”一面說著,又悄悄的用手指著賈璉和他二姨努嘴。二姐倒不好意思說什麼,只見三姐似笑非笑、似惱非惱的罵道:“壞透了的小猴兒崽子!沒了你孃的說了!多早晚我才撕他那嘴呢!”一面說著,便趕了過來。賈蓉早笑著跑了出去,賈璉也笑著辭了出來。走至廳上,又吩咐了家人們不可耍錢喫酒等話。又 那牡難賈蓉,回去急速和他父親說。一面便帶了俞祿過來,將銀子添足,交給他拿去。一面給賈赦請安,又給賈母去請安不提。
\end{parag}


\begin{parag}
    卻說賈蓉見俞祿跟了賈璉去取銀子,自己無事,便仍回至裏面,和他兩個姨娘嘲戲一回,方起身。至晚到寺,見了賈珍回道:“銀子已經交給俞祿了。老太太已大愈了,如今已經不服藥了。”說畢,又趁便將路上賈璉要娶尤二姐做二房之意說了。又說如何在外面置房子住,不使鳳姐知道,“此時總不過爲的是子嗣艱難起見。爲的是二姨是見過的,親上做親,比別處不知道的人家說了來的好。所以二叔再三央我對父親說。”只不說是他自己的主意。賈珍想了想,笑道:“其實倒也罷了。只不知你二姨心中願意不願意。明日你先去和你老孃商量,叫你老孃問準了你二姨,再作定奪。”於是又教了賈蓉一篇話,便走過來將此事告訴了尤氏。尤氏卻知此事不妥,因而極力勸止。無奈賈珍主意已定,素日又是順從慣了的,況且他與二姐本非一母,不便深管,因而也只得由他們鬧去了。
\end{parag}


\begin{parag}
    至次日一早,果然賈蓉復進城來見他老孃,將他父親之意說了。又添上許多話,說賈璉做人如何好,目今鳳姐身子有病,已是不能好的了,暫且買了房子在外面住著,過個一年半載,只等鳳姐一死,便接了二姨進去做正室。又說他父親此時如何聘,賈璉那邊如何娶,如何接了你老人家養老,往後三姨也是那邊應了替聘,說得天花亂墜,不由得尤老孃不肯。況且素日全虧賈珍賙濟,此時又是賈珍作主替聘,而且妝奩不用自己置買,賈璉又是青年公子,比張華勝強十倍,遂連忙過來與二姐商議。二姐又是水性的人,在先已和姐夫不妥,又常怨恨當時錯許張華,致使後來終身失所,今見賈璉有情,況是姐夫將他聘嫁,有何不肯,也便點頭依允。當下回覆了賈蓉,賈蓉回了他父親。
\end{parag}


\begin{parag}
    次日命人請了賈璉到寺中來,賈珍當面告訴了他尤老孃應允之事。賈璉自是喜出望外,感謝賈珍賈蓉父子不盡。於是二人商量著,使人看房子打首飾,給二姐置買妝奩及新房中應用牀帳等物。不過幾日,早將諸事辦妥。已於寧榮街後二里遠近小花枝巷內買定一所房子,共二十餘間。又買了兩個小丫鬟。忽然想起家人鮑二來,當初他女人因和賈璉偷情,被鳳姐兒打鬧了一陣,含羞吊死了,賈璉叫林之孝許了二百兩發送,又梯己給鮑二些銀兩,另日挑個媳婦給他。鮑二又有體面,又有銀子,便仍然奉承賈璉。一時想起來,便叫了他兩口兒到新房子裏來,以備二姐兒過來時伏侍。又使人將張華父子叫來,逼勒著與尤老孃寫退婚書。卻說張華之祖,原當皇糧莊頭,後來死去。至張華父親時,仍充此役,因與尤老孃前夫相好,所以將張華與尤二姐指腹爲婚。後來不料遭了官司,敗落了家產,弄得衣食不周,那裏還娶得起媳婦呢。尤老孃又自那家嫁了出來,兩家有十數年音信不通。今被賈府家人喚至,逼他與二姐退婚,心中雖不願意,無奈懼怕賈珍等勢焰,不敢不依,只得寫了一張退婚文約。尤老孃與了二十兩銀子,兩家退親不提。
\end{parag}


\begin{parag}
    這裏賈璉等見諸事已妥,遂擇了初三黃道吉日,以便迎娶二姐過門。下回分解。正是:
\end{parag}


\begin{poem}
    \begin{pl}只爲同枝貪色慾,致叫連理起戈矛。\end{pl}

\end{poem}



\begin{parag}
    \begin{note}蒙回後總評:五首新詩何所居,顰兒應自日欷歔。柔腸一段千般結,豈是尋常望雁魚。\end{note}
\end{parag}


\begin{parag}
    \begin{note}蒙回後總評:五百年風流債,一見了偏作怪。你貪我愛自難休,天巧姻緣渾無奈。父母者,子女間,莫失教訓說前緣。防微之處休弛謝,嚴厲才能真愛憐。\end{note}
\end{parag}