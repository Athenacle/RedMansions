\chap{一百一十六}{得通灵幻境悟仙缘 送慈柩故乡全孝道}



\begin{parag}
    话说宝玉一听麝月的话,身往后仰,复又死去,急得王夫人等哭叫不止。麝月自知失言致祸,此时王夫人等也不及说他。那麝月一面哭着,一面打定主意,心想:“若是宝玉一死,我便自尽跟了他去!”不言麝月心里的事。且言王夫人等见叫不回来,赶着叫人出来找和尚救治。岂知贾政进内出去时,那和尚已不见了。贾政正在诧异,听见里头又闹,急忙进来。见宝玉又是先前的样子,口关紧闭,脉息全无。用手在心窝中一摸,尚是温热。贾政只得急忙请医灌药救治。
\end{parag}


\begin{parag}
    那知那宝玉的魂魄早已出了窍了。你道死了不成?却原来恍恍惚惚赶到前厅,见那送玉的和尚坐着,便施了礼。那知和尚站起身来,拉着宝玉就走。宝玉跟了和尚,觉得身轻如叶,飘飘摇摇,也没出大门,不知从那里走了出来。行了一程,到了个荒野地方,远远的望见一座牌楼,好象曾到过的。正要问那和尚时,只见恍恍惚惚来了一个女人。宝玉心里想道:“这样旷野地方,那得有如此的丽人,必是神仙下界了。”宝玉想着,走近前来细细一看,竟有些认得的,只是一时想不起来。见那女人和和尚打了一个照面就不见了。宝玉一想,竟是尤三姐的样子,越发纳闷:“怎么他也在这里?”又要问时,那和尚拉着宝玉过了那牌楼,只见牌上写着“真如福地”四个大字,两边一幅对联,乃是:
\end{parag}


\begin{poem}
    \begin{pl}
        假去真来真胜假,无原有是有非无。
    \end{pl}
\end{poem}


\begin{parag}
    转过牌坊,便是一座宫门。门上横书四个大字道“福善祸淫”。又有一副对子,大书云:
\end{parag}


\begin{poem}
    \begin{pl}
        过去未来,莫谓智贤能打破,
    \end{pl}


    \begin{pl}
        前因后果,须知亲近不相逢。
    \end{pl}

\end{poem}


\begin{parag}
    宝玉看了,心下想道:“原来如此。我倒要问问因果来去的事了。”这么一想,只见鸳鸯站在那里招手儿叫他。宝玉想道:“我走了半日,原不曾出园子,怎么改了样子了呢?”赶着要和鸳鸯说话,岂知一转眼便不见了,心里不免疑惑起来。走到鸳鸯站的地方儿,乃是一溜配殿,各处都有匾额。宝玉无心去看,只向鸳鸯立的所在奔去。见那一间配殿的门半掩半开,宝玉也不敢造次进去,心里正要问那和尚一声,回过头来,和尚早已不见了。宝玉恍惚,见那殿宇巍峨,绝非大观园景象。便立住脚,抬头看那匾额上写道:“引觉情痴”。两边写的对联道:
\end{parag}

\begin{poem}
    \begin{pl}
        喜笑悲哀都是假,贪求思慕总因痴。
    \end{pl}
\end{poem}


\begin{parag}
    宝玉看了,便点头叹息。想要进去找鸳鸯问他是什么所在,细细想来甚是熟识,便仗着胆子推门进去。满屋一瞧,并不见鸳鸯,里头只是黑漆漆的,心下害怕。正要退出,见有十数个大橱,橱门半掩。
\end{parag}


\begin{parag}
    宝玉忽然想起:“我少时做梦曾到过这个地方。如今能够亲身到此,也是大幸。”恍惚间,把找鸳鸯的念头忘了。便壮着胆把上首的大橱开了橱门一瞧,见有好几本册子,心里更觉喜欢,想道:“大凡人做梦,说是假的,岂知有这梦便有这事。我常说还要做这个梦再不能的,不料今儿被我找着了。但不知那册子是那个见过的不是?”伸手在上头取了一本,册上写着“金陵十二钗正册”。宝玉拿着一想道:“我恍惚记得是那个,只恨记不得清楚。”便打开头一页看去,见上头有画,但是画迹模糊,再瞧不出来。后面有几行字迹也不清楚,尚可摹拟,便细细的看去,见有什么“玉带”,上头有个好象“林”字,心里想道:“不要是说林妹妹罢?”便认真看去,底下又有“金簪雪里”四字,诧异道”怎么又象他的名字呢。”复将前后四句合起来一念道:“也没有什么道理,只是暗藏着他两个名字,并不为奇。独有那‘怜’字‘叹’字不好。这是怎么解?”想到那里,又自啐道:“我是偷着看,若只管呆想起来,倘有人来,又看不成了。”遂往后看去,也无暇细玩那图画,只从头看去。看到尾儿有几句词,什么“相逢大梦归”一句,便恍然大悟道:“是了,果然机关不爽,这必是元春姐姐了。若都是这样明白,我要抄了去细玩起来,那些姊妹们的寿夭穷通没有不知的了。我回去自不肯泄漏,只做一个未卜先知的人,也省了多少闲想。”又向各处一瞧,并没有笔砚,又恐人来,只得忙着看去。只见图上影影有一个放风筝的人儿,也无心去看。急急的将那十二首诗词都看遍了。也有一看便知的,也有一想便得的,也有不大明白的,心下牢牢记着。一面叹息,一面又取那《金陵又副册》一看,看到“堪羡优伶有福,谁知公子无缘”先前不懂,见上面尚有花席的影子,便大惊痛哭起来。
\end{parag}


\begin{parag}
    待要往后再看,听见有人说道:“你又发呆了!林妹妹请你呢。”好似鸳鸯的声气,回头却不见人。心中正自惊疑,忽鸳鸯在门外招手。宝玉一见,喜得赶出来。但见鸳鸯在前影影绰绰的走,只是赶不上。宝玉叫道:“好姐姐,等等我。”那鸳鸯并不理,只顾前走。宝玉无奈,尽力赶去,忽见别有一洞天,楼阁高耸,殿角玲珑,且有好些宫女隐约其间。宝玉贪看景致,竟将鸳鸯忘了。宝玉顺步走入一座宫门,内有奇花异卉,都也认不明白。惟有白石花阑围着一颗青草,叶头上略有红色,但不知是何名草,这样矜贵。只见微风动处,那青草已摇摆不休,虽说是一枝小草,又无花朵,其妩媚之态,不禁心动神怡,魂消魄丧。宝玉只管呆呆的看着,只听见旁边有一人说道:“你是那里来的蠢物,在此窥探仙草!”宝玉听了,吃了一惊,回头看时,却是一位仙女,便施礼道:“我找鸳鸯姐姐,误入仙境,恕我冒昧之罪。请问神仙姐姐,这里是何地方?怎么我鸳鸯姐姐到此还说是林妹妹叫我?望乞明示。”那人道:“谁知你的姐姐妹妹,我是看管仙草的,不许凡人在此逗留。”宝玉欲待要出来,又舍不得,只得央告道:“神仙姐姐既是那管理仙草的,必然是花神姐姐了。但不知这草有何好处?”那仙女道:“你要知道这草,说起来话长着呢。那草本在灵河岸上,名曰绛珠草。因那时萎败,幸得一个神瑛侍者日以甘露灌溉,得以长生。后来降凡历劫,还报了灌溉之恩,今返归真境。所以警幻仙子命我看管,不令蜂缠蝶恋。”宝玉听了不解,一心疑定必是遇见了花神了,今日断不可当面错过,便问:“管这草的是神仙姐姐了。还有无数名花必有专管的,我也不敢烦问,只有看管芙蓉花的是那位神仙?”那仙女道:“我却不知,除是我主人方晓。”宝玉便问道:“姐姐的主人是谁?”那仙女道:“我主人是潇湘妃子。”宝玉听道:“是了,你不知道这位妃子就是我的表妹林黛玉。”那仙女道:“胡说。此地乃上界神女之所,虽号为潇湘妃子,并不是娥皇女英之辈,何得与凡人有亲。你少来混说,瞧着叫力士打你出去。”
\end{parag}


\begin{parag}
    宝玉听了发怔,只觉自形秽浊,正要退出,又听见有人赶来说道:“里面叫请神瑛侍者。”那人道:“我奉命等了好些时,总不见有神瑛侍者过来,你叫我那里请去。”那一个笑道:“才退去的不是么?”那侍女慌忙赶出来说:“请神瑛侍者回来。”宝玉只道是问别人,又怕被人追赶,只得踉跄而逃。正走时,只见一人手提宝剑迎面拦住说:“那里走!”唬得宝玉惊慌无措,仗着胆抬头一看却不是别人,就是尤三姐。宝玉见了,略定些神,央告道:“姐姐怎么你也来逼起我来了。”那人道:“你们兄弟没有一个好人,败人名节,破人婚姻。今儿你到这里,是不饶你的了!”宝玉听去话头不好,正自着急,只听后面有人叫道:“姐姐快快拦住,不要放他走了。”尤三姐道:“我奉妃子之命等侯已久,今儿见了,必定要一剑斩断你的尘缘。”宝玉听了益发着忙,又不懂这些话到底是什么意思,只得回头要跑。岂知身后说话的并非别人,却是晴雯。宝玉一见,悲喜交集,便说:“我一个人走迷了道儿,遇见仇人,我要逃回,却不见你们一人跟着我。如今好了,晴雯姐姐,快快的带我回家去罢。”晴雯道:“侍者不必多疑,我非晴雯,我是奉妃子之命特来请你一会,并不难为你。”宝玉满腹狐疑,只得问道:“姐姐说是妃子叫我,那妃子究是何人?”晴雯道:“此时不必问,到了那里自然知道。”宝玉没法,只得跟着走。细看那人背后举动恰是晴雯,那面目声音是不错的了,”怎么他说不是?我此时心里模糊。且别管他,到了那边见了妃子,就有不是,那时再求他,到底女人的心肠是慈悲的,必是恕我冒失。”正想着,不多时到了一个所在。只见殿宇精致,色彩辉煌,庭中一丛翠竹,户外数本苍松。廊檐下立着几个侍女,都是宫妆打扮,见了宝玉进来,便悄悄的说道:“这就是神瑛侍者么?”引着宝玉的说道:“就是。你快进去通报罢。”有一侍女笑着招手,宝玉便跟着进去。过了几层房舍,见一正房,珠帘高挂。那侍女说:“站着候旨。”宝玉听了,也不敢则声,只得在外等着。那侍女进去不多时,出来说:“请侍者参见。”又有一人卷起珠帘。只见一女子,头戴花冠,身穿绣服,端坐在内。宝玉略一抬头,见是黛玉的形容,便不禁的说道:“妹妹在这里!叫我好想。”那帘外的侍女悄咤道:“这侍者无礼,快快出去。”说犹未了,又见一个侍儿将珠帘放下。宝玉此时欲待进去又不敢,要走又不舍,待要问明,见那些侍女并不认得,又被驱逐,无奈出来。心想要问晴雯,回头四顾,并不见有晴雯。心下狐疑,只得怏怏出来,又无人引着,正欲找原路而去,却又找不出旧路了。正在为难,见凤姐站在一所房檐下招手。宝玉看见喜欢道:“可好了,原来回到自己家里了。我怎么一时迷乱如此。”急奔前来说:“姐姐在这里么,我被这些人捉弄到这个分儿。林妹妹又不肯见我,不知何原故。”说着,走到凤姐站的地方,细看起来并不是凤姐,原来却是贾蓉的前妻秦氏。宝玉只得立住脚要问“凤姐姐在那里”,那秦氏也不答言,竟自往屋里去了。宝玉恍恍惚惚的又不敢跟进去,只得呆呆的站着,叹道:“我今儿得了什么不是,众人都不理我。”便痛哭起来。见有几个黄巾力士执鞭赶来,说是”何处男人敢闯入我们这天仙福地来,快走出去!”宝玉听得,不敢言语。正要寻路出来,远远望见一群女子说笑前来。宝玉看时,又象有迎春等一干人走来,心里喜欢,叫道:“我迷住在这里,你们快来救我!”正嚷着,后面力士赶来。宝玉急得往前乱跑,忽见那一群女子都变作鬼怪形像,也来追扑。
\end{parag}


\begin{parag}
    宝玉正在情急,只见那送玉来的和尚手里拿着一面镜子一照,说道:“我奉元妃娘娘旨意,特来救你。”登时鬼怪全无仍是一片荒郊。宝玉拉着和尚说道:“我记得是你领我到这里,你一时又不见了。看见了好些亲人,只是都不理我,忽又变作鬼怪,到底是梦是真,望老师明白指示。”那和尚道:“你到这里曾偷看什么东西没有?”宝玉一想道:“他既能带我到天仙福地,自然也是神仙了,如何瞒得他。况且正要问个明白。”便道:“我倒见了好些册子来着。”那和尚道:“可又来,你见了册子还不解么!世上的情缘都是那些魔障。只要把历过的事情细细记着,将来我与你说明。”说着,把宝玉狠命的一推,说:“回去罢!”宝玉站不住脚,一交跌倒,口里嚷道:“阿哟!”
\end{parag}


\begin{parag}
    王夫人等正在哭泣,听见宝玉苏来,连忙叫唤。宝玉睁眼看时,仍躺在炕上,见王夫人宝钗等哭的眼泡红肿。定神一想,心里说道:“是了,我是死去过来的。”遂把神魂所历的事呆呆的细想,幸喜多还记得,便哈哈的笑道:“是了,是了。”王夫人只道旧病复发,便好延医调治,即命丫头婆子快去告诉贾政,说是”宝玉回过来了,头里原是心迷住了,如今说出话来,不用备办后事了。”贾政听了,即忙进来看视,果见宝玉苏来,便道:“没福的痴儿你要唬死谁么!”说着,眼泪也不知不觉流下来了。又叹了几口气,仍出去叫人请医生诊脉服药。这里麝月正思自尽,见宝玉一过来,也放了心。只见王夫人叫人端了桂圆汤叫他喝了几口,渐渐的定了神。王夫人等放心,也没有说麝月,只叫人仍把那玉交给宝钗给他带上,”想起那和尚来,这玉不知那里找来的,也是古怪。怎么一时要银一时又不见了,莫非是神仙不成?”宝钗道:“说起那和尚来的踪迹去的影响,那玉并不是找来的。头里丢的时候,必是那和尚取去的。”王夫人道:“玉在家里怎么能取的了去?”宝钗道:“既可送来,就可取去。”袭人麝月道:“那年丢了玉,林大爷测了个字,后来二奶奶过了门,我还告诉过二奶奶,说测的那字是什么‘赏’字。二奶奶还记得么?”宝钗想道:“是了。你们说测的是当铺里找去,如今才明白了,竟是个和尚的‘尚’字在上头,可不是和尚取了去的么。”王夫人道:“那和尚本来古怪。那年宝玉病的时候,那和尚来说是我们家有宝贝可解,说的就是这块玉了。他既知道,自然这块玉到底有些来历。况且你女婿养下来就嘴里含着的。古往今来,你们听见过这么第二个么。只是不知终久这块玉到底是怎么着,就连咱们这一个也还不知是怎么着。病也是这块玉,好也是这块玉,生也是这块玉——”说到这里忽然住了,不免又流下泪来。宝玉听了,心里却也明白,更想死去的事愈加有因,只不言语,心里细细的记忆。那时惜春便说道:“那年失玉,还请妙玉请过仙,说是‘青埂峰下倚古松’,还有什么‘入我门来一笑逢’的话,想起来‘入我门’三字大有讲究。佛教的法门最大,只怕二哥不能入得去。”宝玉听了,又冷笑几声。宝钗听了,不觉的把眉头儿肐揪着发起怔来。尤氏道:“偏你一说又是佛门了。你出家的念头还没有歇么?”惜春笑道:“不瞒嫂子说,我早已断了荤了。”王夫人道:“好孩子,阿弥陀佛,这个念头是起不得的。”惜春听了,也不言语。宝玉想“青灯古佛前”的诗句,不禁连叹几声。忽又想起一床席一枝花的诗句来,拿眼睛看着袭人,不觉又流下泪来。众人都见他忽笑忽悲,也不解是何意,只道是他的旧病。岂知宝玉触处机来,竟能把偷看册上诗句俱牢牢记住了,只是不说出来,心中早有一个成见在那里了。暂且不题。
\end{parag}


\begin{parag}
    且说众人见宝玉死去复生,神气清爽,又加连日服药,一天好似一天,渐渐的复原起来。便是贾政见宝玉已好,现在丁忧无事,想起贾赦不知几时遇赦,老太太的灵柩久停寺内,终不放心,欲要扶柩回南安葬,便叫了贾琏来商议。贾琏便道:“老爷想得极是,如今趁着丁忧干了一件大事更好。将来老爷起了服,生恐又不能遂意了。但是我父亲不在家,侄儿呢又不敢僭越。老爷的主意很好,只是这件事也得好几千银子。衙门里缉赃那是再缉不出来的。”贾政道:“我的主意是定了,只为大爷不在家,叫你来商议商议怎么个办法。你是不能出门的。现在这里没有人,我为是好几口材都要带回去的,一个怎么样的照应呢,想起把蓉哥儿带了去。况且有他媳妇的棺材也在里头。还有你林妹妹的,那是老太太的遗言说跟着老太太一块儿回去的。我想这一项银子只好在那里挪借几千,也就够了。”贾琏道:“如今的人情过于淡薄。老爷呢,又丁忧,我们老爷呢,又在外头,一时借是借不出来的了。只好拿房地文书出去押去。”贾政道:“住的房子是官盖的,那里动得。”贾琏道:“住房是不能动的。外头还有几所可以出脱的,等老爷起复后再赎也使得。将来我父亲回来了,倘能也再起用,也好赎的。只是老爷这么大年纪,辛苦这一场,侄儿们心里实不安。”贾政道:“老太太的事,是应该的。只要你在家谨慎些,把持定了才好。”贾琏道:“老爷这倒只管放心,侄儿虽糊涂,断不敢不认真办理的。况且老爷回南少不得多带些人去,所留下的人也有限了,这点子费用还可以过的来。就是老爷路上短少些,必经过赖尚荣的地方,可也叫他出点力儿。”贾政道:“自己的老人家的事,叫人家帮什么。”贾琏答应了“是”,便退出来打算银钱。
\end{parag}


\begin{parag}
    贾政便告诉了王夫人,叫他管了家,自己便择了发引长行的日子,就要起身。宝玉此时身体复元,贾环贾兰倒认真念书,贾政都交付给贾琏,叫他管教,“今年是大比的年头。环儿是有服的,不能入场,兰儿是孙子,服满了也可以考的,务必叫宝玉同着侄儿考去。能够中一个举人,也好赎一赎咱们的罪名。”贾琏等唯唯应命。贾政又吩咐了在家的人,说了好些话,才别了宗祠,便在城外念了几天经,就发引下船,带了林之孝等而去。也没有惊动亲友,惟有自家男女送了一程回来。
\end{parag}


\begin{parag}
    宝玉因贾政命他赴考,王夫人便不时催逼查考起他的工课来。那宝钗袭人时常劝勉,自不必说。那知宝玉病后虽精神日长,他的念头一发更奇僻了,竟换了一种。不但厌弃功名仕进,竟把那儿女情缘也看淡了好些。只是众人不大理会,宝玉也并不说出来。一日,恰遇紫鹃送了林黛玉的灵柩回来,闷坐自己屋里啼哭,想道:“宝玉无情,见他林妹妹的灵柩回去并不伤心落泪,见我这样痛哭也不来劝慰,反瞅着我笑。这样负心的人,从前都是花言巧语来哄着我们!前夜亏我想得开,不然几乎又上了他的当。只是一件叫人不解,如今我看他待袭人等也是冷冷儿的。二奶奶是本来不喜欢亲热的,麝月那些人就不抱怨他么?我想女孩子们多半是痴心的,白操了那些时的心,看将来怎样结局!”正想着,只见五儿走来瞧他,见紫鹃满面泪痕,便说:“姐姐又想林姑娘了?想一个人闻名不如眼见,头里听着宝二爷女孩子跟前是最好的,我母亲再三的把我弄进来。岂知我进来了,尽心竭力的伏侍了几次病,如今病好了,连一句好话也没有剩出来,如今索性连眼儿也都不瞧了。”紫鹃听他说的好笑,便噗嗤的一笑,啐道:“呸,你这小蹄子,你心里要宝玉怎么个样儿待你才好?女孩儿家也不害臊,连名公正气的屋里人瞧着他还没事人一大堆呢,有功夫理你去!”因又笑着拿个指头往脸上抹着问道:“你到底算宝玉的什么人哪?”那五儿听了,自知失言,便飞红了脸。待要解说不是要宝玉怎么看待,说他近来不怜下的话,只听院门外乱嚷说:“外头和尚又来了,要那一万银子呢。太太着急,叫琏二爷和他讲去,偏偏琏二爷又不在家。那和尚在外头说些疯话,太太叫请二奶奶过去商量。”不知怎样打发那和尚,下回分解。
\end{parag}