\chap{一百一十三}{懺宿冤鳳姐託村嫗 釋舊憾情婢感癡郎}



\begin{parag}
    話說趙姨娘在寺內得了暴病,見人少了,更加混說起來,唬得衆人都恨,就有兩個女人攙着。趙姨娘雙膝跪在地下,說一回,哭一回,有時爬在地下叫饒,說:“打殺我了!紅鬍子的老爺,我再不敢了。”有一時雙手合着,也是叫疼。眼睛突出,嘴裏鮮血直流,頭髮披散,人人害怕,不敢近前。那時又將天晚,趙姨娘的聲音只管喑啞起來了,居然鬼嚎一般。無人敢在他跟前,只得叫了幾個有膽量的男人進來坐着,趙姨娘一時死去,隔了些時又回過來,整整的鬧了一夜。到了第二天,也不言語,只裝鬼臉,自己拿手撕開衣服,露出胸膛,好象有人剝他的樣子。可憐趙姨娘雖說不出來,其痛苦之狀實在難堪。正在危急,大夫來了,也不敢診,只囑咐“辦理後事罷”,說了起身就走。那送大夫的家人再三央告說:“請老爺看看脈,小的好回稟家主。”那大夫用手一摸,已無脈息。賈環聽了,然後大哭起來。衆人只顧賈環,誰料理趙姨娘。只有周姨娘心裏苦楚,想到:“做偏房側室的下場頭不過如此!況他還有兒子的,我將來死起來還不知怎樣呢!”於是反哭的悲切。且說那人趕回家去回稟了。賈政即派家人去照例料理,陪着環兒住了三天,一同回來。
\end{parag}


\begin{parag}
    那人去了,這裏一人傳十,十人傳百,都知道趙姨娘使了毒心害人被陰司裏拷打死了。又說是“璉二奶奶只怕也好不了,怎麼說璉二奶奶告的呢。”這些話傳到平兒耳內,甚是着急,看着鳳姐的樣子實在是不能好的了,看着賈璉近日並不似先前的恩愛,本來事也多,竟象不與他相干的。平兒在鳳姐跟前只管勸慰,又想着邢王二夫人回家幾日,只打發人來問問,並不親身來看。鳳姐心裏更加悲苦。賈璉回來也沒有一句貼心的話。鳳姐此時只求速死,心裏一想,邪魔悉至。只見尤二姐從房後走來,漸近牀前說:“姐姐,許久的不見了。做妹妹的想念的很,要見不能,如今好容易進來見見姐姐。姐姐的心機也用盡了,咱們的二爺糊塗,也不領姐姐的情,反倒怨姐姐作事過於苛刻,把他的前程去了,叫他如今見不得人。我替姐姐氣不平。”鳳姐恍惚說道:“我如今也後悔我的心忒窄了,妹妹不念舊惡,還來瞧我。”平兒在旁聽見,說道:“奶奶說什麼?”鳳姐一時甦醒,想起尤二姐已死,必是他來索命。被平兒叫醒,心裏害怕,又不肯說出,只得勉強說道:“我神魂不定,想是說夢話。給我捶捶。”平兒上去捶着,見個小丫頭子進來,說是“劉姥姥來了,婆子們帶着來請奶奶的安。”平兒急忙下來說:“在那裏呢?”小丫頭子說:“他不敢就進來,還聽奶奶的示下。”平兒聽了點頭,想鳳姐病裏必是懶待見人,便說道:“奶奶現在養神呢,暫且叫他等着。你問他來有什麼事麼?”小丫頭子說道:“他們問過了,沒有事。說知道老太太去世了,因沒有報纔來遲了。”小丫頭子說着,鳳姐聽見,便叫“平兒,你來,人家好心來瞧,不要冷淡人家。你去請了劉姥姥進來,我和他說說話兒。”平兒只得出來請劉姥姥這裏坐。
\end{parag}


\begin{parag}
    鳳姐剛要閤眼,又見一個男人一個女人走向炕前,就象要上炕似的。鳳姐着忙,便叫平兒說:“那裏來了一個男人跑到這裏來了!”連叫兩聲,只見豐兒小紅趕來說:“奶奶要什麼?”鳳姐睜眼一瞧,不見有人,心裏明白,不肯說出來,便問豐兒道:“平兒這東西那裏去了?”豐兒道:“不是奶奶叫去請劉姥姥去了麼。”鳳姐定了一會神,也不言語。
\end{parag}


\begin{parag}
    只見平兒同劉姥姥帶了一個小女孩兒進來,說:“我們姑奶奶在那裏?”平兒引到炕邊,劉姥姥便說:“請姑奶奶安。”鳳姐睜眼一看,不覺一陣傷心,說:“姥姥你好?怎麼這時候纔來?你瞧你外孫女兒也長的這麼大了。”劉姥姥看着鳳姐骨瘦如柴,神情恍惚,心裏也就悲慘起來,說:“我的奶奶,怎麼這幾個月不見,就病到這個分兒。我糊塗的要死,怎麼不早來請姑奶奶的安!”便叫青兒給姑奶奶請安。青兒只是笑,鳳姐看了倒也十分喜歡,便叫小紅招呼着。劉姥姥道:“我們屯鄉里的人不會病的,若一病了就要求神許願,從不知道吃藥的。我想姑奶奶的病不要撞着什麼了罷?”平兒聽着那話不在理,便在背地裏扯他。劉姥姥會意,便不言語。那裏知道這句話倒合了鳳姐的意,扎掙着說:“姥姥你是有年紀的人,說的不錯。你見過的趙姨娘也死了,你知道麼?”劉姥姥詫異道:“阿彌陀佛!好端端一個人怎麼就死了?我記得他也有一個小哥兒,這便怎麼樣呢?”平兒道:“這怕什麼,他還有老爺太太呢。”劉姥姥道:“姑娘,你那裏知道,不好死了是親生的,隔了肚皮子是不中用的。”這句話又招起鳳姐的愁腸,嗚嗚咽咽的哭起來了。衆人都來勸解。
\end{parag}


\begin{parag}
    巧姐兒聽見他母親悲哭,便走到炕前用手拉着鳳姐的手,也哭起來。鳳姐一面哭着道:“你見過了姥姥了沒有?”巧姐兒道:“沒有。”鳳姐道:“你的名字還是他起的呢,就和乾孃一樣,你給他請個安。”巧姐兒便走到跟前,劉姥姥忙着拉着道:“阿彌陀佛,不要折殺我了!巧姑娘,我一年多不來,你還認得我麼?”巧姐兒道:“怎麼不認得。那年在園裏見的時候我還小,前年你來,我還合你要來年的蟈蟈兒,你也沒有給我,必是忘了。”劉姥姥道:“好姑娘,我是老糊塗了。若說蟈蟈兒,我們屯裏多得很,只是不到我們那裏去,若去了,要一車也容易。”鳳姐道:“不然你帶了他去罷。”劉姥姥笑道:“姑娘這樣千金貴體,綾羅裹大了的,喫的是好東西,到了我們那裏,我拿什麼哄他頑,拿什麼給他喫呢?這倒不是坑殺我了麼。”說着,自己還笑,他說:“那麼着,我給姑娘做個媒罷。我們那裏雖說是屯鄉里,也有大財主人家,幾千頃地,幾百牲口,銀子錢亦不少,只是不象這裏有金的,有玉的。姑奶奶是瞧不起這種人家,我們莊家人瞧着這樣大財主,也算是天上的人了。”鳳姐道:“你說去,我願意就給。”劉姥姥道:“這是頑話兒罷咧。放着姑奶奶這樣,大官大府的人家只怕還不肯給,那裏肯給莊家人。就是姑奶奶肯了,上頭太太們也不給。”巧姐因他這話不好聽,便走了去和青兒說話。兩個女孩兒倒說得上,漸漸的就熟起來了。
\end{parag}


\begin{parag}
    這裏平兒恐劉姥姥話多,攪煩了鳳姐,便拉了劉姥姥說:“你提起太太來,你還沒有過去呢。我出去叫人帶了你去見見,也不枉來這一趟。”劉姥姥便要走。鳳姐道:“忙什麼,你坐下,我問你近來的日子還過的麼?”劉姥姥千恩萬謝的說道:“我們若不仗着姑奶奶”,說着,指着青兒說:“他的老子娘都要餓死了。如今雖說是莊家人苦,家裏也掙了好幾畝地,又打了一眼井,種些菜蔬瓜果,一年賣的錢也不少,儘夠他們嚼喫的了。這兩年姑奶奶還時常給些衣服布匹,在我們村裏算過得的了。阿彌陀佛,前日他老子進城,聽見姑奶奶這裏動了家,我就幾乎唬殺了。虧得又有人說不是這裏,我才放心。後來又聽見說這裏老爺升了,我又喜歡,就要來道喜,爲的是滿地的莊家來不得。昨日又聽說老太太沒有了,我在地裏打豆子,聽見了這話,唬得連豆子都拿不起來了,就在地裏狠狠的哭了一大場。我和女婿說,我也顧不得你們了,不管真話謊話,我是要進城瞧瞧去的。我女兒女婿也不是沒良心的,聽見了也哭了一回子,今兒天沒亮就趕着我進城來了。我也不認得一個人,沒有地方打聽,一徑來到後門,見是門神都糊了,我這一唬又不小。進了門找周嫂子,再找不着,撞見一個小姑娘,說周嫂子他得了不是了,攆了。我又等了好半天,遇見了熟人,才得進來。不打諒姑奶奶也是那麼病。”說着,又掉下淚來。平兒等着急,也不等他說完拉着就走,說:“你老人家說了半天,口乾了,咱們喝碗茶去罷。”拉着劉姥姥到下房坐着,青兒在巧姐兒那邊。劉姥姥道:“茶倒不要。好姑娘,叫人帶了我去請太太的安,哭哭老太太去罷。”平兒道:“你不用忙,今兒也趕不出城的了。方纔我是怕你說話不防頭招的我們奶奶哭,所以催你出來的。別思量。”劉姥姥道:“阿彌陀佛,姑娘是你多心,我知道。倒是奶奶的病怎麼好呢?”平兒道:“你瞧去妨礙不妨礙?”劉姥姥道:“說是罪過,我瞧着不好。”正說着,又聽鳳姐叫呢。平兒及到牀前,鳳姐又不言語了。平兒正問豐兒,賈璉進來,向炕上一瞧,也不言語,走到裏間氣哼哼的坐下。只有秋桐跟了進去,倒了茶,殷勤一回,不知嘁嘁喳喳的說些什麼。回來賈璉叫平兒來問道:“奶奶不吃藥麼?”平兒道:“不吃藥。怎麼樣呢?”賈璉道:“我知道麼!你拿櫃子上的鑰匙來罷。”平兒見賈璉有氣,又不敢問,只得出來鳳姐耳邊說了一聲。鳳姐不言語,平兒便將一個匣子擱在賈璉那裏就走。賈璉道:“有鬼叫你嗎!你擱着叫誰拿呢?”平兒忍氣打開,取了鑰匙開了櫃子,便問道:“拿什麼?”賈璉道:“咱們有什麼嗎?”平兒氣得哭道:“有話明白說,人死了也願意!”賈璉道:“還要說麼!頭裏的事是你們鬧的。如今老太太的還短了四五千銀子,老爺叫我拿公中的地帳弄銀子,你說有麼?外頭拉的帳不開發使得麼?誰叫我應這個名兒!只好把老太太給我的東西折變去罷了。你不依麼?”平兒聽了,一句不言語,將櫃裏東西搬出。只見小紅過來說:“平姐姐快走,奶奶不好呢。”平兒也顧不得賈璉,急忙過來,見鳳姐用手空抓,平兒用手攥着哭叫。賈璉也過來一瞧,把腳一跺道:“若是這樣,是要我的命了。”說着,掉下淚來。豐兒進來說:“外頭找二爺呢。”賈璉只得出去。
\end{parag}


\begin{parag}
    這裏鳳姐愈加不好,豐兒等不免哭起來。巧姐聽見趕來。劉姥姥也急忙走到炕前,嘴裏唸佛,搗了些鬼,果然鳳姐好些。一時王夫人聽了丫頭的信,也過來了,先見鳳姐安靜些,心下略放心,見了劉姥姥,便說:“劉姥姥你好?什麼時候來的?”劉姥姥便說:“請太太安。”不及細說,只言鳳姐的病。講究了半天,彩雲進來說:“老爺請太太呢。”王夫人叮嚀了平兒幾句話,便過去了。鳳姐鬧了一回,此時又覺清楚些,見劉姥姥在這裏,心裏信他求神禱告,便把豐兒等支開,叫劉姥姥坐在頭邊,告訴他心神不寧如見鬼怪的樣。劉姥姥便說我們屯裏什麼菩薩靈,什麼廟有感應。鳳姐道:“求你替我禱告,要用供獻的銀錢我有。”便在手腕上褪下一支金鐲子來交給他。劉姥姥道:“姑奶奶,不用那個。我們村莊人家許了願,好了,花上幾百錢就是了,那用這些。就是我替姑奶奶求去,也是許願。等姑奶奶好了,要花什麼自己去花罷。”鳳姐明知劉姥姥一片好心,不好勉強,只得留下,說:“姥姥,我的命交給你了。我的巧姐兒也是千災百病的,也交給你了。”劉姥姥順口答應,便說:“這麼着,我看天氣尚早,還趕得出城去,我就去了。明兒姑奶奶好了,再請還願去。”鳳姐因被衆冤魂纏繞害怕,巴不得他就去,便說:“你若肯替我用心,我能安穩睡一覺,我就感激你了。你外孫女兒叫他在這裏住下罷。”劉姥姥道:“莊家孩子沒有見過世面,沒的在這裏打嘴。我帶他去的好。”鳳姐道:“這就是多心了。既是咱們一家,這怕什麼。雖說我們窮了,這一個人喫飯也不礙什麼。”劉姥姥見鳳姐真情,落得叫青兒住幾天,又省了家裏的嚼喫。只怕青兒不肯,不如叫他來問問,若是他肯,就留下。於是和青兒說了幾句。青兒因與巧姐兒頑得熟了,巧姐又不願他去,青兒又願意在這裏。劉姥姥便吩咐了幾句,辭了平兒,忙忙的趕出城去。不題。
\end{parag}


\begin{parag}
    且說櫳翠庵原是賈府的地址,因蓋省親園子,將那庵圈在裏頭,向來食用香火併不動賈府的錢糧。今日妙玉被劫,那女尼呈報到官,一則候官府緝盜的下落,二則是妙玉基業不便離散,依舊住下。不過回明瞭賈府。那時賈府的人雖都知道,只爲賈政新喪,且又心事不寧,也不敢將這些沒要緊的事回稟。只有惜春知道此事,日夜不安。漸漸傳到寶玉耳邊,說妙玉被賊劫去,又有的說妙玉凡心動了跟人而走。寶玉聽得十分納悶,想來必是被強徒搶去,這個人必不肯受,一定不屈而死。但是一無下落,心下甚不放心,每日長噓短嘆。還說:“這樣一個人自稱爲‘檻外人’,怎麼遭此結局!”又想到:“當日園中何等熱鬧,自從二姐姐出閣以來,死的死,嫁的嫁,我想他一塵不染是保得住的了,豈知風波頓起,比林妹妹死的更奇!”由是一而二,二而三,追思起來,想到《莊子》上的話,虛無縹緲,人生在世,難免風流雲散,不禁的大哭起來。襲人等又道是他的瘋病發作,百般的溫柔解勸。寶釵初時不知何故,也用話箴規。怎奈寶玉抑鬱不解,又覺精神恍惚。寶釵想不出道理,再三打聽,方知妙玉被劫不知去向,也是傷感,只爲寶玉愁煩,便用正言解釋。因提起“蘭兒自送殯回來,雖不上學,聞得日夜攻苦。他是老太太的重孫,老太太素來望你成人,老爺爲你日夜焦心,你爲閒情癡意糟蹋自己,我們守着你如何是個結果!”說得寶玉無言可答,過了一回才說道:“我那管人家的閒事,只可嘆咱們家的運氣衰頹。”寶釵道:“可又來,老爺太太原爲是要你成人,接續祖宗遺緒。你只是執迷不悟,如何是好。”寶玉聽來,話不投機,便靠在桌上睡去。寶釵也不理他,叫麝月等伺候着,自己卻去睡了。
\end{parag}


\begin{parag}
    寶玉見屋裏人少,想起:“紫鵑到了這裏,我從沒合他說句知心的話兒,冷冷清清撂着他,我心裏甚不過意。他呢,又比不得麝月秋紋,我可以安放得的。想起從前我病的時候,他在我這裏伴了好些時,如今他的那一面小鏡子還在我這裏,他的情義卻也不薄了。如今不知爲什麼,見我就是冷冷的。若說爲我們這一個呢,他是和林妹妹最好的,我看他待紫鵑也不錯。我有不在家的日子,紫鵑原與他有說有講的,到我來了,紫鵑便走開了。想來自然是爲林妹妹死了我便成了家的原故。噯,紫鵑,紫鵑,你這樣一個聰明女孩兒,難道連我這點子苦處都看不出來麼!”因又一想:“今晚他們睡的睡,做活的做活,不如趁着這個空兒我找他去,看他有什麼話。倘或我還有得罪之處,便陪個不是也使得。”想定主意,輕輕的走出了房門,來找紫鵑。
\end{parag}


\begin{parag}
    那紫鵑的下房也就在西廂裏間。寶玉悄悄的走到窗下,只見裏面尚有燈光,便用舌頭舔破窗紙往裏一瞧,見紫鵑獨自挑燈,又不是做什麼,呆呆的坐着。寶玉便輕輕的叫道:“紫鵑姐姐還沒有睡麼?”紫鵑聽了唬了一跳,怔怔的半日才說:“是誰?”寶玉道:“是我。”紫鵑聽着,似乎是寶玉的聲音,便問:“是寶二爺麼?”寶玉在外輕輕的答應了一聲。紫鵑問道:“你來做什麼?”寶玉道:“我有一句心裏的話要和你說說,你開了門,我到你屋裏坐坐。”紫鵑停了一會兒說道:“二爺有什麼話,天晚了,請回罷,明日再說罷。”寶玉聽了,寒了半截。自己還要進去,恐紫鵑未必開門,欲要回去,這一肚子的隱情,越發被紫鵑這一句話勾起。無奈,說道:“我也沒有多餘的話,只問你一句。”紫鵑道:“既是一句,就請說。”寶玉半日反不言語。紫鵑在屋裏不見寶玉言語,知他素有癡病,恐怕一時實在搶白了他,勾起他的舊病倒也不好了,因站起來細聽了一聽,又問道:“是走了,還是傻站着呢?有什麼又不說,盡着在這裏慪人。已經慪死了一個,難道還要慪死一個麼!這是何苦來呢!”說着,也從寶玉舔破之處往外一張,見寶玉在那裏呆聽。紫鵑不便再說,回身剪了剪燭花。忽聽寶玉嘆了一聲道:“紫鵑姐姐,你從來不是這樣鐵心石腸,怎麼近來連一句好好兒的話都不和我說了?我固然是個濁物,不配你們理我,但只我有什麼不是,只望姐姐說明了,那怕姐姐一輩子不理我,我死了倒作個明白鬼呀!”紫鵑聽了,冷笑道:“二爺就是這個話呀,還有什麼?若就是這個話呢,我們姑娘在時我也跟着聽俗了!若是我們有什麼不好處呢,我是太太派來的,二爺倒是回太太去,左右我們丫頭們更算不得什麼了。”說到這裏,那聲兒便哽咽起來,說着又醒鼻涕,寶玉在外知他傷心哭了,便急的跺腳道:“這是怎麼說,我的事情你在這裏幾個月還有什麼不知道的。就便別人不肯替我告訴你,難道你還不叫我說,叫我憋死了不成!”說着,也嗚咽起來了。
\end{parag}


\begin{parag}
    寶玉正在這裏傷心,忽聽背後一個人接言道:“你叫誰替你說呢?誰是誰的什麼?自己得罪了人自己央及呀,人家賞臉不賞在人家,何苦來拿我們這些沒要緊的墊喘兒呢。”這一句話把裏外兩個人都嚇了一跳。你道是誰,原來卻是麝月。寶玉自覺臉上沒趣。只見麝月又說道:“到底是怎麼着?一個陪不是,一個人又不理。你倒是快快的央及呀。噯,我們紫鵑姐姐也就太狠心了,外頭這麼怪冷的,人家央及了這半天,總連個活動氣兒也沒有。”又向寶玉道:“剛纔二奶奶說了,多早晚了,打量你在那裏呢,你卻一個人站在這房檐底下做什麼!”紫鵑裏面接着說道:“這可是什麼意思呢?早就請二爺進去,有話明日說罷。這是何苦來!”寶玉還要說話,因見麝月在那裏,不好再說別的,只得一面同麝月走回,一面說道:“罷了,罷了!我今生今世也難剖白這個心了!惟有老天知道罷了!”說到這裏,那眼淚也不知從何處來的,滔滔不斷了。麝月道:“二爺,依我勸你死了心罷,白陪眼淚也可惜了兒的。”寶玉也不答言,遂進了屋子。只見寶釵睡了,寶玉也知寶釵裝睡。卻是襲人說了一句道:“有什麼話明日說不得,巴巴兒的跑那裏去鬧,鬧出——”說到這裏也就不肯說,遲了一遲才接着道:“身上不覺怎麼樣?”寶玉也不言語,只搖搖頭兒,襲人一面纔打發睡下。一夜無眠,自不必說。
\end{parag}


\begin{parag}
    這裏紫鵑被寶玉一招,越發心裏難受,直直的哭了一夜。思前想後,“寶玉的事,明知他病中不能明白,所以衆人弄鬼弄神的辦成了。後來寶玉明白了,舊病復發,常時哭想,並非忘情負義之徒。今日這種柔情,一發叫人難受,只可憐我們林姑娘真真是無福消受他。如此看來,人生緣分都有一定,在那未到頭時,大家都是癡心妄想。乃至無可如何,那糊塗的也就不理會了,那情深義重的也不過臨風對月,灑淚悲啼。可憐那死的倒未必知道,這活的真真是苦惱傷心,無休無了。算來竟不如草木石頭,無知無覺,倒也心中乾淨!”想到此處,倒把一片酸熱之心一時冰冷了。纔要收拾睡時,只聽東院裏吵嚷起來。未知何事,下回分解。
\end{parag}