\chap{七}{送宮花周瑞嘆英蓮 談肄業秦鍾結寶玉}


\begin{parag}
    \begin{note}蒙:苦盡甘來遞轉,正強忽弱誰明?惺惺自古惜惺惺,時運文章操勁。無縫機關難見,多少筆墨偏精。有情情處特無情,何是人人不醒?\end{note}
\end{parag}


\begin{parag}
    \begin{note}靖:他小說中一筆作兩三筆者、一事啓兩事者均曾見之。豈有似“送花”一回間三帶四攢花簇錦之文哉?\end{note}
\end{parag}


\begin{parag}
    題曰:
\end{parag}


\begin{poem}
    \begin{pl}十二花容色最新,不知誰是惜花人?\end{pl}

    \begin{pl}相逢若問名何氏,家住江南本姓秦。\end{pl}
\end{poem}


\begin{parag}
    話說周瑞家的送了劉姥姥去後,便上來回王夫人話。\begin{note}甲側:不回鳳姐,卻回王夫人,不交代處,正交代得清楚。\end{note}誰知王夫人不在上房,問丫鬟們時,方知往薛姨媽那邊閒話去了。\begin{note}甲側:文章只是隨筆寫來,便有流離生動之妙。\end{note}周瑞家的聽說,便轉出東角門至東院,往梨香院來。剛至院門前,只見王夫人的丫鬟名金釧兒\begin{note}甲側:金釧、寶釵互相映射。妙!\end{note}者,和一個才留了頭的小女孩兒\begin{note}甲側:蓮卿別來無恙否?\end{note}站在臺階坡上頑。見周瑞家的來了,便知有話回,因向內努嘴兒。\begin{note}甲側:畫。\end{note}周瑞家的輕輕掀簾進去,只見王夫人和薛姨媽長篇大套的說些家務人情等語。
\end{parag}


\begin{parag}
    周瑞家的不敢驚動,遂進裏間來。\begin{note}甲雙夾:總用雙歧岔路之筆,令人估料不到之文。\end{note}只見薛寶釵\begin{note}甲側:自入梨香院,至此方寫。\end{note}穿著家常衣服,\begin{note}甲雙夾:好!寫一人換一副筆墨,另出一花樣。甲眉:“家常愛著舊衣裳”是也。\end{note}頭上只散挽著䰖兒,坐在炕邊裏,伏在小炕桌上同丫鬟鶯兒正描花樣子呢。\begin{note}甲側:一幅《繡窗仕女圖》,虧想得周到。\end{note}見他進來,寶釵才放下筆,轉過身來,滿面堆笑讓:“周姐姐坐。”周瑞家的也忙陪笑問:“姑娘好?”一面炕沿上坐了,因說:“這有兩三天也沒見姑娘到那邊逛逛去,只怕是你寶兄弟衝撞了你不成?”\begin{note}甲側:一人不漏,一筆不板。\end{note}寶釵笑道:“那裏的話。只因我那種病又發了,\begin{note}甲眉:“那種病”“那”字,與前二玉“不知因何”二“又”字,皆得天成地設之體;且省卻多少閒文,所謂“惜墨如金”是也。\end{note}所以這兩天沒出屋子。”\begin{note}甲側:得空便入。\end{note}周瑞家的道:“正是呢,姑娘到底有什麼病根兒,也該趁早兒請個大夫來,好生開個方子,認真喫幾劑,一勢兒除了根纔是。小小的年紀倒作下個病根兒,也不是頑的。”寶釵聽了便笑道:“再不要提吃藥,爲這病請大夫吃藥,也不知白花了多少銀子錢呢。憑你什麼名醫仙藥,從不見一點兒效。後來還虧了一個禿頭和尚,\begin{note}甲側:奇奇怪怪,真雲龍作雨,忽隱忽見,使人逆料不到。\end{note}說專治無名之症,因請他看了。他說我這是從胎裏帶來的一股熱毒,\begin{note}甲側:凡心偶熾,是以孽火齊攻。\end{note}幸而先天壯,還不相干。\begin{note}甲側:渾厚故也,假使顰、鳳輩,不知又何如治之。\end{note}若喫尋常藥,是不中用的。他就說了一個海上方,又給了一包藥末子作引子,異香異氣的。不知是那裏弄了來的。他說發了時喫一丸就好。倒也奇怪,喫他的藥倒效驗些。”\begin{note}甲雙夾:卿不知從那裏弄來,餘則深知是從放春山採來,以灌愁海水和成,煩廣寒玉兔搗碎,在太虛幻境空靈殿上炮製配合者也。\end{note}
\end{parag}


\begin{parag}
    周瑞家的因問:“不知是個什麼海上方兒?姑娘說了,我們也記著,說與人知道,倘遇見這樣病,也是行好的事。”寶釵見問,乃笑道:“不用這方兒還好,若用了這方兒,真真把人瑣碎死。東西藥料一概都有限,只難得‘可巧’二字:要春天開的白牡丹花蕊十二兩,\begin{note}甲側:凡用“十二”字樣,皆照應十二釵。\end{note}夏天開的白荷花蕊十二兩,秋天的白芙蓉蕊十二兩,冬天的白梅花蕊十二兩。將這四樣花蕊,於次年春分這日曬幹,和在藥末子一處,一齊研好。又要雨水這日的雨水十二錢,……”周瑞家的忙道:“噯喲!這麼說來,這就得三年的工夫。倘或雨水這日竟不下雨,這卻怎處呢?”寶釵笑道:“所以說那裏有這樣可巧的雨,便沒雨也只好再等罷了。白露這日的露水十二錢,霜降這日的霜十二錢,小雪這日的雪十二錢。把這四樣水調勻,和了藥,再加十二錢蜂蜜,十二錢白糖,丸了龍眼大的丸子,盛在舊磁壇內,埋在花根底下。若發了病時,拿出來喫一丸,用十二分黃柏煎湯送下。”\begin{note}甲雙夾:末用黃柏更妙。可知“甘苦”二字,不獨十二釵,世皆同有者。\end{note}
\end{parag}


\begin{parag}
    周瑞家的聽了笑道:“阿彌陀佛,真坑死人的事兒!等十年未必都這樣巧的呢。”寶釵道:“竟好,自他說了去後,一二年間可巧都得了,好容易配成一料。如今從南帶至北,現在就埋在梨花樹底下呢。”\begin{note}甲側:“梨香”二字有著落,並未白白虛設。\end{note}周瑞家的又問道:“這藥可有名子沒有呢?”寶釵道:“有。\begin{note}甲側:一字句。\end{note}這也是那癩頭和尚說下的。叫作‘冷香丸’。”\begin{note}甲側:新雅奇甚。\end{note}周瑞家的聽了點頭兒,因又說:“這病發了時到底覺怎麼著?”寶釵道:“也不覺甚怎麼著,只不過喘嗽些,喫一丸下去也就好些了。”\begin{note}甲雙夾:以花爲藥,可是吃煙火人想得出者?諸公且不必問其事之有無,只據此新奇妙文悅我等心目,便當浮一大白。\end{note}
\end{parag}


\begin{parag}
    周瑞家的還欲說話時,忽聽王夫人問:“誰在房裏呢?”周瑞家的忙出去答應了,趁便回了劉姥姥之事。略待半刻,見王夫人無語,方欲退出,\begin{note}甲雙夾:行文原只在一二字,便有許多省力處。不得此竅者,便在窗下百般扭捏。\end{note}薛姨媽忽又笑道:\begin{note}甲雙夾:“忽”字“又”字與“方欲”二字對射。\end{note}“你且站住。我有一宗東西,你帶了去罷。”說著便叫香菱。\begin{note}甲雙夾:二字仍從“蓮”上起來。蓋“英蓮”者,“應憐”也,“香菱”者亦“相憐”之意。此是改名之“英蓮”也。\end{note}只聽簾櫳響處,方纔和金釧頑的那個小丫頭進來了,問:“奶奶叫我作什麼?”\begin{note}甲雙夾:這是英蓮天生成的口氣,妙甚!\end{note}薛姨媽道:“把匣子裏的花兒拿來。”香菱答應了,向那邊捧了個小錦匣來。薛姨媽道:“這是宮裏頭的新鮮樣法,拿紗堆的花兒十二支。昨兒我想起來,白放著可惜了兒的,何不給他們姊妹們戴去。昨兒要送去,偏又忘了。你今兒來的巧,就帶了去罷。你家的三位姑娘,每人一對,剩下的六枝,送林姑娘兩枝,那四枝給了鳳哥罷。”\begin{note}甲側:妙文!今古小說中可有如此口吻者?\end{note}王夫人道:“留著給寶丫頭戴罷了,又想著他們。”薛姨媽道:“姨娘不知道,寶丫頭古怪\begin{note}甲側:“古怪”二字,正是寶卿身份。\end{note}著呢,他從來不愛這些花兒粉兒的。”\begin{note}甲雙夾:可知周瑞一回,正爲寶菱二人所有,正《石頭記》得力處也。\end{note}
\end{parag}


\begin{parag}
    說著,周瑞家的拿了匣子,走出房門,見金釧仍在那裏曬日陽兒。周瑞家的因問他道:“那香菱小丫頭子,可就是常說臨上京時買的,爲他打人命官司的那個小丫頭子麼?”金釧道:“可不就是。”\begin{note}甲側:出明英蓮。\end{note}正說著,只見香菱笑嘻嘻的走來。周瑞家的便拉了他的手,細細的看了一會,因向金釧兒笑道:“倒好個模樣兒,竟有些象咱們東府裏蓉大奶奶的品格兒。”\begin{note}甲雙夾:一擊兩鳴法,二人之美,並可知矣。再忽然想到秦可卿,何玄幻之極。假使說像榮府中所有之人,則死板之至,故遠遠以可卿之貌爲譬,似極扯淡,然卻是天下必有之情事。\end{note}金釧兒笑道:“我也是這們說呢。”周瑞家的又問香菱:“你幾歲投身到這裏?”又問:“你父母今在何處?今年十幾歲了?本處是那裏人?”香菱聽問,都搖頭說:“不記得了。”\begin{note}甲雙夾:傷痛之極,亦必如此收住方妙。不然,則又將作出香菱思鄉一段文字矣。\end{note}周瑞家的和金釧兒聽了,倒反爲嘆息傷感一回。
\end{parag}


\begin{parag}
    一時間周瑞家的攜花至王夫人正房後頭來。原來近日賈母說孫女兒們太多了,一處擠著倒不方便,只留寶玉、黛玉二人這邊解悶,卻將迎、探、惜三人移到王夫人這邊房後三間小抱廈內居住,令李紈陪伴照管。\begin{note}甲側:不作一筆安逸之筆矣。\end{note}如今周瑞家的故順路先往這裏來,只見幾個小丫頭子都在抱廈內聽呼喚呢。迎春的丫鬟司棋與探春的丫鬟侍書\begin{note}甲雙夾:妙名。賈家四釵之鬟,暗以琴、棋、書、畫四字列名,省力之甚,醒目之甚,卻是俗中不俗處。\end{note}二人正掀簾子出來,手裏都捧著茶鍾,周瑞家的便知他們姊妹在一處坐著呢,遂進入內房,只見迎春探春二人正在窗下圍棋。周瑞家的將花送上,說明緣故。二人忙住了棋,都欠身道謝,命丫鬟們收了。
\end{parag}


\begin{parag}
    周瑞家的答應了,因說:“四姑娘不在房裏?只怕在老太太那邊呢。”丫鬟們道:“在這屋裏不是?”\begin{note}甲雙夾:用畫家三五聚散法寫來,方不死板。\end{note}周瑞家的聽了,便往這邊屋裏來。只見惜春正同水月庵\begin{note}[即饅頭庵。]\end{note}的小姑子智能兒一處頑笑,\begin{note}甲雙夾:總是得空便入。百忙中又帶出王夫人喜施捨等事,可知一支筆作千百支用。又伏後文。甲眉:閒閒一筆,卻將後半部線索提動。\end{note}見周瑞家的進來,惜春便問他何事。周瑞家的便將花匣打開,說明原故。惜春笑道:“我這裏正和智能兒說,我明兒也剃了頭同他作姑子去呢,可巧又送了花兒來,若剃了頭,可把這花兒戴在那裏呢?”說著,大家取笑一回,惜春命丫鬟入畫\begin{note}甲側:曰司棋,曰侍書,曰入畫;後文補抱琴。琴、棋、書、畫四字最俗,上添一虛字則覺新雅。\end{note}來收了。
\end{parag}


\begin{parag}
    周瑞家的因問智能兒:“你是什麼時候來的?你師父那禿歪剌往那裏去了?”智能兒道:“我們一早就來了,我師父見了太太,就往於老爺府內去了,叫我在這裏等他呢。”\begin{note}甲雙夾:又虛貼一個於老爺,可知尚僧尼者,悉愚人也。\end{note}周瑞家的又道:“十五的月例香供銀子可曾得了沒有?”智能兒搖頭兒說:“我不知道。”\begin{note}甲雙夾:妙!年輕未任事也。一應騙佈施、哄齋供諸惡,皆是老禿賊設局。寫一種人,一種人活像。\end{note}惜春聽了,便問周瑞家的:“如今各廟月例銀子是誰管著?”周瑞家的道:“是餘信\begin{note}甲側:明點“愚信”二字。\end{note}管著。”惜春聽了笑道:“這就是了。他師父一來,餘信家的就趕上來,和他師父咕唧了半日,想是就爲這事了。”\begin{note}甲雙夾:一人不落,一事不忽,伏下多少後文,豈真爲送花哉!\end{note}
\end{parag}


\begin{parag}
    那周瑞家的又和智能兒勞叨了一會,便往鳳姐兒處來。穿夾道從李紈後窗下過,\begin{note}甲雙夾:細極!李紈雖無花,豈可失而不寫者?故用此順筆便墨,間三帶四,使觀者不忽。\end{note}越過西花牆,出西角門進入鳳姐院中。走至堂屋,只見小丫頭豐兒坐在鳳姐房中門檻上,見周瑞家的來了,連忙\begin{note}甲側:二字著緊。\end{note}擺手兒叫他往東屋裏去。周瑞家的會意,忙躡手躡足往東邊房裏來,只見奶子正拍著大姐兒睡覺呢。\begin{note}甲側:總不重犯,寫一次有一次的新樣文法。\end{note}周瑞家的悄問奶子道:“奶奶睡中覺呢?也該請醒了。”奶子搖頭兒。\begin{note}甲側:有神理。\end{note}正說著,只聽那邊一陣笑聲,卻有賈璉的聲音。接著房門響處,平兒拿著大銅盆出來,叫豐兒舀水進去。\begin{note}甲雙夾:妙文奇想!阿鳳之爲人,豈有不著意於“風月”二字之理哉?若直以明筆寫之,不但唐突阿鳳身價,亦且無妙文可賞。若不寫之,又萬萬不可。故只用 “柳藏鸚鵡語方知”之法,略一皴染,不獨文字有隱微,亦且不至污瀆阿鳳之英風俊骨。所謂此書無一不妙。甲眉:餘素所藏仇十洲《幽窗聽鶯暗春圖》,其心思筆墨,已是無雙,今見此阿鳳一傳,則覺畫工太板。\end{note}平兒便到這邊來,一見了周瑞家的便問:“你老人家又跑了來作什麼?”周瑞家的忙起身,拿匣子與他,說送花兒一事。平兒聽了,便打開匣子,拿了四枝,轉身去了。半刻工夫,手裏拿出兩枝來,\begin{note}甲側:攢花簇錦之文,故使人耳目眩亂。\end{note}先叫彩明吩咐道:“送到那邊府裏給小蓉大奶奶戴去。”\begin{note}甲側:忙中更忙,又曰“密處不容針”,此等處是也。\end{note}次後方命周瑞家的回去道謝。
\end{parag}


\begin{parag}
    周瑞家的這才往賈母這邊來。穿過了穿堂,抬頭忽見他女兒打扮著才從他婆家來。周瑞家的忙問:“你這會跑來作什麼?”他女兒笑道:“媽一向身上好?我在家裏等了這半日,媽竟不出去,什麼事情這樣忙的不回家?我等煩了,自己先到了老太太跟前請了安了,這會子請太太的安去。媽還有什麼不了的差事,手裏是什麼東西?”周瑞家的笑道:“噯!今兒偏偏的來了個劉姥姥,我自己多事,爲他跑了半日,這會子又被姨太太看見了,送這幾枝花兒與姑娘奶奶們。這會子還沒送清楚呢。你這會子跑了來,一定有什麼事。”他女兒笑道:“你老人家倒會猜。實對你老人家說,你女婿前兒因多吃了兩杯酒,和人分爭,不知怎的被人放了一把邪火,說他來歷不明,告到衙門裏,要遞解還鄉。所以我來和你老人家商議商議,這個情分,求那一個可了事呢?”周瑞家的聽了道:“我就知道呢。這有什麼大不了的!你且家去等我,我給林姑娘送了花兒去就回家去。此時太太二奶奶都不得閒兒,你回去等我。這有什麼,忙的如此。”女兒聽說,便回去了,又說:“媽,好歹快來。”周瑞家的道:“是了。小人兒家沒經過什麼事,就急得你這樣了。”說著。便到黛玉房中去了。\begin{note}甲雙夾:又生出一小段來,是榮、寧中常事,亦是阿鳳正文,若不如此穿插,直用一送花到底,亦太死板,不是《石頭記》筆墨矣。\end{note}
\end{parag}


\begin{parag}
    誰知此時黛玉不在自己房中,卻在寶玉房中大家解九連環頑呢。\begin{note}甲側:妙極!又一花樣。此時二玉已隔房矣。\end{note}周瑞家的進來笑道:“林姑娘,姨太太著我送花兒與姑娘帶。”寶玉聽說,便先問:“什麼花兒?拿來給我。”一面早伸手接過來了。\begin{note}甲側:瞧他夾寫寶玉。\end{note}開匣看時,原來是宮制堆紗新巧的假花兒。\begin{note}甲側:此處方一細寫花形。\end{note}黛玉只就寶玉手中看了一看,\begin{note}甲側:妙!看他寫黛玉。\end{note}便問道:“還是單送我一人的,還是別的姑娘們都有呢?”\begin{note}甲雙夾:在黛玉心中,不知有何丘壑。\end{note}周瑞家的道:“各位都有了,這兩枝是姑娘的了。”黛玉冷笑道:“我就知道,別人不挑剩下的也不給我。”\begin{note}甲側:吾實不知黛卿胸中有何丘壑,在“看一看”上傳神。\end{note}周瑞家的聽了,一聲兒不言語。\begin{note}甲眉:餘閱送花一回,薛姨媽雲“寶丫頭不喜這些花兒粉兒的”,則謂是寶釵正傳。又出阿鳳、惜春一段,則又知是阿鳳正傳。今又到顰兒一段,卻又將阿顰之天性,從骨中一寫,方知亦系顰兒正傳。小說中一筆作兩三筆者有之,一事啓兩事者有之,未有如此恆河沙數之筆也。\end{note}寶玉便問道:“周姐姐,你作什麼到那邊去了。”周瑞家的因說:“太太在那裏,因回話去了,姨太太就順便叫我帶來了。”寶玉道:“寶姐姐在家作什麼呢?怎麼這幾日也不過這邊來?”周瑞家的道:“身上不大好呢。”寶玉聽了,便和丫頭說:“誰去瞧瞧?只說我和林姑娘\begin{note}甲側:“和林姑娘”四字著眼。\end{note}打發了來請姨太太姐姐安,問姐姐是什麼病,現喫什麼藥。論理我該親自來的,就說才從學裏來,也著了些涼,異日再親自來看罷。”\begin{note}甲眉:餘觀“才從學裏來”幾句,忽追思昔日情景,可嘆!想紈絝小兒,自開口雲“學裏”,亦如市俗人開口便雲“有些小事”,然何嘗真有事哉!此掩飾推托之詞耳。寶玉若不雲“從學房裏來涼著”,然則便雲“因憨頑時涼著”者哉?寫來一笑,繼之一嘆。\end{note}說著,茜雪便答應去了。周瑞家的自去,無話。
\end{parag}


\begin{parag}
    原來這周瑞的女婿,便是雨村的好友冷子興,\begin{note}甲側:著眼。\end{note}近因賣古董和人打官司,故教女人來討情分。周瑞家的仗著主子的勢利,把這些事也不放在心上,晚間只求求鳳姐兒便完了。
\end{parag}


\begin{parag}
    至掌燈時分,鳳姐已卸了妝,來見王夫人回話:“今兒甄家\begin{note}甲側:又提甄家。\end{note}送了來的東西,我已收了。\begin{note}甲側:不必細說方妙。\end{note}咱們送他的,趁著他家有年下進鮮的船回去,一併都交給他們帶了去罷?”王夫人點頭。鳳姐又道:“臨安伯老太太生日的禮已經打點了,派誰送去呢?”\begin{note}甲側:阿鳳一生尖處。\end{note}王夫人道:“你瞧誰閒著,就叫他們去四個女人就是了,又來當什麼正經事問我。”\begin{note}甲雙夾:虛描二事,真真千頭萬緒,紙上雖一回兩回中或有不能寫到阿鳳之事,然亦有阿鳳在彼處手忙心忙矣,觀此回可知。\end{note}鳳姐又笑道:“今日珍大嫂子來,請我明日過去逛逛,明日倒沒有什麼事情。”王夫人道:“有事沒事都害不著什麼。每常他來請,有我們,你自然不便意,他既不請我們,單請你,可知是他誠心叫你散淡散淡,別辜負了他的心,便有事也該過去纔是。”鳳姐答應了。當下李紈、迎、探等姐妹們亦來定省畢,各自歸房無話。
\end{parag}


\begin{parag}
    次日鳳姐梳洗了,先回王夫人畢,方來辭賈母。寶玉聽了,也要跟了逛去。鳳姐只得答應,立等著換了衣服,姐兒兩個坐了車,一時進入寧府。早有賈珍之妻尤氏與賈蓉之妻秦氏婆媳兩個,引了多少姬妾丫鬟媳婦等接出儀門。那尤氏一見了鳳姐,必先笑嘲一陣,一手攜了寶玉同入上房來歸坐。秦氏獻茶畢,鳳姐因說:“你們請我來作什麼?有什麼好東西孝敬我,就快獻上來,我還有事呢。”尤氏秦氏未及答話,地下幾個姬妾先就笑說:“二奶奶今兒不來就罷,既來了就依不得二奶奶了。”正說著,只見賈蓉進來請安。寶玉因問:“大哥哥今日不在家麼?”尤氏道:“出城與老爺請安去了。可是你怪悶的,坐在這裏作什麼?何不也去逛逛?”
\end{parag}


\begin{parag}
    秦氏笑道:“今兒巧,上回寶叔立刻要見的我那兄弟,他今兒也在這裏,\begin{note}甲眉:欲出鯨卿,卻先小妯娌閒閒一聚,隨筆帶出,不見一絲作造。\end{note}想在書房裏呢,寶叔何不去瞧一瞧?”寶玉聽了,即便下炕要走。尤氏、鳳姐都忙說:“好生著,忙什麼?”一面便吩咐,“好生小心跟著,別委屈著他,倒比不得跟了老太太過來就罷了。”\begin{note}甲雙夾:“委屈”二字極不通,卻是至情,寫愚婦至矣!\end{note}鳳姐說道:“既這麼著,何不請進這秦小爺來,我也瞧一瞧。難道我見不得他不成?”尤氏笑道:“罷,罷!可以不必見他,比不得咱們家的孩子們,胡打海摔的慣了。\begin{note}甲雙夾:卿家“胡打海摔”,不知誰家方珍憐珠惜?此極相矛盾卻極入情,蓋大家婦人口吻如此。\end{note}人家的孩子都是斯斯文文的慣了,乍見了你這破落戶,還被人笑話死了呢。”鳳姐笑\begin{note}甲側:自負得起。\end{note}道:“普天下的人,我不笑話就罷了,竟叫這小孩子笑話我不成?”賈蓉笑道:“不是這話,他生的靦腆,沒見過大陣仗兒,嬸子見了,沒的生氣。”鳳姐啐道:“他是哪吒,我也要見一見!別放你孃的屁了。再不帶我看看,給你一頓好嘴巴。”賈蓉笑嘻嘻的說:“我不敢扭著,就帶他來。”\begin{note}甲眉:此等處寫阿鳳之放縱,是爲後回伏線。\end{note}
\end{parag}


\begin{parag}
    說著,果然出去帶進一個小後生來,較寶玉略瘦些,清眉秀目,粉面朱脣,身材俊俏,舉止風流,似在寶玉之上,只是羞羞怯怯,有女兒之態,靦腆含糊,慢向鳳姐作揖問好。鳳姐喜的先推寶玉,笑道:“比下去了!”\begin{note}甲側:不知從何處想來。\end{note}便探身一把攜了這孩子的手,就命他身傍坐了,慢慢的問他年紀讀書等事,\begin{note}甲側:分明寫寶玉,卻先偏寫阿鳳。\end{note}方知他學名喚秦鍾。\begin{note}甲雙夾:設雲“情鍾”。古詩云:“未嫁先名玉,來時本姓秦。”二語便是此書大綱目、大比託、大諷刺處。\end{note}早有鳳姐的丫鬟媳婦們見鳳姐初會秦鍾,並未備得表禮來,遂忙過那邊去告訴平兒。平兒知道鳳姐與秦氏厚密,雖是小後生家,亦不可太儉,遂自作主意,拿了一匹尺頭,兩個“狀元及第”的小金錁(kè)子,交付與來人送過去。鳳姐猶笑說太簡薄等語。秦氏等謝畢。一時喫過飯,尤氏、鳳姐、秦氏等抹骨牌,不在話下。\begin{note}甲雙夾:一人不落,又帶出強將手下無弱兵。\end{note}
\end{parag}


\begin{parag}
    寶玉秦鍾二人隨便起坐說話。\begin{note}甲側:淡淡寫來。\end{note}那寶玉只一見了秦鐘的人品出衆,心中便有所失,癡了半日,自己心中又起了呆意,乃自思道:“天下竟有這等人物!如今看來,我竟成了泥豬癩狗了。可恨我爲什麼生在這侯門公府之家,若也生在寒門薄宦之家,早得與他交結,也不枉生了一世。我雖如此比他尊貴,\begin{note}甲雙夾:這一句不是寶玉本意中語,卻是古今歷來膏粱紈絝之意。\end{note}可知錦繡紗羅,也不過裹了我這根死木頭;美酒羊羔,也不過填了我這糞窟泥溝。‘富貴’二字,不料遭我荼毒了!”\begin{note}甲雙夾:一段癡情,翻“賢賢易色”一句筋斗,使此後朋友中無復再敢假談道義,虛論情常。蒙側:此是作者一大發泄處。\end{note}秦鍾自見了寶玉形容出衆,舉止不浮,\begin{note}甲雙夾:“不浮”二字妙,秦卿目中所取正在此。\end{note}更兼金冠繡服,驕婢侈童,\begin{note}甲雙夾:這二句是貶,不是獎。此八字遮飾過多少魑魅紈綺秦卿目中所鄙者。\end{note}秦鍾心中亦自思道:“果然這寶玉怨不得人溺愛他。可恨我偏生於清寒之家,不能與他耳鬢交接,可知‘貧富’二字限人,亦世間之大不快事。”\begin{note}甲雙夾:“貧富”二字中,失卻多少英雄朋友!蒙側:總是作者大發泄處,藉此以伸多少不樂。\end{note}二人一樣的胡思亂想。\begin{note}甲雙夾:作者又欲瞞過衆人。\end{note}忽又\begin{note}甲雙夾:二字寫小兒得神。\end{note}寶玉問他讀什麼書。\begin{note}甲雙夾:寶玉問讀書,亦想不到之大奇事。\end{note}秦鍾見問,便因實而答。\begin{note}甲雙夾:四字普天下朋友來看。\end{note}二人你言我語,十來句後,越覺親密起來。
\end{parag}


\begin{parag}
    一時擺上茶果,寶玉便說:“我兩個又不喫酒,把果子擺在裏間小炕上,我們那裏坐去,省得鬧你們。”\begin{note}甲雙夾:眼見得二人一身一體矣。\end{note}於是二人進裏間來喫茶。秦氏一面張羅與鳳姐擺酒果,一面忙進來囑寶玉道:“寶叔,你侄兒倘或言語不防頭,你千萬看著我,不要理他。他雖靦腆,卻性子左強,不大隨和些是有的。”\begin{note}甲側:實寫秦鍾,又映寶玉。\end{note}寶玉笑道:“你去罷,我知道了。”秦氏又囑了他兄弟一回,方去陪鳳姐。
\end{parag}


\begin{parag}
    一時鳳姐尤氏又打發人來問寶玉:“要喫什麼,外面有,只管要去。”寶玉只答應著,也無心在飲食上,只問秦鍾近日家務等事。\begin{note}甲雙夾:寶玉問讀書已奇,今又問家務,豈不更奇?\end{note}秦鍾因說:“業師於去年病故,家父又年紀老邁,殘疾在身,公務繁冗,因此尚未議及再延師一事,目下不過在家溫習舊課而已。再讀書一事,必須有一二知己爲伴,時常大家討論,才能進益。”寶玉不待說完,便答道:“正是呢,我們卻有個家塾,合族中有不能延師的,便可入塾讀書,子弟們中亦有親戚在內可以附讀。我因業師上年回家去了,也現荒廢著呢。家父之意,亦欲暫送我去溫習舊書,待明年業師上來,再各自在家裏讀。家祖母因說:一則家學裏之子弟太多,生恐大家淘氣,反不好,二則也因我病了幾天,遂暫且耽擱著。如此說來,尊翁如今也爲此事懸心。今日回去,何不稟明,就往我們敝塾中來,我亦相伴,彼此有益,豈不是好事?” 秦鍾笑道:\begin{note}甲眉:真是可兒之弟。\end{note}“家父前日在家提起延師一事,也曾提起這裏的義學倒好,原要來和這裏的親翁商議引薦。因這裏又事忙,不便爲這點小事來聒絮的。寶叔果然度小侄或可磨墨滌硯,何不速速的作成,\begin{note}甲眉:真是可卿之弟。\end{note}又彼此不致荒廢,又可以常相談聚,又可以慰父母之心,又可以得朋友之樂,豈不是美事?”寶玉道:“放心,放心。咱們回來告訴你姐夫、姐姐和璉二嫂子。你今日回家就稟明令尊,我回去再稟明祖母,再無不速成之理。”二人計議一定。那天氣已是掌燈時候,出來又看他們頑了一回牌。算帳時,卻又是秦氏、尤氏二人輸了戲酒的東道,\begin{note}甲側:自然是二人輸。\end{note}言定後日喫這東道,一面就叫送飯。
\end{parag}


\begin{parag}
    喫畢晚飯,因天黑了,尤氏說:“先派兩個小子送了這秦相公家去。”媳婦們傳出去半日,秦鍾告辭起身。尤氏問:“派了誰送去?”媳婦們回說:“外頭派了焦大,誰知焦大醉了,又罵呢。”\begin{note}甲雙夾:可見罵非一次矣。\end{note}尤氏、秦氏都說道:“偏又派他作什麼!放著這些小子們,那一個派不得?偏要惹他去。”\begin{note}甲側:便奇。\end{note}鳳姐道:“我成日家說你太軟弱了,縱的家裏人這樣還了得了。”尤氏嘆道:“你難道不知這焦大的?連老爺都不理他的,你珍大哥哥也不理他。只因他從小兒跟著太爺們出過三四回兵,從死人堆裏把太爺背了出來,得了命,自己挨著餓,卻偷了東西來給主子喫。兩日沒得水,得了半碗水給主子喝,他自己喝馬溺。不過仗著這些功勞情分,有祖宗時都另眼相待,如今誰肯難爲他去。他自己又老了,又不顧體面,一味喫酒,喫醉了,無人不罵。我常說給管事的,不要派他差事,全當一個死的就完了。今兒又派了他。”\begin{note}蒙側:有此功勞,實不可輕易摧折,亦當處之道,厚其贍養,尊其等次。送人回家,原非酬功之事。所謂漢之功臣不得保其首領者,我知之矣。\end{note}鳳姐道:“我何曾不知這焦大。倒是你們沒主意,有這樣的,何不打發他遠遠的莊子上去就完了。”\begin{note}甲眉:這是爲後協理寧國伏線。\end{note}說著,因問:“我們的車可齊備了?”地下衆人都應道:“伺候齊了。”
\end{parag}


\begin{parag}
    鳳姐起身告辭,和寶玉攜手同行。尤氏等送至大廳,只見燈燭輝煌,衆小廝在丹墀侍立。那焦大又恃賈珍不在家,即在家亦不好怎樣他,更可以任意灑落灑落。因趁著酒興,先罵大總管賴二,\begin{note}甲雙夾:記清,榮府中則是賴大,又故意綜錯的妙。\end{note}說他不公道,欺軟怕硬:“有了好差事就派別人,象這等黑更半夜送人的事,就派我。沒良心的王八羔子!瞎充管家!你也不想想,焦大太爺蹺蹺腳,比你的頭還高呢。二十年頭裏的焦大太爺眼裏有誰?別說你們這把子的雜種王八羔子們!”
\end{parag}


\begin{parag}
    正罵的興頭上,賈蓉送鳳姐的車出去,衆人喝他不聽,賈蓉忍不得,便罵了他兩句,使人捆起來,“等明日酒醒了,問他還尋死不尋死了!”那焦大那裏把賈蓉放在眼裏,反大叫起來,趕著賈蓉叫:“蓉哥兒,你別在焦大跟前使主子性兒。別說你這樣兒的,就是你爹,你爺爺,也不敢和焦大挺腰子!不是焦大一個人,你們就做官兒,享榮華,受富貴?你祖宗九死一生掙下這家業,到如今了,不報我的恩,反和我充起主子來了。\begin{note}甲側:忽接此焦大一段,真可驚心駭目,一字化一淚,一淚化一血珠。\end{note}不和我說別的還可,若再說別的,咱們紅刀子進去白刀子出來!”\begin{note}甲雙夾:是醉人口中文法。一段借醉奴口角閒閒補出寧榮往事近故,特爲天下世家一笑。\end{note}鳳姐在車上說與賈蓉道:“以後還不早打發了這個沒王法的東西!留在這裏豈不是禍害?倘或親友知道了,豈不笑話咱們這樣的人家,連個王法規矩都沒有。”賈蓉答應“是”。
\end{parag}


\begin{parag}
    衆小廝見他太撒野了,只得上來幾個,揪翻捆倒,拖往馬圈裏去。焦大越發連賈珍都說出來,亂嚷亂叫說:“我要往祠堂裏哭太爺去。那裏承望到如今生下這些畜牲來!每日家偷狗戲雞,爬灰的爬灰,養小叔子的養小叔子,我什麼不知道?咱們‘胳膊折了往袖子裏藏’!”\begin{note}甲眉:“不如意事常八九,可與人言無二三。”以二句批是段,聊慰石兄。\end{note}\begin{note}蒙側;放筆痛罵一回,富貴之家,每罹此禍。\end{note}衆小廝聽他說出這些沒天日的話來,唬的魂飛魄散,也不顧別的了,便把他捆起來,用土和馬糞滿滿的填了他一嘴。
\end{parag}


\begin{parag}
    鳳姐和賈蓉等也遙遙的聞得,便都裝作沒聽見。寶玉在車上見這般醉鬧,倒也有趣,因問鳳姐道:“姐姐,你聽他說‘爬灰的爬灰’,什麼是‘爬灰’?”\begin{note}蒙側:暗伏後來史湘雲之問。\end{note}鳳姐聽了,連忙立眉嗔目斷喝道:“少胡說!那是醉漢嘴裏混唚。你是什麼樣的人,不說沒聽見,還倒細問!等我回去回了太太,仔細捶你不捶你!”唬的寶玉忙央告道:“好姐姐,我再不敢了。”鳳姐亦忙回色哄道:“這纔是呢。等到了家,咱們回了老太太,打發你同秦家侄兒學裏唸書去要緊。”說著,卻自回往榮府而來。正是:
\end{parag}


\begin{poem}
    \begin{pl} 不因俊俏難爲友,正爲風流始讀書。\end{pl}
    \begin{note}甲側:原來不讀書即蠢物矣。\end{note}
\end{poem}
