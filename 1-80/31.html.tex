\chap{三十一}{撕扇子作千金一笑 因麒麟伏白首双星}


\begin{parag}
    \begin{note}庚辰:“撕扇子”是以不情之物供娇嗔不知情时之人一笑,所谓“情不情”。\end{note}
\end{parag}


\begin{parag}
    \begin{note}庚辰:“金玉姻缘”已定,又写一金麒麟,是间色法也。何颦儿为其所感?故颦儿谓“情情”。\end{note}
\end{parag}


\begin{parag}
    话说袭人见了自己吐的鲜血在地,也就冷了半截,想著往日常听人说:“少年吐血,年月不保,纵然命长,终是废人了。”想起此言,不觉将素日想著后来争荣 夸耀之心尽皆灰了,眼中不觉滴下泪来。宝玉见他哭了,也不觉心酸起来,因问道:“你心里觉的怎么样?”袭人勉强笑道:“好好的,觉怎么呢。”宝玉的意思即 刻便要叫人烫黄酒,要山羊血黎洞丸来。袭人拉了他的手,笑道:“你这一闹不打紧,闹起多少人来,倒抱怨我轻狂。分明人不知道,倒闹的人知道了,你也不好, 我也不好。正经明儿你打发小子问问王太医去,弄点子药吃吃就好了。人不知鬼不觉的可不好?”宝玉听了有理,也只得罢了,向案上斟了茶来,给袭人漱了口。袭人知宝玉心内是不安稳的,待要不叫他伏侍,他又必不依;二则定要惊动别人,不如由他去罢:因此只在榻上由宝玉去伏侍。一交五更,宝玉也顾不的梳洗,忙穿衣 出来,将王济仁叫来,亲自确问。王济仁问其原故,不过是伤损,便说了个丸药的名字,怎么服,怎么敷。宝玉记了,回园依方调治。不在话下。
\end{parag}


\begin{parag}
    这日正是端阳佳节,蒲艾簪门,虎符系臂。午间,王夫人治了酒席,请薛家母女等赏午。宝玉见宝钗淡淡的,也不和他说话,自知是昨儿的原故。王夫人见宝玉 没精打彩,也只当是金钏儿昨日之事,他没好意思的,越发不理他。林黛玉见宝玉懒懒的,只当是他因为得罪了宝钗的原故,心中不自在,形容也就懒懒的。凤姐昨 日晚间王夫人就告诉了他宝玉金钏的事,知道王夫人不自在,自己如何敢说笑,也就随著王夫人的气色行事,更觉淡淡的。贾迎春姊妹见众人无意思,也都无意思了。因此,大家坐了一坐就散了。
\end{parag}


\begin{parag}
    林黛玉天性喜散不喜聚。他想的也有个道理,他说,“人有聚就有散,聚时欢喜,到散时岂不清冷?既清冷则生伤感,所以不如倒是不聚的好。比如那花开时令人爱慕,谢时则增惆怅,所以倒是不开的好。”故此人以为喜之时,他反以为悲。那宝玉的情性只愿常聚,生怕一时散了添悲;那花只愿常开,生怕一时谢了没趣; 只到筵散花谢,虽有万种悲伤,也就无可如何了。因此,今日之筵,大家无兴散了,林黛玉倒不觉得,倒是宝玉心中闷闷不乐,回至自己房中长吁短叹。偏生晴雯上来换衣服,不防又把扇子失了手跌在地下,将股子跌折。宝玉因叹道:“蠢才,蠢才!将来怎么样?明日你自己当家立事,难道也是这么顾前不顾后的?”晴雯冷笑 道:“二爷近来气大的很,行动就给脸子瞧。前儿连袭人都打了,今儿又来寻我们的不是。要踢要打凭爷去。就是跌了扇子,也是平常的事。先时连那么样的玻璃缸、玛瑙碗不知弄坏了多少,也没见个大气儿,这会子一把扇子就这么著了。何苦来!要嫌我们就打发我们,再挑好的使。好离好散的,倒不好?”宝玉听了这些话,气的浑身乱战,因说道:“你不用忙,将来有散的日子!”
\end{parag}


\begin{parag}
    袭人在那边早已听见,忙赶过来向宝玉道:“好好的,又怎么了?可是我说的:‘一时我不到,就有事故儿。’”晴雯听了冷笑道:“姐姐既会说,就该早来, 也省了爷生气。自古以来,就是你一个人伏侍爷的,我们原没伏侍过。因为你伏侍的好,昨日才挨窝心脚;我们不会伏侍的,到明儿还不知是个什么罪呢!”袭人听了这话,又是恼,又是愧,待要说几句话,又见宝玉已经气的黄了脸,少不得自己忍了性子,推晴雯道:“好妹妹,你出去逛逛,原是我们的不是。”晴雯听他说 “我们”两个字,自然是他和宝玉了,不觉又添了酸意,冷笑几声,道:“我倒不知道你们是谁,别教我替你们害臊了!便是你们鬼鬼祟祟干的那事儿,也瞒不过我去,那里就称起‘我们’来了。明公正道,连个姑娘还没挣上去呢,也不过和我似的,那里就称上‘我们’了!”袭人羞的脸紫胀起来,想一想,原来是自己把话说 错了。宝玉一面说:“你们气不忿,我明儿偏抬举他。”袭人忙拉了宝玉的手道:“他一个糊涂人,你和他分证什么?况且你素日又是有担待的,比这大的过去了多 少,今儿是怎么了?”晴雯冷笑道:“我原是糊涂人,那里配和我说话呢!”袭人听说道:“姑娘倒是和我拌嘴呢,是和二爷拌嘴呢?要是心里恼我,你只和我说, 不犯著当著二爷吵;要是恼二爷,不该这们吵的万人知道。我才也不过为了事,进来劝开了,大家保重。姑娘倒寻上我的晦气。又不象是恼我,又不象是恼二爷,夹枪带棒,终久是个什么主意?我就不多说,让你说去。”说著便往外走。宝玉向晴雯道:“你也不用生气,我也猜著你的心事了。我回太太去,你也大了,打发你出去好不好?”晴雯听了这话,不觉又伤起心来,含恨说道:“为什么我出去?要嫌我,变著法儿打发我出去,也不能够。”宝玉道:“我何曾经过这个吵闹?一定是 你要出去了。不如回太太,打发你去吧。”说著,站起来就要走。袭人忙回身拦住,笑道:“往那里去?”宝玉道:“回太太去。”袭人笑道:“好没意思!真个的去回,你也不怕臊了?便是他认真的要去,也等把这气下去了,等无事中说话儿回了太太也不迟。这会子急急的当作一件正经事去回,岂不叫太太犯疑?”宝玉道: “太太必不犯疑,我只明说是他闹著要去的。”晴雯哭道:“我多早晚闹著要去了?饶生了气,还拿话压派我。只管去回,我一头碰死了也不出这门儿。”宝玉道: “这也奇了。你又不去,你又闹些什么?我经不起这吵,不如去了倒干净。”说著一定要去回。袭人见拦不住,只得跪下了。碧痕、秋纹、麝月等众丫鬟见吵闹,都鸦雀无闻的在外头听消息,这会子听见袭人跪下央求,便一齐进来都跪下了。宝玉忙把袭人扶起来,叹了一声,在床上坐下,叫众人起去,向袭人道:“叫我怎么样 才好!这个心使碎了也没人知道。”说著不觉滴下泪来。袭人见宝玉流下泪来,自己也就哭了。
\end{parag}


\begin{parag}
    晴雯在旁哭著,方欲说话,只见林黛玉进来,便出去了。林黛玉笑道:“大节下怎么好好的哭起来?难道是为争粽子吃争恼了不成?”宝玉和袭人嗤的一笑。黛玉道:“二哥哥不告诉我,我问你就知道了。”一面说,一面拍著袭人的肩,笑道:“好嫂子,你告诉我。必定是你两个拌了嘴了。告诉妹妹,替你们和劝和劝。” 袭人推他道:“林姑娘你闹什么?我们一个丫头,姑娘只是混说。”黛玉笑道:“你说你是丫头,我只拿你当嫂子待。”宝玉道:“你何苦来替他招骂名儿。饶这么 著,还有人说闲话,还搁的住你来说他。”袭人笑道:“林姑娘,你不知道我的心事,除非一口气不来死了倒也罢了。”林黛玉笑道:“你死了,别人不知怎么样, 我先就哭死了。”宝玉笑道:“你死了,我作和尚去。”袭人笑道:“你老实些罢,何苦还说这些话。”林黛玉将两个指头一伸,抿嘴笑道:“作了两个和尚了。我从今以后都记著你作和尚的遭数儿。”宝玉听得,知道是他点前儿的话,自己一笑也就罢了。
\end{parag}


\begin{parag}
    一时黛玉去后,就有人说“薛大爷请”,宝玉只得去了。原来是吃酒,不能推辞,只得尽席而散。晚间回来,已带了几分酒,踉跄来至自己院内,只见院中早把乘凉枕榻设下,榻上有个人睡著。宝玉只当是袭人,一面在榻沿上坐下,一面推他,问道:“疼的好些了?”只见那人翻身起来说:“何苦来,又招我!”宝玉一 看,原来不是袭人,却是晴雯。宝玉将他一拉,拉在身旁坐下,笑道:“你的性子越发惯娇了。早起就是跌了扇子,我不过说了那两句,你就说上那些话。说我也罢了,袭人好意来劝,你又括上他,你自己想想,该不该?”晴雯道:“怪热的,拉拉扯扯作什么!叫人来看见象什么!我这身子也不配坐在这里。”宝玉笑道:“你既知道不配,为什么睡著呢?”晴雯没的话,嗤的又笑了,说:“你不来便使得,你来了就不配了。起来,让我洗澡去。袭人麝月都洗了澡,我叫了他们来。”宝玉笑道:“我才又吃了好些酒,还得洗一洗。你既没有洗,拿了水来咱们两个洗。”晴雯摇手笑道:“罢,罢,我不敢惹爷。还记得碧痕打发你洗澡,足有两三个时 辰,也不知道作什么呢。我们也不好进去的。后来洗完了,进去瞧瞧,地下的水淹著床腿,连席子上都汪著水,也不知是怎么洗了,笑了几天。我也没那工夫收拾, 也不用同我洗去。今儿也凉快,那会子洗了,可以不用再洗。我倒舀一盆水来,你洗洗脸通通头。才刚鸳鸯送了好些果子来,都湃在那水晶缸里呢,叫他们打发你吃。”宝玉笑道:“既这么著,你也不许洗去,只洗洗手来拿果子来吃罢。”晴雯笑道:“我慌张的很,连扇子还跌折了,那里还配打发吃果子。倘或再打破了盘 子,还更了不得呢。”宝玉笑道:“你爱打就打,这些东西原不过是借人所用,你爱这样,我爱那样,各自性情不同。比如那扇子原是扇的,你要撕著玩也可以使 得,只是不可生气时拿他出气。就如杯盘,原是盛东西的,你喜听那一声响,就故意的碎了也可以使得,只是别在生气时拿他出气。这就是爱物了。”晴雯听了,笑道:“既这么说,你就拿了扇子来我撕。我最喜欢撕的。”宝玉听了,便笑著递与他。晴雯果然接过来,嗤的一声,撕了两半,接著嗤嗤又听几声。宝玉在旁笑著 说:“响的好,再撕响些!”正说著,只见麝月走过来,笑道:“少作些孽罢。”宝玉赶上来,一把将他手里的扇子也夺了递与晴雯。晴雯接了,也撕了几半子,二人都大笑。麝月道:“这是怎么说,拿我的东西开心儿?”宝玉笑道:“打开扇子匣子你拣去,什么好东西!”麝月道:“既这么说,就把匣子搬了出来,让他尽力 的撕,岂不好?”宝玉笑道:“你就搬去。”麝月道:“我可不造这孽。他也没折了手,叫他自己搬去。”晴雯笑著,倚在床上说道:“我也乏了,明儿再撕罢。” 宝玉笑道:“古人云:‘千金难买一笑。’几把扇子能值几何!”一面说著,一面叫袭人。袭人才换了衣服走出来,小丫头佳蕙过来拾去破扇,大家乘凉,不消细说。
\end{parag}


\begin{parag}
    至次日午间,王夫人、薛宝钗、林黛玉众姊妹正在贾母房内坐著,就有人回:“史大姑娘来了。”一时果见史湘云带领众多丫鬟媳妇走进院来。宝黛玉等忙迎至 阶下相见。青年姊妹间经月不见,一旦相逢,其亲密自不必细说。一时进入房中,请安问好,都见过了。贾母因说:“天热,把外头的衣服脱脱罢。” 史湘云忙起身宽衣。王夫人因笑道:“也没见穿上这些作什么?”史湘云笑道:“都是二婶婶叫穿的,谁愿意穿这些。”宝钗一旁笑道:“姨娘不知道,他穿衣裳还 更爱穿别人的衣裳。可记得旧年三四月里,他在这里住著,把宝兄弟的袍子穿上,靴子也穿上,额子也勒上,猛一瞧倒象是宝兄弟,就是多两个坠子。他站在那椅子 后边,哄的老太太只是叫‘宝玉,你过来,仔细那上头挂的灯穗子招下灰来迷了眼’。他只是笑,也不过去。后来大家撑不住笑了,老太太才笑了,说:‘倒扮上男人好看了。’”林黛玉道:“这算什么。惟有前年正月里接了他来,住了没两日就下起雪来,老太太和舅母那日想是才拜了影回来,老太太的一个新新的大红猩猩毡斗蓬放在那里,谁知眼错不见他就披了,又大又长,他就拿了个汗巾子拦腰系上,和丫头们在后院子扑雪人儿去,一跤栽到沟跟前,弄了一身泥水。”说著,大家想 著前情,都笑了。宝钗笑向那周奶妈道:“周妈,你们姑娘还是那么淘气不淘气了?”周奶娘也笑了。迎春笑道:“淘气也罢了,我就嫌他爱说话。也没见睡在那里还是咭咭呱呱,笑一阵,说一阵,也不知那里来的那些话。”王夫人道:“只怕如今好了。前日有人家来相看,眼见有婆婆家了,还是那们著。”贾母因问:“今儿还是住著,还是家去呢?”周奶娘笑道:“老太太没有看见衣服都带了来,可不住两天?”史湘云问道:“宝玉哥哥不在家么?”宝钗笑道:“他再不想著别人,只想宝兄弟,两个人好憨的。这可见还没改了淘气。”贾母道:“如今你们大了,别提小名儿了。”
\end{parag}


\begin{parag}
    刚只说著,只见宝玉来了,笑道:“云妹妹来了。怎么前儿打发人接你去,怎么不来?”王夫人道:“这里老太太才说这一个,他又来提名道姓的了。”林黛玉 道:“你哥哥得了好东西,等著你呢。”史湘云道:“什么好东西?”宝玉笑道:“你信他呢!几日不见,越发高了。”湘云笑道:“袭人姐姐好?”宝玉道:“多谢你记挂。”湘云道:“我给他带了好东西来了。”说著,拿出手帕子来,挽著一个疙瘩。宝玉道:“什么好的?你倒不如把前儿送来的那种绛纹石的戒指儿带两个 给他。”湘云笑道:“这是什么?”说著便打开。众人看时,果然就是上次送来的那绛纹戒指,一包四个。林黛玉笑道:“你们瞧瞧他这主意。前儿一般的打发人给我们送了来,你就把他的带来岂不省事?今儿巴巴的自己带了来,我当又是什么新奇东西,原来还是他。真真你是糊涂人。”史湘云笑道:“你才糊涂呢!我把这理说出来,大家评一评谁糊涂。给你们送东西,就是使来的不用说话,拿进来一看,自然就知是送姑娘们的了;若带他们的东西,这得我先告诉来人,这是那一个丫头 的,那是那一个丫头的,那使来的人明白还好,再糊涂些,丫头的名字他也不记得,混闹胡说的,反连你们的东西都搅糊涂了。若是打发个女人素日知道的还罢了, 偏生前儿又打发小子来,可怎么说丫头们的名字呢?横竖我来给他们带来,岂不清白。”说著,把四个戒指放下,说道:“袭人姐姐一个,鸳鸯姐姐一个,金钏儿姐 姐一个,平儿姐姐一个:这倒是四个人的,难道小子们也记得这们清白?”众人听了都笑道:“果然明白。”宝玉笑道:“还是这么会说话,不让人。”林黛玉听 了,冷笑道:“他不会说话,他的金麒麟会说话。”一面说著,便起身走了。幸而诸人都不曾听见,只有薛宝钗抿嘴一笑。宝玉听见了,倒自己后悔又说错了话,忽见宝钗一笑,由不得也笑了。宝钗见宝玉笑了,忙起身走开,找了林黛玉去说话。
\end{parag}


\begin{parag}
    贾母向湘云道:“吃了茶歇一歇,瞧瞧你的嫂子们去。园里也凉快,同你姐姐们去逛逛。”湘云答应了,将三个戒指儿包上,歇了一歇,便起身要瞧凤姐等人 去。众奶娘丫头跟著,到了凤姐那里,说笑了一回,出来便往大观园来,见过了李宫裁,少坐片时,便往怡红院来找袭人。因回头说道:“你们不必跟著,只管瞧你 们的朋友亲戚去,留下翠缕伏侍就是了。”众人听了,自去寻姑觅嫂,早剩下湘云翠缕两个人。翠缕道:“这荷花怎么还不开?”史湘云道:“时候没到。”翠缕道: “这也和咱们家池子里的一样,也是楼子花?”湘云道:“他们这个还不如咱们的。”翠缕道:“他们那边有棵石榴,接连四五枝,真是楼子上起楼子,这也难为他 长。”史湘云道:“花草也是同人一样,气脉充足,长的就好。”翠缕把脸一扭,说道:“我不信这话。若说同人一样,我怎么不见头上又长出一个头来的人?”湘 云听了由不得一笑,说道:“我说你不用说话,你偏好说。这叫人怎么好答言?天地间都赋阴阳二气所生,或正或邪,或奇或怪,千变万化,都是阴阳顺逆。多少一 生出来,人罕见的就奇,究竟理还是一样。”翠缕道:“这么说起来,从古至今,开天辟地,都是阴阳了?”湘云笑道:“糊涂东西,越说越放屁。什么‘都是些阴 阳’,难道还有个阴阳不成!‘阴’‘阳’两个字还只是一字,阳尽了就成阴,阴尽了就成阳,不是阴尽了又有个阳生出来,阳尽了又有个阴生出来。”翠缕道: “这糊涂死了我!什么是个阴阳,没影没形的。我只问姑娘,这阴阳是怎么个样儿?”湘云道:“阴阳可有什么样儿,不过是个气,器物赋了成形。比如天是阳,地 就是阴;水是阴,火就是阳;日是阳,月就是阴。”翠缕听了,笑道:“是了,是了,我今儿可明白了。怪道人都管著日头叫‘太阳’呢,算命的管著月亮叫什么 ‘太阴星’,就是这个理了。”湘云笑道:“阿弥陀佛!刚刚的明白了。”翠缕道:“这些大东西有阴阳也罢了,难道那些蚊子、虼蚤、蠓虫儿、花儿、草儿、瓦片儿、砖头儿也有阴阳不成?”湘云道:“怎么有没阴阳的呢?比如那一个树叶儿还分阴阳呢,那边向上朝阳的便是阳,这边背阴覆下的便是阴。”翠缕听 了,点头笑道:“原来这样,我可明白了。只是咱们这手里的扇子,怎么是阳,怎么是阴呢?”湘云道:“这边正面就是阳,那边反面就为阴。”翠缕又点头笑了, 还要拿几件东西问,因想不起个什么来,猛低头就看见湘云宫绦上系的金麒麟,便提起来问道:“姑娘,这个难道也有阴阳?”湘云道:“走兽飞禽,雄为阳,雌为阴;牝为阴,牡为阳。怎么没有呢!”翠缕道:“这是公的,到底是母的呢?”湘云道:“这连我也不知道。”翠缕道:“这也罢了,怎么东西都有阴阳,咱们人倒 没有阴阳呢?”湘云照脸啐了一口道:“下流东西,好生走罢!越问越问出好的来了!” 翠缕笑道:“这有什么不告诉我的呢?我也知道了,不用难我。”湘云笑道:“你知道什么?”翠缕道:“姑娘是阳,我就是阴。”说著,湘云拿手帕子握著嘴,呵 呵的笑起来。翠缕道:“说是了,就笑的这样了。”湘云道:“很是,很是。”翠缕道:“人规矩主子为阳,奴才为阴。我连这个大道理也不懂得?”湘云笑道: “你很懂得。”
\end{parag}


\begin{parag}
    一面说,一面走,刚到蔷薇架下,湘云道:“你瞧那是谁掉的首饰,金晃晃在那里。”翠缕听了,忙赶上拾在手里攥著,笑道:“可分出阴阳来了。”说著,先 拿史湘云的麒麟瞧。湘云要他拣的瞧,翠缕只管不放手,笑道:“是件宝贝,姑娘瞧不得。这是从那里来的?好奇怪!我从来在这里没见有人有这个。”湘云笑道: “拿来我看。”翠缕将手一撒,笑道:“请看。”湘云举目一验,却是文彩辉煌的一个金麒麟,比自己佩的又大又有文彩。湘云伸手擎在掌上,只是默默不语,正自 出神,忽见宝玉从那边来了,笑问道:“你两个在这日头底下作什么呢?怎么不找袭人去?”湘云连忙将那麒麟藏起道:“正要去呢。咱们一处走。”说著,大家进入怡红院来。袭人正在阶下倚槛追风,忽见湘云来了,连忙迎下来,携手笑说一向久别情况。一时进来归坐,宝玉因笑道:“你该早来,我得了一件好东西,专等你呢。”说著,一面在身上摸掏,掏了半天,呵呀了一声,便问袭人“那个东西你收起来了么?”袭人道:“什么东西?”宝玉道:“前儿得的麒麟。”袭人道:“你 天天带在身上的,怎么问我?”宝玉听了,将手一拍说道:“这可丢了,往那里找去!”就要起身自己寻去。湘云听了,方知是他遗落的,便笑问道:“你几时又有了麒麟了?”宝玉道:“前儿好容易得的呢,不知多早晚丢了,我也糊涂了。”湘云笑道:“幸而是顽的东西,还是这么慌张。”说著,将手一撒,“你瞧瞧,是这个不是?”宝玉一见由不得欢喜非常,因说道……不知是如何,且听下回分解。
\end{parag}


\begin{parag}
    \begin{note}庚辰:后数十回若兰在射圃所佩之麒麟正此麒麟也。提纲伏于此回中,所谓“草蛇灰线,在千里之外”。\end{note}
\end{parag}

