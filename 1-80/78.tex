\chap{七十八}{老學士閒徵姽嫿詞 癡公子杜撰芙蓉誄}


\begin{parag}
    \begin{note}蒙回前總:文有賓主,不可誤。此文以《芙蓉誄》爲主,以《姽嫿詞》爲賓,以寶玉古詩䛴爲主,以賈蘭賈環詩絕爲賓。文有賓中賓,不可誤。以清客作序爲賓,以寶玉出遊作詩爲賓中賓。由虛入實,可歌可詠。\end{note}
\end{parag}


\begin{parag}
    話說兩個尼姑領了芳官等去後,王夫人便往賈母處來省晨,見賈母喜歡,便趁便回道:“寶玉屋裏有個晴雯,那個丫頭也大了,而且一年之間,病不離身;我常見他比別人份外淘氣,也懶;前日又病倒了十幾天,叫大夫瞧,說是女兒癆,所以我就趕著叫他下去了。若養好了也不用叫他進來,就賞他家配人去也罷了。再那幾個學戲的女孩子,我也作主放出去了。一則他們都會戲,口裏沒輕沒重,只會混說,女孩兒們聽了如何使得?二則他們既唱了會子戲,白放了他們,也是應該的。況丫頭們也太多,若說不夠使,再挑上幾個來也是一樣。”賈母聽了,點頭道:“這倒是正理,我也正想著如此呢。但晴雯那丫頭我看他甚好,怎麼就這樣起來。我的意思這些丫頭的模樣爽利言談針線多不及他,將來只他還可以給寶玉使喚得。誰知變了。”王夫人笑道:“老太太挑中的人原不錯。只怕他命裏沒造化,所以得了這個病。俗語又說:‘女大十八變。’況且有本事的人,未免就有些調歪。老太太還有什麼不曾經驗過的。三年前我也就留心這件事。先只取中了他,我便留心。冷眼看去,他色色雖比人強,只是不大沉重。若說沉重知大禮,莫若襲人第一。雖說賢妻美妾,然也要性情和順舉止沉重的更好些。就是襲人模樣雖比晴雯略次一等,然放在房裏,也算得一二等的了。況且行事大方,心地老實,這幾年來,從未逢迎著寶玉淘氣。凡寶玉十分胡鬧的事,他只有死勸的。因此品擇了二年,一點不錯了,我就悄悄的把他丫頭的月分錢止住,我的月分銀子裏批出二兩銀子來給他。不過使他自己知道越發小心學好之意。且不明說者,一則寶玉年紀尚小,老爺知道了又恐說耽誤了書;二則寶玉再自爲已是跟前的人不敢勸他說他,反倒縱性起來。所以直到今日纔回明老太太。”賈母聽了,笑道:“原來這樣,如此更好了。襲人本來從小兒不言不語,我只說他是沒嘴的葫蘆。既是你深知,豈有大錯誤的。而且你這不明說與寶玉的主意更好。且大家別提這事,只是心裏知道罷了。我深知寶玉將來也是個不聽妻妾勸的。我也解不過來,也從未見過這樣的孩子。別的淘氣都是應該的,只他這種和丫頭們好卻是難懂。我爲此也耽心,每每的冷眼查看他。只和丫頭們鬧,必是人大心大,知道男女的事了,所以愛親近他們。既細細查試,究竟不是爲此。豈不奇怪。想必原是個丫頭錯投了胎不成。”說著,大家笑了。王夫人又回今日賈政如何誇獎,又如何帶他們逛去,賈母聽了,更加喜悅。
\end{parag}


\begin{parag}
    一時,只見迎春妝扮了前來告辭過去。鳳姐也來省晨,伺候過早飯,又說笑了一回。賈母歇晌後,王夫人便喚了鳳姐,問他丸藥可曾配來。鳳姐兒道:“還不曾呢,如今還是喫湯藥。太太只管放心,我已大好了。”\begin{note}庚雙夾:總是勉強。\end{note}王夫人見他精神復初,也就信了。\begin{note}庚雙夾:只用此一句,便又伏下後文。\end{note}因告訴攆逐晴雯等事,又說:“怎麼寶丫頭私自回家睡了,你們都不知道?我前兒順路都查了一查。誰知蘭小子這一個新進來的奶子也十分的妖喬,我也不喜歡他。我也說與你嫂子了,好不好叫他各自去罷。況且蘭小子也大了,用不著奶子了。我因問你大嫂子:‘寶丫頭出去難道你也不知道不成?’他說是告訴了他的,不過住兩三日,等你姨媽好了就進來。姨媽究竟沒甚大病,不過還是咳嗽腰疼,年年是如此的。他這去必有原故,敢是有人得罪了他不成?那孩子心重,親戚們住一場,別得罪了人,反不好了。”鳳姐笑道:“誰可好好的得罪著他?況且他天天在園裏,左不過是他們姊妹那一羣人。”王夫人道:“別是寶玉有嘴無心,傻子似的從沒個忌諱,高興了信嘴胡說也是有的。”鳳姐笑道:“這可是太太過於操心了。若說他出去於正經事說正經話去,卻象個傻子;若只叫進來在這些姊妹跟前以至於大小的丫頭們跟前,他最有儘讓,又恐怕得罪了人,那是再不得有人惱他的。我想薛妹妹此去,想必爲著前時搜檢衆丫頭的東西的原故。他自然爲信不及園裏的人才搜檢,他又是親戚,現也有丫頭老婆在內,我們又不好去搜檢,恐我們疑他,所以多了這個心,自己迴避了。也是應該避嫌疑的。”
\end{parag}


\begin{parag}
    王夫人聽了這話不錯,自己遂低頭想了一想,便命人請了寶釵來分晰前日的事以解他疑心,又仍命他進來照舊居住。寶釵陪笑道:“我原要早出去的,只是姨娘有許多的大事,所以不便來說。可巧前日媽又不好了,家裏兩個靠得的女人也病著,我所以趁便出去了。姨娘今日既已知道了,我正好明講出情理來,就從今日辭了好搬東西的。”王夫人鳳姐都笑著:“你太固執了。正經再搬進來爲是,休爲沒要緊的事反疏遠了親戚。”寶釵笑道:“這話說的太不解了,並沒爲什麼事我出去。我爲的是媽近來神思比先大減,而且夜間晚上沒有得靠的人,通共只我一個。二則如今我哥哥眼看要娶嫂子,多少針線活計並家裏一切動用的器皿,尚有未齊備的,我也須得幫著媽去料理料理。姨媽和鳳姐姐都知道我們家的事,不是我撒謊。三則自我在園裏,東南上小角門子就常開著,原是爲我走的,保不住出入的人就圖省路也從那裏走,又沒人盤查,設若從那裏生出一件事來,豈不兩礙臉面。而且我進園裏來住原不是什麼大事,因前幾年年紀皆小,且家裏沒事,有在外頭的,不如進來姊妹相共,或作針線,或頑笑,皆比在外頭悶坐著好,如今彼此都大了,也彼此皆有事。況姨娘這邊歷年皆遇不遂心的事故,那園子也太大,一時照顧不到,皆有關係,惟有少幾個人,就可以少操些心。所以今日不但我執意辭去,之外還要勸姨娘如今該減些的就減些,也不爲失了大家的體統。據我看,園裏這一項費用也竟可以免的,說不得當日的話。姨娘深知我家的,難道我們當日也是這樣冷落不成。”鳳姐聽了這篇話,便向王夫人笑道:“這話竟是,不必強了。”王夫人點頭道:“我也無可回答,只好隨你便罷了。”
\end{parag}


\begin{parag}
    話說之間,只見寶玉等已回來,因說他父親還未散,恐天黑了,所以先叫我們回來了。王夫人忙問:“今日可有丟了醜?”寶玉笑道:“不但不丟醜,倒拐了許多東西來。”接著,就有老婆子們從二門上小廝手內接了東西來。王夫人一看時,只見扇子三把,扇墜三個,筆墨共六匣,香珠三串,玉絛環三個。寶玉說道:“這是梅翰林送的,那是楊侍郎送的,這是李員外送的,每人一分。”說著又向懷中取出一個旃檀香小護身佛來,說:“這是慶國公單給我的。”王夫人又問在席何人,作何詩詞等語畢,只將寶玉一分令人拿著,同寶玉蘭環前來見過賈母。賈母看了,喜歡不盡,不免又問些話。無奈寶玉一心記著晴雯,答應完了話時,便說騎馬顛了,骨頭疼。賈母便說:“快回房去換了衣服,疏散疏散就好了,不許睡倒。”寶玉聽了,便忙入園來。
\end{parag}


\begin{parag}
    當下麝月秋紋已帶了兩個丫頭來等候,見寶玉辭了賈母出來,秋紋便將筆墨拿起來,一同隨寶玉進園來。寶玉滿口裏說“好熱”,一壁走,一壁便摘冠解帶,將外面的大衣服都脫下來麝月拿著,\begin{note}庚雙夾:看他用智之處。\end{note}只穿著一件松花綾子夾襖,襖內露出血點般大紅褲子來。秋紋見這條紅褲是晴雯手內針線,因嘆道:“這條褲子以後收了罷,真是物件在人去了。”麝月忙也笑道:“這是晴雯的針線。”又嘆道:“真真物在人亡了!”秋紋將麝月拉了一把,笑道:“這褲子配著松花色襖兒、石青靴子,越顯出這靛青的頭,雪白的臉來了。”寶玉在前只裝聽不見,又走了兩步,便止步道:“我要走一走,這怎麼好?”麝月道:“大白日裏,還怕什麼?還怕丟了你不成!”因命兩個小丫頭跟著,“我們送了這些東西去再來。”寶玉道:“好姐姐,等一等我再去。”麝月道:“我們去了就來。兩個人手裏都有東西,倒向擺執事的,一個捧著文房四寶,一個捧著冠袍帶履,成個什麼樣子。”寶玉聽見,正中心懷,便讓他兩個去了。
\end{parag}


\begin{parag}
    他便帶了兩個小丫頭到一石後,也不怎麼樣,只問他二人道:“自我去了,你襲人姐姐打發人瞧晴雯姐姐去了不曾?”這一個答道:“打發宋媽媽瞧去了。”寶玉道:“回來說什麼?”小丫頭道:“回來說晴雯姐姐直著脖子叫了一夜,今日早起就閉了眼,住了口,世事不知,也出不得一聲兒,只有倒氣的分兒了。”寶玉忙道:“一夜叫的是誰?”小丫頭子說:“一夜叫的是娘。”寶玉拭淚道:“還叫誰?”小丫頭子道:“沒有聽見叫別人了。”寶玉道:“你糊塗,想必沒有聽真。” 旁邊那一個小丫頭最伶俐,聽寶玉如此說,便上來說:“真個他糊塗。”又向寶玉道:“不但我聽得真切,我還親自偷著看去的。”寶玉聽說,忙問:“你怎麼又親自看去?”小丫頭道:“我因想晴雯姐姐素日與別人不同,待我們極好。如今他雖受了委屈出去,我們不能別的法子救他,只親去瞧瞧,也不枉素日疼我們一場。就是人知道了回了太太,打我們一頓,也是願受的。所以我拚著挨一頓打,偷著下去瞧了一瞧。誰知他平生爲人聰明,至死不變。他因想著那起俗人不可說話,所以只閉眼養神,見我去了便睜開眼,拉我的手問:‘寶玉那去了?’我告訴他實情。他嘆了一口氣說:‘不能見了。’我就說:‘姐姐何不等一等他回來見一面,豈不兩完心願?’他就笑道:‘你們還不知道。我不是死,如今天上少了一位花神,玉皇敕命我去司主。我如今在未正二刻到任司花,寶玉須待未正三刻纔到家,只少得一刻的工夫,不能見面。世上凡該死之人閻王勾取了過去,是差些小鬼來捉人魂魄。若要遲延一時半刻,不過燒些紙錢澆些漿飯,那鬼只顧搶錢去了,該死的人就可多待些個工夫。\begin{note}庚雙夾:好奇之至!古來皆說“閻王註定三更死,誰敢留人至五更”之語,今忽藉此小女兒一篇無稽之談,反成無人敢翻之案,且又寓意調侃,罵盡世態。豈非……之至文章耶?寄語觀者:至此一浮一大白者,以後不必看書也。\end{note}我這如今是有天上的神仙來召請,豈可捱得時刻!’我聽了這話,竟不大信,及進來到房裏留神看時辰表時,果然是未正二刻他嚥了氣,正三刻上就有人來叫我們,說你來了。這時候倒都對合。”寶玉忙道:“你不識字看書,所以不知道。這原是有的,不但花有一個神,一樣花有一位神之外還有總花神。但他不知是作總花神去了,還是單管一樣花的神?”這丫頭聽了,一時謅不出來。恰好這是八月時節,園中池上芙蓉正開。這丫頭便見景生情,忙答道:“我也曾問他是管什麼花的神,告訴我們日後也好供養的。他說:‘天機不可泄漏。你既這樣虔誠,我只告訴你,你只可告訴寶玉一人。除他之外若泄了天機,五雷就來轟頂的。’他就告訴我說,他就是專管這芙蓉花的。”寶玉聽了這話,不但不爲怪,亦且去悲而生喜,乃指芙蓉笑道:“此花也須得這樣一個人去司掌。我就料定他那樣的人必有一番事業做的。雖然超出苦海,從此不能相見,也免不得傷感思念。”因又想:“雖然臨終未見,如今且去靈前一拜,也算盡這五六年的情常。”
\end{parag}


\begin{parag}
    想畢忙至房中,又另穿戴了,只說去看黛玉,遂一人出園來,往前次之處去,意爲停柩在內。誰知他哥嫂見他一嚥氣便回了進去,希圖早些得幾兩發送例銀。王夫人聞知,便命賞了十兩燒埋銀子。又命:“即刻送到外頭焚化了罷。女兒癆死的,斷不可留!”他哥嫂聽了這話,一面得銀,一面就僱了人來入殮,抬往城外化人場上去了。剩的衣履簪環,約有三四百金之數,他兄嫂自收了爲後日之計。二人將門鎖上,一同送殯去未回。寶玉走來撲了個空。\begin{note}庚雙夾:收拾晴雯,故爲紅顏一哭。然亦大令人不堪。上雲王夫人怕女兒癆不祥,今則忽從寶玉心中其苦,又模擬出非是已抑鬱詞其母子至心中體貼眷愛之情曲委已盡。\end{note}(按:此句不解。)
\end{parag}


\begin{parag}
    寶玉自立了半天,別無法兒,只得復身進入園中。待回至房中,甚覺無味,因乃順路來找黛玉。偏黛玉不在房中,問其何往,丫鬟們回說:“往寶姑娘那裏去了。”寶玉又至蘅蕪苑中,只見寂靜無人,房內搬的空空落落的,不覺喫一大驚。(不覺喫一大驚,纔想起前日彷彿聽見寶釵要搬出去,只因這兩日工課忙就混忘了,這時看見如此,才知道果然搬出。怔了半天,因轉念一想:“不如還是和襲人廝混,再與黛玉相伴。)忽見個老婆子走來,寶玉忙問這是什麼原故。老婆子道:“寶姑娘出去了。這裏交我們看著,還沒有搬清楚。我們幫著送了些東西去,這也就完了。你老人家請出去罷,讓我們掃掃灰塵也好,從此你老人家省跑這一處的腿子了。”寶玉聽了,怔了半天,因看著那院中的香藤異蔓,仍是翠翠青青,忽比昨日好似改作淒涼了一般,更又添了傷感。默默出來,又見門外的一條翠樾埭上也半日無人來往,不似當日各處房中丫鬟不約而來者絡繹不絕。又俯身看那埭下之水,仍是溶溶脈脈的流將過去。心下因想:“天地間竟有這樣無情的事!”悲感一番,忽又想到去了司棋、入畫、芳官等五個;死了晴雯;今又去了寶釵等一處;迎春雖尚未去,然連日也不見回來,且接連有媒人來求親:大約園中之人不久都要散的了。縱生煩惱,也無濟於事。不如還是找黛玉去相伴一日,回來還是和襲人廝混,只這兩三個人,只怕還是同死同歸的。想畢,仍往瀟湘館來,偏黛玉尚未回來。寶玉想亦當出去候送纔是,無奈不忍悲感,還是不去的是,遂又垂頭喪氣的回來。( 只這兩三個人,只怕還是同死同歸。”想畢,仍往瀟湘館來。偏黛玉還未回來。正在不知所之,忽見王夫人的丫頭進來找他,說:“老爺回來了,找你呢。又得了好題目了。快走,快走。”寶玉聽了,只得跟了出來。到王夫人屋裏,他父親已出去了,王夫人命人送寶玉至書房裏。)正在不知所以之際,忽見王夫人的丫頭進來找他說:“老爺回來了,找你呢,又得了好題目來了。快走,快走。”寶玉聽了,只得跟了出來。到王夫人房中,他父親已出去了。王夫人命人送寶玉至書房中。
\end{parag}


\begin{parag}
    彼時賈政正與衆幕友們談論尋秋之勝,又說:“快散時忽然談及一事,最是千古佳談,‘風流雋逸,忠義慷慨’八字皆備,倒是個好題目,大家要作一首輓詞。”衆幕賓聽了,都忙請教系何等妙事。賈政乃道:“當日曾有一位王封曰恆王,出鎮青州。這恆王最喜女色,且公餘好武,因選了許多美女,日習武事。每公餘輒開宴連日,令衆美女習戰鬥攻拔之事。其姬中有姓林行四者,姿色既冠,且武藝更精,皆呼爲林四娘。恆王最得意,遂超拔林四娘統轄諸姬,又呼爲‘姽嫿將軍’。”衆清客都稱“妙極神奇。竟以‘姽嫿’下加‘將軍’二字,反更覺嫵媚風流,真絕世奇文也。想這恆王也是千古第一風流人物了。”賈政笑道:“這話自然是如此,但更有可奇可嘆之事。”衆清客都愕然驚問道:“不知底下有何奇事?”賈政道:“誰知次年便有‘黃巾’‘赤眉’一干流賊餘黨復又烏合,搶掠山左一帶。\begin{note}庚雙夾:妙!“赤眉”“黃巾”兩時之事,今合而爲一,蓋雲一過是此等衆類,非特歷歷指明某赤某黃。若雲不合兩用便呆矣。此書全是如此,爲混人也。\end{note}恆王意爲犬羊之惡,不足大舉,因輕騎前剿。不意賊衆頗有詭譎智術,兩戰不勝,恆王遂爲衆賊所戮。於是青州城內文武官員,各各皆謂:‘王尚不勝,你我何爲!’遂將有獻城之舉。林四娘得聞兇報,遂集聚衆女將,發令說道:‘你我皆向蒙王恩,戴天履地,不能報其萬一。今王既殞身國事,我意亦當殞身於王。爾等有願隨者,即時同我前往;有不願者,亦早各散。’衆女將聽他這樣,都一齊說願意。於是林四娘帶領衆人連夜出城,直殺至賊營裏頭。衆賊不防,也被斬戮了幾員首賊。然後大家見是不過幾個女人,料不能濟事,遂回戈倒兵,奮力一陣,把林四娘等一個不曾留下,倒作成了這林四孃的一片忠義之志。後來報至中都,自天子以至百官,無不驚駭道奇。其後朝中自然又有人去剿滅,天兵一到,化爲烏有,不必深論。只就林四娘一節,衆位聽了,可羨不可羨呢?”衆幕友都嘆道:“實在可羨可奇,實是個妙題,原該大家挽一挽纔是。”說著,早有人取了筆硯,按賈政口中之言稍加改易了幾個字,便成了一篇短序,遞與賈政看了。賈政道:“不過如此。他們那裏已有原序。昨日因又奉恩旨,著察覈前代以來應加褒獎而遺落未經請奏各項人等,無論僧尼乞丐與女婦人等,有一事可嘉,即行匯送履歷至禮部備請恩獎。所以他這原序也送往禮部去了。大家聽見這新聞,所以都要作一首《姽嫿詞》,以志其忠義。”衆人聽了,都又笑道:“這原該如此。只是更可羨者,本朝皆系千古未有之曠典隆恩,實歷代所不及處,可謂‘聖朝無闕事’,唐朝人預先竟說了,竟應在本朝。如今年代方不虛此一句。”賈政點頭道:“正是。”
\end{parag}


\begin{parag}
    說話間,賈環叔侄亦到。賈政命他們看了題目。他兩個雖能詩,較腹中之虛實雖也去寶玉不遠,但第一件他兩個終是別路,若論舉業一道,似高過寶玉,若論雜學,則遠不能及;第二件他二人才思滯鈍,不及寶玉空靈娟逸,每作詩亦如八股之法,未免拘板庸澀。那寶玉雖不算是個讀書人,然虧他天性聰敏,且素喜好些雜書,他自爲古人中也有杜撰的,也有誤失之處,拘較不得許多;若只管怕前怕後起來,縱堆砌成一篇,也覺得甚無趣味。因心裏懷著這個念頭,每見一題,不拘難易,他便毫無費力之處,就如世上的流嘴滑舌之人,無風作有,信著伶口俐舌,長篇大論,胡扳亂扯,敷演出一篇話來。雖無稽考,卻都說得四座春風。雖有正言厲語之人,亦不得壓倒這一種風流去。近日賈政年邁,名利大灰,然起初天性也是個詩酒放誕之人,因在子侄輩中,少不得規以正路。近見寶玉雖不讀書,竟頗能解此,細評起來,也還不算十分玷辱了祖宗。就思及祖宗們,各各亦皆如此,雖有深精舉業的,也不曾發跡過一個,看來此亦賈門之數。況母親溺愛,遂也不強以舉業逼他了。所以近日是這等待他。又要環蘭二人舉業之餘,怎得亦同寶玉纔好,所以每欲作詩,必將三人一齊喚來對作。\begin{note}庚雙夾:妙!世事皆不可無足饜,只有“讀書”二字是萬不可足饜的。父母之心可不甚哉!近之父母只怕兒子不能名利,豈不可嘆乎?\end{note}
\end{parag}


\begin{parag}
    閒言少述。且說賈政又命他三人各吊一首,誰先成者賞,佳者額外加賞。賈環賈蘭二人近日當著多人皆作過幾首了,膽量逾壯,今看了題,遂自去思索。一時,賈蘭先有了。賈環生恐落後也就有了。二人皆已錄出,寶玉尚出神。\begin{note}庚雙夾:妙篇寫出鈍態來。\end{note}賈政與衆人且看他二人的二首。賈蘭的是一首七言絕,寫道是:
\end{parag}


\begin{poem}
    \begin{pl}姽嫿將軍林四娘,玉爲肌骨鐵爲腸,\end{pl}

    \begin{pl}捐軀自報恆王后,此日青州土亦香。\end{pl}
\end{poem}


\begin{parag}
    衆幕賓看了,便皆大讚:“小哥兒十三歲的人就如此,可知家學淵源,真不誣矣。”賈政笑道:“稚子口角,也還難爲他。”又看賈環的,是首五言律,寫道是:
\end{parag}


\begin{poem}
    \begin{pl}紅粉不知愁,將軍意未休。\end{pl}

    \begin{pl}掩啼離繡幕,抱恨出青州。\end{pl}

    \begin{pl}自謂酬王德,詎能復寇仇。\end{pl}

    \begin{pl}誰題忠義墓,千古獨風流。\end{pl}


\end{poem}


\begin{parag}
    衆人道:“更佳。倒是大幾歲年紀,立意又自不同。”賈政道:“還不甚大錯,終不懇切。”衆人道:“這就罷了。三爺才大不多兩歲,在未冠之時如此,用了工夫,再過幾年,怕不是大阮小阮了。”賈政笑道:“過獎了。只是不肯讀書過失。”因又問寶玉怎樣。衆人道:“二爺細心鏤刻,定又是風流悲感,不同此等的了。”寶玉笑道:“這個題目似不稱近體,須得古體,或歌或行,長篇一首,方能懇切。”衆人聽了,都立身點頭拍手道:“我說他立意不同!每一題到手必先度其體格宜與不宜,這便是老手妙法。就如裁衣一般,未下剪時,須度其身量。這題目名曰《姽嫿詞》,且既有了序,此必是長篇歌行方合體的。或擬白樂天《長恨歌》,或擬詠古詞,半敘半詠,流利飄逸,始能盡妙。”賈政聽說,也合了主意,遂自提筆向紙上要寫,又向寶玉笑道:“如此,你念我寫。不好了,我捶你那肉。誰許你先大言不慚了!”寶玉只得唸了一句,道是:
\end{parag}


\begin{poem}
    \begin{pl}恆王好武兼好色,\end{pl}
\end{poem}


\begin{parag}
    賈政寫了看時,搖頭道:“粗鄙。”一幕賓道:“要這樣方古,究竟不粗。且看他底下的。”賈政道:“姑存之。”寶玉又道:
\end{parag}


\begin{poem}
    \begin{pl}遂教美女習騎射。\end{pl}

    \begin{pl}穠歌豔舞不成歡,列陣挽戈爲自得。\end{pl}


\end{poem}


\begin{parag}
    賈政寫出,衆人都道:“只這第三句便古樸老健,極妙。這四句平敘出,也最得體。”賈政道:“休謬加獎譽,且看轉的如何。”寶玉念道:
\end{parag}


\begin{poem}
    \begin{pl}眼前不見塵沙起,將軍俏影紅燈裏。\end{pl}

\end{poem}


\begin{parag}
    衆人聽了這兩句,便都叫:“妙!好個‘不見塵沙起’!又承了一句‘俏影紅燈裏’,用字用句,皆入神化了。”寶玉道:
\end{parag}


\begin{poem}
    \begin{pl}叱吒時聞口舌香,霜矛雪劍嬌難舉。\end{pl}
\end{poem}


\begin{parag}
    衆人聽了,便拍手笑道:“益發畫出來了。當日敢是寶公也在座,見其嬌且聞其香否?不然,何體貼至此。”寶玉笑道:“閨閣習武,任其勇悍,怎似男人。\begin{note}庚雙夾:賈老在座,故不便出“濁物”二字,妙甚細甚!\end{note}不待問而可知嬌怯之形的了。”賈政道:“還不快續,這又有你說嘴的了。”寶玉只得又想了一想,念道:
\end{parag}


\begin{poem}
    \begin{pl}丁香結子芙蓉絛,\end{pl}
\end{poem}


\begin{parag}
    衆人都道:“轉‘絛’,蕭韻,更妙,這才流利飄蕩。而且這一句也綺靡秀媚的妙。”賈政寫了,看道:“這一句不好。已寫過‘口舌香’‘嬌難舉’,何必又如此。這是力量不加,故又用這些堆砌貨來搪塞。”寶玉笑道:“長歌也須得要些詞藻點綴點綴,不然便覺蕭索。”賈政道:“你只顧用這些,但這一句底下如何能轉至武事?若再多說兩句,豈不蛇足了。”寶玉道:“如此,底下一句轉煞住,想亦可矣。”賈政冷笑道:“你有多大本領?上頭說了一句大開門的散話,如今又要一句連轉帶煞,豈不心有餘而力不足些。”寶玉聽了,垂頭想了一想,說了一句道:
\end{parag}


\begin{poem}
    \begin{pl}不繫明珠系寶刀。\end{pl}
\end{poem}


\begin{parag}
    忙問:“這一句可還使得?”衆人拍案叫絕。賈政寫了,看著笑道:“且放著,再續。”寶玉道:“若使得,我便要一氣下去了。若使不得,越性塗了,我再想別的意思出來,再另措詞。”賈政聽了,便喝道:“多話!不好了再作,便作十篇百篇,還怕辛苦了不成!”寶玉聽說,只得想了一會,便念道:
\end{parag}


\begin{poem}
    \begin{pl}戰罷夜闌心力怯,脂痕粉漬污鮫鮹。\end{pl}
\end{poem}


\begin{parag}
    賈政道:“又一段。底下怎樣?”寶玉道:
\end{parag}


\begin{poem}
    \begin{pl}明年流寇走山東,強吞虎豹勢如蜂。\end{pl}
\end{poem}


\begin{parag}
    衆人道:“好個‘走’字!便見得高低了。且通句轉的也不板。”寶玉又念道:
\end{parag}


\begin{poem}
    \begin{pl}王率天兵思剿滅,一戰再戰不成功。\end{pl}

    \begin{pl}腥風吹折隴頭麥,日照旌旗虎帳空。\end{pl}

    \begin{pl}青山寂寂水澌澌,正是恆王戰死時。\end{pl}

    \begin{pl}雨淋白骨血染草,月冷黃沙鬼守屍。\end{pl}

\end{poem}


\begin{parag}
    衆人都道:“妙極,妙極!佈置,敘事,詞藻,無不盡美。且看如何至四娘,必另有妙轉奇句。”寶玉又念道:
\end{parag}


\begin{poem}
    \begin{pl}紛紛將士只保身,青州眼見皆灰塵,\end{pl}

    \begin{pl}不期忠義明閨閣,憤起恆王得意人。\end{pl}

\end{poem}


\begin{parag}
    衆人都道:“鋪敘得委婉。”賈政道:“太多了,底下只怕累贅呢。”寶玉乃又念道:
\end{parag}


\begin{poem}
    \begin{pl}恆王得意數誰行,姽嫿將軍林四娘,\end{pl}

    \begin{pl}號令秦姬驅趙女,豔李穠桃臨戰場。\end{pl}

    \begin{pl}繡鞍有淚春愁重,鐵甲無聲夜氣涼。\end{pl}

    \begin{pl}勝負自然難預定,誓盟生死報前王。\end{pl}

    \begin{pl}賊勢猖獗不可敵,柳折花殘實可傷,\end{pl}

    \begin{pl}魂依城郭家鄉近,馬踐胭脂骨髓香。\end{pl}

    \begin{pl}星馳時報入京師,誰家兒女不傷悲!\end{pl}

    \begin{pl}天子驚慌恨失守,此時文武皆垂首。\end{pl}

    \begin{pl}何事文武立朝綱,不及閨中林四娘!\end{pl}

    \begin{pl}我爲四娘長太息,歌成餘意尚傍徨。\end{pl}

\end{poem}


\begin{parag}
    念畢,衆人都大讚不止,又都從頭看了一遍。賈政笑道:“雖然說了幾句,到底不大懇切。”因說:“去罷。”三人如得了赦的一般,一齊出來,各自回房。
\end{parag}


\begin{parag}
    衆人皆無別話,不過至晚安歇而已。獨有寶玉一心悽楚,回至園中,猛然見池上芙蓉,想起小鬟說晴雯作了芙蓉之神,不覺又喜歡起來,乃看著芙蓉嗟嘆了一會。忽又想起死後並未到靈前一祭,如今何不在芙蓉前一祭,豈不盡了禮,比俗人去靈前祭弔又更覺別緻。想畢,便欲行禮。忽又止住道:“雖如此,亦不可太草率,也須得衣冠整齊,奠儀周備,方爲誠敬。”想了一想,“如今若學那世俗之奠禮,斷然不可;竟也還別開生面,另立排場,風流奇異,於世無涉,方不負我二人之爲人。況且古人有云:‘潢污行潦,蘋蘩蘊藻之賤,可以羞王公,薦鬼神。’原不在物之貴賤,全在心之誠敬而已。此其一也。二則誄文輓詞也須另出己見,自放手眼,亦不可蹈襲前人的套頭,填寫幾字搪塞耳目之文,亦必須灑淚泣血,一字一咽,一句一啼,寧使文不足悲有餘,萬不可尚文藻而反失悲慼。況且古人多有微詞,非自我今作俑也。奈今人全惑於功名二字,尚古之風一洗皆盡,恐不合時宜,於功名有礙之故。我又不希罕那功名,不爲世人觀閱稱讚,何必不遠師楚人之《大言》、《招魂》、《離騷》、《九辯》、《枯樹》、《問難》、《秋水》、《大人先生傳》等法,或雜參單句,或偶成短聯,或用實典,或設譬寓,隨意所之,信筆而去,喜則以文爲戲,悲則以言志痛,辭達意盡爲止,何必若世俗之拘拘於方寸之間哉。”寶玉本是個不讀書之人,再心中有了這篇歪意,怎得有好詩文作出來。他自己卻任意纂著,並不爲人知慕,所以大肆妄誕,竟杜撰成一篇長文,用晴雯素日所喜之冰鮫縠一幅楷字寫成,名曰《芙蓉女兒誄》,前序後歌。又備了四樣晴雯所喜之物,於是夜月下,命那小丫頭捧至芙蓉花前。先行禮畢,將那誄文即掛於芙蓉枝上,乃泣涕唸曰:
\end{parag}


\begin{qute2sp}
    \bfseries
    維太平不易之元,\begin{note}庚雙夾:年便奇。\end{note}蓉桂競芳之月,\begin{note}庚雙夾:是八月。\end{note}無可奈何之日,\begin{note}庚雙夾:日更奇。試思日何難於直說某某,今偏用如此說,則可知矣。\end{note}怡紅院濁玉,\begin{note}庚雙夾:自謙得更奇。蓋常以“濁”字許天下之男子,竟自謂,所謂“以責人之心責己”矣。\end{note}謹以羣花之蕊,\begin{note}庚雙夾:奇香。\end{note}冰鮫之縠,\begin{note}庚雙夾:奇帛。\end{note}沁芳之泉,\begin{note}庚雙夾:奇奠。\end{note}楓露之茗,\begin{note}庚雙夾:奇名。\end{note}四者雖微,聊以達誠申信,乃致祭於白帝宮中撫司秋豔芙蓉女兒\begin{note}庚雙夾:奇稱。\end{note}之前曰:



    \begin{parag}
        竊思女兒自臨濁世,
        \begin{note}庚雙夾:世不濁,內物所混而濁也,前後便有照應。“女兒”稱妙!蓋思普天下之稱斷不能有如此二字之清潔者。亦是寶玉真心。\end{note}
        迄今凡十有六載。
        \begin{note}庚雙夾:方十六歲而夭,亦可傷矣。\end{note}
        其先之鄉籍姓氏,湮淪而莫能考者久矣。
        \begin{note}庚雙夾:忽又有此文,不可後來,亦可傷矣。\end{note}
        而玉得於衾枕櫛沐之間,棲息宴遊之夕,親暱狎褻,相與共處者,僅五年八月有畸。
        \begin{note}庚雙夾:相共不足六載,一旦夭別,豈不可傷!\end{note}
        憶女兒曩生之昔,其爲質則金玉不足喻其貴,其爲性則冰雪不足喻其潔,其爲神則星日不足喻其精,其爲貌則花月不足喻其色。
        姊妹悉慕媖嫺,嫗媼鹹仰惠德。孰料鳩鴆惡其高,鷹鷙翻遭罦罬;
        \begin{note}庚雙夾:《離騷》:“鷙鳥之不羣兮。”又語:“吾令鴆爲媒兮,鴆告餘以不好。雄鳩之鳴逝兮,餘猶惡其佻巧。”注:鷙特立不羣,故不於……。羽毒殺人。鳩多聲有如人之多言不實。罦罬,音孚拙。翻車網。《詩經》:“雉離於罦。”《爾雅》:“罬謂之罦。”\end{note}
        薋葹妒其臭,茝蘭竟被芟鉏!
        \begin{note}庚雙夾:《離騷》。薋、葹皆惡草,以別邪佞。茝蘭,芳草,以別君子。\end{note}
        花原自怯,豈奈狂飆;柳本多愁,何禁驟雨。偶遭蠱蠆之讒,遂抱膏肓之疚。故爾櫻脣紅褪,韻吐呻吟;杏臉香枯,色陳顑頷。
        \begin{note}庚雙夾:《離騷》:“長顑頷亦何傷。”面黃色。\end{note}
        諑謠謑詬,出自屏幃;荊棘蓬榛,蔓延戶牖。豈招尤則替,實攘詬而終。
        \begin{note}庚雙夾:《離騷》:“謇朝誶而夕替。”廢也。“忍尤□鄰浮!比同取也。\end{note}
        既忳幽沉於不盡,復含罔屈於無窮。高標見嫉,閨幃恨比長沙;
        \begin{note}庚雙夾:汲黯輩嫉賈誼之才,譎貶長沙。\end{note}
        直烈遭危,巾幗慘於羽野。
        \begin{note}庚雙夾:鯀剛直自命,舜殛於羽山。《離騷》曰:“鯀婞直以亡身兮,終然殀乎羽之野。”\end{note}
        自蓄辛酸,誰憐夭折!仙雲既散,芳趾難尋。洲迷聚窟,何來卻死香?海失靈槎,不獲回生之藥。眉黛煙青,昨猶我畫;指環玉冷,今倩誰溫?鼎爐之剩藥猶存,襟淚之餘痕尚漬。鏡分鸞別,愁開麝月之奩;梳化龍飛,哀折檀雲之齒。
        委金鈿於草莽,拾翠盒於塵埃。樓空鳷鵲,徒懸七夕之針;
        帶斷鴛鴦,誰續五絲之縷?況乃金天屬節,白帝司時,孤衾有夢,空室無人。桐階月暗,芳魂與倩影同銷;蓉帳香殘,嬌喘共細言皆絕。
        連天衰草,豈獨蒹葭;匝地悲聲,無非蟋蟀。
        露苔晚砌,穿簾不度寒砧;雨荔秋垣,隔院希聞怨笛。芳名未泯,檐前鸚鵡猶呼;豔質將亡,檻外海棠預老。
        \begin{note}庚雙夾:極恰!\end{note}
        捉迷屏後,蓮瓣無聲;
        \begin{note}庚雙夾:元微之詩:“小樓深迷藏。”\end{note}
        鬥草庭前,蘭芽枉待。拋殘繡線,銀箋彩縷誰裁?折斷冰絲,金斗御香未熨。昨承嚴命,既趨車而遠陟芳園;今犯慈威,復拄杖而遽拋孤匶。\begin{note}庚雙夾:柩本字。
        \end{note}及聞槥棺被燹,慚違共穴之盟;石槨成災,愧迨同灰之誚。
        \begin{note}庚雙夾:唐詩云:“光開石棺,木可爲棺。”晉楊公回詩云:“生爲並身楊,死作同棺灰。”\end{note}
        爾乃西風古寺,淹滯青燐;落日荒丘,零星白骨。楸榆颯颯,蓬艾蕭蕭。隔霧壙以啼猿,繞煙塍而泣鬼。自爲紅綃帳裏,公子情深;始信黃土壟中,女兒命薄!汝南淚血,斑斑灑向西風;梓澤餘衷,默默訴憑冷月。嗚呼!固鬼蜮之爲災,豈神靈而亦妒。箝詖奴之口,討豈從寬;剖悍婦之心,忿猶未釋!
        \begin{note}庚雙夾:《莊子》:“箝楊墨之口。”《孟子》謂:“詖辭知其所蔽。”\end{note}
        在君之塵緣雖淺,然玉之鄙意豈終。因蓄惓惓之思,不禁諄諄之問。始知上帝垂旌,花宮待詔,生儕蘭蕙,死轄芙蓉。聽小婢之言,似涉無稽;以濁玉之思,則深爲有據。何也?昔葉法善攝魂以撰碑,李長吉被詔而爲記,事雖殊,其理則一也。故相物以配才,苟非其人,惡乃濫乎?始信上帝委託權衡,可謂至洽至協,庶不負其所秉賦也。因希其不昧之靈,或陟降於茲;特不揣鄙俗之詞,有污慧聽。乃歌而招之曰:
    \end{parag}


    \begin{poem}
        \begin{pl}天何如是之蒼蒼兮,乘玉虯以遊乎穹窿耶?\end{pl}
        \begin{note}庚雙行夾批:《楚辭》:“駟玉虯以乘鷖兮。”\end{note}

        \begin{pl}地何如是之茫茫兮,駕瑤象以降乎泉壤耶?\end{pl}
        \begin{note}庚雙行夾批:《楚辭》:“雜瑤象以爲車。”\end{note}

        \begin{pl}望徹蓋之陸離兮,抑箕尾之光耶?\end{pl}

        \begin{pl}列羽葆而爲前導兮,衛危虛於旁耶?\end{pl}

        \begin{pl}驅豐隆以爲比從兮,望舒月以離耶?\end{pl}
        \begin{note}庚雙行夾批:“危”“虛”二星爲衛護星。“豐隆”,電師“舒月”御也。\end{note}

        \begin{pl}聽車軌而伊軋兮,御鸞鷖以徵耶?\end{pl}

        \begin{pl}問馥郁而薆然兮,紉蘅杜以爲纕耶?\end{pl}

        \begin{pl}餘中心爲之慨然兮,\end{pl}
        \begin{note}庚雙夾:《莊子•至樂篇》:“我獨何能無慨然?”\end{note}徒噭噭而何爲耶?
        \begin{note}庚雙夾:《莊子》:“噭噭然隨而哭之。”\end{note}

        \begin{pl}卿偃然而長寢兮,豈天運之變於斯耶?\end{pl}
        \begin{note}庚雙夾:《莊子》:“偃然寢於巨室。”謂人死也。又“變而有氣,氣變而有形,形變而有生,今又變而之死,是相與爲春秋冬夏四時行也。”“其生也天行,其死也物化。”\end{note}

        \begin{pl}既窀穸且安穩兮,反其真而復奚化耶?\end{pl}
        \begin{note}庚雙夾:《左傳》:“窀穸之事,墓穴幽堂也。”《莊子•大宗師》:“而已反其真。”以死爲真。“方將不化,惡知已化哉?”言人死猶如化去。\end{note}

        \begin{pl}瞻雲氣而凝盼兮,彷彿有所覘耶?\end{pl}

        \begin{pl}俯窈窕而屬耳兮,恍惚有所聞耶?\end{pl}

        \begin{pl}期汗漫而無夭閾兮,忍捐棄餘於塵埃耶?\end{pl}
        \begin{note}庚雙行夾批:《逍遙遊》:“夭閼。上也。”\end{note}

        \begin{pl}倩風廉之爲餘驅車兮,冀聯轡而攜歸耶?\end{pl}

        \begin{pl}餘中心爲之慨然兮,\end{pl}
        \begin{note}庚雙行夾批:《莊子•至樂篇》:“我獨何能無慨然?”\end{note}徒噭噭而何爲耶?
        \begin{note}庚雙行夾批:《莊子》:“噭噭然隨而哭之。”\end{note}

        \begin{pl}卿偃然而長寢兮,豈天運之變於斯耶?\end{pl}
        \begin{note}庚雙行夾批:《莊子》:“偃然寢於巨室。”謂人死也。又“變而有氣,氣變而有形,形變而有生,今又變而之死,是相與爲春秋冬夏四時行也。”“其生也天行,其死也物化。”\end{note}

        \begin{pl}既窀穸且安穩兮,反其真而復奚化耶?\end{pl}
        \begin{note}庚雙行夾批:《左傳》:“窀穸之事,墓穴幽堂也。”《莊子•大宗師》:“而已反其真。”以死爲真。“方將不化,惡知已化哉?”言人死猶如化去。\end{note}

        \begin{pl}餘猶桎梏而懸附兮,靈格餘以嗟來耶?\end{pl}
        \begin{note}庚雙行夾批:《莊子•大宗師》:“彼以生爲附 感 疣,以死爲決環潰癰。”“嗟來 戶乎!嗟來 戶乎!”桑戶,人名。孟子反、子琴張二人招其魂而語之也。\end{note}

        \begin{pl}來兮止兮,君其來耶!\end{pl}
    \end{poem}

    \begin{parag}
        若夫鴻蒙而居,寂靜以處,雖臨於茲,餘亦莫睹。搴煙蘿而爲步幛,列槍蒲而森行伍。警柳眼之貪眠,釋蓮心之味苦。素女約於桂巖,宓妃迎於蘭渚。弄玉吹笙,寒簧擊敔。徵嵩嶽之妃,啓驪山之姥。瓿事迤之靈,獸作咸池之舞。潛赤水兮龍吟,集珠林兮鳳翥。爰格爰誠,匪簠匪筥。發軔乎霞城,返旌乎玄圃。既顯微而若通,復氤氳而倏阻。離合兮煙雲,空濛兮霧雨。塵霾斂兮星高,溪山麗兮月午。何心意之忡忡,若寤寐之栩栩。餘乃欷歔悵望,泣涕傍徨。人語兮寂歷,天籟兮篔簹。鳥驚散而飛,魚唼喋以響。誌哀兮是禱,成禮兮期祥。嗚呼哀哉!尚饗!
    \end{parag}

\end{qute2sp}

\begin{parag}
    讀畢,遂焚帛奠茗,猶依依不捨。小鬟催至再四,方纔回身。忽聽山石之後有一人笑道:“且請留步。”二人聽了,不免一驚。那小鬟回頭一看,卻是個人影從芙蓉花中走出來,他便大叫:“不好,有鬼。晴雯真來顯魂了!”唬得寶玉也忙看時,──且聽下回分解。
\end{parag}


\begin{parag}
    \begin{note}蒙回末總:前文入一院,必敘一番養竹種花,爲諸婆爭利渲染。此文入一院,必敘一番樹枯香老,爲親眷凋零悽楚。字字實境,字字奇情,令我把玩不釋。\end{note}
\end{parag}


\begin{parag}
    \begin{note}蒙回末總:《姽嫿詞》一段,與前後文似斷似連,如羅浮二山,煙雨爲連合,時有精氣來往。\end{note}
\end{parag}
