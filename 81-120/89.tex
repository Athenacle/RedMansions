\chap{八十九}{人亡物在公子填詞 蛇影杯弓顰卿絕粒}


\begin{parag}
    卻說鳳姐正自起來納悶,忽聽見小丫頭這話,又唬了一跳,連忙問道:“什麼官事?”小丫頭道:“也不知道。剛纔二門上小廝回進來,回老爺有要緊的官事,所以太太叫我請二爺來了。”鳳姐聽是工部裏的事,才把心略略的放下,因說道:“你回去回太太,就說二爺昨日晚上出城有事,沒有回來。打發人先回珍大爺去罷。”那丫頭答應着去了。
\end{parag}


\begin{parag}
    一時賈珍過來見了部裏的人,問明瞭,進來見了王夫人,回道:“部中來報,昨日總河奏到河南一帶決了河口,湮沒了幾府州縣。又要開銷國帑,修理城工。工部司官又有一番照料,所以部裏特來報知老爺的。”說完退出,及賈政回家來回明。從此直到冬間,賈政天天有事,常在衙門裏。寶玉的工課也漸漸鬆了,只是怕賈政覺察出來,不敢不常在學房裏去唸書,連黛玉處也不敢常去。
\end{parag}


\begin{parag}
    那時已到十月中旬,寶玉起來要往學房中去。這日天氣陡寒,只見襲人早已打點出一包衣服,向寶玉道:“今日天氣很冷,早晚寧使暖些。”說着,把衣服拿出來給寶玉挑了一件穿。又包了一件,叫小丫頭拿出交給焙茗,囑咐道:“天氣涼,二爺要換時,好生預備着。”焙茗答應了,抱着氈包,跟着寶玉自去。寶玉到了學房中,做了自己的工課,忽聽得紙窗呼喇喇一派風聲。代儒道:“天氣又發冷。”把風門推開一看,只見西北上一層層的黑雲漸漸往東南撲上來。焙茗走進來回寶玉道:“二爺,天氣冷了,再添些衣服罷。”寶玉點點頭兒。只見焙茗拿進一件衣服來,寶玉不看則已,看了時神已癡了。那些小學生都巴着眼瞧,卻原是晴雯所補的那件雀金裘。寶玉道:“怎麼拿這一件來!是誰給你的?”焙茗道:“是裏頭姑娘們包出來的。”寶玉道:“我身上不大冷,且不穿呢,包上罷。”代儒只當寶玉可惜這件衣服,卻也心裏喜他知道儉省。焙茗道:“二爺穿上罷,着了涼,又是奴才的不是了。二爺只當疼奴才罷。”寶玉無奈,只得穿上,呆呆的對著書坐着。代儒也只當他看書,不甚理會。晚間放學時,寶玉便往代儒託病告假一天。代儒本來上年紀的人,也不過伴着幾個孩子解悶兒,時常也八病九痛的,樂得去一個少操一日心。況且明知賈政事忙,賈母溺愛,便點點頭兒。
\end{parag}


\begin{parag}
    寶玉一徑回來,見過賈母王夫人,也是這樣說,自然沒有不信的,略坐一坐便回園中去了。見了襲人等,也不似往日有說有笑的,便和衣躺在炕上。襲人道:“晚飯預備下了,這會兒喫還是等一等兒?”寶玉道:“我不吃了,心裏不舒服。你們喫去罷。”襲人道:“那麼着你也該把這件衣服換下來了,那個東西那裏禁得住揉搓。”寶玉道:“不用換。”襲人道:“倒也不但是嬌嫩物兒,你瞧瞧那上頭的針線也不該這麼糟蹋他呀。”寶玉聽了這話,正碰在他心坎兒上,嘆了一口氣道:“那麼着,你就收拾起來給我包好了,我也總不穿他了。”說着,站起來脫下。襲人才過來接時,寶玉已經自己迭起。襲人道:“二爺怎麼今日這樣勤謹起來了?”寶玉也不答言,迭好了,便問:“包這個的包袱呢?”麝月連忙遞過來,讓他自己包好,回頭卻和襲人擠着眼兒笑。寶玉也不理會,自己坐着,無精打彩,猛聽架上鐘響,自己低頭看了看錶,針已指到酉初二刻了。一時小丫頭點上燈來。襲人道:“你不喫飯,喝一口粥兒罷。別淨餓着,看仔細餓上虛火來,那又是我們的累贅了。”寶玉搖搖頭兒,說:“不大餓,強吃了倒不受用。”襲人道:“既這麼着,就索性早些歇着罷。”於是襲人麝月鋪設好了,寶玉也就歇下,翻來覆去只睡不着,將及黎明,反朦朧睡去,不一頓飯時,早又醒了。
\end{parag}


\begin{parag}
    此時襲人麝月也都起來。襲人道:“昨夜聽着你翻騰到五更多,我也不敢問你。後來我就睡着了,不知到底你睡着了沒有?”寶玉道:“也睡了一睡,不知怎麼就醒了。”襲人道:“你沒有什麼不受用?”寶玉道:“沒有,只是心上發煩。”襲人道:“今日學房裏去不去?”寶玉道:“我昨兒已經告了一天假了,今兒我要想園裏逛一天,散散心,只是怕冷。你叫他們收拾一間房子,備下一爐香,擱下紙墨筆硯。你們只管幹你們的,我自己靜坐半天才好。別叫他們來攪我。”麝月接着道:“二爺要靜靜兒的用工夫,誰敢來攪。”襲人道:“這麼着很好,也省得着了涼。自己坐坐,心神也不散。”因又問:“你既懶待喫飯,今日喫什麼?早說好傳給廚房裏去。”寶玉道:“還是隨便罷,不必鬧的大驚小怪的。倒是要幾個果子擱在那屋裏,借點果子香。”襲人道:“那個屋裏好?別的都不大幹淨,只有晴雯起先住的那一間,因一向無人,還乾淨,就是清冷些。”寶玉道:“不妨,把火盆挪過去就是了。”襲人答應了。正說着,只見一個小丫頭端了一個茶盤兒,一個碗,一雙牙箸,遞給麝月道:“這是剛纔花姑娘要的,廚房裏老婆子送了來了。”麝月接了一看,卻是一碗燕窩湯,便問襲人道:“這是姐姐要的麼?”襲人笑道:“昨夜二爺沒喫飯,又翻騰了一夜,想來今日早起心裏必是發空的,所以我告訴小丫頭們叫廚房裏作了這個來的。”襲人一面叫小丫頭放桌兒,麝月打發寶玉喝了,漱了口。只見秋紋走來說道:“那屋裏已經收拾妥了,但等着一時炭勁過了,二爺再進去罷。”寶玉點頭,只是一腔心事,懶怠說話。一時小丫頭來請,說筆硯都安放妥當了。寶玉道:“知道了。”又一個小丫頭回道:“早飯得了。二爺在那裏喫?”寶玉道:“就拿了來罷,不必累贅了。”小丫頭答應了自去。一時端上飯來,寶玉笑了一笑,向襲人麝月道:“我心裏悶得很,自己喫只怕又喫不下去,不如你們兩個同我一塊兒喫,或者喫的香甜,我也多喫些。”麝月笑道:“這是二爺的高興,我們可不敢。”襲人道:“其實也使得,我們一處喝酒,也不止今日。只是偶然替你解悶兒還使得,若認真這樣,還有什麼規矩體統呢。”說着三人坐下。寶玉在上首,襲人麝月兩個打橫陪着。吃了飯,小丫頭端上漱口茶,兩個看着撤了下去。寶玉因端着茶,默默如有所思,又坐了一坐,便問道:“那屋裏收拾妥了麼?”麝月道:“頭裏就回過了,這回子又問。”
\end{parag}


\begin{parag}
    寶玉略坐了一坐,便過這間屋子來,親自點了一炷香,擺上些果品,便叫人出去,關上了門。外面襲人等都靜悄無聲。寶玉拿了一幅泥金角花的粉紅箋出來,口中祝了幾句,便提起筆來寫道:
\end{parag}


\begin{qute2sp}
    怡紅主人焚付晴姐知之,酌茗清香,庶幾來饗。其詞雲:隨身伴,獨自意綢繆。誰料風波平地起,頓教軀命實時休。孰與話輕柔?東逝水,無復向西流。想象更無懷夢草,添衣還見翠雲裘。脈脈使人愁!
\end{qute2sp}


\begin{parag}
    寫畢,就在香上點個火焚化了。靜靜兒等着,直待一炷香點盡了,纔開門出來。襲人道:“怎麼出來了?想來又悶的慌了。”
\end{parag}

\begin{parag}
    寶玉笑了一笑,假說道:“我原是心裏煩,才找個地方兒靜坐坐兒。這會子好了,還要外頭走走去呢。”說着,一徑出來,到了瀟湘館中,在院裏問道:“林妹妹在家裏呢麼?”紫鵑接應道:“是誰?”掀簾看時,笑道:“原來是寶二爺。姑娘在屋裏呢,請二爺到屋裏坐着。”寶玉同着紫鵑走進來。黛玉卻在裏間呢,說道:“紫鵑,請二爺屋裏坐罷。”寶玉走到裏間門口,看見新寫的一付紫墨色泥金雲龍箋的小對,上寫着:“綠窗明月在,青史古人空。”寶玉看了,笑了一笑,走入門去,笑問道:“妹妹做什麼呢?”黛玉站起來迎了兩步,笑着讓道:“請坐。我在這裏寫經,只剩得兩行了,等寫完了再說話兒。”因叫雪雁倒茶。寶玉道:“你別動,只管寫。”說着,一面看見中間掛着一幅單條,上面畫着一個嫦娥,帶着一個侍者,又一個女仙,也有一個侍者,捧着一個長長兒的衣囊似的,二人身邊略有些雲護,別無點綴,全仿李龍眠白描筆意,上有”鬪寒圖”三字,用八分書寫着。寶玉道:“妹妹這幅《鬪寒圖》可是新掛上的?”黛玉道:“可不是。昨日他們收拾屋子,我想起來,拿出來叫他們掛上的。”寶玉道:“是什麼出處?”黛玉笑道:“眼前熟的很的,還要問人。”寶玉笑道:“我一時想不起,妹妹告訴我罷。”黛玉道:“豈不聞‘青女素娥俱耐冷,月中霜裏鬪嬋娟’。”寶玉道:“是啊。這個實在新奇雅緻,卻好此時拿出來掛。”說着,又東瞧瞧,西走走。
\end{parag}


\begin{parag}
    雪雁沏了茶來,寶玉喫着。又等了一會子,黛玉經才寫完,站起來道:“簡慢了。”寶玉笑道:“妹妹還是這麼客氣。”但見黛玉身上穿着月白繡花小毛皮襖,加上銀鼠坎肩,頭上挽着隨常雲髻,簪上一枝赤金匾簪,別無花朵,腰下繫着楊妃色繡花綿裙。真比如:
\end{parag}


\begin{qute2sp}
    亭亭玉樹臨風立,冉冉香蓮帶露開。寶玉因問道:“妹妹這兩日彈琴來着沒有?”黛玉道:“兩日沒彈了。因爲寫字已經覺得手冷,那裏還去彈琴。”寶玉道:“不彈也罷了。我想琴雖是清高之品,卻不是好東西,從沒有彈琴裏彈出富貴壽考來的,只有彈出憂思怨亂來的。再者彈琴也得心裏記譜,未免費心。依我說,妹妹身子又單弱,不操這心也罷了。”黛玉抿着嘴兒笑。寶玉指着壁上道:“這張琴可就是麼?怎麼這麼短?”黛玉笑道:“這張琴不是短,因我小時學撫的時候別的琴都夠不着,因此特地做起來的。雖不是焦尾枯桐,這鶴山鳳尾還配得齊整,龍池雁足高下還相宜。你看這斷紋不是牛旄似的麼,所以音韻也還清越。”寶玉道:“妹妹這幾天來做詩沒有?”黛玉道:“自結社以後沒大作。”寶玉笑道:“你別瞞我,我聽見你吟的什麼‘不可惙,素心如何天上月’,你擱在琴裏覺得音響分外的響亮。有的沒有?”黛玉道:“你怎麼聽見了?”寶玉道:“我那一天從蓼風軒來聽見的,又恐怕打斷你的清韻,所以靜聽了一會就走了。我正要問你:前路是平韻,到末了兒忽轉了仄韻,是個什麼意思?”黛玉道:“這是人心自然之音,做到那裏就到那裏,原沒有一定的。”寶玉道:“原來如此。可惜我不知音,枉聽了一會子。”黛玉道:“古來知音人能有幾個?”寶玉聽了。又覺得出言冒失了,又怕寒了黛玉的心,坐了一坐,心裏象有許多話,卻再無可講的。黛玉因方纔的話也是衝口而出,此時回想,覺得太冷淡些,也就無話。寶玉一發打量黛玉設疑,遂訕訕的站起來說道:“妹妹坐着罷。我還要到三妹妹那裏瞧瞧去呢。”黛玉道:“你若是見了三妹妹,替我問候一聲罷。”寶玉答應着便出來了。
\end{qute2sp}


\begin{parag}
    黛玉送至屋門口,自己回來悶悶的坐着,心裏想道:“寶玉近來說話半吐半吞,忽冷忽熱,也不知他是什麼意思。”正想着,紫鵑走來道:“姑娘,經不寫了?我把筆硯都收好了?”黛玉道:“不寫了,收起去罷。”說着,自己走到裏間屋裏牀上歪着,慢慢的細想。紫鵑進來問道:“姑娘喝碗茶罷?”黛玉道:“不喝呢。我略歪歪兒,你們自己去罷。”
\end{parag}


\begin{parag}
    紫鵑答應着出來,只見雪雁一個人在那裏發呆。紫鵑走到他跟前問道:“你這會子也有了什麼心事了麼?”雪雁只顧發呆,倒被他唬了一跳,因說道:“你別嚷,今日我聽見了一句話,我告訴你聽,奇不奇。你可別言語。”說着,往屋裏努嘴兒。因自己先行,點着頭兒叫紫鵑同他出來,到門外平臺底下,悄悄兒的道:“姐姐你聽見了麼?寶玉定了親了!”紫鵑聽見,唬了一跳,說道:“這是那裏來的話?只怕不真罷。”雪雁道:“怎麼不真,別人大概都知道,就只咱們沒聽見。”紫鵑道:“你是那裏聽來的?”雪雁道:“我聽見侍書說的,是個什麼知府家,家資也好,人才也好。”紫鵑正聽時,只聽得黛玉咳嗽了一聲,似乎起來的光景。紫鵑恐怕他出來聽見,便拉了雪雁搖搖手兒,往裏望望,不見動靜,才又悄悄兒的問道:“他到底怎麼說來?”雪雁道:“前兒不是叫我到三姑娘那裏去道謝嗎,三姑娘不在屋裏,只有侍書在那裏。大家坐着,無意中說起寶二爺的淘氣來,他說寶二爺怎麼好,只會頑兒,全不象大人的樣子,已經說親了,還是這麼呆頭呆腦。我問他定了沒有,他說是定了,是個什麼王大爺做媒的。那王大爺是東府裏的親戚,所以也不用打聽,一說就成了。”紫鵑側着頭想了一想,“這句話奇!”又問道:“怎麼家裏沒有人說起?”雪雁道:“侍書也說的是老太太的意思。若一說起,恐怕寶玉野了心,所以都不提起。侍書告訴了我,又叮囑千萬不可露風,說出來只道是我多嘴。”把手往裏一指,“所以他面前也不提。今日是你問起,我不犯瞞你。”正說到這裏,只聽鸚鵡叫喚,學着說:“姑娘回來了,快倒茶來!”倒把紫鵑雪雁嚇了一跳,回頭並不見有人,便罵了鸚鵡一聲,走進屋內。只見黛玉喘吁吁的剛坐在椅子上,紫鵑搭訕着問茶問水。黛玉問道:“你們兩個那裏去了?再叫不出一個人來。”說着便走到炕邊,將身子一歪,仍舊倒在炕上,往裏躺下,叫把帳子撩下。紫鵑雪雁答應出去。他兩個心裏疑惑方纔的話只怕被他聽了去了,只好大家不提。誰知黛玉一腔心事,又竊聽了紫鵑雪雁的話,雖不很明白,已聽得了七八分,如同將身撂在大海里一般。思前想後,竟應了前日夢中之讖,千愁萬恨,堆上心來。左右打算,不如早些死了,免得眼見了意外的事情,那時反倒無趣。又想到自己沒了爹孃的苦,自今以後,把身子一天一天的糟踏起來,一年半載,少不得身登清淨。打定了主意,被也不蓋,衣也不添,竟是閤眼裝睡。紫鵑和雪雁來伺候幾次,不見動靜,又不好叫喚。晚飯都不喫。點燈已後,紫鵑掀開帳子,見已睡着了,被窩都蹬在腳後。怕他着了涼,輕輕兒拿來蓋上。黛玉也不動,單待他出去,仍然褪下。那紫鵑只管問雪雁:“今兒的話到底是真的是假的?”雪雁道:“怎麼不真。”紫鵑道:“侍書怎麼知道的?”雪雁道:“是小紅那裏聽來的。”紫鵑道:“頭裏咱們說話,只怕姑娘聽見了,你看剛纔的神情,大有原故。今日以後,咱們倒別提這件事了。”說着,兩個人也收拾要睡。紫鵑進來看時,只見黛玉被窩又蹬下來,復又給他輕輕蓋上。一宿晚景不提。
\end{parag}


\begin{parag}
    次日,黛玉清早起來,也不叫人,獨自一個呆呆的坐着。紫鵑醒來,看見黛玉已起,便驚問道:“姑娘怎麼這麼早?”黛玉道:“可不是,睡得早,所以醒得早。”紫鵑連忙起來,叫醒雪雁,伺候梳洗。那黛玉對着鏡子,只管呆呆的自看。看了一回,那淚珠兒斷斷連連,早已溼透了羅帕。正是:
\end{parag}


\begin{poem}
    \begin{pl}
        瘦影正臨春水照,卿須憐我我憐卿。
    \end{pl}
\end{poem}


\begin{parag}
    紫鵑在旁也不敢勸,只怕倒把閒話勾引舊恨來。遲了好一會,黛玉才隨便梳洗了,那眼中淚漬終是不幹。又自坐了一會,叫紫鵑道:“你把藏香點上。”紫鵑道:“姑娘,你睡也沒睡得幾時,如何點香?不是要寫經?”黛玉點點頭兒。紫鵑道:“姑娘今日醒得太早,這會子又寫經,只怕太勞神了罷。”黛玉道:“不怕,早完了早好。況且我也並不是爲經,倒藉着寫字解解悶兒。以後你們見了我的字跡,就算見了我的面兒了。”說着,那淚直流下來。紫鵑聽了這話,不但不能再勸,連自己也掌不住滴下淚來。原來黛玉立定主意,自此已後,有意糟踏身子,茶飯無心,每日漸減下來。寶玉下學時,也常抽空問候,只是黛玉雖有萬千言語,自知年紀已大,又不便似小時可以柔情挑逗,所以滿腔心事,只是說不出來。寶玉欲將實言安慰,又恐黛玉生嗔,反添病症。兩個人見了面,只得用浮言勸慰,真真是親極反疏了。那黛玉雖有賈母王夫人等憐恤,不過請醫調治,只說黛玉常病,那裏知他的心病。紫鵑等雖知其意,也不敢說。從此一天一天的減,到半月之後,腸胃日薄,一日果然粥都不能吃了。黛玉日間聽見的話,都似寶玉娶親的話,看見怡紅院中的人,無論上下,也象寶玉娶親的光景。薛姨媽來看,黛玉不見寶釵,越發起疑心,索性不要人來看望,也不肯吃藥,只要速死。睡夢之中,常聽見有人叫寶二奶奶的。一片疑心,竟成蛇影。一日竟是絕粒,粥也不喝,懨懨一息,垂斃殆盡。未知黛玉性命如何,且看下回分解。
\end{parag}