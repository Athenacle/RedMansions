\chap{一百一十四}{王熙凤历幻返金陵 甄应嘉蒙恩还玉阙}



\begin{parag}
    却说宝玉宝钗听说凤姐病的危急,赶忙起来。丫头秉烛伺候。正要出院,只见王夫人那边打发人来说:“琏二奶奶不好了,还没有咽气,二爷二奶奶且慢些过去罢。琏二奶奶的病有些古怪,从三更天起到四更时候,琏二奶奶没有住嘴说些胡话,要船要轿的,说到金陵归入册子去。众人不懂,他只是哭哭喊喊的。琏二爷没有法儿,只得去糊了船轿,还没拿来,琏二奶奶喘着气等呢。叫我们过来说,等琏二奶奶去了再过去罢。”宝玉道:“这也奇,他到金陵做什么?”袭人轻轻的和宝玉说道:“你不是那年做梦,我还记得说有多少册子,不是琏二奶奶也到那里去么?”宝玉听了点头道:“是呀,可惜我都不记得那上头的话了。这么说起来,人都有个定数的了。但不知林妹妹又到那里去了?我如今被你一说,我有些懂得了。若再做这个梦时,我得细细的瞧一瞧,便有未卜先知的分儿了。”袭人道:“你这样的人可是不可和你说话的,偶然提了一句,你便认起真来了吗?就算你能先知了,你有什么法儿!”宝玉道:“只怕不能先知,若是能了,我也犯不着为你们瞎操心了。”
\end{parag}


\begin{parag}
    两个正说着,宝钗走来问道:“你们说什么?”宝玉恐他盘诘,只说:“我们谈论凤姐姐。”宝钗道:“人要死了,你们还只管议论人。旧年你还说我咒人,那个签不是应了么?”宝玉又想了一想,拍手道:“是的,是的。这么说起来,你倒能先知了。我索性问问你,你知道我将来怎么样?”宝钗笑道:“这是又胡闹起来了。我是就他求的签上的话混解的,你就认了真了。你就和邢妹妹一样的了,你失了玉,他去求妙玉扶乩,批出来的众人不解,他还背地里和我说妙玉怎么前知,怎么参禅悟道。如今他遭此大难,他如何自己都不知道,这可是算得前知吗?就是我偶然说着了二奶奶的事情,其实知道他是怎么样了,只怕我连我自己也不知道呢。这样下落可不是虚诞的事,是信得的么!”宝玉道:“别提他了。你只说邢妹妹罢,自从我们这里连连的有事,把他这件事竟忘记了。你们家这么一件大事怎么就草草的完了,也没请亲唤友的。”宝钗道:“你这话又是迂了。我们家的亲戚只有咱们这里和王家最近。王家没了什么正经人了。咱们家遭了老太太的大事,所以也没请,就是琏二哥张罗了张罗。别的亲戚虽也有一两门子,你没过去,如何知道。算起来我们这二嫂子的命和我差不多,好好的许了我二哥哥,我妈妈原想体体面面的给二哥哥娶这房亲事的。一则为我哥哥在监里,二哥哥也不肯大办,二则为咱家的事,三则为我二嫂子在大太太那边忒苦,又加着抄了家,大太太是苛刻一点的,他也实在难受:所以我和妈妈说了,便将将就就的娶了过去。我看二嫂子如今倒是安心乐意的孝敬我妈妈,比亲媳妇还强十倍呢。待二哥哥也是极尽妇道的,和香菱又甚好,二哥哥不在家,他两个和和气气的过日子。虽说是穷些,我妈妈近来倒安逸好些。就是想起我哥哥来不免悲伤。况且常打发人家里来要使用,多亏二哥哥在外头帐头儿上讨来应付他的。我听见说城里有几处房子已经典去,还剩了一所在那里,打算着搬去住。”宝玉道:“为什么要搬?住在这里你来去也便宜些,若搬远了,你去就要一天了。”宝钗道:“虽说是亲戚,倒底各自的稳便些。那里有个一辈子住在亲戚家的呢。”
\end{parag}


\begin{parag}
    宝玉还要讲出不搬去的理,王夫人打发人来说:“琏二奶奶咽了气了。所有的人多过去了,请二爷二奶奶就过去。”宝玉听了,也掌不住跺脚要哭。宝钗虽也悲戚,恐宝玉伤心,便说:“有在这里哭的,不如到那边哭去。”于是两人一直到凤姐那里。只见好些人围着哭呢。宝钗走到跟前,见凤姐已经停床,便大放悲声。宝玉也拉着贾琏的手大哭起来。贾琏也重新哭泣。平儿等因见无人劝解,只得含悲上来劝止了。众人都悲哀不止。贾琏此时手足无措,叫人传了赖大来,叫他办理丧事。自己回明了贾政去,然后行事。但是手头不济,诸事拮据,又想起凤姐素日来的好处,更加悲哭不已,又见巧姐哭的死去活来,越发伤心。哭到天明,即刻打发人去请他大舅子王仁过来。那王仁自从王子腾死后,王子胜又是无能的人,任他胡为,已闹的六亲不和。今知妹子死了,只得赶着过来哭了一场。见这里诸事将就,心下便不舒服,说:“我妹妹在你家辛辛苦苦当了好几年家,也没有什么错处,你们家该认真的发送发送才是。怎么这时候诸事还没有齐备!”贾琏本与王仁不睦,见他说些混账话,知他不懂的什么,也不大理他。王仁便叫了他外甥女儿巧姐过来说:“你娘在时,本来办事不周到,只知道一味的奉承老太太,把我们的人都不大看在眼里。外甥女儿,你也大了,看见我曾经沾染过你们没有!如今你娘死了,诸事要听着舅舅的话。你母亲娘家的亲戚就是我和你二舅舅了。你父亲的为人我也早知道的了,只有重别人,那年什么尤姨娘死了,我虽不在京,听见人说花了好些银子。如今你娘死了,你父亲倒是这样的将就办去吗!你也不快些劝劝你父亲。”巧姐道:“我父亲巴不得要好看,只是如今比不得从前了。现在手里没钱,所以诸事省些是有的。”王仁道:“你的东西还少么!”巧姐儿道:“旧年抄去,何尝还了呢。”王仁道:“你也这样说。我听见老太太又给了好些东西,你该拿出来。”巧姐又不好说父亲用去,只推不知道。王仁便道:“哦,我知道了,不过是你要留着做嫁妆罢咧。”巧姐听了,不敢回言,只气得哽噎难鸣的哭起来了。平儿生气说道:“舅老爷有话,等我们二爷进来再说,姑娘这么点年纪,他懂的什么。”王仁道:“你们是巴不得二奶奶死了,你们就好为王了。我并不要什么,好看些也是你们的脸面。”说着,赌气坐着。巧姐满怀的不舒服,心想:“我父亲并不是没情,我妈妈在时舅舅不知拿了多少东西去,如今说得这样干净。”于是便不大瞧得起他舅舅了。岂知王仁心里想来,他妹妹不知攒积了多少,虽说抄了家,那屋里的银子还怕少吗。”必是怕我来缠他们,所以也帮着这么说,这小东西儿也是不中用的。”从此王仁也嫌了巧姐儿了。
\end{parag}


\begin{parag}
    贾琏并不知道,只忙着弄银钱使用。外头的大事叫赖大办了,里头也要用好些钱,一时实在不能张罗。平儿知他着急,便叫贾琏道:“二爷也别过于伤了自己的身子。”贾琏道:“什么身子,现在日用的钱都没有,这件事怎么办!偏有个糊涂行子又在这里蛮缠,你想有什么法儿!”平儿道:“二爷也不用着急,若说没钱使唤,我还有些东西旧年幸亏没有抄去,在里头。二爷要就拿去当着使唤罢。”贾琏听了,心想难得这样,便笑道:“这样更好,省得我各处张罗。等我银子弄到手了还你。”平儿道:“我的也是奶奶给的,什么还不还,只要这件事办的好看些就是了。”贾琏心里倒着实感激他,便将平儿的东西拿了去当钱使用,诸凡事情便与平儿商量。秋桐看着心里就有些不甘,每每口角里头便说:“平儿没有了奶奶,他要上去了。我是老爷的人,他怎么就越过我去了呢。”平儿也看出来了,只不理他。倒是贾琏一时明白,越发把秋桐嫌了,一时有些烦恼便拿着秋桐出气。邢夫人知道,反说贾琏不好。贾琏忍气。不题。
\end{parag}


\begin{parag}
    再说凤姐停了十余天,送了殡。贾政守着老太太的孝,总在外书房。那时清客相公渐渐的都辞去了,只有个程日兴还在那里,时常陪着说说话儿。提起“家运不好,一连人口死了好些,大老爷和珍大爷又在外头,家计一天难似一天。外头东庄地亩也不知道怎么样,总不得了呀!”程日兴道:“我在这里好些年,也知道府上的人那一个不是肥己的。一年一年都往他家里拿,那自然府上是一年不够一年了。又添了大老爷珍大爷那边两处的费用,外头又有些债务,前儿又破了好些财,要想衙门里缉贼追赃是难事。老世翁若要安顿家事,除非传那些管事的来,派一个心腹的人各处去清查清查,该去的去,该留的留,有了亏空着在经手的身上赔补,这就有了数儿了。那一座大的园子人家是不敢买的。这里头的出息也不少,又不派人管了。那年老世翁不在家,这些人就弄神弄鬼儿的,闹的一个人不敢到园里。这都是家人的弊。此时把下人查一查,好的使着,不好的便撵了,这才是道理。”贾政点头道:“先生你所不知,不必说下人,便是自己的侄儿也靠不住。若要我查起来,那能一一亲见亲知。况我又在服中,不能照管这些了。我素来又兼不大理家,有的没的,我还摸不着呢。”程日兴道:“老世翁最是仁德的人,若在别家的,这样的家计,就穷起来,十年五载还不怕,便向这些管家的要也就够了。我听见世翁的家人还有做知县的呢。”贾政道:“一个人若要使起家人们的钱来,便了不得了,只好自己俭省些。但是册子上的产业,若是实有还好,生怕有名无实了。”程日兴道:“老世翁所见极是。晚生为什么说要查查呢!”贾政道:“先生必有所闻。”程日兴道:“我虽知道些那些管事的神通,晚生也不敢言语的。”贾政听了,便知话里有因,便叹道:“我自祖父以来都是仁厚的,从没有刻薄过下人。我看如今这些人一日不似一日了。在我手里行出主子样儿来,又叫人笑话。”
\end{parag}


\begin{parag}
    两人正说着,门上的进来回道:“江南甄老爷到来了。”贾政便问道:“甄老爷进京为什么?”那人道:“奴才也打听了,说是蒙圣恩起复了。”贾政道:“不用说了,快请罢。”那人出去请了进来。那甄老爷即是甄宝玉之父,名叫甄应嘉,表字友忠,也是金陵人氏,功勋之后。原与贾府有亲,素来走动的。因前年挂误革了职,动了家产。今遇主上眷念功臣,赐还世职,行取来京陛见。知道贾母新丧,特备祭礼择日到寄灵的地方拜奠,所以先来拜望。贾政有服不能远接,在外书房门口等着。那位甄老爷一见,便悲喜交集,因在制中不便行礼,便拉着了手叙了些阔别思念的话,然后分宾主坐下,献了茶,彼此又将别后事情的话说了。贾政问道:“老亲翁几时陛见的?”甄应嘉道:“前日。”贾政道:“主上隆恩,必有温谕。”甄应嘉道:“主上的恩典真是比天还高,下了好些旨意。”贾政道:“什么好旨意?”甄应嘉道:“近来越寇猖獗,海疆一带小民不安,派了安国公征剿贼寇。主上因我熟悉土疆,命我前往安抚,但是即日就要起身。昨日知老太太仙逝,谨备瓣香至灵前拜奠,稍尽微忱。”贾政即忙叩首拜谢,便说:“老亲翁即此一行,必是上慰圣心,下安黎庶,诚哉莫大之功,正在此行。但弟不克亲睹奇才,只好遥聆捷报。现在镇海统制是弟舍亲,会时务望青照。”甄应嘉道:“老亲翁与统制是什么亲戚?”贾政道:“弟那年在江西粮道任时,将小女许配与统制少君,结褵已经三载。因海口案内未清,继以海寇聚奸,所以音信不通。弟深念小女,俟老亲翁安抚事竣后,拜恳便中请为一视。弟即修数行烦尊纪带去,便感激不尽了。”甄应嘉道:“儿女之情,人所不免,我正在有奉托老亲翁的事。日蒙圣恩召取来京,因小儿年幼,家下乏人,将贱眷全带来京。我因钦限迅速,昼夜先行,贱眷在后缓行,到京尚需时日。弟奉旨出京,不敢久留。将来贱眷到京,少不得要到尊府,定叫小犬叩见。如可进教,遇有姻事可图之处,望乞留意为感。”贾政一一答应。那甄应嘉又说了几句话,就要起身,说:“明日在城外再见。”贾政见他事忙,谅难再坐,只得送出书房。
\end{parag}


\begin{parag}
    贾琏宝玉早已伺候在那里代送,因贾政未叫,不敢擅入。甄应嘉出来,两人上去请安。应嘉一见宝玉,呆了一呆,心想:“这个怎么甚象我家宝玉?只是浑身缟素。”因问:“至亲久阔,爷们都不认得了。”贾政忙指贾琏道:“这是家兄名赦之子琏二侄儿。”又指着宝玉道:“这是第二小犬,名叫宝玉。”应嘉拍手道奇:“我在家听见说老亲翁有个衔玉生的爱子,名叫宝玉。因与小儿同名,心中甚为罕异。后来想着这个也是常有的事,不在意了。岂知今日一见,不但面貌相同,且举止一般,这更奇了。”问起年纪,比这里的哥儿略小一岁。贾政便因提起承属包勇,问及令郎哥儿与小儿同名的话述了一遍。应嘉因属意宝玉,也不暇问及那包勇的得妥,只连连的称道:“真真罕异!”因又拉了宝玉的手,极致殷勤。又恐安国公起身甚速,急须预备长行,勉强分手徐行。贾琏宝玉送出,一路又问了宝玉好些的话。及至登车去后,贾琏宝玉回来见了贾政,便将应嘉问的话回了一遍。
\end{parag}


\begin{parag}
    贾政命他二人散去。贾琏又去张罗算明凤姐丧事的账目。宝玉回到自己房中,告诉了宝钗,说是:“常提的甄宝玉,我想一见不能,今日倒先见了他父亲了。我还听得说宝玉也不日要到京了,要来拜望我老爷呢。又人人说和我一模一样的,我只不信。若是他后儿到了咱们这里来,你们都去瞧去,看他果然和我象不象。”宝钗听了道:“嗳,你说话怎么越发不留神了,什么男人同你一样都说出来了,还叫我们瞧去吗!”宝玉听了,知是失言,脸上一红,连忙的还要解说。不知何话,下回分解。
\end{parag}