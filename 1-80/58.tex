\chap{五十八}{杏子陰假鳳泣虛凰 茜紗窗真情揆癡理}


\begin{parag}
    話說他三人因見探春等進來,忙將此話掩住不提。探春等問候過,大家說笑了一會方散。
\end{parag}


\begin{parag}
    誰知上回所表的那位老太妃已薨,凡誥命等皆入朝隨班按爵守制。敕諭天下:凡有爵之家,一年內不得筵宴音樂,庶民皆三月不得婚嫁。賈母、邢、王、尤、許婆媳祖孫等皆每日入朝隨祭,至未正以後方回。在大內偏宮二十一日後,方請靈入先陵,地名曰孝慈縣。\begin{note}庚雙夾:隨事命名。\end{note}這陵離都來往得十來日之功,如今請靈至此,還要停放數日,方入地宮,故得一月光景。\begin{note}庚雙夾:周到細膩之至。真細之至,不獨寫侯府得理,亦且將皇宮赫赫寫得令人不敢坐閱。\end{note}寧府賈珍夫妻二人,也少不得是要去的。兩府無人,因此大家計議,家中無主,便報了尤氏產育,將他騰挪出來,協理榮寧兩處事體。因又託了薛姨媽在園內照管他姊妹丫鬟。薛姨媽只得也挪進園來。因寶釵處有湘雲香菱;李紈處目今李嬸母女雖去,然有時亦來住三五日不定,賈母又將寶琴送與他去照管;迎春處有岫煙;探春因家務冗雜,且不時有趙姨娘與賈環來嘈聒,甚不方便;惜春處房屋狹小;況賈母又千叮嚀萬囑咐託他照管林黛玉,薛姨媽素習也最憐愛他的,今既巧遇這事,便挪至瀟湘館來和黛玉同房,一應藥餌飲食十分經心。黛玉感戴不盡,以後便亦如寶釵之呼,連寶釵前亦直以姐姐呼之,寶琴前直以妹妹呼之,儼似同胞共出,較諸人更似親切。賈母見如此,也十分喜悅放心。薛姨媽只不過照管他姊妹,禁約得丫頭輩,一應家中大小事務也不肯多口。尤氏雖天天過來,也不過應名點卯,亦不肯亂作威福,且他家內上下也只剩他一個料理,再者每日還要照管賈母王夫人的下處一應所需飲饌鋪設之物,所以也甚操勞。
\end{parag}


\begin{parag}
    當下榮寧兩處主人既如此不暇,並兩處執事人等,或有人跟隨入朝的,或有朝外照理下處事務的,又有先跴踏下處的,也都各各忙亂。因此兩處下人無了正經頭緒,也都偷安,或乘隙結黨,與權暫執事者竊弄威福。榮府只留得賴大並幾個管事照管外務。這賴大手下常用幾個人已去,雖另委人,都是些生的,只覺不順手。且他們無知,或賺騙無節,或呈告無據,或舉薦無因,種種不善,在在生事,也難備述。
\end{parag}


\begin{parag}
    又見各官宦家,凡養優伶男女者,一概蠲免遣發,尤氏等便議定,待王夫人回家回明,也欲遣發十二個女孩子,又說:“這些人原是買的,如今雖不學唱,儘可留著使喚,令其教習們自去也罷了。”王夫人因說:“這學戲的倒比不得使喚的,他們也是好人家的兒女,因無能賣了做這事,裝醜弄鬼的幾年。如今有這機會,不如給他們幾兩銀子盤纏,各自去罷。當日祖宗手裏都是有這例的。咱們如今損陰壞德,而且還小器。如今雖有幾個老的還在,那是他們各有原故,不肯回去的,所以才留下使喚,大了配了咱們家的小廝們了。”尤氏道:“如今我們也去問他十二個,有願意回去的,就帶了信兒,叫上父母來親自來領回去,給他們幾兩銀子盤纏方妥。若不叫上他父母親人來,只怕有混賬人頂名冒領出去又轉賣了,豈不辜負了這恩典。若有不願意回去的,就留下。”王夫人笑道:“這話妥當。”尤氏等又遣人告訴了鳳姐兒。\begin{note}庚雙夾:看他任意鄙俚詼諧之中必有一個“禮”字還清,足是大家光景。\end{note}一面說與總理房中,每教習給銀八兩,令其自便。凡梨香院一應物件,查清註冊收明,派人上夜。將十二個女孩子叫來面問,倒有一多半不願意回家的:也有說父母雖有,他只以賣我們爲事,這一去還被他賣了;也有父母已亡,或被叔伯兄弟所賣的;也有說無人可投的;也有說戀恩不捨的。所願去者止四五人。王夫人聽了,只得留下。將去者四五人皆令其乾孃領回家去,單等他親父母來領;將不願去者分散在園中使喚。賈母便留下文官自使,將正旦芳官指與寶玉,將小旦蕊官送了寶釵,將小生藕官指與了黛玉,將大花面葵官送了湘雲,將小花面豆官送了寶琴,將老外艾官送了探春,尤氏便討了老旦茄官去。當下各得其所,就如倦鳥出籠,每日園中游戲。衆人皆知他們不能針黹,不慣使用,皆不大責備。其中或有一二個知事的,愁將來無應時之技,亦將本技丟開,便學起針黹紡績女工諸務。
\end{parag}


\begin{parag}
    一日正是朝中大祭,賈母等五更便去了,先到下處用些點心小食,然後入朝。早膳已畢,方退至下處,用過早飯,略歇片刻,復入朝待中晚二祭完畢,方出至下處歇息,用過晚飯方回家。可巧這下處乃是一個大官的家廟,乃比丘尼焚修,房舍極多極淨。東西二院,榮府便賃了東院,北靜王府便賃了西院。太妃少妃每日宴息,見賈母等在東院,彼此同出同入,都有照應。外面細事不消細述。
\end{parag}


\begin{parag}
    且說大觀園中因賈母王夫人天天不在家內,又送靈去一月方回,各丫鬟婆子皆有閒空,多在園內遊玩。更又將梨香院內伏侍的衆婆子一概撤回,並散在園內聽使,更覺園內人多了幾十個。因文官等一干人或心性高傲,或倚勢凌下,或揀衣挑食,或口角鋒芒,大概不安分守理者多。因此衆婆子無不含怨,只是口中不敢與他們分證。如今散了學,大家稱了願,也有丟開手的,也有心地狹窄猶懷舊怨的,因將衆人皆分在各房名下,不敢來廝侵。
\end{parag}


\begin{parag}
    可巧這日乃是清明之日,賈璉已備下年例祭祀,帶領賈環、賈琮、賈蘭三人去往鐵檻寺祭柩燒紙。寧府賈蓉也同族中幾人各辦祭祀前往。因寶玉未大愈,故不曾去得。飯後發倦,襲人因說:“天氣甚好,你且出去逛逛,省得丟下粥碗就睡,存在心裏。”寶玉聽說,只得拄了一支杖,靸著鞋,步出院外。\begin{note}庚雙夾:畫出病勢。\end{note}因近日將園中分與衆婆子料理,各司各業,皆在忙時,也有修竹的,也有□樹的,也有栽花的,也有種豆的,池中又有駕娘們行著船夾泥種藕。香菱、湘雲、寶琴與丫鬟等都坐在山石上,瞧他們取樂。寶玉也慢慢行來。湘雲見了他來,忙笑說:“快把這船打出去,他們是接林妹妹的。”衆人都笑起來。寶玉紅了臉,也笑道:“人家的病,誰是好意的,你也形容著取笑兒。”湘雲笑道:“病也比人家另一樣,原招笑兒,反說起人來。”說著,寶玉便也坐下,看著衆人忙亂了一回。湘雲因說:“這裏有風,石頭上又冷,坐坐去罷。”
\end{parag}


\begin{parag}
    寶玉便也正要去瞧林黛玉,便起身拄拐辭了他們,從沁芳橋一帶堤上走來。只見柳垂金線,桃吐丹霞,山石之後,一株大杏樹,花已全落,葉稠陰翠,上面已結了豆子大小的許多小杏。寶玉因想道:“能病了幾天,竟把杏花辜負了!不覺倒‘綠葉成蔭子滿枝’了!”因此仰望杏子不捨。又想起邢岫煙已擇了夫婿一事,雖說是男女大事,不可不行,但未免又少了一個好女兒。不過兩年,便也要“綠葉成蔭子滿枝”了。再過幾日,這杏樹子落枝空,再幾年,岫煙未免烏髮如銀,紅顏似槁了,因此不免傷心,只管對杏流淚嘆息。\begin{note}庚雙夾:近之淫書滿紙傷春,究竟不知傷春原委。看他並不提“傷春”字樣,卻豔恨穠愁香流滿紙矣。\end{note}正悲嘆時,忽有一個雀兒飛來,落於枝上亂啼。寶玉又發了呆性,心下想道:“這雀兒必定是杏花正開時他曾來過,今見無花空有子葉,故也亂啼。這聲韻必是啼哭之聲,可恨公冶長不在眼前,不能問他。但不知明年再發時,這個雀兒可還記得飛到這裏來與杏花一會了?”
\end{parag}


\begin{parag}
    正胡思間,忽見一股火光從山石那邊發出,將雀兒驚飛。寶玉喫一大驚,又聽那邊有人喊道:“藕官,你要死,怎弄些紙錢進來燒?我回去回奶奶們去,仔細你的肉!”寶玉聽了,益發疑惑起來,忙轉過山石看時,只見藕官滿面淚痕,蹲在那裏,手裏還拿著火,守著些紙錢灰作悲。寶玉忙問道:“你與誰燒紙錢?快不要在這裏燒。你或是爲父母兄弟,你告訴我姓名,外頭去叫小廝們打了包袱寫上名姓去燒。”藕官見了寶玉,只不作一聲。寶玉數問不答,忽見一婆子惡恨恨走來拉藕官,口內說道:“我已經回了奶奶們了,奶奶氣的了不得。”藕官聽了,終是孩氣,怕辱沒了沒臉,便不肯去。婆子道:“我說你們別太興頭過餘了,如今還比你們在外頭隨心亂鬧呢。這是尺寸地方兒。”指寶玉道:“連我們的爺還守規矩呢,你是什麼阿物兒,跑來胡鬧。怕也不中用,跟我快走罷!”\begin{note}庚雙夾:如何?必是含怨之人,又拉上寶玉,畫出小人得意來。\end{note}寶玉忙道:“他並沒燒紙錢,原是林妹妹叫他來燒那爛字紙的。你沒看真,反錯告了他。”藕官正沒了主意,見了寶玉,也正添了畏懼,忽聽他反掩飾,心內轉憂成喜,也便硬著口說道:“你很看真是紙錢了麼?我燒的是林姑娘寫壞了的字紙!”那婆子聽如此,亦發狠起來,便彎腰向紙灰中揀那不曾化盡的遺紙,揀了兩點在手內,說道:“你還嘴硬,有據有證在這裏。我只和你廳上講去!”說著,拉了袖子,就拽著要走。寶玉忙把藕官拉住,用杖敲開那婆子的手,說道:“你只管拿了那個回去。實告訴你:我夜作了一個夢,夢見杏花神和我要一掛白紙錢,不可叫本房人燒,要一個生人替我燒了,我的病就好的快。所以我請了白錢,巴巴兒的和林姑娘煩了他來,替我燒了祝讚。原不許一個人知道的,所以我今日才能起來,偏你看見了。我這會子又不好了,都是你衝了!你還要告他去。藕官,只管去,見了他們你就照依我這話說。等老太太回來,我就說他故意來衝神祗,保佑我早死。”藕官聽了益發得了主意,反倒拉著婆子要走。那婆子聽了這話,忙丟下紙錢,陪笑央告寶玉道:“我原不知道,二爺若回了老太太,我這老婆子豈不完了?我如今回奶奶們去,就說是爺祭神,我看錯了。”寶玉道:“你也不許再回去了,我便不說。”婆子道:“我已經回了,叫我來帶他,我怎好不回去的。也罷,就說我已經叫到了他,林姑娘叫了去了。”寶玉想了一想,方點頭應允。那婆子只得去了。
\end{parag}


\begin{parag}
    這裏寶玉問他:“到底是爲誰燒紙?我想來若是爲父母兄弟,你們皆煩人外頭燒過了,這裏燒這幾張,必有私自的情理。”藕官因方纔護庇之情感激於衷,便知他是自已一流的人物,便含淚說道:“我這事,除了你屋裏的芳官並寶姑娘的蕊官,並沒第三個人知道。今日被你遇見,又有這段意思,少不得也告訴了你,只不許再對人言講。”又哭道:“我也不便和你面說,你只回去揹人悄問芳官就知道了。”說畢,佯常而去。
\end{parag}


\begin{parag}
    寶玉聽了,心下納悶,\begin{note}庚雙夾:連觀書者亦納悶。\end{note}只得踱到瀟湘館,瞧黛玉益發瘦的可憐,問起來,比往日已算大愈了。\begin{note}庚雙夾:好,若只管病亦不好。\end{note}黛玉見他也比先大瘦了,想起往日之事,不免流下淚來,些微談了談,便催寶玉去歇息調養。寶玉只得回來。因記掛著要問芳官那原委,偏有湘雲香菱來了,正和襲人芳官說笑,不好叫他,恐人又盤詰,只得耐著。
\end{parag}


\begin{parag}
    一時芳官又跟了他乾孃去洗頭。他乾孃偏又先叫了他親女兒洗過了後,才叫芳官洗。芳官見了這般,便說他偏心,“把你女兒剩水給我洗。我一個月的月錢都是你拿著,沾我的光不算,反倒給我剩東剩西的。”他乾孃羞愧變成惱,便罵他:“不識抬舉的東西!怪不得人人說戲子沒一個好纏的。憑你甚麼好人,入了這一行,都弄壞了。這一點子屄崽子,也挑幺挑六,鹹屄淡話,咬羣的騾子似的!”孃兒兩個吵起來。襲人忙打發人去說:“少亂嚷,瞅著老太太不在家,一個個連句安靜話也不說。”晴雯因說:“都是芳官不省事,不知狂的什麼也不是,會兩齣戲,倒象殺了賊王,擒了反叛來的。”襲人道:“一個巴掌拍不響,老的也太不公些,小的也太可惡些。”寶玉道:“怨不得芳官。自古說:‘物不平則鳴。’\begin{note}庚雙夾:自來經語未遭如是用也。\end{note}他少親失眷的,在這裏沒人照看,賺了他的錢。又作踐他,如何怪得。”因又向襲人道:“他一月多少錢?以後不如你收了過來照管他,豈不省事?”襲人道:“我要照看他那裏不照看了,又要他那幾個錢才照看他?沒的討人罵去了。”說著,便起身至那屋裏取了一瓶花露油並些雞卵、香皂、頭繩之類,叫一個婆子來送給芳官去,叫他另要水自洗,不要吵鬧了。他乾孃益發羞愧,便說芳官“沒良心,花掰我剋扣你的錢。”便向他身上拍了幾把,芳官便哭起來。寶玉便走出,襲人忙勸:“作什麼?我去說他。”晴雯忙先過來,指他乾孃說道:“你老人家太不省事。你不給他洗頭的東西,我們饒給他東西,你不自臊,還有臉打他。他要還在學裏學藝,你也敢打他不成!”那婆子便說:“一日叫孃,終身是母。他排場我,我就打得!”襲人喚麝月道:“我不會和人拌嘴,晴雯性太急,你快過去震嚇他兩句。” 麝月聽了,忙過來說道:“你且別嚷。我且問你,別說我們這一處,你看滿園子裏,誰在主子屋裏教導過女兒的?便是你的親女兒,既分了房,有了主子,自有主子打得罵得,再者大些的姑娘姐姐們打得罵得,誰許老子娘又半中間管閒事了?都這樣管,又要叫他們跟著我們學什麼?越老越沒了規矩!你見前兒墜兒的娘來吵,你也來跟他學?你們放心,因連日這個病那個病,老太太又不得閒心,所以我沒回。等兩日消閒了,咱們痛回一回,大家把威風煞一煞兒纔好。寶玉纔好了些,連我們不敢大聲說話,你反打的人狼號鬼叫的。上頭能出了幾日門,你們就無法無天的,眼睛裏沒了我們,再兩天你們就該打我們了。他不要你這乾孃,怕糞草埋了他不成?”寶玉恨的用拄杖敲著門檻子說道:“這些老婆子都是些鐵心石頭腸子,也是件大奇的事。不能照看,反倒折挫,天長地久,如何是好!”\begin{note}庚雙夾:畫出寶玉來。\end{note}晴雯道:“什麼‘如何是好’,都攆了出去,不要這些中看不中喫的!”那婆子羞愧難當,一言不發。那芳官只穿著海棠紅的小棉襖,底下絲綢撒花袷褲,敞著褲腿,\begin{note}庚雙夾:四字奇想,寫得紙上跳出一個女優來。\end{note}一頭烏油似的頭髮披在腦後,哭的淚人一般。麝月笑道:“把一個鶯鶯小姐,反弄成拷打紅娘了!這會子又不妝扮了,還是這麼松怠怠的。”寶玉道:“他這本來面目極好,倒別弄緊襯了。”晴雯過去拉了他,替他洗淨了發,用手巾擰乾,鬆鬆的挽了一個慵妝髻,命他穿了衣服過這邊來了。
\end{parag}


\begin{parag}
    接著司內廚的婆子來問:“晚飯有了,可送不送?”小丫頭聽了,進來問襲人。襲人笑道:“方纔胡吵了一陣,也沒留心聽鍾幾下了。”晴雯道:“那勞什子又不知怎麼了,又得去收拾。”說著,便拿過表來瞧了一瞧說:“略等半鍾茶的工夫就是了。”小丫頭去了。麝月笑道:“提起淘氣,芳官也該打幾下。昨兒是他擺弄了那墜子半日,就壞了。”說話之間,便將食具打點現成。一時小丫頭子捧了盒子進來站住。晴雯麝月揭開看時,還是隻四樣小菜。晴雯笑道:“已經好了,還不給兩樣清淡菜喫。這稀飯鹹菜鬧到多早晚?”一面擺好,一面又看那盒中,卻有一碗火腿鮮筍湯,忙端了放在寶玉跟前。寶玉便就桌上喝了一口,\begin{note}庚雙夾:畫出病人。\end{note}說:“好燙!”襲人笑道:“菩薩,能幾日不見葷,饞的這樣起來。”一面說,一面忙端起輕輕用口吹。因見芳官在側,便遞與芳官,笑道:“你也學著些伏侍,別一味呆憨呆睡。口勁輕著,別吹上唾沫星兒。”芳官依言果吹了幾口,甚妥。
\end{parag}


\begin{parag}
    他乾孃也忙端飯在門外伺候。向日芳官等一到時原從外邊認的,就同往梨香院去了。這幹婆子原系榮府三等人物,不過令其與他們漿洗,皆不曾入內答應,故此不知內幃規矩。今亦托賴他們方入園中,隨女歸房。這婆子先領過麝月的排場,方知了一二分,生恐不令芳官認他做乾孃,便有許多失利之處,故心中只要買轉他們。今見芳官吹湯,便忙跑進來笑道:“他不老成,仔細打了碗,讓我吹罷。”一面說,一面就接。晴雯忙喊:“出去!你讓他砸了碗,也輪不到你吹。你什麼空兒跑到這裏槅子來了?還不出去。”一面又罵小丫頭們:“瞎了心的,他不知道,你們也不說給他!”小丫頭們都說:“我們攆他,他不出去;說他,他又不信。如今帶累我們受氣,你可信了?我們到的地方兒,有你到的一半,還有你一半到不去的呢。何況又跑到我們到不去的地方還不算,又去伸手動嘴的了。”一面說,一面推他出去。階下幾個等空盒傢伙的婆子見他出來,都笑道:“嫂子也沒用鏡子照一照,就進去了。”羞的那婆子又恨又氣,只得忍耐下去。
\end{parag}


\begin{parag}
    芳官吹了幾口,寶玉笑道:“好了,仔細傷了氣。你嘗一口,可好了?”芳官只當是頑話,只是笑看著襲人等。襲人道:“你就嘗一口何妨。”晴雯笑道:“你瞧我嘗。”說著就喝了一口。芳官見如此,自己也便嚐了一口,說:“好了。”遞與寶玉。寶玉喝了半碗,吃了幾片筍,又吃了半碗粥就罷了。衆人揀收出去了。小丫頭捧了沐盆,盥漱已畢,襲人等出去喫飯。寶玉使個眼色與芳官,芳官本自伶俐,又學幾年戲,何事不知?便裝說頭疼不喫飯了。襲人道:“既不喫飯,你就在屋裏作伴兒,把這粥給你留著,一時餓了再喫。”說著,都去了。
\end{parag}


\begin{parag}
    這裏寶玉和他只二人,寶玉便將方纔從火光發起,如何見了藕官,又如何謊言護庇,又如何藕官叫我問你,從頭至尾,細細的告訴他一遍,又問他祭的果系何人。芳官聽了,滿面含笑,又嘆一口氣,說道:“這事說來可笑又可嘆。”寶玉聽了,忙問如何。芳官笑道:“你說他祭的是誰?祭的是死了的菂官。”寶玉道:“這是友誼,也應當的。”芳官笑道:“那裏是友誼?他竟是瘋傻的想頭,說他自己是小生,菂官是小旦,常做夫妻,雖說是假的,每日那些曲文排場,皆是真正溫存體貼之事,故此二人就瘋了,雖不做戲,尋常飲食起坐,兩個人竟是你恩我愛。菂官一死,他哭的死去活來,至今不忘,所以每節燒紙。後來補了蕊官,我們見他一般的溫柔體貼,也曾問他得新棄舊的。他說:‘這又有個大道理。比如男子喪了妻,或有必當續絃者,也必要續絃爲是。便只是不把死的丟過不提,便是情深意重了。若一味因死的不續,孤守一世,妨了大節,也不是理,死者反不安了。’你說可是又瘋又呆?說來可是可笑?”寶玉聽說了這篇呆話,獨合了他的呆性,不覺又是歡喜,又是悲嘆,又稱奇道絕,說:“天既生這樣人,又何用我這鬚眉濁物玷辱世界。”因又忙拉芳官囑道:“既如此說,我也有一句話囑咐他,我若親對面與他講未免不便,須得你告訴他。”芳管問何事。寶玉道:“以後斷不可燒紙錢。這紙錢原是後人異端,不是孔子的遺訓。以後逢時按節,只備一個爐,到日隨便焚香,一心誠虔,就可感格了。愚人原不知,無論神佛死人,必要分出等例,各式各例的。殊不知只一‘誠心’二字爲主。即值倉皇流離之日,雖連香亦無,隨便有土有草,只以潔淨,便可爲祭,不獨死者享祭,便是神鬼也來享的。你瞧瞧我那案上,只設一爐,不論日期,時常焚香。他們皆不知原故,我心裏卻各有所因。隨便有清茶便供一鍾茶,有新水就供一盞水,或有鮮花,或有鮮果,甚至葷羹腥菜,只要心誠意潔,便是佛也都可來享,所以說,只在敬不在虛名。以後快命他不可再燒紙。”芳官聽了,便答應著。一時喫過飯,便有人回:“老太太、太太回來了。”
\end{parag}
