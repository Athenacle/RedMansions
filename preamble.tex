\chapter*{戚蓼生序}
\addcontentsline{toc}{chapter}{戚蓼生序}

\begin{qute2sp}

    \begin{parag}
        \large
        吾聞絳樹兩歌,一聲在喉,一聲在鼻;黃華二牘,左腕能楷,右腕能草。神乎技也,吾未之見也。今則兩歌而不分乎喉鼻,二牘而無區乎左右,一聲也而兩歌,一手也而二牘,此萬萬不能有之事,不可得之奇,而竟得之《石頭記》一書。嘻!異矣。夫敷華掞藻、立意遣詞無一落前人窠臼,此固有目共賞,姑不具論;第觀其蘊於心而抒於手也,注彼而寫此,目送而手揮,似譎而正,似則而淫,如春秋之有微詞、史家之多曲筆。試一一讀而繹之:寫閨房則極其雍肅也,而艶冶已滿紙矣;狀閥閱則極其豐整也,而式微已盈睫矣;寫寶玉之淫而癡也,而多情善悟,不減歷下琅琊;寫黛玉之妒而尖也,而篤愛深憐,不啻桑娥石女。他如摹繪玉釵金屋,刻畫薌澤羅襦,靡靡焉幾令讀者心蕩神怡矣,而欲求其一字一句之粗鄙猥褻,不可得也。蓋聲止一聲,手只一手,而淫佚貞靜,悲慼歡愉,不啻雙管之齊下也。噫!異矣。其殆稗官野史中之盲左、腐遷乎?然吾謂作者有兩意,讀者當具一心。譬之繪事,石有三面,佳處不過一峯;路看兩蹊,幽處不逾一樹。必得是意,以讀是書,乃能得作者微旨。如捉水月,只挹清輝;如雨天花,但聞香氣,庶得此書弦外音乎?乃或者以未窺全豹爲恨,不知盛衰本是迴環,萬緣無非幻泡,作者慧眼婆心,正不必再作轉語,而千萬領悟,便具無數慈航矣。彼沾沾焉刻楮葉以求之者,其與開卷而寤者幾希!
    \end{parag}

\end{qute2sp}

