\chap{一百二十}{甄士隐详说太虚情 贾雨村归结红楼梦}



\begin{parag}
    话说宝钗听秋纹说袭人不好,连忙进去瞧看。巧姐儿同平儿也随着走到袭人炕前。只见袭人心痛难禁,一时气厥。宝钗等用开水灌了过来,仍旧扶他睡下,一面传请大夫。巧姐儿问宝钗道:“袭人姐姐怎么病到这个样?”宝钗道:“大前儿晚上哭伤了心了,一时发晕栽倒了。太太叫人扶他回来,他就睡倒了。因外头有事,没有请大夫瞧他,所以致此。”说着,大夫来了,宝钗等略避。大夫看了脉,说是急怒所致,开了方子去了。原来袭人模糊听见说宝玉若不回来,便要打发屋里的人都出去,一急越发不好了。到大夫瞧后,秋纹给他煎药。他各自一人躺着,神魂未定,好象宝玉在他面前,恍惚又象是个和尚,手里拿着一本册子揭着看,还说道:“你别错了主意,我是不认得你们的了。”袭人似要和他说话,秋纹走来说:“药好了,姐姐吃罢。”袭人睁眼一瞧,知是个梦,也不告诉人。吃了药,便自己细细的想:“宝玉必是跟了和尚去。上回他要拿玉出去,便是要脱身的样子,被我揪住,看他竟不象往常,把我混推混揉的,一点情意都没有。后来待二奶奶更生厌烦。在别的姊妹跟前,也是没有一点情意。这就是悟道的样子。但是你悟了道,抛了二奶奶怎么好!我是太太派我服侍你,虽是月钱照着那样的分例,其实我究竟没有在老爷太太跟前回明就算了你的屋里人。若是老爷太太打发我出去,我若死守着,又叫人笑话,若是我出去,心想宝玉待我的情分,实在不忍。”左思右想,实在难处。想到刚纔的梦“好象和我无缘”的话,“倒不如死了干净。”岂知吃药以后,心痛减了好些,也难躺着,只好勉强支持。过了几日,起来服侍宝钗。宝钗想念宝玉,暗中垂泪,自叹命苦。又知他母亲打算给哥哥赎罪,很费张罗,不能不帮着打算。暂且不表。
\end{parag}


\begin{parag}
    且说贾政扶贾母灵柩,贾蓉送了秦氏凤姐鸳鸯的棺木,到了金陵,先安了葬。贾蓉自送黛玉的灵也去安葬。贾政料理坟基的事。一日接到家书,一行一行的看到宝玉贾兰得中,心里自是喜欢。后来看到宝玉走失,复又烦恼,只得赶忙回来。在道儿上又闻得有恩赦的旨意,又接家书,果然赦罪复职,更是喜欢,便日夜趱行。
\end{parag}


\begin{parag}
    一日,行到毘陵驿地方,那天乍寒下雪,泊在一个清静去处。贾政打发众人上岸投帖辞谢朋友,总说即刻开船,都不敢劳动。船中只留一个小厮伺候,自己在船中写家书,先要打发人起旱到家。写到宝玉的事,便停笔。抬头忽见船头上微微的雪影里面一个人,光着头,赤着脚,身上披着一领大红猩猩毡的斗篷,向贾政倒身下拜。贾政尚未认清,急忙出船,欲待扶住问他是谁。那人已拜了四拜,站起来打了个问讯。贾政纔要还揖,迎面一看,不是别人,却是宝玉。贾政吃一大惊,忙问道:“可是宝玉么?”那人只不言语,似喜似悲。贾政又问道:“你若是宝玉,如何这样打扮,跑到这里?”宝玉未及回言,只见舡头上来了两人,一僧一道,夹住宝玉说道:“俗缘已毕,还不快走。”说着,三个人飘然登岸而去。贾政不顾地滑,疾忙来赶。见那三人在前,那里赶得上。只听见他们三人口中不知是那个作歌曰:
\end{parag}


\begin{poem}

    \begin{pl}
        我所居兮,青埂之峰。
    \end{pl}


    \begin{pl}
        我所游兮,鸿蒙太空。
    \end{pl}


    \begin{pl}
        谁与我游兮,吾谁与从。
    \end{pl}


    \begin{pl}
        渺渺茫茫兮,归彼大荒。
    \end{pl}
\end{poem}


\begin{parag}
    贾政一面听着,一面赶去,转过一小坡,倏然不见。贾政已赶得心虚气喘,惊疑不定,回过头来,见自己的小厮也是随后赶来。贾政问道:“你看见方纔那三个人么?”小厮道:“看见的。奴才为老爷追赶,故也赶来。后来只见老爷,不见那三个人了。”贾政还欲前走,只见白茫茫一片旷野,并无一人。贾政知是古怪,只得回来。
\end{parag}


\begin{parag}
    众家人回舡,见贾政不在舱中,问了舡夫,说是“老爷上岸追赶两个和尚一个道士去了。”众人也从雪地里寻踪迎去,远远见贾政来了,迎上去接着,一同回船。贾政坐下,喘息方定,将见宝玉的话说了一遍。众人回禀,便要在这地方寻觅。贾政叹道:“你们不知道,这是我亲眼见的,并非鬼怪。况听得歌声大有元妙。那宝玉生下时衔了玉来,便也古怪,我早知不祥之兆,为的是老太太疼爱,所以养育到今。便是那和尚道士,我也见了三次:头一次是那僧道来说玉的好处,第二次便是宝玉病重,他来了将那玉持诵了一番,宝玉便好了,第三次送那玉来坐在前厅,我一转眼就不见了。我心里便有些诧异,只道宝玉果真有造化,高僧仙道来护佑他的。岂知宝玉是下凡历劫的,竟哄了老太太十九年!如今叫我纔明白。”说到那里,掉下泪来。众人道:“宝二爷果然是下凡的和尚,就不该中举人了。怎么中了纔去?”贾政道:“你们那里知道,大凡天上星宿,山中老僧,洞里的精灵,他自有一种性情。你看宝玉何尝肯念书,他若略一经心,无有不能的。他那一种脾气也是各别另样。”说着,又叹了几声。众人便拿“兰哥得中,家道复兴”的话解了一番。贾政仍旧写家书,便把这事写上,劝谕合家不必想念了。写完封好,即着家人回去。贾政随后赶回。暂且不题。
\end{parag}


\begin{parag}
    且说薛姨妈得了赦罪的信,便命薛蝌去各处借贷。并自己凑齐了赎罪银两。刑部准了,收兑了银子,一角文书将薛蟠放出。他们母子姊妹弟兄见面,不必细述,自然是悲喜交集了。薛蟠自己立誓说道:“若是再犯前病,必定犯杀犯剐!”薛姨妈见他这样,便要握他嘴说:“只要自己拿定主意,必定还要妄口巴舌血淋淋的起这样恶誓么!只香菱跟了你受了多少的苦处,你媳妇已经自己治死自己了,如今虽说穷了,这碗饭还有得吃,据我的主意,我便算他是媳妇了,你心里怎么样?”薛蟠点头愿意。宝钗等也说:“很该这样。”倒把香菱急得脸胀通红,说是:“伏侍大爷一样的,何必如此。”众人便称起大奶奶来,无人不服。薛蟠便要去拜谢贾家,薛姨妈宝钗也都过来。见了众人,彼此聚首,又说了一番的话。正说着,恰好那日贾政的家人回家,呈上书子,说:“老爷不日到了。”王夫人叫贾兰将书子念给听。贾兰念到贾政亲见宝玉的一段,众人听了都痛哭起来,王夫人宝钗袭人等更甚。大家又将贾政书内叫家内“不必悲伤,原是借胎”的话解说了一番。“与其作了官,倘或命运不好,犯了事坏家败产,那时倒不好了。宁可咱们家出一位佛爷,倒是老爷太太的积德,所以纔投到咱们家来。不是说句不顾前后的话,当初东府里太爷倒是修炼了十几年,也没有成了仙。这佛是更难成的。太太这么一想,心里便开豁了。”王夫人哭着和薛姨妈道:“宝玉抛了我,我还恨他呢。我叹的是媳妇的命苦,纔成了一二年的亲,怎么他就硬着肠子都撂下了走了呢!”薛姨妈听了也甚伤心。宝钗哭得人事不知。所有爷们都在外头,王夫人便说道:“我为他担了一辈子的惊,刚刚儿的娶了亲,中了举人,又知道媳妇作了胎,我纔喜欢些,不想弄到这样结局!早知这样,就不该娶亲害了人家的姑娘!”薛姨妈道:“这是自己一定的,咱们这样人家,还有什么别的说的吗?幸喜有了胎,将来生个外孙子必定是有成立的,后来就有了结果了。你看大奶奶,如今兰哥儿中了举人,明年成了进士,可不是就做了官了么。他头里的苦也算吃尽的了,如今的甜来,也是他为人的好处。我们姑娘的心肠儿姊姊是知道的,并不是刻薄轻佻的人,姊姊倒不必耽忧。”王夫人被薛姨妈一番言语说得极有理,心想:“宝钗小时候更是廉静寡欲极爱素淡的,他所以纔有这个事,想人生在世真有一定数的。看着宝钗虽是痛哭,他端庄样儿一点不走,却倒来劝我,这是真真难得的!不想宝玉这样一个人,红尘中福分竟没有一点儿!”想了一回,也觉解了好些。又想到袭人身上:“若说别的丫头呢,没有什么难处的,大的配了出去,小的伏侍二奶奶就是了。独有袭人可怎么处呢?”此时人多,也不好说,且等晚上和薛姨妈商量。
\end{parag}


\begin{parag}
    那日薛姨妈并未回家,因恐宝钗痛哭,所以在宝钗房中解劝。那宝钗却是极明理,思前想后,“宝玉原是一种奇异的人。夙世前因,自有一定,原无可怨天尤人。”更将大道理的话告诉他母亲了。薛姨妈心里反倒安了,便到王夫人那里先把宝钗的话说了。王夫人点头叹道:“若说我无德,不该有这样好媳妇了。”说着,更又伤心起来。薛姨妈倒又劝了一会子,因又提起袭人来,说:“我见袭人近来瘦的了不得,他是一心想着宝哥儿。但是正配呢理应守的,屋里人愿守也是有的。惟有这袭人,虽说是算个屋里人,到底他和宝哥儿并没有过明路儿的。”王夫人道:“我纔刚想着,正要等妹妹商量商量。若说放他出去,恐怕他不愿意,又要寻死觅活的,若要留着他也罢,又恐老爷不依。所以难处。”薛姨妈道:“我看姨老爷是再不肯叫守着的。再者姨老爷并不知道袭人的事,想来不过是个丫头,那有留的理呢?只要姊姊叫他本家的人来,狠狠的吩咐他,叫他配一门正经亲事,再多多的陪送他些东西。那孩子心肠儿也好,年纪儿又轻,也不枉跟了姐姐会子,也算姐姐待他不薄了。袭人那里还得我细细劝他。就是叫他家的人来也不用告诉他,只等他家里果然说定了好人家儿,我们还去打听打听,若果然足衣足食,女婿长的象个人儿,然后叫他出去。”王夫人听了道:“这个主意很是。不然叫老爷冒冒失失的一办,我可不是又害了一个人了么!”薛姨妈听了点头道:“可不是么!”又说了几句,便辞了王夫人,仍到宝钗房中去了。
\end{parag}


\begin{parag}
    看见袭人泪痕满面,薛姨妈便劝解譬喻了一会。袭人本来老实,不是伶牙利齿的人,薛姨妈说一句,他应一句,回来说道:“我是做下人的人,姨太太瞧得起我,纔和我说这些话,我是从不敢违拗太太的。”薛姨妈听他的话,“好一个柔顺的孩子!”心里更加喜欢。宝钗又将大义的话说了一遍,大家各自相安。
\end{parag}


\begin{parag}
    过了几日,贾政回家,众人迎接。贾政见贾赦贾珍已都回家,弟兄叔侄相见,大家历叙别来的景况。然后内眷们见了,不免想起宝玉来,又大家伤了一会子心。贾政喝住道:“这是一定的道理。如今只要我们在外把持家事,你们在内相助,断不可仍是从前这样的散慢。别房的事,各有各家料理,也不用承总。我们本房的事,里头全归于你,都要按理而行。”王夫人便将宝钗有孕的话也告诉了,将来丫头们都劝放出去。贾政听了,点头无语。
\end{parag}


\begin{parag}
    次日贾政进内,请示大臣们,说是:“蒙恩感激,但未服阕,应该怎么谢恩之处,望乞大人们指教。”众朝臣说是代奏请旨。于是圣恩浩荡,即命陛见。贾政进内谢了恩,圣上又降了好些旨意,又问起宝玉的事来。贾政据实回奏。圣上称奇,旨意说,宝玉的文章固是清奇,想他必是过来人,所以如此。若在朝中,可以进用。他既不敢受圣朝的爵位,便赏了一个“文妙真人”的道号。贾政又叩头谢恩而出。
\end{parag}


\begin{parag}
    回到家中,贾琏贾珍接着,贾政将朝内的话述了一遍,众人喜欢。贾珍便回说:“宁国府第收拾齐全,回明了要搬过去。栊翠庵圈在园内,给四妹妹静养。”贾政并不言语,隔了半日,却吩咐了一番仰报天恩的话。贾琏也趁便回说:“巧姐亲事,父亲太太都愿意给周家为媳。”贾政昨晚也知巧姐的始末,便说:“大老爷大太太作主就是了。莫说村居不好,只要人家清白,孩子肯念书,能够上进。朝里那些官儿难道都是城里的人么?”贾琏答应了“是”,又说:“父亲有了年纪,况且又有痰症的根子,静养几年,诸事原仗二老爷为主。”贾政道:“提起村居养静,甚合我意。只是我受恩深重,尚未酬报耳。”贾政说毕进内。贾琏打发请了刘姥姥来,应了这件事。刘姥姥见了王夫人等,便说些将来怎样升官,怎样起家,怎样子孙昌盛。正说着,丫头回道:“花自芳的女人进来请安。”王夫人问几句话,花自芳的女人将亲戚作媒,说的是城南蒋家的,现在有房有地,又有铺面,姑爷年纪略大了几岁,并没有娶过的,况且人物儿长的是百里挑一的。王夫人听了愿意,说道:“你去应了,隔几日进来再接你妹子罢。”王夫人又命人打听,都说是好。王夫人便告诉了宝钗,仍请了薛姨妈细细的告诉了袭人。袭人悲伤不已,又不敢违命的,心里想起宝玉那年到他家去,回来说的死也不回去的话,“如今太太硬作主张。若说我守着,又叫人说我不害臊,若是去了,实不是我的心愿”,便哭得咽哽难鸣,又被薛姨妈宝钗等苦劝,回过念头想道:“我若是死在这里,倒把太太的好心弄坏了。我该死在家里纔是。”于是,袭人含悲叩辞了众人,那姐妹分手时自然更有一番不忍说。袭人怀着必死的心肠上车回去,见了哥哥嫂子,也是哭泣,但只说不出来。那花自芳悉把蒋家的娉礼送给他看,又把自己所办妆奁一一指给他瞧,说那是太太赏的,那是置办的。袭人此时更难开口,住了两天,细想起来:“哥哥办事不错,若是死在哥哥家里,岂不又害了哥哥呢。”千思万想,左右为难,真是一缕柔肠,几乎牵断,只得忍住。
\end{parag}


\begin{parag}
    那日已是迎娶吉期,袭人本不是那一种泼辣人,委委屈屈的上轿而去,心里另想到那里再作打算。岂知过了门,见那蒋家办事极其认真,全都按着正配的规矩。一进了门,丫头仆妇都称奶奶。袭人此时欲要死在这里,又恐害了人家,辜负了一番好意。那夜原是哭着不肯俯就的,那姑爷却极柔情曲意的承顺。到了第二天开箱,这姑爷看见一条猩红汗巾,方知是宝玉的丫头。原来当初只知是贾母的侍儿,益想不到是袭人。此时蒋玉菡念着宝玉待他的旧情,倒觉满心惶愧,更加周旋,又故意将宝玉所换那条松花绿的汗巾拿出来。袭人看了,方知这姓蒋的原来就是蒋玉菡,始信姻缘前定。袭人纔将心事说出,蒋玉菡也深为叹息敬服,不敢勉强,并越发温柔体贴,弄得个袭人真无死所了。看官听说:虽然事有前定,无可奈何。但孽子孤臣,义夫节妇,这“不得已”三字也不是一概推委得的。此袭人所以在又一副册也。正是前人过那桃花庙的诗上说道:
\end{parag}


\begin{poem}
    \begin{pl}
        千古艰难惟一死,伤心岂独息夫人!
    \end{pl}
\end{poem}


\begin{parag}
    不言袭人从此又是一番天地。且说那贾雨村犯了婪索的案件,审明定罪,今遇大赦,褫籍为民。雨村因叫家眷先行,自己带了一个小厮,一车行李,来到急流津觉迷渡口。只见一个道者从那渡头草棚里出来,执手相迎。雨村认得是甄士隐,也连忙打恭,士隐道:“贾先生别来无恙?”雨村道:“老仙长到底是甄老先生!何前次相逢觌面不认?后知火焚草亭,下鄙深为惶恐。今日幸得相逢,益叹老仙翁道德高深。奈鄙人下愚不移,致有今日。”甄士隐道:“前者老大人高官显爵,贫道怎敢相认!原因故交,敢赠片言,不意老大人相弃之深。然而富贵穷通,亦非偶然,今日复得相逢,也是一桩奇事。这里离草庵不远,暂请膝谈,未知可否?”
\end{parag}


\begin{parag}
    雨村欣然领命,两人携手而行,小厮驱车随后,到了一座茅庵。士隐让进雨村坐下,小童献上茶来。雨村便请教仙长超尘的始末。士隐笑道:“一念之间,尘凡顿易。老先生从繁华境中来,岂不知温柔富贵乡中有一宝玉乎?”雨村道:“怎么不知。近闻纷纷传述,说他也遁入空门。下愚当时也曾与他往来过数次,再不想此人竟有如是之决绝。”士隐道:“非也。这一段奇缘,我先知之。昔年我与先生在仁清巷旧宅门口叙话之前,我已会过他一面。”雨村惊讶道:“京城离贵乡甚远,何以能见?”士隐道:“神交久矣。”雨村道:“既然如此,现今宝玉的下落,仙长定能知之。”士隐道:“宝玉,即宝玉也。那年荣宁查抄之前,钗黛分离之日,此玉早已离世。一为避祸,二为撮合,从此夙缘一了,形质归一,又复稍示神灵,高魁贵子,方显得此玉那天奇地灵之宝,非凡间可比。前经茫茫大士渺渺真人携带下凡,如今尘缘已满,仍是此二人携归本处,这便是宝玉的下落。”雨村听了,虽不能全然明白,却也十知四五,便点头叹道:“原来如此,下愚不知。但那宝玉既有如此的来历,又何以情迷至此,复又豁悟如此?还要请教。”士隐笑道:“此事说来,老先生未必尽解。太虚幻境即是真如福地。一番阅册,原始要终之道,历历生平,如何不悟?仙草归真,焉有通灵不复原之理呢!”雨村听着,却不明白了。知仙机也不便更问,因又说道:“宝玉之事既得闻命,但是敝族闺秀如此之多,何元妃以下算来结局俱属平常呢?”士隐叹息道:“老先生莫怪拙言,贵族之女俱属从情天孽海而来。大凡古今女子,那‘淫’字固不可犯,只这‘情’字也是沾染不得的。所以崔莺苏小,无非仙子尘心,宋玉相如,大是文人口孽。凡是情思缠绵的,那结果就不可问了。”雨村听到这里,不觉拈须长叹,因又问道:“请教老仙翁,那荣宁两府,尚可如前否?”士隐道:“福善祸淫,古今定理。现今荣宁两府,善者修缘,恶者悔祸,将来兰桂齐芳,家道复初,也是自然的道理。”雨村低了半日头,忽然笑道:“是了,是了。现在他府中有一个名兰的已中乡榜,恰好应着‘兰’字。适间老仙翁说‘兰桂齐芳’,又道宝玉‘高魁子贵’,莫非他有遗腹之子,可以飞黄腾达的么?”士隐微微笑道:“此系后事,未便预说。”雨村还要再问,士隐不答,便命人设俱盘飧,邀雨村共食。
\end{parag}


\begin{parag}
    食毕,雨村还要问自己的终身,士隐便道:“老先生草庵暂歇,我还有一段俗缘未了,正当今日完结。”雨村惊讶道:“仙长纯修若此,不知尚有何俗缘?”士隐道:“也不过是儿女私情罢了。”雨村听了益发惊异:“请问仙长,何出此言?”士隐道:“老先生有所不知,小女英莲幼遭尘劫,老先生初任之时曾经判断。今归薛姓,产难完劫,遗一子于薛家以承宗祧。此时正是尘缘脱尽之时,只好接引接引。”士隐说着拂袖而起。雨村心中恍恍惚惚,就在这急流津觉迷渡口草庵中睡着了。
\end{parag}


\begin{parag}
    这士隐自去度脱了香菱,送到太虚幻境,交那警幻仙子对册,刚过牌坊,见那一僧一道,缥渺而来。士隐接着说道:“大士,真人,恭喜,贺喜!情缘完结,都交割清楚了么?”那僧道说:“情缘尚未全结,倒是那蠢物已经回来了。还得把他送还原所,将他的后事叙明,不枉他下世一回。”士隐听了,便供手而别。那僧道仍携了玉到青埂峰下,将宝玉安放在女娲炼石补天之处,各自云游而去。从此后,“天外书传天外事,两番人作一番人。”
\end{parag}


\begin{parag}
    这一日空空道人又从青埂峰前经过,见那补天未用之石仍在那里,上面字迹依然如旧,又从头的细细看了一遍,见后面偈文后又历叙了多少收缘结果的话头,便点头叹道:“我从前见石兄这段奇文,原说可以闻世传奇,所以曾经抄录,但未见返本还原。不知何时复有此一佳话,方知石兄下凡一次,磨出光明,修成圆觉,也可谓无复遗憾了。只怕年深日久,字迹模糊,反有舛错,不如我再抄录一番,寻个世上清闲无事的人,托他传遍,知道奇而不奇,俗而不俗,真而不真,假而不假。或者尘梦劳人,聊倩鸟呼归去,山灵好客,更从石化飞来,亦未可知。”想毕,便又抄了,仍袖至那繁华昌盛的地方,遍寻了一番,不是建功立业之人,即系饶口谋衣之辈,那有闲情更去和石头饶舌。直寻到急流津觉迷渡口,草庵中睡着一个人,因想他必是闲人,便要将这抄录的《石头记》给他看看。那知那人再叫不醒。空空道人复又使劲拉他,纔慢慢的开眼坐起,便草草一看,仍旧掷下道:“这事我早已亲见尽知。你这抄录的尚无舛错,我只指与你一个人,托他传去,便可归结这一新鲜公案了。”空空道人忙问何人,那人道:“你须待某年某月某日到一个悼红轩中,有个曹雪芹先生,只说贾雨村言托他如此如此。”说毕,仍旧睡下了。
\end{parag}


\begin{parag}
    那空空道人牢牢记着此言,又不知过了几世几劫,果然有个悼红轩,见那曹雪芹先生正在那里翻阅历来的古史。空空道人便将贾雨村言了,方把这《石头记》示看。那雪芹先生笑道:“果然是‘贾雨村言’了!”空空道人便问:“先生何以认得此人,便肯替他传述?”曹雪芹先生笑道:“说你空,原来你肚里果然空空。既是假语村言,但无鲁鱼亥豕以及背谬矛盾之处,乐得与二三同志,酒余饭饱,雨夕灯窗之下,同消寂寞,又不必大人先生品题传世,似你这样寻根问底,便是刻舟求剑,胶柱鼓瑟了。”那空空道人听了,仰天大笑,掷下抄本,飘然而去。一面走着,口中说道:“果然是敷衍荒唐!不但作者不知,抄者不知,并阅者也不知。不过游戏笔墨,陶情适性而已!”后人见了这本奇传,亦曾题过四句偈语,为作者缘起之言更转一竿头云:
\end{parag}


\begin{poem}
    \begin{pl}
        说到辛酸处,荒唐愈可悲。
    \end{pl}

    \begin{pl}
        由来同一梦,休笑世人痴!
    \end{pl}
\end{poem}