\chap{三十二}{诉肺腑心迷活宝玉 含耻辱情烈死金钏}
\begin{parag}
    \begin{note}庚辰:前明显祖汤先生有《怀人》诗一截,堪合此回,故录之以待知音。曰:无情无尽却情多,情到无多得尽么?解道多情情尽处,月中无树影无波。\end{note}
\end{parag}


\begin{parag}
    话说宝玉见那麒麟,心中甚是欢喜,便伸手来拿,笑道:“亏你拣著了。你是那里拣的?”史湘云笑道:“幸而是这个,明儿倘或把印也丢了,难道也就罢了不 成?”宝玉笑道:“倒是丢了印平常,若丢了这个,我就该死了。”袭人斟了茶来与史湘云吃,一面笑道:“大姑娘,听见前儿你大喜了。”史湘云红了脸,吃茶不答。袭人道:“这会子又害臊了。你还记得十年前,咱们在西边暖阁住著,晚上你同我说的话儿?那会子不害臊,这会子怎么又害臊了?”史湘云笑道:“你还说呢。那会子咱们那么好。后来我们太太没了,我家去住了一程子,怎么就把你派了跟二哥哥,我来了,你就不象先待我了。”袭人笑道:“你还说呢。先姐姐长姐姐短哄著我替你梳头洗脸,作这个弄那个,\begin{note}蒙侧:大家风范,情景逼真。\end{note}如今大了,就拿出小姐的款来。你既拿小姐的款,我怎敢亲近呢?”史湘云道:“阿弥陀佛,冤枉冤哉!我要这样,就立刻死了。你瞧瞧,这么大热天,我来了,必定赶来先瞧瞧你。不信你问问缕儿,我在家时时刻刻那一回不念你几声。”话未了,忙的袭人和宝玉都劝道:“顽话你又认真了。还是这么性急。”史湘云道:“你不说你的话噎人,倒说人性急。”一面说,一面打开手帕子,将戒指递与袭人。\begin{note}蒙侧 批:心中意中多少情致。\end{note}袭人感谢不尽,因笑道:“你前儿送你姐姐们的,我已得了;今儿你亲自又送来,可见是没忘了我。只这个就试出你来了。戒指儿能值多 少,可见你的心真。”史湘云道:“是谁给你的?”袭人道:“是宝姑娘给我的。”湘云笑道:“我只当是林姐姐给你的,原来是宝钗姐姐给了你。我天天在家里想 著,这些姐姐们再没一个比宝姐姐好的。可惜我们不是一个娘养的。\begin{note}蒙侧:感知己之一欢。\end{note}我但凡有这么个亲姐姐,就是没了父母,也是没妨碍的。”说著, 眼睛圈儿就红了。宝玉道:“罢,罢,罢!不用提这个话。”史湘云道:“提这个便怎么?我知道你的心病,恐怕你的林妹妹听见,又怪嗔我赞了宝姐姐。可是为这个不是?”袭人在旁嗤的一笑,说道:“云姑娘,你如今大了,越发心直口快了。”宝玉笑道:“我说你们这几个人难说话,果然不错。”史湘云道:“好哥哥,你 不必说话教我恶心。只会在我们跟前说话,见了你林妹妹,又不知怎么了。”\begin{note}蒙侧:豪爽情性如画。\end{note}
\end{parag}


\begin{parag}
    袭人道:“且别说顽话,正有一件事还要求你呢。”史湘云便问:“什么事?”袭人道:“有一双鞋,抠了垫心子。我这两日身上不好,不得做,你可有工夫替 我做做?”史湘云笑道:“这又奇了,你家放著这些巧人不算,还有什么针线上的,裁剪上的,怎么教我做起来?你的活计叫谁做,谁好意思不做呢。”袭人笑道: “你又糊涂了。你难道不知道,我们这屋里的针线,\begin{note}蒙侧:“我们这屋里”等字精神活跳。\end{note}是不要那些针线上的人做的。”史湘云听了,便知是宝玉的鞋了, 因笑道:“既这么说,我就替你做了罢。只是一件,你的我才作,别人的我可不能。”袭人笑道:“又来了,我是个什么,就烦你做鞋了。实告诉你,可不是我的。 你别管是谁的,横竖我领情就是了。”史湘云道:“论理,你的东西也不知烦我做了多少了,今儿我倒不做了的原故,你必定也知道。”袭人道:“倒也不知道。\begin{note}蒙侧:反观叠起,灵活之至。\end{note}” 史湘云冷笑道:“前儿我听见把我做的扇套子拿著和人家比,赌气又铰了。我早就听见了,你还瞒我。这会子又叫我做,我成了你们的奴才了。”宝玉忙笑道:“前 儿的那事,本不知是你做的。”袭人也笑道:“他本不知是你做的。是我哄他的话,说是新近外头有个会做活的女孩子,说扎的出奇的花,我叫他拿了一个扇套子试 试看好不好。他就信了,拿出去给这个瞧给那个看的。不知怎么又惹恼了林姑娘,铰了两段。回来他还叫赶著做去,我才说了是你作的,他后悔的什么似的。\begin{note}蒙侧:描神!\end{note}”史湘云道:“越发奇了。林姑娘他也犯不上生气,他既会剪,就叫他做。”袭人道:“他可不作呢。饶这么著,老太太还怕他劳碌著了。大夫又说好 生静养才好,谁还烦他做?旧年好一年的工夫,做了个香袋儿;今年半年,还没见拿针线呢。”
\end{parag}


\begin{parag}
    正说著,有人来回说:“兴隆街的大爷来了,老爷叫二爷出去会。”宝玉听了,便知是贾雨村来了,心中好不自在。袭人忙去拿衣服。宝玉一面蹬著靴子,一面抱怨道:“有老爷和他坐著就罢了,\begin{note}蒙侧:原本烦俗。\end{note}回回定要见我。”史湘云一边摇著扇子,笑道:“自然你能会宾接客,老爷才叫你出去呢。”宝玉道: “那里是老爷,都是他自己要请我去见的。”湘云笑道:“主雅客来勤,自然你有些警他的好处,他才只要会你。”宝玉道:“罢,罢,我也不敢称雅,俗中又俗的 一个俗人,并不愿同这些人往来。”\begin{note}蒙侧:我也不知宝玉是俗是雅,请诸同类一拟。\end{note}湘云笑道:“还是这个情性不改。如今大了,你就不愿读书去考举人进士的,也该常常的会会这些为官做宰的人们,谈谈讲讲些仕途经济的学问,也好将来应酬世务,日后也有个朋友。没见你成年家只在我们队里搅些什么!”宝玉听了 道:“姑娘请别的姊妹屋里坐坐,我这里仔细污了你知经济学问的。”袭人道:“云姑娘快别说这话。\begin{note}蒙侧:此际不同湘云一语,湘云也定难出一语。\end{note}上回也是宝姑娘也说过一回,他也不管人脸上过的去过不去,他就咳了一声,拿起脚来走了。这里宝姑娘的话也没说完,见他走了,登时羞的脸通红,说又不是,不说又不是。幸而是宝姑娘,那要是林姑娘,不知又闹到怎么样,哭的怎么样呢。提起这个话来,真真的宝姑娘叫人敬重,自己讪了一会子去了。我倒过不去,\begin{note}蒙侧:袭人善解忿。\end{note}只当他恼了。谁知过后还是照旧一样,真真有涵养,心地宽大。谁知这一个反倒同他生分了。那林姑娘见你赌气不理他,你得赔多少不是呢。”宝玉道:“林姑娘从来说过这些混帐话不曾?若他也说过这些混帐话,我早和他生分了。”\begin{note}蒙侧:花爱水清明,水怜花色新。浮落虽同流,空惹鱼龙涎。\end{note}袭人和湘云都点头笑道:“这原是混帐话。”
\end{parag}


\begin{parag}
    原来林黛玉知道史湘云在这里,宝玉又赶来,一定说麒麟的原故。因此心下忖度著,近日宝玉弄来的外传野史,多半才子佳人都因小巧玩物上撮合,或有鸳鸯, 或有凤凰,或玉环金珮,或鲛帕鸾绦,皆由小物而遂终身。今忽见宝玉亦有麒麟,便恐借此生隙,同史湘云也做出那些风流佳事来。因而悄悄走来,见机行事,以察二人之意。不想刚走来,正听见史湘云说经济一事,宝玉又说:“林妹妹不说这样混帐话,若说这话,我也和他生分了。”林黛玉听了这话,不觉又喜又惊,又悲又 叹。所喜者,果然自己眼力不错,素日认他是个知己,果然是个知己。所惊者,他在人前一片私心称扬于我,其亲热厚密,竟不避嫌疑。所叹者,你既为我之知己, 自然我亦可为你之知己矣;既你我为知己,则又何必有金玉之论哉;既有金玉之论,亦该你我有之,则又何必来一宝钗哉!所悲者,父母早逝,虽有铭心刻骨之言, 无人为我主张。况近日每觉神思恍惚,病已渐成,医者更云气弱血亏,恐致劳怯之症。你我虽为知己,但恐自不能久待;你纵为我知己,奈我薄命何!想到此间,不 禁滚下泪来。\begin{note}蒙侧:普天下才子佳人英雄侠士都同来一哭!我虽愚浊,也愿同声一哭。\end{note}待进去相见,自觉无味,便一面拭泪,一面抽身回去了。
\end{parag}


\begin{parag}
    这里宝玉忙忙的穿了衣裳出来,忽见林黛玉在前面慢慢的走著,似有拭泪之状,便忙赶上来,\begin{note}蒙侧:关心情致。\end{note}笑道:“妹妹往那里去?怎么又哭了?又 是谁得罪了你?”林黛玉回头见是宝玉,便勉强笑道:“好好的,我何曾哭了。”宝玉笑道:“你瞧瞧,眼睛上的泪珠儿未干,还撒谎呢。”一面说,一面禁不住抬 起手来替他拭泪。林黛玉忙向后退了几步,说道:“你又要死了!\begin{note}蒙侧:娇羞态!\end{note}作什么这么动手动脚的!”宝玉笑道:“说话忘了情,不觉的动了手,也就顾不的死活。”林黛玉道:“你死了倒不值什么,只是丢下了什么金,又是什么麒麟,可怎么样呢?”一句话又把宝玉说急了,赶上来问道:“你还说这话,到底是咒我还是气我呢?”林黛玉见问,方想起前日的事来,遂自悔自己又说造次了,忙笑道:“你别著急,我原说错了。这有什么的,筋都暴起来,急的一脸汗。”一面说,一面禁不住近前伸手替他拭面上的汗\begin{note}蒙侧:痴情态。\end{note}。宝玉瞅了半天,方说道“你放心”三个字。\begin{note}蒙侧:连我今日看之,也不懂是何等文章。\end{note}林黛玉听了,怔了半天,方说道:“我有什么不放心的?我不明白这话。你倒说说怎么放心不放心?”宝玉叹了一口气,问道:“你果不明白这话?难道我素日在你身上的心都用错了?连你的意思若体贴不著,就难怪你天天为我生气了。”林黛玉道:“果然我不明白放心不放心的话。”宝玉点头叹道:“好妹妹,你别哄我。果然不 明白这话,不但我素日之意白用了,且连你素日待我之意也都辜负了。\begin{note}蒙侧:第二层。\end{note}你皆因总是不放心的原故,才弄了一身病。但凡宽慰些,\begin{note}蒙侧:真 疼真爱真怜真惜中,每每生出此等心病来。\end{note}这病也不得一日重似一日。”林黛玉听了这话,如轰雷掣电,细细思之,竟比自己肺腑中掏出来的还觉恳切,\begin{note}蒙侧 批:何等神佛,开慧眼照见众生孽障,为现此锦绣文章,说此上乘功德法。\end{note}竟有万句言语,满心要说,只是半个字也不能吐,却怔怔的望著他。此时宝玉心中也有万句言语,不知从那一句上说起,却也怔怔的望著黛玉。两个人怔了半天,林黛玉只咳了一声,两眼不觉滚下泪来,回身便要走。\begin{note}蒙侧:下笔时,用一“走”, 文之大力,孟愤(愤之右半)不苦也。\end{note}宝玉忙上前拉住,说道:“好妹妹,且略站住,我说一句话再走。”林黛玉一面拭泪,一面将手推开,说道:“有什么可说 的。你的话我早知道了!”口里说著,却头也不回竟去了。
\end{parag}


\begin{parag}
    宝玉站著,只管发起呆来。原来方才出来慌忙,不曾带得扇子,袭人怕他热,忙拿了扇子赶来送与他,忽抬头见了林黛玉和他站著。一时黛玉走了,他还站著不动,因而赶上来说道:“你也不带了扇子去,亏我看见,赶了送来。”宝玉出了神,见袭人和他说话,并未看出是何人来,便一把拉住,说道:“好妹妹,我的这心 事,从来也不敢说,今儿我大胆说出来,死也甘心!我为你也弄了一身的病在这里,又不敢告诉人,只好掩著。只等你的病好了,只怕我的病才得好呢。睡里梦里也 忘不了你!”袭人听了这话,吓得魄消魂散,只叫“神天菩萨,坑死我了!”便推他道:“这是那里的话!敢是中了邪?还不快去?”宝玉一时醒过来,方知是袭人 送扇子来,羞的满面紫涨,夺了扇子,便忙忙的抽身跑了。
\end{parag}


\begin{parag}
    这里袭人见他去了,自思方才之言,一定是因黛玉而起,如此看来,将来难免不才之事,令人可惊可畏。想到此间,也不觉怔怔的滴下泪来,心下暗度如何处治 方免此丑祸。正裁疑间,忽有宝钗从那边走来,笑道:“大毒日头地下,出什么神呢?”袭人见问,忙笑道:“那边两个雀儿打架,倒也好玩,我就看住了。”宝钗道:“宝兄弟这会子穿了衣服,忙忙的那去了?我才看见走过去,倒要叫住问他呢。他如今说话越发没了经纬,我故此没叫他了,由他过去罢。”袭人道:“老爷叫 他出去。”宝钗听了,忙道:“嗳哟!这么黄天暑热的,叫他做什么!别是想起什么来生了气,\begin{note}蒙侧:偏是近。\end{note}叫出去教训一场。”袭人笑道:“不是这个, 想是有客要会。”宝钗笑道:“这个客也没意思,这么热天,不在家里凉快,还跑些什么!”袭人笑道:“倒是你说说罢。”
\end{parag}


\begin{parag}
    宝钗因而问道:“云丫头在你们家做什么呢?”袭人笑道:“才说了一会子闲话。你瞧,我前儿粘的那双鞋,明儿叫他做去。”宝钗听见这话,便两边回头,看 无人来往,便笑道:“你这么个明白人,怎么一时半刻的就不会体谅人情。我近来看著云丫头神情,再风里言风里语的听起来,那云丫头在家里竟一点儿作不得主。 他们家嫌费用大,竟不用那些针线上的人,差不多的东西多是他们娘儿们动手。为什么这几次他来了,他和我说话儿,见没人在跟前,他就说家里累的很。我再问他 两句家常过日子的话,他就连眼圈儿都红了,口里含含糊糊待说不说的。想其形景来,自然从小儿没爹娘的苦。\begin{note}蒙侧:真是知己,不枉湘云前言。\end{note}我看著他, 也不觉的伤起心来。”袭人见说这话,将手一拍,说:“是了,是了。怪道上月我烦他打十根蝴蝶结子,过了那些日子才打发人送来,还说‘打的粗,且在别处能著 使罢;要匀净的,等明儿来住著再好生打罢’。如今听宝姑娘这话,想来我们烦他他不好推辞,不知他在家里怎么三更半夜的做呢。可是我也糊涂了,早知是这样, 我也不烦他了。”宝钗道:“上次他就告诉我,在家里做活做到三更天,若是替别人做一点半点,他家的那些奶奶太太们还不受用呢。”袭人道:“偏生我们那个牛心左性的小爷,\begin{note}蒙侧:多情的当有这样牛心左性之癖。\end{note}凭著小的大的活计,一概不要家里这些活计上的人作。我又弄不开这些。”宝钗笑道:“你理他呢!只 管叫人做去,只说是你做的就是了。”袭人笑道:“那里哄的信他,他才是认得出来呢。说不得我只好慢慢的累去罢了。\begin{note}蒙侧:痴心的情愿。\end{note}”宝钗笑道: “你不必忙,我替你作些如何?”袭人笑道:“当真的这样,就是我的福了。晚上我亲自送过来。”
\end{parag}


\begin{parag}
    一句话未了,忽见一个老婆子忙忙走来,说道:“这是那里说起!金钏儿姑娘好好的投井死了!”袭人唬了一跳,忙问:“那个金钏儿?”那老婆子道:“那里 还有两个金钏儿呢?就是太太屋里的。前儿不知为什么撵他出去,在家里哭天哭地的,也都不理会他,谁知找他不见了。刚才打水的人在那东南角上井里打水,见一个尸首,赶著叫人打捞起来,谁知是他。他们家里还只管乱著要救活,那里中用了!”宝钗道:“这也奇了。”袭人听说,点头赞叹,想素日同气之情,不觉流下泪 来。\begin{note}蒙侧:又一哭法。\end{note}宝钗听见这话,忙向王夫人处来道安慰。这里袭人回去不提。
\end{parag}


\begin{parag}
    却说宝钗来至王夫人处,只见鸦雀无闻,独有王夫人在里间房内坐著垂泪。\begin{note}蒙侧:又一哭法。\end{note}宝钗便不好提这事,只得一旁坐了。王夫人便问:“你从那里来?”宝钗道:“从园里来。”王夫人道:“你从园里来,可见你宝兄弟?”\begin{note}蒙侧:世人多是凡事欲瞒人,偏不意中将要著开露,理之所无,事则多有,何 也?\end{note}宝钗道:“才倒看见了。他穿了衣服出去了,不知那里去。”王夫人点头哭道:“你可知道一桩奇事?金钏儿忽然投井死了!”宝钗见说,道:“怎么好好的 投井?这也奇了。”王夫人道:“原是前儿他把我一件东西弄坏了,我一时生气,打了他几下,撵了他下去。我只说气他两天,还叫他上来,谁知他这么气性大,就 投井死了。岂不是我的罪过。”宝钗叹道:“姨娘是慈善人,固然这么想。据我看来,他并不是赌气投井。多半他下去住著,或是在井跟前憨顽,失了脚掉下去的。 他在上头拘束惯了,这一出去,自然要到各处去顽顽逛逛,岂有这样大气的理!纵然有这样大气,也不过是个糊涂人,也不为可惜。\begin{note}蒙侧:善劝人大见解!惜乎?不知其情,虽精美玉之言不中,奈何?\end{note}” 王夫人点头叹道:“这话虽然如此说,到底我心不安。”宝钗叹道:“姨娘也不必念念于兹,十分过不去,不过多赏他几两银子发送他,也就尽主仆之情了。”王夫 人道:“刚才我赏了他娘五十两银子,原要还把你妹妹们的新衣服拿两套给他妆裹。谁知凤丫头说可巧都没什么新做的衣服,只有你林妹妹作生日的两套。我想你林 妹妹那个孩子素日是个有心的,况且他也三灾八难的,既说了给他过生日,这会子又给人妆裹去,岂不忌讳。因为这么样,我现叫裁缝赶两套给他。要是别的丫头, 赏他几两银子也就完了,只是金钏儿虽然是个丫头,素日在我跟前比我的女儿也差不多。”口里说著,不觉泪下。宝钗忙道:“姨娘这会子又何用叫裁缝赶去,我前 儿倒做了两套,拿来给他岂不省事。况且他活著的时候也穿过我的旧衣服,身量又相对。”王夫人道:“虽然这样,难道你不忌讳?”宝钗笑道:“姨娘放心,我从 来不计较这些。”一面说,一面起身就走。王夫人忙叫了两个人来跟宝姑娘去。
\end{parag}


\begin{parag}
    一时宝钗取了衣服回来,只见宝玉在王夫人旁边坐著垂泪。王夫人正才说他,因宝钗来了,却掩了口不说了。\begin{note}蒙侧:云龙现影法,可爱煞人。\end{note}宝钗见此光景,察言观色,早知觉了八分,于是将衣服交割明白。王夫人将他母亲叫来拿了去。再看下回便知。
\end{parag}


\begin{parag}
    \begin{note}蒙回末总:世上无情空大地,人间少爱景何穷。其中世界其中了,含笑同归造化功。\end{note}
\end{parag}


\begin{parag}
    \begin{note}蒙回末总:袭人湘云黛玉宝钗等之爱之哭,各具一心,各具一见。而宝玉黛玉之痴情痴性,行文如绘真,是现身说法,岂三家村老学究之可能实现者!不尽炷香再拜!\end{note}
\end{parag}

