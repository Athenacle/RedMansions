\chap{九十六}{瞞消息鳳姐設奇謀 泄機關顰兒迷本性}


\begin{parag}
    話說賈璉拿了那塊假玉忿忿走出,到了書房。那個人看見賈璉的氣色不好,心裏先發了虛了,連忙站起來迎着。剛要說話,只見賈璉冷笑道:“好大膽,我把你這個混賬東西!這裏是什麼地方兒,你敢來掉鬼!”回頭便問:“小廝們呢?”外頭轟雷一般幾個小廝齊聲答應。賈璉道:“取繩子去捆起他來。等老爺回來問明瞭,把他送到衙門裏去。”衆小廝又一齊答應“預備着呢。”嘴裏雖如此,卻不動身。那人先自唬的手足無措,見這般勢派,知道難逃公道,只得跪下給賈璉碰頭,口口聲聲只叫:“老太爺別生氣。是我一時窮極無奈,纔想出這個沒臉的營生來。那玉是我借錢做的,我也不敢要了,只得孝敬府裏的哥兒頑罷。”說畢,又連連磕頭。賈璉啐道:“你這個不知死活的東西!這府裏希罕你的那朽不了的浪東西!”正鬧着,只見賴大進來,陪着笑向賈璉道:“二爺別生氣了。靠他算個什麼東西,饒了他,叫他滾出去罷。”賈璉道:“實在可惡。”賴大賈璉作好作歹,衆人在外頭都說道:“糊塗狗攮的,還不給爺和賴大爺磕頭呢。快快的滾罷,還等窩心腳呢!”那人趕忙磕了兩個頭,抱頭鼠竄而去。從此街上鬧動了“賈寶玉弄出‘假寶玉’”來。
\end{parag}


\begin{parag}
    且說賈政那日拜客回來,衆人因爲燈節底下,恐怕賈政生氣,已過去的事了,便也都不肯回。只因元妃的事忙碌了好些時,近日寶玉又病着,雖有舊例家宴,大家無興,也無有可記之事。到了正月十七日,王夫人正盼王子騰來京,只見鳳姐進來回說”今日二爺在外聽得有人傳說,我們家大老爺趕着進京,離城只二百多里地,在路上沒了。太太聽見了沒有?”王夫人喫驚道:“我沒有聽見,老爺昨晚也沒有說起,到底在那裏聽見的?”鳳姐道:“說是在樞密張老爺家聽見的。”王夫人怔了半天,那眼淚早流下來了,因拭淚說道:“回來再叫璉兒索性打聽明白了來告訴我。”鳳姐答應去了。王夫人不免暗裏落淚,悲女哭弟,又爲寶玉耽憂。如此連三接二,都是不隨意的事,那裏擱得住,便有些心口疼痛起來。又加賈璉打聽明白了來說道:“舅太爺是趕路勞乏,偶然感冒風寒,到了十里屯地方,延醫調治。無奈這個地方沒有名醫,誤用了藥,一劑就死了。但不知家眷可到了那裏沒有?”王夫人聽了,一陣心酸,便心口疼得坐不住,叫彩雲等扶了上炕,還扎掙着叫賈璉去回了賈政,“即速收拾行裝迎到那裏,幫着料理完畢,既刻回來告訴我們。好叫你媳婦兒放心。”賈璉不敢違拗,只得辭了賈政起身。賈政早已知道,心裏很不受用,又知寶玉失玉以後神志惛憒,醫藥無效,又值王夫人心疼。那年正值京察,工部將賈政保列一等。二月,吏部帶領引見。皇上念賈政勤儉謹慎,即放了江西糧道。即日謝恩,已奏明起程日期。雖有衆親朋賀喜,賈政也無心應酬,只念家中人口不寧,又不敢耽延在家。正在無計可施,只聽見賈母那邊叫“請老爺。”
\end{parag}


\begin{parag}
    賈政即忙進去,看見王夫人帶着病也在那裏。便向賈母請了安。賈母叫他坐下,便說:“你不日就要赴任,我有多少話與你說,不知你聽不聽?”說着,掉下淚來。賈政忙站起來說道:“老太太有話只管吩咐,兒子怎敢不遵命呢。”賈母咽哽着說道:“我今年八十一歲的人了,你又要做外任去,偏有你大哥在家,你又不能告親老。你這一去了,我所疼的只有寶玉,偏偏的又病得糊塗,還不知道怎麼樣呢。我昨日叫賴升媳婦出去叫人給寶玉算算命,這先生算得好靈,說要娶了金命的人幫扶他,必要衝沖喜纔好,不然只怕保不住。我知道你不信那些話,所以教你來商量。你的媳婦也在這裏。你們兩個也商量商量,還是要寶玉好呢,還是隨他去呢?”賈政陪笑說道:“老太太當初疼兒子這麼疼的,難道做兒子的就不疼自己的兒子不成麼。只爲寶玉不上進,所以時常恨他,也不過是恨鐵不成鋼的意思。老太太既要給他成家,這也是該當的,豈有逆着老太太不疼他的理。如今寶玉病着,兒子也是不放心。因老太太不叫他見我,所以兒子也不敢言語。我到底瞧瞧寶玉是個什麼病。”王夫人見賈政說着也有些眼圈兒紅,知道心裏是疼的,便叫襲人扶了寶玉來。寶玉見了他父親,襲人叫他請安,他便請了個安。賈政見他臉面很瘦,目光無神,大有瘋傻之狀,便叫人扶了進去,便想到:“自己也是望六的人了,如今又放外任,不知道幾年回來。倘或這孩子果然不好,一則年老無嗣,雖說有孫子,到底隔了一層,二則老太太最疼的是寶玉,若有差錯,可不是我的罪名更重了。”瞧瞧王夫人,一包眼淚,又想到他身上,復站起來說:“老太太這麼大年紀,想法兒疼孫子,做兒子的還敢違拗?老太太主意該怎麼便怎麼就是了。但只姨太太那邊不知說明白了沒有?”王夫人便道:“姨太太是早應了的。只爲蟠兒的事沒有結案,所以這些時總沒提起。”賈政又道:“這就是第一層的難處。他哥哥在監裏,妹子怎麼出嫁。況且貴妃的事雖不禁婚嫁,寶玉應照已出嫁的姐姐有九個月的功服,此時也難娶親。再者我的起身日期已經奏明,不敢耽擱,這幾天怎麼辦呢?”賈母想了一想:“說的果然不錯。若是等這幾件事過去,他父親又走了。倘或這病一天重似一天,怎麼好?只可越些禮辦了纔好。”想定主意,便說道:“你若給他辦呢,我自然有個道理,包管都礙不着。姨太太那邊我和你媳婦親自過去求他。蟠兒那裏我央蝌兒去告訴他,說是要救寶玉的命,諸事將就,自然應的。若說服裏娶親,當真使不得。況且寶玉病着,也不可教他成親,不過是沖沖喜,我們兩家願意,孩子們又有金玉的道理,婚是不用合的了。即挑了好日子,按着咱們家分兒過了禮。趕着挑個娶親日子,一概鼓樂不用,倒按宮裏的樣子,用十二對提燈,一乘八人轎子抬了來,照南邊規矩拜了堂,一樣坐牀撒帳,可不是算娶了親了麼。寶丫頭心地明白,是不用慮的。內中又有襲人,也還是個妥妥當當的孩子。再有個明白人常勸他更好。他又和寶丫頭合的來。再者姨太太曾說,寶丫頭的金鎖也有個和尚說過,只等有玉的便是婚姻,焉知寶丫頭過來,不因金鎖倒招出他那塊玉來,也定不得。從此一天好似一天,豈不是大家的造化。這會子只要立刻收拾屋子,鋪排起來。這屋子是要你派的。一概親友不請,也不排筵席,待寶玉好了,過了功服,然後再擺席請人。這麼着都趕的上。你也看見了他們小兩口的事,也好放心的去。”賈政聽了,原不願意,只是賈母做主,不敢違命,勉強陪笑說道:“老太太想的極是,也很妥當。只是要吩咐家下衆人,不許吵嚷得裏外皆知,這要耽不是的。姨太太那邊,只怕不肯,若是果真應了,也只好按着老太太的主意辦去。”賈母道:“姨太太那裏有我呢。你去吧。”賈政答應出來,心中好不自在。因赴任事多,部裏領憑,親友們薦人,種種應酬不絕,竟把寶玉的事,聽憑賈母交與王夫人鳳姐兒了。惟將榮禧堂後身王夫人內屋旁邊一大跨所二十餘間房屋指與寶玉,餘者一概不管。賈母定了主意叫人告訴他去,賈政只說很好,此是後話。
\end{parag}


\begin{parag}
    且說寶玉見過賈政,襲人扶回裏間炕上。因賈政在外,無人敢與寶玉說話,寶玉便昏昏沉沉的睡去。賈母與賈政所說的話,寶玉一句也沒有聽見。襲人等卻靜靜兒的聽得明白。頭裏雖也聽得些風聲,到底影響,只不見寶釵過來,卻也有些信真。今日聽了這些話,心裏方纔水落歸漕,倒也喜歡。心裏想道:“果然上頭的眼力不錯,這才配得是。我也造化。若他來了,我可以卸了好些擔子。但是這一位的心理只有一個林姑娘,幸虧他沒有聽見,若知道了,又不知要鬧到什麼分兒了。”襲人想到這裏,轉喜爲悲,心想:“這件事怎麼好?老太太,太太那裏知道他們心裏的事。一時高興說給他知道,原想要他病好。若是他仍似前的心事:初見林姑娘便要摔玉砸玉,況且那年夏天在園裏把我當作林姑娘,說了好些私心話,後來因爲紫鵑說了句頑話兒,便哭得死去活來。若是如今和他說要娶寶姑娘,竟把林姑娘撂開,除非是他人事不知還可,若稍明白些,只怕不但不能沖喜,竟是催命了!我再不把話說明,那不是一害三個人了麼。”襲人想定主意,待等賈政出去,叫秋紋照看着寶玉,便從裏間出來,走到王夫人身旁,悄悄的請了王夫人到賈母后身屋裏去說話。賈母只道是寶玉有話,也不理會,還在那裏打算怎麼過禮,怎麼娶親。
\end{parag}


\begin{parag}
    那襲人同了王夫人到了後間,便跪下哭了。王夫人不知何意,把手拉着他說:“好端端的,這是怎麼說?有什麼委屈起來說。”襲人道:“這話奴才是不該說的,這會子因爲沒有法兒了。”王夫人道:“你慢慢說。”襲人道:“寶玉的親事老太太,太太已定了寶姑娘了,自然是極好的一件事。只是奴才想着,太太看去寶玉和寶姑娘好,還是和林姑娘好呢?”王夫人道:“他兩個因從小兒在一處,所以寶玉和林姑娘又好些。”襲人道:“不是好些。”便將寶玉素與黛玉這些光景一一的說了,還說:“這些事都是太太親眼見的。獨是夏天的話我從沒敢和別人說。”王夫人拉着襲人道:“我看外面兒已瞧出幾分來了。你今兒一說,更加是了。但是剛纔老爺說的話想必都聽見了,你看他的神情兒怎麼樣?”襲人道:“如今寶玉若有人和他說話他就笑,沒人和他說話他就睡。所以頭裏的話卻倒都沒聽見。”王夫人道:“倒是這件事叫人怎麼樣呢?”襲人道:“奴才說是說了,還得太太告訴老太太,想個萬全的主意纔好。”王夫人便道:“既這麼着,你去幹你的,這時候滿屋子的人,暫且不用提起,等我瞅空兒回明老太太,再作道理。”說着,仍到賈母跟前。
\end{parag}


\begin{parag}
    賈母正在那裏和鳳姐兒商議,見王夫人進來,便問道:“襲人丫頭說什麼?這麼鬼鬼祟祟的。”王夫人趁問,便將寶玉的心事,細細回明賈母。賈母聽了,半日沒言語。王夫人和鳳姐也都不再說了。只見賈母嘆道:“別的事都好說。林丫頭倒沒有什麼,若寶玉真是這樣,這可叫人作了難了。”只見鳳姐想了一想,因說道:“難倒不難,只是我想了個主意,不知姑媽肯不肯。”王夫人道:“你有主意只管說給老太太聽,大家娘兒們商量着辦罷了。”鳳姐道:“依我想,這件事只有一個掉包兒的法子。”賈母道:“怎麼掉包兒?”鳳姐道:“如今不管寶兄弟明白不明白,大家吵嚷起來,說是老爺做主,將林姑娘配了他了。瞧他的神情兒怎麼樣。要是他全不管,這個包兒也就不用掉了。若是他有些喜歡的意思,這事卻要大費周折呢。”王夫人道:“就算他喜歡,你怎麼樣辦法呢?”鳳姐走到王夫人耳邊,如此這般的說了一遍。王夫人點了幾點頭兒,笑了一笑說道:“也罷了。”賈母便問道:“你孃兒兩個搗鬼,到底告訴我是怎麼着呀?”鳳姐恐賈母不懂,露泄機關,便也向耳邊輕輕的告訴了一遍。賈母果真一時不懂,鳳姐笑着又說了幾句。賈母笑道:“這麼着也好,可就只忒苦了寶丫頭了。倘或吵嚷出來,林丫頭又怎麼樣呢?”鳳姐道:“這個話原只說給寶玉聽,外頭一概不許提起,有誰知道呢。”正說間,丫頭傳進話來說:“璉二爺回來了。”王夫人恐賈母問及,使個眼色與鳳姐。鳳姐便迎着賈璉努了個嘴兒,同到王夫人屋裏等着去了。一回兒王夫人進來,已見鳳姐哭的兩眼通紅。賈璉請了安,將到十里屯料理王子騰的喪事的話說了一遍,便說:“有恩旨賞了內閣的職銜,諡了文勤公,命本宗扶柩回籍,着沿途地方官員照料。昨日起身,連家眷回南去了。舅太太叫我回來請安問好,說如今想不到不能進京,有多少話不能說。聽見我大舅子要進京,若是路上遇見了,便叫他來到咱們這裏細細的說。”王夫人聽畢,其悲痛自不必言。鳳姐勸慰了一番,“請太太略歇一歇,晚上來再商量寶玉的事罷。”說畢,同了賈璉回到自己房中,告訴了賈璉,叫他派人收拾新房。不題。
\end{parag}


\begin{parag}
    一日,黛玉早飯後帶着紫鵑到賈母這邊來,一則請安,二則也爲自己散散悶。出了瀟湘館,走了幾步,忽然想起忘了手絹子來,因叫紫鵑回去取來,自己卻慢慢的走着等他。剛走到沁芳橋那邊山石背後,當日同寶玉葬花之處,忽聽一個人嗚嗚咽咽在那裏哭。黛玉煞住腳聽時,又聽不出是誰的聲音,也聽不出哭着叨叨的是些什麼話。心裏甚是疑惑,便慢慢的走去。及到了跟前,卻見一個濃眉大眼的丫頭在那裏哭呢。黛玉未見他時,還只疑府裏這些大丫頭有什麼說不出的心事,所以來這裏發泄發泄,及至見了這個丫頭,卻又好笑,因想到:這種蠢貨有什麼情種,自然是那屋裏作粗活的丫頭受了大女孩子的氣了。細瞧了一瞧,卻不認得。那丫頭見黛玉來了,便也不敢再哭,站起來拭眼淚。黛玉問道:“你好好的爲什麼在這裏傷心?”那丫頭聽了這話,又流淚道:“林姑娘你評評這個理。他們說話我又不知道,我就說錯了一句話,我姐姐也不犯就打我呀。”黛玉聽了,不懂他說的是什麼,因笑問道:“你姐姐是那一個?”那丫頭道:“就是珍珠姐姐。”黛玉聽了,才知道他是賈母屋裏的,因又問:“你叫什麼?”那丫頭道:“我叫傻大姐兒。”黛玉笑了一笑,又問:“你姐姐爲什麼打你?你說錯了什麼話了?”那丫頭道:“爲什麼呢,就是爲我們寶二爺娶寶姑娘的事情。”黛玉聽了這一句,如同一個疾雷,心頭亂跳。略定了定神,便叫了這丫頭”你跟了我這裏來。”那丫頭跟着黛玉到那畸角兒上葬桃花的去處,那裏背靜。黛玉因問道:“寶二爺娶寶姑娘,他爲什麼打你呢?”傻大姐道:“我們老太太和太太二奶奶商量了,因爲我們老爺要起身,說就趕着往姨太太商量把寶姑娘娶過來罷。頭一宗,給寶二爺衝什麼喜,第二宗——”說到這裏,又瞅着黛玉笑了一笑,才說道:“趕着辦了,還要給林姑娘說婆婆家呢。”黛玉已經聽呆了。這丫頭只管說道:“我又不知道他們怎麼商量的,不叫人吵嚷,怕寶姑娘聽見害臊。我白和寶二爺屋裏的襲人姐姐說了一句:‘咱們明兒更熱鬧了,又是寶姑娘,又是寶二奶奶,這可怎麼叫呢!’林姑娘你說我這話害着珍珠姐姐什麼了嗎,他走過來就打了我一個嘴巴,說我混說,不遵上頭的話,要攆出我去。我知道上頭爲什麼不叫言語呢,你們又沒告訴我,就打我。”說着,又哭起來。
\end{parag}


\begin{parag}
    那黛玉此時心裏竟是油兒醬兒糖兒醋兒倒在一處的一般,甜苦酸鹹,竟說不上什麼味兒來了。停了一會兒,顫巍巍的說道:“你別混說了。你再混說,叫人聽見又要打你了。你去罷。”說着,自己移身要回瀟湘館去。那身子竟有千百斤重的,兩隻腳卻象踩着棉花一般,早已軟了。只得一步一步慢慢的走將來。走了半天,還沒到沁芳橋畔,原來腳下軟了。走的慢,且又迷迷癡癡,信着腳從那邊繞過來,更添了兩箭地的路。這時剛到沁芳橋畔,卻又不知不覺的順着堤往回裏走起來。紫鵑取了絹子來,卻不見黛玉。正在那裏看時,只見黛玉顏色雪白,身子恍恍蕩蕩的,眼睛也直直的,在那裏東轉西轉。又見一個丫頭往前頭走了,離的遠,也看不出是那一個來。心中驚疑不定,只得趕過來輕輕的問道:“姑娘怎麼又回去?是要往那裏去?”黛玉也只模糊聽見,隨口應道:“我問問寶玉去!”紫鵑聽了,摸不着頭腦,只得攙着他到賈母這邊來。
\end{parag}


\begin{parag}
    黛玉走到賈母門口,心裏微覺明晰,回頭看見紫鵑攙着自己,便站住了問道:“你作什麼來的?”紫鵑陪笑道:“我找了絹子來了。頭裏見姑娘在橋那邊呢,我趕着過來問姑娘,姑娘沒理會。”黛玉笑道:“我打量你來瞧寶二爺來了呢,不然怎麼往這裏走呢。”紫鵑見他心裏迷惑,便知黛玉必是聽見那丫頭什麼話了,惟有點頭微笑而已。只是心裏怕他見了寶玉,那一個已經是瘋瘋傻傻,這一個又這樣恍恍惚惚,一時說出些不大體統的話來,那時如何是好?心裏雖如此想,卻也不敢違拗,只得攙他進去。那黛玉卻又奇怪了,這時不似先前那樣軟了,也不用紫鵑打簾子,自己掀起簾子進來,卻是寂然無聲。因賈母在屋裏歇中覺,丫頭們也有脫滑頑去的,也有打盹兒的,也有在那裏伺候老太太的。倒是襲人聽見簾子響,從屋裏出來一看,見是黛玉,便讓道:“姑娘屋裏坐罷。”黛玉笑着道:“寶二爺在家麼?”襲人不知底裏,剛要答言,只見紫鵑在黛玉身後和他努嘴兒,指着黛玉,又搖搖手兒。襲人不解何意,也不敢言語。黛玉卻也不理會,自己走進房來。看見寶玉在那裏坐着,也不起來讓坐,只瞅着嘻嘻的傻笑。黛玉自己坐下,卻也瞅着寶玉笑。兩個人也不問好,也不說話,也無推讓,只管對着臉傻笑起來。襲人看見這番光景,心裏大不得主意,只是沒法兒。忽然聽着黛玉說道:“寶玉,你爲什麼病了?”寶玉笑道:“我爲林姑娘病了。”襲人紫鵑兩個嚇得面目改色,連忙用言語來岔。兩個卻又不答言,仍舊傻笑起來。襲人見了這樣,知道黛玉此時心中迷惑不減於寶玉,因悄和紫鵑說道:“姑娘纔好了,我叫秋紋妹妹同着你攙回姑娘歇歇去罷。”因回頭向秋紋道:“你和紫鵑姐姐送林姑娘去罷,你可別混說話。”秋紋笑着,也不言語,便來同着紫鵑攙起黛玉。
\end{parag}


\begin{parag}
    那黛玉也就起來,瞅着寶玉只管笑,只管點頭兒。紫鵑又催道:“姑娘回家去歇歇罷。”黛玉道:“可不是,我這就是回去的時候兒了。”說着,便回身笑着出來了,仍舊不用丫頭們攙扶,自己卻走得比往常飛快。紫鵑秋紋後面趕忙跟着走。黛玉出了賈母院門,只管一直走去。紫鵑連忙攙住叫道:“姑娘往這麼來。”黛玉仍是笑着隨了往瀟湘館來。離門口不遠,紫鵑道:“阿彌陀佛,可到了家了!”只這一句話沒說完,只見黛玉身子往前一栽,哇的一聲,一口血直吐出來。未知性命如何,且聽下回分解。
\end{parag}