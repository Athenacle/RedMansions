\chap{九十九}{守官箴惡奴同破例 閱邸報老舅自擔驚}



\begin{parag}
    話說鳳姐見賈母和薛姨媽爲黛玉傷心,便說:“有個笑話兒說給老太太和姑媽聽”,未從開口,先自笑了,因說道:“老太太和姑媽打諒是那裏的笑話兒?就是咱們家的那二位新姑爺新媳婦啊。”賈母道:“怎麼了?”鳳姐拿手比着道:“一個這麼坐着,一個這麼站着。一個這麼扭過去,一個這麼轉過來。一個又……”說到這裏,賈母已經大笑起來,說道:“你好生說罷,倒不是他們兩口兒,你倒把人慪的受不得了。”薛姨媽也笑道:“你往下直說罷,不用比了。”鳳姐才說道:“剛纔我到寶兄弟屋裏,我看見好幾個人笑。我只道是誰,巴着窗戶眼兒一瞧,原來寶妹妹坐在炕沿上,寶兄弟站在地下。寶兄弟拉着寶妹妹的袖子,口口聲聲只叫:‘寶姐姐,你爲什麼不會說話了?你這麼說一句話,我的病包管全好。’寶妹妹卻扭着頭只管躲。寶兄弟卻作了一個揖,上前又拉寶妹妹的衣服。寶妹妹急得一扯,寶兄弟自然病後是腳軟的,索性一撲,撲在寶妹妹身上了。寶妹妹急得紅了臉,說道:‘你越發比先不尊重了。’”說到這裏,賈母和薛姨媽都笑起來。鳳姐又道:“寶兄弟便立起身來笑道:‘虧了跌了這一交,好容易才跌出你的話來了。’”薛姨媽笑道:“這是寶丫頭古怪。這有什麼的,既作了兩口兒,說說笑笑的怕什麼。他沒見他璉二哥和你。”鳳姐兒笑道:“這是怎麼說呢,我饒說笑話給姑媽解悶兒,姑媽反倒拿我打起卦來了。”賈母也笑道:“要這麼着纔好。夫妻固然要和氣,也得有個分寸兒。我愛寶丫頭就在這尊重上頭。只是我愁着寶玉還是那麼傻頭傻腦的,這麼說起來,比頭裏竟明白多了。你再說說,還有什麼笑話兒沒有?”鳳姐道:“明兒寶玉圓了房,親家太太抱了外孫子,那時侯不更是笑話兒了麼。”賈母笑道:“猴兒,我在這裏同着姨太太想你林妹妹,你來慪個笑兒還罷了,怎麼臊起皮來了。你不叫我們想你林妹妹,你不用太高興了,你林妹妹恨你,將來不要獨自一個到園裏去,堤防他拉着你不依。”鳳姐笑道:“他倒不怨我。他臨死咬牙切齒倒恨着寶玉呢。”賈母薛姨媽聽着,還道是頑話兒,也不理會,便道:“你別胡拉扯了。你去叫外頭挑個很好的日子給你寶兄弟圓了房兒罷。”鳳姐去了,擇了吉日,重新擺酒唱戲請親友。這不在話下。
\end{parag}


\begin{parag}
    卻說寶玉雖然病好復原,寶釵有時高興翻書觀看,談論起來,寶玉所有眼前常見的尚可記憶,若論靈機,大不似從前活變了,連他自己也不解,寶釵明知是通靈失去,所以如此。倒是襲人時常說他:“你何故把從前的靈機都忘了?那些舊毛病忘了纔好,爲什麼你的脾氣還覺照舊,在道理上更糊塗了呢?”寶玉聽了並不生氣,反是嘻嘻的笑。有時寶玉順性胡鬧,多虧寶釵勸說,諸事略覺收斂些。襲人倒可少費些脣舌,惟知悉心伏侍。別的丫頭素仰寶釵貞靜和平,各人心服,無不安靜。只有寶玉到底是愛動不愛靜的,時常要到園裏去逛。賈母等一則怕他招受寒暑,二則恐他睹景傷情,雖黛玉之柩已寄放城外庵中,然而瀟湘館依然人亡屋在,不免勾起舊病來,所以也不使他去。況且親戚姊妹們,薛寶琴已回到薛姨媽那邊去了,史湘雲因史侯回京,也接了家去了,又有了出嫁的日子,所以不大常來,只有寶玉娶親那一日與喫喜酒這天來過兩次,也只在賈母那邊住下,爲着寶玉已經娶過親的人,又想自己就要出嫁的,也不肯如從前的詼諧談笑,就是有時過來,也只和寶釵說話,見了寶玉不過問好而已,那邢岫煙卻是因迎春出嫁之後便隨着邢夫人過去,李家姊妹也另住在外,即同着李嬸孃過來,亦不過到太太們與姐妹們處請安問好,即回到李紈那裏略住一兩天就去了:所以園內的只有李紈,探春,惜春了。賈母還要將李紈等挪進來,爲着元妃薨後,家中事情接二連三,也無暇及此。現今天氣一天熱似一天,園裏尚可住得,等到秋天再挪。此是後話,暫且不提。
\end{parag}


\begin{parag}
    且說賈政帶了幾個在京請的幕友,曉行夜宿,一日到了本省,見過上司,即到任拜印受事,便查盤各屬州縣糧米倉庫。賈政向來作京官,只曉得郎中事務都是一景兒的事情,就是外任,原是學差,也無關於吏治上。所以外省州縣折收糧米勒索鄉愚這些弊端,雖也聽見別人講究,卻未嘗身親其事。只有一心做好官,便與幕賓商議出示嚴禁,並諭以一經查出,必定詳參揭報。初到之時,果然胥吏畏懼,便百計鑽營,偏遇賈政這般古執。那些家人跟了這位老爺在都中一無出息,好容易盼到主人放了外任,便在京指着在外發財的名頭向人借貸,做衣裳裝體面,心裏想着,到了任,銀錢是容易的了。不想這位老爺呆性發作,認真要查辦起來,州縣饋送一概不受。門房簽押等人心裏盤算道:“我們再挨半個月,衣服也要當完了。債又逼起來,那可怎麼樣好呢。眼見得白花花的銀子,只是不能到手。”那些長隨也道:“你們爺們到底還沒花什麼本錢來的。我們才冤,花了若干的銀子打了個門子,來了一個多月,連半個錢也沒見過。想來跟這個主兒是不能撈本兒的了。明兒我們齊打夥兒告假去。”次日果然聚齊,都來告假。賈政不知就裏,便說:“要來也是你們,要去也是你們。既嫌這裏不好,就都請便。”那些長隨怨聲載道而去。只剩下些家人,又商議道:“他們可去的去了,我們去不了的,到底想個法兒纔好。”內中有一個管門的叫李十兒,便說:“你們這些沒能耐的東西,着什麼忙!我見這長字號兒的在這裏,不犯給他出頭。如今都餓跑了,瞧瞧你十太爺的本領,少不得本主兒依我。只是要你們齊心,打夥兒弄幾個錢回家受用,若不隨我,我也不管了,橫豎拚得過你們。”衆人都說:“好十爺,你還主兒信得過。若你不管,我們實在是死症了。”李十兒道:“不要我出了頭得了銀錢,又說我得了大分兒了。窩兒裏反起來,大家沒意思。”衆人道:“你萬安,沒有的事。就沒有多少,也強似我們腰裏掏錢。”正說着,只見糧房書辦走來找週二爺。李十兒坐在椅子上,蹺着一隻腿,挺着腰說道:“找他做什麼?”書辦便垂手陪着笑說道:“本官到了一個多月的任,這些州縣太爺見得本官的告示利害,知道不好說話,到了這時侯都沒有開倉。若是過了漕,你們太爺們來做什麼的。”李十兒道:“你別混說。老爺是有根蒂的,說到那裏是要辦到那裏。這兩天原要行文催兌,因我說了緩幾天才歇的。你到底找我們週二爺做什麼?”書辦道:“原爲打聽催文的事,沒有別的。”李十兒道:“越發胡說,方纔我說催文,你就信嘴胡謅。可別鬼鬼祟祟來講什麼帳,我叫本官打了你,退你。”書辦道:“我在衙門內已經三代了。外頭也有些體面,家裏還過得,就規規矩矩伺侯本官升了還能夠,不象那些等米下鍋的。”說着,回了一聲“二太爺,我走了。”李十兒便站起,堆着笑說:“這麼不禁頑,幾句話就臉急了。”書辦道:“不是我臉急,若再說什麼,豈不帶累了二太爺的清名呢。”李十兒過來拉著書辦的手說:“你貴姓啊?”書辦道:“不敢,我姓詹,單名是個‘會’字,從小兒也在京裏混了幾年。”李十兒道:“詹先生,我是久聞你的名的。我們兄弟們是一樣的,有什麼話晚上到這裏咱們說一說。”書辦也說:“誰不知道李十太爺是能事的,把我一詐就嚇毛了。”大家笑着走開。那晚便與書辦咕唧了半夜,第二天拿話去探賈政,被賈政痛罵了一頓。
\end{parag}


\begin{parag}
    隔一天拜客,裏頭吩咐伺侯,外頭答應了。停了一會子,打點已經三下了,大堂上沒有人接鼓。好容易叫個人來打了鼓。賈政踱出暖閣,站班喝道的衙役只有一個。賈政也不查問,在墀下上了轎,等轎伕又等了好一回。來齊了,擡出衙門,那個炮只響得一聲,吹鼓亭的鼓手只有一個打鼓,一個吹號筒。賈政便也生氣說:“往常還好,怎麼今兒不齊集至此。”抬頭看那執事,卻是攙前落後。勉強拜客回來,便傳誤班的要打,有的說因沒有帽子誤的,有的說是號衣當了誤的,又有的說是三天沒喫飯抬不動。賈政生氣,打了一兩個也就罷了。隔一天,管廚房的上來要錢,賈政帶來銀兩付了。
\end{parag}


\begin{parag}
    以後便覺樣樣不如意,比在京的時侯倒不便了好些。無奈,便喚李十兒問道:“我跟來這些人怎樣都變了?你也管管。現在帶來銀兩早使沒有了,藩庫俸銀尚早,該打發京裏取去。”李十兒稟道:“奴才那一天不說他們,不知道怎麼樣這些人都是沒精打彩的,叫奴才也沒法兒。老爺說家裏取銀子,取多少?現在打聽節度衙門這幾天有生日,別的府道老爺都上千上萬的送了,我們到底送多少呢?”賈政道:“爲什麼不早說?”李十兒說:“老爺最聖明的。我們新來乍到,又不與別位老爺很來往,誰肯送信。巴不得老爺不去,便好想老爺的美缺。”賈政道:“胡說,我這官是皇上放的,不與節度做生日便叫我不做不成!”李十兒笑着回道:“老爺說的也不錯。京裏離這裏很遠,凡百的事都是節度奏聞。他說好便好,說不好便喫不住。到得明白,已經遲了。就是老太太,太太們,那個不願意老爺在外頭烈烈轟轟的做官呢。”賈政聽了這話,也自然心裏明白,道:“我正要問你,爲什麼都說起來?”李十兒回說:“奴才本不敢說。老爺既問到這裏,若不說是奴才沒良心,若說了少不得老爺又生氣。”賈政道:“只要說得在理。”李十兒說道:“那些書吏衙役都是花了錢買着糧道的衙門,那個不想發財?俱要養家活口。自從老爺到了任,並沒見爲國家出力,倒先有了口碑載道。”賈政道:“民間有什麼話?”李十兒道:“百姓說,凡有新到任的老爺,告示出得愈利害,愈是想錢的法兒。州縣害怕了,好多多的送銀子。收糧的時侯,衙門裏便說新道爺的法令,明是不敢要錢,這一留難叨蹬,那些鄉民心裏願意花幾個錢早早了事,所以那些人不說老爺好,反說不諳民情。便是本家大人是老爺最相好的,他不多幾年已巴到極頂的分兒,也只爲識時達務能夠上和下睦罷了。”賈政聽到這話,道:“胡說,我就不識時務嗎?若是上和下睦,叫我與他們貓鼠同眠嗎。”李十兒回說道:“奴才爲着這點忠心兒掩不住,才這麼說,若是老爺就是這樣做去,到了功不成名不就的時侯,老爺又說奴才沒良心,有什麼話不告訴老爺了。”賈政道:“依你怎麼做纔好?”李十兒道:“也沒有別的。趁着老爺的精神年紀,裏頭的照應,老太太的硬朗,爲顧着自己就是了。不然到不了一年,老爺家裏的錢也都貼補完了,還落了自上至下的人抱怨,都說老爺是做外任的,自然弄了錢藏着受用。倘遇着一兩件爲難的事,誰肯幫着老爺?那時辦也辦不清,悔也悔不及。”賈政道:“據你一說,是叫我做貪官嗎?送了命還不要緊,必定將祖父的功勳抹了纔是?”李十兒回稟道:“老爺極聖明的人,沒看見舊年犯事的幾位老爺嗎?這幾位都與老爺相好,老爺常說是個做清官的,如今名在那裏!現有幾位親戚,老爺向來說他們不好的,如今升的升,遷的遷。只在要做的好就是了。老爺要知道,民也要顧,官也要顧。若是依着老爺不準州縣得一個大錢,外頭這些差使誰辦。只要老爺外面還是這樣清名聲原好,裏頭的委屈只要奴才辦去,關礙不着老爺的。奴才跟主兒一場,到底也要掏出忠心來。”賈政被李十兒一番言語,說得心無主見,道:“我是要保性命的,你們鬧出來不與我相干。”說着,便踱了進去。
\end{parag}


\begin{parag}
    李十兒便自己做起威福,鉤連內外一氣的哄着賈政辦事,反覺得事事周到,件件隨心。所以賈政不但不疑,反多相信。便有幾處揭報,上司見賈政古樸忠厚,也不查察。惟是幕友們耳目最長,見得如此,得便用言規諫,無奈賈政不信,也有辭去的,也有與賈政相好在內維持的。於是漕務事畢,尚無隕越。
\end{parag}


\begin{parag}
    一日,賈政無事,在書房中看書。簽押上呈進一封書子,外面官封上開着:“鎮守海門等處總制公文一角,飛遞江西糧道衙門。”賈政拆封看時,只見上寫道:
\end{parag}


\begin{qute2sp}
    金陵契好,桑梓情深。昨歲供職來都,竊喜常依座右。仰蒙雅愛,許結朱陳,至今佩德勿諼。祗因調任海疆,未敢造次奉求,衷懷歉仄,自嘆無緣。今幸戟戟遙臨,快慰平生之願。正申燕賀,先蒙翰教,邊帳光生,武夫額手。雖隔重洋,尚叨樾蔭。想蒙不棄卑寒,希望蔦蘿之附。小兒已承青盼,淑媛素仰芳儀。如蒙踐諾,即遣冰人。途路雖遙,一水可通。不敢雲百輛之迎,敬備仙舟以俟。茲修寸幅,恭賀升祺,並求金允。臨穎不勝待命之至。世弟周瓊頓首。
\end{qute2sp}


\begin{parag}
    賈政看了,心想:“兒女姻緣果然有一定的。舊年因見他就了京職,又是同鄉的人,素來相好,又見那孩子長得好,在席間原提起這件事。因未說定,也沒有與他們說起。後來他調了海疆,大家也不說了。不料我今升任至此,他寫書來問。我看起門戶卻也相當,與探春到也相配。但是我並未帶家眷,只可寫字與他商議。”正在躊躇,只見門上傳進一角文書,是議取到省會議事件。賈政只得收拾上省,侯節度派委。
\end{parag}


\begin{parag}
    一日在公館閒坐,見桌上堆着一堆字紙,賈政一一看去,見刑部一本:“爲報明事,會看得金陵籍行商薛蟠——”賈政便喫驚道:“了不得,已經提本了!”隨用心看下去,是“薛蟠毆傷張三身死,串囑屍證捏供誤殺一案。”賈政一拍桌道:“完了!”只得又看,底下是:
\end{parag}


\begin{qute2sp}
    據京營節度使諮稱:緣薛蟠籍隸金陵,行過太平縣,在李家店歇宿,與店內當槽之張三素不相認,於某年月日薛蟠令店主備酒邀請太平縣民吳良同飲,令當槽張三取酒。因酒不甘,薛蟠令換好酒。張三因稱酒已沽定難換。薛蟠因伊倔強,將酒照臉潑去,不期去勢甚猛,恰值張三低頭拾箸,一時失手,將酒碗擲在張三囟門,皮破血出,逾時殞命。李店主趨救不及,隨向張三之母告知。伊母張王氏往看,見已身死,隨喊稟地保赴縣呈報。前署縣詣驗,仵作將骨破一寸三分及腰眼一傷,漏報填格,詳府審轉。看得薛蟠實系潑酒失手,擲碗誤傷張三身死,將薛蟠照過失殺人,準鬥殺罪收贖等因前來。臣等細閱各犯證屍親前後供詞不符,且查《鬥殺律》注云:“相爭爲鬥,相打爲毆。必實無爭鬥情形,邂逅身死,方可以過失殺定擬。”應令該節度審明實情,妥擬具題。今據該節度疏稱:薛蟠因張三不肯換酒,醉後拉着張三右手,先毆腰眼一拳。張三被毆回罵,薛蟠將碗擲出,致傷囟門深重,骨碎腦破,立時殞命。是張三之死實由薛蟠以酒碗砸傷深重致死,自應以薛蟠擬抵。將薛蟠依《鬥殺律》擬絞監侯,吳良擬以杖徒。承審不實之府州縣應請……以下注着“此稿未完”。
\end{qute2sp}


\begin{parag}
    賈政因薛姨媽之託曾託過知縣,若請旨革審起來,牽連着自己,好不放心。即將下一本開看,偏又不是。只好翻來覆去將報看完,終沒有接這一本的。心中狐疑不定,更加害怕起來。正在納悶,只見李十兒進來:“請老爺到官廳伺侯去,大人衙門已經打了二鼓了。”賈政只是發怔,沒有聽見。李十兒又請了一遍。賈政道:“這便怎麼處?”李十兒道:“老爺有什麼心事?”賈政將看報之事說了一遍。李十兒道:“老爺放心。若是部裏這麼辦了,還算便宜薛大爺呢。奴才在京的時侯聽見,薛大爺在店裏叫了好些媳婦,都喝醉了生事,直把個當槽兒的活活打死的。奴才聽見不但是託了知縣,還求璉二爺去花了好些錢各衙門打通了才提的。不知道怎麼部裏沒有弄明白。如今就是鬧破了,也是官官相護的,不過認個承審不實革職處分罷,那裏還肯認得銀子聽情呢。老爺不用想,等奴才再打聽罷。不要誤了上司的事。”賈政道:“你們那裏知道,只可惜那知縣聽了一個情,把這個官都丟了,還不知道有罪沒有呢。”李十兒道:“如今想他也無益,外頭伺侯着好半天了,請老爺就去罷。”賈政不知節度傳辦何事,且聽下回分解。
\end{parag}