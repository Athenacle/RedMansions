\chap{一十七}{會芳園試才題對額 賈寶玉機敏動諸賓}

\begin{parag}
    \begin{note}庚:此回宜分二回方妥。\end{note}\begin{subnote}按:己本與庚本第十七、十八回尚未分回。\end{subnote}
\end{parag}


\begin{parag}
    \begin{note}戚本:寶玉系諸豔之貫,故大觀園對額必得玉兄題跋,且暫題燈匾聯上,再請賜題,此千妥萬當之章法。\end{note}
\end{parag}


\begin{parag}
    詩曰:
\end{parag}

\begin{poem}
    \begin{pl}豪華雖足羨,離別卻難堪。\end{pl}

    \begin{pl}博得虛名在,誰人識苦甘?\end{pl}
    \begin{note}庚側:好詩,全是諷刺。近之諺雲:“又要馬兒好,又要馬兒不喫草。”真罵盡無厭貪癡之輩。\end{note}
\end{poem}


\begin{parag}
    話說秦鍾既死,寶玉痛哭不已,李貴等好容易勸解半日方住,歸時猶是悽惻哀痛。賈母幫了幾十兩銀子,外又備奠儀,寶玉去弔紙。七日後便送殯掩埋了,別無記述。只有寶玉日日思慕感悼,然亦無可如何了。\begin{note}庚雙夾:每於此等文後使用此語作結,是板定大章法,亦是此書大旨。\end{note}
\end{parag}


\begin{parag}
    又不知歷過幾日何時,\begin{note}庚側:慣用此等章法。\end{note}\begin{note}庚雙夾:年表如此寫,亦妙!\end{note}這日賈珍等來回賈政:“園內工程俱已告竣,大老爺已瞧過了,只等老爺瞧了,或有不妥之處,再行改造,好題匾額對聯的。”賈政聽了,沉思一回,說道:“這匾額對聯倒是一件難事。論理該請貴妃賜題纔是,然貴妃若不親睹其景,大約亦必不肯妄擬;若直待貴妃遊幸過再請題,偌大景緻,若干亭榭,無字標題,也覺寥落無趣,任有花柳山水,也斷不能生色。”衆清客在旁笑答道:“老世翁所見極是。如今我們有個愚見:各處匾額對聯斷不可少,亦斷不可定名。如今且按其景緻,或兩字、三字、四字,虛合其意,擬了出來,暫且做出燈匾聯懸了。待貴妃遊幸時,再請定名,豈不兩全?”賈政等聽了,都道:“所見不差。我們今日且看看去,只管題了,若妥當便用;不妥時,然後將雨村請來,令他再擬。”\begin{note}庚雙夾:點雨村,照應前文。\end{note}衆人笑道:“老爺今日一擬定佳,何必又待雨村。”賈政笑道:“你們不知,我自幼於花鳥山水題詠上就平平;\begin{note}庚側:是紗帽頭口氣。\end{note}如今上了年紀,且案牘紛煩,於這怡情悅性文章上更生疏了,縱擬了出來,不免迂腐古板,反不能使花柳園亭生色,似不妥協,反沒意思。”\begin{note}庚眉:政老情字如此寫。壬午季春。畸笏。\end{note}衆清客笑道:“這也無妨。我們大家看了公擬,各舉其長,優則存之,劣則刪也,未爲不可。”賈政道:“此論極是。且喜今日天氣和暖,大家去逛逛。”\begin{note}庚雙夾:音光,字去聲,出《諧聲字箋》。\end{note}說著起身,引衆人前往。
\end{parag}


\begin{parag}
    賈珍先去園中知會衆人。可巧近日寶玉因思念秦鍾,憂戚不盡,賈母常命人帶他到園中來戲耍。\begin{note}庚側:現成榫楔,一絲不費力。若特喚出寶玉來,則成何文字?\end{note}此時亦才進去,忽見賈珍走來,向他笑道:“你還不出去,老爺就來了。”寶玉聽了,帶著奶孃小廝們,一溜煙就出園來。\begin{note}庚側:不肖子弟來看形容。餘初看之,不覺怒焉,蓋謂作者形容餘幼年往事,因思彼亦自寫其照,何獨餘哉?信筆書之,供諸大衆同一發笑。\end{note}方轉過彎,頂頭賈政引衆客來了,躲之不及,只得一邊站了。賈政近日因聞得塾掌稱讚寶玉專能對對聯,雖不喜讀書,偏倒有些歪才情似的,\begin{note}蒙側:如此順寫,筆間寫來,然卻是寶玉正傳。\end{note}今日偶然撞見這機會,便命他跟來。\begin{note}庚雙夾:如此偶然方妙,若特特喚來題額,真不成文矣。\end{note}寶玉只得隨往,尚不知何意。
\end{parag}


\begin{parag}
    賈政剛至園門前,只見賈珍帶領許多執事人來,一旁侍立。賈政道:“你且把園門都關上,我們先瞧了外面再進去。”\begin{note}庚雙夾:是行家看法。\end{note}賈珍聽說,命人將門關了。賈政先秉正看門。只見正門五間,上面桶瓦泥鰍脊;那門欄窗隔,皆是細雕新鮮花樣,並無朱粉塗飾;一色水磨羣牆,\begin{note}庚雙夾:門雅,牆雅,不落俗套。\end{note}下面白石臺磯,鑿成西番草花樣。左右一望,皆雪白粉牆,下面虎皮石,隨勢砌去,果然不落富麗俗套,自是歡喜。遂命開門,只見迎門一帶翠嶂擋在前面。\begin{note}庚雙夾:掩映好極。\end{note}衆清客都道:“好山,好山!”賈政道:“非此一山,一進來園中所有之景悉入目中,則有何趣。”衆人道:“極是。非胸中大有邱壑,焉想及此。”說著,往前一望,見白石崚嶒,\begin{note}庚雙夾:想入其中,一時難辯方向。用“前”“後”“這邊”“那邊”等字,正是不辨東西。\end{note}或如鬼怪,或如猛獸,縱橫拱立,上面苔蘚成斑,藤蘿掩映,\begin{note}庚雙夾:曾用兩處舊有之園所改,故如此寫方可,細極。\end{note}其中微露羊腸小徑,\begin{note}庚雙夾:好景界,山子野精於此技。此是小徑,非行車蔫通道,今賈政原欲遊覽其景,故指此等處寫之。想其通路大道,自是堂堂冠冕氣象,無庸細寫者也。後於省親之時已得知矣。\end{note}賈政道:“我們就從此小徑游去,回來由那一邊出去,方可遍覽。”
\end{parag}


\begin{parag}
    說畢,命賈珍在前引導,自己扶了寶玉,逶迤進入山口。\begin{note}庚側:寶玉此刻已料定吉多兇少。\end{note}\begin{note}庚雙夾:此回乃一部之綱緒,不得不細寫,尤不可不細批註。蓋後文十二釵書,出入來往之境,方不能錯亂,觀者亦如身臨足到矣。今賈政雖進的是正門。卻行的是僻路,按此一大園,羊腸鳥道不止幾百十條,穿東度西,臨山過水,萬勿以今日賈政所行之徑,考其方向基址。故正殿反於末後寫之,足見未由大道而往,乃逶迤轉折而經也。\end{note}抬頭忽見山上有鏡面白石一塊,\begin{note}庚側:新奇。\end{note}正是迎面留題處。\begin{note}庚雙夾:留題處便精,不必限定鑿 鏤銀一色惡俗,賴及棗梨之力。\end{note}賈政回頭笑道:“諸公請看,此處題以何名方妙?”衆人聽說,也有說該題“疊翠”二字,也有說該題“錦嶂”的,又有說“賽香爐”的,又有說“小終南”的,種種名色,不止幾十個。原來衆客心中早知賈政要試寶玉的功業進益何如,只將些俗套來敷衍。寶玉亦料定此意。\begin{note}庚雙夾:補明好。\end{note}賈政聽了,便回頭命寶玉擬來。寶玉道:“嘗聞古人有云:‘編新不如述舊,刻古終勝雕今。’\begin{note}庚雙夾:未聞古人說此兩句,卻又似有者。\end{note}況此處並非主山正景,原無可題之處,不過是探景一進步耳。\begin{note}庚雙夾:此論卻是。\end{note}莫如直書‘曲徑通幽處’這舊句舊詩在上,倒還大方氣派。”衆人聽了,都讚道:“是極!二世兄天分高,才情遠,不似我們讀腐了書的。”賈政笑道:“不可謬獎。他年小,不過以一知充十知用,取笑罷了。再俟選擬。”
\end{parag}


\begin{parag}
    說著,進入石洞來,只見佳木籠蔥,奇花熌灼,一帶清流,從花木深處曲折瀉於石隙之下。 \begin{note}庚雙夾:這水是人力引來做的。\end{note} 再進數步,漸向北邊, \begin{note}庚雙夾:細極。後文所以雲進賈母臥房後之角門,是諸釵日相來往之境也。後文又云,諸釵所居之處,只在西北一帶,最近賈母臥室之後,皆從此“北”字而來。\end{note} 平坦寬豁,兩邊飛樓插空,雕甍繡檻,皆隱於山坳樹杪之間。俯而視之,則清溪瀉雪,石磴穿雲, \begin{note}庚雙夾:前已寫山至寬處,此則由低至高處,各景皆遍。\end{note} 白石爲欄,環抱池沿,石橋三港,獸面銜吐。橋上有亭。 \begin{note}庚雙夾:前已寫山寫石,今則寫池寫樓,各景皆遍。\end{note} 賈政與諸人上了亭子,倚欄坐了, \begin{note}庚雙夾:此亭大抵四通八達,爲諸小徑之咽喉要路。\end{note} 因問:“諸公以何題此?”諸人都道:“當日歐陽公《醉翁亭記》有云:‘有亭翼然。’就名‘翼然’。”賈政笑道:“‘翼然’雖佳,但此亭壓水而成,還須偏於水題方稱。依我拙裁,歐陽公之‘瀉出於兩峯之間’,竟用他這一個‘瀉’字。”有一客道:“是極,是極。竟是‘瀉玉’二字妙。”賈政拈髯尋思,因抬頭見寶玉侍側,便笑命他也擬一個來。寶玉聽說,連忙回道:“老爺方纔所議已是。但是如今追究了去,似乎當日歐陽公題釀泉用一‘瀉’字則妥,今日此泉若亦用‘瀉’字,則覺不妥。況此處雖爲省親駐蹕別墅,亦當入於應制之例,用此等字眼,亦覺粗陋不雅。求再擬較此蘊藉含蓄者。”賈政笑道:“諸公聽此論若如?方纔衆人編新,你又說不如述古;如今我們述古,你又說粗陋不妥。你且說你的來我聽。”寶玉道:“有用‘瀉玉’二字,則莫若‘沁芳’ \begin{note}庚側:真新雅。\end{note} 二字, \begin{note}庚雙夾:果然。\end{note} 豈不新雅?”賈政拈髯點頭不語。 \begin{note}庚眉:六字是嚴父大露悅容也。壬午春。\end{note} 衆人都忙迎合,贊寶玉才情不凡。賈政道:“匾上二字容易,再作一副七言對聯來。”寶玉聽說,立於亭上,四顧一望,便機上心來,乃念道:
\end{parag}


\begin{poem}
    \begin{pl}繞堤柳借三篙翠,\end{pl}
    \begin{note}庚雙夾:要緊,貼切水字。\end{note}

    \begin{pl}隔岸花分一脈香。\end{pl}
    \begin{note}庚雙夾:恰極,工極!綺靡秀媚,香奩正體。\end{note}
\end{poem}


\begin{parag}
    賈政聽了,點頭微笑。衆人先稱讚不已。
\end{parag}


\begin{parag}
    於是出亭過池,一山一石,一花一木,莫不著意觀覽。\begin{note}庚雙夾:渾寫兩句,已見經行處愈遠,更至北一路矣。\end{note}忽抬頭看見前面一帶粉垣,裏面數楹修舍,有千百竿翠竹遮映。衆人都道:“好個所在!”\begin{note}庚側:此方可爲顰兒之居。\end{note}於是大家進入,只見入門便是曲折遊廊,\begin{note}庚雙夾:不犯超手遊廊。\end{note}階下石子漫成甬路。上面小小兩三間房舍,一明兩暗,裏面都是合著地步打就的牀几椅案。從裏間房內又得一小門,出去則是後院,有大株梨花兼著芭蕉。又有兩間小小退步。後院牆下忽開一隙,得泉一派,開溝僅尺許,灌入牆內,繞階緣屋至前院,盤旋竹下而出。
\end{parag}


\begin{parag}
    賈政笑道:“這一處還罷了。\begin{note}庚側:一處。\end{note}若能月夜坐此窗下讀書,不枉虛生一世。”說畢,看著寶玉,唬的寶玉忙垂了頭。\begin{note}庚雙夾:點一筆。\end{note}衆客忙用話開釋,\begin{note}庚雙夾:客不可不有。\end{note}又說道:“此處的匾該題四個字。”賈政笑問:“那四字?”一個道是“淇水遺風。”賈政道:“俗。”\begin{note}庚雙夾:餘亦如此。\end{note}又一個是“睢園遺蹟” 。賈政道:“也俗。”賈珍笑道:“還是寶兄弟擬一個來。”\begin{note}庚眉:又換一章法。壬午春。\end{note}賈政道:“他未曾作,先要議論人家的好歹,可見就是個輕薄人。”\begin{note}庚側:知子者莫如父。\end{note}衆客道:“議論的極是,其奈他何。”賈政道:“休如此縱了他。”因命他道:“今日任你狂爲亂道,先設議論來,然後方許你作。\begin{note}庚雙夾:又一格式,不然,不獨死板,且亦大失嚴父素體。\end{note}\begin{note}庚眉:於作詩文時雖政老亦有如此令旨,可知嚴父亦無可奈何也。不學紈絝來看。畸笏。\end{note}方纔衆人說的,可有使得的?”寶玉見問,答道:“都似不妥。”\begin{note}庚雙夾:明知是故意要他搬駁議論,落得肆行施展。\end{note}賈政冷笑道:“怎麼不妥?”寶玉道:“這是第一處行幸之處,必須頌聖方可。若用四字的匾,又有古人現成的,何必再作。”賈政道:“難道‘淇水’‘睢園’不是古人的?”寶玉道:“這太板腐了。莫若‘有鳳來儀’四字。”\begin{note}庚雙夾:果然,妙在雙關暗合。\end{note}衆人都鬨然叫妙。賈政點頭道:“畜生,畜生,可謂‘管窺蠡測’矣。”因命:“再題一聯來。”寶玉便念道:
\end{parag}


\begin{poem}
    \begin{pl}寶鼎茶閒煙尚綠,\end{pl}
    \begin{note}庚雙夾:“尚”字妙極!不必說竹,然恰恰是竹中精舍。\end{note}

    \begin{pl}幽窗棋罷指猶涼。\end{pl}
    \begin{note}庚雙夾:“猶”字妙!“尚綠”、“猶涼”四字,便如置身於森森萬竿之中。\end{note}
\end{poem}


\begin{parag}
    賈政搖頭說道:“也未見長。”說畢,引衆人出來。
\end{parag}


\begin{parag}
    方欲走時,忽又想起一事來,\begin{note}己側:不板。\end{note}因問賈珍道:“這些院落房宇並几案桌椅都算有了,\begin{note}庚側:此一頓少不得。\end{note}還有那些帳幔簾子並陳設玩器古董,可也都是一處一處合式配就的?”\begin{note}庚雙夾:大篇長文不如此頓,則成何話說?\end{note}賈珍回道:“那陳設的東西早已添了許多,自然臨期合式陳設。帳幔簾子,昨日聽見璉兄弟說,還不全。那原是一起工程之時就畫了各處的圖樣,量準尺寸,就打發人辦去的。想必昨日得了一半。”\begin{note}庚雙夾:補出近日忙冗,千頭萬緒景況。\end{note}賈政聽了,便知此事不是賈珍的首尾,便令人去喚賈璉。
\end{parag}


\begin{parag}
    一時賈璉趕來。\begin{note}庚雙夾:寫出忙冗景況。\end{note}賈政問他共有幾種,現今得了幾種,尚欠幾種。賈璉見問,忙向靴桶取靴掖內裝的一個紙折略節來,\begin{note}庚雙夾:細極!從頭至尾,誓不作一筆逸安苟且之筆。\end{note}看了一看,回道:“妝蟒繡堆、\begin{note}庚雙夾:一字一句。\end{note}刻絲彈墨\begin{note}庚雙夾:二字一句。\end{note}並各色綢綾大小幔子一百二十架,昨日得了八十架,下欠四十架。簾子二百掛,昨日俱得了。外有猩猩氈簾二百掛,金絲藤紅漆竹簾二百掛,墨漆竹簾二百掛,五彩線絡盤花簾二百掛,每樣得了一半,也不過秋天都全了。椅搭、桌圍、牀裙、桌套,每分一千二百件,也有了。”
\end{parag}


\begin{parag}
    一面走,一面說,\begin{note}庚雙夾:是極!\end{note}倏爾青山斜阻。\begin{note}庚雙夾:“斜”字細,不必拘定方向。諸釵所居之處,若稻香村、瀟湘館、怡紅院、秋爽齋、蘅蕪苑等,都相隔不遠,究竟只在一隅。然處置得巧妙,使人見其千邱萬壑,恍然不知所窮,所謂會心處不在乎遠。大抵一山一水,一木一石,全在人之穿插佈置耳。\end{note}轉過山懷中,隱隱露出一帶黃泥築就牆,牆頭上皆稻莖掩護。\begin{note}庚雙夾:配的好!\end{note}有幾百株杏花,如噴火蒸霞一般。裏面數楹茅屋。外面卻是桑、 榆、槿、柘,各色樹稚新條,隨其曲折,編就兩溜青籬。籬外山坡之下,有一土井,旁有桔槔轆轤之屬。下面分畦列畝,佳蔬菜花,漫然無際。\begin{note}庚雙夾:閱至此,又笑別部小說中,一方個花園中,皆是牡丹亭、芍藥圃、雕欄畫揀、瓊榭朱樓,略不差別。\end{note}
\end{parag}


\begin{parag}
    賈政笑道:“倒是此處有些道理。固然系人力穿鑿,此時一見,未免勾引起我歸農之意。\begin{note}庚雙夾:極熱中偏以冷筆點之,所以爲妙。\end{note}我們且進去歇息歇息。”說畢,方欲進籬門去,忽見路旁有一石碣,亦爲留題之備。\begin{note}庚側:真妙真新。\end{note}\begin{note}庚雙夾:更恰當。若有懸額之處,或再用鏡面石,豈覆成文哉?忽想到“石碣”二字,又托出許多郊野氣色來,一肚皮千邱萬壑,只在這石碣上。\end{note}衆人笑道:“更妙,更妙!此處若懸匾待題,則田舍家風一洗盡矣。立此一碣,又覺生色許多,非範石湖田家之詠不足以盡其妙。”\begin{note}庚側:贊得是,這個蔑翁有些意思。\end{note}\begin{note}庚雙夾:客不可不養。\end{note}賈政道:“諸公請題。”衆人道:“方纔世兄有云,‘編新不如述舊’,此處古人已道盡矣,莫若直書‘杏花村’妙極。”賈政聽了,笑向賈珍道:“正虧提醒了我。此處都妙極,只是還少一個酒幌,明日竟作一個,不必華麗,就依外面村莊的式樣作來,用竹竿挑在樹梢。”賈珍答應了,又回道:“此處竟還不可養別的雀鳥,只是買些鵝鴨雞類,才都相稱了。”賈政與衆人都道:“更妙。”賈政又向衆人道:“‘杏花村’固佳,只是犯了正名,村名直待請名方可。”衆客都道:“是呀。如今虛的,便是什麼字樣好?”大家想著,寶玉卻等不得了,\begin{note}庚雙夾:又換一格方不板。\end{note}也不等賈政的命,\begin{note}庚雙夾:忘情有理。\end{note}便說道:“舊詩云:‘紅杏梢頭掛酒旗。’如今莫若‘杏簾在望’\begin{note}庚雙夾:妙在一“在”字。\end{note}四字。”衆人都道:“好個‘在望’!又暗合‘杏花村’意。”寶玉冷笑道:\begin{note}庚雙夾:忘情最妙。\end{note}“村名若用‘杏花’二字,則俗陋不堪了。又有古人詩云:‘柴門臨水稻花香。’何不就用‘稻香村’的妙?”衆人聽了,亦發哄聲拍手道:“妙!”賈政一聲喝斷:“無知的業障!\begin{note}庚眉:愛之至,喜之至,故作此語。作者至此,寧不笑殺?壬午春。\end{note}你能知道幾個古人,能記得幾首熟詩,也敢在老先生前賣弄!你方纔那些胡說的,不過是試你的清濁,取笑而已,你就認真了!”說著,引衆人步入茆堂,裏面紙窗木榻,富貴氣象一洗皆盡。賈政心中自是喜歡,卻瞅寶玉道:“此處如何?”衆人見問,都忙悄悄的推寶玉,教他說好。寶玉不聽人言,便應聲道:“不及‘有鳳來儀 ’多矣。”\begin{note}庚雙夾:公然自定名,妙!\end{note}賈政聽了道:“無知的蠢物!你只知朱樓畫棟,惡賴富麗爲佳,那裏知道這清幽氣象。終是不讀書之過!”寶玉忙答道:“老爺教訓的固是,但古人常雲‘天然’二字,不知何意?”
\end{parag}


\begin{parag}
    衆人見寶玉牛心,都怪他呆癡不改。今見問“天然”二字,衆人忙道:“別的都明白,爲何連‘天然’不知?‘天然’者,天之自然而有,非人力之所成也。”寶玉道:“卻又來!此處置一田莊,分明見得人力穿鑿扭捏而成。遠無鄰村,近不負郭,背山山無脈,臨水水無源,高無隱寺之塔,下無通市之橋,峭然孤出,似非大觀。爭似先處有自然之理,得自然之氣,雖種竹引泉,亦不傷於穿鑿。古人云‘天然圖畫’四字,正畏非其地而強爲其地,非其山而強爲其山,雖百般精而終不相宜……”未及說完,賈政氣的喝命:“叉出去!”剛出去,又喝命:“回來!”命再題一聯:“若不通,一併打嘴!”\begin{note}庚眉:所謂奈何他不得也,呵呵!畸笏。\end{note}寶玉只得念道:
\end{parag}


\begin{poem}
    \begin{pl} 新漲綠添浣葛處,\end{pl}
    \begin{note}庚雙夾:採《詩》頌聖最恰當。\end{note}

    \begin{pl} 好雲香護採芹人。\end{pl}
    \begin{note}庚雙夾:採《風》採《雅》都恰當。然冠冕中又不失香奩格調。\end{note}
\end{poem}


\begin{parag}
    賈政聽了,搖頭說:“更不好。”一面引人出來,轉過山坡,穿花度柳,撫石依泉,過了茶蘼架,再入木香棚,越牡丹亭,度芍藥圃,入薔薇院,出芭蕉 塢,盤旋曲折。\begin{note}庚雙夾:略用套語一束,與前頓破格不板。\end{note}忽聞水聲潺湲,瀉出石洞,上則蘿薜倒垂,下則落花浮蕩。\begin{note}庚雙夾:仍是沁芳溪矣,究竟基址不大,全是曲折掩映之巧可知。\end{note}衆人都道:“好景,好景!”賈政道:“諸公題以何名?”衆人道:“再不必擬了,恰恰乎是‘武陵源’三個字。”賈政笑道:“又落實了,而且陳舊。”衆人笑道:“不然就用‘秦人舊舍’四字也罷了。”寶玉道:“這越發過露了。‘秦人舊舍’說避亂之意,如何使得?莫若‘蓼汀花漵’四字。”賈政聽了,更批胡說。
\end{parag}


\begin{parag}
    於是要進港洞時,又想起有船無船。賈珍道:“採蓮船共四隻,座船一隻,如今尚未造成。”賈政笑道:“可惜不得入了。”賈珍道:“從山上盤道亦可進去。” 說畢,在前導引,大家攀藤撫樹過去。只見水上落花愈多,其水愈清,溶溶蕩蕩,曲折縈迂。池邊兩行垂柳,雜著桃杏,遮天蔽日,真無一些塵土。忽見柳陰中又露出一個折帶朱欄板橋來,\begin{note}庚雙夾:此處才見一朱粉字樣,綠柳紅橋,此等點綴亦不可少。後文寫蘆雪廣\end{note}\begin{subnote}按:廣,音眼。就山築成之房屋。韓愈《陪杜侍御遊湘西兩寺》詩:“剖竹走泉源,開廊架崖广。”各本或作“庵”“庭”“廬”,皆非。今從庚本改。\end{subnote}\begin{note}則曰蜂腰板橋,都施之得宜,非一幅死稿也。\end{note}度過橋去,諸路可通,\begin{note}庚雙夾:補四字,細極!不然,後文寶釵來往,則將日日爬山越嶺矣。記清此處,則知後文寶玉所行常徑,非此處也。\end{note}便見一所清涼瓦舍,一色水磨磚牆,清瓦花堵。那大主山所分之脈,\begin{note}庚雙夾:兩見大主山,稻香村又云懷中,不寫主山,而主山處處映帶連絡不斷可知矣。\end{note}皆穿牆而過。\begin{note}庚雙夾:好想。\end{note}
\end{parag}


\begin{parag}
    賈政道:“此處這所房子,無味的很。”\begin{note}庚雙夾:先故頓此一筆,使後文愈覺生色,未揚先抑之法。蓋釵、顰對峙有甚難寫者。\end{note}因而步入門時,忽迎面突出插天的大玲瓏山石來,四面羣繞各式石塊,竟把裏面所有房屋悉皆遮住,而且一株花木也無。\begin{note}庚雙夾:更奇妙!\end{note}只見許多異草:或有牽藤的,或有引蔓的,或垂山巔,或穿石隙,甚至垂檐繞柱,縈砌盤階,\begin{note}庚雙夾:更妙?\end{note}或如翠帶飄搖,或如金繩盤屈,或實若丹砂,或花如金桂,味芬氣馥,非花香之可比。\begin{note}庚雙夾:前三處皆還在人意之中,此一處則今古書中未見之工程也。連用幾“或”字,是從昌黎《南山詩》中學得。\end{note}賈政不禁笑道:“有趣!\begin{note}庚雙夾:前有“無味”二字,及雲“有趣”二字,更覺生色,更覺重大。\end{note}只是不大認識。”有的說:“是薜荔藤蘿。” 政道:“薜荔藤蘿不得如此異香。”寶玉道:“果然不是。這些之中也有藤蘿薜荔。那香的是杜若蘅蕪,那一種大約是茝蘭,這一種大約是清葛,那一種是金簦草,這一種是玉蕗藤,紅的自然是紫芸,綠的定是青芷。\begin{note}庚雙夾:金簦草,見《字彙》。玉蕗,見《楚辭》“菎蕗雜於黀蒸”。茝、葛、芸、芷,皆不必注,見者太多。此書中異物太多,有人生之未聞未見者,然實系所有之物,或名差理同者亦有之。\end{note}想來《離騷》《文選》等書上所有的那些異草,也有叫作什麼藿蒳姜蕁的,也有叫什麼綸組紫絳的,還有石帆、水松、扶留等樣,\begin{note}庚雙夾:左太沖《吳都賦》。\end{note}又有叫作什麼綠荑的,還有什麼丹椒、蘼蕪、風連。\begin{note}庚雙夾:以上《蜀都賦》。\end{note}如今年深歲改,人不能識,故皆象形奪名,漸漸的喚差了,也是有的。”\begin{note}庚雙夾:自實注一筆,妙!\end{note}未及說完,賈政喝道:“誰問你來!”\begin{note}庚雙夾:又一樣止法。\end{note}唬的寶玉倒退,不敢再說。
\end{parag}


\begin{parag}
    賈政因見兩邊俱是超手遊廊,便順著遊廊步入。只見上面五間清廈連著捲棚,四面出廊,綠窗油壁,更比前幾處清雅不同。賈政嘆道:“此軒中煮茶操琴,亦不必再焚香矣。\begin{note}庚雙夾:前二處,一曰“月下讀書”,一曰“勾引起歸農之意”,此則“操琴煮茶”,斷語皆妙。\end{note}此造已出意外,諸公必有佳作新題以顏其額,方不負此。”衆人笑道:“再莫若‘蘭風蕙露’貼切了。”賈政道:“也只好用這四字。其聯若何?”一人道:“我倒想了一對,大家批削改正。”念道是:
\end{parag}


\begin{poem}
    \begin{pl}    麝蘭芳靄斜陽院,\end{pl}

    \begin{pl}    杜若香飄明月洲。\end{pl}
\end{poem}


\begin{parag}
    衆人道:“妙則妙矣,只是‘斜陽’二字不妥。”那人道:“古人詩云:‘蘼蕪滿手泣斜暉’。”衆人道:“頹喪,頹喪 ”又一人道:“我也有一聯,諸公評閱評閱。”因念道:
\end{parag}


\begin{poem}
    \begin{pl}三徑香風飄玉蕙,\end{pl}

    \begin{pl}一庭明月照金蘭。\end{pl}
    \begin{note}庚雙夾:此二聯皆不過爲釣寶玉之餌,不必認真批評。\end{note}
\end{poem}


\begin{parag}
    賈政拈髯沉吟,意欲也題一聯。忽抬頭見寶玉在旁不敢則聲,因喝道:“怎麼你應說話時又不說了?還要等人請教你不成!”寶玉聽說,便回道:“此處並沒有什麼‘蘭麝’、‘明月’、‘洲渚’之類,若要這樣著跡說來,就題二百聯也不能完。”賈政道:“誰按著你的頭,叫你必定說這些字樣呢?”寶玉道:“如此說,匾上則莫若‘蘅芷清芬’四字。對聯則是:
\end{parag}


\begin{poem}
    \begin{pl}吟成豆蔻詩猶豔,\end{pl}
    \begin{pl}睡足荼蘼夢亦香。\end{pl}\begin{note}庚雙夾:實佳。\end{note}
\end{poem}


\begin{parag}
    賈政笑道:“這是套的‘書成蕉葉文猶綠’,不足爲奇。”衆客道:“李太白‘鳳凰臺’之作,全套‘黃鶴樓’,\begin{note}庚側:這一位蔑翁更有意思。\end{note}只要套得妙。如今細評起來,方纔這一聯,竟比‘書成蕉葉’尤覺幽嫺活潑。視‘書成’之句,竟似套此而來。”賈政笑說:“豈有此理!”
\end{parag}


\begin{parag}
    說著,大家出來。行不多遠,則見崇閣巍峨,層樓高起,面面琳宮合抱,迢迢複道縈紆,青松拂檐,玉蘭繞砌,金輝獸面,彩煥螭頭。賈政道:“這是正殿了。\begin{note}庚雙夾:想來此殿在園之正中。按園不是殿方之基,西北一帶通賈母臥室後,可知西北一帶是多寬出一帶來的,諸釵始便於行也。\end{note}只是太富麗了些。”衆人都道:“要如此方是。雖然貴妃崇尚節儉,天性惡繁悅樸,\begin{note}庚側:寫出賈妃身分天性。\end{note}然今日之尊,禮儀如此,不爲過也。”一面說,一面走,只見正面\begin{note}庚雙夾:正面,細。\end{note}現出一座玉石牌坊來,上面龍蟠螭護,玲瓏鑿就。賈政道:“此處書以何文?”衆人道:“必是‘蓬萊仙境’方妙。”賈政搖頭不語。寶玉見了這個所在,心中忽有所動,尋思起來,倒像在那裏曾見過的一般,卻一時想不起那年那月日的事了。\begin{note}庚雙夾:仍歸於葫蘆一夢之太虛玄境。\end{note}賈政又命他作題,寶玉只顧細思前景,全無心於此了。衆人不知其意,只當他受了這半日的折磨,精神耗散,才盡辭窮了;再要考難逼迫,著了急,或生出事來,倒不便。遂忙都勸賈政:“罷,罷,明日再題罷了。”賈政心中也怕賈母不放心,\begin{note}庚雙夾:一筆不漏。\end{note}遂冷笑道:“你這畜生,也竟有不能之時了。也罷,限你一日,明日若再不能,我定不饒。這是要緊之處,更要好生作來!”\begin{note}庚眉:一路順順逆逆,已成千邱萬壑之景,若不有此一段大江截住,直成一盆景矣。作者從何落筆著想!\end{note}
\end{parag}


\begin{parag}
    說著,引人出來,再一觀望,原來自進門起,所行至此,才遊了十之五六。\begin{note}庚雙夾:總住,妙!伏下後文所補等處。若都入此回寫完,不獨太繁,使後文冷落,亦且非《石頭記》之筆。\end{note}又值人來回,有雨村處遣人來回話。\begin{note}庚雙夾:又一緊,故不能終局也。此處漸漸寫雨村親切,正爲後文地步。伏脈千里,橫雲斷嶺法。\end{note}賈政笑道:“此數處不能遊了。雖如此,到底從那一邊出去,縱不能細觀,也可稍覽。”說著,引衆客行來,至一大橋前,水如晶簾一般奔入。原來這橋便是通外河之閘,引泉而入者。\begin{note}庚雙夾:寫出水源,要緊之極!近之畫家著意于山,若不講水。又造園圃者,唯知弄莽憨頑石壅笨冢輒謂之景,皆不知水爲先著。此園大概一描,處處未嘗離水,蓋又未寫明水之從來,今終補出,精細之至!\end{note}賈政因問:“此閘何名?”寶玉道:“此乃沁芳泉之正源,就名‘沁芳閘’。”\begin{note}庚雙夾:究竟只一脈,賴人力引導之功,園不易造,景非泛寫。\end{note}賈政道:“胡說!偏不用‘沁芳’二字。”\begin{note}庚雙夾:此以下皆系文終之餘波,收的方不突。\end{note}
\end{parag}


\begin{parag}
    於是一路行來,或清堂茅舍,或堆石爲垣,或編花爲牖,或山下得幽尼佛寺,或林中藏女道丹房,或長廊曲洞,或方廈圓亭,賈政皆不及進去。\begin{note}庚雙夾:伏下櫳翠庵、蘆雪廣、凸碧山莊、凹晶溪館、暖香塢等諸處,於後文一段一段補之,方得雲龍作雨之勢。\end{note}因說半日腿痠,未嘗歇息,忽又見前面又露出一所院落來,\begin{note}庚眉:問卿此居比大荒山若何?\end{note}賈政笑道:“到此可要進去歇息歇息了。”說著,一徑引人繞著碧桃花,\begin{note}庚雙夾:怡紅院如此寫來,用無意之筆,卻是極精細文字。\end{note}穿過一層竹籬花障編就的月洞門,\begin{note}庚雙夾:未寫其居,先寫其境。\end{note}俄見粉牆環護,綠柳周垂。\begin{note}庚雙夾:與“萬竿修竹”遙映。\end{note}賈政與衆人進去,一入門,兩邊都是遊廊相接。院中點襯幾塊山石,一邊種著數本芭蕉;那一邊乃是一顆西府海棠,其勢若傘,綠垂碧縷,葩吐丹砂。衆人讚道:“好花,好花!從來也見過許多海棠,那裏有這樣妙的。”賈政道:“這叫作‘女兒棠’,\begin{note}庚雙夾:妙名。\end{note}乃是外國之種。俗傳系出‘女兒國’中,\begin{note}庚旁批:出自政老口中,奇特之至!\end{note}雲彼國此種最盛,亦荒唐不經之說罷了。”\begin{note}庚側:政老應如此語。\end{note}衆人笑道:“然雖不經,如何此名傳久了?”寶玉道:“大約騷人詠士,以花之色紅暈若施脂,輕弱似扶病,\begin{note}庚雙夾:體貼的切,故形容的妙。\end{note}\begin{note}庚眉:十字若海棠有知,必深深謝之。\end{note}大近乎閨閣風度,所以以‘女兒’命名。想因被世間俗惡聽了,他便以野史纂入爲證,以俗傳俗,以訛傳訛,都認真了。”\begin{note}庚雙夾:不獨此花,近之謬傳者不少,不能悉道,只借此花數語駁盡。\end{note}衆人都搖身贊妙。
\end{parag}


\begin{parag}
    一面說話,一面都在廊外抱廈下打就的榻上坐了。\begin{note}庚雙夾:至階又至檐,不肯輕易寫過。\end{note}賈政因問:“想幾個什麼新鮮字來題此?”一客道:“‘蕉鶴’二字最妙。”又一個道:“‘崇光泛彩’方妙。”賈政與衆人都道:“好個‘崇光泛彩’!”寶玉也道:“妙極。”又嘆:“只是可惜了。”衆人問:“如何可惜?”寶玉道:“此處蕉棠兩植,其意暗蓄‘紅’‘綠’二字在內。若只說蕉,則棠無著落;若只說棠,蕉亦無著落。固有蕉無棠不可,有棠無蕉更不可。”賈政道:“依你如何?”寶玉道:“依我,題‘紅香綠玉’四字,方兩全其妙。”賈政搖頭道:“不好,不好!”
\end{parag}


\begin{parag}
    說著,引人進入房內。只見這幾間房內收拾的與別處不同,竟分不出間隔來的,\begin{note}庚側:特爲青埂峯下淒涼與別處不同耳。庚雙夾:新奇希見之式。\end{note}原來四面皆是雕空玲瓏木板,或“流雲百蝠”,或“歲寒三友”,或山水人物,或翎毛花卉,或集錦,或博古,\begin{note}庚雙夾:花樣周全之極!然必用下文者,正是作者無聊,撰出新異筆墨,使觀者眼目一新。所謂集小說之大成,遊戲筆墨,雕蟲之技,無所不備,可謂善戲者矣。又供諸人同學一戲,洵爲妙極。\end{note}或,\begin{note}前庚雙夾:金玉篆文是可考正籙,今則從俗花樣,真是醒睡魔。其中詩詞雅謎以及各種風俗字文,一概不必究,只據此等處便是一絕。\end{note}各種花樣,皆是名手雕鏤,五彩銷金嵌寶的。\begin{note}庚雙夾:至此方見一朱彩之處,亦必如此式方可。可笑近之園庭,行動便以粉油從事。\end{note}一隔一隔,或有貯書處,或有設鼎處,或安置筆硯處,或供花設瓶、安放盆景處,其隔各式各樣,或天圓地方,或葵花蕉葉,或連環半壁。真是花團錦簇,剔透玲瓏。倏爾五色紗糊就,竟系小窗;倏爾彩綾輕覆,竟系幽戶。\begin{note}庚雙夾:精工之極!\end{note}且滿牆滿壁,皆系隨依古董玩器之形摳成的槽子。諸如琴、劍、懸瓶、\begin{note}庚雙夾:懸於壁上之瓶也。\end{note}桌屏之類,雖懸於壁,卻都是與壁相平的。\begin{note}庚雙夾:皆系人意想不到,日所未見之文,若雲擬編虛想出來,焉能如此?一段極清極細,後文鴛鴦瓶、紫瑪瑙碟、西洋酒令、自行船等文,不必細表。\end{note}衆人都道:“好精緻想頭!難爲怎麼想來?”\begin{note}庚雙夾:誰不如此贊?\end{note}
\end{parag}


\begin{parag}
    原來賈政等走了進來,未進兩層,便都迷了舊路,左瞧也有門可通,右瞧又有窗暫隔,及到了跟前,又被一架書擋住。回頭再走,又有窗紗明透,門徑可行;及至門前,忽見迎面也進來了一羣人,都與自己形相一樣,——卻是一架玻璃大鏡相照。及轉過鏡去,\begin{note}庚側:石兄迷否?\end{note}益發見門子多了。\begin{note}庚側:所謂投投是道是也。\end{note}賈珍笑道:“老爺隨我來。從這門出去,便是後院,從後院出去,倒比先近了。”說著,又轉了兩層紗廚錦隔,果得一門出去,\begin{note}庚側:此方便門也。\end{note}院中滿架薔薇、寶相。轉過花障,則見清溪前阻。衆人吒異:“這股水又是從何而來?”賈珍遙指道:“原從那閘起流至那洞口,從東北山坳裏引到那村莊裏,又開一道岔口,引到西南上,共總流到這裏,仍舊合在一處,\begin{note}庚側:於怡紅總一園之看,是書中大立意。\end{note}從那牆下出去。”衆人聽了,都道:“神妙之極!”說著,忽見大山阻路。衆人都道:“迷了路了。”賈珍笑道:“隨我來。”仍在前導引,衆人隨他,直由山腳邊忽一轉,便是平坦寬闊大路,\begin{note}庚側:衆善歸緣,自然有平坦大道。\end{note}豁然大門前見。\begin{note}庚雙夾:可見前進來是小路徑,此雲忽一轉,便是平坦寬闊之正甬路也,細極!\end{note}衆人都道:“有趣,有趣,真搜神奪巧之至也!”於是大家出來。\begin{note}庚眉:以上可當《大觀園記》。\end{note}那寶玉一心只記掛著裏邊,又不見賈政吩咐,少不得跟到書房。賈政忽想起他來,方喝道:“你還不去?難道還逛不足!\begin{note}庚側:冤哉冤哉!\end{note}也不想逛了這半日,老太太必懸掛著。快進去,疼你也白疼了。”\begin{note}庚雙夾:如此去法,大家嚴父風範,無家法者不知。\end{note}寶玉聽說,方退了出來。
\end{parag}


\begin{parag}
    \begin{note}蒙回末總評:好將富貴回頭看,總有文章如意難。零落機緣君記去,黃金萬斗大觀攤。\end{note}
\end{parag}

