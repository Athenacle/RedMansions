\chap{五十三}{寧國府除夕祭宗祠 榮國府元宵開夜宴}


\begin{parag}
    \begin{note}蒙回前總:除夕祭宗祠一題極博大,開夜宴一題極富麗,擬此二題於一回中,早令人驚心動魄。不知措手處,乃作者偏就寶琴眼中款款敘來。首敘院宇匾對,次敘抱廈匾對,後敘正堂匾對,字字古豔。檻以外,檻以內,是男女分界處;儀門以外,儀門以內,是主僕分界處。獻帛獻爵擇其人,應昭應穆從其諱,是一篇絕大典制。文字最高妙是神主看不真切,一句最苦心是用賈蓉爲檻邊傳蔬人,用賈芷等爲儀門傳蔬人,體貼入微。噫!文心至此,脈絕血枯矣。是知音者。\end{note}
\end{parag}


\begin{parag}
    \begin{note}靖:“祭宗祠”“開夜宴”一番鋪敘,隱後回無限文字。\end{note}
\end{parag}


\begin{parag}
    話說寶玉見晴雯將雀裘補完,已使的力盡神危,忙命小丫頭子來替他捶著,彼此捶打了一會歇下。沒一頓飯的工夫,天已大亮,且不出門,只叫快傳大夫。一時王太醫來了,診了脈,疑惑說道:“昨日已好了些,今日如何反虛微浮縮起來,敢是喫多了飲食?不然就是勞了神思。外感卻倒清了,這汗後失於調養,非同小可。一面說,一面出去開了藥方進來。寶玉看時,已將疏散驅邪諸藥減去了,倒添了茯苓、地黃、當歸等益神養血之劑。寶玉忙命人煎去,一面嘆說:“這怎麼處!倘或有個好歹,都是我的罪孽。”晴雯睡在枕上(口害)道:“好太爺!你幹你的去罷!那裏就得癆病了。”寶玉無奈,只得去了。至下半天,說身上不好就回來了。晴雯此症雖重,幸虧他素習是個使力不使心的;再者素習飲食清淡,飢飽無傷。這賈宅中的風俗祕法,無論上下,只一略有些傷風咳嗽,總以淨餓爲主,次則服藥調養。故於前日一病時,淨餓了兩三日,又謹慎服藥調治,如今勞碌了些,又加倍培養了幾日,便漸漸的好了。近日園中姊妹皆各在房中喫飯,炊爨飲食亦便,寶玉自能變法要湯要羹調停,不必細說。
\end{parag}


\begin{parag}
    襲人送母殯後,業已回來,麝月便將平兒所說宋媽墜兒一事,並晴雯攆逐出去等話,一一也曾回過寶玉。襲人也沒別說,只說太性急了些。只因李紈亦因時氣感冒;邢夫人又正害火眼,迎春岫煙皆過去朝夕侍藥;\begin{note}庚雙夾:妙在一人不落,事事皆到。\end{note}李嬸之弟又接了李嬸和李紋李綺家去住幾日;\begin{note}庚雙夾:來得也有理,去得也有情。\end{note}寶玉又見襲人常常思母含悲,晴雯猶未大愈:因此詩社之日,皆未有人作興,便空了幾社。
\end{parag}


\begin{parag}
    當下已是臘月,離年日近,王夫人與鳳姐治辦年事。王子騰昇了九省都檢點,賈雨村補授了大司馬,協理軍機參贊朝政,不題。
\end{parag}


\begin{parag}
    且說賈珍那邊,開了宗祠,著人打掃,收拾供器,請神主,又打掃上房,以備懸供遺真影像。此時榮寧二府內外上下,皆是忙忙碌碌。這日寧府中尤氏正起來同賈蓉之妻打點送賈母這邊針線禮物,正值丫頭捧了一茶盤押歲錁子進來,回說:“興兒回奶奶,前兒那一包碎金子共是一百五十三兩六錢七分,裏頭成色不等,共總傾了二百二十個錁子。”說著遞上去。尤氏看了看,只見也有梅花式的,也有海棠式的,也有筆錠如意的,也有八寶聯春的。尤氏命:“收起這個來,叫他把銀錁子快快交了進來。”丫鬟答應去了。
\end{parag}


\begin{parag}
    一時賈珍進來喫飯,賈蓉之妻迴避了。\begin{note}庚眉:自可卿死後未見賈蓉續娶,此回有“蓉妻迴避”語,是書中遺漏處。綺園。\end{note}賈珍因問尤氏:“咱們春祭的恩賞可領了不曾?”尤氏道:“今兒我打發蓉兒關去了。”賈珍道:“咱們家雖不等這幾兩銀子使,多少是皇上天恩。早關了來,給那邊老太太見過,置了祖宗的供,上領皇上的恩,下則是託祖宗的福。咱們那怕用一萬銀子供祖宗,到底不如這個又體面,又是霑恩錫福的。除咱們這樣一二家之外,那些世襲窮官兒家,若不仗著這銀子,拿什麼上供過年?真正皇恩浩大,想的周到。”尤氏道:“正是這話。”
\end{parag}


\begin{parag}
    二人正說著,只見人回:“哥兒來了。” 賈珍便命叫他進來。只見賈蓉捧了一個小黃布口袋進來。賈珍道:“怎麼去了這一日。”賈蓉陪笑回說:“今兒不在禮部關領,又分在光祿寺庫上,因又到了光祿寺才領了下來。光祿寺的官兒們都說問父親好,多日不見,都著實想念。”賈珍笑道:“他們那裏是想我。這又到了年下了,不是想我的東西,就是想我的戲酒了。” 一面說,一面瞧那黃布口袋,上有印就是“皇恩永錫”四個大字,那一邊又有禮部祠祭司的印記,又寫著一行小字,
    道是“寧國公賈演榮國公賈源恩賜永遠春祭賞共二分,淨折銀若干兩,某年月日龍禁尉候補侍衛賈蓉當堂領訖,值年寺丞某人”,下面一個硃筆花押。
\end{parag}


\begin{parag}
    賈珍喫過飯,盥漱畢,換了靴帽,命賈蓉捧著銀子跟了來,回過賈母王夫人,又至這邊回過賈赦邢夫人,方回家去,取出銀子,命將口袋向宗祠大爐內焚了。又命賈蓉道:“你去問問你璉二嬸子,正月裏請喫年酒的日子擬了沒有。若擬定了,叫書房裏明白開了單子來,咱們再請時,就不能重犯了。舊年不留心重了幾家,不說咱們不留神,倒象兩宅商議定了送虛情怕費事一樣。”賈蓉忙答應了過去。一時,拿了請人喫年酒的日期單子來了。賈珍看了,命交與賴升去看了,請人別重這上頭日子。因在廳上看著小廝們抬圍屏,擦抹几案金銀供器。只見小廝手裏拿著個稟帖並一篇帳目,回說:“黑山村的烏莊頭來了。”
\end{parag}


\begin{parag}
    賈珍道:“這個老砍頭的今兒纔來。”說著,賈蓉接過稟帖和帳目,忙展開捧著,賈珍倒背著兩手,向賈蓉手內只看紅稟帖上寫著:
\end{parag}


\begin{qute2sp}
    門下莊頭烏進孝叩請爺、奶奶萬福金安,並公子小姐金安。新春大喜大福,榮貴平安,加官進祿,萬事如意。
\end{qute2sp}


\begin{parag}
    賈珍笑道:“莊家人有些意思。”賈蓉也忙笑說:“別看文法,只取個吉利罷了。”一面忙展開單子看時,只見上面寫著:
\end{parag}


\begin{qute2sp}
    大鹿三十隻,獐子五十隻,狍子五十隻,暹豬二十個,湯豬二十個,龍豬二十個,野豬二十個,家臘豬二十個,野羊二十個,青羊二十個,家湯羊二十個,家風羊二十個,鱘鰉魚二個,各色雜魚二百斤,活雞、鴨、鵝各二百隻,風雞、鴨、鵝二百隻,野雞、兔子各二百對,熊掌二十對,鹿筋二十斤,海蔘五十斤,鹿舌五十條,牛舌五十條,蟶乾二十斤,榛、松、桃、杏穰各二口袋,大對蝦五十對,幹蝦二百斤,銀霜炭上等選用一千斤、中等二千斤,柴炭三萬斤,御田胭脂米二石,\begin{note}庚雙夾:在園雜字曾有此說。\end{note}碧糯五十斛,白糯五十斛,粉粳五十斛,雜色粱谷各五十斛,下用常米一千石,各色乾菜一車,外賣粱谷、牲口各項之銀共折銀二千五百兩。外門下孝敬哥兒姐兒頑意:活鹿兩對,活白兔四對,黑兔四對,活錦雞兩對,西洋鴨兩對。
\end{qute2sp}


\begin{parag}
    賈珍便命帶進他來。一時,只見烏進孝進來,只在院內磕頭請安。賈珍命人拉他起來,笑說:“你還硬朗。”烏進孝笑回:“託爺的福,還能走得動。”賈珍道:“你兒子也大了,該叫他走走也罷了。”烏進孝笑道:“不瞞爺說,小的們走慣了,不來也悶的慌。他們可不是都願意來見見天子腳下世面?他們到底年輕,怕路上有閃失,再過幾年就可放心了。”賈珍道:“你走了幾日?”烏進孝道:“回爺的話,今年雪大,外頭都是四五尺深的雪,前日忽然一暖一化,路上竟難走的很,耽擱了幾日。雖走了一個月零兩日,因日子有限了,怕爺心焦,可不趕著來了。”賈珍道:“我說呢,怎麼今兒纔來。我纔看那單子上,今年你這老貨又來打擂臺來了。”烏進孝忙進前了兩步,回道:“回爺說,今年年成實在不好。從三月下雨起,接接連連直到八月,竟沒有一連晴過五日。九月裏一場碗大的雹子,方近一千三百里地,連人帶房並牲口糧食,打傷了上千上萬的,所以才這樣。小的並不敢說謊。”賈珍皺眉道:“我算定了你至少也有五千兩銀子來,這夠作什麼的!如今你們一共只剩了八九個莊子,今年倒有兩處報了旱澇,你們又打擂臺,真真是又教別過年了。”烏進孝道:“爺的這地方還算好呢!我兄弟離我那裏只一百多里,誰知竟大差了。他現管著那府裏八處莊地,比爺這邊多著幾倍,今年也只這些東西,不過多二三千兩銀子,也是有饑荒打呢。”賈珍道:“正是呢,我這邊都可,已沒有什麼外項大事,不過是一年的費用費些。我受些委屈就省些。再者年例送人請人,我把臉皮厚些,可省些也就完了。比不得那府裏,這幾年添了許多花錢的事,一定不可免是要花的,卻又不添些銀子產業。這一二年倒賠了許多,不和你們要,找誰去!”烏進孝笑道:“那府裏如今雖添了事,有去有來,娘娘和萬歲爺豈不賞的!”\begin{note}庚雙夾:是莊頭口中語氣。脂硯。\end{note}賈珍聽了,笑向賈蓉等道:“你們聽,他這話可笑不可笑?”賈蓉等忙笑道:“你們山坳海沿子上的人,那裏知道這道理。娘娘難道把皇上的庫給了我們不成!他心裏縱有這心,他也不能作主。豈有不賞之理,按時到節不過是些綵緞古董頑意兒。縱賞銀子,不過一百兩金子,才值了一千兩銀子,夠一年的什麼?這二年那一年不多賠出幾千銀子來!頭一年省親連蓋花園子,你算算那一注共花了多少,就知道了。再兩年再一回省親,只怕就精窮了。”賈珍笑道:“所以他們莊家老實人,外明不知裏暗的事。黃柏木作磬槌子──外頭體面裏頭苦。”\begin{note}庚雙夾:新鮮趣語。\end{note}賈蓉又笑向賈珍道: “果真那府裏窮了。前兒我聽見鳳姑娘\begin{note}庚雙夾:此亦南北互用之文,前注不謬。\end{note}和鴛鴦悄悄商議,要偷出老太太的東西去當銀子呢。”賈珍笑道:“那又是你鳳姑娘的鬼,那裏就窮到如此。他必定是見去路太多了,實在賠的狠了,不知又要省那一項的錢,先設此法使人知道,說窮到如此了。我心裏卻有一個算盤,還不至如此田地。”說著,命人帶了烏進孝出去,好生待他,不在話下。
\end{parag}


\begin{parag}
    這裏賈珍吩咐將方纔各物,留出供祖的來,將各樣取了些,命賈蓉送過榮府裏。然後自己留了家中所用的,餘者派出等例來,一分一分的堆在月臺下,命人將族中的子侄喚來與他們。接著榮國府也送了許多供祖之物及與賈珍之物。賈珍看著收拾完備供器,靸著鞋,披著猞猁猻大裘,命人在廳柱下石磯上太陽中鋪了一個大狼皮褥子,負暄閒看各子弟們來領取年物。因見賈芹亦來領物,賈珍叫他過來,說道:“你作什麼也來了?誰叫你來的?”賈芹垂手回說:“聽見大爺這裏叫我們領東西,我沒等人去就來了。”賈珍道:“我這東西,原是給你那些閒著無事的無進益的小叔叔兄弟們的。那二年你閒著,我也給過你的。你如今在那府裏管事,家廟裏管和尚道士們,一月又有你的分例外,這些和尚的分例銀子都從你手裏過,你還來取這個,太也貪了!你自己瞧瞧,你穿的象個手裏使錢辦事的?先前說你沒進益,如今又怎麼了?比先倒不象了。”賈芹道:“我家裏原人口多,費用大。”賈珍冷笑道:“你還支吾我。你在家廟裏乾的事,打諒我不知道呢。你到了那裏自然是爺了,沒人敢違拗你。你手裏又有了錢,離著我們又遠,你就爲王稱霸起來,夜夜招聚匪類賭錢,\begin{note}庚雙夾:這一回文字斷不可少。\end{note}養老婆小子。\begin{note}靖藏眉: “招匪類賭錢,養老婆小子”,即是敗家的根本。\end{note}這會子花的這個形象,你還敢領東西來?領不成東西,領一頓馱水棍去才罷。等過了年,我必和你璉二叔說,換回你來。”賈芹紅了臉,不敢答應。人回:“北府水王爺送了字聯、荷包來了。”賈珍聽說,忙命賈蓉出去款待,“只說我不在家。”賈蓉去了,這裏賈珍看著領完東西,回房與尤氏喫畢晚飯,一宿無話。至次日,更比往日忙,都不必細說。
\end{parag}


\begin{parag}
    已到了臘月二十九日了,各色齊備,兩府中都換了門神、聯對、掛牌,新油了桃符,煥然一新。寧國府從大門、儀門、大廳、暖閣、內廳、內三門、內儀門並內塞門,直到正堂,一路正門大開,兩邊階下一色硃紅大高照,點的兩條金龍一般。次日,由賈母有誥封者,皆按品級著朝服,先坐八人大轎,帶領著衆人進宮朝賀,行禮領宴畢回來,便到寧國府暖閣下轎。諸子弟有未隨入朝者,皆在寧府門前排班伺候,然後引入宗祠。且說寶琴是初次,一面細細留神打諒這宗祠,原來寧府西邊另一個院子,黑油柵欄內五間大門,上懸一塊匾,寫著是“賈氏宗祠”四個字,旁書“衍聖公孔繼宗書”。兩旁有一副長聯,寫道是:
\end{parag}


\begin{poem}
    \begin{pl}肝腦塗地,兆姓賴保育之恩;\end{pl}

    \begin{pl}功名貫天,百代仰蒸嘗之盛。\end{pl}
    \begin{note}庚眉:此聯宜掉轉。\end{note}
\end{poem}


\begin{parag}
    亦衍聖公所書。進入院中,白石甬路,兩邊皆是蒼松翠柏。月臺上設著青銅古銅鼎彞等器。抱廈前上面懸一九龍金匾,寫道是:“星輝輔弼”。乃先皇御筆。兩邊一副對聯,寫道是:
\end{parag}


\begin{poem}
    \begin{pl}勳業有光昭日月,功名無間及兒孫。\end{pl}

\end{poem}


\begin{parag}
    亦是御筆。五間正殿前懸一鬧龍填青匾,寫道是:“慎終追遠”。旁邊一副對聯,寫道是:
\end{parag}


\begin{poem}
    \begin{pl}已後兒孫承福德,至今黎庶念榮寧。\end{pl}

\end{poem}


\begin{parag}
    俱是御筆。裏邊香燭輝煌,錦帳繡幕,雖列著神主,卻看不真切。只見賈府人分昭穆排班立定:賈敬主祭,賈赦陪祭,賈珍獻爵,賈璉賈琮獻帛,寶玉捧香,賈菖賈菱展拜墊,守焚池。青衣樂奏,三獻爵,拜興畢,焚帛奠酒。禮畢,樂止,退出。衆人圍隨賈母至正堂上,影前錦幔高掛,彩屏張護,香燭輝煌。上面正居中懸著寧榮二祖遺像,皆是披蟒腰玉;兩邊還有幾軸列祖遺影。賈荇賈芷等從內儀門挨次列站,直到正堂廊下。檻外方是賈敬賈赦,檻內是各女眷。衆家人小廝皆在儀門之外。每一道菜至,傳至儀門,賈荇賈芷等便接了,按次傳至階上賈敬手中。賈蓉系長房長孫,獨他隨女眷在檻內,每賈敬捧菜至,傳於賈蓉,賈蓉便傳於他妻子,又傳於鳳姐尤氏諸人,直傳至供桌前,方傳於王夫人。王夫人傳於賈母,賈母方捧放在桌上。邢夫人在供桌之西,東向立,同賈母供放。直至將菜飯湯點酒菜傳完,賈蓉方退出下階,歸入賈芹階位之首。凡從文旁之名者,賈敬爲首;下則從玉者,賈珍爲首;再下從草頭者,賈蓉爲首;左昭右穆,男東女西;俟賈母拈香下拜,衆人方一齊跪下,將五間大廳,三間抱廈,內外廊檐,階上階下兩丹墀內,花軒錦簇,塞的無一些空地。鴉雀無聞,只聽鏗鏘叮噹,金鈴玉佩微微搖曳之聲,並起跪靴履颯沓之響。一時禮畢,賈敬賈赦等便忙退出,至榮府專候與賈母行禮。
\end{parag}


\begin{parag}
    尤氏上房地下早已襲地鋪滿紅氈,當地放著象鼻三足鰍沿鎏金琺琅大火盆,正面炕上鋪新猩紅氈子,設著大紅彩繡雲龍捧壽的靠背引枕,外另黑狐皮的袱子搭在上面,大白狐皮坐褥,請賈母上去坐了。兩邊又鋪皮褥,讓賈母一輩的兩三個妯娌坐了。這邊橫頭排插之後小炕上,也鋪了皮褥,讓邢夫人等坐了。地下兩面相對十二張雕漆椅上,都是一色灰鼠椅搭小褥,每一張椅下一個大銅腳爐,讓寶琴等姊妹坐了。尤氏用茶盤親捧茶與賈母,蓉妻捧與衆老祖母,然後尤氏又捧與邢夫人等,蓉妻又捧與衆姊妹。鳳姐李紈等只在地下伺候。茶畢,邢夫人等便先起身來侍賈母。賈母喫茶,與老妯娌閒話了兩三句,便命看轎,鳳姐兒忙上去挽起來。尤氏笑回說:“已經預備下老太太的晚飯。每年都不肯賞些體面用過晚飯過去,果然我們就不及鳳丫頭不成?”鳳姐兒攙著賈母笑道:“老祖宗快走,咱們家去喫去,別理他。”賈母笑道:“你這裏供著祖宗,忙的什麼似的,那裏還擱得住鬧。況且每年我不喫,你們也要送去的。不如還送了來,我吃不了留著明兒再喫,豈不多喫些。”說的衆人都笑了。又吩咐他:”好生派妥當人夜裏看香火,不是大意得的。”尤氏答應了。一面走出來至暖閣前上了轎。尤氏等閃過屏風,小廝們才領轎伕,請了轎出大門。尤氏亦隨邢夫人等同至榮府。
\end{parag}


\begin{parag}
    這裏轎出大門,這一條街上,東一邊合面設列著寧國公的儀仗執事樂器,西一邊合面設列著榮國公的儀仗執事樂器,來往行人皆屏退不從此過。一時來至榮府,也是大門正廳直開到底。如今便不在暖閣下轎了,過了大廳,便轉彎向西,至賈母這邊正廳上下轎。衆人圍隨同至賈母正室之中,亦是錦裀繡屏,煥然一新。當地火盆內焚著松柏香、百合草。賈母歸了座,老嬤嬤來回:“老太太們來行禮。”賈母忙又起身要迎,只見兩三個老妯娌已進來了。大家挽手,笑了一回,讓了一回。喫茶去後,賈母只送至內儀門便回來,歸正坐。賈敬賈赦等領諸子弟進來。賈母笑道:“一年價難爲你們,不行禮罷。”一面說著,一面男一起,女一起,一起一起俱行過了禮。左右兩旁設下交椅,然後又按長幼挨次歸坐受禮。兩府男婦小廝丫鬟亦按差役上中下行禮畢,散押歲錢、荷包、金銀錁,擺上合歡宴來。男東女西歸坐,獻屠蘇酒、合歡湯、吉祥果、如意糕畢,賈母起身進內間更衣,衆人方各散出。那晚各處佛堂竈王前焚香上供,王夫人正房院內設著天地紙馬香供,大觀園正門上也挑著大明角燈,兩溜高照,各處皆有路燈。上下人等,皆打扮的花團錦簇,一夜人聲嘈雜,語笑喧闐,爆竹起火,絡繹不絕。
\end{parag}


\begin{parag}
    至次日五鼓,賈母等又按品大妝,擺全副執事進宮朝賀,兼祝元春千秋。領宴回來,又至寧府祭過列祖,方回來受禮畢,便換衣歇息。所有賀節來的親友一概不會,只和薛姨媽李嬸二人說話取便,或者同寶玉、寶琴、釵、玉等姊妹趕圍棋抹牌作戲。王夫人與鳳姐是天天忙著請人喫年酒,那邊廳上院內皆是戲酒,親友絡繹不絕,一連忙了七八日才完了。早又元宵將近,寧榮二府皆張燈結綵。十一日是賈赦請賈母等,次日賈珍又請,賈母皆去隨便領了半日。王夫人和鳳姐兒連日被人請去喫年酒,不能勝記。
\end{parag}


\begin{parag}
    至十五日之夕,賈母便在大花廳上命擺几席灑,定一班小戲,滿掛各色佳燈,帶領榮寧二府各子侄孫男孫媳等家宴。賈敬素不茹酒,也不去請他,於後十七日祖祀已完,他便仍出城去修養。便這幾日在家內,亦是靜室默處,一概無聽無聞,不在話下。賈赦略領了賈母之賜,也便告辭而去。賈母知他在此彼此不便,也就隨他去了。賈赦自到家中與衆門客賞燈喫酒,自然是笙歌聒耳,錦繡盈眸,其取便快樂另與這邊不同。\begin{note}庚雙夾:又交代一個。\end{note}
\end{parag}


\begin{parag}
    這邊賈母花廳之上共擺了十來席。每一席旁邊設一幾,几上設爐瓶三事,焚著御賜百合宮香。又有八寸來長四五寸寬二三寸高的點著山石佈滿青苔的小盆景,俱是新鮮花卉。又有小洋漆茶盤,內放著舊窯茶杯並十錦小茶吊,裏面泡著上等名茶。一色皆是紫檀透雕,嵌著大紅紗透繡花卉並草字詩詞的瓔珞。原來繡這瓔珞的也是個姑蘇女子,名喚慧娘。因他亦是書香宦門之家,他原精於書畫,不過偶然繡一兩件針線作耍,並非市賣之物。凡這屏上所繡之花卉,皆仿的是唐、宋、元、明各名家的折枝花卉,故其格式配色皆從雅,本來非一味濃豔匠工可比。每一枝花側皆用古人題此花之舊句,或詩詞歌賦不一,皆用黑絨繡出草字來,且字跡勾踢、轉折、輕重、連斷皆與筆草無異,亦不比市繡字跡板強可恨。他不仗此技獲利,所以天下雖知,得者甚少,凡世宦富貴之家,無此物者甚多,當今便稱爲“慧繡”。竟有世俗射利者,近日仿其針跡,愚人獲利。偏這慧娘命夭,十八歲便死了,如今竟不能再得一件的了。凡所有之家,縱有一兩件,皆珍藏不用。有那一干翰林文魔先生們,因深惜“慧繡”之佳,便說這“繡”字不能盡其妙,這樣筆跡說一“繡”字,反似乎唐突了,便大家商議了,將“繡”字便隱去,換了一個“紋”字,所以如今都稱爲“慧紋”。若有一件真“慧紋”之物,價則無限。賈府之榮,也只有兩三件,上年將那兩件已進了上,目下只剩這一副瓔珞,一共十六扇,賈母愛如珍寶,不入在請客各色陳設之內,只留在自己這邊,高興擺酒時賞玩。又有各色舊窯小瓶中都點綴著“歲寒三友”“玉堂富貴”等鮮花草。
\end{parag}


\begin{parag}
    上面兩席是李嬸薛姨媽二位。賈母於東邊設一透雕夔龍護屏矮足短榻,靠背引枕皮褥俱全。榻之上一頭又設一個極輕巧洋漆描金小几,几上放著茶吊、茶碗、漱盂、洋巾之類,又有一個眼鏡匣子。賈母歪在榻上,與衆人說笑一回,又自取眼鏡向戲臺上照一回,又向薛姨媽李嬸笑說:“恕我老了,骨頭疼,放肆,容我歪著相陪罷。”因又命琥珀坐在榻上,拿著美人拳捶腿。榻下並不擺席面,只有一張高几,卻設著瓔珞花瓶香爐等物。外另設一精緻小高桌,設著酒杯匙箸,將自己這一席設於榻旁,命寶琴、湘雲、黛玉、寶玉四人坐著。每一饌一果來,先捧與賈母看了,喜則留在小桌上嘗一嘗,仍撤了放在他四人席上,只算他四人是跟著賈母坐。故下面方是邢夫人王夫人之位,再下便是尤氏、李紈、鳳姐、賈蓉之妻。西邊一路便是寶釵、李紋、李綺、岫煙、迎春姊妹等。兩邊大梁上,掛著一對聯三聚五玻璃芙蓉彩穗燈。每一席前豎一柄漆幹倒垂荷葉,葉上有燭信插著彩燭。這荷葉乃是鏨琺琅的,活信可以扭轉,如今皆將荷葉扭轉向外,將燈影逼住全向外照,看戲分外真切。窗格門戶一齊摘下,全掛彩穗各種宮燈。廊檐內外及兩邊遊日棚,將各色羊角、玻璃、戳紗、料絲、或繡、或畫、或堆、或摳、或絹、或紙諸燈掛滿。廊上几席,便是賈珍、賈璉、賈環、賈琮、賈蓉、賈芹、賈芸、賈菱、賈菖等。
\end{parag}


\begin{parag}
    賈母也賈母也曾差人去請衆族中男女,奈他們或有年邁懶於熱鬧的;或有家內沒有人不便來的;或有疾病淹纏,欲來竟不能來的;或有一等妒富愧貧不來的;甚至於有一等憎畏鳳姐之爲人而賭氣不來的;或有羞手羞腳,不慣見人,不敢來的:因此族衆雖多,女客來者只不過賈菌之母婁氏帶了賈藍來了,男子只有賈芹、賈芸、賈菖、賈菱四個現是在鳳姐麾下辦事的來了。當下人雖不全,在家庭間小宴中,數來也算是熱鬧的了。當又有林之孝之妻帶了六個媳婦,抬了三張炕桌,每一張上搭著一條紅氈,氈上放著選淨一般大新出局的銅錢,用大紅彩繩串著,每二人搭一張,共三張。林之孝家的指示將那兩張擺至薛姨媽李嬸的席下,將一張送至賈母榻下來。賈母便說:“放在當地罷。”這媳婦們都素知規矩的,放下桌子,一併將錢都打開,將彩繩抽去,散堆在桌上。正唱《西樓•樓會》這出將終,於叔夜因賭氣去了,那文豹便發科諢道:“你賭氣去了,恰好今日正月十五,榮國府中老祖宗家宴,待我騎了這馬,趕進去討些果子喫是要緊的。”說畢,引的賈母等都笑了。薛姨媽等都說:“好個鬼頭孩子,可憐見的。”鳳姐便說:“這孩子才九歲了。”賈母笑說:“難爲他說的巧。”便說了一個“賞”字。早有三個媳婦已經手下預備下簸籮,聽見一個“賞”字,走上去向桌上的散錢堆內,每人便撮了一簸籮,走出來向戲臺說:“老祖宗、姨太太、親家太太賞文豹買果子喫的!”說著,向臺上便一撒,只聽豁啷啷滿臺的錢響。賈珍賈璉已命小廝們抬了大簸籮的錢來,暗暗的預備在那裏。聽見賈母一賞,要知端的--
\end{parag}


\begin{parag}
    \begin{note}蒙回末總:敘元宵一宴,卻不敘酒,何以青菜?何以馨客?何以盛令?何以行先?於香茗古玩上渲染,兒榻坐次上鋪陳,隱隱爲下回張本,有無限含蓄,超邁獺祭者百倍。\end{note}
\end{parag}


\begin{parag}
    \begin{note}蒙回末總:前半整飭,後半蕭(?)落,濃淡相間。宗祠在寧國府,開宴在榮國府,分敘不犯手,是作者胸有成竹處。\end{note}
\end{parag}
