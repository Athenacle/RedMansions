\chap{七十三}{癡丫頭誤拾繡春囊 懦小姐不問累金鳳}


\begin{parag}
    \begin{note}蒙回前總:賈母一席話隱隱照起,全文便可一直敘去,接筆卻置賊不論,轉出賭錢,接筆卻置賭錢不論,轉出奸證,接筆卻置奸證不論,轉出討情,一波未平,一波又起,勢如怒蛇出穴,蜿蜒不就捕。\end{note}
\end{parag}


\begin{parag}
    話說那趙姨娘和賈政說話,忽聽外面一聲響,不知何物。忙問時,原來是外間窗屜不曾扣好,塌了屈戍了吊下來。趙姨娘罵了丫頭幾句,自己帶領丫鬟上好,方進來打發賈政安歇。不在話下。
\end{parag}


\begin{parag}
    卻說怡紅院中寶玉正才睡下,丫鬟們正欲各散安歇,忽聽有人擊院門。老婆子開了門,見是趙姨娘房內的丫鬟名喚小鵲的。問他什麼事,小鵲不答,直往房內來找寶玉。\begin{note}庚雙夾:奇,從未見此婢也。\end{note}只見寶玉才睡下,晴雯等猶在牀邊坐著,大家頑笑,見他來了,都問:“什麼事,這時候又跑了來作什麼?”\begin{note}庚雙夾:又是補出前文矣,非只張一回也。\end{note}小鵲笑向寶玉道:“我來告訴你一個信兒。方纔我們奶奶這般如此在老爺前說了。你仔細明兒老爺問你話。”說著回身就去了。襲人命留他喫茶,因怕關門,遂一直去了。
\end{parag}


\begin{parag}
    這裏寶玉聽了,便如孫大聖聽見了緊箍咒一般,登時四肢五內一齊皆不自在起來。想來想去,別無他法,且理熟了書預備明兒盤考。口內不舛錯,便有他事,也可搪塞一半。想罷,忙披衣起來要讀書。心中又自後悔,這些日子只說不提了,偏又丟生,早知該天天好歹溫習些的。如今打算打算,肚子內現可背誦的,不過只有《學》《庸》《二論》是帶注背得出的。至上本《孟子》,就有一半是夾生的,若憑空提一句,斷不能接背的,至《下孟》,就有一大半忘了。算起五經來,因近來作詩,常把《詩經》讀些,雖不甚精闡,還可塞責。\begin{note}庚雙夾:妙!寶玉讀書原系從問中□而有。\end{note}別的雖不記得,素日賈政也幸未吩咐過讀的,縱不知,也還不妨。至於古文,這是那幾年所讀過的幾篇,連《左傳》《國策》《公羊》《谷粱》漢唐等文,不過幾十篇,這幾年竟未曾溫得半篇片語,雖閒時也曾遍閱,不過一時之興,隨看隨忘,未下苦工夫,如何記得。這是斷難塞責的。更有時文八股一道,因平素深惡此道,原非聖賢之制撰,焉能闡發聖賢之微奧,不過作後人餌名釣祿之階。雖賈政當日起身時選了百十篇命他讀的,不過偶因見其中或一二股內,或承起之中,有作的或精緻,或流蕩,或遊戲,或悲感,稍能動性者,偶一讀之,不過供一時之興趣,究竟何曾成篇潛心玩索。\begin{note}庚雙夾:妙!寫寶玉讀書非爲功名也。\end{note}如今若溫習這個,又恐明日盤詰那個,若溫習那個,又恐盤駁這個。況一夜之功,亦不能全然溫習,因此越添了焦燥。自己讀書不致緊要,卻帶累著一房丫鬟們皆不能睡。襲人麝月晴雯等幾個大的是不用說,在旁剪燭斟茶,那些小的,都困眼朦朧,前仰後合起來。晴雯因罵道:“什麼蹄子們,一個個黑日白夜挺屍挺不夠,偶然一次睡遲了些,就裝出這腔調來了。再這樣,我拿針戳給你們兩下子!”
\end{parag}


\begin{parag}
    話猶未了,只聽外間咕咚一聲,急忙看時,原來是一個小丫頭子坐著打盹,一頭撞到壁上了,從夢中驚醒,恰正是晴雯說這話之時,他怔怔的只當是晴雯打了他一下,遂哭央說:“好姐姐,我再不敢了。”衆人都發起笑來。寶玉忙勸道:“饒他去罷,原該叫他們都睡去纔是。你們也該替換著睡去。”襲人忙道:“小祖宗,你只顧你的罷。通共這一夜的功夫,你把心暫且用在這幾本書上,等過了這一關,由你再張羅別的去,也不算誤了什麼。”寶玉聽他說的懇切,只得又讀。讀了沒有幾句,麝月又斟了一杯茶來潤舌,寶玉接茶吃了。因見麝月只穿著短襖,解了裙子,寶玉道:“夜靜了,冷,到底穿一件大衣裳纔是。”麝月笑指著書道:“你暫且把我們忘了,把心且略對著他些罷。”\begin{note}庚雙夾:此處豈是讀書之處,又豈是伴讀之人?古今天下誤盡多少紈絝!何況又是此等時之怡紅院,此等之鬟婢,又是此等一個寶玉哉!\end{note}
\end{parag}


\begin{parag}
    話猶未了,只聽金星玻璃從後房門跑進來,口內喊說:“不好了,一個人從牆上跳下來了!”衆人聽說,忙問在那裏,即喝起人來,各處尋找。晴雯因見寶玉讀書苦惱,勞費一夜神思,明日也未必妥當,心下正要替寶玉想出一個主意來脫此難,正好忽然逢此一驚,即便生計,向寶玉道:“趁這個機會快裝病,只說唬著了。” 此話正中寶玉心懷,因而遂傳起上夜人等來,打著燈籠,各處搜尋,並無蹤跡,都說:“小姑娘們想是睡花了眼出去,風搖的樹枝兒,錯認作人了。”晴雯便道: “別放謅屁!你們查的不嚴,怕得不是,還拿這話來支吾。纔剛並不是一個人見的,寶玉和我們出去有事,大家親見的。如今寶玉唬的顏色都變了,滿身發熱,我如今還要上房裏取安魂丸藥去。太太問起來,是要回明白的,難道依你說就罷了不成。”衆人聽了,嚇的不敢則聲,只得又各處去找。雯和玻璃二人果出去要藥,故意鬧的衆人皆知寶玉嚇著了。王夫人聽了,忙命人來看視給藥,又吩咐各上夜人仔細搜查,又一面叫查二門外鄰園牆上夜的小廝們。於是園內燈籠火把,直鬧了一夜。至五更天,就傳管家男女,命仔細查一查,拷問內外上夜男女等人。
\end{parag}


\begin{parag}
    賈母聞知寶玉被嚇,細問原由,不敢再隱,只得回明。賈母道:“我必料到有此事。如今各處上夜都不小心,還是小事,只怕他們就是賊也未可知。”當下邢夫人並尤氏等都過來請安,鳳姐及李紈姊妹等皆陪侍,聽賈母如此說,都默無所答。獨探春出位笑道:“近因鳳姐姐身子不好,幾日園內的人比先放肆了許多。先前不過是大家偷著一時半刻,或夜裏坐更時,三四個人聚在一處,或擲骰或鬥牌,小小的頑意,不過爲熬困。近來漸次放誕,竟開了賭局,甚至有頭家局主,或三十吊五十吊三百吊的大輸贏。半月前竟有爭鬥相打之事。”賈母聽了,忙說:“你既知道,爲何不早回我們來?”探春道:“我因想著太太事多,且連日不自在,所以沒回。只告訴了大嫂子和管事的人們,戒飭過幾次,近日好些。”賈母忙道:“你姑娘家,如何知道這裏頭的利害。你自爲耍錢常事,不過怕起爭端。殊不知夜間既耍錢,就保不住不喫酒,既喫酒,就免不得門戶任意開鎖。或買東西,尋張覓李,其中夜靜人稀,趨便藏賊引奸引盜,何等事作不出來。況且園內的姊妹們起居所伴者皆系丫頭媳婦們,賢愚混雜,賊盜事小,再有別事,倘略沾帶些,關係不小。這事豈可輕恕。”探春聽說,便默然歸坐。鳳姐雖未大愈,精神因此比常稍減,\begin{note}庚雙夾:看他漸次寫來,從不作一筆安逸之筆,況阿鳳之文哉。\end{note}今見賈母如此說,便忙道:“偏生我又病了。”遂回頭命人速傳林之孝家的等總理家事四個媳婦到來,當著賈母申飭了一頓。賈母命即刻查了頭家賭家來,有人出首者賞,隱情不告者罰。
\end{parag}


\begin{parag}
    林之孝家的等見賈母動怒,誰敢徇私,忙至園內傳齊人,一一盤查。雖不免大家賴一回,終不免水落石出。查得大頭家三人,小頭家八人,聚賭者通共二十多人,都帶來見賈母,跪在院內磕響頭求饒。賈母先問大頭家名姓和錢之多少。原來這三個大頭家,一個就是林之孝家的兩姨親家,一個就是園內廚房內柳家媳婦之妹,一個就是迎春之乳母。這是三個爲首的,餘者不能多記。賈母便命將骰子牌一併燒燬,所有的錢入官分散與衆人,將爲首者每人四十大板,攆出,總不許再入;從者每人二十大板,革去三月月錢,撥入圊廁行內。又將林之孝家的申飭了一番。林之孝家的見他的親戚又與他打嘴,自己也覺沒趣。迎春在坐,也覺沒意思。黛玉、寶釵、探春等見迎春的乳母如此,也是物傷其類的意思,遂都起身笑向賈母討情說:“這個媽媽素日原不頑的,不知怎麼也偶然高興。求看二姐姐面上,饒他這次罷。”賈母道:“你們不知。大約這些奶子們,一個個仗著奶過哥兒姐兒,原比別人有些體面,他們就生事,比別人更可惡,專管調唆主子護短偏向。我都是經過的。況且要拿一個作法,恰好果然就遇見了一個。你們別管,我自有道理。”寶釵等聽說,只得罷了。
\end{parag}


\begin{parag}
    一時賈母歇晌,大家散出,都知賈母今日生氣,皆不敢各散回家,只得在此暫候。尤氏便往鳳姐處來閒話了一回,因他也不自在,只得往園內尋衆姑嫂閒談。邢夫人在王夫人處坐了一回,也就往園內散散心來。剛至園門前,只見賈母房內的小丫頭子名喚傻大姐的笑嘻嘻走來,手內拿著個花紅柳綠的東西,低頭一壁瞧著,一壁只管走,不防迎頭撞見邢夫人,抬頭看見,方纔站住。邢夫人因說:“這癡丫頭,又得了個什麼狗不識兒這麼歡喜?拿來我瞧瞧。”原來這傻大姐年方十四五歲,是新挑上來的與賈母這邊提水桶掃院子專作粗活的一個丫頭。只因他生得體肥面闊,兩隻大腳作粗活簡捷爽利,且心性愚頑,一無知識,行事出言,常在規矩之外。賈母因喜歡他爽利便捷,又喜他出言可以發笑,便起名爲“呆大姐”,常悶來便引他取笑一回,毫無避忌,因此又叫他作“癡丫頭”。他縱有失禮之處,見賈母喜歡他,衆人也就不去苛責。這丫頭也得了這個力,若賈母不喚他時,便入園內來頑耍。今日正在園內掏促織,忽在山石背後得了一個五彩繡香囊,其華麗精緻,固是可愛,但上面繡的並非花鳥等物,一面卻是兩個人赤條條的盤踞相抱,一面是幾個字。這癡丫頭原不認得是春意,便心下盤算:“敢是兩個妖精打架?不然必是兩口子相打。”左右猜解不來,正要拿去與賈母看,\begin{note}庚雙夾:險極妙極!榮府堂堂詩禮之家,且大觀園又何等嚴肅清幽之地,金閨玉閣尚有此等穢物,天下淺浦募之家寧不慎乎!雖然,但此等偏出大官世族之中者,蓋因其房室香宵、鬟婢混雜,焉保其個個守禮持節哉?此正爲大官世族而告誡。淺浦募之處毋如主婢日夕耳鬢交磨,一止一動悉在耳目之中,又何必諄諄再四焉!\end{note}是以笑嘻嘻的一壁看,一壁走,忽見了邢夫人如此說,便笑道:“太太真個說的巧,真個是狗不識呢。\begin{note}庚雙夾:妙!寓言也,大凡知此交媾之情者真狗畜之說耳,非肆言惡詈凡識此事者即狗矣。然則雲先與賈母看,則先罵賈母矣。此處邢夫人亦看,然則又罵邢夫人乎?故作者又難。\end{note}太太請瞧一瞧。”說著,便送過去。邢夫人接來一看,嚇得連忙死緊攥住,\begin{note}庚雙夾:妙!這一“嚇”字方是寫邢夫人之筆,雖前文明寫邢夫人之爲人稍劣,然不在情理之中,若不用慎重之筆,則邢夫人直系一小家卑污極輕賊極輕之人矣,豈得與榮府賜房哉?所謂此書針綿慎密處全在無意中一字一句之間耳,看者細心方得。\end{note}忙問:“你是那裏得的?”傻大姐道:“我掏促織兒在山石上揀的。”邢夫人道:“快休告訴一人。這不是好東西,連你也要打死。皆因你素日是傻子,以後再別提起了。”這傻大姐聽了,反嚇的黃了臉,說:“再不敢了。”磕了個頭,呆呆而去。邢夫人回頭看時,都是些女孩兒,不便遞與,自己便塞在袖內,心內十分罕異,揣摩此物從何而至,且不形於聲色,且來至迎春室中。
\end{parag}


\begin{parag}
    迎春正因他乳母獲罪,自覺無趣,心中不自在,忽報母親來了,遂接入內室。奉茶畢,邢夫人因說道:“你這麼大了,你那奶媽子行此事,你也不說說他。如今別人都好好的,偏咱們的人做出這事來,什麼意思。”\begin{note}庚雙夾:“咱們”二字便見自懷異心,從上文生離異發瀝而來,謹密之至。更有人於此者君未知也,一笑。\end{note}迎春低著頭弄衣帶,半晌答道:“我說他兩次,他不聽也無法。況且他是媽媽,只有他說我的,沒有我說他的。”\begin{note}庚雙夾:妙極!直畫出一個懦弱小姐來。\end{note}邢夫人道:“胡說!你不好了他原該說,如今他犯了法,你就該拿出小姐的身分來。他敢不從,你就回我去纔是。如今直等外人共知,是什麼意思。\begin{note}庚雙夾:我竟問:外人爲誰?\end{note}再者,只他去放頭兒,還恐怕他巧言花語的和你借貸些簪環衣履作本錢,你這心活面軟,未必不周接他些。若被他騙去,我是一個錢沒有的,看你明日怎麼過節。”迎春不語,只低頭弄衣帶。邢夫人見他這般,因冷笑道:“總是你那好哥哥好嫂子,一對兒赫赫揚揚,璉二爺鳳奶奶,兩口子遮天蓋日,百事周到,竟通共這一個妹子,全不在意。\begin{note}庚雙夾:加在於璉鳳的是父母常情,極是何必又如此說來便見私意。\end{note}但凡是我身上吊下來的,又有一話說──只好憑他們罷了。\begin{note}庚雙夾:如何?此皆婦女私假之意,大不可者。\end{note}況且你又不是我養的,\begin{note}庚雙夾:更不好。\end{note}你雖然不是同他一娘所生,到底是同出一父,也該彼此瞻顧些,也免別人笑話。\begin{note}庚雙夾:又問:別人爲誰?又問:彼二人雖不同母,終是同父。彼二人既系同父,其父又系君之何人?籲!婦人私心,今古有之。\end{note}我想天下的事也難較定,你是大老爺跟前人養的,這裏探丫頭也是二老爺跟前人養的,出身一樣。如今你娘死了,從前看來你兩個的娘,只有你娘比如今趙姨娘強十倍的,你該比探丫頭強纔是。怎麼反不及他一半!誰知竟不然,這可不是異事。倒是我一生無兒無女的,一生乾淨,也不能惹人笑話議論爲高。”\begin{note}庚雙夾:最可恨婦人無嗣者引此話是說。\end{note}旁邊伺侯的媳婦們便趁機道:“我們的姑娘老實仁德,那裏像他們三姑娘伶牙俐齒,會要姊妹們的強。他們明知姐姐這樣,他竟不顧恤一點兒。”\begin{note}庚雙夾:殺殺殺!此輩專生離異。餘因實受其蠱,今讀此文,直欲拔劍劈紙。又不知作者多少眼淚灑出此回也。又問:不知如何顧恤些?又不知有何可顧恤之處?直令人不解愚奴賤婢之言。酷肖之至。\end{note}邢夫人道:“連他哥哥嫂子還如是,別人又作什麼呢。”一言未了,人回:“璉二奶奶來了。”邢夫人聽了,冷笑兩聲,命人出去說:“請他自去養病,我這裏不用他伺候。”接著又有探春的小丫頭來報說:“老太太醒了。”邢夫人方起身前邊來。迎春送至院外方回。
\end{parag}


\begin{parag}
    繡桔因說道:“如何,前兒我回姑娘,那一個攢珠累絲金鳳竟不知那裏去了。回了姑娘,姑娘竟不問一聲兒。我說必是老奶奶拿去典了銀子放頭兒的,姑娘不信,只說司棋收著呢。問司棋,司棋雖病著,心裏卻明白。我去問他,他說沒有收起來,還在書架上匣內暫放著,預備八月十五日恐怕要戴呢。姑娘就該問老奶奶一聲,只是臉軟怕人惱。如今竟怕無著,明兒要都戴時,獨咱們不戴,是何意思呢。”\begin{note}庚雙夾:這個“咱們”使得恰,是女兒 喁私語,非前文之一例可比。寫得出,批得出。\end{note}迎春道:“何用問,自然是他拿去暫時借一肩兒。我只說他悄悄的拿了出去,不過一時半晌,仍舊悄悄的送來就完了,誰知他就忘了。今日偏又鬧出來,問他想也無益。”繡桔道:“何曾是忘記!他是試準了姑娘的性格,所以才這樣。如今我有個主意:我竟走到二奶奶房裏將此事回了他,或他著人去要,或他省事拿幾吊錢來替他賠補。如何?”\begin{note}庚雙夾:寫女兒各有機變,個個不同。\end{note}迎春忙道:“罷,罷,罷,省些事罷。寧可沒有了,又何必生事。”\begin{note}庚雙夾:總是懦語。\end{note}繡桔道:“姑娘怎麼這樣軟弱。都要省起事來,將來連姑娘還騙了去呢,我竟去的是。”說著便走。迎春便不言語,只好由他。
\end{parag}


\begin{parag}
    誰知迎春乳母子媳王住兒媳婦正因他婆婆得了罪,來求迎春去討情,聽他們正說金鳳一事,且不進去。也因素日迎春懦弱,他們都不放在心上。如今見繡桔立意去回鳳姐,估著這事脫不去的,且又有求迎春之事,只得進來,陪笑先向繡桔說:“姑娘,你別去生事。姑娘的金絲鳳,原是我們老奶奶老糊塗了,輸了幾個錢,沒的撈梢,所以暫借了去。原說一日半晌就贖的,因總未撈過本兒來,就遲住了。可巧今兒又不知是誰走了風聲,弄出事來。雖然這樣,到底主子的東西,我們不敢遲誤下,終久是要贖的。如今還要求姑娘看從小兒喫奶的情常,往老太太那邊去討個情面,救出他老人家來纔好。”迎春先便說道:“好嫂子,你趁早兒打了這妄想,要等我去說情兒,等到明年也不中用的。方纔連寶姐姐林妹妹大夥兒說情,老太太還不依,何況是我一個人。我自己愧還愧不來,反去討臊去。”繡桔便說:“贖金鳳是一件事,說情是一件事,別絞在一處說。難道姑娘不去說情,你就不贖了不成?嫂子且取了金鳳來再說。”王住兒家的聽見迎春如此拒絕他,繡桔的話又鋒利無可回答,一時臉上過不去,也明欺迎春素日好性兒,乃向繡桔發話道:“姑娘,你別太仗勢了。你滿家子算一算,誰的媽媽奶子不仗著主子哥兒多得些益,偏咱們就這樣丁是丁卯是卯的,只許你們偷偷摸摸的哄騙了去。自從邢姑娘來了,太太吩咐一個月儉省出一兩銀子來與舅太太去,這裏饒添了邢姑娘的使費,反少了一兩銀子。常時短了這個,少了那個,那不是我們供給?誰又要去?不過大家將就些罷了。算到今日,少說些也有三十兩了。我們這一向的錢,豈不白填了限呢。”繡桔不待說完,便啐了一口,道:“作什麼的白填了三十兩,我且和你算算帳,姑娘要了些什麼東西?”迎春聽見這媳婦發邢夫人之私意,\begin{note}庚雙夾:大書此句,誅心之筆。\end{note}忙止道:“罷,罷,罷。你不能拿了金鳳來,不必牽三扯四亂嚷。我也不要那鳳了。便是太太們問時,我只說丟了,也妨礙不著你什麼的,出去歇息歇息倒好。”一面叫繡桔倒茶來。繡桔又氣又急,因說道:“姑娘雖不怕,我們是作什麼的,把姑娘的東西丟了。他倒賴說姑娘使了他們的錢,這如今竟要准折起來。倘或太太問姑娘爲什麼使了這些錢,敢是我們就中取勢了?這還了得!”一行說,一行就哭了。司棋聽不過,只得勉強過來,幫著繡桔問著那媳婦。迎春勸止不住,自拿了一《太上感應篇》來看。\begin{note}庚雙夾:神妙之至!從紙上跳出一位懦弱小姐,且書又有奇,大妙!\end{note}
\end{parag}


\begin{parag}
    三人正沒開交,可巧寶釵、黛玉、寶琴、探春等因恐迎春今日不自在,都約來安慰他。走至院中,聽得兩三個人較口。探春從紗窗內一看,只見迎春倚在牀上看書,若有不聞之狀。\begin{note}庚雙夾:看他寫迎春,雖稍劣,然亦大家千金之格也。\end{note}探春也笑了。小丫鬟們忙打起簾子,報導:“姑娘們來了。”迎春方放下書起身。那媳婦見有人來,且又有探春在內,不勸而自止了,遂趁便要去。探春坐下,便問:“纔剛誰在這裏說話?倒象拌嘴似的。”\begin{note}庚雙夾:瞧他寫探春氣宇。\end{note}迎春笑道:“沒有說什麼,左不過是他們小題大作罷了。何必問他。”探春笑道:“我才聽見什麼‘金鳳’,又是什麼‘沒有錢只和我們奴才要’,誰和奴才要錢了?難道姐姐和奴才要錢了不成?難道姐姐不是和我們一樣有月錢的,一樣有用度不成?”司棋繡桔道:“姑娘說的是了。姑娘們都是一樣的,那一位姑娘的錢不是由著奶奶媽媽們使,連我們也不知道怎麼是算帳,不過要東西只說得一聲兒。如今他偏要說姑娘使過了頭兒,他賠出許多來了。究竟姑娘何曾和他要什麼了。” 探春笑道:“姐姐既沒有和他要,必定是我們或者和他們要了不成!你叫他進來,我倒要問問他。”迎春笑道:“這話又可笑。你們又無沾礙,何得帶累於他。”探春笑道:“這倒不然。我和姐姐一樣,姐姐的事和我的也是一般,他說姐姐就是說我。我那邊的人有怨我的,姐姐聽見也即同怨姐姐是一理。咱們是主子,自然不理論那些錢財小事,只知想起什麼要什麼,也是有的事。但不知金累絲鳳因何又夾在裏頭?”那王住兒媳婦生恐繡桔等告出他來,遂忙進來用話飾。探春深知其意,因笑道:“你們所以糊塗。如今你奶奶已得了不是,趁此求求二奶奶,把方纔的錢尚未散人的拿出些來贖取了就完了。比不得沒鬧出來,大家都藏著留臉面,如今既是沒了臉,趁此時縱有十個罪,也只一人受罰,沒有砍兩顆頭的理。你依我,竟是和二奶奶說說。在這裏大聲小氣,如何使得。”這媳婦被探春說出真病,也無可賴了,只不敢往鳳姐處自首。探春笑道:“我不聽見便罷,既聽見,少不得替你們分解分解。”誰知探春早使個眼色與待書出去了。
\end{parag}


\begin{parag}
    這裏正說話,忽見平兒進來。寶琴拍手笑說道:“三姐姐敢是有驅神召將的符術?”黛玉笑道:“這倒不是道家玄術,倒是用兵最精的,所謂‘守如處女,脫如狡兔’,出其不備之妙策也。”二人取笑。寶釵便使眼色與二人,令其不可,遂以別話岔開。探春見平兒來了,遂問:“你奶奶可好些了?真是病糊塗了,事事都不在心上,叫我們受這樣的委曲。”平兒忙道:“姑娘怎麼委曲?誰敢給姑娘氣受,姑娘快吩咐我。”當時住兒媳婦兒方慌了手腳,遂上來趕著平兒叫:“姑娘坐下,讓我說原故請聽。”平兒正色道:“姑娘這裏說話,也有你我混插口的禮!你但凡知禮,只該在外頭伺候。不叫你進不來的地方,幾曾有外頭的媳婦子們無故到姑娘們房裏來的例。”繡桔道:“你不知我們這屋裏是沒禮的,誰愛來就來。”平兒道:“都是你們的不是。姑娘好性兒,你們就該打出去,然後再回太太去纔是。”王住兒媳婦見平兒出了言,紅了臉方退出去。探春接著道:“我且告訴你,若是別人得罪了我,倒還罷了。如今那住兒媳婦和他婆婆仗著是媽媽,又瞅著二姐姐好性兒,如此這般私自拿了首飾去賭錢,而且還捏造假帳妙算,威逼著還要去討情,和這兩個丫頭在臥房裏大嚷大叫,二姐姐竟不能轄治,所以我看不過,才請你來問一聲:還是他原是天外的人,不知道理?還是誰主使他如此,先把二姐姐制伏,然後就要治我和四姑娘了?”平兒忙陪笑道:“姑娘怎麼今日說這話出來?我們奶奶如何當得起!”探春冷笑道:“俗語說的,‘物傷其類’,‘齒竭脣亡’,我自然有些驚心。”平兒道:“若論此事,還不是大事,極好處置。但他現是姑娘的奶嫂,據姑娘怎麼樣爲是?”當下迎春只和寶釵閱《感應篇》故事,究竟連探春之語亦不曾聞得,忽見平兒如此說,乃笑道:“問我,我也沒什麼法子。他們的不是,自作自受,我也不能討情,我也不去苛責就是了。至於私自拿去的東西,送來我收下,不送來我也不要了。太太們要問,我可以隱瞞遮飾過去,是他的造化,若瞞不住,我也沒法,沒有個爲他們反欺枉太太們的理,少不得直說。你們若說我好性兒,沒個決斷,竟有好主意可以八面周全,不使太太們生氣,任憑你們處治,我總不知道。”衆人聽了,都好笑起來。黛玉笑道:“真是‘虎狼屯於階陛尚談因果’。若使二姐姐是個男人,這一家上下若許人,又如何裁治他們。”迎春笑道:“正是。多少男人尚如此,何況我哉?”一語未了,只見又有一個人進來。正不知道是那個,且聽下回分解。
\end{parag}


\begin{parag}
    \begin{note}蒙回末總:一篇奸盜淫邪文字,反以四子五經《公羊》《穀梁》秦漢諸作起,以《太上感應篇》結後,何心哉!他深見“書中自有黃金屋”“書中有女美如雨”等語誤盡天下蒼生,而大奸大盜從此出。故特作此一起結,爲五淫濁世頂門一聲棒喝也。眼空似箕,筆大如椽,何得以尋行數墨繩之。\end{note}
\end{parag}


\begin{parag}
    \begin{note}蒙回末總:探春處處出頭,人謂其能,吾謂其苦;迎春處處藏舌,謂其怯,吾謂其超。探春運符咒,因及役鬼驅神;迎春說因果,更可降狼伏虎。\end{note}
\end{parag}
