\chap{三}{金陵城起復賈雨村 榮國府收養林黛玉}

\begin{parag}
    金陵城起復賈雨村 榮國府收養\begin{note}甲側:二字觸目淒涼之至!\end{note}林黛玉
\end{parag}

\begin{parag}
    \begin{note}蒙、戚回前:我爲你持戒,我爲你喫齋,我爲你百行百計不舒懷,我爲你淚眼愁眉難解。無人處,自疑猜,生怕那慧性靈心偷改。
        寶玉通靈可愛,天生有眼堪穿。萬年幸一遇仙緣,從此春光美滿。隨時喜怒哀樂,遠卻離合悲歡。地久天長香影連,可意方舒心眼。
        寶玉銜來,是補天之餘,落地已久,得地氣收藏,因人而現。其性質內陽外陰,其形體光白溫潤,天生有眼可穿,故名曰寶玉,將欲得者盡寶愛此玉之意也。
        天地循環秋復春,生生死死舊重新。君家著筆描風月,寶玉顰顰解愛人。\end{note}
\end{parag}


\begin{parag}
    卻說雨村忙回頭看時,不是別人,乃是當日同僚一案參革的號張如圭\begin{note}甲側:蓋言如鬼如蜮也,亦非正人正言。\end{note}者。他本系此地人,革後家居,今打聽得都中奏準起復舊員之信,他便四下裏尋情找門路,忽遇見雨村,故忙道喜。二人見了禮,張如圭便將此信告訴雨村,雨村自是歡喜,忙忙的敘了兩句,\begin{note}甲側:畫出心事。\end{note}遂作別各自回家。冷子興聽得此言,便忙獻計,\begin{note}甲側:畢肖趕熱竈者。\end{note}令雨村央煩林如海,轉向都中去央煩賈政。雨村領其意,作別回至館中,忙尋邸報看真確了。\begin{note}甲側:細。\end{note}次日,面謀之如海。如海道:“天緣湊巧,因賤荊去世,都中家岳母念及小女無人依傍教育,前已遣了男女船隻來接,因小女未曾大痊,故未及行。此刻正思向蒙訓教之恩未經酬報,遇此機會,豈有不盡心圖報之理。但請放心,弟已預爲籌劃至此,已修下薦書一封,轉託內兄務爲周全協佐,方可稍盡弟之鄙誠,即有所費用之例,弟於內兄信中已註明白,亦不勞尊兄多慮矣。”雨村一面打恭,謝不釋口,一面又問:“不知令親大人現居何職?\begin{note}甲側:奸險小人欺人語。\end{note}只怕晚生草率,不敢驟然入都幹瀆。”\begin{note}甲側:全是假,全是詐。\end{note}如海笑道:“若論舍親,與尊兄猶系同譜,乃榮公之孫。大內兄現襲一等將軍,名赦,字恩侯,二內兄名政,字存周,\begin{note}甲側:二名二字皆頌德而來,與子興口中作證。\end{note}現任工部員外郎,其爲人謙恭厚道,大有祖父遺風,非膏粱輕薄仕宦之流,\begin{note}復醒一筆。\end{note}故弟方致書煩託。否則不但有污尊兄之清操,即弟亦不屑爲矣。”\begin{note}甲側:寫如海實寫政老。所謂此書有不寫之寫是也。\end{note}雨村聽了,心下方信了昨日子興之言,於是又謝了林如海。如海乃說:“已擇了出月初二日小女入都,尊兄即同路而往,豈不兩便?”雨村唯唯聽命,心中十分得意。
\end{parag}


\begin{parag}
    如海遂打點禮物幷餞行之事,雨村一一領了。
\end{parag}


\begin{parag}
    那女學生黛玉,身體方愈,原不忍棄父而往,無奈他外祖母致意務去,且兼如海說:“汝父年將半百,再無續室之意,且汝多病,年又極小,上無親母教養,下無姊妹兄弟扶持,\begin{note}甲側:可憐!一句一滴血,一句一滴血之文。\end{note}今依傍外祖母及舅氏姊妹去,正好減我顧盼之憂,何反雲不往?”黛玉聽了,方灑淚拜別,\begin{note}甲側:實寫黛玉。蒙側:此一段是不肯使黛玉作棄父樂爲遠遊者。以此可見作者之心寶愛黛玉如己。\end{note}隨了奶孃及榮府幾個老婦人登舟而去。雨村另有一隻船,帶兩個小童,依附黛玉而行。\begin{note}甲側:老師依附門生,怪道今時以收納門生爲幸。\end{note}
\end{parag}


\begin{parag}
    有日到了都中,\begin{note}甲側:繁中簡筆。\end{note}進入神京,雨村先整了衣冠,\begin{note}甲側:且按下黛玉以待細寫。今故先將雨村安置過一邊,方起榮府中之正文也。\end{note}帶了小童,\begin{note}甲側:至此漸漸好看起來也。\end{note}拿著宗侄的名帖,\begin{note}甲側:此帖妙極,可知雨村的品行矣。\end{note}至榮府的門前投了。彼時賈政已看了妹丈之書,即忙請入相會。見雨村相貌魁偉,言語不俗,且這賈政最喜讀書人,禮賢下士,濟弱扶危,大有祖風,況又系妹丈致意,因此優待雨村,\begin{note}甲側:君子可欺其方也,況雨村正在王莽謙恭下士之時,雖政老亦爲所惑,在作者係指東說西也。\end{note}更又不同,便竭力內中協助,題奏之日,輕輕謀\begin{note}甲側:《春秋》字法。\end{note}了一個復職候缺,不上兩個月,金陵應天府缺出,便謀補\begin{note}甲側:《春秋》字法。\end{note}了此缺,拜辭了賈政,擇日上任去了。\begin{note}甲側:因寶釵故及之,一語過至下回。\end{note}不在話下。
\end{parag}


\begin{parag}
    且說黛玉自那日棄舟登岸時,\begin{note}甲側:這方是正文起頭處。此後筆墨,與前兩回不同。\end{note}便有榮國府打發了轎子幷拉行李的車輛久候了。這林黛玉常聽得\begin{note}甲側:三字細。\end{note}母親說過,他外祖母家與別家不同。他近日所見的這幾個三等僕婦,喫穿用度,已是不凡了,何況今至其家。因此步步留心,時時在意,不肯輕易多說一句話,多行一步路,惟恐被人恥笑了他去。\begin{note}甲側:寫黛玉自幼之心機。[黛玉自忖之語。]\end{note}自上了轎,進入城中,從紗窗向外瞧了一瞧,其街市之繁華,人煙之阜盛,自與別處不同。\begin{note}甲側:先從街市寫來。\end{note}又行了半日,忽見街北蹲著兩個大石獅子,三間獸頭大門,門前列坐著十來個華冠麗服之人。正門卻不開,只有東西兩角門有人出入。正門之上有一匾,匾上大書“敕造寧國府”五個大字。\begin{note}甲側:先寫寧府,這是由東向西而來。\end{note}黛玉想道:“這必是外祖之長房了。”想著,又往西行,不多遠,照樣也是三間大門,方是榮國府了。卻不進正門,只進了西邊角門。那轎伕抬進去,走了一射之地,將轉彎時,便歇下退出去了。後面的婆子們已都下了轎,趕上前來。另換了三四個衣帽周全十七八歲的小廝上來,復抬起轎子。衆婆子步下圍隨至一垂花門前落下。衆小廝退出,衆婆子上來打起轎簾,扶黛玉下轎。林黛玉扶著婆子的手,進了垂花門,兩邊是抄手遊廊,當中是穿堂,當地放著一個紫檀架子大理石的大插屏。轉過插屏,小小的三間廳,廳後就是後面的正房大院。正面五間上房,皆雕樑畫棟,兩邊穿山遊廊廂房,掛著各色鸚鵡、畫眉等鳥雀。臺磯之上,坐著幾個穿紅著綠的丫頭,一見他們來了,便忙都笑迎上來,說:“剛纔老太太還念呢,可巧就來了。”\begin{note}甲側:如見如聞,活現於紙上之筆。好看煞!\end{note}於是三四人爭著打起簾籠,\begin{note}甲側:真有是事,真有是事!\end{note}一面聽得人回話:“林姑娘到了。”\begin{note}甲眉:此書得力處,全是此等地方,所謂“頰上三毫”也。\end{note}
\end{parag}


\begin{parag}
    黛玉方進入房時,只見兩個人攙著一位鬢髮如銀的老母迎上來,黛玉便知是他外祖母。方欲拜見時,早被他外祖母一把摟入懷中,心肝兒肉叫著大哭起來。\begin{note}甲側:幾千斤力量寫此一筆。\end{note}當下地下侍立之人,無不掩面涕泣,\begin{note}甲側:旁寫一筆,更妙!\end{note}黛玉也哭個不住。\begin{note}甲側:自然順寫一筆。\end{note}一時衆人慢慢解勸住了,黛玉方拜見了外祖母。\begin{note}甲眉:書中正文之人,卻如此寫出,卻是天生地設章法,不見一絲勉強。\end{note}此即冷子興所云之史氏太君,賈赦、賈政之母也。\begin{note}甲側:書中人目太繁,故明注一筆,使觀者省眼。\end{note}當下賈母一一指與黛玉:“這是你大舅母,\begin{note}邢氏。\end{note}這是你二舅母,\begin{note}王氏。\end{note}這是你先珠大哥的媳婦珠大嫂子。”黛玉一一拜見過。賈母又說:“請姑娘們來。今日遠客纔來,可以不必上學去了。”衆人答應了一聲,便去了兩個。
\end{parag}


\begin{parag}
    不一時,只見三個奶嬤嬤幷五六個丫鬟,簇擁著三個姊妹來了。\begin{note}甲側:聲勢如現紙上。甲眉:從黛玉眼中寫三人。\end{note}第一個肌膚微豐,\begin{note}甲側:不犯寶釵。\end{note}閤中身材,腮凝新荔,鼻膩鵝脂,溫柔沉默,觀之可親。\begin{note}甲側:爲迎春寫照。\end{note}第二個削肩細腰,\begin{note}甲側:《洛神賦》中雲“肩若削成”是也。\end{note}長挑身材,鴨蛋臉面,俊眼修眉,顧盼神飛,文彩精華,見之忘俗。\begin{note}甲側:爲探春寫照。\end{note}第三個身量未足,形容尚小。\begin{note}甲眉:渾寫一筆更妙!必個個寫去則板矣。可笑近之小說中有一百個女子,皆是如花似玉一副臉面。\end{note}其釵環裙襖,\begin{note}甲側:是極。\end{note}三人皆是一樣的妝飾。\begin{note}甲側:畢肖。\end{note}黛玉忙起身迎上來見禮,\begin{note}甲側:此筆亦不可少。\end{note}互相廝認過,大家歸了坐。丫鬟們斟上茶來。不過說些黛玉之母如何得病,如何請醫服藥,如何送死發喪。不免賈母又傷感起來,\begin{note}甲側:妙!\end{note}因說:“我這些兒女,所疼者獨有你母,今日一旦先舍我而去,連面也不能一見,今見了你,我怎不傷心!”說著,摟了黛玉在懷,又嗚咽起來。衆人忙都寬慰解釋,方略略止住。\begin{note}甲側:總爲黛玉自此不能別往。\end{note}
\end{parag}


\begin{parag}
    衆人見黛玉年貌雖小,其舉止言談不俗,身體面龐雖怯弱不勝,\begin{note}甲側:寫美人是如此筆仗,看官怎得不叫絕稱賞!\end{note}卻有一段自然的風流態度,\begin{note}甲側:爲黛玉寫照。衆人目中,只此一句足矣。甲眉:從衆人目中寫黛玉。草胎卉質,豈能勝物耶?想其衣裙皆不得不勉強支持者也。\end{note}便知他有不足之症。因問:“常服何藥,如何不急爲療治?”黛玉道:“我自來是如此,從會喫飲食時便吃藥,到今日未斷,請了多少名醫修方配藥,皆不見效。那一年我三歲時,聽得說\begin{note}甲側:文字細如牛毛。\end{note}來了一個癩頭和尚,\begin{note}甲眉:奇奇怪怪一至於此。通部中假借癩僧、跛道二人點明迷情幻海中有數之人也。非襲《西遊》中一味無稽、至不能處便用觀世音可比。\end{note}說要化我去出家,我父母固是不從。他又說:‘既捨不得他,只怕他的病一生也不能好的了。若要好時,除非從此以後總不許見哭聲,除父母之外,凡有外姓親友之人,一概不見,方可平安了此一世。’瘋瘋癲癲,說了這些不經之談,\begin{note}甲側:是作書者自注。\end{note}也沒人理他。如今還是喫人蔘養榮丸。”\begin{note}甲側:人生自當自養榮衛。甲眉:甄英蓮乃副十二釵之首,卻明寫癩僧一點。今黛玉爲正十二釵之冠,反用暗筆。蓋正十二釵人或洞悉可知,副十二釵或恐觀者忽略,故寫極力一提,使觀者萬勿稍加翫忽之意耳。\end{note}賈母道:“正好,我這裏正配丸藥呢。叫他們多配一料就是了。”\begin{note}甲側:爲後菖菱伏脈。\end{note}
\end{parag}


\begin{parag}
    一語未了,只聽後院中有人笑聲,\begin{note}甲側:懦筆庸筆何能及此!\end{note}說:“我來遲了,不曾迎接遠客!”\begin{note}甲側:第一筆,阿鳳三魂六魄已被作者拘定了,後文焉得不活跳紙上?此等文字非仙助即神助,從何而得此機括耶?甲眉:另磨新墨,搦銳筆,特獨出熙鳳一人。未見其人,先使聞聲,所謂“繡幡開,遙見英雄俺”也。\end{note}黛玉納罕道:“這些人個個皆斂聲屏氣,恭肅嚴整如此,這來者系誰,這樣放誕無禮?”\begin{note}甲側:原有此一想。\end{note}心下想時,只見一羣媳婦丫鬟圍擁著一個人從後房門進來。這個人打扮與衆姑娘不同,彩繡輝煌,恍若神妃仙子:
\end{parag}

\begin{qute2sp}
    頭上戴著金絲八寶攢珠髻,綰著朝陽五鳳掛珠釵,\begin{note}甲側:頭。\end{note}項上戴著赤金盤螭瓔珞圈,\begin{note}甲側:頸。\end{note}裙邊系著豆綠宮絛,雙衡比目玫瑰佩,\begin{note}甲側:腰。\end{note}身上穿著縷金百蝶穿花大紅洋緞窄褃襖,外罩五彩刻絲石青銀鼠褂,下著翡翠撒花洋縐裙。一雙丹鳳三角眼,兩彎柳葉吊梢眉,身量苗條,體格風騷,粉面含春威不露,丹脣未啓笑先開。\begin{note}甲側:爲阿鳳寫照。甲眉:試問諸公:從來小說中可有寫形追像至此者?\end{note}
\end{qute2sp}

\begin{parag}
    黛玉連忙起身接見。賈母笑\begin{note}甲側:阿鳳一至,賈母方笑,與後文多少笑字作偶。\end{note}道:“你不認得他,他是我們這裏有名的一個潑皮破落戶兒,南省俗謂作‘辣子’,你只叫他‘鳳辣子’就是了。”\begin{note}甲側:阿鳳笑聲進來,老太君打諢,雖是空口傳聲,卻是補出一向晨昏起居,阿鳳於太君處承歡應候一刻不可少之人,看官勿以閒文淡文也。\end{note}黛玉正不知以何稱呼,只見衆姊妹都忙告訴他道:“這是璉嫂子。”黛玉雖不識,也曾聽見母親說過,大舅賈赦之子賈璉,娶的就是二舅母王氏之內侄女,自幼假充男兒教養的,學名王熙鳳。\begin{note}甲側:奇想奇文。以女子曰“學名”固奇,然此偏有學名的反倒不識字,不曰學名者反若假。\end{note}黛玉忙陪笑見禮,以“嫂”呼之。這熙鳳攜著黛玉的手,上下細細打諒了一回,\begin{note}甲側:寫阿鳳全部傳神第一筆也。\end{note}仍送至賈母身邊坐下,因笑道:“天下真有這樣標緻的人物,我今兒纔算見了!\begin{note}甲側:這方是阿鳳言語。若一味浮詞套語,豈復爲阿鳳哉!甲眉:“真有這樣標緻人物”出自鳳口,黛玉丰姿可知。宜作史筆看。\end{note}況且這通身的氣派,竟不象老祖宗的外孫女兒,竟是個嫡親的孫女,\begin{note}甲側:仍歸太君,方不失《石頭記》文字,且是阿鳳身心之至文。\end{note}怨不得老祖宗天天口頭心頭一時不忘。\begin{note}甲側:卻是極淡之語,偏能恰投賈母之意。\end{note}只可憐我這妹妹這樣命苦,\begin{note}甲側:這是阿鳳見黛玉正文。\end{note}怎麼姑媽偏就去世了!”\begin{note}甲側:若無這幾句,便不是賈府媳婦。\end{note}說著,便用帕拭淚。賈母笑道:“我纔好了,你倒來招我。\begin{note}甲側:文字好看之極。\end{note}你妹妹遠路纔來,身子又弱,也才勸住了,快再休提前話!”\begin{note}甲側:反用賈母勸,看阿鳳之術亦甚矣。\end{note}這熙鳳聽了,忙轉悲爲喜道:“正是呢!我一見了妹妹,一心都在他身上了,又是喜歡,又是傷心,竟忘記了老祖宗。該打,該打!”又忙攜黛玉之手,問:“妹妹幾歲了?可也上過學?現喫什麼藥?在這裏不要想家,想要什麼喫的,什麼玩的,只管告訴我,丫頭老婆們不好了,也只管告訴我。”一面又問婆子們:“林姑娘的行李東西可搬進來了?帶了幾個人來?\begin{note}甲側:當家的人事如此,畢肖!\end{note}你們趕早打掃兩間下房,讓他們去歇歇。”
\end{parag}



\begin{parag}
    說話時,已擺了茶果上來,熙鳳親爲捧茶捧果。\begin{note}甲側:總爲黛玉眼中寫出。\end{note}又見二舅母問他:“月錢放過了不曾?”\begin{note}甲側:不見後文,不見此筆之妙。\end{note}熙鳳道:“月錢已放完了。纔剛帶著人到後樓上找緞子,\begin{note}甲側:接閒文,是本意避繁也。\end{note}找了這半日,也幷沒有見昨日太太說的那樣的。\begin{note}甲側:卻是日用家常實事。\end{note}想是太太記錯了?”王夫人道:“有沒有,什麼要緊。”因又說道:“該隨手拿出兩個來給你這妹妹去裁衣裳的,\begin{note}甲側:仍歸前文。妙妙!\end{note}等晚上想著叫人再去拿罷,可別忘了。”熙鳳道:“這倒是我先料著了,知道妹妹不過這兩日到的,我已預備下了,\begin{note}甲眉:餘知此緞阿鳳幷未拿出,此借王夫人之語機變欺人處耳。若信彼果拿出預備,不獨被阿鳳瞞過,亦且被石頭瞞過了。\end{note}等太太回去過了目好送來。”\begin{note}甲側:試看他心機。\end{note}王夫人一笑,點頭不語。\begin{note}甲側:深取之意。[鳳姐是個當家人。]\end{note}
\end{parag}


\begin{parag}
    當下茶果已撤,賈母命兩個老嬤嬤帶了黛玉去見兩個母舅。時賈赦之妻邢氏忙亦起身,笑回道:“我帶了外甥女過去,倒也便宜。”賈母笑道:“正是呢,你也去罷,不必過來了。”邢夫人答應了一聲“是”字,遂帶了黛玉與王夫人作辭,大家送至穿堂前。出了垂花門,早有衆小廝們拉過一輛翠幄青綢車。邢夫人攜了黛玉,坐在上面,\begin{note}[未識黛卿能乘此否?]\end{note}衆婆子們放下車簾,方命小廝們抬起,拉至寬處,方駕上馴騾,亦出了西角門,往東過榮府正門,便入一黑油大門中,至儀門前方下來。衆小廝退出,方打起車簾,邢夫人攙著黛玉的手,進入院中。黛玉度其房屋院宇,必是榮府中花園隔斷過來的。\begin{note}甲側:黛玉之心機眼力。\end{note}進入三層儀門,果見正房廂廡遊廊,悉皆小巧別緻,不似方纔那邊軒峻壯麗,且院中隨處之樹木山石皆有。\begin{note}甲側:爲大觀園伏脈。試思榮府園今在西,後之大觀園偏寫在東,何不畏難之若此?\end{note}一時進入正室,早有許多盛妝麗服之姬妾丫鬟迎著,邢夫人讓黛玉坐了,一面命人到外面書房去請賈赦。\begin{note}甲側:這一句都是寫賈赦,妙在全是指東擊西打草驚蛇之筆。若看其寫一人即作此一人看,先生便呆了。\end{note}一時人來回話說:“老爺說了:‘連日身上不好,見了姑娘彼此倒傷心,\begin{note}甲側:追魂攝魄。甲眉:餘久不作此語矣,見此語未免一醒。\end{note}暫且不忍相見。\begin{note}甲側:若一見時,不獨死板,且亦大失情理,亦不能有此等妙文矣。\end{note}勸姑娘不要傷心想家,跟著老太太和舅母,即同家裏一樣。姊妹們雖拙,大家一處伴著,亦可以解些煩悶。\begin{note}甲側:赦老亦能作此語,嘆嘆!\end{note}或有委屈之處,只管說得,不要外道纔是。’”黛玉忙站起來,一一聽了。再坐一刻,便告辭。邢夫人苦留喫過晚飯去,黛玉笑回道:“舅母愛惜賜飯,原不應辭,只是還要過去拜見二舅舅,恐領了賜去不恭,\begin{note}甲側:得體。\end{note}異日再領,未爲不可。望舅母容諒。”邢夫人聽說,笑道:“這倒是了。”遂令兩三個嬤嬤用方纔的車好生送了姑娘過去,於是黛玉告辭。邢夫人送至儀門前,又囑咐了衆人幾句,眼看著車去了方回來。
\end{parag}


\begin{parag}
    一時黛玉進了榮府,下了車。衆嬤嬤引著,便往東轉彎,穿過一個東西的穿堂,\begin{note}甲側:這一個穿堂是賈母正房之南者,鳳姐處所通者則是賈母正房之北。\end{note}向南大廳之後,儀門內大院落,上面五間大正房,兩邊廂房鹿頂耳房鑽山,四通八達,軒昂壯麗,比賈母處不同。黛玉便知這方是正經正內室,一條大甬路,直接出大門的。進入堂屋中,抬頭迎面先看見一個赤金九龍青地大匾,匾上寫著斗大的三個大字,是“榮禧堂”,後有一行小字“某年月日,書賜榮國公賈源”,又有“萬幾宸翰之寶”。大紫檀雕螭案上,設著三尺來高青綠古銅鼎,懸著待漏隨朝墨龍大畫,一邊是金蜼彝,\begin{note}甲側:蜼,音壘。周器也。\end{note}一邊是玻璃臺。\begin{note}甲側:(上臺下皿),音海。盛酒之大器也。\end{note}地下兩溜十六張楠木交椅。又有一副對聯,乃烏木聯牌,鑲著鏨銀的字跡,\begin{note}甲側:雅而麗,富而文。\end{note}道是:
\end{parag}


\begin{poem}
    \begin{pl} 座上珠璣昭日月,\end{pl}

    \begin{pl}堂前黼黻煥煙霞。\end{pl}\begin{note}甲夾:實貼。\end{note}
\end{poem}


\begin{parag}
    下面一行小字,道是:“同鄉世教弟勳襲東安郡王穆蒔拜手書。”\begin{note}甲側:先虛陪一筆。\end{note}
\end{parag}


\begin{parag}
    原來王夫人時常居坐宴息,亦不在這正室,\begin{note}甲側:黛玉由正室一段而來,是爲拜見政老耳,故進東房。\end{note}只在這正室東邊的三間耳房內。\begin{note}甲側:若見王夫人,直寫引至東廊小正室內矣。\end{note}於是老嬤嬤引黛玉進東房門來。臨窗大炕上鋪著猩紅洋罽,正面設著大紅金錢蟒靠背,石青金錢蟒引枕,秋香色金錢蟒大條褥。兩邊設一對梅花式洋漆小几。左邊几上文王鼎匙箸香盒,右邊几上汝窯美人觚“”觚內插著時鮮花卉,幷茗碗痰盒等物。地下面西一溜四張椅上,都搭著銀紅撒花椅搭,底下四副腳踏。椅之兩邊,也有一對高几,几上茗碗瓶花俱備。其餘陳設,自不必細說。\begin{note}甲側:此不過略敘榮府家常之禮數,特使黛玉一識階級座次耳,餘則繁。\end{note}老嬤嬤們讓黛玉炕上坐,炕沿上卻有兩個錦褥對設,黛玉度其位次,便不上炕,只向東邊椅子上坐了。\begin{note}甲側:寫黛玉心意。\end{note}本房內的丫鬟忙捧上茶來。黛玉一面喫茶,一面打諒這些丫鬟們,裝飾衣裙,舉止行動,果亦與別家不同。
\end{parag}


\begin{parag}
    茶未吃了,只見一個穿紅綾襖青緞掐牙背心的丫鬟\begin{note}甲側:金乎?玉乎?\end{note}走來笑說道:“太太說,請林姑娘到那邊坐罷。”老嬤嬤聽了,於是又引黛玉出來,到了東廊三間小正房內。正房炕上橫設一張炕桌,桌上磊著書籍茶具,\begin{note}甲側:傷心筆,墮淚筆。\end{note}靠東壁面西設著半舊的青緞靠背引枕。王夫人卻坐在西邊下首,亦是半舊的青緞靠背坐褥。見黛玉來了,便往東讓。黛玉心中料定這是賈政之位。\begin{note}甲側:寫黛玉心到眼到,傖夫但云爲賈府敘坐位,豈不可笑?\end{note}因見挨炕一溜三張椅子上,也搭著半舊的\begin{note}甲側:三字有神。此處則一色舊的,可知前正室中亦非家常之用度也。可笑近之小說中,不論何處,則曰商彝周鼎、繡幕珠簾、孔雀屏、芙蓉褥等樣字眼。甲眉:近聞一俗笑語云:一莊農人進京回家,衆人問曰:“你進京去可見些個世面否?”莊人曰:“連皇帝老爺都見了。”衆罕然問曰:“皇帝如何景況?”莊人曰:“皇帝左手拿一金元寶,右手拿一銀元寶,馬上稍著一口袋人蔘,行動人蔘不離口。一時要屙屎了,連擦屁股都用的是鵝黃緞子,所以京中掏茅廁的人都富貴無比。”試思凡稗官寫富貴字眼者,悉皆莊農進京之一流也。蓋此時彼實未身經目睹,所言皆在情理之外焉。又如人嘲作詩者亦往往愛說富麗語,故有“脛骨變成金玳瑁,,眼睛嵌作碧璃琉”之誚。餘自是評《石頭記》,非鄙棄前人也。\end{note}彈墨椅袱,黛玉便向椅上坐了。王夫人再四攜他上炕,他方挨王夫人坐了。王夫人因說:“你舅舅今日齋戒去了,\begin{note}甲側:點綴宦途。\end{note}再見罷。\begin{note}甲側:赦老不見,又寫政老。政老又不能見,是重不見重,犯不見犯。作者慣用此等章法。\end{note}只是有一句話囑咐你:你三個姊妹倒都極好,以後一處唸書認字學針線,或是偶一頑笑,都有儘讓的。但我不放心的最是一件:我有一個孽根禍胎,\begin{note}甲側:四字是血淚盈面,不得已無奈何而下。四字是作者痛哭。\end{note}是家裏的‘混世魔王’,\begin{note}甲側:與“絳洞花王”爲對看。\end{note}今日因廟裏還願去了,\begin{note}甲側:是富貴公子。\end{note}尚未回來,晚間你看見便知了。你只以後不要睬他,你這些姊妹都不敢沾惹他的。”
\end{parag}


\begin{parag}
    黛玉亦常聽得母親說過,二舅母生的有個表兄,乃銜玉而誕,頑劣異常,\begin{note}甲側:與甄家子恰對。\end{note}極惡讀書,\begin{note}甲側:是極惡每日“詩云”“子曰”的讀書。\end{note}最喜在內幃廝混,外祖母又極溺愛,無人敢管。今見王夫人如此說,便知說的是這表兄了。\begin{note}甲側:這是一段反襯筆法。黛玉心用“猜度蠢物”等句對著去,方不失作者本旨。\end{note}因陪笑道:“舅母說的,可是銜玉所生的這位哥哥?在家時亦曾聽見母親常說,這位哥哥比我大一歲,小名就喚寶玉,\begin{note}甲側:以黛玉道寶玉名,方不失正文。\end{note}雖\begin{note}甲側:“雖”字是有情字,宿根而發,勿得泛泛看過。\end{note}極憨頑,說在姊妹情中極好的。況我來了,自然只和姊妹同處,兄弟們自是別院另室的,\begin{note}甲側:又登開一筆,妙妙!\end{note}豈得去沾惹之理?”王夫人笑道:“你不知道原故。他與別人不同,自幼因老太太疼愛,原系同姊妹們一處嬌養慣了的。\begin{note}甲側:此一筆收回,是明通部同處原委也。\end{note}若姊妹們有日不理他,他倒還安靜些,縱然他沒趣,不過出了二門,背地裏拿著他兩個小麼兒出氣,咕唧一會子就完了。\begin{note}甲側:這可是寶玉本性真情,前四十九字迥異之批今始方知。蓋小人口碑累累如是。是是非非任爾口角,大都皆然。\end{note}若這一日姊妹們和他多說一句話,他心裏一樂,便生出多少事來。所以囑咐你別睬他。他嘴裏一時甜言蜜語,一時有天無日,一時又瘋瘋傻傻,只休信他。”
\end{parag}


\begin{parag}
    黛玉一一的都答應著。\begin{note}甲眉:不寫黛玉眼中之寶玉,卻先寫黛玉心中已早有一寶玉矣,幻妙之至!自冷子興口中之後,餘已極思欲一見,及今尚未得見,狡猾之至!\end{note}只見一個丫鬟來回:“老太太那裏傳晚飯了。”王夫人忙攜黛玉從後房門\begin{note}甲側:後房門。\end{note}由後廊\begin{note}甲側:是正房後廊也。\end{note}往西,出了角門,\begin{note}甲側:這是正房後西界牆角門。\end{note}是一條南北寬夾道。南邊是倒座三間小小的抱廈廳,北邊立著一個粉油大影壁,後有一半大門,小小一所房室。王夫人笑指向黛玉道:“這是你鳳姐姐的屋子,回來你好往這裏找他來,少什麼東西,你只管和他說就是了。”這院門上也有\begin{note}甲側:二字是他處不寫之寫也。\end{note}四五個才總角的小廝,都垂手侍立。王夫人遂攜黛玉穿過一個東西穿堂,\begin{note}甲眉:這正是賈母正室後之穿堂也,與前穿堂是一帶之屋,中一帶乃賈母之下室也。記清。\end{note}便是賈母的後院了。\begin{note}甲側:寫得清,一絲不錯。\end{note}於是,進入後房門,已有多人在此伺候,見王夫人來了,方安設桌椅。\begin{note}甲側:不是待王夫人用膳,是恐使王夫人有失侍膳之禮耳。\end{note}賈珠之妻李氏捧飯,熙鳳安箸,王夫人進羹。賈母正面榻上獨坐,兩邊四張空椅,熙鳳忙拉了黛玉在左邊第一張椅上坐了,黛玉十分推讓。賈母笑道:“你舅母你嫂子們不在這裏喫飯。你是客,原應如此坐的。”黛玉方告了座,坐了。賈母命王夫人坐了。迎春姊妹三個告了座方上來。迎春便坐右手第一,探春左第二,惜春右第二。旁邊丫鬟執著拂塵、漱盂、巾帕。李、鳳二人立於案旁佈讓。外間伺候之媳婦丫鬟雖多,卻連一聲咳嗽不聞。寂然飯畢,各有丫鬟用小茶盤捧上茶來。當日林如海教女以惜福養身,雲飯後務待飯粒咽盡,過一時再喫茶,方不傷脾胃。\begin{note}甲側:夾寫如海一派書氣,最妙!\end{note}今黛玉見了這裏許多事情不合家中之式,不得不隨的,少不得一一改過來,因而接了茶。早見人又捧過漱盂來,黛玉也照樣漱了口。盥手畢,又捧上茶來,這方是喫的茶。\begin{note}甲側:總寫黛玉以後之事,故只以此一件小事略爲一表也。甲眉:餘看至此,故想日前所閱“王敦初尚公主,登廁時不知塞鼻用棗,敦輒取而啖之,早爲宮人鄙誚多矣”。今黛玉若不漱此茶,或飲一口,不爲榮婢所誚乎?觀此則知黛玉平生之心思過人。\end{note}賈母便說:“你們去罷,讓我們自在說話兒。”王夫人聽了,忙起身,又說了兩句閒話,方引鳳、李二人去了。賈母因問黛玉念何書。黛玉道:“只剛唸了《四書》。”\begin{note}甲側:好極!稗官專用“腹隱五車書”者來看。\end{note}黛玉又問姊妹們讀何書。賈母道:“讀的是什麼書,不過是認得兩個字,不是睜眼的瞎子罷了!”
\end{parag}


\begin{parag}
    一語未了,只聽外面一陣腳步響,\begin{note}甲側:與阿鳳之來相映而不相犯。\end{note}丫鬟進來笑道:“寶玉來了!”\begin{note}甲側:餘爲一樂。\end{note}黛玉心中正疑惑著:“這個寶玉,不知是怎生個憊懶人物,懵懂頑童?”\begin{note}甲側:文字不反,不見正文之妙,似此應從《國策》得來。\end{note}倒不見那蠢物\begin{note}甲側:這蠢物不是那蠢物,卻有個極蠢之物相待。妙極!\end{note}也罷了。心中想著,忽見丫鬟話未報完,已進來了一位年輕的公子:
\end{parag}


\begin{qute2sp}
    頭上戴著束髮嵌寶紫金冠,齊眉勒著二龍搶珠金抹額,穿一件二色金百蝶穿花大紅箭袖,束著五彩絲攢花結長穗宮絛,外罩石青起花八團倭鍛排穗褂,登著青緞粉底小朝靴。面若中秋之月,\begin{note}甲眉:此非套“滿月”,蓋人生有面扁而青白色者,則皆可謂之秋月也。用“滿月”者不知此意。\end{note}色如春曉之花。\begin{note}甲眉:“少年色嫩不堅牢”,以及“非夭即貧”之語,餘猶在心。今閱至此,放聲一哭。\end{note}鬢若刀裁,眉如墨畫,面如桃瓣,目若秋波。雖怒時而若笑,即嗔視而有情。
\end{qute2sp}


\begin{parag}
    \begin{note}甲側:真真寫殺。\end{note}項上金螭瓔珞,又有一根五色絲絛,系著一塊美玉。黛玉一見,便喫一大驚,心下想道:“好生奇怪,倒象在那裏見過一般,何等眼熟到如此!”\begin{note}甲側:正是想必在靈河岸上三生石畔曾見過。\end{note}只見這寶玉向賈母請了安,賈母便命:“去見你娘來。”寶玉即轉身去了。一時回來,再看,已換了冠帶:頭上週圍一轉的短髮,都結成小辮,紅絲結束,共攢至頂中胎髮,總編一根大辮,黑亮如漆,從頂至梢,一串四顆大珠,用金八寶墜角,身上穿著銀紅撒花半舊大襖,仍舊帶著項圈、寶玉、寄名鎖、護身符等物,下面半露松花撒花綾褲腿,錦邊彈墨襪,厚底大紅鞋。越顯得面如敷粉,脣若施脂,轉盼多情,語言常笑。天然一段風騷,全在眉梢,平生萬種情思,悉堆眼角。看其外貌最是極好,卻難知其底細。後人有《西江月》二詞,批寶玉極恰,\begin{note}甲眉:二詞更妙。最可厭野史“貌如潘安”“才如子建”等語。\end{note}其詞曰:
\end{parag}


\begin{poem}
    \begin{pl}無故尋愁覓恨,有時似傻如狂。\end{pl}
    \begin{pl}縱然生得好皮囊,腹內原來草莽。\end{pl}

    \begin{pl}潦倒不通世務,愚頑怕讀文章。\end{pl}
    \begin{pl}行爲偏僻性乖張,那管世人誹謗!\end{pl}

    \begin{pl}富貴不知樂業,貧窮難耐淒涼。\end{pl}
    \begin{pl}可憐辜負好韶光,於國於家無望。\end{pl}

    \begin{pl}天下無能第一,古今不肖無雙。\end{pl}
    \begin{pl}寄言紈絝與膏粱,莫效此兒形狀!\end{pl}
    \begin{note}甲眉:末二語最緊要。只是紈絝膏粱,亦未必不見笑我玉卿。可知能效一二者,亦必不是蠢然紈絝矣。\end{note}
\end{poem}


\begin{parag}
    賈母因笑道:“外客未見,就脫了衣裳,還不去見你妹妹!”寶玉早已看見多了一個姊妹,便料定是林姑媽之女,忙來作揖。廝見畢歸坐,細看形容,\begin{note}甲眉:又從寶玉目中細寫一黛玉,直畫一美人圖。\end{note}與衆各別:兩彎似蹙非蹙罥煙眉,\begin{note}甲側:奇眉妙眉,奇想妙想。\end{note}一雙似泣非泣含露目。\begin{note}甲側:奇目妙目,奇想妙想。\end{note}態生兩靨之愁,嬌襲一身之病。淚光點點,嬌喘微微。閒靜時如姣花照水,行動處似弱柳扶風。\begin{note}甲側:至此八句是寶玉眼中。\end{note}心較比干多一竅,\begin{note}甲側:此一句是寶玉心中。甲眉:更奇妙之至!多一竅固是好事,然未免偏僻了,所謂“過猶不及”也。\end{note}病如西子勝三分。\begin{note}甲側:此十句定評,直抵一賦。甲眉:不寫衣裙妝飾,正是寶玉眼中不屑之物,故不曾看見。黛玉之舉止容貌,亦是寶玉眼中看、心中評。若不是寶玉,斷不能知黛玉是何等品貌。\end{note}寶玉看罷,因笑\begin{note}甲眉:黛玉見寶玉寫一“驚”字,寶玉見黛玉寫一“笑”字,一存於中,一發乎外,可見文於下筆必推敲的準穩,方纔用字。\end{note}道:\begin{note}甲側:看他第一句是何話。\end{note}“這個妹妹我曾見過的。”\begin{note}甲側:瘋話。與黛玉同心,卻是兩樣筆墨。觀此則知玉卿心中有則說出,一毫宿滯皆無。\end{note}賈母笑道:“可又是胡說,你又何曾見過他?”寶玉笑道:“雖然未曾見過他,然我看著面善,心裏就算是舊相識,\begin{note}甲側:一見便作如是語,宜乎王夫人謂之瘋瘋傻傻也。\end{note}今日只作遠別重逢,亦未爲不可。”\begin{note}甲側:妙極奇語,全作如是等語。無怪人謂曰癡狂。\end{note}賈母笑道:“更好,更好。\begin{note}甲側:作小兒語瞞過世人亦可。\end{note}若如此,更相和睦了。”\begin{note}甲側:亦是真話。\end{note}寶玉便走近黛玉身邊坐下,又細細打諒一番,\begin{note}甲側:與黛玉兩次打諒一對。\end{note}因問:“妹妹可曾讀書?”\begin{note}甲側:自己不讀書,卻問到人,妙!\end{note}黛玉道:“不曾讀,只上了一年學,些須認得幾個字。”寶玉又道:“妹妹尊名是那兩個字?”黛玉便說了名。寶玉又問表字,黛玉道:“無字。”寶玉笑道:“我送妹妹一妙字,莫若‘顰顰’二字極妙。”探春\begin{note}甲側:寫探春。\end{note}便問何出。寶玉道:“《古今人物通考》上說:‘西方有石名黛,可代畫眉之墨。’況這林妹妹眉尖若蹙,用取這兩個字,豈不兩妙!”探春笑道:“只恐又是你的杜撰。”寶玉笑道:“除《四書》外,杜撰的太多,偏只我是杜撰不成?”\begin{note}甲側:如此等語,焉得怪彼世人謂之怪?只瞞不過批書者。\end{note}又問黛玉:“可也有玉沒有?”\begin{note}甲側:奇極怪極,癡極愚極,焉得怪人目爲癡哉?\end{note}衆人不解其語,黛玉便忖度著:“因他有玉,故問我有也無。”\begin{note}甲眉:奇之至,怪之至,又忽將黛玉亦寫成一極癡女子,觀此初會二人之心,則可知以後之事矣。\end{note}因答道:“我沒有那個。想來那玉是一件罕物,豈能人人有的。”寶玉聽了,登時發作起癡狂病來,摘下那玉,就狠命摔去,\begin{note}甲側:試問石兄:此一摔,比在青埂峯下蕭然坦臥何如?\end{note}罵道:“什麼罕物,連人之高低不擇,還說‘通靈’不‘通靈’呢!我也不要這勞什子了!”嚇的衆人一擁爭去拾玉。賈母急的摟了寶玉道:“孽障!\begin{note}甲側:如聞其聲,恨極語卻是疼極語。\end{note}你生氣,要打罵人容易,何苦摔那命根子!”\begin{note}甲側:一字一千斤重。\end{note}寶玉滿面淚痕泣\begin{note}甲側:千奇百怪,不寫黛玉泣,卻反先寫寶玉泣。\end{note}道:“家裏姐姐妹妹都沒有,單我有,我說沒趣,如今來了這們一個神仙似的妹妹也沒有,可知這不是個好東西。”\begin{note}甲眉:“不是冤家不聚頭”第一場也。\end{note}賈母忙哄他道:“你這妹妹原有這個來的,因你姑媽去世時,捨不得你妹妹,無法處,遂將他的玉帶了去了。一則全殉葬之禮,盡你妹妹之孝心,二則你姑媽之靈,亦可權作見了女兒之意。因此他只說沒有這個,不便自己誇張之意。你如今怎比得他?還不好生慎重帶上,仔細你娘知道了。”說著,便向丫鬟手中接來,親與他帶上。寶玉聽如此說,想一想大有情理,也就不生別論了。\begin{note}甲側:所謂小兒易哄,餘則謂“君子可欺以其方”雲。\end{note}
\end{parag}


\begin{parag}
    當下,奶孃來請問黛玉之房舍。賈母說:“今將寶玉挪出來,同我在套間暖閣兒裏,把你林姑娘暫安置  紗櫥裏。等過了殘冬,春天再與他們收拾房屋,另作一番安置罷。”寶玉道:“好祖宗,\begin{note}甲側:跳出一小兒。\end{note}我就在  紗櫥外的牀上很妥當,何必又出來鬧的老祖宗不得安靜。”賈母想了一想說:“也罷了。”每人一個奶孃幷一個丫頭照管,餘者在外間上夜聽喚。一面早有熙鳳命人送了一頂藕合色花帳,幷幾件錦被緞褥之類。
\end{parag}


\begin{parag}
    黛玉只帶了兩個人來:一個是自幼奶孃王嬤嬤,一個是十歲的小丫頭,亦是自幼隨身的,名喚作雪雁。\begin{note}甲側:新雅不落套,是黛玉之文章也。\end{note}賈母見雪雁甚小,一團孩氣,王嬤嬤又極老,料黛玉皆不遂心省力的,便將自己身邊的一個二等丫頭,名喚鸚哥\begin{note}甲眉:妙極!此等名號方是賈母之文章。最厭近之小說中,不論何處,滿紙皆是紅娘、小玉、嬌紅、香翠等俗字。\end{note}者與了黛玉。外亦如迎春等例,每人除自幼乳母外,另有四個教引嬤嬤,除貼身掌管釵釧盥沐兩個丫鬟外,另有五六個灑掃房屋來往使役的小丫鬟。當下,王嬤嬤與鸚哥陪侍黛玉在碧紗櫥內。寶玉之乳母李嬤嬤,幷大丫鬟名喚襲人\begin{note}甲側:奇名新名,必有所出。\end{note}者,陪侍在外面大牀上。
\end{parag}


\begin{parag}
    原來這襲人亦是賈母之婢,本名珍珠。\begin{note}甲側:亦是賈母之文章。前鸚哥已伏下一鴛鴦,今珍珠又伏下一  珀矣。以下乃寶玉之文章。\end{note}賈母因溺愛寶玉,生恐寶玉之婢無竭力盡忠之人,素喜襲人心地純良,克盡職任,遂與了寶玉。寶玉因知他本姓花,又曾見舊人詩句上有“花氣襲人”之句,遂回明賈母,更名襲人。這襲人亦有些癡處:\begin{note}甲側:只如此寫又好極!最厭近之小說中,滿紙“千伶百俐”“這妮子亦通文墨”等語。\end{note}伏侍賈母時,心中眼中只有一個賈母,如今服侍寶玉,心中眼中又只有一個寶玉。只因寶玉性情乖僻,每每規諫寶玉,心中著實憂鬱。\begin{note}蒙側:我讀至此,不覺放聲大哭。\end{note}
\end{parag}


\begin{parag}
    是晚,寶玉李嬤嬤已睡了,他見裏面黛玉和鸚哥猶未安息,他自卸了妝,悄悄進來,笑問:“姑娘怎麼還不安息?”黛玉忙讓:“姐姐請坐。”襲人在牀沿上坐了。鸚哥笑道:“林姑娘正在這裏傷心,\begin{note}甲側:可知前批不謬。\end{note}自己淌眼抹淚\begin{note}甲側:黛玉第一次哭卻如此寫來。甲眉:前文反明寫寶玉之哭,今卻反如此寫黛玉,幾被作者瞞過。這是第一次算還,不知下剩還該多少?\end{note}的說:‘今兒纔來,就惹出你家哥兒的狂病,倘或摔壞了那玉,豈不是因我之過!’\begin{note}甲側:所謂寶玉知己,全用體貼功夫。蒙:我也心疼,豈獨顰顰!\end{note}因此便傷心,我好容易勸好了。”襲人道:“姑娘快休如此,將來只怕比這個更奇怪的笑話兒還有呢!若爲他這種行止,你多心傷感,只怕你傷感不了呢。快別多心!”\begin{note}蒙側:後百十回黛玉之淚,總不能出此二語。“月上窗紗人到階,窗上影兒先進來”,筆未到而境先到矣。[應知此非傷感,來還甘露水也。]\end{note}黛玉道:“姐姐們說的,我記著就是了。究竟那玉不知是怎麼個來歷?上面還有字跡?”襲人道:“連一家子也不知來歷,上頭還有現成的眼兒,聽得說,落草時是從他口裏掏出來的。\begin{note}甲側:癩僧幻術亦太奇矣。蒙側:天生帶來美玉,有現成可穿之眼,豈不可愛,豈不可惜!\end{note}等我拿來你看便知。”黛玉忙止道:“罷了,此刻夜深,明日再看也不遲。”\begin{note}甲側:總是體貼,不肯多事。蒙側:他天生帶來的美玉,他自己不愛惜,遇知己替他愛惜,連我看書的人也著實心疼不了,不覺揹人一哭,以謝作者。\end{note}大家又敘了一回,方纔安歇。
\end{parag}


\begin{parag}
    次日起來,省過賈母,因往王夫人處來,正值王夫人與熙鳳在一處拆金陵來的書信看,又有王夫人之兄嫂處遣了兩個媳婦來說話的。黛玉雖不知原委,探春等卻都曉得是議論金陵城中所居的薛家姨母之子姨表兄薛蟠,倚財仗勢,打死人命,現在應天府案下審理。如今母舅王子騰得了信息,故遣他家內的人來告訴這邊,意欲喚取進京之意。
\end{parag}


\begin{parag}
    \begin{note}蒙:補不完的是離恨天,所餘之石豈非離恨石乎。而絳珠之淚偏不因離恨而落,爲惜其石而落。可見惜其石必惜其人,其人不自惜,而知己能不千方百計爲之惜乎?所以絳珠之淚至死不幹,萬苦不怨。所謂求仁得仁,又何怨。悲夫!\end{note}
\end{parag}