\chap{一百一十一}{鴛鴦女殉主登太虛 狗彘奴欺天招夥盜}



\begin{parag}
    話說鳳姐聽了小丫頭的話,又氣又急又傷心,不覺吐了一口血,便昏暈過去,坐在地下。平兒急來靠着,忙叫了人來攙扶着,慢慢的送到自己房中,將鳳姐輕輕的安放在炕上,立刻叫小紅斟上一杯開水送到鳳姐脣邊。鳳姐呷了一口,昏迷仍睡。秋桐過來略瞧了一瞧,卻便走開,平兒也不叫他。只見豐兒在旁站着,平兒叫他快快的去回明白了二奶奶吐血發暈不能照應的話,告訴了邢王二夫人。邢夫人打諒鳳姐推病藏躲,因這時女親在內不少,也不好說別的,心裏卻不全信,只說:“叫他歇着去罷。”衆人也並無言語。只說這晚人客來往不絕,幸得幾個內親照應。家下人等見鳳姐不在,也有偷閒歇力的,亂亂吵吵,已鬧的七顛八倒,不成事體了。到二更多天遠客去後,便預備辭靈。孝幕內的女眷大家都哭了一陣。只見鴛鴦已哭的昏暈過去了,大家扶住捶鬧了一陣才醒過來,便說“老太太疼我一場我跟了去”的話。衆人都打諒人到悲哭俱有這些言語,也不理會。到了辭靈之時,上上下下也有百十餘人,只鴛鴦不在。衆人忙亂之時,誰去撿點。到了琥珀等一干的人哭奠之時,卻不見鴛鴦,想來是他哭乏了,暫在別處歇着,也不言語。辭靈以後,外頭賈政叫了賈璉問明送殯的事,便商量着派人看家。賈璉回說:“上人裏頭派了芸兒在家照應,不必送殯,下人裏頭派了林之孝的一家子照應拆棚等事。但不知裏頭派誰看家?”賈政道:“聽見你母親說是你媳婦病了不能去,就叫他在家的。你珍大嫂子又說你媳婦病得利害,還叫四丫頭陪着,帶領了幾個丫頭婆子照看上屋裏纔好。”賈璉聽了,心想:“珍大嫂子與四丫頭兩個不合,所以攛掇着不叫他去,若是上頭就是他照應,也是不中用的。我們那一個又病着,也難照應。”想了一回,回賈政道:“老爺且歇歇兒,等進去商量定了再回。”賈政點了點頭,賈璉便進去了。
\end{parag}


\begin{parag}
    誰知此時鴛鴦哭了一場,想到“自己跟着老太太一輩子,身子也沒有着落。如今大老爺雖不在家,大太太的這樣行爲我也瞧不上。老爺是不管事的人,以後便亂世爲王起來了,我們這些人不是要叫他們掇弄了麼。誰收在屋子裏,誰配小子,我是受不得這樣折磨的,倒不如死了乾淨。但是一時怎麼樣的個死法呢?”一面想,一面走回老太太的套間屋內。剛跨進門,只見燈光慘淡,隱隱有個女人拿着汗巾子好似要上吊的樣子。鴛鴦也不驚怕,心裏想道:“這一個是誰?和我的心事一樣,倒比我走在頭裏了。”便問道:“你是誰?咱們兩個人是一樣的心,要死一塊兒死。”那個人也不答言。鴛鴦走到跟前一看,並不是這屋子的丫頭,仔細一看,覺得冷氣侵人時就不見了。鴛鴦呆了一呆,退出在炕沿上坐下,細細一想道:“哦,是了,這是東府裏的小蓉大奶奶啊!他早死了的了,怎麼到這裏來?必是來叫我來了。他怎麼又上吊呢?”想了一想道:“是了,必是教給我死的法兒。”鴛鴦這麼一想,邪侵入骨,便站起來,一面哭,一面開了妝匣,取出那年絞的一綹頭髮,揣在懷裏,就在身上解下一條汗巾,按着秦氏方纔比的地方拴上。自己又哭了一回,聽見外頭人客散去,恐有人進來,急忙關上屋門,然後端了一個腳凳自己站上,把汗巾拴上扣兒套在咽喉,便把腳凳蹬開。可憐咽喉氣絕,香魂出竅,正無投奔,只見秦氏隱隱在前,鴛鴦的魂魄疾忙趕上說道:“蓉大奶奶,你等等我。”那個人道:“我並不是什麼蓉大奶奶,乃警幻之妹可卿是也。”鴛鴦道:“你明明是蓉大奶奶,怎麼說不是呢?”那人道:“這也有個緣故,待我告訴你,你自然明白了。我在警幻宮中原是個鍾情的首坐,管的是風情月債,降臨塵世,自當爲第一情人,引這些癡情怨女早早歸入情司,所以該當懸粱自盡的。因我看破凡情,超出情海,歸入情天,所以太虛幻境癡情一司竟自無人掌管。今警幻仙子已經將你補入,替我掌管此司,所以命我來引你前去的。”鴛鴦的魂道:“我是個最無情的,怎麼算我是個有情的人呢?”那人道:“你還不知道呢。世人都把那淫慾之事當作‘情’字,所以作出傷風敗化的事來,還自謂風月多情,無關緊要。不知‘情’之一字,喜怒哀樂未發之時便是個性,喜怒哀樂已發便是情了。至於你我這個情,正是未發之情,就如那花的含苞一樣,欲待發泄出來,這情就不爲真情了。”鴛鴦的魂聽了點頭會意,便跟了秦氏可卿而去。
\end{parag}


\begin{parag}
    這裏琥珀辭了靈,聽邢王二夫人分派看家的人,想着去問鴛鴦明日怎樣坐車的,在賈母的外間屋裏找了一遍不見,便找到套間裏頭。剛到門口,見門兒掩着,從門縫裏望裏看時,只見燈光半明不滅的,影影綽綽,心裏害怕,又不聽見屋裏有什麼動靜,便走回來說道:“這蹄子跑到那裏去了?”劈頭見了珍珠,說:“你見鴛鴦姐姐來着沒有?”珍珠道:“我也找他,太太們等他說話呢。必在套間裏睡着了罷。”琥珀道:“我瞧了,屋裏沒有。那燈也沒人夾蠟花兒,漆黑怪怕的,我沒進去。如今咱們一塊兒進去瞧,看有沒有。”琥珀等進去正夾蠟花,珍珠說:“誰把腳凳撂在這裏,幾乎絆我一跤。”說着往上一瞧,唬的噯喲一聲,身子往後一仰,咕咚的栽在琥珀身上。琥珀也看見了,便大嚷起來,只是兩隻腳挪不動。
\end{parag}


\begin{parag}
    外頭的人也都聽見了,跑進來一瞧,大家嚷着報與邢王二夫人知道。王夫人寶釵等聽了,都哭着去瞧。邢夫人道:“我不料鴛鴦倒有這樣志氣,快叫人去告訴老爺。”只有寶玉聽見此信,便唬的雙眼直豎。襲人等慌忙扶着,說道:“你要哭就哭,別憋着氣。”寶玉死命的才哭出來了,心想“鴛鴦這樣一個人偏又這樣死法,”又想“實在天地間的靈氣獨鍾在這些女子身上了。他算得了死所,我們究竟是一件濁物,還是老太太的兒孫,誰能趕得上他。”復又喜歡起來。那時寶釵聽見寶玉大哭,也出來了,及到跟前,見他又笑。襲人等忙說:“不好了,又要瘋了。”寶釵道:“不妨事,他有他的意思。”寶玉聽了,更喜歡寶釵的話,“倒是他還知道我的心,別人那裏知道。”正在胡思亂想,賈政等進來,着實的嗟嘆着,說道:“好孩子,不枉老太太疼他一場!”即命賈璉出去吩咐人連夜買棺盛殮,“明日便跟着老太太的殯送出,也停在老太太棺後,全了他的心志。”賈璉答應出去。這裏命人將鴛鴦放下,停放裏間屋內。平兒也知道了,過來同襲人鶯兒等一干人都哭的哀哀欲絕。內中紫鵑也想起自己終身一無着落,“恨不跟了林姑娘去,又全了主僕的恩義,又得了死所。如今空懸在寶玉屋內,雖說寶玉仍是柔情蜜意,究竟算不得什麼?”於是更哭得哀切。
\end{parag}


\begin{parag}
    王夫人即傳了鴛鴦的嫂子進來,叫他看着入殮。逐與邢夫人商量了,在老太太項內賞了他嫂子一百兩銀子,還說等閒了將鴛鴦所有的東西俱賞他們。他嫂子磕了頭出去,反喜歡說:“真真的我們姑娘是個有志氣的,有造化的,又得了好名聲,又得了好發送。”旁邊一個婆子說道:“罷呀嫂子,這會子你把一個活姑娘賣了一百銀子便這麼喜歡了,那時候兒給了大老爺,你還不知得多少銀錢呢,你該更得意了。”一句話戳了他嫂子的心,便紅了臉走開了。剛走到二門上,見林之孝帶了人抬進棺材來了,他只得也跟進去幫着盛殮,假意哭嚎了幾聲。賈政因他爲賈母而死,要了香來上了三炷,作了一個揖,說:“他是殉葬的人,不可作丫頭論。你們小一輩都該行個禮。”寶玉聽了,喜不自勝,走上來恭恭敬敬磕了幾個頭。賈璉想他素日的好處,也要上來行禮,被邢夫人說道:“有了一個爺們便罷了,不要折受他不得超生。”賈璉就不便過來了。寶釵聽了,心中好不自在,便說道:“我原不該給他行禮,但只老太太去世,咱們都有未了之事,不敢胡爲,他肯替咱們盡孝,咱們也該託託他好好的替咱們伏侍老太太西去,也少盡一點子心哪。”說着扶了鶯兒走到靈前,一面奠酒,那眼淚早撲簌簌流下來了,奠畢拜了幾拜,狠狠的哭了他一場。衆人也有說寶玉的兩口子都是傻子,也有說他兩個心腸兒好的,也有說他知禮的。賈政反倒合了意。一面商量定了看家的仍是鳳姐惜春,餘者都遣去伴靈。一夜誰敢安眠,一到五更,聽見外面齊人。到了辰初發引,賈政居長,衰麻哭泣,極盡孝子之禮。靈柩出了門,便有各家的路祭,一路上的風光不必細述。走了半日,來至鐵檻寺安靈,所有孝男等俱應在廟伴宿,不題。
\end{parag}


\begin{parag}
    且說家中林之孝帶領拆了棚,將門窗上好,打掃淨了院子,派了巡更的人到晚打更上夜。只是榮府規例,一二更,三門掩上,男人便進不去了,裏頭只有女人們查夜。鳳姐雖隔了一夜漸漸的神氣清爽了些,只是那裏動得。只有平兒同着惜春各處走了一走,咐吩了上夜的人,也便各自歸房。卻說周瑞的乾兒子何三,去年賈珍管事之時,因他和鮑二打架,被賈珍打了一頓,攆在外頭,終日在賭場過日。近知賈母死了,必有些事情領辦,豈知探了幾天的信,一些也沒有想頭,便噯聲嘆氣的回到賭場中,悶悶的坐下。那些人便說道:“老三,你怎麼樣?不下來撈本了麼?”何三道:“倒想要撈一撈呢,就只沒有錢麼。”那些人道:“你到你們周大太爺那裏去了幾日,府裏的錢你也不知弄了多少來,又來和我們裝窮兒了。”何三道:“你們還說呢,他們的金銀不知有幾百萬,只藏着不用。明兒留着不是火燒了就是賊偷了,他們才死心呢。”那些人道:“你又撒謊,他家抄了家,還有多少金銀?”何三道:“你們還不知道呢,抄去的是撂不了的。如今老太太死還留了好些金銀,他們一個也不使,都在老太太屋裏擱着,等送了殯回來才分呢。”內中有一個人聽在心裏,擲了幾骰,便說:“我輸了幾個錢,也不翻本兒了,睡去了。”說着,便走出來拉了何三道:“老三,我和你說句話。”何三跟他出來。那人道:“你這樣一個伶俐人,這樣窮,爲你不服這口氣。”何三道:“我命裏窮,可有什麼法兒呢。”那人道:“你才說榮府的銀子這麼多,爲什麼不去拿些使喚使喚?”何三道:“我的哥哥,他家的金銀雖多,你我去白要一二錢他們給咱們嗎!”那人笑道:“他不給咱們,咱們就不會拿嗎!”何三聽了這話裏有話,便問道:“依你說怎麼樣拿呢?”那人道:“我說你沒有本事,若是我,早拿了來了。”何三道:“你有什麼本事?”那人便輕輕的說道:“你若要發財,你就引個頭兒。我有好些朋友都是通天的本事,不要說他們送殯去了,家裏剩下幾個女人,就讓有多少男人也不怕。只怕你沒這麼大膽子罷咧。”何三道:“什麼敢不敢!你打諒我怕那個幹老子麼,我是瞧着乾媽的情兒上頭才認他作幹老子罷咧,他又算了人了!你剛纔的話,就只怕弄不來倒招了饑荒。他們那個衙門不熟?別說拿不來,倘或拿了來也要鬧出來的。”那人道:“這麼說你的運氣來了。我的朋友還有海邊上的呢,現今都在這裏看個風頭,等個門路。若到了手,你我在這裏也無益,不如大家下海去受用不好麼?你若撂不下你乾媽,咱們索性把你幹媽也帶了去,大家夥兒樂一樂好不好?”何三道:“老大,你別是醉了罷,這些話混說的什麼。”說着,拉了那人走到一個僻靜地方,兩個人商量了一回,各人分頭而去。暫且不題。
\end{parag}


\begin{parag}
    且說包勇自被賈政吆喝派去看園,賈母的事出來也忙了,不曾派他差使,他也不理會,總是自做自喫,悶來睡一覺,醒時便在園裏耍刀弄棍,倒也無拘無束。那日賈母一早出殯,他雖知道,因沒有派他差事,他任意閒遊。只見一個女尼帶了一個道婆來到園內腰門那裏扣門,包勇走來說道:“女師父那裏去?”道婆道:“今日聽得老太太的事完了,不見四姑娘送殯,想必是在家看家。想他寂寞,我們師父來瞧他一瞧。”包勇道:“主子都不在家,園門是我看的,請你們回去罷。要來呢,等主子們回來了再來。”婆子道:“你是那裏來的個黑炭頭,也要管起我們的走動來了。”包勇道:“我嫌你們這些人,我不叫你們來,你們有什麼法兒!”婆子生了氣,嚷道:“這都是反了天的事了!連老太太在日還不能攔我們的來往走動呢,你是那裏的這麼個橫強盜,這樣沒法沒天的。我偏要打這裏走!”說着,便把手在門環上狠狠的打了幾下。妙玉已氣的不言語,正要回身便走,不料裏頭看二門的婆子聽見有人拌嘴似的,開門一看,見是妙玉,已經回身走去,明知必是包勇得罪了走了。近日婆子們都知道上頭太太們四姑娘都親近得很,恐他日後說出門上不放他進來,那時如何擔得住,趕忙走來說:“不知師父來,我們開門遲了。我們四姑娘在家裏還正想師父呢,快請回來。看園子的小子是個新來的,他不知咱們的事,回來回了太太,打他一頓攆出去就完了。”妙玉雖是聽見,總不理他。那經得看腰門的婆子趕上再四央求,後來才說出怕自己擔不是,幾乎急的跪下,妙玉無奈,只得隨了那婆子過來。包勇見這般光景,自然不好攔他,氣得瞪眼嘆氣而回。
\end{parag}


\begin{parag}
    這裏妙玉帶了道婆走到惜春那裏,道了惱,敘了些閒話。說起“在家看家,只好熬個幾夜。但是二奶奶病着,一個人又悶又是害怕,能有一個人在這裏我就放心。如今裏頭一個男人也沒有,今兒你既光降,肯伴我一宵,咱們下棋說話兒,可使得麼?”妙玉本自不肯,見惜春可憐,又提起下棋,一時高興應了,打發道婆回去取了他的茶具衣褥,命侍兒送了過來,大家坐談一夜。惜春欣幸異常,便命彩屏去開上年蠲的雨水,預備好茶。那妙玉自有茶具。那道婆去了不多一時,又來了個侍者,帶了妙玉日用之物。惜春親自烹茶。兩人言語投機,說了半天,那時已是初更時候,彩屏放下棋枰,兩人對弈。惜春連輸兩盤,妙玉又讓了四個子兒,惜春方贏了半子。這時已到四更,天空地闊,萬籟無聲。妙玉道:“我到五更須得打坐一回,我自有人伏侍,你自去歇息。”惜春猶是不捨,見妙玉要自己養神,不便扭他。正要歇去,猛聽得東邊上屋內上夜的人一片聲喊起,惜春那裏的老婆子們也接着聲嚷道:“了不得了!有了人了!”唬得惜春彩屏等心膽俱裂,聽見外頭上夜的男人便聲喊起來。妙玉道:“不好了,必是這裏有了賊了。”正說着,這裏不敢開門,便掩了燈光。在窗戶眼內往外一瞧,只是幾個男人站在院內,唬得不敢作聲,回身擺着手輕輕的爬下來說:“了不得,外頭有幾個大漢站着。”說猶未了,又聽得房上響聲不絕,便有外頭上夜的人進來吆喝拿賊。一個人說道:“上屋裏的東西都丟了,並不見人。東邊有人去了,咱們到西邊去。”惜春的老婆子聽見有自己的人,便在外間屋裏說道:“這裏有好些人上了房了。”上夜的都道:“你瞧,這可不是嗎。”大家一齊嚷起來。只聽房上飛下好些瓦來,衆人都不敢上前。正在沒法,只聽園門腰門一聲大響,打進門來,見一個梢長大漢,手執木棍。衆人唬得藏躲不及,聽得那人喊說道:“不要跑了他們一個!你們都跟我來。”這些家人聽了這話,越發唬得骨軟筋酥,連跑也跑不動了。只見這人站在當地只管亂喊,家人中有一個眼尖些的看出來了,你道是誰,正是甄家薦來的包勇。這些家人不覺膽壯起來,便顫巍巍的說道:“有一個走了,有的在房上呢。”包勇便向地下一撲,聳身上房追趕那賊。這些賊人明知賈家無人,先在院內偷看惜春房內,見有個絕色女尼,便頓起淫心,又欺上屋俱是女人,且又畏懼,正要踹進門去,因聽外面有人進來追趕,所以賊衆上房。見人不多,還想抵擋,猛見一人上房趕來,那些賊見是一人,越發不理論了,便用短兵抵住。那經得包勇用力一棍打去,將賊打下房來。那些賊飛奔而逃,從園牆過去,包勇也在房上追捕。豈知園內早藏下了幾個在那裏接贓,已經接過好些,見賊夥跑回,大家舉械保護,見追的只有一人,明欺寡不敵衆,反倒迎上來。包勇一見,生氣道:“這些毛賊!敢來和我鬪鬪!”那夥賊便說:“我們有一個夥計被他們打倒了,不知死活,咱們索性搶了他出來。”這裏包勇聞聲即打,那夥賊便掄起器械,四五個人圍住包勇亂打起來。外頭上夜的人也都仗着膽子,只顧趕了來。衆賊見鬪他不過,只得跑了。包勇還要趕時,被一個箱子一絆,立定看時,心想東西未丟,衆賊遠逃,也不追趕。便叫衆人將燈照着,地下只有幾個空箱,叫人收拾,他便欲跑回上房。因路徑不熟,走到鳳姐那邊,見裏面燈燭輝煌,便問:“這裏有賊沒有?”裏頭的平兒戰兢兢的說道:“這裏也沒開門,只聽上屋叫喊說有賊呢。你到那裏去罷。”包勇正摸不着路頭,遙見上夜的人過來,纔跟着一齊尋到上屋。見是門開戶啓,那些上夜的在那裏啼哭。
\end{parag}


\begin{parag}
    一時賈芸林之孝都進來了,見是失盜。大家着急進內查點,老太太的房門大開,將燈一照,鎖頭擰折,進內一瞧,箱櫃已開,便罵那些上夜女人道:“你們都是死人麼!賊人進來你們不知道的麼!”那些上夜的人啼哭着說道:“我們幾個人輪更上夜,是管二三更的,我們都沒有住腳前後走的。他們是四更五更,我們的下班兒。只聽見他們喊起來,並不見一個人,趕着照看,不知什麼時候把東西早已丟了。求爺們問管四五更的。”林之孝道:“你們個個要死,回來再說。咱們先到各處看去。”上夜的男人領着走到尤氏那邊,門兒關緊,有幾個接音說:“唬死我們了。”林之孝問道:“這裏沒有丟東西?”裏頭的人方開了門道:“這裏沒丟東西。”林之孝帶着人走到惜春院內,只聽得裏面說道:“了不得了!唬死了姑娘了,醒醒兒罷。”林之孝便叫人開門,問是怎樣了。裏頭婆子開門說:“賊在這裏打仗,把姑娘都唬壞了,虧得妙師父和彩屏纔將姑娘救醒。東西是沒失。”林之孝道:“賊人怎麼打仗?”上夜的男人說:“幸虧包大爺上了房把賊打跑了去了,還聽見打倒一個人呢。”包勇道:“在園門那裏呢。”賈芸等走到那邊,果見一人躺在地下死了。細細一瞧,好象周瑞的乾兒子。衆人見了詫異,派一個人看守着,又派兩個人照看前後門,俱仍舊關鎖着。
\end{parag}


\begin{parag}
    林之孝便叫人開了門,報了營官,立刻到來查勘。踏察賊跡是從後夾道上屋的,到了西院房上,見那瓦破碎不堪,一直過了後園去了。衆上夜的齊聲說道:“這不是賊,是強盜。”營官着急道:“並非明火執杖,怎算是盜。”上夜的道:“我們趕賊,他在房上擲瓦,我們不能近前,幸虧我們家的姓包的上房打退。趕到園裏,還有好幾個賊竟與姓包的打仗,打不過姓包的才都跑了。”營官道:“可又來,若是強盜,倒打不過你們的人麼。不用說了,你們快查清了東西,遞了失單,我們報就是了。”
\end{parag}


\begin{parag}
    賈芸等又到上屋,已見鳳姐扶病過來,惜春也來。賈芸請了鳳姐的安,問了惜春的好。大家查看失物,因鴛鴦已死,琥珀等又送靈去了,那些東西都是老太太的,並沒見數,只用封鎖,如今打從那裏查去。衆人都說:“箱櫃東西不少,如今一空,偷的時候不小,那些上夜的人管什麼的!況且打死的賊是周瑞的乾兒子,必是他們通同一氣的。”鳳姐聽了,氣的眼睛直瞪瞪的便說:“把那些上夜的女人都拴起來,交給營裏審問。”衆人叫苦連天,跪地哀求。不知怎生髮放,並失去的物有無着落,下回分解。
\end{parag}