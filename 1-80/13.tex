\chap{一十三}{秦可卿死封龍禁尉 王熙鳳協理寧國府}

\begin{parag}
    \begin{note}蒙、戚:生死窮通何處真?英明難遏是精神。微密久藏偏自露,幻中夢裏語驚人。\end{note}
\end{parag}


\begin{parag}
    \begin{note}甲:賈珍尚奢,豈有不請父命之理?因敬□□□要緊,不問家事,故得恣意放爲。\end{note}
\end{parag}


\begin{parag}
    \begin{note}
        甲:若明指一州名,似若《西遊》之套,故曰至中之地,不待言可知是光天化日仁風德雨之下矣。不雲國名更妙,可知是堯街舜巷衣冠禮義之鄉矣。直與第一回呼應相接。
    \end{note}
\end{parag}


\begin{parag}
    \begin{note}
        今秦可卿託□□□□□□□□□□□□□理寧府,亦□□□□□□□□□□□□□凡□□□□□□□□□□□□□□□□在封龍禁尉寫乃褒中之貶,隱去天香樓一節,是不忍下筆也。
    \end{note}
    \begin{subnote}按:甲本此頁被對角撕去,故缺字甚多。此頁原有三則評語,第二則與本回庚本眉同。今補。\end{subnote}
\end{parag}


\begin{parag}
    \begin{note}
        庚:此回可卿夢阿鳳,蓋作者大有深意存焉。可惜生不逢時,奈何奈何!然必寫出自可卿之意也,則又有他意寓焉。
    \end{note}
\end{parag}


\begin{parag}
    \begin{note}
        榮、寧世家未有不尊家訓者。雖賈珍尚奢,豈明逆父哉?故寫敬老不管,然後恣意,方見筆筆周到。
    \end{note}
\end{parag}


\begin{parag}
    \begin{note}靖:此回可卿夢阿鳳,作者大有深意,惜已爲末世,奈何奈何!賈珍雖奢淫,豈能逆父哉?特因敬老不管,然後恣意,足爲世家之戒。“秦可卿淫喪天香樓”,作者用史筆也。老朽因有魂託鳳姐賈家後事二件,豈是安富尊榮坐享人能想得到者?其事雖未行,其言其意,令人悲切感服,姑赦之,因命芹溪刪去“遺簪”、“更衣”諸文,是以此回只十頁,刪去天香樓一節,少去四五頁也。\end{note}
\end{parag}


\begin{parag}
    話說鳳姐兒自賈璉送黛玉往揚州去後,心中實在無趣,每到晚間,不過和平兒說笑一回,就胡亂\begin{note}甲側:“胡亂”二字奇。\end{note}睡了。
\end{parag}


\begin{parag}
    這日夜間,正和平兒燈下擁爐倦繡,早命濃薰繡被,二人睡下,屈指算行程該到何處,\begin{note}甲側:所謂“計程今日到梁州”是也。\end{note}不知不覺已交三鼓。平兒已睡熟了。鳳姐方覺星眼微蒙,恍惚只見秦氏從外走來,含笑說道:“嬸嬸好睡!我今日回去,你也不送我一程。因娘兒們素日相好,我捨不得嬸子,故來別你一別。還有一件心願未了,非告訴嬸子,別人未必中用。”\begin{note}甲側:一語貶盡賈家一族空頂冠束帶者。\end{note}
\end{parag}


\begin{parag}
    鳳姐聽了,恍惚問道:“有何心事?你只管託我就是了。”秦氏道:“嬸嬸,你是個脂粉隊裏的英雄,\begin{note}甲側:稱得起。\end{note}連那些束帶頂冠的男子也不能過你,你如何連兩句俗語也不曉得?常言‘月滿則虧,水滿則溢’;又道是‘登高必跌重’。如今我們家赫赫揚揚,已將百載,一日倘或\begin{note}甲側:“倘或”二字酷肖婦女口氣。\end{note}樂極悲生,若應了那句‘樹倒猢猻散’的俗語,\begin{note}甲眉:“樹倒猢猻散”之語,今猶在耳,屈指三十五年矣。哀哉傷哉,寧不痛殺!\end{note}豈不虛稱了一世詩書舊族了!”鳳姐聽了此話,心胸大快,十分敬畏,忙問道:“這話慮的極是,但有何法可以永保無虞?”\begin{note}甲側:非阿鳳不明,該古今名利場中患失之同意也。\end{note}秦氏冷笑道:“嬸子好癡也。否極泰來,榮辱自古週而復始,豈人力能可常保的。但如今能於榮時籌劃下將來衰時的世業,亦可謂常保永全了。即如今日諸事都妥,只有兩件未妥,若把此事如此一行,則後日可保永全了。”
\end{parag}


\begin{parag}
    鳳姐便問何事。秦氏道:“目今祖塋雖四時祭祀,只是無一定的錢糧;第二,家塾雖立,無一定的供給。依我想來,如今盛時固不缺祭祀供給,但將來敗落之時,此二項有何出處?莫若依我定見,趁今日富貴,將祖塋附近多置田莊房舍地畝,以備祭祀供給之費皆出自此處,將家塾亦設於此。合同族中長幼,大家定了則例,日後按房掌管這一年的地畝、錢糧、祭祀、供給之事。如此周流,又無競爭,亦不有典賣諸弊。便是有了罪,凡物可入官,這祭祀產業連官也不入的。便敗落下來,子孫回家讀書務農,也有個退步,\begin{note}蒙雙夾:幻情文字中忽入此等警句,提醒多少熱心人。\end{note}祭祀又可永繼。若目今以爲榮華不絕,不思後日,終非長策。眼見不日又有一件非常喜事,真是烈火烹油、鮮花著錦之盛。要知道,也不過是瞬息的繁華,一時的歡樂,萬不可忘了那‘盛筵必散’的俗語。\begin{note}蒙側:“瞬息繁華,一時歡樂”二語,可共天下有志事業功名者同來一哭。但天生人非無所爲,遇機會,成事業,留名於後世者,辦必有奇傳奇遇,方能成不世之功。此亦皆蒼天暗中扶助,雖有波瀾,而無甚害,反覺其錚錚有聲。其不成也,亦由天命。其好人傾險之計,亦非天命不能行。其繁華歡樂,亦自天命。人於其間,知天命而存好生之心,盡已力以周旋其間,不計其功之成否,所謂心安而理盡,又何患乎?一時瞬息,隨緣遇緣,烏乎不可!\end{note}此時若不早爲後慮,臨期只恐後悔無益了。”\begin{note}甲眉:語語見道,字字傷心,讀此一段,幾不知此身爲何物矣。松齋。\end{note}鳳姐忙問:“有何喜事?”秦氏道:“天機不可泄漏。\begin{note}甲側:伏得妙!\end{note}只是我與嬸子好了一場,臨別贈你兩句話,須要記著。”因念道:
\end{parag}


\begin{poem}
    \begin{pl}三春去後諸芳盡,各自須尋各自門。\end{pl}
    \begin{note}甲側:此句令批書人哭死。甲眉:不必看完,見此二句,即欲墮淚。梅溪。\end{note}
\end{poem}


\begin{parag}
    鳳姐還欲問時,只聽二門上傳事雲牌連叩四下,將鳳姐驚醒。人回:“東府蓉大奶奶沒了。”鳳姐聞聽,嚇了一身冷汗,出了一回神,只得忙忙的穿衣,往王夫人處來。
\end{parag}


\begin{parag}
    彼時閤家皆知,無不納罕,都有些疑心。\begin{note}甲眉:九個字寫盡天香樓事,是不寫之寫。[靖本多署名“棠村”。]庚眉:可從此批。靖眉:可從此批。通回將可卿如何死故隱去,是餘大發慈悲也。嘆嘆!壬午季春。 笏叟。\end{note}那長一輩的想他素日孝順;平一輩的,想他平日和睦親密,\begin{note}庚眉:松齋雲:好筆力。此方是文字佳處。\end{note}下一輩的想他素日慈愛,以及家中僕從老小想他素日憐貧惜賤、慈老愛幼\begin{note}庚側:八字乃爲上人之當銘於五衷。\end{note}之恩,莫不悲嚎痛哭者。\begin{note}庚側:老健。\end{note}
\end{parag}


\begin{parag}
    閒言少敘,卻說寶玉因近日林黛玉回去,剩得自己孤悽,也不和人頑耍,\begin{note}甲側:與鳳姐反對。淡淡寫來,方是二人自幼氣味相投,可知後文皆非突然文字。\end{note}每到晚間便索然睡了。如今從夢中聽見說秦氏死了,連忙翻身爬起來,只覺心中似戮了一刀的不忍,哇的一聲,直奔出一口血來。\begin{note}甲側:寶玉早已看定可繼家務事者可卿也,今聞死了,大失所望。急火攻心,焉得不有此血?爲玉一嘆!\end{note}襲人等慌慌忙忙上來搊(校者注:蒙古王府本此處作“摟”)扶,問是怎麼樣,又要回賈母來請大夫。寶玉笑道:“不用忙,不相干,\begin{note}庚側:又淡淡抹去。\end{note}這是急火攻心,\begin{note}甲側:如何自己說出來了?\end{note}血不歸經。”說著便爬起來,要衣服換了,來見賈母,即時要過去。\begin{note}庚眉:如此總是淡描輕寫,全無痕跡,方見得有生以來,天分中自然所賦之性如此,非因色所感也。\end{note}襲人見他如此,心中雖放不下,又不敢攔,只是由他罷了。賈母見他要去,因說:“才嚥氣的人,那裏不乾淨;二則夜裏風大,明早再去不遲。”寶玉那裏肯依。賈母命人備車,多派跟從人役,擁護前來。
\end{parag}


\begin{parag}
    一直到了寧國府前,只見府門洞開,兩邊燈籠照如白晝,亂烘烘人來人往,裏面哭聲搖山振嶽。\begin{note}甲側:寫大族之喪,如此起緒。\end{note}寶玉下了車,忙忙奔至停靈之室,痛哭一番。然後見過尤氏。誰知尤氏正犯了胃疼舊疾,睡在牀上。\begin{note}甲側:妙!非此何以出阿鳳!\end{note}\begin{note}庚側:緊處愈緊,密處愈密。\end{note}\begin{note}庚眉:所謂層巒疊翠之法也。野史中從無此法。即觀者到此,亦爲寫秦氏未必全到,豈料更又寫一尤氏哉!\end{note}然後又出來見賈珍。彼時賈代儒帶領賈敕、賈效、賈敦、賈赦、賈政、賈琮、賈㻞、賈珩、賈珖、賈琛、賈瓊、賈璘、賈薔、賈菖、賈菱、賈芸、賈芹、賈蓁、賈萍、賈藻、賈蘅、賈芬、賈芳、賈蘭、賈菌、賈芝等\begin{note}庚側:將賈族約略一總,觀者方不惑。\end{note}都來了。賈珍哭的淚人一般,\begin{note}甲側:可笑,如喪考妣,此作者刺心筆也。\end{note}正和賈代儒等說道:“合家大小,遠親近友,誰不知我這媳婦比兒子還強十倍。如今伸腿去了,可見這長房內絕滅無人了。”說著又哭起來。衆人忙勸道:“人已辭世,哭也無益,且商議如何料理要緊。”\begin{note}庚側:淡淡一句,勾出賈珍多少文字來。\end{note}賈珍拍手道:“如何料理,不過盡我所有罷了!”\begin{note}蒙雙夾:“盡我所有”,爲媳婦是非禮之談,父母又將何以待之?故前此有思織酒後狂言,及今復見此語,含而不露,吾不能爲賈珍隱諱。\end{note}
\end{parag}


\begin{parag}
    正說著,只見秦業、秦鍾並尤氏的幾個眷屬\begin{note}甲側:伏後文。\end{note}尤氏姊妹也都來了。賈珍便命賈瓊、賈琛、賈璘、賈薔四個人去陪客,一面吩咐去請欽天監陰陽司來擇日,推準停靈七七四十九日,三日後開喪送訃聞。這四十九日,單請一百單八衆禪僧在大廳上拜大悲懺,超度前亡後化諸魂,以免亡者之罪;另設一罈於天香樓上,\begin{note}甲側:刪卻,是未刪之筆。\end{note}\begin{note}靖眉:何必定用“西”字?讀之令人酸鼻!\end{note}\begin{subnote}按:此條所評正文之「天香樓」,靖藏本作「西帆樓」。\end{subnote}是九十九位全真道士,打四十九日解冤洗業醮。然後停靈於會芳園中,靈前另有五十衆高僧、五十衆高道,對壇按七作好事。那賈敬聞得長孫媳婦死了,因自爲早晚就要飛昇,\begin{note}庚側:可笑可嘆。古今之儒,中途多惑老佛。王梅隱雲:“若能再加東坡十年壽,亦能跳出這圈子來。”斯言信矣。\end{note}\begin{note}蒙側:“就要飛昇”的“要”,用得得當。凡“要”者,則身心急切;急切之者,百事無成。正爲後文作引線。\end{note}如何肯又回家染了紅塵,將前功盡棄呢,因此並不在意,只憑賈珍料理。
\end{parag}


\begin{parag}
    賈珍見父親不管,亦發恣意奢華。看板時,幾副杉木板皆不中用。可巧薛蟠來弔問,因見賈珍尋好板,便說道:“我們木店裏有一副板,叫做什麼檣木,\begin{note}甲眉:檣者,舟具也。所謂“人生若泛舟”而已,寧不可嘆!\end{note}出在潢海鐵網山上,\begin{note}甲側:所謂迷津易墮,塵網難逃也。\end{note}作了棺材,萬年不壞。這還是當年先父帶來,原系義忠親王老千歲要的,因他壞了事,\begin{note}蒙側:“壞了事”等字毒極,寫盡勢利場中故套。\end{note}就不曾拿去。現今還封在店裏,也沒人出價敢買。你若要,就抬來罷了。”賈珍聽了,喜之不盡,即命人抬來。大家看時,只見幫底皆厚八寸,紋若檳榔,味若檀麝,以手扣之,玎璫如金玉。大家都奇異稱賞。賈珍笑問:“價值幾何?”薛蟠笑道:“拿一千兩銀子來,只怕也沒處買去。什麼價不價,賞他們幾兩工錢就是了。”\begin{note}甲側:的是阿呆兄口氣。\end{note}賈珍聽說,忙謝不盡,即命解鋸糊漆。賈政因勸道:“此物恐非常人可享者,\begin{note}甲側:政老有深意存焉。\end{note}殮以上等杉木也就是了。”\begin{note}甲側:夾寫賈政。\end{note}\begin{note}甲眉:寫個個皆到,全無安逸之筆,深得《金瓶》壺奧!\end{note}此時賈珍恨不能代秦氏之死,這話如何肯聽。\begin{note}蒙側:“代秦氏死”等句,總是填實前文。\end{note}
\end{parag}


\begin{parag}
    因忽又聽得秦氏之丫鬟名喚瑞珠者,見秦氏死了,他也觸柱而亡。\begin{note}甲側:補天香樓未刪之文。\end{note}\begin{note}靖側:是亦未刪之筆。\end{note}此事可罕,合族中人也都稱讚。賈珍遂以孫女之禮殮殯,一併停靈於會芳園中之登仙閣。小丫鬟名寶珠者,因見秦氏身無所出,乃甘心願爲義女,誓任摔喪駕靈之任。賈珍喜之不盡,即時傳下,從此皆呼寶珠爲小姐。那寶珠按未嫁女之喪,在靈前哀哀欲絕。\begin{note}甲側:非恩惠愛人,那能如是?惜哉可卿,惜哉可卿!\end{note}於是,合族人丁並家下諸人,都各遵舊制行事,自不敢紊亂。\begin{note}甲側:兩句寫盡大家。\end{note}
\end{parag}


\begin{parag}
    賈珍因想著賈蓉不過是個黌門監,\begin{note}庚側:又起波瀾,卻不突然。\end{note}靈幡經榜上寫時不好看,便是執事也不多,因此心下甚不自在。\begin{note}甲側:善起波瀾。\end{note}可巧這日正是首七第四日,早有大明宮掌宮內相戴權,\begin{note}甲側:妙!大權也。\end{note}先備了祭禮遣人來,次後坐了大轎,打傘嗚鑼,親來上祭。賈珍忙接著,讓至逗蜂軒\begin{note}甲側:軒名可思。\end{note}獻茶。賈珍心中打算定了主意,因而趁便就說要與賈蓉捐個前程的話。戴權會意,因笑道:“想是爲喪禮上風光些?”\begin{note}甲側:難得內相機括之快如此。\end{note}賈珍忙笑道:“老內相所見不差。”戴權道:“事倒湊巧,正有個美缺。如今三百員龍禁尉短了兩員,昨日襄陽侯的兄弟老三來求我,現拿了一千五百兩銀子,送到我家裏。你知道,咱們都是老相與,不拘怎麼樣,看著他爺爺的分上,胡亂應了。\begin{note}甲側:忙中寫閒。\end{note}還剩了一個缺,誰知永興節度使馮胖子來求,要與他孩子捐,我就沒工夫應他。既是咱們的孩子\begin{note}甲側:奇談,畫盡閹官口吻。\end{note}要捐,快寫個履歷來。”賈珍聽說,忙吩咐:“快命書房裏人恭敬寫了大爺的履歷來。”小廝不敢怠慢,去了一刻,便拿了一張紅紙來與賈珍。賈珍看了,忙送與戴權。看時,上面寫道:
\end{parag}


\begin{qute2sp}
    江南江寧府江寧縣監生賈蓉,年二十歲。曾祖,原任京營節度使世襲一等神威將軍賈代化;祖,乙卯科進士賈敬;父,世襲三品爵威烈將軍賈珍。
\end{qute2sp}


\begin{parag}
    戴權看了,回手便遞與一個貼身的小廝收了,說道:“回來送與戶部堂官老趙,說我拜上他,起一張五品龍禁尉的票,再給個執照,就把那履歷填上,明兒我來兌銀子送去。”小廝答應了,戴權也就告辭了。賈珍十分款留不住,只得送出府門。臨上轎,賈珍因問:“銀子還是我到部兌,還是一併送入老內相府中?”戴權道:“若到部裏,你又喫虧了。不如平準一千二百銀子送到我家裏就完了。”賈珍感謝不盡,只說:“待服滿後,親帶小犬到府叩謝。”於是作別。
\end{parag}


\begin{parag}
    接著,便又聽喝道之聲,原來是忠靖侯史鼎的夫人來了。\begin{note}甲側:史小姐湘雲消息也。\end{note}王夫人、邢夫人、鳳姐等剛迎入上房,又見錦鄉侯、川寧侯、壽山伯三家祭禮擺在靈前。少時,三人下轎,賈政等忙接上大廳。如此親朋你來我去,也不能勝數。只這四十九日,\begin{note}庚側:就簡去繁。\end{note}寧國府街上一條白漫漫人來人往,\begin{note}甲側:是有服親朋並家下人丁之盛。\end{note}花簇簇官去官來。\begin{note}甲側:是來往祭弔之盛。\end{note}
\end{parag}


\begin{parag}
    賈珍命賈蓉次日換了吉服,領憑回來。靈前供用執事等物,俱按五品職例。靈牌疏上皆寫“天朝誥授賈門秦氏恭人之靈位”。會芳園臨街大門洞開,旋在兩邊起了鼓樂廳,兩班青衣按時奏樂,一對對執事擺的刀斬斧齊。更有兩面硃紅銷金大字牌對豎在門外,上面大書:“防護內廷紫禁道御前侍衛龍禁尉”。對面高起著宣壇,僧道對壇榜文,榜上大書:“世襲寧國公冢孫婦、防護內廷御前侍衛龍禁尉賈門秦氏恭人之喪。\begin{note}庚眉:賈珍是亂費,可卿卻實如此。\end{note}四大部州至中之地,奉天承運太平之國,\begin{note}庚眉:奇文。若明指一州名,似若《西遊》之套,故曰至中之地,不待言可知是光天化日仁風德雨之下矣。不雲國名更妙,可知是堯街舜巷衣冠禮義之鄉矣。直與第一回呼應相接。\end{note}總理虛無寂靜教門僧錄司正堂萬虛、總理元始三一教門道錄司正堂葉生等,敬謹修齋,朝天叩佛”,以及“恭請諸伽藍、揭諦、功曹等神,聖恩普錫,神威遠鎮,四十九日消災洗業平安水陸道場”等語,亦不消繁記。
\end{parag}


\begin{parag}
    只是賈珍雖然此時心意滿足,\begin{note}蒙側:可笑。\end{note}但裏面尤氏又犯了舊疾,不能料理事務,惟恐各誥命來往,虧了禮數,怕人笑話,因此心中不自在。當下正憂慮時,因寶玉\begin{note}甲側:餘正思如何高擱起玉兄了。\end{note}在側問道:“事事都算安貼了,大哥哥還愁什麼?”賈珍見問,便將裏面無人的話說了出來。寶玉聽說笑道:“這有何難,我薦一個人\begin{note}甲側:薦鳳姐須得寶玉,俱龍華會上人也。\end{note}與你權理這一個月的事,管必妥當。”賈珍忙問:“是誰?”寶玉見座間還有許多親友,不便明言,走至賈珍耳邊說了兩句。賈珍聽了喜不自禁,連忙起身道:“果然安貼,如今就去。”說著拉了寶玉,辭了衆人,便往上房裏來。
\end{parag}


\begin{parag}
    可巧這日非正經日期,親友來的少,裏面不過幾位近親堂客,邢夫人、王夫人、鳳姐併合族中的內眷陪坐。聞人報:“大爺進來了。”唬的衆婆娘唿的一聲,往後藏之不迭,\begin{note}甲側:數日行止可知。作者自是筆筆不空,批者亦字字留神之至矣。\end{note}獨鳳姐款款站了起來。\begin{note}庚側:又寫鳳姐。\end{note}賈珍此時也有些病症在身,二則過於悲痛了,因拄拐踱了進來。邢夫人等因說道:“你身上不好,又連日事多,該歇歇纔是,又進來做什麼?”賈珍一面扶拐,\begin{note}庚側:一絲不亂。\end{note}\begin{note}靖眉:刺心之筆。\end{note}扎掙著要蹲身跪下請安道乏。邢夫人等忙叫寶玉攙住,命人挪椅子來與他坐。賈珍斷不肯坐,因勉強陪笑道:“侄兒進來有一件事要求二位嬸子並大妹。”邢夫人等忙問:“什麼事?”賈珍忙道:“嬸子自然知道,如今孫子媳婦沒了,侄兒媳婦偏又病倒,我看裏頭著實不成個體統。怎麼屈尊大妹妹一個月,\begin{note}庚側:不見突然。\end{note}在這裏料理料理,我就放心了。”\begin{note}庚側:阿鳳此刻心癢矣。\end{note}邢夫人笑道:“原來爲這個。你大妹妹現在你二嬸子家,只和你二嬸子說就是了。”王夫人忙道:“他一個小孩子\begin{note}庚側:三字愈令人可愛可憐。\end{note}家何曾經過這樣事,倘或料理不清,反叫人笑話,倒是再煩別人好。”賈珍笑道:“嬸子的意思侄兒猜著了,是怕大妹妹勞苦了。若說料理不開,我包管必料理的開,便是錯一點兒,別人看著還是不錯的。從小兒大妹妹頑笑著就有殺伐決斷,\begin{note}庚側:阿鳳身份。\end{note}如今出了閣,又在那府裏辦事,越發歷練老成了。我想了這幾日,除了大妹妹再無人了。嬸子不看侄兒、侄兒媳婦的分上,只看死了的分上罷!”說著滾下淚來。\begin{note}庚側:有筆力。\end{note}
\end{parag}


\begin{parag}
    王夫人心中怕的是鳳姐未經過喪事,怕他料理不清,惹人恥笑。今見賈珍苦苦的說到這步田地,心中已活了幾分,卻又眼看著鳳姐出神。那鳳姐素日最喜攬事辦,好賣弄才幹,雖然當家妥當,也因未辦過婚喪大事,恐人還不伏,巴不得遇見這事。今見賈珍如此一來,他心中早已歡喜。先見王夫人不允,後見賈珍說的情真,王夫人有活動之意,便向王夫人道:“大哥哥說的這麼懇切,太太就依了罷。”王夫人悄悄的道:“你可能麼?”鳳姐道:“有什麼不能的。外面的大事已經大哥哥\begin{note}庚旁批:王夫人是悄言,鳳姐是響應,故稱“大哥哥”。\end{note}料理清了,\begin{note}庚側:已得三昧矣。\end{note}不過是裏頭照管照管,便是我有不知道的,問問太太就是了。”\begin{note}甲側:胸中成見已有之語。\end{note}王夫人見說的有理,便不作聲。賈珍見鳳姐允了,又陪笑道:“也管不得許多了,橫豎要求大妹妹辛苦辛苦。我這裏先與妹妹行禮,等事完了,我再到那府裏去謝。”說著,就作揖下去,鳳姐兒還禮不迭。
\end{parag}


\begin{parag}
    賈珍便忙向袖中取了寧國府對牌出來,命寶玉送與鳳姐,又說:“妹妹愛怎樣就怎樣,要什麼只管拿這個取去,也不必問我。只求別存心替我省錢,只要好看爲上;二則也要同那府裏一樣待人才好,不要存心怕人抱怨。只這兩件外,我再沒不放心的了。”鳳姐不敢就接牌,\begin{note}蒙雙夾:凡有本領者斷不越禮。接牌小事而必待命於王夫人也,誠家道之規範,亦天下之規範也。看是書者不可草草從事。\end{note}只看著王夫人。王夫人道:“你哥哥既這麼說,你就照看照看罷了。只是別自作主意,有了事,打發人問你哥哥、嫂子要緊。”寶玉早向賈珍手裏接過對牌來,強遞與鳳姐了。又問:“妹妹住在這裏,還是天天來呢?若是天天來,越發辛苦了。不如我這裏趕著收拾出一個院落來,妹妹住過這幾日倒安穩。”鳳姐笑道:“不用。\begin{note}甲側:二字句,有神。\end{note}那邊也離不得我,倒是天天來的好。”賈珍聽說,只得罷了。然後又說了一回閒話,方纔出去。
\end{parag}


\begin{parag}
    一時女眷散後,王夫人因問鳳姐:“你今兒怎麼樣?”鳳姐兒道:“太太只管請回去,我須得先理出一個頭緒來,纔回去得呢。”王夫人聽說,便先同邢夫人等回去,不在話下。
\end{parag}


\begin{parag}
    這裏鳳姐兒來至三間一所抱廈內坐了,因想:頭一件是人口混雜,遺失東西;第二件,事無專責,臨期推委;第三件,需用過費,濫支冒領;第四件,任無大小,苦樂不均;第五件,家人豪縱,有臉者不服鈐束,無臉者不能上進。\begin{note}甲眉:舊族後輩受此五病者頗多,餘家更甚。三十年前事見書於三十年後,令餘悲痛血淚盈面。\end{note}\begin{note}庚眉:讀五件事未完,餘不禁失聲大哭,三十年前作書人在何處耶?\end{note}此五件實是寧國府中風俗。不知鳳姐如何處治,且聽下回分解。\begin{note}甲眉:此回只十頁,因刪去天香樓一節,少去四五頁也。\end{note}
\end{parag}


\begin{parag}
    正是:
\end{parag}


\begin{poem}
    \begin{pl} 金紫萬千誰治國,裙釵一二可齊家。\end{pl}
    \begin{note}蒙:五件事若能如法整理得當,豈獨家庭,國家天下治之不難。\end{note}
\end{poem}


\begin{parag}
    \begin{note}甲:“秦可卿淫喪天香樓”,作者用史筆也。老朽因有魂託鳳姐賈家後事二件,的是安富尊榮坐享人不能想得到處。其事雖未行,其言其意則令人悲切感服,姑赦之,因命芹溪刪去。\end{note}
\end{parag}


\begin{parag}
    \begin{note}庚:通回將可卿如何死故隱去,是大發慈悲心也,嘆嘆!壬午春。\end{note}
\end{parag}


\begin{parag}
    \begin{note}蒙回末總評:借可卿之死,又寫出情之變態,上下大小,男女老少,無非情感而生情。且又藉鳳姐之夢,更化就幻空中一片貼切之情,所謂寂然不動,感而遂通。所感之象,所動之萌,深淺誠僞,隨種必報,所謂幻者此也,情者亦此也。何非幻,何非情?情即是幻,幻即是情,明眼者自見。\end{note}
\end{parag}

