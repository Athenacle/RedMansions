\chap{四十}{史太君兩宴大觀園 金鴛鴦三宣牙牌令}


\begin{parag}
    \begin{note}蒙回前總:兩宴不覺已深秋,惜春只如畫春遊。可憐富貴誰能保,只有恩情得到頭。\end{note}
\end{parag}


\begin{parag}
    話說寶玉聽了,忙進來看時,只見琥珀站在屏風跟前說:“快去吧,立等你說話呢。”寶玉來至上房,只見賈母正和王夫人衆姊妹商議給史湘雲還席。寶玉因說道:“我有個主意。既沒有外客,喫的東西也別定了樣數,誰素日愛喫的揀樣兒做幾樣。也不要按桌席,每人跟前擺一張高几,各人愛喫的東西一兩樣,再一個什錦攢心盒子,自斟壺,豈不別緻。”賈母聽了,說“很是”,忙命傳與廚房:“明日就揀我們愛喫的東西作了,按著人數,再裝了盒子來。早飯也擺在園裏喫。”商議之間早又掌燈,一夕無話。
\end{parag}


\begin{parag}
    次日清早起來,可喜這日天氣清朗。李紈侵晨先起,看著老婆子丫頭們掃那些落葉,\begin{note}蒙雙夾:八月盡的光景。\end{note}並擦抹桌椅,預備茶酒器皿。只見豐兒帶了劉姥姥板兒進來,說“大奶奶倒忙的緊。”李紈笑道:“我說你昨兒去不成,只忙著要去。”劉姥姥笑道:“老太太留下我,叫我也熱鬧一天去。”豐兒拿了幾把大小鑰匙,說道:“我們奶奶說了,外頭的高几恐不夠使,不如開了樓把那收著的拿下來使一天罷。奶奶原該親自來的,因和太太說話呢,請大奶奶開了,帶著人搬罷。”李氏便令素雲接了鑰匙,又令婆子出去把二門上的小廝叫幾個來。李氏站在大觀樓下往上看,令人上去開了綴錦閣,一張一張往下抬。小廝老婆子丫頭齊動手,抬了二十多張下來。李紈道:“好生著,別慌慌張張鬼趕來似的,仔細碰了牙子。”又回頭向劉姥姥笑道:“姥姥,你也上去瞧瞧。”劉姥姥聽說,巴不得一聲兒,便拉了板兒登梯上去進裏面,只見烏壓壓的堆著些圍屏、桌椅、大小花燈之類,雖不大認得,只見五彩炫耀,各有奇妙。唸了幾聲佛,便下來了。然後鎖上門,一齊纔下來。李紈道:“恐怕老太太高興,越性把舡上划子、篙槳、遮陽幔子都搬了下來預備著。”衆人答應,復又開了,色色的搬了下來。令小廝傳駕娘們到舡塢裏撐出兩隻船來。
\end{parag}


\begin{parag}
    正亂著安排,只見賈母已帶了一羣人進來了。李紈忙迎上去,笑道:“老太太高興,倒進來了。我只當還沒梳頭呢,才擷了菊花要送去。”一面說,一面碧月早捧過一個大荷葉式的翡翠盤子來,裏面盛著各色的折枝菊花。賈母便揀了一朵大紅的簪於鬢上。因回頭看見了劉姥姥,忙笑道:“過來帶花兒。”一語未完,鳳姐便拉過劉姥姥,笑道:“讓我打扮你。”說著,將一盤子花橫三豎四的插了一頭。賈母和衆人笑的了不得。劉姥姥笑道:“我這頭也不知修了什麼福,今兒這樣體面起來。”衆人笑道:“你還不拔下來摔到他臉上呢,把你打扮的成了個老妖精了。”劉姥姥笑道:“我雖老了,年輕時也風流,愛個花兒粉兒的,今兒老風流纔好。”
\end{parag}


\begin{parag}
    說笑之間,已來至沁芳亭子上。丫鬟們抱了一個大錦褥子來,鋪在欄杆榻板上。賈母倚柱坐下,命劉姥姥也坐在旁邊,因問他:“這園子好不好?”劉姥姥唸佛說道:“我們鄉下人到了年下,都上城來買畫兒貼。時常閒了,大家都說,怎麼得也到畫兒上去逛逛。想著那個畫兒也不過是假的,那裏有這個真地方呢。誰知我今兒進這園裏一瞧,竟比那畫兒還強十倍。怎麼得有人也照著這個園子畫一張,我帶了家去,給他們見見,死了也得好處。”賈母聽說,便指著惜春笑道:“你瞧我這個小孫女兒,他就會畫。等明兒叫他畫一張如何?”劉姥姥聽了,喜的忙跑過來,拉著惜春說道:“我的姑娘,你這麼大年紀兒,又這麼個好模樣,還有這個能幹,別是神仙託生的罷。”
\end{parag}


\begin{parag}
    賈母少歇一回,自然領著劉姥姥都見識見識。先到了瀟湘館。一進門,只見兩邊竹夾路,土地下蒼苔佈滿,中間羊腸一條石子漫的路。劉姥姥讓出路來賈母衆人走,自己卻赾土地。琥珀拉著他說道:“姥姥,你上來走,仔細蒼苔滑了。”劉姥姥道:“不相干的,我們走熟了的,姑娘們只管走罷。可惜你們的那繡鞋,別沾髒了。”他只顧上頭和人說話,不防底下果跴滑了,具一跤跌倒。衆人拍手都哈哈的笑起來。賈母笑罵道:“小蹄子們,還不攙起來,只站著笑。”說話時,劉姥姥已爬了起來,自己也笑了,說道:“才說嘴就打了嘴。”賈母問他:“可扭了腰了不曾?叫丫頭們捶一捶。”劉姥姥道:“那裏說的我這麼嬌嫩了。那一天不跌兩下子,都要捶起來,還了得呢。”紫鵑早打起湘簾,賈母等進來坐下。林黛玉親自用小茶盤捧了一蓋碗茶來奉與賈母。王夫人道:“我們不喫茶,姑娘不用倒了。”林黛玉聽說,便命丫頭把自己窗下常坐的一張椅子挪到下首,請王夫人坐了。劉姥姥因見窗下案上設著筆硯,又見書架上磊著滿滿的書,劉姥姥道:“這必定是那位哥兒的書房了。”賈母笑指黛玉道:“這是我這外孫女兒的屋子。”劉姥姥留神打量了黛玉一番,方笑道:“這那象個小姐的繡房,竟比那上等的書房還好。”賈母因問:“寶玉怎麼不見?”衆丫頭們答說:“在池子裏舡上呢。”賈母道:“誰又預備下舡了?”李紈忙回說:“纔開樓拿幾,我恐怕老太太高興,就預備下了。”賈母聽了方欲說話時,有人回說: “姨太太來了。” 賈母等剛站起來,只見薛姨媽早進來了,一面歸坐,笑道:“今兒老太太高興,這早晚就來了。”賈母笑道:“我才說來遲了的要罰他,不想姨太太就來遲了。”
\end{parag}


\begin{parag}
    說笑一會,賈母因見窗上紗的顏色舊了,便和王夫人說道:“這個紗新糊上好看,過了後來就不翠了。這個院子裏頭又沒有個桃杏樹,這竹子已是綠的,再拿這綠紗糊上反不配。我記得咱們先有四五樣顏色糊窗的紗呢,明兒給他把這窗上的換了。”鳳姐兒忙道:“昨兒我開庫房,看見大板箱裏還有好些匹銀紅蟬翼紗,也有各樣折枝花樣的,也有流雲萬福花樣的,也有百蝶穿花花樣的,顏色又鮮,紗又輕軟,我竟沒見過這樣的。拿了兩匹出來,作兩牀綿紗被,想來一定是好的。”賈母聽了笑道:“呸,人人都說你沒有不經過不見過,連這個紗還不認得呢,明兒還說嘴。”薛姨媽等都笑說:“憑他怎麼經過見過,如何敢比老太太呢。老太太何不教導了他,我們也聽聽。”鳳姐兒也笑說:“好祖宗,教給我罷。”賈母笑向薛姨媽衆人道:“那個紗,比你們的年紀還大呢。怪不得他認作蟬翼紗,原也有些象,不知道的,都認作蟬翼紗。正經名字叫作‘軟煙羅’。”鳳姐兒道:“這個名兒也好聽。只是我這麼大了,紗羅也見過幾百樣,從沒聽見過這個名色。”賈母笑道: “你能夠活了多大,見過幾樣沒處放的東西,就說嘴來了。那個軟煙羅只有四樣顏色:一樣雨過天晴,一樣秋香色,一樣松綠的,一樣就是銀紅的。若是做了帳子,糊了窗屜,遠遠的看著,就似煙霧一樣,所以叫作‘軟煙羅’,那銀紅的又叫作‘霞影紗’。如今上用的府紗也沒有這樣軟厚輕密的了。”薛姨媽笑道:“別說鳳丫頭沒見,連我也沒聽見過。”鳳姐兒一面說,早命人取了一匹來了。賈母說:“可不是這個!先時原不過是糊窗屜,後來我們拿這個作被作帳子,試試也竟好。明兒就找出幾匹來,拿銀紅的替他糊窗子。”鳳姐答應著。衆人都看了,稱讚不已。劉姥姥也覷著眼看個不了,唸佛說道:“我們想他作衣裳也不能,拿著糊窗子,豈不可惜?”賈母道:“倒是做衣裳不好看。”鳳姐忙把自己身上穿的一件大紅綿紗襖子襟兒拉了出來,向賈母薛姨媽道:“看我的這襖兒。”賈母薛姨媽都說:“這也是上好的了,這是如今的上用內造的,竟比不上這個。”鳳姐兒道:“這個薄片子,還說是上用內造呢,竟連官用的也比不上了。” 賈母道:“再找一找,只怕還有青的。若有時都拿出來,送這劉親家兩匹,做一個帳子我掛,下剩的添上裏子,做些夾背心子給丫頭們穿,白收著黴壞了。”鳳姐忙答應了,仍令人送去。賈母起身笑道:“這屋裏窄,再往別處逛去。”劉姥姥唸佛道:“人人都說大家子住大房。昨兒見了老太太正房,配上大箱大櫃大桌子大牀,果然威武。那櫃子比我們那一間房子還大還高。怪道後院子裏有個梯子。我想並不上房曬東西,預備個梯子作什麼?後來我想起來,定是爲開頂櫃收放東西,非離了那梯子,怎麼得上去呢。如今又見了這小屋子,更比大的越發齊整了。滿屋裏的東西都只好看,都不知叫什麼,我越看越捨不得離了這裏。”鳳姐道:“還有好的呢,我都帶你去瞧瞧。”說著一徑離了瀟湘館。
\end{parag}


\begin{parag}
    遠遠望見池中一羣人在那裏撐舡。賈母道:“他們既預備下船,咱們就坐。”一面說著,便向紫菱洲蓼漵一帶走來。未至池前,只見幾個婆子手裏都捧著一色捏絲戧金五彩大盒子走來。鳳姐忙問王夫人早飯在那裏擺。王夫人道:“問老太太在那裏,就在那裏罷了。”賈母聽說,便回頭說:“你三妹妹那裏就好。你就帶了人擺去,我們從這裏坐了舡去。”鳳姐聽說,便回身同了探春、李紈、鴛鴦、琥珀帶著端飯的人等,抄著近路到了秋爽齋,就在曉翠堂上調開桌案。鴛鴦笑道:“天天咱們說外頭老爺們喫酒喫飯都有一個篾片相公,拿他取笑兒。咱們今兒也得了一個女篾片了。”李紈是個厚道人,聽了不解。鳳姐兒卻知是說的是劉姥姥了,也笑說道:“咱們今兒就拿他取個笑兒。”二人便如此這般的商議。李紈笑勸道:“你們一點好事也不做,又不是個小孩兒,還這麼淘氣,仔細老太太說。”鴛鴦笑道: “很不與你相干,有我呢。”
\end{parag}


\begin{parag}
    正說著,只見賈母等來了,各自隨便坐下。先著丫鬟端過兩盤茶來,大家喫畢。鳳姐手裏拿著西洋布手巾,裹著一把烏木三鑲銀箸,敁敠人位,按席擺下。賈母因說:“把那一張小楠木桌子抬過來,讓劉親家近我這邊坐著。”衆人聽說,忙抬了過來。鳳姐一面遞眼色與鴛鴦,鴛鴦便拉了劉姥姥出去,那牡囑咐了劉姥姥一席話,又說:“這是我們家的規矩,若錯了我們就笑話呢。”調停已畢,然後歸坐。薛姨媽是喫過飯來的,不喫,只坐在一邊喫茶。\begin{note}庚雙夾:妙!若只管寫薛姨媽來則喫飯,則成何義理?\end{note}賈母帶著寶玉、湘雲、黛玉、寶釵一桌,王夫人帶著迎春姊妹三個人一桌,劉姥姥傍著賈母一桌。賈母素日喫飯,皆有小丫鬟在旁邊,拿著漱盂麈尾巾帕之物。如今鴛鴦是不當這差的了,今日鴛鴦偏接過麈尾來拂著。丫鬟們知道他要撮弄劉姥姥,便躲開讓他。鴛鴦一面侍立,一面悄向劉姥姥說道:“別忘了。”劉姥姥道:“姑娘放心。”那劉姥姥入了坐,拿起箸來,沉甸甸的不伏手。原是鳳姐和鴛鴦商議定了,單拿一雙老年四楞象牙鑲金的筷子與劉姥姥。劉姥姥見了,說道:“這叉爬子比俺那裏鐵掀還沉,那裏犟的過他。”說的衆人都笑起來。
\end{parag}


\begin{parag}
    只見一個媳婦端了一個盒子站在當地,一個丫鬟上來揭去盒蓋,裏面盛著兩碗菜。李紈端了一碗放在賈母桌上。鳳姐兒偏揀了一碗鴿子蛋放在劉姥姥桌上。賈母這邊說聲“請”,劉姥姥便站起身來,高聲說道:“老劉,老劉,食量大似牛,喫一個老母豬不抬頭。”自己卻鼓著腮不語。衆人先是發怔,後來一聽,上上下下都哈哈的大笑起來。史湘雲撐不住,一口飯都噴了出來;林黛玉笑岔了氣,伏著桌子噯喲;寶玉早滾到賈母懷裏,賈母笑的摟著寶玉叫“心肝”;王夫人笑的用手指著鳳姐兒,只說不出話來;薛姨媽也撐不住,口裏茶噴了探春一裙子;探春手裏的飯碗都合在迎春身上;惜春離了坐位,拉著他奶母叫揉一揉腸子。地下的無一個不彎腰屈背,也有躲出去蹲著笑去的,也有忍著笑上來替他姊妹換衣裳的,獨有鳳姐鴛鴦二人撐著,還只管讓劉姥姥。劉姥姥拿起箸來,只覺不聽使,又說道:“這裏的雞兒也俊,下的這蛋也小巧,怪俊的。我且肏攮一個。”衆人方住了笑,聽見這話又笑起來。賈母笑的眼淚出來,琥珀在後捶著。賈母笑道:“這定是鳳丫頭促狹鬼兒鬧的,快別信他的話了。”那劉姥姥正誇雞蛋小巧,要肏攮一個,鳳姐兒笑道:“一兩銀子一個呢,你快嚐嚐罷,那冷了就不好吃了。”劉姥姥便伸箸子要夾,那裏夾的起來,滿碗裏鬧了一陣好的,好容易撮起一個來,才伸著脖子要喫,偏又滑下來滾在地下,忙放下箸子要親自去撿,早有地下的人撿了出去了。劉姥姥嘆道: “一兩銀子,也沒聽見響聲兒就沒了。”衆人已沒心喫飯,都看著他笑。賈母又說:“這會子又把那個筷子拿了出來,又不請客擺大筵席。都是鳳丫頭支使的,還不換了呢。”地下的人原不曾預備這牙箸,本是鳳姐和鴛鴦拿了來的,聽如此說,忙收了過去,也照樣換上一雙烏木鑲銀的。劉姥姥道:“去了金的,又是銀的,到底不及俺們那個伏手。”鳳姐兒道:“菜裏若有毒,這銀子下去了就試的出來。”劉姥姥道:“這個菜裏若有毒,俺們那菜都成了砒霜了。那怕毒死了也要吃盡了。” 賈母見他如此有趣,喫的又香甜,把自己的也都端過來與他喫。又命一個老嬤嬤來,將各樣的菜給板兒夾在碗上。
\end{parag}


\begin{parag}
    一時喫畢,賈母等都往探春臥室中去說閒話。這裏收拾過殘桌,又放了一桌。劉姥姥看著李紈與鳳姐兒對坐著喫飯,嘆道:“別的罷了,我只愛你們家這行事。怪道說‘禮出大家’。”鳳姐兒忙笑道:“你可別多心,纔剛不過大家取笑兒。”一言未了,鴛鴦也進來笑道:“姥姥別惱,我給你老人家賠個不是。”劉姥姥笑道:“姑娘說那裏話,咱們哄著老太太開個心兒,可有什麼惱的!你先囑咐我,我就明白了,不過大家取個笑兒。我要心裏惱,也就不說了。”鴛鴦便罵人“爲什麼不倒茶給姥姥喫?”劉姥姥忙道:“剛纔那個嫂子倒了茶來,我喫過了。姑娘也該用飯了。”鳳姐兒便拉鴛鴦:“你坐下和我們吃了罷,省的回來又鬧。”鴛鴦便坐下了。婆子們添上碗箸來,三人喫畢。劉姥姥笑道:“我看你們這些人都只吃這一點兒就完了,虧你們也不餓。怪只道風兒都吹的倒。” 鴛鴦便問:“今兒剩的菜不少,都那去了?”婆子們道:“都還沒散呢,在這裏等著一齊散與他們喫。”鴛鴦道:“他們吃不了這些,挑兩碗給二奶奶屋裏平丫頭送去。”鳳姐兒道:“他早吃了飯了,不用給他。”鴛鴦道:“他不吃了,餵你們的貓。”婆子聽了,忙揀了兩樣拿盒子送去。鴛鴦道:“素雲那去了?”李紈道: “他們都在這裏一處喫,又找他作什麼。”鴛鴦道:“這就罷了。”鳳姐兒道:“襲人不在這裏,你倒是叫人送兩樣給他去。”鴛鴦聽說,便命人也送兩樣去後,鴛鴦又問婆子們:“回來喫酒的攢盒可裝上了?”婆子道:“想必還得一會子。”鴛鴦道:“催著些兒。”婆子應喏了。
\end{parag}


\begin{parag}
    鳳姐兒等來至探春房中,只見他娘兒們正說笑。探春素喜闊朗,這三間屋子並不曾隔斷。當地放著一張花梨大理石大案,案上磊著各種名人法帖,並數十方寶硯,各色筆筒,筆海內插的筆如樹林一般。那一邊設著斗大的一個汝窯花囊,插著滿滿的一囊水晶球兒的白菊。西牆上當中掛著一大幅米襄陽《煙雨圖》,左右掛著一副對聯,乃是顏魯公墨跡,其詞雲:
\end{parag}


\begin{poem}
    \begin{pl}煙霞閒骨格,泉石野生涯。\end{pl}
\end{poem}


\begin{parag}
    案上設著大鼎。左邊紫檀架上放著一個大觀窯的大盤,盤內盛著數十個嬌黃玲瓏大佛手。右邊洋漆架上懸著一個白玉比目磬,旁邊掛著小錘。那板兒略熟了些,便要摘那錘子要擊,丫鬟們忙攔住他。他又要佛手喫,探春揀了一個與他說:“頑罷,喫不得的。”東邊便設著臥榻,拔步牀上懸著蔥綠雙繡卉草蟲的紗帳。板兒又跑過來看,說:“這是蟈蟈,這是螞蚱。”劉姥姥忙打了他一巴掌,罵道:“下作黃子,沒乾沒淨的亂鬧。倒叫你進來瞧瞧,就上臉了。”打的板兒哭起來,衆人忙勸解方罷。賈母因隔著紗窗往後院內看了一回,說道:“後廊檐下的梧桐也好了,就只細些。”正說話,忽一陣風過,隱隱聽得鼓樂之聲。賈母問“是誰家娶親呢?這裏臨街倒近。”王夫人等笑回道:“街上的那裏聽的見,這是咱們的那十幾個女孩子們演習吹打呢。”賈母便笑道:“既是他們演,何不叫他們進來演習。他們也逛一逛,咱們可又樂了。”鳳姐聽說,忙命人出去叫來,又一面吩咐擺下條桌,鋪上紅氈子。賈母道:“就鋪排在藕香榭的水亭子上,藉著水音更好聽。回來咱們就在綴錦閣底下喫酒,又寬闊,又聽的近。”衆人都說那裏好。賈母向薛姨媽笑道:“咱們走罷。他們姊妹們都不大喜歡人來坐著,怕髒了屋子。咱們別沒眼色,正經坐一回子船喝酒去。”說著大家起身便走。探春笑道:“這是那裏的話,求著老太太姨太太來坐坐還不能呢。”賈母笑道:“我的這三丫頭卻好,只有兩個玉兒可惡。回來喫醉了,咱們偏往他們屋裏鬧去。”
\end{parag}


\begin{parag}
    說著,衆人都笑了,一齊出來。走不多遠,已到了 葉渚。 姑蘇選來的幾個駕娘早把兩隻棠舫撐來,衆人扶了賈、王夫人、薛姨媽、劉姥姥、鴛鴦、玉釧兒上了這一隻,落後李紈也跟上去。鳳姐兒也上去,立在舡頭上,也要撐舡。賈母在艙內道:“這不是頑的,雖不是河裏,也有好深的。你快不給我進來。”鳳姐兒笑道:“怕什麼!老祖宗只管放心。”說著便一篙點開。到了池當中,舡小人多,鳳姐只覺亂晃,忙把篙子遞與駕娘,方蹲下了。然後迎春姊妹等並寶玉上了那隻,隨後跟來。其餘老嬤嬤散衆丫鬟俱沿河隨行。寶玉道:“這些破荷葉可恨,怎麼還不叫人來拔去。”寶釵笑道:“今年這幾日,何曾饒了這園子閒了,天天逛,那裏還有叫人來收拾的工夫。”林黛玉道:“我最不喜歡李義山的詩,只喜他這一句‘留得殘荷聽雨聲’。偏你們又不留著殘荷了。”寶玉道:“果然好句,以後咱們就別叫人拔去了。”說著已到了花漵的蘿港之下,覺得陰森透骨,兩灘上衰草殘菱,更助秋情。更助秋情。
\end{parag}


\begin{parag}
    賈母因見岸上的清廈曠朗,便問“這是你薛姑娘的屋子不是?”衆人道:“是。”賈母忙命攏岸,順著雲步石梯上去,一同進了蘅蕪苑,只覺異香撲鼻。那些奇草仙藤愈冷愈蒼翠,都結了實,似珊瑚豆子一般,累垂可愛。及進了房屋,雪洞一般,一色玩器全無,案上只有一個土定瓶中供著數枝菊花,並兩部書,茶奩茶杯而已。牀上只吊著青紗帳幔,衾褥也十分樸素。賈母嘆道:“這孩子太老實了。你沒有陳設,何妨和你姨娘要些。我也不理論,也沒想到,你們的東西自然在家裏沒帶了來。”說著,命鴛鴦去取些古董來,又嗔著鳳姐兒:“不送些玩器來與你妹妹,這樣小器。”王夫人鳳姐兒等都笑回說:“他自己不要的。我們原送了來,他都退回去了。”薛姨媽也笑說:“他在家裏也不大弄這些東西的。”賈母搖頭道:“使不得。雖然他省事,倘或來一個親戚,看著不象;二則年輕的姑娘們,房裏這樣素淨,也忌諱。我們這老婆子,越發該住馬圈去了。你們聽那些書上戲上說的小姐們的繡房,精緻的還了得呢。他們姊妹們雖不敢比那些小姐們,也不要很離了格兒。有現成的東西,爲什麼不擺?若很愛素淨,少幾樣倒使得。我最會收拾屋子的,如今老了,沒有這些閒心了。他們姊妹們也還學著收拾的好,只怕俗氣,有好東西也擺壞了。我看他們還不俗。如今讓我替你收拾,包管又大方又素淨。我的梯己兩件,收到如今,沒給寶玉看見過,若經了他的眼,也沒了。”說著叫過鴛鴦來,親吩咐道:“你把那石頭盆景兒和那架紗桌屏,還有個墨煙凍石鼎,這三樣擺在這案上就夠了。再把那水墨字畫白綾帳子拿來,把這帳子也換了。”鴛鴦答應著,笑道: “這些東西都擱在東樓上的不知那個箱子裏,還得慢慢找去,明兒再拿去也罷了。”賈母道:“明日後日都使得,只別忘了。”說著,坐了一回方出來,一徑來至錦閣下。文官等上來請過安,因問“演習何曲”。賈母道:“只揀你們生的演習幾套罷。”文官等下來,往藕香榭去不提。
\end{parag}


\begin{parag}
    這裏鳳姐兒已帶著人擺設整齊,上面左右兩張榻,榻上都鋪著錦裀蓉簟,每一榻前有兩張雕漆幾,也有海棠式的,也有梅花式的,也有荷葉式的,也有葵花式的,也有方的,也有圓的,其式不一。一個上面放著爐瓶,一分攢盒,一個上面空設著,預備放人所喜食物。上面二榻四幾,是賈母薛姨媽;下面一椅兩幾,是王夫人的,餘者都是一椅一幾。東邊是劉姥姥,劉姥姥之下便是王夫人。西邊便是史湘雲,第二便是寶釵,第三便是黛玉,第四迎春、探春、惜春挨次下去,寶玉在末。李紈鳳姐二人之幾設於三層檻內,二層紗廚之外。攢盒式樣,亦隨幾之式樣。每人一把烏銀洋鏨自斟壺,一個十錦琺琅杯。
\end{parag}


\begin{parag}
    大家坐定,賈母先笑道:“咱們先喫兩杯,今日也行一令纔有意思。”薛姨媽等笑道:“老太太自然有好酒令,我們如何會呢,安心要我們醉了。我們都多喫兩杯就有了。”賈母笑道:“姨太太今兒也過謙起來,想是厭我老了。”薛姨媽笑道:“不是謙,只怕行不上來倒是笑話了。”王夫人忙笑道:“便說不上來,就便多喫一杯酒,醉了睡覺去,還有誰笑話咱們不成。”薛姨媽點頭笑道:“依令。老太太到底喫一杯令酒纔是。”賈母笑道:“這個自然。”說著便吃了一杯。
\end{parag}


\begin{parag}
    鳳姐兒忙走至當地,笑道:“既行令,還叫鴛鴦姐姐來行更好。”衆人都知賈母所行之令必得鴛鴦提著,故聽了這話,都說:“很是。”鳳姐兒便拉了鴛鴦過來。王夫人笑道:“既在令內,沒有站著的理。”回頭命小丫頭子:“端一張椅子,放在你二位奶奶的席上。”鴛鴦也半推半就,謝了坐,便坐下,也吃了一鍾酒,笑道:“酒令大如軍令,不論尊卑,惟我是主。違了我的話,是要受罰的。”王夫人等都笑道:“一定如此,快些說來。”鴛鴦未開口,劉姥姥便下了席,擺手道: “別這樣捉弄人家,我家去了。”衆人都笑道:“這卻使不得。”鴛鴦喝令小丫頭子們:“拉上席去!”小丫頭子們也笑著,果然拉入席中。劉姥姥只叫:“饒了我罷!”鴛鴦道:“再多言的罰一壺。”劉姥姥方住了聲。鴛鴦道:“如今我說骨牌副兒,從老太太起,順領說下去,至劉姥姥止。比如我說一副兒,將這三張牌拆開,先說頭一張,次說第二張,再說第三張,說完了,合成這一副兒的名字。無論詩詞歌賦,成語俗話,比上一句,都要叶韻。錯了的罰一杯。”衆人笑道:“這個令好,就說出來。”鴛鴦道:“有了一副了。左邊是張‘天’。”賈母道:“頭上有青天。”衆人道:“好。”鴛鴦道:“當中是個‘五與六’。”賈母道:“六橋梅花香徹骨。”鴛鴦道:“剩得一張‘六與幺 ’。”賈母道:“一輪紅日出雲霄。”鴛鴦道:“湊成便是個‘蓬頭鬼’。”賈母道:“這鬼抱住鍾馗腿。”說完,大家笑說:“極妙。”賈母飲了一杯。鴛鴦又道:“有了一副。左邊是個‘大長五’。”薛姨媽道:“梅花朵朵風前舞。”鴛鴦道:“右邊還是個‘大五長’。”薛姨媽道:“十月梅花嶺上香。”鴛鴦道:“當中‘二五’是雜七。”薛姨媽道:“織女牛郎會七夕。”鴛鴦道:“湊成‘二郎遊五嶽’。”薛姨媽道:“世人不及神仙樂。”說完,大家稱賞,飲了酒。鴛鴦又道:“有了一副。左邊‘長幺’兩點明。”湘雲道:“雙懸日月照乾坤。”鴛鴦道:“右邊‘長幺’兩點明。”湘雲道:“閒花落地聽無聲。”鴛鴦道:“中間還得 ‘幺四’來。”湘雲道:“日邊紅杏倚雲栽。”鴛鴦道:“湊成‘櫻桃九熟’。”湘雲道:“御園卻被鳥銜出。”說完飲了一杯。鴛鴦道:“有了一副。左邊是‘長三’。”寶釵道:“雙雙燕子語梁間。” 鴛鴦道:“右邊是‘三長’。”寶釵道:“水荇牽風翠帶長。”鴛鴦道:“當中‘三六’九點在。”寶釵道:“三山半落青天外。”鴛鴦道:“湊成‘鐵鎖練孤舟 ’。”寶釵道:“處處風波處處愁。”說完飲畢。鴛鴦又道:“左邊一個‘天’。”黛玉道:“良辰美景奈何天。”寶釵聽了,回頭看著他。黛玉只顧怕罰,也不理論。鴛鴦道:“中間‘錦屏’顏色俏。”黛玉道:“紗窗也沒有紅娘報。”鴛鴦道:“剩了‘二六’八點齊。”黛玉道:“雙瞻玉座引朝儀。”鴛鴦道:“湊成‘籃子’好採花。”黛玉道:“仙杖香挑芍藥花。”說完,飲了一口。鴛鴦道:“左邊‘四五’成花九。”迎春道:“桃花帶雨濃。”衆人道:“該罰!錯了韻,而且又不象。”迎春笑著飲了一口。原是鳳姐兒和鴛鴦都要聽劉姥姥的笑話,故意都令說錯,都罰了。至王夫人,鴛鴦代說了個,下便該劉姥姥。劉姥姥道:“我們莊家人閒了,也常會幾個人弄這個,但不如說的這麼好聽。少不得我也試一試。”衆人都笑道:“容易說的。你只管說,不相干。”鴛鴦笑道:“左邊‘四四’是個人。” 劉姥姥聽了,想了半日,說道:“是個莊家人罷。”衆人鬨堂笑了。賈母笑道:“說的好,就是這樣說。”劉姥姥也笑道:“我們莊家人,不過是現成的本色,衆位別笑。”鴛鴦道:“中間‘三四’綠配紅。”劉姥姥道:“大火燒了毛毛蟲。”衆人笑道:“這是有的,還說你的本色。”鴛鴦道:“右邊‘幺四’真好看。”劉姥姥道:“一個蘿蔔一頭蒜。”衆人又笑了。鴛鴦笑道:“湊成便是一枝花。”劉姥姥兩隻手比著,說道:“花兒落了結個大倭瓜。”衆人大笑起來。只聽外面亂嚷 ——
\end{parag}


\begin{parag}
    \begin{note}蒙回末總:寓貧賤輩低首豪門,凌辱不計,誠可悲乎!此故作者以警貧窮。而富室貴家亦當於其間著意。\end{note}
\end{parag}

