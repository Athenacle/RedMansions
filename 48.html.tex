\chap{四十八}{滥情人情误思游艺 慕雅女雅集苦吟诗}
\begin{parag}

    \begin{note}庚辰:题曰“柳湘莲走他乡”,必谓写湘莲如何走,今却不写,反写阿呆兄之游艺,了却柳湘莲之分内走者而不细写其走,反写阿呆不应走而写其走,文牵歧路,令人不识者如此。\end{note}
\end{parag}


\begin{parag}


    \begin{note}庚辰:至“情小妹”回中方写湘莲文字,真神化之笔。\end{note}
\end{parag}


\begin{parag}


    \begin{note}蒙回前总批:心地聪明性自灵,喜同雅品讲诗经,娇柔倍觉可怜形。皓齿朱唇真嬝嬝,痴情专意更娉娉,宜人解语小星星。\end{note}
\end{parag}

\begin{parag}

    且说薛蟠听见如此说了,气方渐平。三五日后,疼痛虽愈,伤痕未平,只装病在家,愧见亲友。
\end{parag}


\begin{parag}


    展眼已到十月,因有各铺面伙计内有算年帐要回家的,少不得家内治酒饯行。内有一个张德辉,年过六十,自幼在薛家当铺内揽总,家内也有二三千金的过活,今岁也要回家,明春方来。因说起“今年纸札香料短少,明年必是贵的。明年先打发大小儿上来当铺内照管,赶端阳前我顺路贩些纸札香扇来卖。除去关税花销,亦可以剩得几倍利息。”薛蟠听了,心中忖度:“我如今挨了打,正难见人,想著要躲个一年半载,又没处去躲。天天装病,也不是事。况且我长了这么大,文又不文,武又不武,虽说做买卖,究竟戥子算盘从没拿过,地土风俗远近道路又不知道,不如也打点几个本钱,和张德辉逛一年来。赚钱也罢,不赚钱也罢,且躲躲羞去。二则逛逛山水也是好的。”心内主意已定,至酒席散后,便和张德辉说知,命他等一二日一同前往。
\end{parag}


\begin{parag}


    晚间薛蟠告诉了他母亲。薛姨妈听了虽是欢喜,但又恐他在外生事,花了本钱倒是末事,因此不命他去,只说:“好歹你守著我,我还能放心些。况且也不用做这买卖,也不等著这几百银子来用。你在家里安分守己的,就强似这几百银子了。”薛蟠主意已定,那里肯依,只说:“天天又说我不知世事,这个也不知,那个也不学。如今我发狠把那些没要紧的都断了,如今要成人立事,学习著做买卖,又不准我了,叫我怎么样呢?我又不是个丫头,把我关在家里,何日是个了日?况且那张德辉又是个年高有德的,咱们和他世交,我同他去,怎么得有舛错?我就一时半刻有不好的去处,他自然说我劝我。就是东西贵贱行情,他是知道的,自然色色问他,何等顺利,倒不叫我去。过两日我不告诉家里,私自打点了一走,明年发了财回家,那时才知道我呢。”说毕,赌气睡觉去了。
\end{parag}


\begin{parag}


    薛姨妈听他如此说,因和宝钗商议。宝钗笑道:“哥哥果然要经历正事,正是好的了。只是他在家时说著好听,到了外头旧病复犯,越发难拘束他了。但也愁不得许多。他若是真改了,是他一生的福。若不改,妈也不能又有别的法子。一半尽人力,一半听天命罢了。这么大人了,若只管怕他不知世路,出不得门,干不得事,今年关在家里,明年还是这个样儿。他既说的名正言顺,妈就打谅著丢了八百一千银子,竟交与他试一试。横竖有伙计们帮著,也未必好意思哄骗他的。二则他出去了,左右没有助兴的人,又没了倚仗的人,到了外头,谁还怕谁,有了的吃,没了的饿著,举眼无靠,他见这样,只怕比在家里省了事也未可知。”\begin{note}庚辰双行夹批:作书者曾吃此亏,批书者亦曾吃此亏,故特于此注明,使后来人深思默戒。脂砚斋。\end{note}薛姨妈听了,思忖半晌说道:“倒是你说的是。花两个钱,叫他学些乖来也值了。”商议已定,一宿无话。
\end{parag}


\begin{parag}


    至次日,薛姨妈命人请了张德辉来,在书房中命薛蟠款待酒饭,自己在后廊下,隔著窗子,向里千言万语嘱托张德辉照管薛蟠。张德辉满口应承,吃过饭告辞,又回说:“十四日是上好出行日期,大世兄即刻打点行李,雇下骡子,十四一早就长行了。”薛蟠喜之不尽,将此话告诉了薛姨妈。薛姨妈便和宝钗香菱并两个老年的嬷嬷连日打点行装,派下薛蟠之乳父老苍头一名,当年谙事旧仆二名,外有薛蟠随身常使小厮二人,主仆一共六人,雇了三辆大车,单拉行李使物,又雇了四个长行骡子。薛蟠自骑一匹家内养的铁青大走骡,外备一匹坐马。诸事完毕,薛姨妈宝钗等连夜劝戒之言,自不必备说。
\end{parag}


\begin{parag}


    至十三日,薛蟠先去辞了他舅舅,然后过来辞了贾宅诸人。贾珍等未免又有饯行之说,也不必细述。至十四日一早,薛姨妈宝钗等直同薛蟠出了仪门,母女两个四只泪眼看他去了,方回来。
\end{parag}


\begin{parag}


    薛姨妈上京带来的家人不过四五房,并两三个老嬷嬷小丫头,今跟了薛蟠一去,外面只剩了一两个男子。因此薛姨妈即日到书房,将一应陈设玩器并帘幔等物尽行搬了进来收贮,命那两个跟去的男子之妻一并也进来睡觉。又命香菱将他屋里也收拾严紧,“将门锁了,晚间和我去睡。”宝钗道:“妈既有这些人作伴,不如叫菱姐姐和我作伴去。我们园里又空,夜长了,我每夜作活,越多一个人岂不越好。”薛姨妈听了,笑道:“正是我忘了,原该叫他同你去才是。我前日还同你哥哥说,文杏又小,道三不著两,莺儿一个人不够伏侍的,还要买一个丫头来你使。”宝钗道:“买的不知底里,倘或走了眼,花了钱小事,没的淘气。倒是慢慢的打听著,有知道来历的,买个还罢了。”\begin{note}庚辰双行夹批:闲言过耳无迹,然又伏下一事矣。\end{note}一面说,一面命香菱收拾了衾褥妆奁,命一个老嬷嬷并臻儿送至蘅芜苑去,然后宝钗和香菱才同回园中来。\begin{note}庚辰双行夹批:细想香菱之为人也,根基不让迎、探,容貌不让凤、秦,端雅不让纨、钗,风流不让湘、黛,贤惠不让袭、平,所惜者青年罹祸,命运乖蹇,至为侧室,且虽曾读书,不能与林、湘辈并驰于海棠之社耳。然此一人岂可不入园哉?故欲令入园,终无可入之隙,筹划再四,欲令入园必呆兄远行后方可。然阿呆兄又如何方可远行?曰名,不可;利,不可;无事,不可;必得万人想不到,自己忽发一机之事方可。因此思及“情”之一字及呆素所误者,故借“情误”二字生出一事,使阿呆游艺之志已坚,则菱卿入园之隙方妥。回思因欲香菱入园,是写阿呆情误,因欲阿呆情误,先写一赖尚荣,实委婉严密之甚也。脂砚斋评。\end{note}\begin{note}靖眉批:此批甚当。\end{note}
\end{parag}


\begin{parag}


    香菱道:“我原要和奶奶说的,大爷去了,我和姑娘作伴儿去。又恐怕奶奶多心,说我贪著园里来顽;谁知你竟说了。”宝钗笑道:“我知道你心里羡慕这园子不是一日两日了,只是没个空儿。就每日来一趟,慌慌张张的,也没趣儿。所以趁著机会,越性住上一年,我也多个作伴的,你也遂了心。”香菱笑道:“好姑娘,你趁著这个功夫,教给我作诗罢。”\begin{note}庚辰双行夹批:写得何其有趣,今忽见菱卿此句,合卷从纸上另走出一娇小美人来,并不是湘、林、探、凤等一样口气声色。真神骏之技,虽驱驰万里而不见有倦怠之色。\end{note}宝钗笑道:“我说你‘得陇望蜀’呢。我劝你今儿头一日进来,先出园东角门,从老太太起,各处各人你都瞧瞧,问候一声儿,也不必特意告诉他们说搬进园来。若有提起因由,你只带口说我带了你进来作伴儿就完了。回来进了园,再到各姑娘房里走走。”
\end{parag}


\begin{parag}


    香菱应著才要走时,只见平儿忙忙的走来。\begin{note}庚辰双行夹批:“忙忙”二字奇,不知有何妙文。\end{note}香菱忙问了好,平儿只得陪笑相问。宝钗因向平儿笑道:“我今儿带了他来作伴儿,正要去回你奶奶一声儿。”平儿笑道:“姑娘说的是那里话?我竟没话答言了。” 宝钗道:“这才是正理。店房也有个主人,庙里也有个住持。虽不是大事,到底告诉一声,便是园里坐更上夜的人知道添了他两个,也好关门候户的了。你回去告诉一声罢,我不打发人去了。”平儿答应著,因又向香菱笑道:“你既来了,也不拜一拜街坊邻舍去?”\begin{note}庚辰双行夹批:是极,恰是戏言,实欲支出香菱去也。\end{note}宝钗笑道:“我正叫他去呢。”平儿道:“你且不必往我们家去,二爷病了在家里呢。”香菱答应著去了,先从贾母处来,不在话下。
\end{parag}


\begin{parag}


    且说平儿见香菱去了,便拉宝钗忙说道:“姑娘可听见我们的新闻了?”宝钗道:“我没听见新闻。因连日打发我哥哥出门,所以你们这里的事,一概也不知道,连姊妹们这两日也没见。”平儿笑道:“老爷把二爷打了个动不得,难道姑娘就没听见?”宝钗道:“早起恍惚听见了一句,也信不真。我也正要瞧你奶奶去呢,不想你来了。又是为了什么打他?”平儿咬牙骂道:“都是那贾雨村什么风村,半路途中那里来的饿不死的野杂种!认了不到十年,生了多少事出来!今年春天,老爷不知在那个地方看见了几把旧扇子,回家看家里所有收著的这些好扇子都不中用了,立刻叫人各处搜求。谁知就有一个不知死的冤家,混号儿世人叫他作石呆子,穷的连饭也没的吃,偏他家就有二十把旧扇子,死也不肯拿出大门来。二爷好容易烦了多少情,见了这个人,说之再三,把二爷请到他家里坐著,拿出这扇子略瞧了一瞧。据二爷说,原是不能再有的,全是湘妃、棕竹、麋鹿、玉竹的,皆是古人写画真迹,因来告诉了老爷。老爷便叫买他的,要多少银子给他多少。偏那石呆子说:‘我饿死冻死,一千两银子一把我也不卖!’老爷没法子,天天骂二爷没能为。已经许了他五百两,先兑银子后拿扇子。他只是不卖,只说:‘要扇子,先要我的命!’姑娘想想,这有什么法子?谁知雨村那没天理的听见了,便设了个法子,讹他拖欠了官银,拿他到衙门里去,说所欠官银,变卖家产赔补,把这扇子抄了来,作了官价送了来。那石呆子如今不知是死是活。老爷拿著扇子问著二爷说:‘人家怎么弄了来?’二爷只说了一句:‘为这点子小事,弄得人坑家败业,也不算什么能为!’老爷听了就生了气,说二爷拿话堵老爷,因此这是第一件大的。这几日还有几件小的,我也记不清,所以都凑在一处,就打起来了。也没拉倒用板子棍子,就站著,不知拿什么混打一顿,脸上打破了两处。我们听见姨太太这里有一种丸药,上棒疮的,姑娘快寻一丸子给我。”宝钗听了,忙命莺儿去要了一丸来与平儿。宝钗道:“既这样,替我问候罢,我就不去了。”平儿答应著去了,不在话下。
\end{parag}


\begin{parag}


    且说香菱见过众人之后,吃过晚饭,宝钗等都往贾母处去了,自己便往潇湘馆中来。此时黛玉已好了大半,见香菱也进园来住,自是欢喜。香菱因笑道:“我这一进来了,也得了空儿,好歹教给我作诗,就是我的造化了!”黛玉笑道:“既要作诗,你就拜我作师。我虽不通,大略也还教得起你。”香菱笑道:“果然这样,我就拜你作师。你可不许腻烦的。”黛玉道:“什么难事,也值得去学!不过是起承转合,当中承转是两副对子,平声对仄声,虚的对实的,实的对虚的,若是果有了奇句,连平仄虚实不对都使得的。”香菱笑道:“怪道我常弄一本旧诗偷空儿看一两首,又有对的极工的,又有不对的,又听见说‘一三五不论,二四六分明’。看古人的诗上亦有顺的,亦有二四六上错了的,所以天天疑惑。如今听你一说,原来这些格调规矩竟是末事,只要词句新奇为上。”黛玉道:“正是这个道理。词句究竟还是末事,第一立意要紧。若意趣真了,连词句不用修饰,自是好的,这叫做‘不以词害意’。”香菱笑道:“我只爱陆放翁的诗‘重帘不卷留香久,古砚微凹聚墨多’,说的真有趣!”黛玉道:“断不可学这样的诗。你们因不知诗,所以见了这浅近的就爱,一入了这个格局,再学不出来的。你只听我说,你若真心要学,我这里有《王摩诘全集》,你且把他的五言律读一百首,细心揣摩透熟了,然后再读一二百首老杜的七言律,次再李青莲的七言绝句读一二百首。肚子里先有了这三个人作了底子,然后再把陶渊明、应玚、谢、阮、庾、鲍等人的一看。你又是一个极聪敏伶俐的人,不用一年的工夫,不愁不是诗翁了!”香菱听了,笑道:“既这样,好姑娘,你就把这书给我拿出来,我带回去夜里念几首也是好的。”黛玉听说,便命紫鹃将王右丞的五言律拿来,递与香菱,又道:“你只看有红圈的都是我选的,有一首念一首。不明白的问你姑娘,或者遇见我,我讲与你就是了。” 香菱拿了诗,回至蘅芜苑中,诸事不顾,只向灯下一首一首的读起来。宝钗连催他数次睡觉,他也不睡。宝钗见他这般苦心,只得随他去了。
\end{parag}


\begin{parag}


    一日,黛玉方梳洗完了,只见香菱笑吟吟的送了书来,又要换杜律。黛玉笑道:“共记得多少首?”香菱笑道:“凡红圈选的我尽读了。”黛玉道:“可领略了些滋味没有?”香菱笑道:“领略了些滋味,不知可是不是,说与你听听。”黛玉笑道:“正要讲究讨论,方能长进。你且说来我听。”香菱笑道:“据我看来,诗的好处,有口里说不出来的意思,想去却是逼真的。有似乎无理的,想去竟是有理有情的。”黛玉笑道:“这话有了些意思,但不知你从何处见得?”香菱笑道: “我看他《塞上》一首,那一联云:‘大漠孤烟直,长河落日圆。’想来烟如何直?日自然是圆的:这‘直’字似无理,‘圆’字似太俗。合上书一想,倒象是见了这景的。若说再找两个字换这两个,竟再找不出两个字来。再还有‘日落江湖白,潮来天地青’,这‘白’‘青’两个字也似无理。想来,必得这两个字才形容得尽,念在嘴里倒象有几千斤重的一个橄榄。还有‘渡头余落日,墟里上孤烟’,这‘余’字和‘上’字,难为他怎么想来!我们那年上京来,那日下晚便湾住船,岸上又没有人,只有几棵树,远远的几家人家作晚饭,那个烟竟是碧青,连云直上。谁知我昨日晚上读了这两句,倒象我又到了那个地方去了。”
\end{parag}


\begin{parag}


    正说著,宝玉和探春也来了,也都入坐听他讲诗。宝玉笑道:“既是这样,也不用看诗。会心处不在多,听你说了这两句,可知三昧你已得了。”黛玉笑道: “你说他这‘上孤烟’ 好,你还不知他这一句还是套了前人的来。我给你这一句瞧瞧,更比这个淡而现成。”说著便把陶渊明的“暧暧远人村,依依墟里烟”翻了出来,递与香菱。香菱瞧了,点头叹赏,笑道:“原来‘上’字是从‘依依’两个字上化出来的。”宝玉大笑道:“你已得了,不用再讲,越发倒学杂了。你就作起来,必是好的。”探春笑道:“明儿我补一个柬来,请你入社。”香菱笑道:“姑娘何苦打趣我,我不过是心里羡慕,才学著顽罢了。”探春黛玉都笑道:“谁不是顽?难道我们是认真作诗呢!若说我们认真成了诗,出了这园子,把人的牙还笑倒了呢。”宝玉道:“这也算自暴自弃了。前日我在外头和相公们商议画儿,他们听见咱们起诗社,求我把稿子给他们瞧瞧。我就写了几首给他们看看,谁不真心叹服。他们都抄了刻去了。”探春黛玉忙问道:“这是真话么?”宝玉笑道:“说谎的是那架上的鹦哥。”黛玉探春听说,都道:“你真真胡闹!且别说那不成诗,便是成诗,我们的笔墨也不该传到外头去。”宝玉道:“这怕什么!古来闺阁中的笔墨不要传出去,如今也没有人知道了。”说著,只见惜春打发了入画来请宝玉,宝玉方去了。香菱又逼著黛玉换出杜律来,又央黛玉探春二人:“出个题目,让我诌去,诌了来,替我改正。” 黛玉道:“昨夜的月最好,我正要诌一首,竟未诌成,你竟作一首来。‘十四寒’的韵,由你爱用那几个字去。”
\end{parag}


\begin{parag}


    香菱听了,喜的拿回诗来,又苦思一回作两句诗,又舍不得杜诗,又读两首。如此茶饭无心,坐卧不定。宝钗道:“何苦自寻烦恼。都是颦儿引的你,我和他算账去。你本来呆头呆脑的,再添上这个,越发弄成个呆子了。”\begin{note}庚辰双行夹批:“呆头呆脑的”有趣之至!最恨野史有一百个女子皆曰“聪敏伶俐”,究竟看来,他行为也只平平。今以“呆”字为香菱定评,何等妩媚之至也。\end{note}香菱笑道:“好姑娘,别混我。”\begin{note}庚辰双行夹批:如闻如见。\end{note}一面说,一面作了一首,先与宝钗看。宝钗看了笑道:“这个不好,不是这个作法。你别怕臊,只管拿了给他瞧去,看他是怎么说。”香菱听了,便拿了诗找黛玉。黛玉看时,只见写道是:
\end{parag}

\begin{poem}
    \begin{pl}

        月挂中天夜色寒,清光皎皎影团团。
    \end{pl}
    \begin{pl}

        诗人助兴常思玩,野客添愁不忍观。
    \end{pl}
    \begin{pl}

        翡翠楼边悬玉镜,珍珠帘外挂冰盘。
    \end{pl}
    \begin{pl}

        良宵何用烧银烛,晴彩辉煌映画栏。
    \end{pl}
\end{poem}

\begin{parag}

    黛玉笑道:“意思却有,只是措词不雅。皆因你看的诗少,被他缚住了。把这首丢开,再作一首。只管放开胆子去作。”
\end{parag}


\begin{parag}


    香菱听了,默默的回来,越性连房也不入,只在池边树下,或坐在山石上出神,或蹲在地下抠土,来往的人都诧异。李纨、宝钗、探春、宝玉等听得此信,都远远的站在山坡上瞧看他。只见他皱一回眉,又自己含笑一回。宝钗笑道:“这个人定要疯了!昨夜嘟嘟哝哝直闹到五更天才睡下,没一顿饭的工夫天就亮了。我就听见他起来了,忙忙碌碌梳了头就找颦儿去。一回来了,呆了一日,作了一首又不好,这会子自然另作呢。”宝玉笑道:“这正是‘地灵人杰’,老天生人再不虚赋情性的。我们成日叹说可惜他这么个人竟俗了,谁知到底有今日。可见天地至公。”宝钗笑道:“你能够像他这苦心就好了,学什么有个不成的。”宝玉不答。
\end{parag}


\begin{parag}


    只见香菱兴兴头头的又往黛玉那边去了。探春笑道:“咱们跟了去,看他有些意思没有。”说著,一齐都往潇湘馆来。只见黛玉正拿著诗和他讲究。众人因问黛玉作的如何。黛玉道:“自然算难为他了,只是还不好。这一首过于穿凿了,还得另作。”众人因要诗看时,只见作道:
\end{parag}

\begin{poem}

    \begin{pl}

        非银非水映窗寒,试看晴空护玉盘。
    \end{pl}
    \begin{pl}

        淡淡梅花香欲染,丝丝柳带露初干。
    \end{pl}
    \begin{pl}

        只疑残粉涂金砌,恍若轻霜抹玉栏。
    \end{pl}
    \begin{pl}

        梦醒西楼人迹绝,余容犹可隔帘看。
    \end{pl}
\end{poem}
\begin{parag}

    宝钗笑道:“不象吟月了,月字底下添一个‘色’字倒还使得,你看句句倒是月色。这也罢了,原来诗从胡说来,再迟几天就好了。”香菱自为这首妙绝,听如此说,自己扫了兴,不肯丢开手,便要思索起来。因见他姊妹们说笑,便自己走至阶前竹下闲步,挖心搜胆,耳不旁听,目不别视。一时探春隔窗笑说道:“菱姑娘,你闲闲罢。”香菱怔怔答道:“‘闲’字是‘十五删’的,你错了韵了。”众人听了,不觉大笑起来。宝钗道:“可真是诗魔了。都是颦儿引的他!”黛玉笑道:“圣人说:‘诲人不倦。’他又来问我,我岂有不说之理。”李纨笑道:“咱们拉了他往四姑娘房里去,引他瞧瞧画儿,叫他醒一醒才好。”
\end{parag}


\begin{parag}


    说著,真个出来拉了他过藕香榭,至暖香坞中。惜春正乏倦,在床上歪著睡午觉,画缯立在壁间,用纱罩著。众人唤醒了惜春,揭纱看时,十停方有了三停。香菱见画上有几个美人,因指著笑道:“这一个是我们姑娘,那一个是林姑娘。”探春笑道:“凡会作诗的都画在上头,快学罢。”说著,顽笑了一回。
\end{parag}


\begin{parag}


    各自散后,香菱满心中还是想诗。至晚间对灯出了一回神,至三更以后上床卧下,两眼鳏鳏,直到五更方才朦胧睡去了。一时天亮,宝钗醒了,听了一听,他安稳睡了,心下想:“他翻腾了一夜,不知可作成了?这会子乏了,且别叫他。”正想著,只听香菱从梦中笑道:“可是有了,难道这一首还不好?”宝钗听了,又是可叹,又是可笑,连忙唤醒了他,问他:“得了什么?你这诚心都通了仙了。学不成诗,还弄出病来呢。”一面说,一面梳洗了,会同姊妹往贾母处来。原来香菱苦志学诗,精血诚聚,日间做不出,忽于梦中得了八句。梳洗已毕,便忙录出来,自己并不知好歹,便拿来又找黛玉。刚到沁芳亭,只见李纨与众姊妹方从王夫人处回来,宝钗正告诉他们说他梦中作诗说梦话。\begin{note}庚辰双行夹批:一部大书起是梦,宝玉情是梦,贾瑞淫又是梦,秦之家计长策又是梦,今作诗也是梦,一并“风月鉴” 亦从梦中所有,故“红楼梦”也。余今批评亦在梦中,特为梦中之人作此一大梦也。脂砚斋。\end{note}众人正笑,抬头见他来了,便都争著要诗看。且听下回分解。
\end{parag}

\begin{parag}

    \begin{note}蒙回末总批:一扇之微,而害人如此,其毒藏之者。故自无味,构求者更觉可笑。多少没天理处,全不自觉。可见,好爱之端,断不可生求古董于古坟,争盆景而荡产势。所以至可不慎诸。\end{note}
\end{parag}

