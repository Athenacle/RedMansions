\chap{八十四}{試文字寶玉始提親 探驚風賈環重結怨}



\begin{parag}
    卻說薛姨媽一時因被金桂這場氣慪得肝氣上逆,左肋作痛。寶釵明知是這個原故,也等不及醫生來看,先叫人去買了幾錢鉤藤來,濃濃的煎了一碗,給他母親吃了。又和秋菱給薛姨媽捶腿揉胸,停了一會兒,略覺安頓。這薛姨媽只是又悲又氣,氣的是金桂撒潑,悲的是寶釵有涵養,倒覺可憐。寶釵又勸了一回,不知不覺的睡了一覺,肝氣也漸漸平復了。寶釵便說道:“媽媽,你這種閒氣不要放在心上纔好。過幾天走的動了,樂得往那邊老太太姨媽處去說說話兒散散悶也好。家裏橫豎有我和秋菱照看着,諒他也不敢怎麼樣。”薛姨媽點點頭道:“過兩日看罷了。”
\end{parag}


\begin{parag}
    且說元妃疾愈之後,家中俱各喜歡。過了幾日,有幾個老公走來,帶着東西銀兩,宣貴妃娘娘之命,因家中省問勤勞,俱有賞賜。把對象銀兩一一交代清楚。賈赦賈政等稟明瞭賈母,一齊謝恩畢,太監吃了茶去了。大家回到賈母房中,說笑了一回。外面老婆子傳進來說:“小廝們來回道,那邊有人請大老爺說要緊的話呢。”賈母便向賈赦道:“你去罷。”賈赦答應着,退出來自去了。
\end{parag}


\begin{parag}
    這裏賈母忽然想起,和賈政笑道:“娘娘心裏卻甚實惦記着寶玉,前兒還特特的問他來着呢。”賈政陪笑道:“只是寶玉不大肯唸書,辜負了娘娘的美意。”賈母道:“我倒給他上了個好兒,說他近日文章都做上來了。”賈政笑道:“那裏能象老太太的話呢。”賈母道:“你們時常叫他出去作詩作文,難道他都沒作上來麼。小孩子家慢慢的教導他,可是人家說的,‘胖子也不是一口兒喫的’。”賈政聽了這話,忙陪笑道:“老太太說的是。”賈母又道:“提起寶玉,我還有一件事和你商量。如今他也大了,你們也該留神看一個好孩子給他定下。這也是他終身的大事。也別論遠近親戚,什麼窮啊富的,只要深知那姑娘的脾性兒好模樣兒周正的就好。”賈政道:“老太太吩咐的很是。但只一件,姑娘也要好,第一要他自己學好纔好,不然不稂不莠的,反倒耽誤了人家的女孩兒,豈不可惜。”賈母聽了這話,心裏卻有些不喜歡,便說道:“論起來,現放着你們作父母的,那裏用我去張心。但只我想寶玉這孩子從小兒跟着我,未免多疼他一點兒,耽誤了他成人的正事也是有的。只是我看他那生來的模樣兒也還齊整,心性兒也還實在,未必一定是那種沒出息的,必至遭踏了人家的女孩兒。也不知是我偏心,我看着橫豎比環兒略好些,不知你們看着怎麼樣。”幾句話說得賈政心中甚實不安,連忙陪笑道:“老太太看的人也多了,既說他好有造化的,想來是不錯的。只是兒子望他成人性兒太急了一點,或者竟和古人的話相反,倒是‘莫知其子之美’了。”一句話把賈母也慪笑了,衆人也都陪着笑了。賈母因說道:“你這會子也有了幾歲年紀,又居着官,自然越歷練越老成。”說到這裏,回頭瞅着邢夫人和王夫人笑道:“想他那年輕的時侯,那一種古怪脾氣,比寶玉還加一倍呢。直等娶了媳婦,才略略的懂了些人事兒。如今只抱怨寶玉,這會子我看寶玉比他還略體些人情兒呢。”說的邢夫人王夫人都笑了。因說道:“老太太又說起逗笑兒的話兒來了。”說着,小丫頭子們進來告訴鴛鴦:“請示老太太,晚飯伺侯下了。”賈母便問:“你們又咕咕唧唧的說什麼?”鴛鴦笑着回明瞭。賈母道:“那麼着,你們也都喫飯去罷,單留鳳姐兒和珍哥媳婦跟着我喫罷。”賈政及邢王二夫人都答應着,伺侯擺上飯來,賈母又催了一遍,才都退出各散。
\end{parag}


\begin{parag}
    卻說邢夫人自去了。賈政同王夫人進入房中。賈政因提起賈母方纔的話來,說道:“老太太這樣疼寶玉,畢竟要他有些實學,日後可以混得功名,纔好不枉老太太疼他一場,也不至糟踏了人家的女兒。”王夫人道:“老爺這話自然是該當的。”賈政因着個屋裏的丫頭傳出去告訴李貴:“寶玉放學回來,索性喫飯後再叫他過來,說我還要問他話呢。”李貴答應了“是”。至寶玉放了學剛要過來請安,只見李貴道:“二爺先不用過去。老爺吩咐了,今日叫二爺吃了飯再過去呢,聽見還有話問二爺呢。”寶玉聽了這話,又是一個悶雷。只得見過賈母,便回園喫飯。三口兩口喫完,忙漱了口,便往賈政這邊來。
\end{parag}


\begin{parag}
    賈政此時在內書房坐着,寶玉進來請了安,一旁侍立。賈政問道:“這幾日我心上有事,也忘了問你。那一日你說你師父叫你講一個月的書就要給你開筆,如今算來將兩個月了,你到底開了筆了沒有?”寶玉道:“才做過三次。師父說且不必回老爺知道,等好些再回老爺知道罷。因此這兩天總沒敢回。”賈政道:“是什麼題目?”寶玉道:“一個是《吾十有五而志於學》,一個是《人不知而不慍》,一個是《則歸墨》三字。”賈政道:“都有稿兒麼?”寶玉道:“都是做了抄出來師父又改的。”賈政道:“你帶了家來了還是在學房裏呢?”寶玉道:“在學房裏呢。”賈政道:“叫人取了來我瞧。”寶玉連忙叫人傳話與焙茗:“叫他往學房中去,我書桌子抽屜裏有一本薄薄兒竹紙本子,上面寫着‘窗課’兩字的就是,快拿來。”一回兒焙茗拿了來遞給寶玉。寶玉呈與賈政。賈政翻開看時,見頭一篇寫着題目是《吾十有五而志於學》。他原本破的是“聖人有志於學,幼而已然矣。”代儒卻將幼字抹去,明用“十五”。賈政道:“你原本‘幼’字便扣不清題目了。‘幼’字是從小起至十六以前都是‘幼’。這章書是聖人自言學問工夫與年俱進的話,所以十五,三十,四十,五十,六十,七十俱要明點出來,才見得到了幾時有這麼個光景,到了幾時又有那麼個光景。師父把你‘幼’字改了‘十五’,便明白了好些。”看到承題,那抹去的原本雲:“夫不志於學,人之常也。”賈政搖頭道:“不但是孩子氣,可見你本性不是個學者的志氣。”又看後句”聖人十五而志之,不亦難乎”,說道:“這更不成話了。”然後看代儒的改本雲:“夫人孰不學,而志於學者卒鮮。此聖人所爲自信於十五時歟。”便問“改的懂得麼?”寶玉答應道:“懂得。”又看第二藝,題目是《人不知而不慍》,便先看代儒的改本雲:“不以不知而慍者,終無改其說樂矣。”方覷着眼看那抹去的底本,說道:“你是什麼?——‘能無慍人之心,純乎學者也。’上一句似單做了‘而不慍’三個字的題目,下一句又犯了下文君子的分界。必如改筆才合題位呢。且下句找清上文,方是書理。須要細心領略。”寶玉答應着。賈政又往下看,“夫不知,未有不慍者也,而竟不然。是非由說而樂者,曷克臻此。”原本末句“非純學者乎。”賈政道:“這也與破題同病的。這改的也罷了,不過清楚,還說得去。”第三藝是《則歸墨》,賈政看了題目,自己揚着頭想了一想,因問寶玉道:“你的書講到這裏了麼?”寶玉道:“師父說,《孟子》好懂些,所以倒先講《孟子》,大前日纔講完了。如今講‘上論語’呢。”賈政因看這個破承倒沒大改。破題雲:“言於舍楊之外,若別無所歸者焉。”賈政道:“第二句倒難爲你。”’夫墨,非欲歸者也,而墨之言已半天下矣,則舍楊之外,欲不歸於墨,得乎?”賈政道:“這是你做的麼?”寶玉答應道:“是。”賈政點點頭兒,因說道:“這也並沒有什麼出色處,但初試筆能如此,還算不離。前年我在任上時,還出過《惟士爲能》這個題目。那些童生都讀過前人這篇,不能自出心裁,每多抄襲。你念過沒有?”寶玉道:“也念過。”賈政道:“我要你另換個主意,不許雷同了前人,只做個破題也使得。”寶玉只得答應着,低頭搜索枯腸。賈政揹着手,也在門口站著作想。只見一個小小廝往外飛走,看見賈政,連忙側身垂手站住。賈政便問道:“作什麼?”小廝回道:“老太太那邊姨太太來了,二奶奶傳出話來,叫預備飯呢。”賈政聽了,也沒言語。那小廝自去了。
\end{parag}


\begin{parag}
    誰知寶玉自從寶釵搬回家去,十分想念,聽見薛姨媽來了,只當寶釵同來,心中早已忙了,便乍着膽子回道:“破題倒作了一個,但不知是不是。”賈政道:“你念來我聽。”寶玉念道:“天下不皆士也,能無產者亦僅矣。”賈政聽了,點着頭道:“也還使得。以後作文,總要把界限分清,把神理想明白了再去動筆。你來的時侯老太太知道不知道?”寶玉道:“知道的。”賈政道:“既如此,你還到老太太處去罷。”寶玉答應了個“是”,只得拿捏着慢慢的退出,剛過穿廊月洞門的影屏,便一溜煙跑到老太太院門口。急得焙茗在後頭趕着叫:“看跌倒了!老爺來了。”寶玉那裏聽得見。剛進得門來,便聽見王夫人,鳳姐,探春等笑語之聲。
\end{parag}


\begin{parag}
    丫鬟們見寶玉來了,連忙打起簾子,悄悄告訴道:“姨太太在這裏呢。”寶玉趕忙進來給薛姨媽請安,過來纔給賈母請了晚安。賈母便問:“你今兒怎麼這早晚才散學?”寶玉悉把賈政看文章並命作破題的話述了一遍。賈母笑容滿面。寶玉因問衆人道:“寶姐姐在那裏坐着呢?”薛姨媽笑道:“你寶姐姐沒過來,家裏和香菱作活呢。”寶玉聽了,心中索然,又不好就走。只見說着話兒已擺上飯來,自然是賈母薛姨媽上坐,探春等陪坐。薛姨媽道:“寶哥兒呢?”賈母忙笑說道:“寶玉跟着我這邊坐罷。”寶玉連忙回道:“頭裏散學時李貴傳老爺的話,叫吃了飯過去。我趕着要了一碟菜,泡茶吃了一碗飯,就過去了。老太太和姨媽姐姐們用罷。”賈母道:“既這麼着,鳳丫頭就過來跟着我。你太太才說他今兒喫齋,叫他們自己喫去罷。”王夫人也道:“你跟着老太太姨太太喫罷,不用等我,我喫齋呢。”於是鳳姐告了坐,丫頭安了杯箸,鳳姐執壺斟了一巡,才歸坐。
\end{parag}


\begin{parag}
    大家喫着酒。賈母便問道:“可是才姨太太提香菱,我聽見前兒丫頭們說‘秋菱’,不知是誰,問起來才知道是他。怎麼那孩子好好的又改了名字呢?”薛姨媽滿臉飛紅,嘆了一口氣道:“老太太再別提起。自從蟠兒娶了這個不知好歹的媳婦,成日家咕咕唧唧,如今鬧的也不成個人家了。我也說過他幾次,他牛心不聽說,我也沒那麼大精神和他們盡着吵去,只好由他們去。可不是他嫌這丫頭的名兒不好改的。”賈母道:“名兒什麼要緊的事呢?”薛姨媽道:“說起來我也怪臊的,其實老太太這邊有什麼不知道的。他那裏是爲這名兒不好,聽見說他因爲是寶丫頭起的,他纔有心要改。”賈母道:“這又是什麼原故呢?”薛姨媽把手絹子不住的擦眼淚,未曾說,又嘆了一口氣,道:“老太太還不知道呢,這如今媳婦子專和寶丫頭慪氣。前日老太太打發人看我去,我們家裏正鬧呢。”賈母連忙接着問道:“可是前兒聽見姨太太肝氣疼,要打發人看去,後來聽見說好了,所以沒着人去。依我,勸姨太太竟把他們別放在心上。再者,他們也是新過門的小夫妻,過些時自然就好了。我看寶丫頭性格兒溫厚和平,雖然年輕,比大人還強幾倍。前日那小丫頭子回來說,我們這邊還都讚歎了他一會子。都象寶丫頭那樣心胸兒脾氣兒,真是百裏挑一的。不是我說句冒失話,那給人家做了媳婦兒,怎麼叫公婆不疼,家裏上上下下的不賓服呢。”寶玉頭裏已經聽煩了,推故要走,及聽見這話,又坐了呆呆的往下聽。薛姨媽道:“不中用。他雖好,到底是女孩兒家。養了蟠兒這個糊塗孩子,真真叫我不放心,只怕在外頭喝點子酒,鬧出事來。幸虧老太太這裏的大爺二爺常和他在一塊兒,我還放點兒心。”寶玉聽到這裏,便接口道:“姨媽更不用懸心。薛大哥相好的都是些正經買賣大客人,都是有體面的,那裏就鬧出事來。”薛姨媽笑道:“依你這樣說,我敢只不用操心了。”說話間,飯已喫完。寶玉先告辭了,晚間還要看書,便各自去了。
\end{parag}


\begin{parag}
    這裏丫頭們剛捧上茶來,只見琥珀走過來向賈母耳朵旁邊說了幾句,賈母便向鳳姐兒道:“你快去罷,瞧瞧巧姐兒去罷。”鳳姐聽了,還不知何故,大家也怔了。琥珀遂過來向鳳姐道:“剛纔平兒打發小丫頭子來回二奶奶,說巧姐身上不大好,請二奶奶忙着些過來纔好呢。”賈母因說道:“你快去罷,姨太太也不是外人。”鳳姐連忙答應,在薛姨媽跟前告了辭。又見王夫人說道:“你先過去,我就去。小孩子家魂兒還不全呢,別叫丫頭們大驚小怪的,屋裏的貓兒狗兒,也叫他們留點神兒。盡着孩子貴氣,偏有這些瑣碎。”鳳姐答應了,然後帶了小丫頭回房去了。
\end{parag}


\begin{parag}
    這裏薛姨媽又問了一回黛玉的病。賈母道:“林丫頭那孩子倒罷了,只是心重些,所以身子就不大很結實了。要賭靈性兒,也和寶丫頭不差什麼,要賭寬厚待人裏頭,卻不濟他寶姐姐有耽待,有儘讓了。”薛姨媽又說了兩句閒話兒,便道:“老太太歇着罷。我也要到家裏去看看,只剩下寶丫頭和香菱了。打那麼同着姨太太看看巧姐兒。”賈母道:“正是。姨太太上年紀的人看看是怎麼不好,說給他們,也得點主意兒。”薛姨媽便告辭,同着王夫人出來,往鳳姐院裏去了。
\end{parag}


\begin{parag}
    卻說賈政試了寶玉一番,心裏卻也喜歡,走向外面和那些門客閒談。說起方纔的話來,便有新近到來最善大棋的一個王爾調名作梅的說道:“據我們看來,寶二爺的學問已是大進了。”賈政道:“那有進益,不過略懂得些罷咧,‘學問’兩個字早得很呢。”詹光道:“這是老世翁過謙的話。不但王大兄這般說,就是我們看,寶二爺必定要高發的。”賈政笑道:“這也是諸位過愛的意思。”那王爾調又道:“晚生還有一句話,不揣冒昧,和老世翁商議。”賈政道:“什麼事?”王爾調陪笑道:“也是晚生的相與,做過南韶道的張大老爺家有一位小姐,說是生得德容功貌俱全,此時尚未受聘。他又沒有兒子,家資鉅萬。但是要富貴雙全的人家,女婿又要出衆,才肯作親。晚生來了兩個月,瞧着寶二爺的人品學業,都是必要大成的。老世翁這樣門楣,還有何說。若晚生過去,包管一說就成。”賈政道:“寶玉說親卻也是年紀了,並且老太太常說起。但只張大老爺素來尚未深悉。”詹光道:“王兄所提張家,晚生卻也知道。況和大老爺那邊是舊親,老世翁一問便知。”賈政想了一回,道:“大老爺那邊不曾聽得這門親戚。”詹光道:“老世翁原來不知,這張府上原和邢舅太爺那邊有親的。”賈政聽了,方知是邢夫人的親戚。坐了一回,進來了,便要同王夫人說知,轉問邢夫人去。誰知王夫人陪了薛姨媽到鳳姐那邊看巧姐兒去了。那天已經掌燈時候,薛姨媽去了,王夫人才過來了。賈政告訴了王爾調和詹光的話,又問巧姐兒怎麼了。王夫人道:“怕是驚風的光景。”賈政道:“不甚利害呀?”王夫人道:“看着是搐風的來頭,只還沒搐出來呢。”賈政聽了,便不言語,各自安歇,一宿晚景不提。
\end{parag}


\begin{parag}
    卻說次日邢夫人過賈母這邊來請安,王夫人便提起張家的事,一面回賈母,一面問邢夫人。邢夫人道:“張家雖系老親,但近年來久已不通音信,不知他家的姑娘是怎麼樣的。倒是前日孫親家太太打發老婆子來問安,卻說起張家的事,說他家有個姑娘,託孫親家那邊有對勁的提一提。聽見說只這一個女孩兒,十分嬌養,也識得幾個字,見不得大陣仗兒,常在房中不出來的。張大老爺又說,只有這一個女孩兒,不肯嫁出去,怕人家公婆嚴,姑娘受不得委屈,必要女婿過門贅在他家,給他料理些家事。”賈母聽到這裏,不等說完便道:“這斷使不得。我們寶玉別人伏侍他還不夠呢,倒給人家當家去。”邢夫人道:“正是老太太這個話。”賈母因向王夫人道:“你回來告訴你老爺,就說我的話,這張家的親事是作不得的。”王夫人答應了。賈母便問:“你們昨日看巧姐兒怎麼樣?頭裏平兒來回我說很不大好,我也要過去看看呢。”邢王二夫人道:“老太太雖疼他,他那裏耽的住。”賈母道:“卻也不止爲他,我也要走動走動,活活筋骨兒。”說着,便吩咐:“你們喫飯去罷,回來同我過去。”邢王二夫人答應着出來,各自去了。
\end{parag}


\begin{parag}
    一時吃了飯,都來陪賈母到鳳姐房中。鳳姐連忙出來接了進去。賈母便問巧姐兒到底怎麼樣。鳳姐兒道:“只怕是搐風的來頭。”賈母道:“這麼着還不請人趕着瞧!”鳳姐道:“已經請去了。”賈母因同邢王二夫人進房來看,只見奶子抱着,用桃紅綾子小綿被兒裹着,臉皮趣青,眉梢鼻翅微有動意。賈母同邢王二夫人看了看,便出外間坐下。正說間,只見一個小丫頭回鳳姐道:“老爺打發人問姐兒怎麼樣。”鳳姐道:“替我回老爺,就說請大夫去了。一會兒開了方子,就過去回老爺。”賈母忽然想起張家的事來,向王夫人道:“你該就去告訴你老爺,省得人家去說了回來又駁回。”又問邢夫人道:“你們和張家如今爲什麼不走了?”邢夫人因又說:“論起那張家行事,也難和咱們作親,太嗇克,沒的玷辱了寶玉。”鳳姐聽了這話,已知八九,便問道:“太太不是說寶兄弟的親事?”邢夫人道:“可不是麼。”賈母接着因把剛纔的話告訴鳳姐。鳳姐笑道:“不是我當着老祖宗太太們跟前說句大膽的話,現放着天配的姻緣,何用別處去找。”賈母笑問道:“在那裏?”鳳姐道:“一個‘寶玉’,一個‘金鎖’,老太太怎麼忘了?”賈母笑了一笑,因說:“昨日你姑媽在這裏,你爲什麼不提?”鳳姐道:“老祖宗和太太們在前頭,那裏有我們小孩子家說話的地方兒。況且姨媽過來瞧老祖宗,怎麼提這些個,這也得太太們過去求親纔是。”賈母笑了,邢王二夫人也都笑了。賈母因道:“可是我背晦了。”
\end{parag}


\begin{parag}
    說着人回:“大夫來了。”賈母便坐在外間,邢王二夫人略避。那大夫同賈璉進來,給賈母請了安,方進房中。看了出來,站在地下躬身回賈母道:“妞兒一半是內熱,一半是驚風。須先用一劑發散風痰藥,還要用四神散纔好,因病勢來得不輕。如今的牛黃都是假的,要找真牛黃方用得。”賈母道了乏,那大夫同賈璉出去開了方子,去了。鳳姐道:“人蔘家裏常有,這牛黃倒怕未必有,外頭買去,只是要真的纔好。”王夫人道:“等我打發人到姨太太那邊去找找。他家蟠兒是向與那些西客們做買賣,或者有真的也未可知。我叫人去問問。”正說話間,衆姊妹都來瞧來了,坐了一回,也都跟着賈母等去了。
\end{parag}


\begin{parag}
    這裏煎了藥給巧姐兒灌了下去,只聽喀的一聲,連藥帶痰都吐出來,鳳姐才略放了一點兒心。只見王夫人那邊的小丫頭拿着一點兒的小紅紙包兒說道:“二奶奶,牛黃有了。太太說了,叫二奶奶親自把分兩對準了呢。”鳳姐答應着接過來,便叫平兒配齊了真珠,冰片,硃砂,快熬起來。自己用戥子按方稱了,攙在裏面,等巧姐兒醒了好給他喫。只見賈環掀簾進來說:“二姐姐,你們巧姐兒怎麼了?媽叫我來瞧瞧他。”鳳姐見了他母子便嫌,說:“好些了。你回去說,叫你們姨娘想着。”那賈環口裏答應,只管各處瞧看。看了一回,便問鳳姐兒道:“你這裏聽的說有牛黃,不知牛黃是怎麼個樣兒,給我瞧瞧呢。”鳳姐道:“你別在這裏鬧了,妞兒纔好些。那牛黃都煎上了。”賈環聽了,便去伸手拿那吊子瞧時,豈知措手不及,沸的一聲,吊子倒了,火已潑滅了一半。賈環見不是事,自覺沒趣,連忙跑了。鳳姐急的火星直爆,罵道:“真真那一世的對頭冤家!你何苦來還來使促狹!從前你媽要想害我,如今又來害妞兒。我和你幾輩子的仇呢!”一面罵平兒不照應。正罵着,只見丫頭來找賈環。鳳姐道:“你去告訴趙姨娘,說他操心也太苦了。巧姐兒死定了,不用他惦着了!”平兒急忙在那裏配藥再熬,那丫頭摸不着頭腦,便悄悄問平兒道:“二奶奶爲什麼生氣?”平兒將環哥弄倒藥吊子說了一遍。丫頭道:“怪不得他不敢回來,躲了別處去了。這環哥兒明日還不知怎麼樣呢。平姐姐,我替你收拾罷。”平兒說:“這倒不消。幸虧牛黃還有一點,如今配好了,你去罷。”丫頭道:“我一準回去告訴趙姨奶奶,也省得他天天說嘴。”
\end{parag}


\begin{parag}
    丫頭回去果然告訴了趙姨娘。趙姨娘氣的叫:“快找環兒!”環兒在外間屋子裏躲着,被丫頭找了來。趙姨娘便罵道:“你這個下作種子!你爲什麼弄灑了人家的藥,招的人家咒罵。我原叫你去問一聲,不用進去,你偏進去,又不就走,還要虎頭上捉蝨子。你看我回了老爺,打你不打!”這裏趙姨娘正說着,只聽賈環在外間屋子裏更說出些驚心動魄的話來。未知何言,下回分解。
\end{parag}