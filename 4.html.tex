\chap{四}{薄命女偏逢薄命郎 葫芦僧乱判葫芦案}

\begin{parag}
    \begin{note}蒙回前:阴阳交结变无伦,幻境生时即是真。秋月春花谁不见,朝晴暮雨自何因。心肝一点劳牵恋,可意偏长遇喜嗔。我爱世缘随分定,至诚相感作痴人。
        请君著眼护官符,把笔悲伤说世途。作者泪痕同我泪,燕山仍旧窦公无。\end{note}
\end{parag}


\begin{parag}
    题曰:
\end{parag}


\begin{poem}
    \begin{pl}捐躯报国恩,\end{pl}
    \begin{pl}未报躯犹在。\end{pl}

    \begin{pl}眼底物多情,\end{pl}
    \begin{pl}君恩或可待。\end{pl}
\end{poem}


\begin{parag}
    却说黛玉同姊妹们至王夫人处,见王夫人与兄嫂处的来使计议家务,又说姨母家遭人命官司等语。因见王夫人事情冗杂,姊妹们遂出来,至寡嫂李氏房中来了。
\end{parag}


\begin{parag}
    原来这李氏即贾珠之妻。\begin{note}甲戌侧:起笔写薛家事,他偏写宫裁,是结黛玉,明李纨本末,又在人意料之外。\end{note}珠虽夭亡,幸存一子,取名贾兰,今方五岁,已入学攻书。这李氏亦系金陵名宦之女,父名李守中,\begin{note}甲戌侧:妙!盖云人能以理自守,安得为情所陷哉!\end{note}曾为国子监祭酒,族中男女无有不诵诗读书者。\begin{note}甲戌侧:未出李纨,先伏下李纹、李绮。\end{note}至李守中继承以来,便说“女子无才便有\begin{note}甲戌侧:“有”字改得好。\end{note}德”,故生了李氏时,便不十分令其读书,只不过将些《女四书》、《列女传》、《贤媛集》等三四种书,使他认得几个字,记得前朝这几个贤女便罢了,却只以纺绩井臼为要,因取名为李纨,字宫裁。\begin{note}甲戌侧:一洗小说窠臼俱尽,且命名字,亦不见红香翠玉恶俗。\end{note}因此这李纨虽青春丧偶,居家处膏粱锦绣之中,竟如槁木死灰一般,\begin{note}甲戌侧:此时处此境,最能越理生事,彼竟不然,实罕见者。\end{note}一概无见无闻,唯知侍亲养子,外则陪侍小姑等针黹诵读而已。\begin{note}甲戌侧:一段叙出李纨,不犯熙凤。\end{note}今黛玉虽客寄于斯,日有这般姐妹相伴,除老父外,余者也都无庸虑及了。\begin{note}甲戌侧:仍是从黛玉身上写来,以上了结住黛玉,复找前文。\end{note}
\end{parag}


\begin{parag}
    如今且说雨村,因补授了应天府,一下马就有一件人命官司详至案下,乃是两家争买一婢,各不相让,以至殴伤人命。彼时雨村即传原告之人来审。那原告道: “被殴死者乃小人之主人。因那日买了一个丫头,不想是拐子拐来卖的。这拐子先已得了我家的银子,我家小爷原说第三日方是好日子,再接入门。\begin{note}甲戌侧:所谓“迟则有变”,往往世人因不经之谈误却大事。\end{note}这拐子便又悄悄的卖与薛家,被我们知道了,去找拿卖主,夺取丫头。无奈薛家原系金陵一霸, 胁仗势,众豪奴将我小主人竟打死了。凶身主仆已皆逃走,无影无踪,只剩了几个局外之人。小人告了一年的状,竟无人作主。望大老爷拘拿凶犯,剪恶除凶,以救孤寡,死者感戴天地之恩不尽!”
\end{parag}


\begin{parag}
    雨村听了大怒道:“岂有这样放屁的事!打死人命就白白的走了,再拿不来的?”因发签差公人立刻将凶犯族中人拿来拷问,令他们实供藏在何处,一面再动海捕文书。正要发签时,只见案边立的一个门子,使眼色儿不令他发签。雨村心下甚为疑怪,\begin{note}甲戌侧:原可疑怪,余亦疑怪。\end{note}只得停了手。即时退堂,至密室,侍从皆退去,只留门子服侍。这门子忙上来请安,笑问:“老爷一向加官进禄,八九年来就忘了我了?”\begin{note}甲戌侧:语气傲慢,怪甚!\end{note}雨村道:“却十分面善得紧,只是一时想不起来。”那门子笑道:“老爷真是贵人多忘事,把出身之地竟忘了,\begin{note}甲戌侧:刹心语。自招其祸,亦因夸能恃才也。\end{note}不记当年葫芦庙里之事?”雨村听了,如雷震一惊,\begin{note}甲戌侧:余亦一惊,但不知门子何知,尤为怪甚。\end{note}方想起往事。原来这门子本是葫芦庙内一个小沙弥,因被火之后,无处安身,欲投别庙去修行,又耐不得清凉景况,因想这件生意倒还轻省热闹,\begin{note}甲戌侧:新鲜字眼。\end{note}遂趁年纪蓄了发,充了门子。\begin{note}甲戌侧:一路奇奇怪怪,调侃世人,总在人意臆之外。\end{note}雨村那里料得是他,便忙携手笑道:“原来是故人。”\begin{note}甲戌侧:妙称!全是假态。\end{note}又让坐了好谈。\begin{note}甲戌侧:假极!\end{note}这门子不敢坐。雨村笑道:“贫贱之交不可忘,\begin{note}甲戌侧:全是奸险小人态度,活现活跳。\end{note}你我故人也,二则此系私室,既欲长谈,岂有不坐之理?”这门子听说,方告了座,斜签著坐了。
\end{parag}


\begin{parag}
    雨村因问方才何故有不令发签之意。这门子道:“老爷既荣任到这一省,难道就没抄一张本省‘护官符’\begin{note}甲戌侧:可对“聚宝盆”,一笑。三字从来未见,奇之至!\end{note}来不成?”雨村忙问:“何为‘护官符’?\begin{note}甲戌侧:余亦欲问。\end{note}我竟不知。”门子道:“这还了得!连这个不知,怎能作得长远!\begin{note}甲戌侧:骂得爽快!\end{note}如今凡作地方官者,皆有一个私单,上面写的是本省最有权有势,极富极贵的大乡绅名姓,各省皆然,倘若不知,一时触犯了这样的人家,不但官爵,只怕连性命还保不成呢!\begin{note}甲戌侧:可怜可叹,可恨可气,变作一把眼泪也。\end{note}所以绰号叫作‘护官符’。\begin{note}甲戌侧:奇甚趣甚,如何想来?\end{note}方才所说的这薛家,老爷如何惹他!他这件官司并无难断之处,皆因都碍著情分面上,所以如此。”一面说,一面从顺袋中取出一张抄写的‘护官符’来,递与雨村,看时,上面皆是本地大族名宦之家的谚俗口碑。其口碑排写得明白,下面所注的皆是自始祖官爵并房次。石头亦曾抄写了一张,\begin{note}甲戌侧:忙中闲笔用得好。\end{note}今据石上所抄云:
\end{parag}


\begin{poem}
    \begin{pl}贾不假,白玉为堂金作马。\end{pl}
    \begin{note}甲戌侧:宁国、荣国二公之后,共二十房分,除宁、荣亲派八房在都外,现原籍住者十二房。\end{note}

    \begin{pl}阿房宫,三百里,住不下金陵一个史。\end{pl}
    \begin{note}甲戌侧:保龄侯尚书令史公之后,房分共十八。都中现住者十房,原籍现居八房。\end{note}

    \begin{pl}丰年好大雪\end{pl}\begin{note}甲夹:隐“薛”字。\end{note} \begin{pl},珍珠如土金如铁。\end{pl}

    \begin{note}甲戌侧:紫薇舍人薛公之后,现领内府帑银行商,共八房分。\end{note}

    \begin{pl}东海缺少白玉床,龙王来请金陵王。\end{pl}\begin{note}甲戌侧:都太尉统制县伯王公之后,共十二房。都中二房,馀皆在籍。\end{note}

\end{poem}


\begin{parag}
    雨村犹未看完,\begin{note}甲戌眉:妙极!若只有此四家,则死板不活,若再有两家,又觉累赘,故如此断法。\end{note}忽听传点,人报:“王老爷来拜。”雨村听说,忙具衣冠出去迎接。\begin{note}甲戌侧:横云断岭法,是板定大章法。\end{note}有顿饭工夫,方回来细问。这门子道:“这四家皆连络有亲,一损皆损,一荣皆荣,扶持遮饰,俱有照应的。\begin{note}甲戌侧:早为下半部伏根。\end{note}今告打死人之薛,就系丰年大雪之‘雪’也。也不单靠这三家,他的世交亲友在都在外者,本亦不少。老爷如今拿谁去?”雨村听如此说,便笑问门子道:“如你这样说来,却怎么了结此案?你大约也深知这凶犯躲的方向了?”
\end{parag}


\begin{parag}
    门子笑道:“不瞒老爷说,不但这凶犯的方向我知道,一并这拐卖之人\begin{note}甲戌侧:斯何人也。\end{note}我也知道,死鬼买主也深知道。待我细说与老爷听:这个被打之死鬼,乃是本地一个小乡绅之子,名唤冯渊,\begin{note}甲戌侧:真真是冤孽相逢。\end{note}自幼父母早亡,又无兄弟,只他一个人守著些薄产过日子。长到十八九岁上,酷爱男风,最厌女子。\begin{note}甲戌侧:最厌女子,仍为女子丧生,是何等大笔!不是写冯渊,正是写英莲。\end{note}这也是前生冤孽,可巧\begin{note}甲戌侧:善善恶恶,多从可巧而来,可畏可怕。\end{note}遇见这拐子卖丫头,他便一眼看上了这丫头,立意买来作妾,立誓再不交结男子,\begin{note}甲戌侧:谚云:“人若改常,非病即亡。”信有之乎?\end{note}也不再娶第二个了,\begin{note}甲戌侧:虚写一个情种。\end{note}所以三日后方过门。谁晓这拐子又偷卖与薛家,他意欲卷了两家的银子,再逃往他省。谁知又不曾走脱,两家拿住,打了个臭死,都不肯收银,只要领人。那薛家公子岂是让人的,便喝著手下人一打,将冯公子打了个稀烂,抬回家去三日死了。这薛公子原是早已择定日子上京去的,头起身两日前,就偶然遇见这丫头,意欲买了就进京的,谁知闹出这事来。既打了冯公子,夺了丫头,他便没事人一般,只管带了家眷走他的路。他这里自有兄弟奴仆在此料理,也并非为此些些小事值得他一逃走的。\begin{note}甲戌侧:妙极!人命视为些些小事,总是刻画阿呆耳。\end{note}这且别说,老爷你当被卖之丫头是谁?”\begin{note}甲戌侧:问得又怪。\end{note}雨村笑道:“我如何得知?”门子冷笑道:“这人算来还是老爷的大恩人呢!他就是葫芦庙旁住的甄老爷的小姐,名唤英莲的。”\begin{note}甲戌侧:至此一醒。\end{note}雨村罕然道:“原来就是他!闻得养至五岁被人拐去,却如今才来卖呢?”
\end{parag}


\begin{parag}
    门子道:“这一种拐子单管偷拐五六岁的儿女,养在一个僻静之处,到十一二岁,度其容貌,带至他乡转卖。当日这英莲,我们天天哄他顽耍,虽隔了七八年,如今十二三岁的光景,其模样虽然出脱得齐整好些,然大概相貌,自是不改,熟人易认。况且他眉心中原有米粒大小的一点胭脂痣,从胎里带来的,\begin{note}甲戌侧:宝钗之热,黛玉之怯,悉从胎中带来。今英莲有痣,其人可知矣。\end{note}所以我却认得。偏生这拐子又租了我的房舍居住,那日拐子不在家,我也曾问他。他是被拐子打怕了的,\begin{note}甲戌侧:可怜!\end{note}万不敢说,只说拐子系他亲爹,因无钱偿债,故卖他。我又哄之再四,他又哭了,只说:‘我不记得小时之事!’这可无疑了。那日冯公子相看了,兑了银子,拐子醉了,他自叹道:‘我今日罪孽可满了!’后又听见冯公子令三日之后过门,他又转有忧愁之态。我又不忍其形景,等拐子出去,又命内人去解释他:‘这冯公子必待好日期来接,可知必不以丫鬟相看。况他是个绝风流人品,家里颇过得,素习又最厌恶堂客,今竟破价买你,后事不言可知。只耐得三两日,何必忧闷!’他听如此说,方才略解忧闷,自为从此得所。谁料天下竟有这等不如意事,\begin{note}甲戌侧:可怜真可怜!一篇《薄命赋》,特出英莲。\end{note}第二日,他偏又卖与薛家。若卖与第二个人还好,这薛公子的混名人称‘呆霸王’,最是天下第一个弄性尚气的人,而且使钱如土,\begin{note}甲戌侧:世路难行钱作马。\end{note}遂打了个落花流水,生拖死拽,把个英莲拖去,如今也不知死活。\begin{note}甲戌侧:为英莲留后步。\end{note}这冯公子空喜一场,一念未遂,反花了钱,送了命,岂不可叹!”\begin{note}甲戌眉:又一首《薄命叹》。英、冯二人一段小悲欢幻境从葫芦僧口中补出,省却闲文之法也。所谓“美中不足,好事多魔”,先用冯渊作一开路之人。\end{note}
\end{parag}


\begin{parag}
    雨村听了,亦叹道:“这也是他们的 跽遭遇,亦非偶然。不然这冯渊如何偏只看准了这英莲?这英莲受了拐子这几年折磨,才得了个头路,且又是个多情的,若能聚合了,倒是件美事,偏又生出这段事来。这薛家纵比冯家富贵,想其为人,自然姬妾众多,淫佚无度,未必及冯渊定情于一人者。这正是梦幻情缘,恰遇一对薄命儿女。\begin{note}甲戌眉:使雨村一评,方补足上半回之题目。所谓此书有繁处愈繁,省中愈省;又有不怕繁中繁,只有繁中虚;不畏省中省,只要省中实。此则省中实也。\end{note}且不要议论他,只目今这官司,如何剖断才好?”门子笑道:“老爷当年何其明决,今日何反成了个没主意的人了!小的闻得老爷补升此任,亦系贾府王府之力,此薛蟠即贾府之亲,老爷何不顺水行舟,作个整人情,将此案了结,日后也好去见贾府王府。”雨村道:“你说的何尝不是。\begin{note}甲戌侧:可发一长叹。这一句已见奸雄,全是假。\end{note}但事关人命,蒙皇上隆恩,起复委用,\begin{note}甲戌侧:奸雄。\end{note}实是重生再造,正当殚心竭力图报之时,\begin{note}甲戌侧:奸雄。\end{note}岂可因私而废法?\begin{note}甲戌侧:奸雄。\end{note}是我实不能忍为者。”\begin{note}甲戌侧:全是假。\end{note}门子听了,冷笑道:“老爷说的何尝不是大道理,但只是如今世上是行不去的。岂不闻古人有云‘大丈夫相时而动’,又曰‘趋吉避凶者为君子’。\begin{note}甲戌侧:近时错会书意者多多如此。\end{note}依老爷这一说,不但不能报效朝廷,亦且自身不保,还要三思为妥。”
\end{parag}


\begin{parag}
    雨村低了半日头,\begin{note}甲戌侧:奸雄欺人。\end{note}方说道:“依你怎么样?”门子道:“小人已想了一个极好的主意在此:老爷明日坐堂,只管虚张声势,动文书发签拿人。原凶自然是拿不来的,原告固是定要将薛家族中及奴仆人等拿几个来拷问。小的在暗中调停,令他们报个暴病身亡,令族中及地方上共递一张保呈,老爷只说善能扶鸾请仙,堂上设下乩坛,令军民人等只管来看。老爷就说:‘乩仙批了,死者冯渊与薛蟠原因夙孽相逢,今狭路既遇,原应了结。薛蟠今已得了无名之症,\begin{note}甲戌侧:“无名之症”却是病之名,而反曰“无”,妙极!\end{note}被冯魂追索已死。其祸皆因拐子某人而起,拐之人原系某乡某姓人氏,按法处治,余不略及’等语。小人暗中嘱托拐子,令其实招。众人见乩仙批语与拐子相符,余者自然也都不虚了。薛家有的是钱,老爷断一千也可,五百也可,与冯家作烧埋之费。那冯家也无甚要紧的人,不过为的是钱,见有了这个银子,想来也就无话了。老爷细想此计如何?”雨村笑道:“不妥,不妥。\begin{note}甲戌侧:奸雄欺人。\end{note}等我再 斟酌斟酌,或可压服口声。”二人计议,天色已晚,别无话说。
\end{parag}


\begin{parag}
    至次日坐堂,勾取一应有名人犯,雨村详加审问,果见冯家人口稀疏,不过赖此欲多得些烧埋之费,\begin{note}甲戌侧:因此三四语收住,极妙!此则重重写来,轻轻抹去也。\end{note}薛家仗势倚情,偏不相让,故致颠倒未决。雨村便徇情枉法,胡乱判断了此案。\begin{note}甲戌侧:实注一笔,更好。不过是如此等事,又何用细写。可谓此书不敢干涉廊庙者,即此等处也,莫谓写之不到。盖作者立意写闺阁尚不暇,何能又及此等哉!\end{note}冯家得了许多烧埋银子,也就无甚话说了。\begin{note}甲戌眉:盖宝钗一家不得不细写者。若另起头绪,则文字死板,故仍只借雨村一人穿插出阿呆兄人命一事,且又带叙出英莲一向之行踪,并以后之归结,是以故意戏用“葫芦僧乱判”等字样,撰成半回,略一解颐,略一叹世,盖非有意讥刺仕途,实亦出人之闲文耳。甲戌眉:又注冯家一笔,更妥。可见冯家正不为人命,实赖此获利耳。故用“乱判”二字为题,虽曰不涉世事,或亦有微词耳。但其意实欲出宝钗,不得不做此穿插,故云此等皆非《石头记》之正文。\end{note}雨村断了此案,急忙作书信二封,与贾政并京营节度使王子腾,\begin{note}甲戌侧:随笔带出王家。\end{note}不过说“令甥之事已完,不必过虑”等语。此事皆由葫芦庙内之沙弥新门子所出,雨村又恐他对人说出当日贫贱时的事来,因此心中大不乐业。\begin{note}甲戌侧:瞧他写雨村如此,可知雨村终不是大英雄。\end{note}后来到底寻了个不是,远远的充发了他才罢。\begin{note}甲戌侧:至此了结葫芦庙文字。又伏下千里伏线。起用“葫芦”字样,收用“葫芦”字样,盖云一部书皆系葫芦提之意也,此亦系寓意处。\end{note}
\end{parag}


\begin{parag}
    当下言不著雨村。且说那买了英莲打死冯渊的薛公子,\begin{note}甲戌侧:本是立意写此,却不肯特起头绪,故意设出“乱判”一段戏文,其中穿插,至此却淡淡写来。\end{note}亦系金陵人氏,本是书香继世之家。只是如今这薛公子幼年丧父,寡母又怜他是个独根孤种,未免溺爱纵容,遂至老大无成,且家中有百万之富,现领著内帑钱粮,采办杂料。这薛公子学名薛蟠,表字文龙,五岁上就性情奢侈,言语傲慢。虽也上过学,不过略识几字,\begin{note}甲戌侧:这句加于老兄,却是实写。\end{note}终日惟有斗鸡走马,游山玩水而已。虽是皇商,一应经济世事,全然不知,不过赖祖父之旧情分,户部挂虚名,支领钱粮,其余事体,自有伙计老家人等措办。寡母王氏乃现任京营节度使王子腾之妹,与荣国府贾政的夫人王氏,是一母所生的姊妹,今年方四十上下年纪,只有薛蟠一子。还有一女,比薛蟠小两岁,乳名宝钗,生得肌骨莹润,举止娴雅。\begin{note}甲戌侧:写宝钗只如此,更妙!\end{note}当日有他父亲在日,酷爱此女,令其读书识字,较之乃兄竟高过十倍。\begin{note}甲戌侧:又只如此写来,更妙!\end{note}自父亲死后,见哥哥不能依贴母怀,他便不以书字为事,只留心针黹家计等事,好为母亲分忧解劳。近因今上崇诗尚礼,征采才能,降不世出之隆恩,除聘选妃嫔外,凡仕宦名家之女,皆亲名达部,以备选为公主、郡主入学陪侍,充为才人、赞善之职。\begin{note}甲戌侧:一段称功颂德,千古小说中所无。\end{note}二则自薛蟠父亲死后,各省中所有的买卖承局,总管、伙计人等,见薛蟠年轻不谙世事,便趁时拐骗起来,京都中几处生意,渐亦消耗。薛蟠素闻得都中乃第一繁华之地,正思一游,便趁此机会,一为送妹待选,二为望亲,三因亲自入部销算旧帐,再计新支,——实则为游览上国风光之意。因此早已打点下行装细软,以及馈送亲友各色土物人情等类,正择日一定起身,不想偏遇见了拐子重卖英莲。薛蟠见英莲生得不俗,\begin{note}甲戌侧:阿呆兄亦知不俗,英莲人品可知矣。\end{note}立意买他,又遇冯家来夺人,因恃强喝令手下豪奴将冯渊打死。他便将家中事务一一的嘱托了族中人并几个老家人,他便带了母妹竟自起身长行去了。人命官司一事,他竟视为儿戏,自为花上几个臭钱,没有不了的。\begin{note}甲戌侧:是极!人谓薛蟠为呆,余则谓是大彻悟。\end{note}
\end{parag}


\begin{parag}
    在路不记其日。\begin{note}甲戌侧:更妙!必云程限则又有落套,岂暇又记路程单哉?\end{note}那日已将入都时,却又闻得母舅王子腾升了九省统制,奉旨出都查边。薛蟠心中暗喜道:“我正愁进京去有个嫡亲的母舅管辖著,不能任意挥霍挥霍,偏如今又升出去了,可知天从人愿。”\begin{note}甲戌侧:写尽五陵心意。\end{note}因和母亲商议道:“咱们京中虽有几处房舍,只是这十来年没人进京居住,那看守的人未免偷著租赁与人,须得先著几个人去打扫收拾才好。”他母亲道:“何必如此招摇!咱们这一进京,原该先拜望亲友,或是在你舅舅家,\begin{note}甲戌侧:陪笔。\end{note}或是你姨爹家。\begin{note}甲戌侧:正笔。\end{note}他两家的房舍极是便宜的,咱们先能著住下,再慢慢的著人去收拾,岂不消停些。”薛蟠道:“如今舅舅正升了外省去,家里自然忙乱起身。咱们这工夫一窝一拖的奔了去,岂不没眼色。”他母亲道:“你舅舅家虽升了去,还有你姨爹家。况这几年来,你舅舅、姨娘两处,每每带信捎书,接咱们来。如今既来了,你舅舅虽忙著起身,你贾家姨娘未必不苦留我们。咱们且忙忙收拾房屋,岂不使人见怪?\begin{note}甲戌侧:闲语中补出许多前文,此画家之云罩峰尖法也。\end{note}你的意思我却知道,\begin{note}甲戌侧:知子莫如父。\end{note}守著舅舅、姨爹住著,未免拘紧了你,不如你各自住著,好任意施为。\begin{note}甲戌侧:寡母孤儿一段,写得毕肖毕真。\end{note}你既如此,你自去挑所宅子去住。我和你姨娘,姊妹们别了这几年,却要厮守几日,我带了你妹子投你姨娘家去,\begin{note}甲戌侧:薛母亦善训子。\end{note}你道好不好?”薛蟠见母亲如此说,情知扭不过的,只得吩咐人夫一路奔荣国府来。
\end{parag}


\begin{parag}
    那时王夫人已知薛蟠官司一事,亏贾雨村维持了结,才放了心。又见哥哥升了边缺,正愁又少了娘家的亲戚来往,\begin{note}甲戌侧:大家尚义,人情大都是也。\end{note}略加寂寞。过了几日,忽家人传报:“姨太太带了哥儿姐儿,合家进京,正在门外下车。”喜的王夫人忙带了女媳人等,接出大厅,将薛姨妈等接了进去。姊妹们暮年相会,自不必说悲喜交集,泣笑叙阔一番。忙又引了拜见贾母,将人情土物各种酬献了,合家具厮见过,忙又治席接风。
\end{parag}


\begin{parag}
    薛蟠已拜见过贾政,贾琏又引著拜见了贾赦,贾珍等。贾政便使人上来对王夫人说:“姨太太已有了春秋,外甥年轻不知世路,在外住著恐有人生事。咱们东北角上梨香院\begin{note}甲戌侧:好香色。\end{note}一所十来间房,白空闲著,打扫了,请姨太太和姐儿哥儿住了甚好。”\begin{note}甲戌眉:用政老一段,不但王夫人得体,且薛母亦免靠亲之嫌。\end{note}王夫人未及留,贾母也就遣人来说“请姨太太就在这里住下,大家亲密些”等语。\begin{note}甲戌侧:老太君口气得情。偏不写王夫人留,方不死板。\end{note}薛姨妈正要同居一处,方可拘紧些儿子,若另住在外,又恐他纵性惹祸,遂忙道谢应允。又私与王夫人说明:“一应日费供给一概免却,\begin{note}甲戌侧:作者题清,犹恐看官误认今之靠亲投友者一例。\end{note}方是处常之法。”王夫人知他家不难于此,遂亦从其愿。从此后,薛家母子就在梨香院住了。
\end{parag}


\begin{parag}
    原来这梨香院即当日荣公暮年养静之所,小小巧巧,约有十余间房屋,前厅后舍俱全。另有一门通街,薛蟠家人就走此门出入。西南有一角门,通一夹道,出夹道便是王夫人正房的东边了。每日或饭后,或晚间,薛姨妈便过来,或与贾母闲谈,或与王夫人相叙。宝钗日与黛玉迎春姊妹等一处,\begin{note}甲戌眉:金玉初见,却如此写,虚虚实实,总不相犯。\end{note}或看书下棋,或作针黹,倒也十分乐业。\begin{note}甲戌侧:这一句衬出后文黛玉之不能乐业,细甚妙甚!\end{note}只是薛蟠起初之心,原不欲在贾宅居住者,但恐姨父管约拘禁,料必不自在的,无奈母亲执意在此,且宅中又十分殷勤苦留,只得暂且住下,一面使人打扫出自己的房屋,再移居过去的。\begin{note}甲戌侧:交代结构,曲曲折折,笔墨尽矣。\end{note}谁知自从在此住了不上一月的光景,贾宅族中凡有的子侄,俱已认熟了一半,凡是那些纨绔气习者,莫不喜与他来往,今日会酒,明日观花,甚至聚赌嫖娼,渐渐无所不至,引诱的薛蟠比当日更坏了十倍。\begin{note}甲戌侧:虽说为纨绔设鉴,其意原只罪贾宅,故用此等句法写来。此等人家岂必欺霸方始成名耶?总因子弟不肖,招接匪人,一朝生事则百计营求,父为子隐,群小迎合,虽暂时不罹祸,而从此放胆,必破家灭族不已,哀哉!\end{note}虽然贾政训子有方,治家有法,\begin{note}甲戌侧:八字特洗出政老来,又是作者隐意。\end{note}一则族大人多,照管不到这些,二则现任族长乃是贾珍,彼乃宁府长孙,又现袭职,凡族中事,自有他掌管,三则公私冗杂,且素性潇洒,不以俗务为要,每公暇之时,不过看书著棋而已,余事多不介意。况且这梨香院相隔两层房舍,又有街门另开,任意可以出入,所以这些子弟们竟可以放意畅怀的,因此,薛蟠遂将移居之念,渐渐打灭了。
\end{parag}


\begin{parag}
    \begin{note}梦:正是:\end{note}
\end{parag}


\begin{parag}
    \begin{note}渐入鲍鱼肆,反恶芝兰香。\end{note}
\end{parag}

