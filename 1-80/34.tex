\chap{三十四}{情中情因情感妹妹 錯裏錯以錯勸哥哥}


\begin{parag}
    \begin{note}蒙回前總:兩條素怡,一片真心,三首新詩,萬行珠淚。\end{note}
\end{parag}


\begin{parag}
    話說襲人見賈母王夫人等去後,便走來寶玉身邊坐下,含淚問他:“怎麼就打到這步田地?”寶玉嘆氣說道:“不過爲那些事,問他做什麼!只是下半截疼的很,你瞧瞧打壞了那裏。”襲人聽說,便輕輕的伸手進去,將中衣褪下。寶玉略動一動,便咬著牙叫“噯喲”,襲人連忙停住手,如此三四次才褪了下來。襲人看時,只見腿上半段青紫,都有四指寬的僵痕高了起來。襲人咬著牙說道:“我的娘,怎麼下這般的狠手!你但凡聽我一句話,也不得到這步地位。幸而沒動筋骨,倘或打出個殘疾來,可叫人怎麼樣呢!”
\end{parag}


\begin{parag}
    正說著,只聽丫鬟們說:“寶姑娘來了。”襲人聽見,知道穿不及中衣,便拿了一牀袷紗被替寶玉蓋了。只見寶釵手裏託著一丸藥走進來,\begin{note}蒙側:請問是關心不是關心?\end{note}向襲人說道:“晚上把這藥用酒研開,替他敷上,把那淤血的熱毒散開,可以就好了。”說畢,遞與襲人,又問道:“這會子可好些?”寶玉一面道謝說:“好了。”又讓坐。寶釵見他睜開眼說話,不象先時,心中也寬慰了好些,便點頭嘆道:“早聽人一句話,\begin{note}蒙側:同襲人語。\end{note}也不至 今日。別說老太太、太太心疼,就是我們看著,心裏也疼。”剛說了半句又忙嚥住,自悔說的話急了,不覺的就紅了臉,\begin{note}蒙側:行雲流水,微露半含時。\end{note}低下頭來。寶玉聽得這話如此親切稠密,大有深意,忽見他又咽住不往下說,紅了臉,低下頭只管弄衣帶,那一種嬌羞怯怯,非可形容得出者,不覺心中大暢,將疼痛早丟在九霄雲外,心中自思:“我不過捱了幾下打,他們一個個就有這些憐惜悲感之態露出,令人可玩可觀,可憐可敬。假若我一時竟遭殃橫死,他們還不知是何等悲 感呢!\begin{note}蒙側:得遇知己者,多生此等疑思疑喜。\end{note}既是他們這樣,我便一時死了,得他們如此,一生事業縱然盡付東流,亦無足嘆惜,冥冥之中若不怡然自得, 亦可謂糊塗鬼祟矣。”想著,只聽寶釵問襲人道:“怎麼好好的動了氣,就打起來了?”襲人便把焙茗的話說了出來。寶玉原來還不知道賈環的話,見襲人說出方纔知道。因又拉上薛蟠,惟恐寶釵沉心,忙又止住襲人道:“薛大哥哥從來不這樣的,你們不可混猜度。”寶釵聽說,便知道是怕他多心,用話相攔襲人,因心中暗暗 想道:“打的這個形像,疼還顧不過來,還是這樣細心,怕得罪了人,可見在我們身上也算是用心了。\begin{note}蒙側:天下古今英雄同一感慨。\end{note}你既這樣用心,何不在外頭大事上做工夫,老爺也歡喜了,也不能喫這樣虧。但你固然怕我沉心,所以攔襲人的話,難道我就不知我的哥哥素日恣心縱慾,毫無防範的那種心性。當日爲一個秦鍾,還鬧的天翻地覆,自然如今比先又更利害了。”想畢,因笑道:“你們也不必怨這個,怨那個。據我想,到底寶兄弟素日不正,肯和那些人來往,老爺才生 氣。就是我哥哥說話不防頭,一時說出寶兄弟來,也不是有心調唆:一則也是本來的實話,二則他原不理論這些防嫌小事。襲姑娘從小兒只見寶兄弟這麼樣細心的人,\begin{note}蒙側:心頭口頭不覺透漏。\end{note}你何嘗見過天不怕地不怕、心裏有什麼口裏就說什麼的人。”襲人因說出薛蟠來,見寶玉攔他的話,早已明白自己說造次了, 恐寶釵沒意思,聽寶釵如此說,更覺羞愧無言。寶玉又聽寶釵這番話,一半是堂皇正大,一半是去己疑心,更覺比先暢快了。方欲說話時,只見寶釵起身說道:“明兒再來看你,你好生養著罷。方纔我拿了藥來交給襲人,晚上敷上管就好了。\begin{note}蒙側:何等關心。\end{note}”說著便走出門去。襲人趕著送出院外,說:“姑娘倒費心了。改日寶二爺好了,親自來謝。”寶釵回頭笑道:“有什麼謝處。你只勸他好生靜養,別胡思亂想的就好了。\begin{note}蒙側:的確真心。\end{note}不必驚動老太太、太太衆人,倘或吹到老爺耳朵裏,雖然彼時不怎麼樣,將來對景,終是要喫虧的。\begin{note}蒙側:要緊。\end{note}”說著,一面去了。
\end{parag}


\begin{parag}
    襲人抽身回來,心內著實感激寶釵。進來見寶玉沉思默默似睡非睡的模樣,因而退出房外,自去櫛沐。寶玉默默的躺在牀上,無奈臀上作痛,如針挑刀挖一般, 更又熱如火炙,略展轉時,禁不住“噯喲”之聲。那時天色將晚,因見襲人去了,卻有兩三個丫鬟伺候,此時並無呼喚之事,因說道:“你們且去梳洗,等我叫時再來。”衆人聽了,也都退出。
\end{parag}


\begin{parag}
    這裏寶玉昏昏默默,只見蔣玉菡走了進來,訴說忠順府拿他之事;又見金釧兒進來哭說爲他投井之情。寶玉半夢半醒,都不在意。忽又覺有人推他,恍恍忽忽聽 得有人悲慼之聲。寶玉從夢中驚醒,睜眼一看,不是別人,卻是林黛玉。寶玉猶恐是夢,忙又將身子欠起來,向臉上細細一認,只見兩個眼睛腫的桃兒一般,滿面淚光,不是黛玉,卻是那個?寶玉還欲看時,怎奈下半截疼痛難忍,支持不住,便“噯喲”一聲,仍就倒下,嘆了一聲,說道:“你又做什麼跑來!雖說太陽落下去, 那地上的餘熱未散,走兩趟又要受了暑。我雖然捱了打,並不覺疼痛。我這個樣兒,只裝出來哄他們,好在外頭佈散與老爺聽,其實是假的。你不可認真。\begin{note}蒙側:有這樣一段語,方不沒滅顰顰兒之痛哭眼腫。英雄失足,每每至死不改,皆猶此而。\end{note}”此時林黛玉雖不是嚎啕大哭,然越是這等無聲之泣,氣噎喉堵,更覺得利害。聽了寶玉這番話,心中雖然有萬句言詞,只是不能說得,半日,方抽抽噎噎的說道:“你從此可都改了罷!\begin{note}蒙側:心血淋漓釀成此數字。\end{note}”寶玉聽說, 便長嘆一聲,道:“你放心,別說這樣話。就便爲這些人死了,\begin{note}蒙側:文氣斬動。\end{note}也是情願的!(校者注:蒙本此處無“也是情願的”,換作“況已是活過來 了”)”一句話未了,只見院外人說:“二奶奶來了。”林黛玉便知是鳳姐來了,連忙立起身說道:“我從後院子去罷,回來再來。”寶玉一把拉住道:“這可奇了,好好的怎麼怕起他來。”林黛玉急的跺腳,悄悄的說道:“你瞧瞧我的眼睛,又該他取笑開心呢。\begin{note}蒙側:不避嫌疑,不惜聲名,破格牽連,誠爲可嘆,著實 可憐。\end{note}”寶玉聽說趕忙的放手。黛玉三步兩步轉過牀後,出後院而去。鳳姐從前頭已進來了,問寶玉:“可好些了?想什麼喫,叫人往我那裏取去。”接著,薛姨媽又來了。一時賈母又打發了人來。
\end{parag}


\begin{parag}
    至掌燈時分,寶玉只喝了兩口湯,便昏昏沉沉的睡去。接著,周瑞媳婦、吳新登媳婦、鄭好時媳婦這幾個有年紀常往來的,聽見寶玉捱了打,也都進來。襲人忙迎出來,悄悄的笑道:“嬸嬸們來遲了一步,\begin{note}蒙側:襲卿善詞令會周旋。\end{note}二爺才睡著了。”說著,一面帶他們到那邊房裏坐了,倒茶與他們喫。那幾個媳婦子都悄悄的坐了一回,向襲人說:“等二爺醒了,你替我們說罷。”
\end{parag}


\begin{parag}
    襲人答應了,送他們出去。剛要回來,只見王夫人使個婆子來,口稱“太太叫一個跟二爺的人呢”。襲人見說,想了一想,便回身悄悄告訴晴雯、麝月、檀雲、 秋紋等說:“太太叫人,你們好生在房裏,我去了就來。”\begin{note}蒙側:身任其責,不憚勞煩。\end{note}說畢,同那婆子一徑出了園子,來至上房。王夫人正坐在涼榻上搖著芭蕉扇子,見他來了,說:“不管叫個誰來也罷了。你又丟下他來了,誰伏侍他呢?”襲人見說,連忙陪笑回道:“二爺才睡安穩了,那四五個丫頭如今也好了,會伏侍二爺了,太太請放心。恐怕太太有什麼話吩咐,打發他們來,一時聽不明白,倒耽誤了。\begin{note}蒙側:能事解事能了事。\end{note}”王夫人道:“也沒甚話,白問問他這 會子疼的怎麼樣。”襲人道:“寶姑娘送去的藥,我給二爺敷上了,\begin{note}蒙側:補足。\end{note}比先好些了。先疼的躺不穩,這會子都睡沉了,可見好些了。”王夫人又 問:“吃了什麼沒有?”襲人道:“老太太給的一碗湯,喝了兩口,只嚷幹喝,要喫酸梅湯。我想著酸梅是個收斂的東西,纔剛捱了打,又不許叫喊,自然急的那熱 毒熱血未免不存在心裏,倘或喫下這個去激在心裏,再弄出大病來,可怎麼樣呢。因此我勸了半天才沒喫,\begin{note}蒙側:能事態。\end{note}只拿那糖醃的玫瑰滷子和了喫,吃了半碗,又嫌喫絮了,不香甜。”王夫人道:“噯喲,你不該早來和我說。前兒有人送了兩瓶子香露來,原要給他點子的,我怕他胡糟踏了,就沒給。既是他嫌那些 玫瑰膏子絮煩,把這個拿兩瓶子去。一碗水裏只用挑一茶匙兒,就香的了不得呢。”說著就喚彩雲來,“把前兒的那幾瓶香露拿了來。”襲人道:“只拿兩瓶來罷, 多了也白糟踏。等不夠再要,再來取也是一樣。”彩雲聽說,去了半日,果然拿了兩瓶來,付與襲人。襲人看時,只見兩個玻璃小瓶,卻有三寸大小,上面螺絲銀蓋,鵝黃箋上寫著“木樨清露”,那一個寫著“玫瑰清露”。襲人笑道:“好金貴東西!這麼個小瓶兒,能有多少?”王夫人道:“那是進上的,你沒看見鵝黃箋子?你好生替他收著,別糟踏了。”
\end{parag}


\begin{parag}
    襲人答應著,方要走時,王夫人又叫:“站著,我想起一句話來問你。”襲人忙又回來。王夫人見房內無人,便問道:“我恍惚聽見寶玉今兒捱打,是環兒在老 爺跟前說了什麼話。你可聽見這個了?你要聽見,告訴我聽聽,我也不吵出來教人知道是你說的。”襲人道:“我倒沒聽見這話,爲二爺霸佔著戲子,人家來和老爺要,爲這個打的。”王夫人搖頭說道:“也爲這個,還有別的原故。”襲人道:“別的原故實在不知道了。我今兒在太太跟前大膽說句不知好歹的話。論理……”說 了半截忙又咽住。王夫人道:“你只管說。”襲人笑道:“太太別生氣,我就說了。”王夫人道:“我有什麼生氣的,你只管說來。”襲人道:“論理,我們二爺也須得老爺教訓兩頓。若老爺再不管,將來不知做出什麼事來呢。”王夫人一聞此言,便合掌念聲“阿彌陀佛”,\begin{note}蒙側:襲卿之心,所謂良人所仰望而終身也。今若此,能不痛哭流泣以成此語?\end{note}由不得趕著襲人叫了一聲“我的兒,虧了你也明白,這話和我的心一樣。我何曾不知道管兒子,先時你珠大爺在,我是怎麼樣管 他,難道我如今倒不知管兒子了?只是有個原故:如今我想,我已經快五十歲的人,通共剩了他一個,他又長的單弱,況且老太太寶貝似的,若管緊了他,倘或再有個好歹,或是老太太氣壞了,那時上下不安,豈不倒壞了,所以就縱壞了他。我常常掰著口兒勸一陣,說一陣,氣的罵一陣,哭一陣,彼時他好,過後兒還是不相干,端的吃了虧才罷了。若打壞了,將來我靠誰呢!\begin{note}蒙側:變轉之句,勉強之言,真體貼,盡溺愛之心。\end{note}”說著,由不得滾下淚來。
\end{parag}


\begin{parag}
    襲人見王夫人這般悲感,自己也不覺傷了心,陪著落淚。又道:“二爺是太太養的,豈不心疼。便是我們做下人的伏侍一場,大家落個平安,也算是造化了。要這樣起來,連平安都不能了。那一日那一時我不勸二爺,只是再勸不醒。偏生那些人又肯親近他,也怨不得他這樣,總是我們勸的倒不好了。今兒太太提起這話來, 我還記掛著一件事,每要來回太太,討太太個主意。只是我怕太太疑心,不但我的話白說了,且連葬身之地都沒了。\begin{note}蒙側:打進一層。非有前項,如許講究這一 層,即爲唐突了。\end{note}” 王夫人聽了這話內有因,忙問道:“我的兒,你有話只管說。近來我因聽見衆人背前背後都誇你,我只說你不過是在寶玉身上留心,或是諸人跟前和氣,這些小意思好,所以將你和老姨娘一體行事。誰知你方纔和我說的話全是大道理,正和我的想頭一樣。你有什麼只管說什麼,只別教別人知道就是了。”襲人道:“我也沒什麼 別的說。我只想著討太太一個示下,怎麼變個法兒,以後竟還教二爺搬出園外來就好了。”王夫人聽了,喫一大驚,忙拉了襲人的手問道:“寶玉難道和誰作怪了不成?”襲人忙回道:“太太別多心,並沒有這話。這不過是我的小見識。如今二爺也大了,裏頭姑娘們也大了,況且林姑娘寶姑娘又是兩姨姑表姊妹,雖說是姊妹 們,到底是男女之分,日夜一處起坐不方便,由不得叫人懸心,\begin{note}蒙側:遠憂近慮,言言字字真是可人。\end{note}便是外人看著也不象。一家子的事,俗語說的‘沒事常思有事’,世上多少無頭腦的事,多半因爲無心中做出,有心人看見,當做有心事,反說壞了。只是預先不防著,斷然不好。二爺素日性格,太太是知道的。他又偏好在我們隊裏鬧,倘或不防,前後錯了一點半點,不論真假,人多口雜,那起小人的嘴有什麼避諱,心順了,說的比菩薩還好,心不順,就貶的連畜牲不如。二爺將 來倘或有人說好,不過大家直過沒事;若叫人說出一個不好字來,我們不用說,粉身碎骨,罪有萬重,都是平常小事,便後來二爺一生的聲名品行豈不完了,\begin{note}蒙側:襲卿愛人以德,竟至如此。字字逼來,不覺令人靜聽。看官自省,且可闊略戒之。\end{note}二則太太也難見老爺。俗語又說‘君子防不然’,不如這會子防避的爲是。 太太事情多,一時固然想不到。我們想不到則可,既想到了,若不回明太太,罪越重了。近來我爲這事日夜懸心,又不好說與人,惟有燈知道罷了。”王夫人聽了這話,如雷轟電掣一般,正觸了金釧兒之事,心內越發感愛襲人不盡,忙笑道:“我的兒,你竟有這個心胸,想的這樣周全!我何曾又不想到這裏,只是這幾次有事就忘了。你今兒這一番話提醒了我。難爲你成全我孃兒兩個聲名體面,真真我竟不知道你這樣好。罷了,你且去罷,我自有道理。\begin{note}蒙側:溺愛者偏會如此說。\end{note}只是還有一句話:你如今既說了這樣的話,我就把他交給你了,好歹留心,保全了他,就是保全了我。我自然不辜負你。”
\end{parag}


\begin{parag}
    襲人連連答應著去了。回來正值寶玉睡醒,襲人回明香露之事。寶玉喜不自禁,即令調來嘗試,果然香妙非常。因心下記掛著黛玉,滿心裏要打發人去,只是怕襲人,便設一法,先使襲人往寶釵那裏去借書。
\end{parag}


\begin{parag}
    襲人去了,寶玉便命晴雯來\begin{note}蒙雙夾:前文晴雯放肆原有把柄所恃也。\end{note}吩咐道:“你到林姑娘那裏看看他做什麼呢。他要問我,只說我好了。”晴雯道: “白眉赤眼,做什麼去呢?到底說句話兒,也象一件事。”寶玉道:“沒有什麼可說的。”晴雯道:“若不然,或是送件東西,或是取件東西,不然我去了怎麼搭訕 呢?”寶玉想了一想,便伸手拿了兩條手帕子撂與晴雯,笑道:“也罷,就說我叫你送這個給他去了。”晴雯道:“這又奇了。他要這半新不舊的兩條手帕子?他又要惱了,說你打趣他。”寶玉笑道:“你放心,他自然知道。”
\end{parag}


\begin{parag}
    晴雯聽了,只得拿了帕子往瀟湘館來。只見春纖正在欄杆上晾手帕子,\begin{note}蒙側:送的是手帕,晾的是手帕,妙文。\end{note}見他進來,忙擺手兒,說:“睡下了。” 晴雯走進來,滿屋黑魆。並未點燈。黛玉已睡在牀上。問是誰。晴雯忙答道:“晴雯。”黛玉道:“做什麼?”晴雯道:“二爺送手帕子來給姑娘。”黛玉聽了,心中發悶:“做什麼送手帕子來給我?”因問:“這帕子是誰送他的?必是上好的,叫他留著送別人罷,我這會子不用這個。” 晴雯笑道:“不是新的,就是家常舊的。”林黛玉聽見,越發悶住,著實細心搜求,思忖一時,方大悟過來,連忙說:“放下,去罷。”晴雯聽了,只得放下,抽身 回去,一路盤算,不解何意。
\end{parag}


\begin{parag}
    這裏林黛玉體貼出手帕子的意思來,不覺神魂馳蕩:寶玉這番苦心,能領會我這番苦意,又令我可喜;我這番苦意,不知將來如何,又令我可悲;忽然好好的送 兩塊舊帕子來,若不是領我深意,單看了這帕子,又令我可笑;再想令人私相傳遞與我,又可懼;我自己每每好哭,想來也無味,又令我可愧。如此左思右想,一時五內沸然炙起。黛玉由不得餘意綿纏,令掌燈,也想不起嫌疑避諱等事,便向案上研墨蘸筆,便向那兩塊舊帕上走筆寫道:
\end{parag}


\begin{poem}
    \begin{pl}其一\end{pl}

    \begin{pl}眼空蓄淚淚空垂,暗灑閒拋卻爲誰?\end{pl}

    \begin{pl}尺幅鮫鮹勞解贈,叫人焉得不傷悲!\end{pl}
    \emptypl

    \begin{pl}其二\end{pl}

    \begin{pl}拋珠滾玉只偷潸,鎮日無心鎮日閒;\end{pl}

    \begin{pl}枕上袖邊難拂拭,任他點點與斑斑。\end{pl}
    \emptypl

    \begin{pl}其三\end{pl}

    \begin{pl}綵線難收面上珠,湘江舊跡已模糊;\end{pl}

    \begin{pl}窗前亦有千竿竹,不識香痕漬也無?\end{pl}
\end{poem}


\begin{parag}
    林黛玉還要往下寫時,覺得渾身火熱,面上作燒,走至鏡臺揭起錦袱一照,只見腮上通紅,自羨壓倒桃花,卻不知病由此萌。一時方上牀睡去,猶拿著那帕子思索,不在話下。
\end{parag}


\begin{parag}
    卻說襲人來見寶釵,誰知寶釵不在園內,往他母親那裏去了,襲人便空手回來。等至二更,寶釵方回來。原來寶釵素知薛蟠情性,心中已有一半疑是薛蟠調唆了人來告寶玉的,誰知又聽襲人說出來,越發信了。究竟襲人是聽焙茗說的,那焙茗也是私心窺度,並未據實,竟認準是他說的。那薛蟠都因素日有這個名聲,其實這一次卻不是他乾的,被人生生的一口咬死是他,有口難分。這日正從外頭吃了酒回來,見過母親,只見寶釵在這裏,說了幾句閒話,因問:“聽見寶兄弟吃了虧,是 爲什麼?”薛姨媽正爲這個不自在,見他問時,便咬著牙道:“不知好歹的東西,都是你鬧的,你還有臉來問!”薛蟠見說,便怔了,忙問道:“我何嘗鬧什麼?” 薛姨媽道:“你還裝憨呢!人人都知道是你說的,還賴呢。”薛蟠道:“人人說我殺了人,也就信了罷?”薛姨媽道:“連你妹妹都知道是你說的,難道他也賴你不 成?”寶釵忙勸道:“媽和哥哥且別叫喊,消消停停的,就有個青紅皁白了。”因向薛蟠道:“是你說的也罷,不是你說的也罷,事情也過去了,不必較證,倒把小 事兒弄大了。我只勸你從此以後在外頭少去胡鬧,少管別人的事。天天一處大家胡逛,你是個不防頭的人,過後兒沒事就罷了,倘或有事,不是你乾的,人人都也疑惑是你乾的,不用說別人,我就先疑惑。”薛蟠本是個心直口快的人,一生見不得這樣藏頭露尾的事,又見寶釵勸他不要逛去,他母親又說他犯舌,寶玉之打是他治的,早已急的亂跳,賭身發誓的分辯。又罵衆人:“誰這樣贓派我?我把那囚攮的牙敲了才罷!分明是爲打了寶玉,沒的獻勤兒,拿我來作幌子。難道寶玉是天王? 他父親打他一頓,一家子定要鬧幾天。那一回爲他不好,姨爹打了他兩下子,過後老太太不知怎麼知道了,說是珍大哥哥治的,好好的叫了去罵了一頓。今兒越發拉上我了!既拉上,我也不怕,越性進去把寶玉打死了,我替他償了命,大家乾淨。”一面嚷,一面抓起一根門閂來就跑。慌的薛姨媽一把抓住,罵道:“作死的孽障,你打誰去?你先打我來!”薛蟠急的眼似銅鈴一般,嚷道:“何苦來!又不叫我去,又好好的賴我。將來寶玉活一日,我擔一日的口舌,不如大家死了清淨。” 寶釵忙也上前勸道:“你忍耐些兒罷。媽急的這個樣兒,你不說來勸媽,你還反鬧的這樣。別說是媽,便是旁人來勸你,也爲你好,倒把你的性子勸上來了。”薛蟠道:“這會子又說這話。都是你說的!”寶釵道:“你只怨我說,再不怨你顧前不顧後的形景。”薛蟠道:“你只會怨我顧前不顧後,你怎麼不怨寶玉外頭招風惹草的那個樣子!別說多的,只拿前兒琪官的事比給你們聽:那琪官,我們見過十來次的,我並未和他說一句親熱話;怎麼前兒他見了,連姓名還不知道,就把汗巾子給他了?難道這也是我說的不成?”薛姨媽和寶釵急的說道:“還提這個!可不是爲這個打他呢。可見是你說的了。”薛蟠道:“真真的氣死了人了!賴我說的我不 惱,我只爲一個寶玉鬧的這麼天翻地覆的。”寶釵道:“誰鬧了?你先持刀動杖的鬧起來,倒說別人鬧。”薛蟠見寶釵說的句句有理,難以駁正,比母親的話反難回答,因此便要設法拿話堵回他去,就無人敢攔自己的話了;也因正在氣頭兒上,未曾想話之輕重,便說道:“好妹妹,你不用和我鬧,我早知道你的心了。從先媽和 我說,你這金要揀有玉的纔可正配,你留了心,見寶玉有那勞什骨子,你自然如今行動護著他。”話未說了,把個寶釵氣怔了,拉著薛姨媽哭道:“媽媽你聽,哥哥說的是什麼話!”\begin{note}蒙側:描寫薛蟠,不過要補足寶釵告襲人前項之言。\end{note}薛蟠見妹妹哭了,便知自己冒撞了,便堵氣走到自己房裏安歇不提。
\end{parag}


\begin{parag}
    這裏薛姨媽氣的亂戰,一面又勸寶釵道:“你素日知那孽障說話沒道理,明兒我叫他給你陪不是。”寶釵滿心委屈氣忿,待要怎樣,又怕他母親不安,少不得含淚別了母親,各自回來,到房裏整哭了一夜。次日早起來,也無心梳洗,胡亂整理整理,便出來瞧母親。可巧遇見林黛玉獨立在花陰之下,問他那裏去。薛寶釵因說 “家去”,口裏說著,便只管走。黛玉見他無精打采的去了,又見眼上有哭泣之狀,大非往日可比,便在後面笑道:“姐姐也自保重些兒。就是哭出兩缸眼淚來,也 醫不好棒瘡!”\begin{note}蒙側:自己眼腫爲誰?偏是以此笑人。笑人世間人多犯此症。\end{note}不知寶釵如何答對,且聽下回分解。
\end{parag}


\begin{parag}
    \begin{note}蒙回末總:人有百折不撓之真心,方能成曠世稀有之事業。寶玉意中諸多輻輳,所謂“求仁得仁,又和怨?”凡人作臣作子,出入家庭廟朝,能推此心此志,忠孝之不、事業之不立耶?\end{note}
\end{parag}

