\chap{四}{薄命女偏逢薄命郎 葫蘆僧亂判葫蘆案}

\begin{parag}
    \begin{note}蒙回前:陰陽交結變無倫,幻境生時即是真。秋月春花誰不見,朝晴暮雨自何因。心肝一點勞牽戀,可意偏長遇喜嗔。我愛世緣隨分定,至誠相感作癡人。
        請君著眼護官符,把筆悲傷說世途。作者淚痕同我淚,燕山仍舊竇公無。\end{note}
\end{parag}


\begin{parag}
    題曰:
\end{parag}


\begin{poem}
    \begin{pl}捐軀報國恩,\end{pl}
    \begin{pl}未報軀猶在。\end{pl}

    \begin{pl}眼底物多情,\end{pl}
    \begin{pl}君恩或可待。\end{pl}
\end{poem}


\begin{parag}
    卻說黛玉同姊妹們至王夫人處,見王夫人與兄嫂處的來使計議家務,又說姨母家遭人命官司等語。因見王夫人事情冗雜,姊妹們遂出來,至寡嫂李氏房中來了。
\end{parag}


\begin{parag}
    原來這李氏即賈珠之妻。\begin{note}甲側:起筆寫薛家事,他偏寫宮裁,是結黛玉,明李紈本末,又在人意料之外。\end{note}珠雖夭亡,倖存一子,取名賈蘭,今方五歲,已入學攻書。這李氏亦系金陵名宦之女,父名李守中,\begin{note}甲側:妙!蓋雲人能以理自守,安得爲情所陷哉!\end{note}曾爲國子監祭酒,族中男女無有不誦詩讀書者。\begin{note}甲側:未出李紈,先伏下李紋、李綺。\end{note}至李守中繼承以來,便說“女子無才便有\begin{note}甲側:“有”字改得好。\end{note}德”,故生了李氏時,便不十分令其讀書,只不過將些《女四書》、《列女傳》、《賢媛集》等三四種書,使他認得幾個字,記得前朝這幾個賢女便罷了,卻只以紡績井臼爲要,因取名爲李紈,字宮裁。\begin{note}甲側:一洗小說窠臼俱盡,且命名字,亦不見紅香翠玉惡俗。\end{note}因此這李紈雖青春喪偶,居家處膏粱錦繡之中,竟如槁木死灰一般,\begin{note}甲側:此時處此境,最能越理生事,彼竟不然,實罕見者。\end{note}一概無見無聞,唯知侍親養子,外則陪侍小姑等針黹誦讀而已。\begin{note}甲側:一段敘出李紈,不犯熙鳳。\end{note}今黛玉雖客寄於斯,日有這般姐妹相伴,除老父外,餘者也都無庸慮及了。\begin{note}甲側:仍是從黛玉身上寫來,以上了結住黛玉,復找前文。\end{note}
\end{parag}


\begin{parag}
    如今且說雨村,因補授了應天府,一下馬就有一件人命官司詳至案下,乃是兩家爭買一婢,各不相讓,以至毆傷人命。彼時雨村即傳原告之人來審。那原告道: “被毆死者乃小人之主人。因那日買了一個丫頭,不想是柺子拐來賣的。這柺子先已得了我家的銀子,我家小爺原說第三日方是好日子,再接入門。\begin{note}甲側:所謂“遲則有變”,往往世人因不經之談誤卻大事。\end{note}這柺子便又悄悄的賣與薛家,被我們知道了,去找拿賣主,奪取丫頭。無奈薛家原系金陵一霸, 脅仗勢,衆豪奴將我小主人竟打死了。凶身主僕已皆逃走,無影無蹤,只剩了幾個局外之人。小人告了一年的狀,竟無人作主。望大老爺拘拿兇犯,剪惡除兇,以救孤寡,死者感戴天地之恩不盡!”
\end{parag}


\begin{parag}
    雨村聽了大怒道:“豈有這樣放屁的事!打死人命就白白的走了,再拿不來的?”因發籤差公人立刻將兇犯族中人拿來拷問,令他們實供藏在何處,一面再動海捕文書。正要發籤時,只見案邊立的一個門子,使眼色兒不令他發籤。雨村心下甚爲疑怪,\begin{note}甲側:原可疑怪,餘亦疑怪。\end{note}只得停了手。即時退堂,至密室,侍從皆退去,只留門子服侍。這門子忙上來請安,笑問:“老爺一向加官進祿,八九年來就忘了我了?”\begin{note}甲側:語氣傲慢,怪甚!\end{note}雨村道:“卻十分面善得緊,只是一時想不起來。”那門子笑道:“老爺真是貴人多忘事,把出身之地竟忘了,\begin{note}甲側:剎心語。自招其禍,亦因誇能恃才也。\end{note}不記當年葫蘆廟裏之事?”雨村聽了,如雷震一驚,\begin{note}甲側:餘亦一驚,但不知門子何知,尤爲怪甚。\end{note}方想起往事。原來這門子本是葫蘆廟內一個小沙彌,因被火之後,無處安身,欲投別廟去修行,又耐不得清涼景況,因想這件生意倒還輕省熱鬧,\begin{note}甲側:新鮮字眼。\end{note}遂趁年紀蓄了發,充了門子。\begin{note}甲側:一路奇奇怪怪,調侃世人,總在人意臆之外。\end{note}雨村那裏料得是他,便忙攜手笑道:“原來是故人。”\begin{note}甲側:妙稱!全是假態。\end{note}又讓坐了好談。\begin{note}甲側:假極!\end{note}這門子不敢坐。雨村笑道:“貧賤之交不可忘,\begin{note}甲側:全是奸險小人態度,活現活跳。\end{note}你我故人也,二則此係私室,既欲長談,豈有不坐之理?”這門子聽說,方告了座,斜籤著坐了。
\end{parag}


\begin{parag}
    雨村因問方纔何故有不令發籤之意。這門子道:“老爺既榮任到這一省,難道就沒抄一張本省‘護官符’\begin{note}甲側:可對“聚寶盆”,一笑。三字從來未見,奇之至!\end{note}來不成?”雨村忙問:“何爲‘護官符’?\begin{note}甲側:餘亦欲問。\end{note}我竟不知。”門子道:“這還了得!連這個不知,怎能作得長遠!\begin{note}甲側:罵得爽快!\end{note}如今凡作地方官者,皆有一個私單,上面寫的是本省最有權有勢,極富極貴的大鄉紳名姓,各省皆然,倘若不知,一時觸犯了這樣的人家,不但官爵,只怕連性命還保不成呢!\begin{note}甲側:可憐可嘆,可恨可氣,變作一把眼淚也。\end{note}所以綽號叫作‘護官符’。\begin{note}甲側:奇甚趣甚,如何想來?\end{note}方纔所說的這薛家,老爺如何惹他!他這件官司並無難斷之處,皆因都礙著情分面上,所以如此。”一面說,一面從順袋中取出一張抄寫的‘護官符’來,遞與雨村,看時,上面皆是本地大族名宦之家的諺俗口碑。其口碑排寫得明白,下面所注的皆是自始祖官爵並房次。石頭亦曾抄寫了一張,\begin{note}甲側:忙中閒筆用得好。\end{note}今據石上所抄雲:
\end{parag}


\begin{poem}
    \begin{pl}賈不假,白玉爲堂金作馬。\end{pl}
    \begin{note}甲側:寧國、榮國二公之後,共二十房分,除寧、榮親派八房在都外,現原籍住者十二房。\end{note}

    \begin{pl}阿房宮,三百里,住不下金陵一個史。\end{pl}
    \begin{note}甲側:保齡侯尚書令史公之後,房分共十八。都中現住者十房,原籍現居八房。\end{note}

    \begin{pl}豐年好大雪\end{pl}\begin{note}甲夾:隱“薛”字。\end{note} \begin{pl},珍珠如土金如鐵。\end{pl}

    \begin{note}甲側:紫薇舍人薛公之後,現領內府帑銀行商,共八房分。\end{note}

    \begin{pl}東海缺少白玉牀,龍王來請金陵王。\end{pl}\begin{note}甲側:都太尉統制縣伯王公之後,共十二房。都中二房,餘皆在籍。\end{note}

\end{poem}


\begin{parag}
    雨村猶未看完,\begin{note}甲眉:妙極!若只有此四家,則死板不活,若再有兩家,又覺累贅,故如此斷法。\end{note}忽聽傳點,人報:“王老爺來拜。”雨村聽說,忙具衣冠出去迎接。\begin{note}甲側:橫雲斷嶺法,是板定大章法。\end{note}有頓飯工夫,方回來細問。這門子道:“這四家皆連絡有親,一損皆損,一榮皆榮,扶持遮飾,俱有照應的。\begin{note}甲側:早爲下半部伏根。\end{note}今告打死人之薛,就係豐年大雪之‘雪’也。也不單靠這三家,他的世交親友在都在外者,本亦不少。老爺如今拿誰去?”雨村聽如此說,便笑問門子道:“如你這樣說來,卻怎麼了結此案?你大約也深知這兇犯躲的方向了?”
\end{parag}


\begin{parag}
    門子笑道:“不瞞老爺說,不但這兇犯的方向我知道,一併這拐賣之人\begin{note}甲側:斯何人也。\end{note}我也知道,死鬼買主也深知道。待我細說與老爺聽:這個被打之死鬼,乃是本地一個小鄉紳之子,名喚馮淵,\begin{note}甲側:真真是冤孽相逢。\end{note}自幼父母早亡,又無兄弟,只他一個人守著些薄產過日子。長到十八九歲上,酷愛男風,最厭女子。\begin{note}甲側:最厭女子,仍爲女子喪生,是何等大筆!不是寫馮淵,正是寫英蓮。\end{note}這也是前生冤孽,可巧\begin{note}甲側:善善惡惡,多從可巧而來,可畏可怕。\end{note}遇見這柺子賣丫頭,他便一眼看上了這丫頭,立意買來作妾,立誓再不交結男子,\begin{note}甲側:諺雲:“人若改常,非病即亡。”信有之乎?\end{note}也不再娶第二個了,\begin{note}甲側:虛寫一個情種。\end{note}所以三日後方過門。誰曉這柺子又偷賣與薛家,他意欲捲了兩家的銀子,再逃往他省。誰知又不曾走脫,兩家拿住,打了個臭死,都不肯收銀,只要領人。那薛家公子豈是讓人的,便喝著手下人一打,將馮公子打了個稀爛,擡回家去三日死了。這薛公子原是早已擇定日子上京去的,頭起身兩日前,就偶然遇見這丫頭,意欲買了就進京的,誰知鬧出這事來。既打了馮公子,奪了丫頭,他便沒事人一般,只管帶了家眷走他的路。他這裏自有兄弟奴僕在此料理,也並非爲此些些小事值得他一逃走的。\begin{note}甲側:妙極!人命視爲些些小事,總是刻畫阿呆耳。\end{note}這且別說,老爺你當被賣之丫頭是誰?”\begin{note}甲側:問得又怪。\end{note}雨村笑道:“我如何得知?”門子冷笑道:“這人算來還是老爺的大恩人呢!他就是葫蘆廟旁住的甄老爺的小姐,名喚英蓮的。”\begin{note}甲側:至此一醒。\end{note}雨村罕然道:“原來就是他!聞得養至五歲被人拐去,卻如今纔來賣呢?”
\end{parag}


\begin{parag}
    門子道:“這一種柺子單管偷拐五六歲的兒女,養在一個僻靜之處,到十一二歲,度其容貌,帶至他鄉轉賣。當日這英蓮,我們天天哄他頑耍,雖隔了七八年,如今十二三歲的光景,其模樣雖然出脫得齊整好些,然大概相貌,自是不改,熟人易認。況且他眉心中原有米粒大小的一點胭脂痣,從胎裏帶來的,\begin{note}甲側:寶釵之熱,黛玉之怯,悉從胎中帶來。今英蓮有痣,其人可知矣。\end{note}所以我卻認得。偏生這柺子又租了我的房舍居住,那日柺子不在家,我也曾問他。他是被拐子打怕了的,\begin{note}甲側:可憐!\end{note}萬不敢說,只說柺子系他親爹,因無錢償債,故賣他。我又哄之再四,他又哭了,只說:‘我不記得小時之事!’這可無疑了。那日馮公子相看了,兌了銀子,柺子醉了,他自嘆道:‘我今日罪孽可滿了!’後又聽見馮公子令三日之後過門,他又轉有憂愁之態。我又不忍其形景,等柺子出去,又命內人去解釋他:‘這馮公子必待好日期來接,可知必不以丫鬟相看。況他是個絕風流人品,家裏頗過得,素習又最厭惡堂客,今竟破價買你,後事不言可知。只耐得三兩日,何必憂悶!’他聽如此說,方纔略解憂悶,自爲從此得所。誰料天下竟有這等不如意事,\begin{note}甲側:可憐真可憐!一篇《薄命賦》,特出英蓮。\end{note}第二日,他偏又賣與薛家。若賣與第二個人還好,這薛公子的混名人稱‘呆霸王’,最是天下第一個弄性尚氣的人,而且使錢如土,\begin{note}甲側:世路難行錢作馬。\end{note}遂打了個落花流水,生拖死拽,把個英蓮拖去,如今也不知死活。\begin{note}甲側:爲英蓮留後步。\end{note}這馮公子空喜一場,一念未遂,反花了錢,送了命,豈不可嘆!”\begin{note}甲眉:又一首《薄命嘆》。英、馮二人一段小悲歡幻境從葫蘆僧口中補出,省卻閒文之法也。所謂“美中不足,好事多魔”,先用馮淵作一開路之人。\end{note}
\end{parag}


\begin{parag}
    雨村聽了,亦嘆道:“這也是他們的 跽遭遇,亦非偶然。不然這馮淵如何偏只看準了這英蓮?這英蓮受了柺子這幾年折磨,才得了個頭路,且又是個多情的,若能聚合了,倒是件美事,偏又生出這段事來。這薛家縱比馮家富貴,想其爲人,自然姬妾衆多,淫佚無度,未必及馮淵定情於一人者。這正是夢幻情緣,恰遇一對薄命兒女。\begin{note}甲眉:使雨村一評,方補足上半回之題目。所謂此書有繁處愈繁,省中愈省;又有不怕繁中繁,只有繁中虛;不畏省中省,只要省中實。此則省中實也。\end{note}且不要議論他,只目今這官司,如何剖斷纔好?”門子笑道:“老爺當年何其明決,今日何反成了個沒主意的人了!小的聞得老爺補升此任,亦系賈府王府之力,此薛蟠即賈府之親,老爺何不順水行舟,作個整人情,將此案了結,日後也好去見賈府王府。”雨村道:“你說的何嘗不是。\begin{note}甲側:可發一長嘆。這一句已見奸雄,全是假。\end{note}但事關人命,蒙皇上隆恩,起復委用,\begin{note}甲側:奸雄。\end{note}實是重生再造,正當殫心竭力圖報之時,\begin{note}甲側:奸雄。\end{note}豈可因私而廢法?\begin{note}甲側:奸雄。\end{note}是我實不能忍爲者。”\begin{note}甲側:全是假。\end{note}門子聽了,冷笑道:“老爺說的何嘗不是大道理,但只是如今世上是行不去的。豈不聞古人有云‘大丈夫相時而動’,又曰‘趨吉避凶者爲君子’。\begin{note}甲側:近時錯會書意者多多如此。\end{note}依老爺這一說,不但不能報效朝廷,亦且自身不保,還要三思爲妥。”
\end{parag}


\begin{parag}
    雨村低了半日頭,\begin{note}甲側:奸雄欺人。\end{note}方說道:“依你怎麼樣?”門子道:“小人已想了一個極好的主意在此:老爺明日坐堂,只管虛張聲勢,動文書發籤拿人。原兇自然是拿不來的,原告固是定要將薛家族中及奴僕人等拿幾個來拷問。小的在暗中調停,令他們報個暴病身亡,令族中及地方上共遞一張保呈,老爺只說善能扶鸞請仙,堂上設下乩壇,令軍民人等只管來看。老爺就說:‘乩仙批了,死者馮淵與薛蟠原因夙孽相逢,今狹路既遇,原應了結。薛蟠今已得了無名之症,\begin{note}甲側:“無名之症”卻是病之名,而反曰“無”,妙極!\end{note}被馮魂追索已死。其禍皆因柺子某人而起,拐之人原系某鄉某姓人氏,按法處治,餘不略及’等語。小人暗中囑託柺子,令其實招。衆人見乩仙批語與柺子相符,餘者自然也都不虛了。薛家有的是錢,老爺斷一千也可,五百也可,與馮家作燒埋之費。那馮家也無甚要緊的人,不過爲的是錢,見有了這個銀子,想來也就無話了。老爺細想此計如何?”雨村笑道:“不妥,不妥。\begin{note}甲側:奸雄欺人。\end{note}等我再 斟酌斟酌,或可壓服口聲。”二人計議,天色已晚,別無話說。
\end{parag}


\begin{parag}
    至次日坐堂,勾取一應有名人犯,雨村詳加審問,果見馮家人口稀疏,不過賴此欲多得些燒埋之費,\begin{note}甲側:因此三四語收住,極妙!此則重重寫來,輕輕抹去也。\end{note}薛家仗勢倚情,偏不相讓,故致顛倒未決。雨村便徇情枉法,胡亂判斷了此案。\begin{note}甲側:實注一筆,更好。不過是如此等事,又何用細寫。可謂此書不敢幹涉廊廟者,即此等處也,莫謂寫之不到。蓋作者立意寫閨閣尚不暇,何能又及此等哉!\end{note}馮家得了許多燒埋銀子,也就無甚話說了。\begin{note}甲眉:蓋寶釵一家不得不細寫者。若另起頭緒,則文字死板,故仍只借雨村一人穿插出阿呆兄人命一事,且又帶敘出英蓮一向之行蹤,並以後之歸結,是以故意戲用“葫蘆僧亂判”等字樣,撰成半回,略一解頤,略一嘆世,蓋非有意譏刺仕途,實亦出人之閒文耳。甲眉:又注馮家一筆,更妥。可見馮家正不爲人命,實賴此獲利耳。故用“亂判”二字爲題,雖曰不涉世事,或亦有微詞耳。但其意實欲出寶釵,不得不做此穿插,故云此等皆非《石頭記》之正文。\end{note}雨村斷了此案,急忙作書信二封,與賈政並京營節度使王子騰,\begin{note}甲側:隨筆帶出王家。\end{note}不過說“令甥之事已完,不必過慮”等語。此事皆由葫蘆廟內之沙彌新門子所出,雨村又恐他對人說出當日貧賤時的事來,因此心中大不樂業。\begin{note}甲側:瞧他寫雨村如此,可知雨村終不是大英雄。\end{note}後來到底尋了個不是,遠遠的充發了他才罷。\begin{note}甲側:至此了結葫蘆廟文字。又伏下千里伏線。起用“葫蘆”字樣,收用“葫蘆”字樣,蓋雲一部書皆系葫蘆提之意也,此亦系寓意處。\end{note}
\end{parag}


\begin{parag}
    當下言不著雨村。且說那買了英蓮打死馮淵的薛公子,\begin{note}甲側:本是立意寫此,卻不肯特起頭緒,故意設出“亂判”一段戲文,其中穿插,至此卻淡淡寫來。\end{note}亦系金陵人氏,本是書香繼世之家。只是如今這薛公子幼年喪父,寡母又憐他是個獨根孤種,未免溺愛縱容,遂至老大無成,且家中有百萬之富,現領著內帑錢糧,採辦雜料。這薛公子學名薛蟠,表字文龍,五歲上就性情奢侈,言語傲慢。雖也上過學,不過略識幾字,\begin{note}甲側:這句加於老兄,卻是實寫。\end{note}終日惟有鬥雞走馬,遊山玩水而已。雖是皇商,一應經濟世事,全然不知,不過賴祖父之舊情分,戶部掛虛名,支領錢糧,其餘事體,自有夥計老家人等措辦。寡母王氏乃現任京營節度使王子騰之妹,與榮國府賈政的夫人王氏,是一母所生的姊妹,今年方四十上下年紀,只有薛蟠一子。還有一女,比薛蟠小兩歲,乳名寶釵,生得肌骨瑩潤,舉止嫺雅。\begin{note}甲側:寫寶釵只如此,更妙!\end{note}當日有他父親在日,酷愛此女,令其讀書識字,較之乃兄竟高過十倍。\begin{note}甲側:又只如此寫來,更妙!\end{note}自父親死後,見哥哥不能依貼母懷,他便不以書字爲事,只留心針黹家計等事,好爲母親分憂解勞。近因今上崇詩尚禮,徵採才能,降不世出之隆恩,除聘選妃嬪外,凡仕宦名家之女,皆親名達部,以備選爲公主、郡主入學陪侍,充爲才人、贊善之職。\begin{note}甲側:一段稱功頌德,千古小說中所無。\end{note}二則自薛蟠父親死後,各省中所有的買賣承局,總管、夥計人等,見薛蟠年輕不諳世事,便趁時拐騙起來,京都中幾處生意,漸亦消耗。薛蟠素聞得都中乃第一繁華之地,正思一遊,便趁此機會,一爲送妹待選,二爲望親,三因親自入部銷算舊帳,再計新支,——實則爲遊覽上國風光之意。因此早已打點下行裝細軟,以及饋送親友各色土物人情等類,正擇日一定起身,不想偏遇見了柺子重賣英蓮。薛蟠見英蓮生得不俗,\begin{note}甲側:阿呆兄亦知不俗,英蓮人品可知矣。\end{note}立意買他,又遇馮家來奪人,因恃強喝令手下豪奴將馮淵打死。他便將家中事務一一的囑託了族中人並幾個老家人,他便帶了母妹竟自起身長行去了。人命官司一事,他竟視爲兒戲,自爲花上幾個臭錢,沒有不了的。\begin{note}甲側:是極!人謂薛蟠爲呆,餘則謂是大徹悟。\end{note}
\end{parag}


\begin{parag}
    在路不記其日。\begin{note}甲側:更妙!必雲程限則又有落套,豈暇又記路程單哉?\end{note}那日已將入都時,卻又聞得母舅王子騰昇了九省統制,奉旨出都查邊。薛蟠心中暗喜道:“我正愁進京去有個嫡親的母舅管轄著,不能任意揮霍揮霍,偏如今又升出去了,可知天從人願。”\begin{note}甲側:寫盡五陵心意。\end{note}因和母親商議道:“咱們京中雖有幾處房舍,只是這十來年沒人進京居住,那看守的人未免偷著租賃與人,須得先著幾個人去打掃收拾纔好。”他母親道:“何必如此招搖!咱們這一進京,原該先拜望親友,或是在你舅舅家,\begin{note}甲側:陪筆。\end{note}或是你姨爹家。\begin{note}甲側:正筆。\end{note}他兩家的房舍極是便宜的,咱們先能著住下,再慢慢的著人去收拾,豈不消停些。”薛蟠道:“如今舅舅正升了外省去,家裏自然忙亂起身。咱們這工夫一窩一拖的奔了去,豈不沒眼色。”他母親道:“你舅舅家雖升了去,還有你姨爹家。況這幾年來,你舅舅、姨娘兩處,每每帶信捎書,接咱們來。如今既來了,你舅舅雖忙著起身,你賈家姨娘未必不苦留我們。咱們且忙忙收拾房屋,豈不使人見怪?\begin{note}甲側:閒語中補出許多前文,此畫家之雲罩峯尖法也。\end{note}你的意思我卻知道,\begin{note}甲側:知子莫如父。\end{note}守著舅舅、姨爹住著,未免拘緊了你,不如你各自住著,好任意施爲。\begin{note}甲側:寡母孤兒一段,寫得畢肖畢真。\end{note}你既如此,你自去挑所宅子去住。我和你姨娘,姊妹們別了這幾年,卻要廝守幾日,我帶了你妹子投你姨娘家去,\begin{note}甲側:薛母亦善訓子。\end{note}你道好不好?”薛蟠見母親如此說,情知扭不過的,只得吩咐人夫一路奔榮國府來。
\end{parag}


\begin{parag}
    那時王夫人已知薛蟠官司一事,虧賈雨村維持了結,才放了心。又見哥哥升了邊缺,正愁又少了孃家的親戚來往,\begin{note}甲側:大家尚義,人情大都是也。\end{note}略加寂寞。過了幾日,忽家人傳報:“姨太太帶了哥兒姐兒,閤家進京,正在門外下車。”喜的王夫人忙帶了女媳人等,接出大廳,將薛姨媽等接了進去。姊妹們暮年相會,自不必說悲喜交集,泣笑敘闊一番。忙又引了拜見賈母,將人情土物各種酬獻了,閤家具廝見過,忙又治席接風。
\end{parag}


\begin{parag}
    薛蟠已拜見過賈政,賈璉又引著拜見了賈赦,賈珍等。賈政便使人上來對王夫人說:“姨太太已有了春秋,外甥年輕不知世路,在外住著恐有人生事。咱們東北角上梨香院\begin{note}甲側:好香色。\end{note}一所十來間房,白空閒著,打掃了,請姨太太和姐兒哥兒住了甚好。”\begin{note}甲眉:用政老一段,不但王夫人得體,且薛母亦免靠親之嫌。\end{note}王夫人未及留,賈母也就遣人來說“請姨太太就在這裏住下,大家親密些”等語。\begin{note}甲側:老太君口氣得情。偏不寫王夫人留,方不死板。\end{note}薛姨媽正要同居一處,方可拘緊些兒子,若另住在外,又恐他縱性惹禍,遂忙道謝應允。又私與王夫人說明:“一應日費供給一概免卻,\begin{note}甲側:作者題清,猶恐看官誤認今之靠親投友者一例。\end{note}方是處常之法。”王夫人知他家不難於此,遂亦從其願。從此後,薛家母子就在梨香院住了。
\end{parag}


\begin{parag}
    原來這梨香院即當日榮公暮年養靜之所,小小巧巧,約有十餘間房屋,前廳後舍俱全。另有一門通街,薛蟠家人就走此門出入。西南有一角門,通一夾道,出夾道便是王夫人正房的東邊了。每日或飯後,或晚間,薛姨媽便過來,或與賈母閒談,或與王夫人相敘。寶釵日與黛玉迎春姊妹等一處,\begin{note}甲眉:金玉初見,卻如此寫,虛虛實實,總不相犯。\end{note}或看書下棋,或作針黹,倒也十分樂業。\begin{note}甲側:這一句襯出後文黛玉之不能樂業,細甚妙甚!\end{note}只是薛蟠起初之心,原不欲在賈宅居住者,但恐姨父管約拘禁,料必不自在的,無奈母親執意在此,且宅中又十分殷勤苦留,只得暫且住下,一面使人打掃出自己的房屋,再移居過去的。\begin{note}甲側:交代結構,曲曲折折,筆墨盡矣。\end{note}誰知自從在此住了不上一月的光景,賈宅族中凡有的子侄,俱已認熟了一半,凡是那些紈絝氣習者,莫不喜與他來往,今日會酒,明日觀花,甚至聚賭嫖娼,漸漸無所不至,引誘的薛蟠比當日更壞了十倍。\begin{note}甲側:雖說爲紈絝設鑑,其意原只罪賈宅,故用此等句法寫來。此等人家豈必欺霸方始成名耶?總因子弟不肖,招接匪人,一朝生事則百計營求,父爲子隱,羣小迎合,雖暫時不罹禍,而從此放膽,必破家滅族不已,哀哉!\end{note}雖然賈政訓子有方,治家有法,\begin{note}甲側:八字特洗出政老來,又是作者隱意。\end{note}一則族大人多,照管不到這些,二則現任族長乃是賈珍,彼乃寧府長孫,又現襲職,凡族中事,自有他掌管,三則公私冗雜,且素性瀟灑,不以俗務爲要,每公暇之時,不過看書著棋而已,餘事多不介意。況且這梨香院相隔兩層房舍,又有街門另開,任意可以出入,所以這些子弟們竟可以放意暢懷的,因此,薛蟠遂將移居之念,漸漸打滅了。
\end{parag}


\begin{parag}
    \begin{note}夢:正是:\end{note}
\end{parag}


\begin{parag}
    \begin{note}漸入鮑魚肆,反惡芝蘭香。\end{note}
\end{parag}

