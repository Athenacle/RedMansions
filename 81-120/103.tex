\chap{一百零三}{施毒計金桂自焚身 昧真禪雨村空遇舊}



\begin{parag}
    話說賈璉到了王夫人那邊,一一的說了。次日到了部裏打點停妥,回來又到王夫人那邊,將打點吏部之事告知。王夫人便道:“打聽準了麼?果然這樣,老爺也願意,閤家也放心。那外任是何嘗做得的!若不是那樣的參回來,只怕叫那些混賬東西把老爺的性命都坑了呢!”賈璉道:“太太那裏知道?”王夫人道:“自從你二叔放了外任,並沒有一個錢拿回來,把家裏的倒掏摸了好些去了。你瞧那些跟老爺去的人,他男人在外頭不多幾時,那些小老婆子們便金頭銀面的妝扮起來了,可不是在外頭瞞着老爺弄錢?你叔叔便由着他們鬧去,若弄出事來,不但自己的官做不成,只怕連祖上的官也要抹掉了呢。”賈璉道:“嬸子說得很是。方纔我聽見參了,嚇的了不得,直等打聽明白才放心。也願意老爺做個京官,安安逸逸的做幾年,才保得住一輩子的聲名。就是老太太知道了,倒也是放心的,只要太太說得寬緩些。”王夫人道:“我知道。你到底再去打聽打聽。”
\end{parag}


\begin{parag}
    賈璉答應了,纔要出來,只見薛姨媽家的老婆子慌慌張張的走來,到王夫人裏間屋內,也沒說請安,便道:“我們太太叫我來告訴這裏的姨太太,說我們家了不得了,又鬧出事來了。”王夫人聽了,便問:“鬧出什麼事來?”那婆子又說:“了不得,了不得!”王夫人哼道:“糊塗東西!有要緊事你到底說啊!”婆子便說:“我們家二爺不在家,一個男人也沒有。這件事情出來怎麼辦!要求太太打發幾位爺們去料理料理。”王夫人聽着不懂,便急着道:“究竟要爺們去幹什麼事?”婆子道:“我們大奶奶死了。”王夫人聽了,便啐道:“這種女人死,死了罷咧,也值得大驚小怪的!”婆子道:“不是好好兒死的,是混鬧死的。快求太太打發人去辦辦。”說着就要走。王夫人又生氣,又好笑,說:“這婆子好混賬。璉哥兒,倒不如你過去瞧瞧,別理那糊塗東西。”那婆子沒聽見打發人去,只聽見說別理他,他便賭氣跑回去了。這裏薛姨媽正在着急,再等不來,好容易見那婆子來了,便問:“姨太太打發誰來?”婆子嘆說道:“人最不要有急難事,什麼好親好眷,看來也不中用。姨太太不但不肯照應我們,倒罵我糊塗。”薛姨媽聽了,又氣又急道:“姨太太不管,你姑奶奶怎麼說了?”婆子道:“姨太太既不管,我們家的姑奶奶自然更不管了。沒有去告訴。”薛姨媽啐道:“姨太太是外人,姑娘是我養的,怎麼不管!”婆子一時省悟道:“是啊,這麼着我還去。”
\end{parag}


\begin{parag}
    正說着,只見賈璉來了,給薛姨媽請了安,道了惱,回說:“我嬸子知道弟婦死了,問老婆子,再說不明,着急得很,打發我來問個明白,還叫我在這裏料理。該怎麼樣,姨太太只管說了辦去。”薛姨媽本來氣得乾哭,聽見賈璉的話,便笑着說:“倒要二爺費心。我說姨太太是待我們最好的,都是這老貨說不清,幾乎誤了事。請二爺坐下,等我慢慢的告訴你。”便說:“不爲別的事,爲的是媳婦不是好死的。”賈璉道:“想是爲兄弟犯事怨命死的?”薛姨媽道:“若這樣倒好了。前幾個月頭裏,他天天蓬頭赤腳的瘋鬧。後來聽見你兄弟問了死罪,他雖哭了一場,以後倒擦脂抹粉的起來。我若說他,又要吵個了不得,我總不理他。有一天不知怎麼樣來要香菱去作伴,我說:‘你放着寶蟾,還要香菱做什麼,況且香菱是你不愛的,何苦招氣生。’他必不依。我沒法兒,便叫香菱到他屋裏去。可憐這香菱不敢違我的話,帶着病就去了。誰知道他待香菱很好,我倒喜歡。你大妹妹知道了,說:‘只怕不是好心罷。’我也不理會。頭幾天香菱病着,他倒親手去做湯給他喫,那知香菱沒福,剛端到跟前,他自己燙了手,連碗都砸了。我只說必要遷怒在香菱身上,他倒沒生氣,自己還拿笤帚掃了,拿水潑淨了地,仍舊兩個人很好。昨兒晚上,又叫寶蟾去做了兩碗湯來,自己說同香菱一塊兒喝。隔了一回,聽見他屋裏兩隻腳蹬響,寶蟾急的亂嚷,以後香菱也嚷着扶着牆出來叫人。我忙着看去,只見媳婦鼻子眼睛裏都流出血來,在地下亂滾,兩手在心口亂抓,兩腳亂蹬,把我就嚇死了,問他也說不出來,只管直嚷,鬧了一回就死了。我瞧那光景是服了毒的。寶蟾便哭着來揪香菱,說他把藥藥死了奶奶了。我看香菱也不是這麼樣的人,再者他病的起還起不來,怎麼能藥人呢。無奈寶蟾一口咬定。我的二爺,這叫我怎麼辦!只得硬着心腸叫老婆子們把香菱捆了,交給寶蟾,便把房門反扣了。我同你二妹妹守了一夜,等府裏的門開了才告訴去的。二爺你是明白人,這件事怎麼好?”賈璉道:“夏家知道了沒有?”薛姨媽道:“也得撕擄明白了纔好報啊。”賈璉道:“據我看起來,必要經官才了得下來。我們自然疑在寶蟾身上,別人便說寶蟾爲什麼藥死他奶奶,也是沒答對的。若說在香菱身上,竟還裝得上。”正說着,只見榮府女人們進來說:“我們二奶奶來了。”賈璉雖是大伯子,因從小兒見的,也不迴避。寶釵進來見了母親,又見了賈璉,便往裏間屋裏同寶琴坐下。薛姨媽也將前事告訴一遍。寶釵便說:“若把香菱捆了,可不是我們也說是香菱藥死的了麼?媽媽說這湯是寶蟾做的,就該捆起寶蟾來問他呀。一面便該打發人報夏家去,一面報官的是。”薛姨媽聽見有理,便問賈璉。賈璉道:“二妹子說得很是。報官還得我去,託了刑部裏的人,相驗問口供的時候有照應得。只是要捆寶蟾放香菱倒怕難些。”薛姨媽道:“並不是我要捆香菱,我恐怕香菱病中受怨着急,一時尋死,又添了一條人命,才捆了交給寶蟾,也是一個主意。”賈璉道:“雖是這麼說,我們倒幫了寶蟾了。若要放都放,要捆都捆,他們三個人是一處的。只要叫人安慰香菱就是了。”薛姨媽便叫人開門進去,寶釵就派了帶來幾個女人幫着捆寶蟾。只見香菱已哭得死去活來,寶蟾反得意洋洋。以後見人要捆他,便亂嚷起來。那禁得榮府的人吆喝着,也就捆了。竟開着門,好叫人看着。這裏報夏家的人已經去了。
\end{parag}


\begin{parag}
    那夏家先前不住在京裏,因近年消索,又記掛女兒,新近搬進京來。父親已沒,只有母親,又過繼了一個混賬兒子,把家業都花完了,不時的常到薛家。那金桂原是個水性人兒,那裏守得住空房,況兼天天心裏想念薛蝌,便有些飢不擇食的光景。無奈他這一干兄弟又是個蠢貨,雖也有些知覺,只是尚未入港。所以金桂時常回去,也幫貼他些銀錢。這些時正盼金桂回家,只見薛家的人來,心裏就想又拿什麼東西來了。不料說這裏姑娘服毒死了,他便氣得亂嚷亂叫。金桂的母親聽見了,更哭喊起來,說:“好端端的女孩兒在他家,爲什麼服了毒呢!”哭着喊着的,帶了兒子,也等不得僱車,便要走來。那夏家本是買賣人家,如今沒了錢,那顧什麼臉面。兒子頭裏就走,他跟了一個破老婆子出了門,在街上啼啼哭哭的僱了一輛破車,便跑到薛家。
\end{parag}


\begin{parag}
    進門也不打話,便兒一聲肉一聲的要討人命。那時賈璉到刑部託人,家裏只有薛姨媽,寶釵,寶琴,何曾見過個陣仗,都嚇得不敢則聲。便要與他講理,他們也不聽,只說:“我女孩兒在你家得過什麼好處,兩口朝打暮罵的。鬧了幾時,還不容他兩口子在一處,你們商量着把女婿弄在監裏,永不見面。你們娘兒們仗着好親戚受用也罷了,還嫌他礙眼,叫人藥死了他,倒說是服毒!他爲什麼服毒!”說着,直奔着薛姨媽來。薛姨媽只得後退,說:“親家太太且請瞧瞧你女兒,問問寶蟾,再說歪話不遲。”那寶釵寶琴因外面有夏家的兒子,難以出來攔護,只在裏邊着急。恰好王夫人打發周瑞家的照看,一進門來,見一個老婆子指着薛姨媽的臉哭罵。周瑞家的知道必是金桂的母親,便走上來說:“這位是親家太太麼?大奶奶自己服毒死的,與我們姨太太什麼相干,也不犯這麼遭塌呀。”那金桂的母親問:“你是誰?”薛姨媽見有了人,膽子略壯了些,便說:“這就是我親戚賈府裏的。”金桂的母親便說道:“誰不知道,你們有仗腰子的親戚,才能夠叫姑爺坐在監裏。如今我的女孩兒倒白死了不成!”說着,便拉薛姨媽說:“你到底把我女兒怎樣弄殺了?給我瞧瞧!”周瑞家的一面勸說:“只管瞧瞧,用不着拉拉扯扯。”便把手一推。夏家的兒子便跑進來不依道:“你仗着府裏的勢頭兒來打我母親麼!”說着,便將椅子打去,卻沒有打着。裏頭跟寶釵的人聽見外頭鬧起來,趕着來瞧,恐怕周瑞家的喫虧,齊打夥的上去半勸半喝。那夏家的母子索性撒起潑來,說:“知道你們榮府的勢頭兒。我們家的姑娘已經死了,如今也都不要命了!”說着,仍奔薛姨媽拼命。地下的人雖多,那裏擋得住,自古說的“一人拼命,萬夫莫當。”
\end{parag}


\begin{parag}
    正鬧到危急之際,賈璉帶了七八個家人進來,見是如此,便叫人先把夏家的兒子拉出去,便說:“你們不許鬧,有話好好兒的說。快將家裏收拾收拾,刑部裏頭的老爺們就來相驗了。”金桂的母親正在撒潑,只見來了一位老爺,幾個在頭裏吆喝,那些人都垂手侍立。金桂的母親見這個光景,也不知是賈府何人,又見他兒子已被人揪住,又聽見說刑部來驗,他心裏原想看見女兒屍首先鬧了一個稀爛再去喊官去,不承望這裏先報了官,也便軟了些。薛姨媽已嚇糊塗了。還是周瑞家的回說:“他們來了,也沒有去瞧他姑娘,便作踐起姨太太來了。我們爲好勸他,那裏跑進一個野男人,在奶奶們裏頭混撒村混打,這可不是沒有王法了!”賈璉道:“這回子不用和他講理,等一會子打着問他,說:男人有男人的所在,裏頭都是些姑娘奶奶們,況且有他母親還瞧不見他們姑娘麼,他跑進來不是要打搶來了麼!”家人們做好做歹壓伏住了。周瑞家的仗着人多,便說:“夏太太,你不懂事,既來了,該問個青紅皁白。你們姑娘是自己服毒死了,不然便是寶蟾藥死他主子了,怎麼不問明白,又不看屍首,就想訛人來了呢,我們就肯叫一個媳婦兒白死了不成!現在把寶蟾捆着,因爲你們姑娘必要點病兒,所以叫香菱陪着他,也在一個屋裏住,故此兩個人都看守在那裏,原等你們來眼看看刑部相驗,問出道理來纔是啊。”
\end{parag}


\begin{parag}
    金桂的母親此時勢孤,也只得跟着周瑞家的到他女孩兒屋裏,只見滿臉黑血,直挺挺的躺在炕上,便叫哭起來。寶蟾見是他家的人來,便哭喊說:“我們姑娘好意待香菱,叫他在一塊兒住,他倒抽空兒藥死我們姑娘!”那時薛家上下人等俱在,便齊聲吆喝道:“胡說,昨日奶奶喝了湯才藥死的,這湯可不是你做的!”寶蟾道:“湯是我做的,端了來我有事走了,不知香菱起來放些什麼在裏頭藥死的。”金桂的母親聽未說完,就奔香菱。衆人攔住。薛姨媽便道:“這樣子是砒霜藥的,家裏決無此物。不管香菱寶蟾,終有替他買的,回來刑部少不得問出來,才賴不去。如今把媳婦權放平正,好等官來相驗。”衆婆子上來抬放。寶釵道:“都是男人進來,你們將女人動用的東西檢點檢點。”只見炕褥底下有一個揉成團的紙包兒。金桂的母親瞧見便拾起,打開看時,並沒有什麼,便撩開了。寶蟾看見道:“可不是有了憑據了。這個紙包兒我認得,頭幾天耗子鬧得慌,奶奶家去與舅爺要的,拿回來擱在首飾匣內,必是香菱看見了拿來藥死奶奶的。若不信,你們看看首飾匣裏有沒有了。”
\end{parag}


\begin{parag}
    金桂的母親便依着寶蟾的所在取出匣子,只有幾支銀簪子。薛姨媽便說:“怎麼好些首飾都沒有了?”寶釵叫人打開箱櫃,俱是空的,便道:“嫂子這些東西被誰拿去,這可要問寶蟾。”金桂的母親心裏也虛了好些,見薛姨媽查問寶蟾,便說:“姑娘的東西他那裏知道。”周瑞家的道:“親家太太別這麼說呢。我知道寶姑娘是天天跟着大奶奶的,怎麼說不知!”這寶蟾見問得緊,又不好胡賴,只得說道:“奶奶自己每每帶回家去,我管得麼。”衆人便說:“好個親家太太!哄着拿姑娘的東西,哄完了叫他尋死來訛我們。好罷了,回來相驗便是這麼說。”寶釵叫人:“到外頭告訴璉二爺說,別放了夏家的人。”
\end{parag}


\begin{parag}
    裏面金桂的母親忙了手腳,便罵寶蟾道:“小蹄子別嚼舌頭了!姑娘幾時拿東西到我家去。”寶蟾道:“如今東西是小,給姑娘償命是大。”寶琴道:“有了東西就有償命的人了。快請璉二哥哥問準了夏家的兒子買砒霜的話,回來好回刑部裏的話。”金桂的母親着了急道:“這寶蟾必是撞見鬼了,混說起來。我們姑娘何嘗買過砒霜。若這麼說,必是寶蟾藥死了的。”寶蟾急的亂嚷說:“別人賴我也罷了,怎麼你們也賴起我來呢!你們不是常和姑娘說,叫他別受委屈,鬧得他們家破人亡,那時將東西捲包兒一走,再配一個好姑爺。這個話是有的沒有?”金桂的母親還未及答言,周瑞家的便接口說道:“這是你們家的人說的,還賴什麼呢。”金桂的母親恨的咬牙切齒的罵寶蟾說:“我待你不錯呀,爲什麼你倒拿話來葬送我呢!回來見了官,我就說是你藥死姑娘的。”寶蟾氣得瞪着眼說:“請太太放了香菱罷,不犯着白害別人。我見官自有我的話。”
\end{parag}


\begin{parag}
    寶釵聽出這個話頭兒來了,便叫人反倒放開了寶蟾,說:“你原是個爽快人,何苦白冤在裏頭。你有話索性說了,大家明白,豈不完了事了呢。”寶蟾也怕見官受苦,便說:“我們奶奶天天抱怨說:‘我這樣人,爲什麼碰着這個瞎眼的娘,不配給二爺,偏給了這麼個混賬糊塗行子。要是能夠同二爺過一天,死了也是願意的。’說到那裏,便恨香菱。我起初不理會,後來看見與香菱好了,我只道是香菱教他什麼了,不承望昨兒的湯不是好意。”金桂的母親接說道:“益發胡說了,若是要藥香菱,爲什麼倒藥了自己呢?”寶釵便問道:“香菱,昨日你喝湯來着沒有?”香菱道:“頭幾天我病得抬不起頭來,奶奶叫我喝湯,我不敢說不喝,剛要扎掙起來,那碗湯已經灑了,倒叫奶奶收拾了個難,我心裏很過不去。昨兒聽見叫我喝湯,我喝不下去,沒有法兒正要喝的時候兒呢,偏又頭暈起來。只見寶蟾姐姐端了去,我正喜歡,剛合上眼,奶奶自己喝着湯,叫我嚐嚐,我便勉強也喝了。”寶蟾不待說完,便道:“是了,我老實說罷。昨兒奶奶叫我做兩碗湯,說是和香菱同喝。我氣不過,心裏想着香菱那裏配我做湯給他喝呢。我故意的一碗裏頭多抓了一把鹽,記了暗記兒,原想給香菱喝的。剛端進來,奶奶卻攔着我到外頭叫小子們僱車,說今日回家去。我出去說了,回來見鹽多的這碗湯在奶奶跟前呢,我恐怕奶奶喝着鹹,又要罵我。正沒法的時候,奶奶往後頭走動,我眼錯不見就把香菱這碗湯換了過來。也是合該如此,奶奶回來就拿了湯去到香菱牀邊喝着,說:‘你到底嚐嚐。’那香菱也不覺鹹。兩個人都喝完了。我正笑香菱沒嘴道兒,那裏知道這死鬼奶奶要藥香菱,必定趁我不在將砒霜撒上了,也不知道我換碗,這可就是天理昭彰,自害其身了。”於是衆人往前後一想,真正一絲不錯,便將香菱也放了,扶着他仍舊睡在牀上。
\end{parag}


\begin{parag}
    不說香菱得放,且說金桂母親心虛事實,還想辯賴。薛姨媽等你言我語,反要他兒子償還金桂之命。正然吵嚷,賈璉在外嚷說:“不用多說了,快收拾停當,刑部老爺就到了。”此時惟有夏家母子着忙,想來總要喫虧的,不得已反求薛姨媽道:“千不是萬不是,終是我死的女孩兒不長進,這也是自作自受。若是刑部相驗,到底府上臉面不好看。求親家太太息了這件事罷。”寶釵道:“那可使不得,已經報了,怎麼能息呢。”周瑞家的等人大家做好做歹的勸說:“若要息事,除非夏親家太太自己出去攔驗,我們不提長短罷了。”賈璉在外也將他兒子嚇住,他情願迎到刑部具結攔驗。衆人依允。薛姨媽命人買棺成殮。不提。
\end{parag}


\begin{parag}
    且說賈雨村升了京兆府尹兼管稅務,一日出都查勘開墾地畝,路過知機縣,到了急流津。正要渡過彼岸,因待人夫,暫且停轎。只見村旁有一座小廟,牆壁坍頹,露出幾株古松,倒也蒼老。雨村下轎,閒步進廟,但見廟內神像金身脫落,殿宇歪斜,旁有斷碣,字跡模糊,也看不明白。意欲行至後殿,只見一翠柏下蔭着一間茅廬,廬中有一個道士閤眼打坐。雨村走近看時,面貌甚熟,想着倒象在那裏見來的,一時再想不出來。從人便欲吆喝。雨村止住,徐步向前叫一聲:“老道。”那道士雙眼微啓,微微的笑道:“貴官何事?”雨村便道:“本府出都查勘事件,路過此地,見老道靜修自得,想來道行深通,意欲冒昧請教。”那道人說:“來自有地,去自有方。”雨村知是有些來歷的,便長揖請問:“老道從何處修來,在此結廬?此廟何名?廟中共有幾人?或欲真修,豈無名山,或欲結緣,何不通衢?”那道人道:“葫蘆尚可安身,何必名山結舍。廟名久隱,斷碣猶存。形影相隨,何須修募。豈似那‘玉在櫝中求善價,釵於奩內待時飛’之輩耶!”
\end{parag}


\begin{parag}
    雨村原是個穎悟人,初聽見“葫蘆”兩字,後聞“玉釵”一對,忽然想起甄士隱的事來。重複將那道士端詳一回,見他容貌依然,便屏退從人,問道:“君家莫非甄老先生麼?”那道人從容笑道:“什麼真,什麼假!要知道真即是假,假即是真。”雨村聽說出賈字來,益發無疑,便從新施禮道:“學生自蒙慨贈到都,託庇獲雋公交車,受任貴鄉,始知老先生超悟塵凡,飄舉仙境。學生雖溯洄思切,自念風塵俗吏,未由再覲仙顏。今何幸於此處相遇,求老仙翁指示愚蒙。倘荷不棄,京寓甚近,學生當得供奉,得以朝夕聆教。”那道人也站起來回禮道:“我於蒲團之外,不知天地間尚有何物。適才尊官所言,貧道一概不解。”說畢,依舊坐下。雨村復又心疑:“想去若非士隱,何貌言相似若此?離別來十九載,面色如舊,必是修煉有成,未肯將前身說破。但我既遇恩公,又不可當面錯過。看來不能以富貴動之,那妻女之私更不必說了。”想罷又道:“仙師既不肯說破前因,弟子於心何忍!”正要下禮,只見從人進來,稟說天色將晚,快請渡河。雨村正無主意,那道人道:“請尊官速登彼岸,見面有期,遲則風浪頓起。果蒙不棄,貧道他日尚在渡頭候教。”說畢,仍閤眼打坐。雨村無奈,只得辭了道人出廟。正要過渡,只見一人飛奔而來。未知何事,下回分解。
\end{parag}