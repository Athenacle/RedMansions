\chap{八十九}{人亡物在公子填词 蛇影杯弓颦卿绝粒}


\begin{parag}
    却说凤姐正自起来纳闷,忽听见小丫头这话,又唬了一跳,连忙问道:“什么官事?”小丫头道:“也不知道。刚才二门上小厮回进来,回老爷有要紧的官事,所以太太叫我请二爷来了。”凤姐听是工部里的事,才把心略略的放下,因说道:“你回去回太太,就说二爷昨日晚上出城有事,没有回来。打发人先回珍大爷去罢。”那丫头答应着去了。
\end{parag}


\begin{parag}
    一时贾珍过来见了部里的人,问明了,进来见了王夫人,回道:“部中来报,昨日总河奏到河南一带决了河口,湮没了几府州县。又要开销国帑,修理城工。工部司官又有一番照料,所以部里特来报知老爷的。”说完退出,及贾政回家来回明。从此直到冬间,贾政天天有事,常在衙门里。宝玉的工课也渐渐松了,只是怕贾政觉察出来,不敢不常在学房里去念书,连黛玉处也不敢常去。
\end{parag}


\begin{parag}
    那时已到十月中旬,宝玉起来要往学房中去。这日天气陡寒,只见袭人早已打点出一包衣服,向宝玉道:“今日天气很冷,早晚宁使暖些。”说着,把衣服拿出来给宝玉挑了一件穿。又包了一件,叫小丫头拿出交给焙茗,嘱咐道:“天气凉,二爷要换时,好生预备着。”焙茗答应了,抱着毡包,跟着宝玉自去。宝玉到了学房中,做了自己的工课,忽听得纸窗呼喇喇一派风声。代儒道:“天气又发冷。”把风门推开一看,只见西北上一层层的黑云渐渐往东南扑上来。焙茗走进来回宝玉道:“二爷,天气冷了,再添些衣服罢。”宝玉点点头儿。只见焙茗拿进一件衣服来,宝玉不看则已,看了时神已痴了。那些小学生都巴着眼瞧,却原是晴雯所补的那件雀金裘。宝玉道:“怎么拿这一件来!是谁给你的?”焙茗道:“是里头姑娘们包出来的。”宝玉道:“我身上不大冷,且不穿呢,包上罢。”代儒只当宝玉可惜这件衣服,却也心里喜他知道俭省。焙茗道:“二爷穿上罢,着了凉,又是奴才的不是了。二爷只当疼奴才罢。”宝玉无奈,只得穿上,呆呆的对著书坐着。代儒也只当他看书,不甚理会。晚间放学时,宝玉便往代儒托病告假一天。代儒本来上年纪的人,也不过伴着几个孩子解闷儿,时常也八病九痛的,乐得去一个少操一日心。况且明知贾政事忙,贾母溺爱,便点点头儿。
\end{parag}


\begin{parag}
    宝玉一径回来,见过贾母王夫人,也是这样说,自然没有不信的,略坐一坐便回园中去了。见了袭人等,也不似往日有说有笑的,便和衣躺在炕上。袭人道:“晚饭预备下了,这会儿吃还是等一等儿?”宝玉道:“我不吃了,心里不舒服。你们吃去罢。”袭人道:“那么着你也该把这件衣服换下来了,那个东西那里禁得住揉搓。”宝玉道:“不用换。”袭人道:“倒也不但是娇嫩物儿,你瞧瞧那上头的针线也不该这么糟蹋他呀。”宝玉听了这话,正碰在他心坎儿上,叹了一口气道:“那么着,你就收拾起来给我包好了,我也总不穿他了。”说着,站起来脱下。袭人才过来接时,宝玉已经自己迭起。袭人道:“二爷怎么今日这样勤谨起来了?”宝玉也不答言,迭好了,便问:“包这个的包袱呢?”麝月连忙递过来,让他自己包好,回头却和袭人挤着眼儿笑。宝玉也不理会,自己坐着,无精打彩,猛听架上钟响,自己低头看了看表,针已指到酉初二刻了。一时小丫头点上灯来。袭人道:“你不吃饭,喝一口粥儿罢。别净饿着,看仔细饿上虚火来,那又是我们的累赘了。”宝玉摇摇头儿,说:“不大饿,强吃了倒不受用。”袭人道:“既这么着,就索性早些歇着罢。”于是袭人麝月铺设好了,宝玉也就歇下,翻来复去只睡不着,将及黎明,反朦胧睡去,不一顿饭时,早又醒了。
\end{parag}


\begin{parag}
    此时袭人麝月也都起来。袭人道:“昨夜听着你翻腾到五更多,我也不敢问你。后来我就睡着了,不知到底你睡着了没有?”宝玉道:“也睡了一睡,不知怎么就醒了。”袭人道:“你没有什么不受用?”宝玉道:“没有,只是心上发烦。”袭人道:“今日学房里去不去?”宝玉道:“我昨儿已经告了一天假了,今儿我要想园里逛一天,散散心,只是怕冷。你叫他们收拾一间房子,备下一炉香,搁下纸墨笔砚。你们只管干你们的,我自己静坐半天才好。别叫他们来搅我。”麝月接着道:“二爷要静静儿的用工夫,谁敢来搅。”袭人道:“这么着很好,也省得着了凉。自己坐坐,心神也不散。”因又问:“你既懒待吃饭,今日吃什么?早说好传给厨房里去。”宝玉道:“还是随便罢,不必闹的大惊小怪的。倒是要几个果子搁在那屋里,借点果子香。”袭人道:“那个屋里好?别的都不大干净,只有晴雯起先住的那一间,因一向无人,还干净,就是清冷些。”宝玉道:“不妨,把火盆挪过去就是了。”袭人答应了。正说着,只见一个小丫头端了一个茶盘儿,一个碗,一双牙箸,递给麝月道:“这是刚才花姑娘要的,厨房里老婆子送了来了。”麝月接了一看,却是一碗燕窝汤,便问袭人道:“这是姐姐要的么?”袭人笑道:“昨夜二爷没吃饭,又翻腾了一夜,想来今日早起心里必是发空的,所以我告诉小丫头们叫厨房里作了这个来的。”袭人一面叫小丫头放桌儿,麝月打发宝玉喝了,漱了口。只见秋纹走来说道:“那屋里已经收拾妥了,但等着一时炭劲过了,二爷再进去罢。”宝玉点头,只是一腔心事,懒怠说话。一时小丫头来请,说笔砚都安放妥当了。宝玉道:“知道了。”又一个小丫头回道:“早饭得了。二爷在那里吃?”宝玉道:“就拿了来罢,不必累赘了。”小丫头答应了自去。一时端上饭来,宝玉笑了一笑,向袭人麝月道:“我心里闷得很,自己吃只怕又吃不下去,不如你们两个同我一块儿吃,或者吃的香甜,我也多吃些。”麝月笑道:“这是二爷的高兴,我们可不敢。”袭人道:“其实也使得,我们一处喝酒,也不止今日。只是偶然替你解闷儿还使得,若认真这样,还有什么规矩体统呢。”说着三人坐下。宝玉在上首,袭人麝月两个打横陪着。吃了饭,小丫头端上漱口茶,两个看着撤了下去。宝玉因端着茶,默默如有所思,又坐了一坐,便问道:“那屋里收拾妥了么?”麝月道:“头里就回过了,这回子又问。”
\end{parag}


\begin{parag}
    宝玉略坐了一坐,便过这间屋子来,亲自点了一炷香,摆上些果品,便叫人出去,关上了门。外面袭人等都静悄无声。宝玉拿了一幅泥金角花的粉红笺出来,口中祝了几句,便提起笔来写道:
\end{parag}


\begin{qute2sp}
    怡红主人焚付晴姐知之,酌茗清香,庶几来飨。其词云:随身伴,独自意绸缪。谁料风波平地起,顿教躯命实时休。孰与话轻柔?东逝水,无复向西流。想象更无怀梦草,添衣还见翠云裘。脉脉使人愁!
\end{qute2sp}


\begin{parag}
    写毕,就在香上点个火焚化了。静静儿等着,直待一炷香点尽了,才开门出来。袭人道:“怎么出来了?想来又闷的慌了。”
\end{parag}

\begin{parag}
    宝玉笑了一笑,假说道:“我原是心里烦,才找个地方儿静坐坐儿。这会子好了,还要外头走走去呢。”说着,一径出来,到了潇湘馆中,在院里问道:“林妹妹在家里呢么?”紫鹃接应道:“是谁?”掀帘看时,笑道:“原来是宝二爷。姑娘在屋里呢,请二爷到屋里坐着。”宝玉同着紫鹃走进来。黛玉却在里间呢,说道:“紫鹃,请二爷屋里坐罢。”宝玉走到里间门口,看见新写的一付紫墨色泥金云龙笺的小对,上写着:“绿窗明月在,青史古人空。”宝玉看了,笑了一笑,走入门去,笑问道:“妹妹做什么呢?”黛玉站起来迎了两步,笑着让道:“请坐。我在这里写经,只剩得两行了,等写完了再说话儿。”因叫雪雁倒茶。宝玉道:“你别动,只管写。”说着,一面看见中间挂着一幅单条,上面画着一个嫦娥,带着一个侍者,又一个女仙,也有一个侍者,捧着一个长长儿的衣囊似的,二人身边略有些云护,别无点缀,全仿李龙眠白描笔意,上有”鬪寒图”三字,用八分书写着。宝玉道:“妹妹这幅《鬪寒图》可是新挂上的?”黛玉道:“可不是。昨日他们收拾屋子,我想起来,拿出来叫他们挂上的。”宝玉道:“是什么出处?”黛玉笑道:“眼前熟的很的,还要问人。”宝玉笑道:“我一时想不起,妹妹告诉我罢。”黛玉道:“岂不闻‘青女素娥俱耐冷,月中霜里鬪婵娟’。”宝玉道:“是啊。这个实在新奇雅致,却好此时拿出来挂。”说着,又东瞧瞧,西走走。
\end{parag}


\begin{parag}
    雪雁沏了茶来,宝玉吃着。又等了一会子,黛玉经才写完,站起来道:“简慢了。”宝玉笑道:“妹妹还是这么客气。”但见黛玉身上穿着月白绣花小毛皮袄,加上银鼠坎肩,头上挽着随常云髻,簪上一枝赤金匾簪,别无花朵,腰下系着杨妃色绣花绵裙。真比如:
\end{parag}


\begin{qute2sp}
    亭亭玉树临风立,冉冉香莲带露开。宝玉因问道:“妹妹这两日弹琴来着没有?”黛玉道:“两日没弹了。因为写字已经觉得手冷,那里还去弹琴。”宝玉道:“不弹也罢了。我想琴虽是清高之品,却不是好东西,从没有弹琴里弹出富贵寿考来的,只有弹出忧思怨乱来的。再者弹琴也得心里记谱,未免费心。依我说,妹妹身子又单弱,不操这心也罢了。”黛玉抿着嘴儿笑。宝玉指着壁上道:“这张琴可就是么?怎么这么短?”黛玉笑道:“这张琴不是短,因我小时学抚的时候别的琴都够不着,因此特地做起来的。虽不是焦尾枯桐,这鹤山凤尾还配得齐整,龙池雁足高下还相宜。你看这断纹不是牛旄似的么,所以音韵也还清越。”宝玉道:“妹妹这几天来做诗没有?”黛玉道:“自结社以后没大作。”宝玉笑道:“你别瞒我,我听见你吟的什么‘不可惙,素心如何天上月’,你搁在琴里觉得音响分外的响亮。有的没有?”黛玉道:“你怎么听见了?”宝玉道:“我那一天从蓼风轩来听见的,又恐怕打断你的清韵,所以静听了一会就走了。我正要问你:前路是平韵,到末了儿忽转了仄韵,是个什么意思?”黛玉道:“这是人心自然之音,做到那里就到那里,原没有一定的。”宝玉道:“原来如此。可惜我不知音,枉听了一会子。”黛玉道:“古来知音人能有几个?”宝玉听了。又觉得出言冒失了,又怕寒了黛玉的心,坐了一坐,心里象有许多话,却再无可讲的。黛玉因方才的话也是冲口而出,此时回想,觉得太冷淡些,也就无话。宝玉一发打量黛玉设疑,遂讪讪的站起来说道:“妹妹坐着罢。我还要到三妹妹那里瞧瞧去呢。”黛玉道:“你若是见了三妹妹,替我问候一声罢。”宝玉答应着便出来了。
\end{qute2sp}


\begin{parag}
    黛玉送至屋门口,自己回来闷闷的坐着,心里想道:“宝玉近来说话半吐半吞,忽冷忽热,也不知他是什么意思。”正想着,紫鹃走来道:“姑娘,经不写了?我把笔砚都收好了?”黛玉道:“不写了,收起去罢。”说着,自己走到里间屋里床上歪着,慢慢的细想。紫鹃进来问道:“姑娘喝碗茶罢?”黛玉道:“不喝呢。我略歪歪儿,你们自己去罢。”
\end{parag}


\begin{parag}
    紫鹃答应着出来,只见雪雁一个人在那里发呆。紫鹃走到他跟前问道:“你这会子也有了什么心事了么?”雪雁只顾发呆,倒被他唬了一跳,因说道:“你别嚷,今日我听见了一句话,我告诉你听,奇不奇。你可别言语。”说着,往屋里努嘴儿。因自己先行,点着头儿叫紫鹃同他出来,到门外平台底下,悄悄儿的道:“姐姐你听见了么?宝玉定了亲了!”紫鹃听见,唬了一跳,说道:“这是那里来的话?只怕不真罢。”雪雁道:“怎么不真,别人大概都知道,就只咱们没听见。”紫鹃道:“你是那里听来的?”雪雁道:“我听见侍书说的,是个什么知府家,家资也好,人才也好。”紫鹃正听时,只听得黛玉咳嗽了一声,似乎起来的光景。紫鹃恐怕他出来听见,便拉了雪雁摇摇手儿,往里望望,不见动静,才又悄悄儿的问道:“他到底怎么说来?”雪雁道:“前儿不是叫我到三姑娘那里去道谢吗,三姑娘不在屋里,只有侍书在那里。大家坐着,无意中说起宝二爷的淘气来,他说宝二爷怎么好,只会顽儿,全不象大人的样子,已经说亲了,还是这么呆头呆脑。我问他定了没有,他说是定了,是个什么王大爷做媒的。那王大爷是东府里的亲戚,所以也不用打听,一说就成了。”紫鹃侧着头想了一想,“这句话奇!”又问道:“怎么家里没有人说起?”雪雁道:“侍书也说的是老太太的意思。若一说起,恐怕宝玉野了心,所以都不提起。侍书告诉了我,又叮嘱千万不可露风,说出来只道是我多嘴。”把手往里一指,“所以他面前也不提。今日是你问起,我不犯瞒你。”正说到这里,只听鹦鹉叫唤,学着说:“姑娘回来了,快倒茶来!”倒把紫鹃雪雁吓了一跳,回头并不见有人,便骂了鹦鹉一声,走进屋内。只见黛玉喘吁吁的刚坐在椅子上,紫鹃搭讪着问茶问水。黛玉问道:“你们两个那里去了?再叫不出一个人来。”说着便走到炕边,将身子一歪,仍旧倒在炕上,往里躺下,叫把帐子撩下。紫鹃雪雁答应出去。他两个心里疑惑方才的话只怕被他听了去了,只好大家不提。谁知黛玉一腔心事,又窃听了紫鹃雪雁的话,虽不很明白,已听得了七八分,如同将身撂在大海里一般。思前想后,竟应了前日梦中之谶,千愁万恨,堆上心来。左右打算,不如早些死了,免得眼见了意外的事情,那时反倒无趣。又想到自己没了爹娘的苦,自今以后,把身子一天一天的糟踏起来,一年半载,少不得身登清净。打定了主意,被也不盖,衣也不添,竟是合眼装睡。紫鹃和雪雁来伺候几次,不见动静,又不好叫唤。晚饭都不吃。点灯已后,紫鹃掀开帐子,见已睡着了,被窝都蹬在脚后。怕他着了凉,轻轻儿拿来盖上。黛玉也不动,单待他出去,仍然褪下。那紫鹃只管问雪雁:“今儿的话到底是真的是假的?”雪雁道:“怎么不真。”紫鹃道:“侍书怎么知道的?”雪雁道:“是小红那里听来的。”紫鹃道:“头里咱们说话,只怕姑娘听见了,你看刚才的神情,大有原故。今日以后,咱们倒别提这件事了。”说着,两个人也收拾要睡。紫鹃进来看时,只见黛玉被窝又蹬下来,复又给他轻轻盖上。一宿晚景不提。
\end{parag}


\begin{parag}
    次日,黛玉清早起来,也不叫人,独自一个呆呆的坐着。紫鹃醒来,看见黛玉已起,便惊问道:“姑娘怎么这么早?”黛玉道:“可不是,睡得早,所以醒得早。”紫鹃连忙起来,叫醒雪雁,伺候梳洗。那黛玉对着镜子,只管呆呆的自看。看了一回,那泪珠儿断断连连,早已湿透了罗帕。正是:
\end{parag}


\begin{poem}
    \begin{pl}
        瘦影正临春水照,卿须怜我我怜卿。
    \end{pl}
\end{poem}


\begin{parag}
    紫鹃在旁也不敢劝,只怕倒把闲话勾引旧恨来。迟了好一会,黛玉才随便梳洗了,那眼中泪渍终是不干。又自坐了一会,叫紫鹃道:“你把藏香点上。”紫鹃道:“姑娘,你睡也没睡得几时,如何点香?不是要写经?”黛玉点点头儿。紫鹃道:“姑娘今日醒得太早,这会子又写经,只怕太劳神了罢。”黛玉道:“不怕,早完了早好。况且我也并不是为经,倒借着写字解解闷儿。以后你们见了我的字迹,就算见了我的面儿了。”说着,那泪直流下来。紫鹃听了这话,不但不能再劝,连自己也掌不住滴下泪来。原来黛玉立定主意,自此已后,有意糟踏身子,茶饭无心,每日渐减下来。宝玉下学时,也常抽空问候,只是黛玉虽有万千言语,自知年纪已大,又不便似小时可以柔情挑逗,所以满腔心事,只是说不出来。宝玉欲将实言安慰,又恐黛玉生嗔,反添病症。两个人见了面,只得用浮言劝慰,真真是亲极反疏了。那黛玉虽有贾母王夫人等怜恤,不过请医调治,只说黛玉常病,那里知他的心病。紫鹃等虽知其意,也不敢说。从此一天一天的减,到半月之后,肠胃日薄,一日果然粥都不能吃了。黛玉日间听见的话,都似宝玉娶亲的话,看见怡红院中的人,无论上下,也象宝玉娶亲的光景。薛姨妈来看,黛玉不见宝钗,越发起疑心,索性不要人来看望,也不肯吃药,只要速死。睡梦之中,常听见有人叫宝二奶奶的。一片疑心,竟成蛇影。一日竟是绝粒,粥也不喝,恹恹一息,垂毙殆尽。未知黛玉性命如何,且看下回分解。
\end{parag}