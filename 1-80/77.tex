\chap{七十七}{俏丫鬟抱屈夭風流 美優伶斬情歸水月}


\begin{parag}
    \begin{note}蒙回前總:司棋一事,前文著實寫來,此卻隨筆□去;晴雯一事,前文不過帶敘,此卻竭力發揮。前文借晴雯一襯,文不寂寞,此是借司棋一引,文愈曲折。\end{note}
\end{parag}


\begin{parag}
    話說王夫人見中秋已過,鳳姐病已比先減了,雖未大愈,可以出入行走得了,仍命大夫每日診脈服藥,又開了丸藥方子來配調經養榮丸。因用上等人蔘二兩,王夫人取時,翻尋了半日,只向小匣內尋了幾枝簪挺粗細的。王夫人看了嫌不好,命再找去,又找了一大包須末出來。王夫人焦躁道:“用不著偏有,但用著了,再找不著。成日家我說叫你們查一查,都歸攏在一處。你們白不聽,就隨手混撂。你們不知他的好處,用起來得多少換買來還不中使呢。”彩雲道:“想是沒了,就只有這個。上次那邊的太太來尋了些去,太太都給過去了。”王夫人道:“沒有的話,你再細找找。”彩雲只得又去找,拿了幾包藥材來說:“我們不認得這個,請太太自看。除這個再沒有了。”王夫人打開看時,也都忘了,不知都是什麼藥,並沒有一枝人蔘。因一面遣人去問鳳姐有無,鳳姐來說:“也只有些參膏蘆須。雖有幾枝,也不是上好的,每日還要煎藥裏用呢。”王夫人聽了,只得向邢夫人那裏問去。邢夫人說:“因上次沒了,才往這裏來尋,早已用完了。”王夫人沒法,只得親身過來請問賈母。賈母忙命鴛鴦取出當日所餘的來,竟還有一大包,皆有手指頭粗細的,遂稱二兩與王夫人。王夫人出來交與周瑞家的拿去令小廝送與醫生家去,又命將那幾包不能辨得的藥也帶了去,命醫生認了,各包記號了來。\begin{note}庚雙夾:此等家常細事豈是揣摩得……此皆……者。\end{note}
\end{parag}


\begin{parag}
    一時,周瑞家的又拿了進來說:“這幾包都各包好記上名字了。但這一包人蔘固然是上好的,如今就連三十換也不能得這樣的了,但年代太陳了。這東西比別的不同,憑是怎樣好的,只過一百年後,便自己就成了灰了。如今這個雖未成灰,然已成了朽糟爛木,也無性力的了。請太太收了這個,倒不拘粗細,好歹再換些新的倒好。”王夫人聽了,低頭不語,半日才說:“這可沒法了,只好去買二兩來罷。”也無心看那些,只命:“都收了罷。”因向周瑞家的說:“你就去說給外頭人們,揀好的換二兩來。倘一時老太太問,你們只說用的是老太太的,不必多說。”周瑞家的方纔要去時,寶釵因在坐,乃笑道:“姨娘且住。如今外頭賣的人蔘都沒好的。雖有一枝全的,他們也必截做兩三段,鑲嵌上蘆泡須枝,摻勻了好賣,看不得粗細。我們鋪子裏常和參行交易,如今我去和媽說了,叫哥哥去託個夥計過去和參行商議說明,叫他把未作的原枝好參兌二兩來。不妨咱們多使幾兩銀子,也得了好的。”王夫人笑道:“倒是你明白。就難爲你親自走一趟更好。”於是寶釵去了,半日回來說:“已遣人去,趕晚就有回信的。明日一早去配也不遲。”王夫人自是喜悅,因說道:“‘賣油的娘子水梳頭’,自來家裏有好的,不知給了人多少。這會子輪到自己用,反倒各處求人去了。”說畢長嘆。寶釵笑道:“這東西雖然值錢,究竟不過是藥,原該濟衆散人才是。咱們比不得那沒見世面的人家,得了這個,就珍藏密斂的。”\begin{note}庚雙夾:調侃語。\end{note}王夫人點頭道:“這話極是。”
\end{parag}


\begin{parag}
    一時寶釵去後,因見無別人在室,遂喚周瑞家的來問前日園中搜檢的事情可得個下落。周瑞家的是已和鳳姐等人商議停妥,一字不隱,遂回明王夫人。王夫人聽了,雖驚且怒,卻又作難,因思司棋系迎春之人,皆系那邊的人,只得令人去回邢夫人。周瑞家的回道:“前日那邊太太嗔著王善保家的多事,打了幾個嘴巴子,如今他也裝病在家,不肯出頭了。況且又是他外孫女兒,自己打了嘴,他只好裝個忘了,日久平服了再說。如今我們過去回時,恐怕又多心,倒象似咱們多事似的。不如直把司棋帶過去,一併連贓證與那邊太太瞧了,不過打一頓配了人,再指個丫頭來,豈不省事。如今白告訴去,那邊太太再推三阻四的,又說‘既這樣你太太就該料理,又來說什麼’,豈不反耽擱了。倘那丫頭瞅空尋了死,反不好了。如今看了兩三天,人都有個偷懶的時候,倘一時不到,豈不倒弄出事來。”王夫人想了一想,說:“這也倒是。快辦了這一件,再辦咱們家的那些妖精。”
\end{parag}


\begin{parag}
    周瑞家的聽說,會齊了那幾個媳婦,先到迎春房裏,回迎春道:“太太們說了,司棋大了,連日他娘求了太太,太太已賞了他娘配人,今日叫他出去,另挑好的與姑娘使。”說著,便命司棋打點走路。迎春聽了,含淚似有不捨之意,因前夜已聞得別的丫鬟悄悄的說了原故,雖數年之情難捨,但事關風化,亦無可如何了。那司棋也曾求了迎春,實指望迎春能死保赦下的,只是迎春語言遲慢,耳軟心活,是不能作主的。司棋見了這般,知不能免,因哭道:“姑娘好狠心!哄了我這兩日,如今怎麼連一句話也沒有?”周瑞家的等說道:“你還要姑娘留你不成?便留下,你也難見園裏的人了。依我們的好話,快快收了這樣子,倒是人不知鬼不覺的去罷,大家體面些。”迎春含淚道:“我知道你幹了什麼大不是,我還十分說情留下,豈不連我也完了。你瞧入畫也是幾年的人,怎麼說去就去了。自然不止你兩個,想這園裏凡大的都要去呢。依我說,將來終有一散,不如你各人去罷。”周瑞家的道:“所以到底是姑娘明白。明兒還有打發的人呢,你放心罷。”司棋無法,只得含淚與迎春磕頭,和衆姊妹告別,又向迎春耳根說:“好歹打聽我要受罪,替我說個情兒,就是主僕一場!”迎春亦含淚答應:“放心。”
\end{parag}


\begin{parag}
    於是周瑞家的人等帶了司棋出了院門,又命兩個婆子將司棋所有的東西都與他拿著。走了沒幾步,後頭只見繡桔趕來,一面也擦著淚,一面遞與司棋一個絹包說:“這是姑娘給你的。主僕一場,如今一旦分離,這個與你作個想念罷。”司棋接了,不覺更哭起來了,又和繡桔哭了一回。周瑞家的不耐煩,只管催促,二人只得散了。司棋因又哭告道:“嬸子大娘們,好歹略徇個情兒,如今且歇一歇,讓我到相好的姊妹跟前辭一辭,也是我們這幾年好了一場。”周瑞家的等人皆各有事務,作這些事便是不得已了,況且又深恨他們素日大樣,如今那裏有工夫聽他的話,因冷笑道:“我勸你走罷,別拉拉扯扯的了。我們還有正經事呢。誰是你一個衣包裏爬出來的,辭他們作什麼,他們看你的笑聲還看不了呢。你不過是挨一會是一會罷了,難道就算了不成!依我說快走罷。”一面說,一面總不住腳,直帶著往後角門出去了。司棋無奈,又不敢再說,只得跟了出來。
\end{parag}


\begin{parag}
    可巧正值寶玉從外而入,一見帶了司棋出去,又見後面抱著些東西,料著此去再不能來了。因聞得上夜之事,又兼晴雯之病亦因那日加重,細問晴雯,又不說是爲何。上日又見入畫已去,今又見司棋亦走,不覺如喪魂魄一般,因忙攔住問道:"那裏去?"周瑞家的等皆知寶玉素日行爲,又恐勞叨誤事,因笑道:“不干你事,快唸書去罷。”寶玉笑道:“好姐姐們,且站一站,我有道理。”周瑞家的便道:“太太不許少挨一刻,又有什麼道理。我們只知遵太太的話,管不得許多。” 司棋見了寶玉,因拉住哭道:"他們做不得主,你好歹求求太太去。"寶玉不禁也傷心,含淚說道:"我不知你作了什麼大事,晴雯也病了,如今你又去。都要去了,這卻怎麼的好。"\begin{note}庚雙夾:寶玉之語全作囫圇意,最是極無味之語偏是極濃極有情之語也。只合如此寫方是寶玉,稍有真切則不是寶玉了。\end{note}周瑞家的發躁向司棋道:“你如今不是副小姐了,若不聽話,我就打得你。別想著往日姑娘護著,任你們作耗。越說著,還不好走。如今和小爺們拉拉扯扯,成個什麼體統!”那幾個媳婦不由分說,拉著司棋便出去了。
\end{parag}


\begin{parag}
    寶玉又恐他們去告舌,恨的只瞪著他們,看已去遠,方指著恨道:“奇怪,奇怪,怎麼這些人只一嫁了漢子,染了男人的氣味,就這樣混帳起來,比男人更可殺了!”\begin{note}庚眉:“染了男人的氣味”實有此情理,非躬親閱歷者亦不知此語之妙。\end{note}守園門的婆子聽了,也不禁好笑起來,因問道:“這樣說,凡女兒個個是好的了,女人個個是壞的了?”寶玉點頭道:“不錯,不錯!”婆子們笑道:“還有一句話我們糊塗不解,倒要請問請問。”方欲說時,只見幾個老婆子走來,忙說道:“你們小心,傳齊了伺候著。此刻太太親自來園裏,在那裏查人呢。只怕還查到這裏來呢。又吩咐快叫怡紅院的晴雯姑娘的哥嫂來,在這裏等著領出他妹妹去。”因笑道:“阿彌陀佛!今日天睜了眼,把這一個禍害妖精退送了,大家清淨些。”寶玉一聞得王夫人進來清查,便料定晴雯也保不住了,早飛也似的趕了去,所以這後來趁願之語竟未得聽見。
\end{parag}


\begin{parag}
    寶玉及到了怡紅院,只見一羣人在那裏,王夫人在屋裏坐著,一臉怒色,見寶玉也不理。晴雯四五日水米不曾沾牙,懨懨弱息,如今現從炕上拉了下來,蓬頭垢面,兩個女人才架起來去了。王夫人吩咐,只許把他貼身衣服撂出去,餘者好衣服留下給好丫頭們穿。又命把這裏所有的丫頭們都叫來一一過目。原來王夫人自那日著惱之後,王善保家的去趁勢告倒了晴雯,本處有人和園中不睦的,也就隨機趁便下了些話。王夫人皆記在心中。因節間有事,故忍了兩日,今日特來親自閱人。一則爲晴雯猶可,二則因竟有人指寶玉爲由,說他大了,已解人事,都由屋裏的丫頭們不長進教習壞了。因這事更比晴雯一人較甚,\begin{note}庚雙夾:暗伏一段。更覺煙迷霧罩之中更有無限溪山矣。\end{note}乃從襲人起以至於極小作粗活的小丫頭們,個個親自看了一遍。因問:“誰是和寶玉一日的生日?”本人不敢答應,老嬤嬤指道: “這一個蕙香,又叫作四兒的,是同寶玉一日生日的。”王夫人細看了一看,雖比不上晴雯一半,卻有幾分水秀。視其行止,聰明皆露在外面,且也打扮的不同。王夫人冷笑道:“這也是個不怕臊的。他背地裏說的,同日生日就是夫妻。這可是你說的?打諒我隔的遠,都不知道呢。可知道我身子雖不大來,我的心耳神意時時都在這裏。難道我通共一個寶玉,就白放心憑你們勾引壞了不成!”這個四兒見王夫人說著他素日和寶玉的私語,不禁紅了臉,低頭垂淚。王夫人即命也快把他家的人叫來,領出去配人。又問,“誰是耶律雄奴?”老嬤嬤們便將芳官指出。王夫人道:“唱戲的女孩子,自然是狐狸精了!上次放你們,你們又懶待出去,可就該安分守己纔是。你就成精鼓搗起來,調唆著寶玉無所不爲。”芳官笑辯道:“並不敢調唆什麼。”王夫人笑道:“你還強嘴。我且問你,前年我們往皇陵上去,是誰調唆寶玉要柳家的丫頭五兒了?幸而那丫頭短命死了,不然進來了,你們又連夥聚黨遭害這園子呢。你連你乾孃都欺倒了,豈止別人!”因喝命:“喚他乾孃來領去,就賞他外頭自尋個女婿去吧。把他的東西一概給他。”又吩咐上年凡有姑娘們分的唱戲的女孩子們,一概不許留在園裏,都令其各人乾孃帶出,自行聘嫁。一語傳出,這些乾孃皆感恩趁願不盡,都約齊與王夫人磕頭領去。王夫人又滿屋裏搜檢寶玉之物。凡略有眼生之物,一併命收的收,卷的卷,著人拿到自己房內去了。因說: “這才幹淨,省得旁人口舌。”因又吩咐襲人麝月等人:“你們小心!往後再有一點份外之事,我一概不饒。因叫人查看了,今年不宜遷挪,暫且捱過今年,明年一併給我仍舊搬出去心淨。”\begin{note}庚雙夾:一段神奇鬼訝之文不知從何想來,王夫人從來未理家務,豈不一木偶哉?且前文隱隱約約已有無限口舌,謾讕之譖原非一日矣。若無此一番更變,不獨終無散場之局,且亦大不近乎情理。況此亦是餘舊日目睹親聞,作者身歷之現成文字,非捏造而成者,故迥不與小說之離合悲窠臼相對。想遭冷落之大族子弟見此雖事有各殊,然其情理似亦有點契於心者焉。此一段不獨批此,直從抄檢大觀園及賈母對月興盡生悲皆可附者也。\end{note}說畢,茶也不喫,遂帶領衆人又往別處去閱人。暫且說不到後文。
\end{parag}


\begin{parag}
    如今且說寶玉只當王夫人不過來搜檢搜檢,無甚大事,誰知竟這樣雷嗔電怒的來了。所責之事皆系平日之語,一字不爽,料必不能挽回的。雖心下恨不能一死,但王夫人盛怒之際,自不敢多言一句,多動一步,一直跟送王夫人到沁芳亭。王夫人命:“回去好生念念那書,仔細明兒問你。才已發下恨了。”寶玉聽如此說,方回來,一路打算:“誰這樣犯舌?況這裏事也無人知道,如何就都說著了。”一面想,一面進來,只見襲人在那裏垂淚。且去了第一等的人,豈不傷心,便倒在牀上也哭起來。襲人知他心內別的還猶可,獨有晴雯是第一件大事,乃推他勸道:“哭也不中用了。你起來我告訴你,晴雯已經好了,他這一家去,倒心淨養幾天。你果然捨不得他,等太太氣消了,你再求老太太,慢慢的叫進來也不難。不過太太偶然信了人的誹言,一時氣頭上如此罷了。”寶玉哭道:“我究竟不知晴雯犯了何等滔天大罪!”\begin{note}庚雙夾:餘亦不知,蓋此等冤實非晴雯一人也。\end{note}襲人道:“太太只嫌他生的太好了,未免輕佻些。在太太是深知這樣美人似的人必不安靜,所以恨嫌他,像我們這粗粗笨笨的倒好。” 寶玉道:“這也罷了。咱們私自頑話怎麼也知道了?又沒外人走風的,這可奇怪。”襲人道:“你有甚忌諱的,一時高興了,你就不管有人無人了。我也曾使過眼色,也曾遞過暗號,倒被那別人已知道了,你反不覺。”寶玉道:“怎麼人人的不是太太都知道,單不挑出你和麝月秋紋來?”襲人聽了這話,心內一動,低頭半日,無可回答,因便笑道:“正是呢。若論我們也有頑笑不留心的孟浪去處,怎麼太太竟忘了?想是還有別的事,等完了再發放我們,也未可知。”寶玉笑道:“你是頭一個出了名的至善至賢之人,他兩個又是你陶冶教育的,焉得還有孟浪該罰之處!只是芳官尚小過於伶俐些,未免倚強壓倒了人,惹人厭。四兒是我誤了他,還是那年我和你拌嘴的那日起,叫上來作些細活,未免奪佔了地位,故有今日。只是晴雯也是和你一樣,從小兒在老太太屋裏過來的,雖然他生得比人強,也沒甚妨礙去處。就是他的性情爽利,口角鋒芒些,究竟也不曾得罪你們。想是他過於生得好了,反被這好所誤。”說畢,復又哭起來。
\end{parag}


\begin{parag}
    襲人細揣此話,好似寶玉有疑他之意,竟不好再勸,因嘆道:“天知道罷了。此時也查不出人來了,白哭一會子也無益。倒是養著精神,等老太太喜歡時,回明白了再要他是正理。”寶玉冷笑道:“你不必虛寬我的心。等到太太平服了再瞧勢頭去要時,知他的病等得等不得。他自幼上來嬌生慣養,何嘗受過一日委屈。連我知道他的性格,還時常衝撞了他。他這一下去,就如同一盆才抽出嫩箭來的蘭花送到豬窩裏去一般。況又是一身重病,裏頭一肚子的悶氣。他又沒有親爺熱娘,只有一個醉泥鰍姑舅哥哥。他這一去,一時也不慣的,那裏還等得幾日。知道還能見他一面兩面不能了!”說著又越發傷心起來。襲人笑道:“可是你‘只許州官放火,不許百姓點燈’。我們偶然說一句略妨礙些的話,就說是不利之談,你如今好好的咒他,是該的了!他便比別人嬌些,也不至這樣起來。”寶玉道:“不是我妄口咒他,今年春天已有兆頭的。”襲人忙問何兆。寶玉道:“這階下好好的一株海棠花,竟無故死了半邊,我就知有異事,果然應在他身上。”襲人聽了,又笑起來,因說道:“我待不說,又撐不住,你太也婆婆媽媽的了。這樣的話,豈是你讀書的男人說的。草木怎又關係起人來?若不婆婆媽媽的,真也成了個呆子了。”寶玉嘆道:“你們那裏知道,不但草木,凡天下之物,皆是有情有理的,也和人一樣,得了知己,便極有靈驗的。若用大題目比,就有孔子廟前之檜,墳前之蓍,諸葛祠前之柏,嶽武穆墳前之松。這都是堂堂正大隨人之正氣,千古不磨之物。世亂則萎,世治則榮,幾千百年了,枯而復生者幾次。這豈不是兆應?小題目比,就有楊太真沉香亭之木芍藥,端正樓之相思樹,王昭君冢上之草,豈不也有靈驗。所以這海棠亦應其人慾亡,故先就死了半邊。”襲人聽了這篇癡話,又可笑,又可嘆,因笑道:“真真的這話越發說上我的氣來了。那晴雯是個什麼東西,就費這樣心思,比出這些正經人來!還有一說,他縱好,也滅不過我的次序去。便是這海棠,也該先來比我,也還輪不到他。想是我要死了。”寶玉聽說,忙握他的嘴,勸道:“這是何苦!一個未清,你又這樣起來。罷了,再別提這事,別弄的去了三個,又饒上一個。”襲人聽說,心下暗喜道:“若不如此,你也不能了局。”寶玉乃道:“從此休提起,全當他們三個死了,不過如此。況且死了的也曾有過,也沒有見我怎麼樣,此一理也。\begin{note}庚雙夾:寶玉至終一省全作如是想,所以始於情終於悟者。既能終於悟而止,則情不得濫漫而涉於淫佚之事矣。一人之事一人了法,皆非 “棄竹而復憫筍”之意。\end{note}如今且說現在的,倒是把他的東西,作瞞上不瞞下,悄悄的打發人送出去與了他。再或有咱們常時積攢下的錢,拿幾吊出去給他養病,也是你姊妹好了一場。”襲人聽了,笑道:“你太把我們看的又小器又沒人心了。這話還等你說,我才已將他素日所有的衣裳以至各什各物總打點下了,都放在那裏。如今白日裏人多眼雜,又恐生事,且等到晚上,悄悄的叫宋媽給他拿出去。我還有攢下的幾吊錢也給他罷。”寶玉聽了,感謝不盡。襲人笑道:“我原是久已出了名的賢人,連這一點子好名兒還不會買來不成!”寶玉聽他方纔的話,忙陪笑撫慰一時。晚間果密遣宋媽送去。
\end{parag}


\begin{parag}
    寶玉將一切人穩住,便獨自得便出了后角門,央一個老婆子帶他到晴雯家去瞧瞧。先是這婆子百般不肯,只說怕人知道,“回了太太,我還喫飯不喫飯!”無奈寶玉死活央告,又許他些錢,那婆子方帶了他來。這晴雯當日系賴大家用銀子買的,那時晴雯才得十歲,尚未留頭。因常跟賴嬤嬤進來,賈母見他生得伶俐標緻,十分喜愛。故此賴嬤嬤就孝敬了賈母使喚,後來所以到了寶玉房裏。這晴雯進來時,也不記得家鄉父母。只知有個姑舅哥哥,專能庖宰,也淪落在外,故又求了賴家的收買進來喫工食。賴家的見晴雯雖到賈母跟前,千伶百俐,嘴尖性大,卻倒還不忘舊,\begin{note}庚雙夾:只此一句便是晴雯正傳。可知晴雯爲聰明風流所害也。一篇爲晴雯寫傳,是哭晴雯也。非哭晴雯,乃哭風流也。\end{note}故又將他姑舅哥哥收買進來,把家裏一個女孩子配了他。成了房後,誰知他姑舅哥哥一朝身安泰,就忘卻當年流落時,任意喫死酒,家小也不顧。偏又娶了個多情美色之妻,見他不顧身命,不知風月,一味死喫酒,便不免有蒹葭倚玉之嘆,紅顏寂寞之悲。又見他器量寬宏,\begin{note}庚雙夾:趣極!“器量寬宏”如此用,真掃地矣。\end{note}並無嫉衾妒枕之意,這媳婦遂恣情縱欲,滿宅內便延攬英雄,收納材俊,上上下下竟有一半是他考試過的。若問他夫妻姓甚名誰,便是上回賈璉所接見的多渾蟲燈姑娘兒的便是了。\begin{note}庚雙夾:奇奇怪怪,左盤右旋,千絲萬縷,皆自一體也。\end{note}目今晴雯只有這一門親戚,所以出來就在他家。
\end{parag}


\begin{parag}
    此時多渾蟲外頭去了,那燈姑娘吃了飯去串門子,只剩下晴雯一人,在外間房內爬著。\begin{note}庚雙夾:總哭晴雯。\end{note}寶玉命那婆子在院門瞭哨,他獨自掀起草簾\begin{note}庚雙夾:草簾。\end{note}進來,一眼就看見晴雯睡在蘆蓆土炕上,\begin{note}庚雙夾:蘆蓆土炕。\end{note}幸而衾褥還是舊日鋪的。心內不知自己怎麼纔好,因上來含淚伸手輕輕拉他,悄喚兩聲。當下晴雯又因著了風,又受了他哥嫂的歹話,病上加病,嗽了一日,才朦朧睡了。忽聞有人喚他,強展星眸,一見是寶玉,又驚又喜,又悲又痛,忙一把死攥住他的手。哽咽了半日,方說出半句話來:“我只當不得見你了。”接著便嗽個不住。寶玉也只有哽咽之分。晴雯道:“阿彌陀佛,你來的好,且把那茶倒半碗我喝。渴了這半日,叫半個人也叫不著。”寶玉聽說,忙拭淚問:“茶在那裏?”晴雯道:“那爐臺上就是。”寶玉看時,雖有個黑沙吊子,卻不象個茶壺。只得桌上去拿了一個碗,也甚大甚粗,不象個茶碗,未到手內,先就聞得油羶之氣。\begin{note}庚雙夾:不獨爲晴雯一哭,且爲寶玉一哭亦可。\end{note}寶玉只得拿了來,先拿些水洗了兩次,復又用水汕過,方提起沙壺斟了半碗。看時,絳紅的,也太不成茶。晴雯扶枕道:“快給我喝一口罷!這就是茶了。那裏比得咱們的茶!”寶玉聽說,先自己嚐了一嘗,並無清香,且無茶味,只一味苦澀,略有茶意而已。嘗畢,方遞與晴雯。只見晴雯如得了甘露一般,一氣都灌下去了。寶玉心下暗道:“往常那樣好茶,他尚有不如意之處;今日這樣。看來,可知古人說的‘飽飫烹宰,飢饜糟糠’,又道是‘飯飽弄粥’,可見都不錯了。”\begin{note}庚雙夾:妙!通篇寶玉最惡書者,每因女子之所曆始信其可,此爲觸類旁通之妙訣矣。\end{note}一面想,一面流淚問道:“你有什麼說的,趁著沒人告訴我。”晴雯嗚咽道:“有什麼可說的!不過挨一刻是一刻,挨一日是一日。我已知橫豎不過三五日的光景,就好回去了。只是一件,我死也不甘心的:我雖生的比別人略好些,並沒有私情密意勾引你怎樣,如何一口死咬定了我是個狐狸精!我太不服。今日既已擔了虛名,而且臨死,不是我說一句後悔的話,早知如此,我當日也另有個道理。不料癡心傻意,只說大家橫豎是在一處。不想平空裏生出這一節話來,有冤無處訴。”說畢又哭。寶玉拉著他的手,只覺瘦如枯柴,腕上猶戴著四個銀鐲,因泣道:“且卸下這個來,等好了再戴上罷。”因與他卸下來,塞在枕下。又說:“可惜這兩個指甲,好容易長了二寸長,這一病好了,又損好些。”晴雯拭淚,就伸手取了剪刀,將左手上兩根蔥管一般的指甲齊根鉸下;又伸手向被內將貼身穿著的一件舊紅綾襖脫下,並指甲都與寶玉道:“這個你收了,以後就如見我一般。快把你的襖兒脫下來我穿。我將來在棺材內獨自躺著,也就象還在怡紅院的一樣了。論理不該如此,只是擔了虛名,我可也是無可如何了。”寶玉聽說,忙寬衣換上,藏了指甲。晴雯又哭道:“回去他們看見了要問,不必撒謊,就說是我的。既擔了虛名,越性如此,也不過這樣了。”
\end{parag}


\begin{parag}
    一語未了,只見他嫂子笑嘻嘻掀簾進來,道:“好呀,你兩個的話,我已都聽見了。”又向寶玉道:“你一個作主子的,跑到下人房裏作什麼?看我年輕又俊,敢是來調戲我麼?”寶玉聽說,嚇的忙陪笑央道:“好姐姐,快別大聲。他伏侍我一場,我私自來瞧瞧他。”燈姑娘便一手拉了寶玉進裏間來,笑道:“你不叫嚷也容易,只是依我一件事。”說著,便坐在炕沿上,卻緊緊的將寶玉摟入懷中。寶玉如何見過這個,心內早突突的跳起來了,急的滿面紅漲,又羞又怕,只說:“好姐姐,別鬧。”\begin{note}庚雙夾:如聞如見,“別鬧”二字活跳。\end{note}燈姑娘乜斜醉眼,笑道:“呸!成日家聽見你風月場中慣作工夫的,怎麼今日就反訕起來。”寶玉紅了臉,笑道:“姐姐放手,有話咱們好說。外頭有老媽媽,聽見什麼意思。”燈姑娘笑道:“我早進來了,卻叫婆子去園門等著呢。我等什麼似的,今兒等著了你。雖然聞名,不如見面,空長了一個好模樣兒,竟是沒藥信的炮仗,只好裝幌子罷了,倒比我還發訕怕羞。可知人的嘴一概聽不得的。就比如方纔我們姑娘下來,我也料定你們素日偷雞盜狗的。我進來一會在窗下細聽,屋內只你二人,若有偷雞盜狗的事,豈有不談及於此,誰知你兩個竟還是各不相擾。可知天下委屈事也不少。如今我反後悔錯怪了你們。既然如此,你但放心。以後你只管來,我也不羅唣你。”寶玉聽說,才放下心來,方起身整衣央道:“好姐姐,你千萬照看他兩天。我如今去了。”說畢出來,又告訴晴雯。二人自是依依不捨,也少不得一別。晴雯知寶玉難行,遂用被矇頭,總不理他,寶玉方出來。意欲到芳官四兒處去,無奈天黑,出來了半日,恐裏面人找他不見,又恐生事,遂且進園來了,明日再作計較。因乃至后角門,小廝正抱鋪蓋,裏邊嬤嬤們正查人,若再遲一步也就關了。
\end{parag}


\begin{parag}
    寶玉進入園中,且喜無人知道。到了自己房內,告訴襲人只說在薛姨媽家去的,也就罷了。一時鋪牀,襲人不得不問今日怎麼睡。寶玉道:“不管怎麼睡罷了。”原來這一二年間襲人因王夫人看重了他了,越發自要尊重。凡揹人之處,或夜晚之間,總不與寶玉狎暱,較先幼時反倒疏遠了。況雖無大事辦理,然一應針線並寶玉及諸小丫頭們凡出入銀錢衣履什物等事,也甚煩瑣;且有吐血舊症雖愈,然每因勞碌風寒所感,即嗽中帶血,故邇來夜間總不與寶玉同房。寶玉夜間常醒,又極膽小,每醒必喚人。因晴雯睡臥警醒,且舉動輕便,故夜晚一應茶水起坐呼喚之任皆悉委他一人,所以寶玉外牀只是他睡。今他去了,襲人只得要問,因思此任比日間緊要之意。寶玉既答不管怎樣,襲人只得還依舊年之例,遂仍將自己舖蓋搬來設於牀外。
\end{parag}


\begin{parag}
    寶玉發了一晚上呆。\begin{note}庚雙夾:一句是矣。\end{note}及催他睡下,襲人等也都睡後,聽著寶玉在枕上長吁短嘆,覆去翻來,直至三更以後。方漸漸的安頓了,略有齁聲。襲人方放心,也就朦朧睡著。沒半盞茶時,只聽寶玉叫“晴雯”。襲人忙睜開眼連聲答應,問作什麼。寶玉因要喫茶。襲人忙下去向盆內蘸過手,從暖壺內倒了半盞茶來喫過。寶玉乃笑道:\begin{note}庚雙夾:“笑”字好極,有文章,蓋恐冷落襲人也。\end{note}“我近來叫慣了他,卻忘了是你。”襲人笑道:“他一乍來時你也曾睡夢中直叫我,半年後才改了。我知道這晴雯人雖去了,這兩個字只怕是不能去的。”說著,大家又臥下。寶玉又翻轉了一個更次,至五更方睡去時,只見晴雯從外頭走來,仍是往日形景,進來笑向寶玉道:“你們好生過罷,我從此就別過了。”說畢,翻身便走。寶玉忙叫時,又將襲人叫醒。襲人還只當他慣了口亂叫,卻見寶玉哭了,說道:“晴雯死了。”襲人笑道:“這是那裏的話!你就知道胡鬧,被人聽著什麼意思。”寶玉那裏肯聽,恨不得一時亮了就遣人去問信。
\end{parag}


\begin{parag}
    及至天亮時,就有王夫人房裏小丫頭立等叫開前角門傳王夫人的話:“即時叫起寶玉,快洗臉,換了衣裳快來,因今兒有人請老爺尋秋賞桂花,老爺因喜歡他前兒作得詩好,故此要帶他們去。這都是太太的話,一句別錯了。你們快飛跑告訴他去,立刻叫他快來,老爺在上屋裏還等他吃麪茶呢。環哥兒已來了。快跑,快跑。再著一個人去叫蘭哥兒,也要這等說。”裏面的婆子聽一句,應一句,一面扣扭子,一面開門。一面早有兩三個人一行扣衣,一行分頭去了。襲人聽得叩院門,便知有事,忙一面命人問時,自己已起來了。聽得這話,促人來舀了麪湯,催寶玉起來盥漱。他自去取衣。因思跟賈政出門,便不肯拿出十分出色的新鮮衣履來,只拿那二等成色的來。寶玉此時亦無法,只得忙忙的前來。果然賈政在那裏喫茶,十分喜悅。寶玉忙行了省晨之禮。賈環賈蘭二人也都見過寶玉。賈政命坐喫茶,向環蘭二人道:“寶玉讀書不如你兩個,論題聯和詩這種聰明,你們皆不及他。今日此去,未免強你們做詩,寶玉須聽便助他們兩個。”王夫人等自來不曾聽見這等考語,真是意外之喜。
\end{parag}


\begin{parag}
    一時候他父子二人等去了,方欲過賈母這邊來時,就有芳官等三個的乾孃走來,回說:“芳官自前日蒙太太的恩典賞了出去,他就瘋了似的,茶也不喫,飯也不用,勾引上藕官蕊官,三個人尋死覓活,只要剪了頭髮做尼姑去。我只當是小孩子家一時出去不慣也是有的,不過隔兩日就好了。誰知越鬧越兇,打罵著也不怕。實在沒法,所以來求太太,或者就依他們做尼姑去,或教導他們一頓,賞給別人作女兒去罷,我們也沒這福。”王夫人聽了道:“胡說!那裏由得他們起來,佛門也是輕易人進去的!每人打一頓給他們,看還鬧不鬧了!”當下因八月十五日各廟內上供去,皆有各廟內的尼姑來送供尖之例,王夫人曾於十五日就留下水月庵的智通與地藏庵的圓心住兩日,至今日未回,聽得此信,巴不得又拐兩個女孩子去作活使喚,因都向王夫人道:“咱們府上到底是善人家。因太太好善,所以感應得這些小姑娘們皆如此。雖說佛門輕易難入,也要知道佛法平等。我佛立願,原是一切衆生無論雞犬皆要度他,無奈迷人不醒。若果有善根能醒悟,即可以超脫輪迴。所以經上現有虎狼蛇蟲得道者就不少。如今這兩三個姑娘既然無父無母,家鄉又遠,他們既經了這富貴,又想從小兒命苦入了這風流行次,將來知道終身怎麼樣,所以苦海回頭,出家修修來世,也是他們的高意。太太倒不要限了善念。”王夫人原是個好善的,先聽彼等之語不肯聽其自由者,因思芳官等不過皆系小兒女,一時不遂心,故有此意,但恐將來熬不得清淨,反致獲罪。今聽這兩個柺子的話大近情理;且近日家中多故,又有邢夫人遣人來知會,明日接迎春家去住兩日,以備人家相看;且又有官媒婆來求說探春等事,心緒正煩,那裏著意在這些小事上。既聽此言,便笑答道:“你兩個既這等說,你們就帶了作徒弟去如何?”兩個姑子聽了,念一聲佛道:“善哉!善哉!若如此,可是你老人家陰德不小。”說畢,便稽首拜謝。王夫人道:“既這樣,你們問他們去。若果真心,即上來當著我拜了師父去罷。”這三個女人聽了出去,果然將他三人帶來。王夫人問之再三,他三人已是立定主意,遂與兩個姑子叩了頭,又拜辭了王夫人。王夫人見他們意皆決斷,知不可強了,反倒傷心可憐,忙命人取了些東西來齎賞了他們,又送了兩個姑子些禮物。從此芳官跟了水月庵的智通,蕊官藕官二人跟了地藏庵的圓心,各自出家去了。再聽下回分解。
\end{parag}


\begin{parag}
    \begin{note}蒙回末總:看晴雯與寶玉永絕一段,的是消魂文字;看寶玉幾番□論,真是至誠種子;看寶玉給晴雯斟茶,又真是阿公子。前文敘襲人奔喪時,寶玉夜來喫茶,先呼襲人,此又夜來喫茶,先呼晴雯。字字龍跳天門,虎臥鳳闕,語語嬰兒戀母,雛鳥尋巢。\end{note}
\end{parag}
