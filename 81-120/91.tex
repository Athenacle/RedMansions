\chap{九十一}{縱淫心寶蟾工設計 佈疑陣寶玉妄談禪}



\begin{parag}
    話說薛蝌正在狐疑,忽聽窗外一笑,唬了一跳,心中想道:“不是寶蟾,定是金桂。只不理他們,看他們有什麼法兒。”聽了半日,卻又寂然無聲。自己也不敢喫那酒果。掩上房門,剛要脫衣時,只聽見窗紙上微微一響。薛蝌此時被寶蟾鬼混了一陣,心中七上八下,竟不知是如何是可。聽見窗紙微響,細看時,又無動靜,自己反倒疑心起來,掩了懷,坐在燈前,呆呆的細想,又把那果子拿了一塊,翻來覆去的細看。猛回頭,看見窗上紙溼了一塊,走過來覷着眼看時,冷不防外面往裏一吹,把薛蝌唬了一大跳。聽得吱吱的笑聲,薛蝌連忙把燈吹滅了,屏息而臥。只聽外面一個人說道:“二爺爲什麼不喝酒喫果子,就睡了?”這句話仍是寶蟾的語音。薛蝌只不作聲裝睡。又隔有兩句話時,又聽得外面似有恨聲道:“天下那裏有這樣沒造化的人。”薛蝌聽了是寶蟾又似是金桂的語音,這才知道他們原來是這一番意思,翻來覆去,直到五更後才睡着了。剛到天明,早有人來扣門。薛蝌忙問是誰,外面也不答應。薛蝌只得起來,開了門看時,卻是寶蟾,攏着頭髮,掩着懷,穿一件片錦邊琵琶襟小緊身,上面系一條松花綠半新的汗巾,下面並未穿裙,正露着石榴紅灑花夾褲,一雙新繡紅鞋。原來寶蟾尚未梳洗,恐怕人見,趕早來取傢伙。薛蝌見他這樣打扮,便走進來,心中又是一動,只得陪笑問道:“怎麼這樣早就起來了?”寶蟾把臉紅着,並不答言,只管把果子折在一個碟子裏,端着就走。薛蝌見他這般,知是昨晚的原故,心裏想道:“這也罷了。倒是他們惱了,索性死了心,也省得來纏。”於是把心放下,喚人舀水洗臉。自己打算在家裏靜坐兩天,一則養養心神,二則出去怕人找他。原來和薛蟠好的那些人因見薛家無人,只有薛蝌在那裏辦事,年紀又輕,便生許多覬覦之心。也有想插在裏頭做跑腿的,也有能做狀子的,認得一二個書役的,要給他上下打點的,甚至有叫他在內趁錢的,也有造作謠言恐嚇的:種種不一。薛蝌見了這些人,遠遠躲避,又不敢面辭,恐怕激出意外之變,只好藏在家中,聽候傳詳。不提。
\end{parag}


\begin{parag}
    且說金桂昨夜打發寶蟾送了些酒果去探探薛蝌的消息,寶蟾回來將薛蝌的光景一一的說了。金桂見事有些不大投機,便怕白鬧一場,反被寶蟾瞧不起,欲把兩三句話遮飾改過口來,又可惜了這個人,心裏倒沒了主意,怔怔的坐着。那知寶蟾亦知薛蟠難以回家,正欲尋個頭路,因怕金桂拿他,所以不敢透漏。今見金桂所爲先已開了端了,他便樂得借風使船,先弄薛蝌到手,不怕金桂不依,所以用言挑撥。見薛蝌似非無情,又不甚兜攬,一時也不敢造次,後來見薛蝌吹燈自睡,大覺掃興,回來告訴金桂,看金桂有甚方法,再作道理。及見金桂怔怔的,似乎無技可施,他也只得陪金桂收拾睡了。夜裏那裏睡得着,翻來覆去,想出一個法子來:不如明兒一早起來,先去取了傢伙,卻自己換上一兩件動人的衣服,也不梳洗,越顯出一番嬌媚來。只看薛蝌的神情,自己反倒裝出一番惱意,索性不理他。那薛蝌若有悔心,自然移船泊岸,不愁不先到手。及至見了薛蝌,仍是昨晚這般光景,並無邪僻之意,自己只得以假爲真,端了碟子回來,卻故意留下酒壺,以爲再來搭轉之地。只見金桂問道:“你拿東西去有人碰見麼?”寶蟾道:“沒有。”“二爺也沒問你什麼?”寶蟾道:“也沒有。”金桂因一夜不曾睡着,也想不出一個法子來,只得回思道:“若作此事,別人可瞞,寶蟾如何能瞞?不如我分惠於他,他自然沒有不盡心的。我又不能自去,少不得要他作腳,倒不如和他商量一個穩便主意。”因帶笑說道:“你看二爺到底是個怎麼樣的人?”寶蟾道:“倒象個糊塗人。”金桂聽了笑道:“你如何說起爺們來了。”寶蟾也笑道:“他辜負奶奶的心,我就說得他。”金桂道:“他怎麼辜負我的心,你倒得說說。”寶蟾道:“奶奶給他好東西喫,他倒不喫,這不是辜負奶奶的心麼。”說着,卻把眼溜着金桂一笑。金桂道:“你別胡想。我給他送東西,爲大爺的事不辭勞苦,我所以敬他,又怕人說瞎話,所以問你。你這些話向我說,我不懂是什麼意思。”寶蟾笑道:“奶奶別多心,我是跟奶奶的,還有兩個心麼。但是事情要密些,倘或聲張起來,不是頑的。”金桂也覺得臉飛紅了,因說道:“你這個丫頭就不是個好貨!想來你心裏看上了,卻拿我作筏子,是不是呢?”寶蟾道:“只是奶奶那麼想罷咧,我倒是替奶奶難受。奶奶要真瞧二爺好,我倒有個主意。奶奶想,那個耗子不偷油呢,他也不過怕事情不密,大家鬧出亂子來不好看。依我想,奶奶且別性急,時常在他身上不周不備的去處張羅張羅。他是個小叔子,又沒娶媳婦兒,奶奶就多盡點心兒和他貼個好兒,別人也說不出什麼來。過幾天他感奶奶的情,他自然要謝候奶奶。那時奶奶再備點東西兒在咱們屋裏,我幫着奶奶灌醉了他,怕跑了他?他要不應,咱們索性鬧起來,就說他調戲奶奶。他害怕,他自然得順着咱們的手兒。他再不應,他也不是人,咱們也不至白丟了臉面。奶奶想怎麼樣?”金桂聽了這話,兩顴早已紅暈了,笑罵道:“小蹄子,你倒偷過多少漢子的似的,怪不得大爺在家時離不開你。”寶蟾把嘴一撇,笑說道:“罷喲,人家倒替奶奶拉縴,奶奶倒往我們說這個話咧。”從此金桂一心籠絡薛蝌,倒無心混鬧了。家中也少覺安靜。
\end{parag}


\begin{parag}
    當日寶蟾自去取了酒壺,仍是穩穩重重一臉的正氣。薛蝌偷眼看了,反倒後悔,疑心或者是自己錯想了他們,也未可知。果然如此,倒辜負了他這一番美意,保不住日後倒要和自己也鬧起來,豈非自惹的呢。過了兩天,甚覺安靜。薛蝌遇見寶蟾,寶蟾便低頭走了,連眼皮兒也不抬,遇見金桂,金桂卻一盆火兒的趕着。薛蝌見這般光景,反倒過意不去。這且不表。
\end{parag}


\begin{parag}
    且說寶釵母女覺得金桂幾天安靜,待人忽親熱起來,一家子都爲罕事。薛姨媽十分歡喜,想到必是薛蟠娶這媳婦時衝犯了什麼,才敗壞了這幾年。目今鬧出這樣事來,虧得家裏有錢,賈府出力,方纔有了指望。媳婦兒忽然安靜起來,或者是蟠兒轉過運氣來了,也未可知,於是自己心裏倒以爲希有之奇。這日飯後扶了同貴過來,到金桂房裏瞧瞧。走到院中,只聽一個男人和金桂說話。同貴知機,便說道:“大奶奶,老太太過來了。”說着已到門口。只見一個人影兒在房門後一躲,薛姨媽一嚇,倒退了出來。金桂道:“太太請裏頭坐。沒有外人,他就是我的過繼兄弟,本住在屯裏,不慣見人,因沒有見過太太。今兒纔來,還沒去請太太的安。”薛姨媽道:“既是舅爺,不妨見見。”金桂叫兄弟出來,見了薛姨媽,作了一個揖,問了好。薛姨媽也問了好,坐下敘起話來。薛姨媽道:“舅爺上京幾時了?”那夏三道:“前月我媽沒有人管家,把我過繼來的。前日才進京,今日來瞧姐姐。”薛姨媽看那人不尷尬,於是略坐坐兒,便起身道:“舅爺坐着罷。”回頭向金桂道:“舅爺頭上末下的來,留在咱們這裏吃了飯再去罷。”金桂答應着,薛姨媽自去了。金桂見婆婆去了,便向夏三道:“你坐着,今日可是過了明路的了,省得我們二爺查考你。我今日還叫你買些東西,只別叫衆人看見。”夏三道:“這個交給我就完了。你要什麼,只要有錢,我就買得來。”金桂道:“且別說嘴,你買上了當,我可不收。”說着,二人又笑了一回,然後金桂陪夏三吃了晚飯,又告訴他買的東西,又囑咐一回,夏三自去。從此夏三往來不絕。雖有個年老的門上人,知是舅爺,也不常回,從此生出無限風波,這是後話。不表。
\end{parag}


\begin{parag}
    一日薛蟠有信寄回,薛姨媽打開叫寶釵看時,上寫:
\end{parag}


\begin{qute2sp}
    男在縣裏也不受苦,母親放心。但昨日縣裏書辦說,府裏已經準詳,想是我們的情到了。豈知府裏詳上去,道里反駁下來。虧得縣裏主文相公好,即刻做了迴文頂上去了。那道里卻把知縣申飭。現在道里要親提,若一上去,又要喫苦。必是道里沒有託到。母親見字,快快託人求道爺去。還叫兄弟快來,不然就要解道。銀子短不得。火速,火速。
\end{qute2sp}


\begin{parag}
    薛姨媽聽了,又哭了一場,自不必說。薛蝌一面勸慰,一面說道:“事不宜遲。”薛姨媽沒法,只得叫薛蝌到縣照料,命人即便收拾行李,兌了銀子,家人李祥本在那裏照應的,薛蝌又同了一個當中夥計連夜起程。
\end{parag}


\begin{parag}
    那時手忙腳亂,雖有下人辦理,寶釵又恐他們思想不到,親來幫着,直鬧至四更才歇。到底富家女子嬌養慣的,心上又急,又苦勞了一會,晚上就發燒。到了明日,湯水都喫不下。鶯兒去回了薛姨媽。薛姨媽急來看時,只見寶釵滿面通紅,身如燔灼,話都不說。薛姨媽慌了手腳,便哭得死去活來。寶琴扶着勸薛姨媽。秋菱也淚如泉湧,只管叫着。寶釵不能說話,手也不能搖動,眼乾鼻塞。叫人請醫調治,漸漸甦醒回來。薛姨媽等大家略略放心。早驚動榮寧兩府的人,先是鳳姐打發人送十香返魂丹來,隨後王夫人又送至寶丹來。賈母邢王二夫人以及尤氏等都打發丫頭來問候,卻都不叫寶玉知道。一連治了七八天,終不見效,還是他自己想起冷香丸,吃了三丸,才得病好。後來寶玉也知道了,因病好了,沒有瞧去。
\end{parag}


\begin{parag}
    那時薛蝌又有信回來,薛姨媽看了,怕寶釵耽憂,也不叫他知道。自己來求王夫人,並述了一會子寶釵的病。薛姨媽去後,王夫人又求賈政。賈政道:“此事上頭可託,底下難託,必須打點纔好。”王夫人又提起寶釵的事來,因說道:“這孩子也苦了。既是我家的人了,也該早些娶了過來纔是,別叫他糟踏壞了身子。”賈政道:“我也是這麼想。但是他家亂忙,況且如今到了冬底,已經年近歲逼,不無各自要料理些家務。今冬且放了定,明春再過禮,過了老太太的生日,就定日子娶。你把這番話先告訴薛姨太太。”王夫人答應了。到了明日,王夫人將賈政的話向薛姨媽述了。薛姨媽想着也是。到了飯後,王夫人陪着來到賈母房中,大家讓了坐。賈母道:“姨太太纔過來?”薛姨媽道:“還是昨兒過來的。因爲晚了,沒得過來給老太太請安。”王夫人便把賈政昨夜所說的話向賈母述了一遍,賈母甚喜。說着,寶玉進來了。賈母便問道:“吃了飯了沒有?”寶玉道:“纔打學房裏回來,吃了要往學房裏去,先見見老太太。又聽見說姨媽來了,過來給姨媽請請安。”因問:“寶姐姐可大好了?”薛姨媽笑道:“好了。”原來方纔大家正說着,見寶玉進來,都煞住了。寶玉坐了坐,見薛姨媽情形不似從前親熱,”雖是此刻沒有心情,也不犯大家都不言語。”滿腹猜疑,自往學中去了。
\end{parag}


\begin{parag}
    晚間回來,都見過了,便往瀟湘館來。掀簾進去,紫鵑接着,見裏間屋內無人,寶玉道:“姑娘那裏去了?”紫鵑道:“上屋裏去了。知道姨太太過來,姑娘請安去了。二爺沒有到上屋裏去麼?”寶玉道:“我去了來的,沒有見你姑娘。”紫鵑道:“這也奇了。”寶玉問:“姑娘到底那裏去了?”紫鵑道:“不定。”寶玉往外便走。剛出屋門,只見黛玉帶着雪雁,冉冉而來。寶玉道:“妹妹回來了。”縮身退步進來。
\end{parag}


\begin{parag}
    黛玉進來,走入裏間屋內,便請寶玉里頭坐。紫鵑拿了一件外罩換上,然後坐下,問道:“你上去看見姨媽沒有?”寶玉道:“見過了。”黛玉道:“姨媽說起我沒有?”寶玉道:“不但沒有說起你,連見了我也不象先時親熱。今日我問起寶姐姐病來,他不過笑了一笑,並不答言。難道怪我這兩天沒有去瞧他麼。”黛玉笑了一笑道:“你去瞧過沒有?”寶玉道:“頭幾天不知道,這兩天知道了,也沒有去。”黛玉道:“可不是。”寶玉道:“老太太不叫我去,太太也不叫我去,老爺又不叫我去,我如何敢去。若是象從前這扇小門走得通的時候,要我一天瞧他十趟也不難。如今把門堵了,要打前頭過去,自然不便了。”黛玉道:“他那裏知道這個原故。”寶玉道:“寶姐姐爲人是最體諒我的。”黛玉道:“你不要自己打錯了主意。若論寶姐姐,更不體諒,又不是姨媽病,是寶姐姐病。向來在園中,做詩賞花飲酒,何等熱鬧,如今隔開了,你看見他家裏有事了,他病到那步田地,你象沒事人一般,他怎麼不惱呢。”寶玉道:“這樣難道寶姐姐便不和我好了不成?”黛玉道:“他和你好不好我卻不知,我也不過是照理而論。”寶玉聽了,瞪着眼呆了半晌。黛玉看見寶玉這樣光景,也不睬他,只是自己叫人添了香,又翻出書來細看了一會。只見寶玉把眉一皺,把腳一跺道:“我想這個人生他做什麼!天地間沒有了我,倒也乾淨!”黛玉道:“原是有了我,便有了人,有了人,便有無數的煩惱生出來,恐怖,顛倒,夢想,更有許多纏礙。——纔剛我說的都是頑話,你不過是看見姨媽沒精打彩,如何便疑到寶姐姐身上去?姨媽過來原爲他的官司事情心緒不寧,那裏還來應酬你?都是你自己心上胡思亂想,鑽入魔道里去了。”寶玉豁然開朗,笑道:“很是,很是。你的性靈比我竟強遠了,怨不得前年我生氣的時候,你和我說過幾句禪語,我實在對不上來。我雖丈六金身,還借你一莖所化。”黛玉乘此機會說道:“我便問你一句話,你如何回答?”寶玉盤着腿,合着手,閉着眼,噓着嘴道:“講來。”黛玉道:“寶姐姐和你好你怎麼樣?寶姐姐不和你好你怎麼樣?寶姐姐前兒和你好,如今不和你好你怎麼樣?今兒和你好,後來不和你好你怎麼樣?你和他好他偏不和你好你怎麼樣?你不和他好他偏要和你好你怎麼樣?”寶玉呆了半晌,忽然大笑道:“任憑弱水三千,我只取一瓢飲。”黛玉道:“瓢之漂水奈何?”寶玉道:“非瓢漂水,水自流,瓢自漂耳!”黛玉道:“水止珠沉,奈何?”寶玉道:“禪心已作沾泥絮,莫向春風舞鷓鴣。”黛玉道:“禪門第一戒是不打誑語的。”寶玉道:“有如三寶。”黛玉低頭不語。只聽見檐外老鴰呱呱的叫了幾聲,便飛向東南上去,寶玉道:“不知主何吉凶。”黛玉道:“人有吉凶事,不在鳥聲中。”忽見秋紋走來說道:“請二爺回去。老爺叫人到園裏來問過,說二爺打學裏回來了沒有。襲人姐姐只說已經來了。快去罷。”嚇得寶玉站起身來往外忙走,黛玉也不敢相留。未知何事,下回分解。
\end{parag}