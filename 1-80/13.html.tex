\chap{一十三}{秦可卿死封龙禁尉 王熙凤协理宁国府}

\begin{parag}
    \begin{note}蒙、戚:生死穷通何处真?英明难遏是精神。微密久藏偏自露,幻中梦里语惊人。\end{note}
\end{parag}


\begin{parag}
    \begin{note}甲:贾珍尚奢,岂有不请父命之理?因敬□□□要紧,不问家事,故得恣意放为。\end{note}
\end{parag}


\begin{parag}
    \begin{note}
        甲:若明指一州名,似若《西游》之套,故曰至中之地,不待言可知是光天化日仁风德雨之下矣。不云国名更妙,可知是尧街舜巷衣冠礼义之乡矣。直与第一回呼应相接。
    \end{note}
\end{parag}


\begin{parag}
    \begin{note}
        今秦可卿托□□□□□□□□□□□□□理宁府,亦□□□□□□□□□□□□□凡□□□□□□□□□□□□□□□□在封龙禁尉写乃褒中之贬,隐去天香楼一节,是不忍下笔也。
    \end{note}
    \begin{subnote}按:甲本此页被对角撕去,故缺字甚多。此页原有三则评语,第二则与本回庚本眉同。今补。\end{subnote}
\end{parag}


\begin{parag}
    \begin{note}
        庚:此回可卿梦阿凤,盖作者大有深意存焉。可惜生不逢时,奈何奈何!然必写出自可卿之意也,则又有他意寓焉。
    \end{note}
\end{parag}


\begin{parag}
    \begin{note}
        荣、宁世家未有不尊家训者。虽贾珍尚奢,岂明逆父哉?故写敬老不管,然后恣意,方见笔笔周到。
    \end{note}
\end{parag}


\begin{parag}
    \begin{note}靖:此回可卿梦阿凤,作者大有深意,惜已为末世,奈何奈何!贾珍虽奢淫,岂能逆父哉?特因敬老不管,然后恣意,足为世家之戒。“秦可卿淫丧天香楼”,作者用史笔也。老朽因有魂托凤姐贾家后事二件,岂是安富尊荣坐享人能想得到者?其事虽未行,其言其意,令人悲切感服,姑赦之,因命芹溪删去“遗簪”、“更衣”诸文,是以此回只十页,删去天香楼一节,少去四五页也。\end{note}
\end{parag}


\begin{parag}
    话说凤姐儿自贾琏送黛玉往扬州去后,心中实在无趣,每到晚间,不过和平儿说笑一回,就胡乱\begin{note}甲侧:“胡乱”二字奇。\end{note}睡了。
\end{parag}


\begin{parag}
    这日夜间,正和平儿灯下拥炉倦绣,早命浓薰绣被,二人睡下,屈指算行程该到何处,\begin{note}甲侧:所谓“计程今日到梁州”是也。\end{note}不知不觉已交三鼓。平儿已睡熟了。凤姐方觉星眼微蒙,恍惚只见秦氏从外走来,含笑说道:“婶婶好睡!我今日回去,你也不送我一程。因娘儿们素日相好,我舍不得婶子,故来别你一别。还有一件心愿未了,非告诉婶子,别人未必中用。”\begin{note}甲侧:一语贬尽贾家一族空顶冠束带者。\end{note}
\end{parag}


\begin{parag}
    凤姐听了,恍惚问道:“有何心事?你只管托我就是了。”秦氏道:“婶婶,你是个脂粉队里的英雄,\begin{note}甲侧:称得起。\end{note}连那些束带顶冠的男子也不能过你,你如何连两句俗语也不晓得?常言‘月满则亏,水满则溢’;又道是‘登高必跌重’。如今我们家赫赫扬扬,已将百载,一日倘或\begin{note}甲侧:“倘或”二字酷肖妇女口气。\end{note}乐极悲生,若应了那句‘树倒猢狲散’的俗语,\begin{note}甲眉:“树倒猢狲散”之语,今犹在耳,屈指三十五年矣。哀哉伤哉,宁不痛杀!\end{note}岂不虚称了一世诗书旧族了!”凤姐听了此话,心胸大快,十分敬畏,忙问道:“这话虑的极是,但有何法可以永保无虞?”\begin{note}甲侧:非阿凤不明,该古今名利场中患失之同意也。\end{note}秦氏冷笑道:“婶子好痴也。否极泰来,荣辱自古周而复始,岂人力能可常保的。但如今能于荣时筹划下将来衰时的世业,亦可谓常保永全了。即如今日诸事都妥,只有两件未妥,若把此事如此一行,则后日可保永全了。”
\end{parag}


\begin{parag}
    凤姐便问何事。秦氏道:“目今祖茔虽四时祭祀,只是无一定的钱粮;第二,家塾虽立,无一定的供给。依我想来,如今盛时固不缺祭祀供给,但将来败落之时,此二项有何出处?莫若依我定见,趁今日富贵,将祖茔附近多置田庄房舍地亩,以备祭祀供给之费皆出自此处,将家塾亦设于此。合同族中长幼,大家定了则例,日后按房掌管这一年的地亩、钱粮、祭祀、供给之事。如此周流,又无竞争,亦不有典卖诸弊。便是有了罪,凡物可入官,这祭祀产业连官也不入的。便败落下来,子孙回家读书务农,也有个退步,\begin{note}蒙双夹:幻情文字中忽入此等警句,提醒多少热心人。\end{note}祭祀又可永继。若目今以为荣华不绝,不思后日,终非长策。眼见不日又有一件非常喜事,真是烈火烹油、鲜花著锦之盛。要知道,也不过是瞬息的繁华,一时的欢乐,万不可忘了那‘盛筵必散’的俗语。\begin{note}蒙侧:“瞬息繁华,一时欢乐”二语,可共天下有志事业功名者同来一哭。但天生人非无所为,遇机会,成事业,留名于后世者,办必有奇传奇遇,方能成不世之功。此亦皆苍天暗中扶助,虽有波澜,而无甚害,反觉其铮铮有声。其不成也,亦由天命。其好人倾险之计,亦非天命不能行。其繁华欢乐,亦自天命。人于其间,知天命而存好生之心,尽已力以周旋其间,不计其功之成否,所谓心安而理尽,又何患乎?一时瞬息,随缘遇缘,乌乎不可!\end{note}此时若不早为后虑,临期只恐后悔无益了。”\begin{note}甲眉:语语见道,字字伤心,读此一段,几不知此身为何物矣。松斋。\end{note}凤姐忙问:“有何喜事?”秦氏道:“天机不可泄漏。\begin{note}甲侧:伏得妙!\end{note}只是我与婶子好了一场,临别赠你两句话,须要记著。”因念道:
\end{parag}


\begin{poem}
    \begin{pl}三春去后诸芳尽,各自须寻各自门。\end{pl}
    \begin{note}甲侧:此句令批书人哭死。甲眉:不必看完,见此二句,即欲堕泪。梅溪。\end{note}
\end{poem}


\begin{parag}
    凤姐还欲问时,只听二门上传事云牌连叩四下,将凤姐惊醒。人回:“东府蓉大奶奶没了。”凤姐闻听,吓了一身冷汗,出了一回神,只得忙忙的穿衣,往王夫人处来。
\end{parag}


\begin{parag}
    彼时合家皆知,无不纳罕,都有些疑心。\begin{note}甲眉:九个字写尽天香楼事,是不写之写。[靖本多署名“棠村”。]庚眉:可从此批。靖眉:可从此批。通回将可卿如何死故隐去,是余大发慈悲也。叹叹!壬午季春。 笏叟。\end{note}那长一辈的想他素日孝顺;平一辈的,想他平日和睦亲密,\begin{note}庚眉:松斋云:好笔力。此方是文字佳处。\end{note}下一辈的想他素日慈爱,以及家中仆从老小想他素日怜贫惜贱、慈老爱幼\begin{note}庚侧:八字乃为上人之当铭于五衷。\end{note}之恩,莫不悲嚎痛哭者。\begin{note}庚侧:老健。\end{note}
\end{parag}


\begin{parag}
    闲言少叙,却说宝玉因近日林黛玉回去,剩得自己孤凄,也不和人顽耍,\begin{note}甲侧:与凤姐反对。淡淡写来,方是二人自幼气味相投,可知后文皆非突然文字。\end{note}每到晚间便索然睡了。如今从梦中听见说秦氏死了,连忙翻身爬起来,只觉心中似戮了一刀的不忍,哇的一声,直奔出一口血来。\begin{note}甲侧:宝玉早已看定可继家务事者可卿也,今闻死了,大失所望。急火攻心,焉得不有此血?为玉一叹!\end{note}袭人等慌慌忙忙上来搊(校者注:蒙古王府本此处作“搂”)扶,问是怎么样,又要回贾母来请大夫。宝玉笑道:“不用忙,不相干,\begin{note}庚侧:又淡淡抹去。\end{note}这是急火攻心,\begin{note}甲侧:如何自己说出来了?\end{note}血不归经。”说著便爬起来,要衣服换了,来见贾母,即时要过去。\begin{note}庚眉:如此总是淡描轻写,全无痕迹,方见得有生以来,天分中自然所赋之性如此,非因色所感也。\end{note}袭人见他如此,心中虽放不下,又不敢拦,只是由他罢了。贾母见他要去,因说:“才咽气的人,那里不干净;二则夜里风大,明早再去不迟。”宝玉那里肯依。贾母命人备车,多派跟从人役,拥护前来。
\end{parag}


\begin{parag}
    一直到了宁国府前,只见府门洞开,两边灯笼照如白昼,乱烘烘人来人往,里面哭声摇山振岳。\begin{note}甲侧:写大族之丧,如此起绪。\end{note}宝玉下了车,忙忙奔至停灵之室,痛哭一番。然后见过尤氏。谁知尤氏正犯了胃疼旧疾,睡在床上。\begin{note}甲侧:妙!非此何以出阿凤!\end{note}\begin{note}庚侧:紧处愈紧,密处愈密。\end{note}\begin{note}庚眉:所谓层峦叠翠之法也。野史中从无此法。即观者到此,亦为写秦氏未必全到,岂料更又写一尤氏哉!\end{note}然后又出来见贾珍。彼时贾代儒带领贾敕、贾效、贾敦、贾赦、贾政、贾琮、贾㻞、贾珩、贾珖、贾琛、贾琼、贾璘、贾蔷、贾菖、贾菱、贾芸、贾芹、贾蓁、贾萍、贾藻、贾蘅、贾芬、贾芳、贾兰、贾菌、贾芝等\begin{note}庚侧:将贾族约略一总,观者方不惑。\end{note}都来了。贾珍哭的泪人一般,\begin{note}甲侧:可笑,如丧考妣,此作者刺心笔也。\end{note}正和贾代儒等说道:“合家大小,远亲近友,谁不知我这媳妇比儿子还强十倍。如今伸腿去了,可见这长房内绝灭无人了。”说著又哭起来。众人忙劝道:“人已辞世,哭也无益,且商议如何料理要紧。”\begin{note}庚侧:淡淡一句,勾出贾珍多少文字来。\end{note}贾珍拍手道:“如何料理,不过尽我所有罢了!”\begin{note}蒙双夹:“尽我所有”,为媳妇是非礼之谈,父母又将何以待之?故前此有思织酒后狂言,及今复见此语,含而不露,吾不能为贾珍隐讳。\end{note}
\end{parag}


\begin{parag}
    正说著,只见秦业、秦钟并尤氏的几个眷属\begin{note}甲侧:伏后文。\end{note}尤氏姊妹也都来了。贾珍便命贾琼、贾琛、贾璘、贾蔷四个人去陪客,一面吩咐去请钦天监阴阳司来择日,推准停灵七七四十九日,三日后开丧送讣闻。这四十九日,单请一百单八众禅僧在大厅上拜大悲忏,超度前亡后化诸魂,以免亡者之罪;另设一坛于天香楼上,\begin{note}甲侧:删却,是未删之笔。\end{note}\begin{note}靖眉:何必定用“西”字?读之令人酸鼻!\end{note}\begin{subnote}按:此条所评正文之「天香楼」,靖藏本作「西帆楼」。\end{subnote}是九十九位全真道士,打四十九日解冤洗业醮。然后停灵于会芳园中,灵前另有五十众高僧、五十众高道,对坛按七作好事。那贾敬闻得长孙媳妇死了,因自为早晚就要飞升,\begin{note}庚侧:可笑可叹。古今之儒,中途多惑老佛。王梅隐云:“若能再加东坡十年寿,亦能跳出这圈子来。”斯言信矣。\end{note}\begin{note}蒙侧:“就要飞升”的“要”,用得得当。凡“要”者,则身心急切;急切之者,百事无成。正为后文作引线。\end{note}如何肯又回家染了红尘,将前功尽弃呢,因此并不在意,只凭贾珍料理。
\end{parag}


\begin{parag}
    贾珍见父亲不管,亦发恣意奢华。看板时,几副杉木板皆不中用。可巧薛蟠来吊问,因见贾珍寻好板,便说道:“我们木店里有一副板,叫做什么樯木,\begin{note}甲眉:樯者,舟具也。所谓“人生若泛舟”而已,宁不可叹!\end{note}出在潢海铁网山上,\begin{note}甲侧:所谓迷津易堕,尘网难逃也。\end{note}作了棺材,万年不坏。这还是当年先父带来,原系义忠亲王老千岁要的,因他坏了事,\begin{note}蒙侧:“坏了事”等字毒极,写尽势利场中故套。\end{note}就不曾拿去。现今还封在店里,也没人出价敢买。你若要,就抬来罢了。”贾珍听了,喜之不尽,即命人抬来。大家看时,只见帮底皆厚八寸,纹若槟榔,味若檀麝,以手扣之,玎珰如金玉。大家都奇异称赏。贾珍笑问:“价值几何?”薛蟠笑道:“拿一千两银子来,只怕也没处买去。什么价不价,赏他们几两工钱就是了。”\begin{note}甲侧:的是阿呆兄口气。\end{note}贾珍听说,忙谢不尽,即命解锯糊漆。贾政因劝道:“此物恐非常人可享者,\begin{note}甲侧:政老有深意存焉。\end{note}殓以上等杉木也就是了。”\begin{note}甲侧:夹写贾政。\end{note}\begin{note}甲眉:写个个皆到,全无安逸之笔,深得《金瓶》壶奥!\end{note}此时贾珍恨不能代秦氏之死,这话如何肯听。\begin{note}蒙侧:“代秦氏死”等句,总是填实前文。\end{note}
\end{parag}


\begin{parag}
    因忽又听得秦氏之丫鬟名唤瑞珠者,见秦氏死了,他也触柱而亡。\begin{note}甲侧:补天香楼未删之文。\end{note}\begin{note}靖侧:是亦未删之笔。\end{note}此事可罕,合族中人也都称赞。贾珍遂以孙女之礼殓殡,一并停灵于会芳园中之登仙阁。小丫鬟名宝珠者,因见秦氏身无所出,乃甘心愿为义女,誓任摔丧驾灵之任。贾珍喜之不尽,即时传下,从此皆呼宝珠为小姐。那宝珠按未嫁女之丧,在灵前哀哀欲绝。\begin{note}甲侧:非恩惠爱人,那能如是?惜哉可卿,惜哉可卿!\end{note}于是,合族人丁并家下诸人,都各遵旧制行事,自不敢紊乱。\begin{note}甲侧:两句写尽大家。\end{note}
\end{parag}


\begin{parag}
    贾珍因想著贾蓉不过是个黉门监,\begin{note}庚侧:又起波澜,却不突然。\end{note}灵幡经榜上写时不好看,便是执事也不多,因此心下甚不自在。\begin{note}甲侧:善起波澜。\end{note}可巧这日正是首七第四日,早有大明宫掌宫内相戴权,\begin{note}甲侧:妙!大权也。\end{note}先备了祭礼遣人来,次后坐了大轿,打伞呜锣,亲来上祭。贾珍忙接著,让至逗蜂轩\begin{note}甲侧:轩名可思。\end{note}献茶。贾珍心中打算定了主意,因而趁便就说要与贾蓉捐个前程的话。戴权会意,因笑道:“想是为丧礼上风光些?”\begin{note}甲侧:难得内相机括之快如此。\end{note}贾珍忙笑道:“老内相所见不差。”戴权道:“事倒凑巧,正有个美缺。如今三百员龙禁尉短了两员,昨日襄阳侯的兄弟老三来求我,现拿了一千五百两银子,送到我家里。你知道,咱们都是老相与,不拘怎么样,看著他爷爷的分上,胡乱应了。\begin{note}甲侧:忙中写闲。\end{note}还剩了一个缺,谁知永兴节度使冯胖子来求,要与他孩子捐,我就没工夫应他。既是咱们的孩子\begin{note}甲侧:奇谈,画尽阉官口吻。\end{note}要捐,快写个履历来。”贾珍听说,忙吩咐:“快命书房里人恭敬写了大爷的履历来。”小厮不敢怠慢,去了一刻,便拿了一张红纸来与贾珍。贾珍看了,忙送与戴权。看时,上面写道:
\end{parag}


\begin{qute2sp}
    江南江宁府江宁县监生贾蓉,年二十岁。曾祖,原任京营节度使世袭一等神威将军贾代化;祖,乙卯科进士贾敬;父,世袭三品爵威烈将军贾珍。
\end{qute2sp}


\begin{parag}
    戴权看了,回手便递与一个贴身的小厮收了,说道:“回来送与户部堂官老赵,说我拜上他,起一张五品龙禁尉的票,再给个执照,就把那履历填上,明儿我来兑银子送去。”小厮答应了,戴权也就告辞了。贾珍十分款留不住,只得送出府门。临上轿,贾珍因问:“银子还是我到部兑,还是一并送入老内相府中?”戴权道:“若到部里,你又吃亏了。不如平准一千二百银子送到我家里就完了。”贾珍感谢不尽,只说:“待服满后,亲带小犬到府叩谢。”于是作别。
\end{parag}


\begin{parag}
    接著,便又听喝道之声,原来是忠靖侯史鼎的夫人来了。\begin{note}甲侧:史小姐湘云消息也。\end{note}王夫人、邢夫人、凤姐等刚迎入上房,又见锦乡侯、川宁侯、寿山伯三家祭礼摆在灵前。少时,三人下轿,贾政等忙接上大厅。如此亲朋你来我去,也不能胜数。只这四十九日,\begin{note}庚侧:就简去繁。\end{note}宁国府街上一条白漫漫人来人往,\begin{note}甲侧:是有服亲朋并家下人丁之盛。\end{note}花簇簇官去官来。\begin{note}甲侧:是来往祭吊之盛。\end{note}
\end{parag}


\begin{parag}
    贾珍命贾蓉次日换了吉服,领凭回来。灵前供用执事等物,俱按五品职例。灵牌疏上皆写“天朝诰授贾门秦氏恭人之灵位”。会芳园临街大门洞开,旋在两边起了鼓乐厅,两班青衣按时奏乐,一对对执事摆的刀斩斧齐。更有两面朱红销金大字牌对竖在门外,上面大书:“防护内廷紫禁道御前侍卫龙禁尉”。对面高起著宣坛,僧道对坛榜文,榜上大书:“世袭宁国公冢孙妇、防护内廷御前侍卫龙禁尉贾门秦氏恭人之丧。\begin{note}庚眉:贾珍是乱费,可卿却实如此。\end{note}四大部州至中之地,奉天承运太平之国,\begin{note}庚眉:奇文。若明指一州名,似若《西游》之套,故曰至中之地,不待言可知是光天化日仁风德雨之下矣。不云国名更妙,可知是尧街舜巷衣冠礼义之乡矣。直与第一回呼应相接。\end{note}总理虚无寂静教门僧录司正堂万虚、总理元始三一教门道录司正堂叶生等,敬谨修斋,朝天叩佛”,以及“恭请诸伽蓝、揭谛、功曹等神,圣恩普锡,神威远镇,四十九日消灾洗业平安水陆道场”等语,亦不消繁记。
\end{parag}


\begin{parag}
    只是贾珍虽然此时心意满足,\begin{note}蒙侧:可笑。\end{note}但里面尤氏又犯了旧疾,不能料理事务,惟恐各诰命来往,亏了礼数,怕人笑话,因此心中不自在。当下正忧虑时,因宝玉\begin{note}甲侧:余正思如何高搁起玉兄了。\end{note}在侧问道:“事事都算安贴了,大哥哥还愁什么?”贾珍见问,便将里面无人的话说了出来。宝玉听说笑道:“这有何难,我荐一个人\begin{note}甲侧:荐凤姐须得宝玉,俱龙华会上人也。\end{note}与你权理这一个月的事,管必妥当。”贾珍忙问:“是谁?”宝玉见座间还有许多亲友,不便明言,走至贾珍耳边说了两句。贾珍听了喜不自禁,连忙起身道:“果然安贴,如今就去。”说著拉了宝玉,辞了众人,便往上房里来。
\end{parag}


\begin{parag}
    可巧这日非正经日期,亲友来的少,里面不过几位近亲堂客,邢夫人、王夫人、凤姐并合族中的内眷陪坐。闻人报:“大爷进来了。”唬的众婆娘唿的一声,往后藏之不迭,\begin{note}甲侧:数日行止可知。作者自是笔笔不空,批者亦字字留神之至矣。\end{note}独凤姐款款站了起来。\begin{note}庚侧:又写凤姐。\end{note}贾珍此时也有些病症在身,二则过于悲痛了,因拄拐踱了进来。邢夫人等因说道:“你身上不好,又连日事多,该歇歇才是,又进来做什么?”贾珍一面扶拐,\begin{note}庚侧:一丝不乱。\end{note}\begin{note}靖眉:刺心之笔。\end{note}扎挣著要蹲身跪下请安道乏。邢夫人等忙叫宝玉搀住,命人挪椅子来与他坐。贾珍断不肯坐,因勉强陪笑道:“侄儿进来有一件事要求二位婶子并大妹。”邢夫人等忙问:“什么事?”贾珍忙道:“婶子自然知道,如今孙子媳妇没了,侄儿媳妇偏又病倒,我看里头著实不成个体统。怎么屈尊大妹妹一个月,\begin{note}庚侧:不见突然。\end{note}在这里料理料理,我就放心了。”\begin{note}庚侧:阿凤此刻心痒矣。\end{note}邢夫人笑道:“原来为这个。你大妹妹现在你二婶子家,只和你二婶子说就是了。”王夫人忙道:“他一个小孩子\begin{note}庚侧:三字愈令人可爱可怜。\end{note}家何曾经过这样事,倘或料理不清,反叫人笑话,倒是再烦别人好。”贾珍笑道:“婶子的意思侄儿猜著了,是怕大妹妹劳苦了。若说料理不开,我包管必料理的开,便是错一点儿,别人看著还是不错的。从小儿大妹妹顽笑著就有杀伐决断,\begin{note}庚侧:阿凤身份。\end{note}如今出了阁,又在那府里办事,越发历练老成了。我想了这几日,除了大妹妹再无人了。婶子不看侄儿、侄儿媳妇的分上,只看死了的分上罢!”说著滚下泪来。\begin{note}庚侧:有笔力。\end{note}
\end{parag}


\begin{parag}
    王夫人心中怕的是凤姐未经过丧事,怕他料理不清,惹人耻笑。今见贾珍苦苦的说到这步田地,心中已活了几分,却又眼看著凤姐出神。那凤姐素日最喜揽事办,好卖弄才干,虽然当家妥当,也因未办过婚丧大事,恐人还不伏,巴不得遇见这事。今见贾珍如此一来,他心中早已欢喜。先见王夫人不允,后见贾珍说的情真,王夫人有活动之意,便向王夫人道:“大哥哥说的这么恳切,太太就依了罢。”王夫人悄悄的道:“你可能么?”凤姐道:“有什么不能的。外面的大事已经大哥哥\begin{note}庚旁批:王夫人是悄言,凤姐是响应,故称“大哥哥”。\end{note}料理清了,\begin{note}庚侧:已得三昧矣。\end{note}不过是里头照管照管,便是我有不知道的,问问太太就是了。”\begin{note}甲侧:胸中成见已有之语。\end{note}王夫人见说的有理,便不作声。贾珍见凤姐允了,又陪笑道:“也管不得许多了,横竖要求大妹妹辛苦辛苦。我这里先与妹妹行礼,等事完了,我再到那府里去谢。”说著,就作揖下去,凤姐儿还礼不迭。
\end{parag}


\begin{parag}
    贾珍便忙向袖中取了宁国府对牌出来,命宝玉送与凤姐,又说:“妹妹爱怎样就怎样,要什么只管拿这个取去,也不必问我。只求别存心替我省钱,只要好看为上;二则也要同那府里一样待人才好,不要存心怕人抱怨。只这两件外,我再没不放心的了。”凤姐不敢就接牌,\begin{note}蒙双夹:凡有本领者断不越礼。接牌小事而必待命于王夫人也,诚家道之规范,亦天下之规范也。看是书者不可草草从事。\end{note}只看著王夫人。王夫人道:“你哥哥既这么说,你就照看照看罢了。只是别自作主意,有了事,打发人问你哥哥、嫂子要紧。”宝玉早向贾珍手里接过对牌来,强递与凤姐了。又问:“妹妹住在这里,还是天天来呢?若是天天来,越发辛苦了。不如我这里赶著收拾出一个院落来,妹妹住过这几日倒安稳。”凤姐笑道:“不用。\begin{note}甲侧:二字句,有神。\end{note}那边也离不得我,倒是天天来的好。”贾珍听说,只得罢了。然后又说了一回闲话,方才出去。
\end{parag}


\begin{parag}
    一时女眷散后,王夫人因问凤姐:“你今儿怎么样?”凤姐儿道:“太太只管请回去,我须得先理出一个头绪来,才回去得呢。”王夫人听说,便先同邢夫人等回去,不在话下。
\end{parag}


\begin{parag}
    这里凤姐儿来至三间一所抱厦内坐了,因想:头一件是人口混杂,遗失东西;第二件,事无专责,临期推委;第三件,需用过费,滥支冒领;第四件,任无大小,苦乐不均;第五件,家人豪纵,有脸者不服钤束,无脸者不能上进。\begin{note}甲眉:旧族后辈受此五病者颇多,余家更甚。三十年前事见书于三十年后,令余悲痛血泪盈面。\end{note}\begin{note}庚眉:读五件事未完,余不禁失声大哭,三十年前作书人在何处耶?\end{note}此五件实是宁国府中风俗。不知凤姐如何处治,且听下回分解。\begin{note}甲眉:此回只十页,因删去天香楼一节,少去四五页也。\end{note}
\end{parag}


\begin{parag}
    正是:
\end{parag}


\begin{poem}
    \begin{pl} 金紫万千谁治国,裙钗一二可齐家。\end{pl}
    \begin{note}蒙:五件事若能如法整理得当,岂独家庭,国家天下治之不难。\end{note}
\end{poem}


\begin{parag}
    \begin{note}甲:“秦可卿淫丧天香楼”,作者用史笔也。老朽因有魂托凤姐贾家后事二件,的是安富尊荣坐享人不能想得到处。其事虽未行,其言其意则令人悲切感服,姑赦之,因命芹溪删去。\end{note}
\end{parag}


\begin{parag}
    \begin{note}庚:通回将可卿如何死故隐去,是大发慈悲心也,叹叹!壬午春。\end{note}
\end{parag}


\begin{parag}
    \begin{note}蒙回末总评:借可卿之死,又写出情之变态,上下大小,男女老少,无非情感而生情。且又藉凤姐之梦,更化就幻空中一片贴切之情,所谓寂然不动,感而遂通。所感之象,所动之萌,深浅诚伪,随种必报,所谓幻者此也,情者亦此也。何非幻,何非情?情即是幻,幻即是情,明眼者自见。\end{note}
\end{parag}

