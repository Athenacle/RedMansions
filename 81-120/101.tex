\chap{一百零一}{大观园月夜感幽魂 散花寺神签惊异兆}


\begin{parag}
    却说凤姐回至房中,见贾琏尚未回来,便分派那管办探春行装奁事的一干人。那天已有黄昏以后,因忽然想起探春来,要瞧瞧他去,便叫丰儿与两个丫头跟着,头里一个丫头打着灯笼。走出门来,见月光已上,照耀如水。凤姐便命打灯笼的“回去罢。”因而走至茶房窗下,听见里面有人嘁嘁喳喳的,又似哭,又似笑,又似议论什么的。凤姐知道不过是家下婆子们又不知搬什么是非,心内大不受用,便命小红进去,装做无心的样子细细打听着,用话套出原委来。小红答应着去了。凤姐只带着丰儿来至园门前,门尚未关,只虚虚的掩着。于是主仆二人方推门进去,只见园中月色比着外面更觉明朗,满地下重重树影,杳无人声,甚是凄凉寂静。刚欲往秋爽斋这条路来,只听忽的一声风过,吹的那树枝上落叶满园中唰喇喇的作响,枝梢上吱喽喽发哨,将那些寒鸦宿鸟都惊飞起来。凤姐吃了酒,被风一吹,只觉身上发噤起来。那丰儿也把头一缩说:“好冷!”凤姐也撑不住,便叫丰儿:“快回去把那件银鼠坎肩儿拿来,我在三姑娘那里等着。”丰儿巴不得一声,也要回去穿衣裳来,答应了一声,回头就跑了。
\end{parag}


\begin{parag}
    凤姐刚举步走了不远,只觉身后咈咈哧哧,似有闻嗅之声,不觉头发森然竖了起来。由不得回头一看,只见黑油油一个东西在后面伸着鼻子闻他呢,那两只眼睛恰似灯光一般。凤姐吓的魂不附体,不觉失声的咳了一声。却是一只大狗。那狗抽头回身,拖着一个扫帚尾巴,一气跑上大土山上方站住了,回身犹向凤姐拱爪儿。凤姐儿此时心跳神移,急急的向秋爽斋来。已将来至门口,方转过山子,只见迎面有一个人影儿一恍。凤姐心中疑惑,心里想着必是那一房里的丫头,便问:“是谁?”问了两声,并没有人出来,已经吓得神魂飘荡。恍恍忽忽的似乎背后有人说道:“婶娘连我也不认得了!”凤姐忙回头一看,只见这人形容俊俏,衣履风流,十分眼熟,只是想不起是那房那屋里的媳妇来。只听那人又说道:“婶娘只管享荣华受富贵的心盛,把我那年说的立万年永远之基都付于东洋大海了。”凤姐听说,低头寻思,总想不起。那人冷笑道:“婶娘那时怎样疼我了,如今就忘在九霄云外了。”凤姐听了,此时方想起来是贾蓉的先妻秦氏,便说道:“嗳呀,你是死了的人哪,怎么跑到这里来了呢!”啐了一口,方转回身,脚下不防一块石头绊了一跤,犹如梦醒一般,浑身汗如雨下。虽然毛发悚然,心中却也明白,只见小红丰儿影影绰绰的来了。凤姐恐怕落人的褒贬,连忙爬起来说道:“你们做什么呢,去了这半天?快拿来我穿上罢。”一面丰儿走至跟前伏侍穿上,小红过来搀扶。凤姐道:“我才到那里,他们都睡了。咱们回去罢。”一面说,一面带了两个丫头急急忙忙回到家中。贾琏已回来了,只是见他脸上神色更变,不似往常,待要问他,又知他素日性格,不敢突然相问,只得睡了。至次日五更,贾琏就起来要往总理内庭都检点太监裘世安家来打听事务。因太早了,见桌上有昨日送来的抄报,便拿起来闲看。第一件是云南节度使王忠一本,新获了一起私带神枪火药出边事,共有十八名人犯。头一名鲍音,口称系太师镇国公贾化家人。第二件苏州刺史李孝一本,参劾纵放家奴,倚势凌辱军民,以致因奸不遂杀死节妇一家人命三口事。凶犯姓时名福,自称系世袭三等职衔贾范家人。贾琏看见这两件,心中早又不自在起来,待要看第三件,又恐迟了不能见裘世安的面,因此急急的穿了衣服,也等不得吃东西,恰好平儿端上茶来,喝了两口,便出来骑马走了。
\end{parag}


\begin{parag}
    平儿在房内收拾换下的衣服。此时凤姐尚未起来,平儿因说道:“今儿夜里我听着奶奶没睡什么觉,我这会子替奶奶捶着,好生打个盹儿罢。”凤姐半日不言语。平儿料着这意思是了,便爬上炕来坐在身边轻轻的捶着。才捶了几拳,那凤姐刚有要睡之意,只听那边大姐儿哭了。凤姐又将眼睁开,平儿连向那边叫道:“李妈,你到底是怎么着?姐儿哭了。你到底拍着他些。你也忒好睡了。”那边李妈从梦中惊醒,听得平儿如此说,心中没好气,只得狠命拍了几下,口里嘟嘟哝哝的骂道:“真真的小短命鬼儿,放着尸不挺,三更半夜嚎你娘的丧!”一面说,一面咬牙便向那孩子身上拧了一把。那孩子哇的一声大哭起来了。凤姐听见,说“了不得!你听听,他该挫磨孩子了。你过去把那黑心的养汉老婆下死劲的打他几下子,把妞妞抱过来。”平儿笑道:“奶奶别生气,他那里敢挫磨姐儿,只怕是不提防错碰了一下子也是有的。这会子打他几下子没要紧,明儿叫他们背地里嚼舌根,倒说三更半夜打人。”凤姐听了,半日不言语,长叹一声说道:“你瞧瞧,这会子不是我十旺八旺的呢!明儿我要是死了,剩下这小孽障,还不知怎么样呢!”平儿笑道:“奶奶这怎么说!大五更的,何苦来呢!”凤姐冷笑道:“你那里知道,我是早已明白了。我也不久了。虽然活了二十五岁,人家没见的也见了,没吃的也吃了,也算全了。所有世上有的也都有了。气也算赌尽了,强也算争足了,就是寿字儿上头缺一点儿,也罢了。”平儿听说,由不的滚下泪来。凤姐笑道:“你这会子不用假慈悲,我死了你们只有欢喜的。你们一心一计和和气气的,省得我是你们眼里的刺似的。只有一件,你们知好歹只疼我那孩子就是了。”平儿听说这话,越发哭的泪人似的。凤姐笑道:“别扯你娘的臊了,那里就死了呢。哭的那么痛!我不死还叫你哭死了呢。”平儿听说,连忙止住哭,道:“奶奶说得这么伤心。”一面说,一面又捶,半日不言语,凤姐又朦胧睡去。
\end{parag}


\begin{parag}
    平儿方下炕来要去,只听外面脚步响。谁知贾琏去迟了,那裘世安已经上朝去了,不遇而回,心中正没好气,进来就问平儿道:“那些人还没起来呢么?”平儿回说:“没有呢。”贾琏一路摔帘子进来,冷笑道:“好,好,这会子还都不起来,安心打擂台打撒手儿!”一迭声又要吃茶。平儿忙倒了一碗茶来。原来那些丫头老婆见贾琏出了门又复睡了,不打谅这会子回来,原不曾预备。平儿便把温过的拿了来。贾琏生气,举起碗来,哗啷一声摔了个粉碎。
\end{parag}


\begin{parag}
    凤姐惊醒,唬了一身冷汗,嗳哟一声,睁开眼,只见贾琏气狠狠的坐在旁边,平儿弯着腰拾碗片子呢。凤姐道:“你怎么就回来了?”问了一声,半日不答应,只得又问一声。贾琏嚷道:“你不要我回来,叫我死在外头罢!”凤姐笑道:“这又是何苦来呢!常时我见你不象今儿回来的快,问你一声,也没什么生气的。”贾琏又嚷道:“又没遇见,怎么不快回来呢!”凤姐笑道:“没有遇见,少不得奈烦些,明儿再去早些儿,自然遇见了。”贾琏嚷道:“我可不吃着自己的饭替人家赶獐子呢。我这里一大堆的事没个动秤儿的,没来由为人家的事,瞎闹了这些日子,当什么呢!正经那有事的人还在家里受用,死活不知,还听见说要锣鼓喧天的摆酒唱戏做生日呢。我可瞎跑他娘的腿子!”一面说,一面往地下啐了一口,又骂平儿。凤姐听了,气的干咽,要和他分证,想了一想,又忍住了,勉强陪笑道:“何苦来生这么大气,大清早起和我叫喊什么。谁叫你应了人家的事?你既应了,就得耐烦些,少不得替人家办办。也没见这个人自己有为难的事还有心肠唱戏摆酒的闹!”贾琏道:“你可说么,你明儿倒也问问他!”凤姐诧异道:“问谁?”贾琏道:“问谁!问你哥哥。”凤姐道:“是他吗?”贾琏道:“可不是他,还有谁呢!”凤姐忙问道:“他又有什么事叫你替他跑?”贾琏道:“你还在坛子里呢。”凤姐道:“真真这就奇了,我连一个字儿也不知道。”贾琏道:“你怎么能知道呢,这个事连太太和姨太太还不知道呢。头一件怕太太和姨太太不放心,二则你身上又常嚷不好,所以我在外头压住了,不叫里头知道的。说起来真真可人恼!你今儿不问我,我也不便告诉你。你打谅你哥哥行事象个人呢,你知道外头人都叫他什么?”凤姐道:“叫他什么?”贾琏道:“叫他什么,叫他‘忘仁’!”凤姐扑哧的一笑:“他可不叫王仁叫什么呢。”贾琏道:“你打谅那个王仁吗,是忘了仁义礼智信的那个‘忘仁’哪!”凤姐道:“这是什么人这么刻薄嘴儿遭塌人。”贾琏道:“不是遭塌他吗,今儿索性告诉你,你也不知道知道你那哥哥的好处,到底知道他给他二叔做生日啊!”凤姐想了一想道:“嗳哟,可是呵,我还忘了问你,二叔不是冬天的生日吗?我记得年年都是宝玉去。前者老爷升了,二叔那边送过戏来,我还偷偷儿的说,二叔为人是最啬刻的,比不得大舅太爷。他们各自家里还乌眼鸡似的。不么,昨儿大舅太爷没了,你瞧他是个兄弟,他还出了个头儿揽了个事儿吗!所以那一天说,赶他的生日咱们还他一班子戏,省了亲戚跟前落亏欠。如今这么早就做生日,也不知道是什么意思。”贾琏道:“你还作梦呢。他一到京,接着舅太爷的首尾就开了一个吊,他怕咱们知道拦他,所以没告诉咱们,弄了好几千银子。后来二舅嗔着他,说他不该一网打尽。他吃不住了,变了个法子就指着你们二叔的生日撒了个网,想着再弄几个钱好打点二舅太爷不生气,也不管亲戚朋友冬天夏天的,人家知道不知道,这么丢脸!你知道我起早为什么?这如今因海疆的事情御史参了一本,说是大舅太爷的亏空,本员已故,应着落其弟王子胜,侄王仁赔补。爷儿两个急了,找了我给他们托人情。我见他们吓的那么个样儿,再者又关系太太和你,我才应了。想着找找总理内庭都检点老裘替办办,或者前任后任挪移挪移。偏又去晚了,他进里头去了,我白起来跑了一趟。他们家里还那里定戏摆酒呢。你说说,叫人生气不生气!”
\end{parag}


\begin{parag}
    凤姐听了,才知王仁所行如此。但他素性要强护短,听贾琏如此说,便道:“凭他怎么样,到底是你的亲大舅儿。再者,这件事死的大太爷活的二叔都感激你。罢了,没什么说的,我们家的事,少不得我低三下四的求你了,省的带累别人受气,背地里骂我。”说着,眼泪早流下来,掀开被窝一面坐起来,一面挽头发,一面披衣裳。贾琏道:“你倒不用这么着,是你哥哥不是人,我并没说你呀。况且我出去了,你身上又不好,我都起来了,他们还睡觉。咱们老辈子有这个规矩么!你如今作好好先生不管事了。我说了一句你就起来,明儿我要嫌这些人,难道你都替了他们么。好没意思啊!”凤姐听了这些话,才把泪止住了,说道:“天呢不早了,我也该起来了。你有这么说的,你替他们家在心的办办,那就是你的情分了。再者也不光为我,就是太太听见也喜欢。”贾琏道:“是了,知道了。‘大萝卜还用屎浇’。”平儿道:“奶奶这么早起来做什么,那一天奶奶不是起来有一定的时候儿呢。爷也不知是那里的邪火,拿着我们出气。何苦来呢,奶奶也算替爷挣够了,那一点儿不是奶奶挡头阵。不是我说,爷把现成儿的也不知吃了多少,这会子替奶奶办了一点子事,又关会着好几层儿呢,就是这么拿糖作醋的起来,也不怕人家寒心。况且这也不单是奶奶的事呀。我们起迟了,原该爷生气,左右到底是奴才呀。奶奶跟前尽着身子累的成了个病包儿了,这是何苦来呢。”说着,自己的眼圈儿也红了。那贾琏本是一肚子闷气,那里见得这一对娇妻美妾又尖利又柔情的话呢,便笑道:“够了,算了罢。他一个人就够使的了,不用你帮着。左右我是外人,多早晚我死了,你们就清净了。”凤姐道:“你也别说那个话,谁知道谁怎么样呢。你不死我还死呢,早死一天早心净。”说着,又哭起来。平儿只得又劝了一回。那时天已大亮,日影横窗。贾琏也不便再说,站起来出去了。
\end{parag}


\begin{parag}
    这里凤姐自己起来,正在梳洗,忽见王夫人那边小丫头过来道:“太太说了,叫问二奶奶今日过舅太爷那边去不去?要去,说叫二奶奶同着宝二奶奶一路去呢。”凤姐因方才一段话,已经灰心丧意,恨娘家不给争气,又兼昨夜园中受了那一惊,也实在没精神,便说道:“你先回太太去,我还有一两件事没办清,今日不能去。况且他们那又不是什么正经事。宝二奶奶要去各自去罢。”小丫头答应着,回去回复了。不在话下。
\end{parag}


\begin{parag}
    且说凤姐梳了头,换了衣服,想了想,虽然自己不去,也该带个信儿。再者,宝钗还是新媳妇,出门子自然要过去照应照应的。于是见过王夫人,支吾了一件事,便过来到宝玉房中。只见宝玉穿着衣服歪在炕上,两个眼睛呆呆的看宝钗梳头。凤姐站在门口,还是宝钗一回头看见了,连忙起身让坐。宝玉也爬起来,凤姐才笑嘻嘻的坐下。宝钗因说麝月道”你们瞧着二奶奶进来也不言语声儿。”麝月笑着道:“二奶奶头里进来就摆手儿不叫言语么。”凤姐因向宝玉道:“你还不走,等什么呢。没见这么大人了还是这么小孩子气的。人家各自梳头,你爬在旁边看什么?成日家一块子在屋里还看不够?也不怕丫头们笑话。”说着,哧的一笑,又瞅着他咂嘴儿。宝玉虽也有些不好意思,还不理会,把个宝钗直臊的满脸飞红,又不好听着,又不好说什么,只见袭人端过茶来,只得搭讪着自己递了一袋烟。凤姐儿笑着站起来接了,道:“二妹妹,你别管我们的事,你快穿衣服罢。”宝玉一面也搭讪着找这个,弄那个。凤姐道:“你先去罢,那里有个爷们等着奶奶们一块儿走的理呢。”宝玉道:“我只是嫌我这衣裳不大好,不如前年穿着老太太给的那件雀金呢好。”凤姐因怄他道:“你为什么不穿?”宝玉道:“穿着太早些。”凤姐忽然想起,自悔失言,幸亏宝钗也和王家是内亲,只是那些丫头们跟前已经不好意思了。袭人却接着说道:“二奶奶还不知道呢,就是穿得,他也不穿了。”凤姐儿道:“这是什么原故?”袭人道:“告诉二奶奶,真真是我们这位爷的行事都是天外飞来的。那一年因二舅太爷的生日,老太太给了他这件衣裳,谁知那一天就烧了。我妈病重了,我没在家。那时候还有晴雯妹妹呢,听见说病着整给他补了一夜,第二天老太太才没瞧出来呢。去年那一天上学天冷,我叫焙茗拿了去给他披披。谁知这位爷见了这件衣裳想起晴雯来了,说了总不穿了,叫我给他收一辈子呢。”凤姐不等说完,便道:“你提晴雯,可惜了儿的,那孩子模样儿手儿都好,就只嘴头子利害些。偏偏儿的太太不知听了那里的谣言,活活儿的把个小命儿要了。还有一件事,那一天我瞧见厨房里柳家的女人他女孩儿,叫什么五儿,那丫头长的和晴雯脱了个影儿似的。我心里要叫他进来,后来我问他妈,他妈说是很愿意。我想着宝二爷屋里的小红跟了我去,我还没还他呢,就把五儿补过来。平儿说太太那一天说了,凡象那个样儿的都不叫派到宝二爷屋里呢。我所以也就搁下了。这如今宝二爷也成了家了,还怕什么呢,不如我就叫他进来。可不知宝二爷愿意不愿意?要想着晴雯,只瞧见这五儿就是了。”宝玉本要走,听见这些话已呆了。袭人道:“为什么不愿意,早就要弄了来的,只是因为太太的话说的结实罢了。”凤姐道:“那么着明儿我就叫他进来。太太的跟前有我呢。”宝玉听了,喜不自胜,才走到贾母那边去了。这里宝钗穿衣服。凤姐儿看他两口儿这般恩爱缠绵,想起贾琏方才那种光景,好不伤心,坐不住,便起身向宝钗笑道:“我和你向老太太屋里去罢。”笑着出了房门,一同来见贾母。
\end{parag}


\begin{parag}
    宝玉正在那里回贾母往舅舅家去。贾母点头说道:“去罢,只是少吃酒,早些回来。你身子才好些。”宝玉答应着出来,刚走到院内,又转身回来向宝钗耳边说了几句不知什么。宝钗笑道:“是了,你快去罢。”将宝玉催着去了。这贾母和凤姐宝钗说了没三句话,只见秋纹进来传说:“二爷打发焙茗转来,说请二奶奶。”宝钗说道:“他又忘了什么,又叫他回来?”秋纹道:“我叫小丫头问了,焙茗说是‘二爷忘了一句话,二爷叫我回来告诉二奶奶:若是去呢,快些来罢,若不去呢,别在风地里站着。’”说的贾母凤姐并地下站着的众老婆子丫头都笑了。宝钗飞红了脸,把秋纹啐了一口,说道:“好个糊涂东西!这也值得这样慌慌张张跑了来说。”秋纹也笑着回去叫小丫头去骂焙茗。那焙茗一面跑着,一面回头说道:“二爷把我巴巴的叫下马来,叫回来说的。我若不说,回来对出来又骂我了。这会子说了,他们又骂我。”那丫头笑着跑回来说了。贾母向宝钗道:“你去罢,省得他这么记挂。”说的宝钗站不住,又被凤姐怄他顽笑,没好意思,才走了。
\end{parag}


\begin{parag}
    只见散花寺的姑子大了来了,给贾母请安,见过了凤姐,坐着吃茶。贾母因问他:“这一向怎么不来?”大了道:“因这几日庙中作好事,有几位诰命夫人不时在庙里起坐,所以不得空儿来。今日特来回老祖宗,明儿还有一家作好事,不知老祖宗高兴不高兴,若高兴也去随喜随喜。”贾母便问:“做什么好事?”大了道:“前月为王大人府里不干净,见神见鬼的,偏生那太太夜间又看见去世的老爷。因此昨日在我庙里告诉我,要在散花菩萨跟前许愿烧香,做四十九天的水陆道场,保佑家口安宁,亡者升天,生者获福。所以我不得空儿来请老太太的安。”却说凤姐素日最厌恶这些事的,自从昨夜见鬼,心中总是疑疑惑惑的,如今听了大了这些话,不觉把素日的心性改了一半,已有三分信意,便问大了道:“这散花菩萨是谁?他怎么就能避邪除鬼呢?”大了见问,便知他有些信意,便说道:“奶奶今日问我,让我告诉奶奶知道。这个散花菩萨来历根基不浅,道行非常。生在西天大树国中,父母打柴为生。养下菩萨来,头长三角,眼横四目,身长三尺,两手拖地。父母说这是妖精,便弃在冰山之后了。谁知这山上有一个得道的老猢狲出来打食,看见菩萨顶上白气冲天,虎狼远避,知道来历非常,便抱回洞中抚养。谁知菩萨带了来的聪慧,禅也会谈,与猢狲天天谈道参禅,说的天花散漫缤纷。至一千年后飞升了。至今山上犹见谈经之处天花散漫,所求必灵,时常显圣,救人苦厄。因此世人才盖了庙,塑了像供奉。”凤姐道:“这有什么凭据呢?”大了道:“奶奶又来搬驳了。一个佛爷可有什么凭据呢?就是撒谎,也不过哄一两个人罢咧,难道古往今来多少明白人都被他哄了不成。奶奶只想,惟有佛家香火历来不绝,他到底是祝国祝民,有些灵验,人才信服。”凤姐听了大有道理,因道:“既这么,我明儿去试试。你庙里可有签?我去求一签,我心里的事签上批的出?批的出来我从此就信了。”大了道:“我们的签最是灵的,明儿奶奶去求一签就知道了。”贾母道:“既这么着,索性等到后日初一你再去求。”说着,大了吃了茶,到王夫人各房里去请了安,回去不提。
\end{parag}


\begin{parag}
    这里凤姐勉强扎挣着,到了初一清早,令人预备了车马,带着平儿并许多奴仆来至散花寺。大了带了众姑子接了进去。献茶后,便洗手至大殿上焚香。那凤姐儿也无心瞻仰圣像,一秉虔诚,磕了头,举起签筒默默的将那见鬼之事并身体不安等故祝告了一回。才摇了三下,只听唰的一声,筒中撺出一支签来。于是叩头拾起一看,只见写着“第三十三签,上上大吉。”大了忙查签薄看时,只见上面写着“王熙凤衣锦还乡”。凤姐一见这几个字,吃一大惊,惊问大了道:“古人也有叫王熙凤的么?”大了笑道:“奶奶最是通今博古的,难道汉朝的王熙凤求官的这一段事也不晓得?”周瑞家的在旁笑道:“前年李先儿还说这一回书的,我们还告诉他重着奶奶的名字不要叫呢。”凤姐笑道:“可是呢,我倒忘了。”说着,又瞧底下的,写的是:
\end{parag}


\begin{poem}
    \begin{pl}
        去国离乡二十年,于今衣锦返家园。
    \end{pl}


    \begin{pl}
        蜂采百花成蜜后,为谁辛苦为谁甜!
    \end{pl}

\end{poem}


\begin{parag}
    行人至,音信迟,讼宜和,婚再议。看完也不甚明白。大了道:“奶奶大喜。这一签巧得很,奶奶自幼在这里长大,何曾回南京去了。如今老爷放了外任,或者接家眷来,顺便还家,奶奶可不是‘衣锦还乡’了?”一面说,一面抄了个签经交与丫头。凤姐也半疑半信的。大了摆了斋来,凤姐只动了一动,放下了要走,又给了香银。大了苦留不住,只得让他走了。凤姐回至家中,见了贾母王夫人等,问起签来,命人一解,都欢喜非常,”或者老爷果有此心,咱们走一趟也好。”凤姐儿见人人这么说,也就信了。不在话下。
\end{parag}


\begin{parag}
    却说宝玉这一日正睡午觉,醒来不见宝钗,正要问时,只见宝钗进来。宝玉问道:“那里去了?半日不见。”宝钗笑道:“我给凤姐姐瞧一回签。”宝玉听说,便问是怎么样的。宝钗把签帖念了一回,又道:“家中人人都说好的。据我看,这‘衣锦还乡’四字里头还有原故,后来再瞧罢了。”宝玉道:“你又多疑了,妄解圣意。‘衣锦还乡’四字从古至今都知道是好的,今儿你又偏生看出缘故来了。依你说,这‘衣锦还乡’还有什么别的解说?”宝钗正要解说,只见王夫人那边打发丫头过来请二奶奶。宝钗立刻过去。未知何事,下回分解。
\end{parag}