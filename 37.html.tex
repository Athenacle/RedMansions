\chap{三十七}{秋爽斋偶结海棠社 蘅芜苑夜拟菊花题}
\begin{parag}
    \begin{note}庚辰:美人用别号亦新奇花样,且韵且雅,呼去觉满口生香。结社出自探春意,作者已伏下回“兴利除弊”之文也。\end{note}
\end{parag}


\begin{parag}
    \begin{note}庚辰:此回才放笔写诗、写词、作扎,看他诗复诗、词复词、扎又扎,总不相犯。\end{note}
\end{parag}


\begin{parag}
    \begin{note}庚辰:湘云,诗客也,前回写之其今才起社,后用不即不离闲人数语数折,仍归社中。何巧活之笔如此?\end{note}
\end{parag}


\begin{parag}
    \begin{note}蒙回前总批:海棠名诗社,林史傲秋闺。纵有才八斗,不如富贵儿。\end{note}
\end{parag}


\begin{parag}
    这年贾政又点了学差,择于八月二十日起身。是日拜过宗祠及贾母起身,宝玉诸子弟等送至洒泪亭。
\end{parag}


\begin{parag}
    却说贾政出门去后,外面诸事不能多记。单表宝玉每日在园中任意纵性的逛荡,真把光阴虚度,岁月空添。这日正无聊之际,只见翠墨进来,手里拿著一副花笺送与他。宝玉因道:“可是我忘了,才说要瞧瞧三妹妹去的,可好些了,你偏走来。”翠墨道:“姑娘好了,今儿也不吃药了,不过是凉著一点儿。”宝玉听说,便展开花笺看时,上面写道:
\end{parag}

\begin{qute2sp}
    娣探谨奉二兄文几:前夕新霁,月色如洗,因惜清景难逢,讵忍就卧,时漏已三转,犹徘徊于桐槛之下,未防风露所欺,致获采薪之患。昨蒙亲劳抚嘱,复又数遣侍儿问切,兼以鲜荔并真卿墨迹见赐,何痌瘝惠爱之深哉!今因伏几凭床处默之时,因思及历来古人中处名攻利敌之场,犹置一些山滴水之区,远招近揖,投辖攀辕,务结二三同志盘桓于其中,或竖词坛,或开吟社,虽一时之偶兴,遂成千古之佳谈。娣虽不才,窃同叨栖处于泉石之间,而兼慕薛林之技。风庭月榭,惜未宴集诗人;帘杏溪桃,或可醉飞吟盏。孰谓莲社之雄才,独许须眉;直以东山之雅会,让余脂粉。若蒙棹雪而来,娣则扫花以待。此谨奉。
\end{qute2sp}

\begin{parag}
    宝玉看了,不觉喜的拍手笑道:“倒是三妹妹的高雅,我如今就去商议。”一面说,一面就走,翠墨跟在后面。刚到了沁芳亭,只见园中后门上值日的婆子手里拿著一个字帖走来,见了宝玉,便迎上去,口内说道:“芸哥儿请安,在后门只等著,叫我送来的。”宝玉打开看时,写道是:
\end{parag}

\begin{qute2sp}
    不肖男芸恭请父亲大人万福金安。男思自蒙天恩,认于膝下,日夜思一孝顺,竟无可孝顺之处。前因买办花草,上托大人金福,竟认得许多花儿匠,\begin{note}庚辰双行夹批:直欲喷饭,真好新鲜文字。\end{note}并认得许多名园。因忽见有白海棠一种,不可多得。故变尽方法,只弄得两盆。大人若视男是亲男一般,\begin{note}庚辰双行夹批:皆千古未有之奇文,初读令人不解,思之则喷饭。\end{note}便留下赏玩。因天气暑热,恐园中姑娘们不便,故不敢面见。奉书恭启,并叩台安。男芸跪书。\begin{note}蒙双行夹批:一笑\end{note}
\end{qute2sp}

\begin{parag}
    宝玉看了,笑道:“独他来了,还有什么人?”婆子道:“还有两盆花儿。”宝玉道:“你出去说,我知道了,难为他想著。你便把花儿送到我屋里去就是了。”一面说,一面同翠墨往秋爽斋来,只见宝钗、黛玉、迎春、惜春已都在那里了。\begin{note}蒙双行夹批:却因芸之一字工夫,已将诸艳请来,省却多少闲文。不然必云如何请如何来,则必至齐犯宝玉,终成重复之文矣。\end{note}
\end{parag}


\begin{parag}
    众人见他进来,都笑说:“又来了一个。”探春笑道:“我不算俗,偶然起个念头,写了几个帖儿试一试,谁知一招皆到。”宝玉笑道:“可惜迟了,早该起个社的。”黛玉道:“你们只管起社,可别算上我,我是不敢的。”迎春笑道:“你不敢谁还敢呢。”\begin{note}庚辰双行夹批:必得如此方是妙文。若也如宝玉说兴头说,则不是黛玉矣。\end{note}宝玉道:“这是一件正经大事,大家鼓舞起来,不要你谦我让的。各有主意自管说出来大家平章。\begin{note}庚辰双行夹批:这是“正经大事”已妙,且曰 “平章”,更妙!的是宝玉的口角。\end{note}宝姐姐也出个主意,林妹妹也说个话儿。”宝钗道:“你忙什么,人还不全呢。”\begin{note}庚辰双行夹批:妙!宝钗自有主见,真不诬也。\end{note}一语未了,李纨也来了,进门笑道:“雅的紧!要起诗社,我自荐我掌坛。前儿春天我原有这个意思的。我想了一想,我又不会作诗,瞎乱些什么,因而也忘了,就没有说得。既是三妹妹高兴,我就帮你作兴起来。”\begin{note}庚辰双行夹批:看他又是一篇文字,分叙单传之法也。\end{note}
\end{parag}


\begin{parag}
    黛玉道:“既然定要起诗社,咱们都是诗翁了,先把这些姐妹叔嫂的字样改了才不俗。”\begin{note}庚辰双行夹批:看他写黛玉,真可人也。\end{note}李纨道:“极是,何不大家起个别号,彼此称呼则雅。\begin{note}庚辰双行夹批:未起诗社,先起别号。\end{note}我是定了‘稻香老农’,再无人占的。”\begin{note}庚辰双行夹批:最妙!一个花样。\end{note}探春笑道: “我就是‘秋爽居士’罢。”宝玉道:“居士,主人到底不恰,且又瘰赘。这里梧桐芭蕉尽有,或指梧桐芭蕉起个倒好。”探春笑道:“有了,我最喜芭蕉,就称 ‘蕉下客’罢。”众人都道别致有趣。黛玉笑道:“你们快牵了他去,炖了脯子吃酒。”众人不解。黛玉笑道:“古人曾云‘蕉叶覆鹿’。他自称‘蕉下客 ’,可不是一只鹿了?快做了鹿脯来。”众人听了都笑起来。探春因笑道:“你别忙中使巧话来骂人,我已替你想了个极当的美号了。”又向众人道:“当日娥皇女英洒泪在竹上成斑,故今斑竹又名湘妃竹。如今他住的是潇湘馆,他又爱哭,将来他想林姐夫,那些竹子也是要变成斑竹的。以后都叫他作‘潇湘妃子’就完了。” 大家听说,都拍手叫妙。林黛玉低了头方不言语。\begin{note}庚辰双行夹批:妙极趣极!所谓“夫人必自侮然后人侮之”,看因一谑便勾出一美号来,何等妙文哉!另一花样。\end{note}李纨笑道:“我替薛大妹妹也早已想了个好的,也只三个字。”惜春迎春都问是什么。\begin{note}庚辰双行夹批:妙文!迎春惜春固不能答言,然不便撕之不叙,故插他二人问。试思近日诸豪宴集雄语伟辩之时,座上或有一二愚夫不敢接谈,然偏好问,亦真可厌之事。\end{note}李纨道:“我是封他‘蘅芜君’了,不知你们如何。”探春笑道:“这个封号极好。”宝玉道:“我呢?你们也替我想一个。”\begin{note}庚辰双行夹批:必有是问。\end{note}宝钗笑道:“你的号早有了,‘无事忙’三字恰当的很。”\begin{note}庚辰双行夹批:真恰当,形容得尽。\end{note}李纨道:“你还是你的旧号‘绛洞花王’就好。”\begin{note}庚辰双行夹批:妙极!又点前文。通部中从头至末,前文已过者恐去之冷落,使人忘怀,得便一点。未来者恐来之突然,或先伏一线。皆行文之妙诀也。\end{note}宝玉笑道:“小时候干的营生,还提他作什么。”\begin{note}庚辰双行夹批:赧言如闻,不知大时又有何营生。\end{note}探春道:“你的号多的很,又起什么。我们爱叫你什么,你就答应著就是了。”\begin{note}庚辰双行夹批:更妙!若只管挨次一个一个乱起,则成何文字?另一花样。\end{note}宝钗道:“还得我送你个号罢。有最俗的一个号,却于你最当。天下难得的是富贵,又难得的是闲散,这两样再不能兼有,不想你兼有了,就叫你 ‘富贵闲人’也罢了。”宝玉笑道:“当不起,当不起,倒是随你们混叫去罢。”李纨道:“二姑娘四姑娘起个什么号?”迎春道:“我们又不大会诗,白起个号作什么?”\begin{note}庚辰双行夹批:假斯文、守钱虏来看这句。\end{note}探春道:“虽如此,也起个才是。”宝钗道:“他住的是紫菱洲,就叫他‘菱洲’;四丫头在藕香榭,就叫他‘藕榭’就完了。”
\end{parag}


\begin{parag}
    李纨道:“就是这样好。但序齿我大,你们都要依我的主意,管情说了大家合意。我们七个人起社,我和二姑娘四姑娘都不会作诗,须得让出我们三个人去。我们三个各分一件事。”探春笑道:“已有了号,还只管这样称呼,不如不有了。以后错了,也要立个罚约才好。”李纨道:“立定了社,再定罚约。我那里地方大,竟在我那里作社。我虽不能作诗,这些诗人竟不厌俗客,我作个东道主人,我自然也清雅起来了。若是要推我作社长,我一个社长自然不够,必要再请两位副社长,就请菱洲藕榭二位学究来,一位出题限韵,一位誊录监场。亦不可拘定了我们三个人不作,若遇见容易些的题目韵脚,我们也随便作一首。你们四个却是要限定的。若如此便起,若不依我,我也不敢附骥了。”迎春惜春本性懒于诗词,又有薛林在前,听了这话便深合己意,二人皆说:“极是。”探春等也知此意,见他二人悦服,也不好强,只得依了。因笑道:“这话也罢了,只是自想好笑,好好的我起了个主意,反叫你们三个来管起我来了。”宝玉道:“既这样,咱们就往稻香村去。”李纨道:“都是你忙,今日不过商议了,等我再请。”宝钗道:“也要议定几日一会才好。”探春道:“若只管会的多,又没趣了。一月之中,只可两三次才好。”宝钗点头道:“一月只要两次就够了。拟定日期,风雨无阻。除这两日外,倘有高兴的,他情愿加一社的,或情愿到他那里去,或附就了来,亦可使得,岂不活泼有趣。”众人都道:“这个主意更好。”
\end{parag}


\begin{parag}
    探春道:“只是原系我起的意,我须得先作个东道主人,方不负我这兴。”李纨道:“既这样说,明日你就先开一社如何?”探春道:“明日不如今日,此刻就很好。你就出题,菱洲限韵,藕榭监场。”迎春道:“依我说,也不必随一人出题限韵,竟是拈阄公道。”李纨道:“方才我来时,看见他们抬进两盆白海棠来,倒是好花。你们何不就咏起他来?”\begin{note}庚辰双行夹批:真正好题。妙在未起诗社先得了题目。\end{note}迎春道:“都还未赏,先倒作诗。”宝钗道:“不过是白海棠,又何必定要见了才作。古人的诗赋,也不过都是寄兴写情耳。若都是等见了作,如今也没这些诗了。”\begin{note}辰夹批:真诗人语。\end{note}迎春道:“既如此,待我限韵。”说著,走到书架前抽出一本诗来,随手一揭,这首竟是一首七言律,递与众人看了,都该作七言律。迎春掩了诗,又向一个小丫头道:“你随口说一个字来。”那丫头正倚门立著,便说了个“门”字。迎春笑道:“就是门字韵,‘十三元’了。头一个韵定要这‘门’字。”说著,又要了韵牌匣子过来,抽出“十三元”一屉,又命那小丫头随手拿四块。那丫头便拿了“盆”“魂”“痕”“昏”四块来。宝玉道:“这‘盆’‘门’两个字不大好作呢!”
\end{parag}


\begin{parag}
    侍书一样预备下四份纸笔,便都悄然各自思索起来。独黛玉或抚梧桐,或看秋色,或又和丫鬟们嘲笑。\begin{note}庚辰双行夹批:看他单写黛玉。\end{note}迎春又令丫鬟炷了一支“梦甜香”。原来这“梦甜香”只有三寸来长,有灯草粗细,以其易烬,故以此烬为限,如香烬未成便要罚。\begin{note}庚辰双行夹批:好香!专能撰此新奇字样。\end{note}一时探春便先有了,自提笔写出,又改抹了一回,递与迎春。因问宝钗:“蘅芜君,你可有了?”宝钗道:“有却有了,只是不好。”宝玉背著手,在回廊上踱来踱去,因向黛玉说道:“你听,他们都有了。”黛玉道:“你别管我。”宝玉又见宝钗已誊写出来,因说道:“了不得!香只剩了一寸了,我才有了四句。”又向黛玉道: “香就完了,只管蹲在那潮地下作什么?”黛玉也不理。宝玉道:“可顾不得你了,好歹也写出来罢。”说著也走在案前写了。李纨道:“我们要看诗了,若看完了还不交卷是必罚的。”宝玉道:“稻香老农虽不善作却善看,又最公道,\begin{note}庚辰双行夹批:理岂不公。\end{note}你就评阅优劣,我们都服的。”众人都道:“自然。”于是先看探春的稿上写道是:
\end{parag}
\begin{poem}

    \begin{pl}

        咏白海棠限门盆魂痕昏
    \end{pl}
    \begin{pl}

        斜阳寒草带重门,苔翠盈铺雨后盆。
    \end{pl}
    \begin{pl}

        玉是精神难比洁,雪为肌骨易消魂。
    \end{pl}
    \begin{pl}

        芳心一点娇无力,倩影三更月有痕。
    \end{pl}
    \begin{pl}

        莫谓缟仙能羽化,多情伴我咏黄昏。
    \end{pl}

\end{poem}


\begin{parag}
    次看宝钗的是:
\end{parag}


\begin{poem}
    \begin{pl}珍重芳姿昼掩门,\end{pl}
    \begin{note}蒙双行夹批:宝钗诗全是自写身份,讽刺时事。只以品行为先,才技为末。……最恨近日小说中一百美人诗词语气只得一个艳稿。\end{note}

    \begin{pl}自携手瓮灌苔盆。\end{pl}

    \begin{pl}胭脂洗出秋阶影,\end{pl}

    \begin{pl}氷雪招来露砌魂。\end{pl}
    \begin{note}庚辰双行夹批:看他清洁自厉,终不肯作一轻浮语。\end{note}
    \begin{pl}淡极始知花更艳,\end{pl}
    \begin{note}庚辰双行夹批:好极!高情巨眼能几人哉!正“鸟鸣山更幽”也。\end{note}

    \begin{pl}愁多焉得玉无痕。\end{pl}
    \begin{note}庚辰双行夹批:看他讽刺林宝二人著手。\end{note}

    \begin{pl}欲偿白帝凭清洁,\end{pl}
    \begin{note}庚辰双行夹批:看他收到自己身上来,是何等身份。\end{note}

    \begin{pl}不语婷婷日又昏。    \end{pl}
\end{poem}


\begin{parag}
    李纨笑道:“到底是蘅芜君。”说著又看宝玉的,道是:
\end{parag}


\begin{poem}
    \begin{pl}秋容浅淡映重门,七节攒成雪满盆。\end{pl}

    \begin{pl}出浴太真氷作影,捧心西子玉为魂。\end{pl}
    \begin{pl}晓风不散愁千点,\end{pl}
    \begin{note}庚辰双行夹批:这句直是自己一生心事。\end{note}\begin{pl}宿雨还添泪一痕。\end{pl}\begin{note}庚辰双行夹批:妙在终不忘黛玉。\end{note}

    \begin{pl}独倚画栏如有意,清砧怨笛送黄昏。\end{pl}
    \begin{note}庚辰双行夹批:宝玉再细心作,只怕还有好的。只是一心挂著黛玉,故手妥不警也。\end{note}
\end{poem}


\begin{parag}
    大家看了,宝玉说探春的好,李纨才要推宝钗这诗有身分,因又催黛玉。黛玉道:“你们都有了。”说著提笔一挥而就,掷与众人。李纨等看他写道是:
\end{parag}


\begin{poem}
    \begin{pl}半卷湘帘半掩门,\end{pl}\begin{note}庚辰双行夹批:且不说花,且说看花的人,起得突然别致。\end{note}

    \begin{pl}碾氷为土玉为盆。\end{pl}\begin{note}庚辰双行夹批:妙极!料定他自与别人不同。\end{note}
\end{poem}


\begin{parag}
    看了这句,宝玉先喝起彩来,只说“从何处想来!”又看下面道:
\end{parag}


\begin{poem}
    \begin{pl}
        偷来梨蕊三分白,借得梅花一缕魂。
    \end{pl}
\end{poem}


\begin{parag}
    众人看了也都不禁叫好,说“果然比别人又是一样心肠。”又看下面道是:
\end{parag}


\begin{poem}
    \begin{pl}
        月窟仙人缝缟袂,秋闺怨女拭啼痕。\end{pl}\begin{note}庚辰双行夹批:虚敲旁比,真逸才也。且不脱落自己。\end{note}

    \begin{pl}
        娇羞默默同谁诉,倦倚西风夜已昏。\end{pl}\begin{note}庚辰双行夹批:看他终结道自己,一人是一人口气。逸才仙品固让颦儿,温雅沉著终是宝钗。今日之作宝玉自应居末。\end{note}
\end{poem}


\begin{parag}
    众人看了,都道是这首为上。李纨道:“若论风流别致,自是这首;若论含蓄浑厚,终让蘅稿。”探春道:“这评的有理,潇湘妃子当居第二。”李纨道:“怡红公子是压尾,你服不服?”宝玉道:“我的那首原不好了,这评的最公。”\begin{note}庚辰双行夹批:话内细思则似有不服先评之意。\end{note}又笑道:“只是蘅潇二首还要斟酌。” 李纨道:“原是依我评论,不与你们相干,再有多说者必罚。”宝玉听说,只得罢了。李纨道:“从此后我定于每月初二、十六这两日开社,出题限韵都要依我。这其间你们有高兴的,你们只管另择日子补开,那怕一个月每天都开社,我只不管。只是到了初二、十六这两日,是必往我那里去。”宝玉道:“到底要起个社名才是。”探春道:“俗了又不好,特新了,刁钻古怪也不好。可巧才是海棠诗开端,就叫个海棠社罢。虽然俗些,因真有此事,也就不碍了。”说毕大家又商议了一回,略用些酒果,方各自散去。也有回家的,也有往贾母王夫人处去的。当下别人无话。\begin{note}庚辰双行夹批:一路总不大写薛林兴头,可见他二人并不著意于此。不写薛林正是大手笔,独他二人长于诗,必使他二人为之则板腐矣。全是错综法。\end{note}
\end{parag}


\begin{parag}
    且说袭人\begin{note}庚辰双行夹批:忽然写到袭人,真令人不解。看他如何终此诗社之文。\end{note}因见宝玉看了字贴儿便慌慌张张的同翠墨去了,也不知是何事。后来又见后门上婆子送了两盆海棠花来。袭人问是那里来的,婆子便将宝玉前一番缘故说了。袭人听说便命他们摆好,让他们在下房里坐了,自己走到自己房内秤了六钱银子封好,又拿了三百钱走来,都递与那两个婆子道:“这银子赏那抬花来的小子们,这钱你们打酒吃罢。”那婆子们站起来,眉开眼笑,千恩万谢的不肯受,见袭人执意不收,方领了。袭人又道:“后门上外头可有该班的小子们?”婆子忙应道:“天天有四个,原预备里面差使的。姑娘有什么差使,我们吩咐去。”袭人笑道:“有什么差使?今儿宝二爷要打发人到小侯爷家与史大姑娘送东西去,可巧你们来了,顺便出去叫后门小子们雇辆车来。回来你们就往这里拿钱,不用叫他们又往前头混碰去。”婆子答应著去了。
\end{parag}


\begin{parag}
    袭人回至房中,拿碟子盛东西与史湘云送去,\begin{note}庚辰双行夹批:线头却牵出,观者犹不理。不知是何碟何物,令人犯思度。\end{note}却见槅子上碟槽空著。\begin{note}庚辰双行夹批:妙极细极!因此处系依古董式样抠成槽子,故无此件此槽遂空。若忘却前文,此句不解。\end{note}因回头见晴雯、秋纹、麝月等都在一处做针黹,袭人问道:“这一个缠丝白玛瑙碟子那去了?”众人见问,都你看我我看你,都想不起来。半日,晴雯笑道:“给三姑娘送荔枝去的,还没送来呢。”袭人道:“家常送东西的家伙也多,巴巴的拿这个去。”晴雯道:“我何尝不也这样说。他说这个碟子配上鲜荔枝才好看。\begin{note}庚辰双行夹批:自然好看,原该如此。可恨今之有一二好花者不背像景而用。\end{note}我送去,三姑娘见了也说好看,叫连碟子放著,就没带来。你再瞧,那槅子尽上头的一对联珠瓶还没收来呢。”秋纹笑道:“提起瓶来,我又想起笑话。我们宝二爷说声孝心一动,也孝敬到二十分。因那日见园里桂花,折了两枝,原是自己要插瓶的,忽然想起来说,这是自己园里的才开的新鲜花,不敢自己先顽,巴巴的把那一对瓶拿下来,亲自灌水插好了,叫个人拿著,亲自送一瓶进老太太,又进一瓶与太太。谁知他孝心一动,连跟的人都得了福了。可巧那日是我拿去的。老太太见了这样,喜的无可无不可,见人就说:‘到底是宝玉孝顺我,连一枝花儿也想的到。别人还只抱怨我疼他。’你们知道,老太太素日不大同我说话的,有些不入他老人家的眼的。那日竟叫人拿几百钱给我,说我可怜见的,生的单柔。这可是再想不到的福气。几百钱是小事,难得这个脸面。及至到了太太那里,太太正和二奶奶、赵姨奶奶、周姨奶奶好些人翻箱子,找太太当日年轻的颜色衣裳,不知给那一个。一见了,连衣裳也不找了,且看花儿。又有二奶奶在旁边凑趣儿,夸宝玉又是怎么孝敬,又是怎样知好歹,有的没的说了两车话。当著众人,太太自为又增了光,堵了众人的嘴。太太越发喜欢了,现成的衣裳就赏了我两件。衣裳也是小事,年年横竖也得,却不象这个彩头。”晴雯笑道:“呸!没见世面的小蹄子!那是把好的给了人,挑剩下的才给你,你还充有脸呢。”秋纹道:“凭他给谁剩的,到底是太太的恩典。”晴雯道:“要是我,我就不要。若是给别人剩下的给我,也罢了。一样这屋里的人,难道谁又比谁高贵些?把好的给他,剩下的才给我,我宁可不要,冲撞了太太,我也不受这口软气。”秋纹忙问:“给这屋里谁的?我因为前儿病了几天,家去了,不知是给谁的。好姐姐,你告诉我知道知道。”晴雯道:“我告诉了你,难道你这会退还太太去不成?”秋纹笑道:“胡说。我白听了喜欢喜欢。那怕给这屋里的狗剩下的,我只领太太的恩典,也不犯管别的事。”众人听了都笑道:“骂的巧,可不是给了那西洋花点子哈巴儿了。”袭人笑道:“你们这起烂了嘴的!得了空就拿我取笑打牙儿。一个个不知怎么死呢。”秋纹笑道:“原来姐姐得了,我实在不知道。我陪个不是罢。”袭人笑道:“少轻狂罢。你们谁取了碟子来是正经。”\begin{note}庚辰双行夹批:看他忽然夹写女儿喁喁一段,总不脱落正事。所谓此书一回是两段,两段中却有无限事体,或有一语透至一回者,或有反补上回者,错综穿插,从不一气直起直泻至终为了。\end{note}麝月道:“那瓶得空儿也该收来了。老太太屋里还罢了,太太屋里人多手杂。别人还可以,赵姨奶奶一伙的人见是这屋里的东西,又该使黑心弄坏了才罢。太太也不大管这些,不如早些收来正经。”晴雯听说,便掷下针黹道:“这话倒是,等我取去。”秋纹道:“还是我取去罢,你取你的碟子去。”晴雯笑道:“我偏取一遭儿去。是巧宗儿你们都得了,难道不许我得一遭儿?”麝月笑道:“通共秋丫头得了一遭儿衣裳,那里今儿又巧,你也遇见找衣裳不成。”晴雯冷笑道:“虽然碰不见衣裳,或者太太看见我勤谨,一个月也把太太的公费里分出二两银子来给我,也定不得。”说著,又笑道:“你们别和我装神弄鬼的,什么事我不知道。”一面说,一面往外跑了。秋纹也同他出来,自去探春那里取了碟子来。
\end{parag}


\begin{parag}
    袭人打点齐备东西,叫过本处的一个老宋妈妈来,\begin{note}庚辰双行夹批:“宋”,送也。随事生文,妙!\end{note}向他说道:“你先好生梳洗了,换了出门的衣裳来,如今打发你与史姑娘送东西去。”那嬷嬷道:“姑娘只管交给我,有话说与我,我收拾了就好一顺去的。”袭人听说,便端过两个小掐丝盒子来。先揭开一个,里面装的是红菱和鸡头两样鲜果;又那一个,是一碟子桂花糖蒸新栗粉糕。又说道:“这都是今年咱们这里园里新结的果子,宝二爷送来与姑娘尝尝。再前日姑娘说这玛瑙碟子好,姑娘就留下顽罢。\begin{note}庚辰双行夹批:妙!隐这一件公案。余想袭人必要玛瑙碟子盛去,何必娇奢轻□如是耶?固有此一案,则无怪矣。\end{note}这绢包儿里头是姑娘上日叫我作的活计,姑娘别嫌粗糙,能著用罢。替我们请安,替二爷问好就是了。”宋嬷嬷道:“宝二爷不知还有什么说的,姑娘再问问去,回来又别说忘了。”袭人因问秋纹:“方才可见在三姑娘那里?”秋纹道:“他们都在那里商议起什么诗社呢,又都作诗。想来没话,你只去罢。” 嬷嬷听了,便拿了东西出去,另外穿戴了。袭人又嘱咐他:“从后门出去,有小子和车等著呢。”宋妈去后,不在话下。
\end{parag}


\begin{parag}
    宝玉回来,先忙著看了一回海棠,至房内告诉袭人起诗社的事。袭人也把打发宋妈妈与史湘云送东西去的话告诉了宝玉。宝玉听了,拍手道:“偏忘了他。我自觉心里有件事,只是想不起来,亏你提起来,正要请他去。这诗社里若少了他还有什么意思。”袭人劝道:“什么要紧,不过玩意儿。他比不得你们自在,家里又作不得主儿。告诉他,他要来又由不得他;不来,他又牵肠挂肚的,没的叫他不受用。”宝玉道:“不妨事,我回老太太打发人接他去。”正说著,宋妈妈已经回来,回复道生受,与袭人道乏,又说:“问二爷作什么呢,我说和姑娘们起什么诗社作诗呢。史姑娘说,他们作诗也不告诉他去,急的了不的。”宝玉听了立身便往贾母处来,立逼著叫人接去。贾母因说:“今儿天晚了,明日一早再去。”宝玉只得罢了,回来闷闷的。
\end{parag}


\begin{parag}
    次日一早,便又往贾母处来催逼人接去。直到午后,史湘云才来,宝玉方放了心,见面时就把始末原由告诉他,又要与他诗看。李纨等因说道:“且别给他诗看,先说与他韵。他后来,先罚他和了诗:若好,便请入社;若不好,还要罚他一个东道再说。”史湘云道:“你们忘了请我,我还要罚你们呢。就拿韵来,我虽不能,只得勉强出丑。容我入社,扫地焚香我也情愿。”众人见他这般有趣,越发喜欢,都埋怨昨日怎么忘了他,遂忙告诉他韵。史湘云一心兴头,等不得推敲删改,一面只管和人说著话,心内早已和成,即用随便的纸笔录出,\begin{note}庚辰双行夹批:可见定是好文字,不管怎样就有了。越用工夫越讲完笔墨终成涂雅。\end{note}先笑说道: “我却依韵和了两首,\begin{note}庚辰双行夹批:更奇!想前四律已将形容尽矣,一首犹恐重犯,不知二首又从何处著笔。\end{note}好歹我却不知,不过应命而已。”说著递与众人。众人道:“我们四首也算想绝了,再一首也不能了。你倒弄了两首,那里有许多话说,必要重了我们。”一面说,一面看时,只见那两首诗写道:
\end{parag}
\begin{poem}
    \begin{pl}神仙昨日降都门,   \end{pl}
    \begin{note}庚辰双行夹批:落想便新奇,不落彼四套。\end{note}
    \begin{pl}种得蓝田玉一盆。   \end{pl}
    \begin{note}庚辰双行夹批:好!“盆”字押得更稳,不落彼四套。\end{note}
    \begin{pl}自是霜娥偏爱冷,    \end{pl}
    \begin{note}庚辰双行夹批:又不脱自己将来形景。\end{note}
    \begin{pl} 非关倩女亦离魂。     \end{pl}

    \begin{pl}秋阴捧出何方雪,\end{pl}
    \begin{note}庚辰双行夹批:拍案叫绝!压倒群芳在此一句。\end{note}
    \begin{pl}雨渍添来隔宿痕。\end{pl}

    \begin{pl}却喜诗人吟不倦,岂令寂寞度朝昏。    \end{pl}\begin{note}庚辰双行夹批:真好!\end{note}

    \begin{pl}
        蘅芷阶通萝薜门,也宜墙角也宜盆。    \end{pl}\begin{note}庚辰双行夹批:更好!\end{note}

    \begin{pl}
        花因喜洁难寻偶,人为题秋易断魂。
    \end{pl}
    \begin{pl}
        玉烛滴干风里泪,晶帘隔破月中痕。
    \end{pl}
    \begin{pl}幽情欲向嫦娥诉,无奈虚廊夜色昏。\end{pl}
    \begin{note}庚辰双行夹批:二首真可压卷。诗是好诗,文是奇奇怪怪之文,总令人想不到忽有二首来压卷。\end{note}
\end{poem}


\begin{parag}
    众人看一句,惊讶一句,看到了,赞到了,都说:“这个不枉作了海棠诗,真该要起海棠社了。”史湘云道:“明日先罚我个东道,就让我先邀一社可使得?” 众人道:“这更妙了。”因又将昨日的与他评论了一回。\begin{note}该批:观湘云作海棠诗,如见其娇憨之态。是乃实有,非作书者杜撰也。\end{note}
\end{parag}


\begin{parag}
    至晚,宝钗将湘云邀往蘅芜苑安歇去。湘云灯下计议如何设东拟题。宝钗听他说了半日,皆不妥当,\begin{note}庚辰双行夹批:却于此刻方写宝钗。\end{note}因向他说道:“既开社,便要作东。虽然是顽意儿,也要瞻前顾后,又要自己便宜,又要不得罪了人,然后方大家有趣。你家里你又作不得主,一个月通共那几串钱,你还不够盘缠呢。这会子又干这没要紧的事,你婶子听见了,越发抱怨你了。况且你就都拿出来,做这个东道也是不够。难道为这个家去要不成?还是往这里要呢?”一席话提醒了湘云,倒踌蹰起来。宝钗道:“这个我已经有个主意。我们当铺里有个伙计,他家田上出的很好的肥螃蟹,前儿送了几斤来。现在这里的人,从老太太起连上园里的人,有多一半都是爱吃螃蟹的。前日姨娘还说要请老太太在园里赏桂花吃螃蟹,因为有事还没有请呢。你如今且把诗社别提起,只管普通一请。等他们散了,咱们有多少诗作不得的。我和我哥哥说,要几篓极肥极大的螃蟹来,再往铺子里取上几坛好酒,再备上四五桌果碟,岂不又省事又大家热闹了。”湘云听了,心中自是感服,极赞他想的周到。宝钗又笑道:“我是一片真心为你的话。你千万别多心,想著我小看了你,咱们两个就白好了。你若不多心,我就好叫他们办去的。”湘云忙笑道:“好姐姐,你这样说,倒多心待我了。凭他怎么糊涂,连个好歹也不知,还成个人了?我若不把姐姐当亲姐姐一样看,上回那些家常话烦难事也不肯尽情告诉你了。”宝钗听说,便叫一个婆子来:“出去和大爷说,依前日的大螃蟹要几篓来,明日饭后请老太太姨娘赏桂花。你说大爷好歹别忘了,我今儿已请下人了。”\begin{note}庚辰双行夹批:必得如此叮咛,阿呆兄方记得。\end{note}那婆子出去说明,回来无话。
\end{parag}


\begin{parag}
    这里宝钗又向湘云道:“诗题也不要过于新巧了。你看古人诗中那些刁钻古怪的题目和那极险的韵了,若题过于新巧,韵过于险,再不得有好诗,终是小家气。诗固然怕说熟话,更不可过于求生,只要头一件立意清新,自然措词就不俗了。究竟这也算不得什么,还是纺绩针黹是你我的本等。一时闲了,倒是于你我深有益的书看几章是正经。”湘云只答应著,因笑道:“我如今心里想著,昨日作了海棠诗,我如今要作个菊花诗如何?”宝钗道:“菊花倒也合景,只是前人太多了。”湘云道:“我也是如此想著,恐怕落套。”宝钗想了一想,说道:“有了,如今以菊花为宾,以人为主,竟拟出几个题目来,都是两个字:一个虚字,一个实字,实字便用‘菊’字,虚字就用通用门的。如此又是咏菊,又是赋事,前人也没作过,也不能落套。赋景咏物两关著,又新鲜,又大方。”湘云笑道:“这却很好。只是不知用何等虚字才好。你先想一个我听听。”宝钗想了一想,笑道:“《菊梦》就好。”湘云笑道:“果然好。我也有一个,《菊影》可使得?”宝钗道:“也罢了。只是也有人作过,若题目多,这个也夹的上。我又有了一个。”湘云道:“快说出来。”宝钗道:“《问菊》如何?”湘云拍案叫妙,因接说道:“我也有了,《访菊》如何?”宝钗也赞有趣,因说道:“越性拟出十个来,写上再来。”说著,二人研墨蘸笔,湘云便写,宝钗便念,一时凑了十个。湘云看了一遍,又笑道:“十个还不成幅,越性凑成十二个便全了,也如人家的字画册页一样。”宝钗听说,又想了两个,一共凑成十二。又说道:“既这样,越性编出他个次序先后来。”湘云道:“如此更妙,竟弄成个菊谱了。”宝钗道:“起首是《忆菊》;忆之不得,故访,第二是《访菊》;访之既得,便种,第三是《种菊》;种既盛开,故相对而赏,第四是《对菊》;相对而兴有余,故折来供瓶为玩,第五是《供菊》;既供而不吟,亦觉菊无彩色,第六便是《咏菊》;既入词章,不可不供笔墨,第七便是《画菊》;既为菊如是碌碌,究竟不知菊有何妙处,不禁有所问,第八便是《问菊》;菊如解语,使人狂喜不禁,第九便是《簪菊》;如此人事虽尽,犹有菊之可咏者,《菊影》《菊梦》二首续在第十第十一;末卷便以《残菊》总收前题之盛。这便是三秋的妙景妙事都有了。”湘云依说将题录出,又看了一回,又问“该限何韵?”宝钗道:“我平生最不喜限韵的,分明有好诗,何苦为韵所缚。咱们别学那小家派,只出题不拘韵。原为大家偶得了好句取乐,并不为此而难人。”湘云道: “这话很是。这样大家的诗还进一层。但只咱们五个人,这十二个题目,难道每人作十二首不成?”宝钗道:“那也太难人了。将这题目誊好,都要七言律,明日贴在墙上。他们看了,谁作那一个就作那一个。有力量者,十二首都作也可;不能的,一首不成也可。高才捷足者为尊。若十二首已全,便不许他后赶著又作,罚他就完了。”湘云道:“这倒也罢了。”二人商议妥贴,方才息灯安寝。要知端的,且听下回分解。
\end{parag}


\begin{parag}
    \begin{note}蒙回末总批:薛家女子何贞侠,总因富贵不须夸。发言行事何其嘉,居心用意不狂奢。世人若可平心度,便解云钗两不暇。\end{note}
\end{parag}

