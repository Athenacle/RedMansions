\chap{八十五}{贾存周报升郎中任 薛文起复惹放流刑}



\begin{parag}
    话说赵姨娘正在屋里抱怨贾环,只听贾环在外间屋里发话道:“我不过弄倒了药吊子,洒了一点子药,那丫头子又没就死了,值的他也骂我,你也骂我,赖我心坏,把我往死里糟踏。等着我明儿还要那小丫头子的命呢,看你们怎么着!只叫他们提防着就是了。”那赵姨娘赶忙从里间出来,握住他的嘴说道:“你还只管信口胡吣,还叫人家先要了我的命呢!”娘儿两个吵了一回。赵姨娘听见凤姐的话,越想越气,也不着人来安慰凤姐一声儿。过了几天,巧姐儿也好了。因此两边结怨比从前更加一层了。
\end{parag}


\begin{parag}
    一日林之孝进来回道:“今日是北静郡王生日,请老爷的示下。”贾政吩咐道:“只按向年旧例办了,回大老爷知道,送去就是了。”林之孝答应了,自去办理。不一时,贾赦过来同贾政商议,带了贾珍,贾琏,宝玉去与北静王拜寿。别人还不理论,惟有宝玉素日仰慕北静王的容貌威仪,巴不得常见才好,遂连忙换了衣服,跟着来到北府。贾赦贾政递了职名候谕。不多时,里面出来了一个太监,手里掐着数珠儿,见了贾赦贾政,笑嘻嘻的说道:“二位老爷好?”贾赦贾政也都赶忙问好。他兄弟三人也过来问了好。那太监道:“王爷叫请进去呢。”于是爷儿五个跟着那太监进入府中,过了两层门,转过一层殿去,里面方是内宫门。刚到门前,大家站住,那太监先进去回王爷去了。这里门上小太监都迎着问了好。一时那太监出来,说了个“请”字,爷儿五个肃敬跟入。只见北静郡王穿着礼服,已迎到殿门廊下。贾赦贾政先上来请安,捱次便是珍,琏,宝玉请安。那北静郡王单拉着宝玉道:“我久不见你,很惦记你。”因又笑问道:“你那块玉儿好?”宝玉躬着身打着一半千儿回道:“蒙王爷福庇,都好。”北静王道:“今日你来,没有什么好东西给你吃的,倒是大家说说话儿罢。”说着,几个老公打起帘子,北静王说“请”,自己却先进去,然后贾赦等都躬着身跟进去。先是贾赦请北静王受礼,北静王也说了两句谦辞,那贾赦早已跪下,次及贾政等捱次行礼,自不必说。
\end{parag}


\begin{parag}
    那贾赦等复肃敬退出。北静王吩咐太监等让在众戚旧一处好生款待,却单留宝玉在这里说话儿,又赏了坐。宝玉又磕头谢了恩,在挨门边绣墩上侧坐,说了一回读书作文诸事。北静王甚加爱惜,又赏了茶,因说道:“昨儿巡抚吴大人来陛见,说起令尊翁前任学政时,秉公办事,凡属生童,俱心服之至。他陛见时,万岁爷也曾问过,他也十分保举,可知是令尊翁的喜兆。”宝玉连忙站起,听毕这一段话,才回启道:“此是王爷的恩典,吴大人的盛情。”正说着,小太监进来回道:“外面诸位大人老爷都在前殿谢王爷赏宴。”说着,呈上谢宴并请午安的帖子来。北静王略看了一看,仍递给小太监,笑了一笑说道:“知道了,劳动他们。”那小太监又回道:“这贾宝玉王爷单赏的饭预备了。”北静王便命那太监带了宝玉到一所极小巧精致的院里,派人陪着吃了饭,又过来谢了恩。北静王又说了些好话儿,忽然笑说道:“我前次见你那块玉倒有趣儿,回来说了个式样,叫他们也作了一块来。今日你来得正好,就给你带回去顽罢。”因命小太监取来,亲手递给宝玉。宝玉接过来捧着,又谢了,然后退出。北静王又命两个小太监跟出来,才同着贾赦等回来了。贾赦便各自回院里去。
\end{parag}


\begin{parag}
    这里贾政带着他三人回来见过贾母,请过了安,说了一回府里遇见的人。宝玉又回了贾政吴大人陛见保举的话。贾政道:“这吴大人本来咱们相好,也是我辈中人,还倒是有骨气的。”又说了几句闲话儿,贾母便叫”歇着去罢。”贾政退出,珍,琏,宝玉都跟到门口。贾政道:“你们都回去陪老太太坐着去罢。”说着,便回房去。刚坐了一坐,只见一个小丫头回道:“外面林之孝请老爷回话。”说着,递上个红单帖来,写着吴巡抚的名字。贾政知是来拜,便叫小丫头叫林之孝进来。贾政出至廊檐下。林之孝进来回道:“今日巡抚吴大人来拜,奴才回了去了。再奴才还听见说,现今工部出了一个郎中缺,外头人和部里都吵嚷是老爷拟正呢。”贾政道:“瞧罢咧。”林之孝又回了几句话,才出去了。
\end{parag}


\begin{parag}
    且说珍,琏,宝玉三人回去,独有宝玉到贾母那边,一面述说北静王待他的光景,并拿出那块玉来。大家看着笑了一回。贾母因命人:“给他收起去罢,别丢了。”因问:“你那块玉好生带着罢?别闹混了。”宝玉在项上摘了下来,说:“这不是我那一块玉,那里就掉了呢。比起来,两块玉差远着呢,那里混得过。我正要告诉老太太,前儿晚上我睡的时候把玉摘下来挂在帐子里,他竟放起光来了,满帐子都是红的。”贾母说道:“又胡说了,帐子的檐子是红的,火光照着,自然红是有的。”宝玉道:“不是。那时候灯已灭了,屋里都漆黑的了,还看得见他呢。”邢王二夫人抿着嘴笑。凤姐道:“这是喜信发动了。”宝玉道:“什么喜信?”贾母道:“你不懂得。今儿个闹了一天,你去歇歇儿去罢,别在这里说呆话了。”宝玉又站了一回儿,才回园中去了。
\end{parag}


\begin{parag}
    这里贾母问道:“正是。你们去看薛姨妈说起这事没有?”王夫人道:“本来就要去看的,因凤丫头为巧姐儿病着,耽搁了两天,今日才去的。这事我们都告诉了,姨妈倒也十分愿意,只说蟠儿这时侯不在家,目今他父亲没了,只得和他商量商量再办。”贾母道:“这也是情理的话。既这么样,大家先别提起,等姨太太那边商量定了再说。”不说贾母处谈论亲事,且说宝玉回到自己房中,告诉袭人道:“老太太与凤姐姐方才说话含含糊糊,不知是什么意思。”袭人想了想,笑了一笑道:“这个我也猜不着。但只刚才说这些话时,林姑娘在跟前没有?”宝玉道:“林姑娘才病起来,这些时何曾到老太太那边去呢。”正说着,只听外间屋里麝月与秋纹拌嘴。袭人道:“你两个又闹什么?”麝月道:“我们两个斗牌,他赢了我的钱他拿了去,他输了钱就不肯拿出来。这也罢了,他倒把我的钱都抢了去了。”宝玉笑道:“几个钱什么要紧,傻丫头,不许闹了。”说的两个人都咕嘟着嘴坐着去了。这里袭人打发宝玉睡下。不提。
\end{parag}


\begin{parag}
    却说袭人听了宝玉方才的话,也明知是给宝玉提亲的事。因恐宝玉每有痴想,这一提起不知又招出他多少呆话来,所以故作不知,自己心上却也是头一件关切的事。夜间躺着想了个主意,不如去见见紫鹃,看他有什么动静,自然就知道了。次日一早起来,打发宝玉上了学,自己梳洗了,便慢慢的去到潇湘馆来。只见紫鹃正在那里掐花儿呢,见袭人进来,便笑嘻嘻的道:“姐姐屋里坐着。”袭人道:“坐着,妹妹掐花儿呢吗?姑娘呢?”紫鹃道:“姑娘才梳洗完了,等着温药呢。”紫鹃一面说着,一面同袭人进来。见了黛玉正在那里拿着一本书看。袭人陪着笑道:“姑娘怨不得劳神,起来就看书。我们宝二爷念书若能象姑娘这样,岂不好了呢。”黛玉笑着把书放下。雪雁已拿着个小茶盘里托着一钟药,一钟水,小丫头在后面捧着痰盒漱盂进来。原来袭人来时要探探口气,坐了一回,无处入话,又想着黛玉最是心多,探不成消息再惹着了他倒是不好,又坐了坐,搭讪着辞了出来了。将到怡红院门口,只见两个人在那里站着呢。袭人不便往前走,那一个早看见了,连忙跑过来。袭人一看,却是锄药,因问”你作什么?”锄药道:“刚才芸二爷来了,拿了个帖儿,说给咱们宝二爷瞧的,在这里候信。”袭人道:“宝二爷天天上学,你难道不知道,还候什么信呢。”锄药笑道:“我告诉他了。他叫告诉姑娘,听姑娘的信呢。”袭人正要说话,只见那一个也慢慢的蹭了过来,细看时,就是贾芸,溜溜湫湫往这边来了。袭人见是贾芸,连忙向锄药道:“你告诉说知道了,回来给宝二爷瞧罢。”那贾芸原要过来和袭人说话,无非亲近之意,又不敢造次,只得慢慢踱来。相离不远,不想袭人说出这话,自己也不好再往前走,只好站住。这里袭人已掉背脸往回里去了。贾芸只得怏怏而回,同锄药出去了。
\end{parag}


\begin{parag}
    晚间宝玉回房,袭人便回道:“今日廊下小芸二爷来了。”宝玉道:“作什么?”袭人道:“他还有个帖儿呢。”宝玉道:“在那里?拿来我看看。”麝月便走去在里间屋里书槅子上头拿了来。宝玉接过看时,上面皮儿上写着“叔父大人安禀”。宝玉道:“这孩子怎么又不认我作父亲了?”袭人道:“怎么?”宝玉道:“前年他送我白海棠时称我作‘父亲大人’今日这帖子封皮上写着‘叔父’,可不是又不认了么。”袭人道:“他也不害臊,你也不害臊。他那么大了,倒认你这么大儿的作父亲,可不是他不害臊?你正经连个——”刚说到这里,脸一红,微微的一笑。宝玉也觉得了,便道:“这倒难讲。俗语说:‘和尚无儿,孝子多着呢。’只是我看着他还伶俐得人心儿,才这么着,他不愿意,我还不希罕呢。”说着,一面拆那帖儿,袭人也笑道:“那小芸二爷也有些鬼鬼头头的。什么时候又要看人,什么时侯又躲躲藏藏的,可知也是个心术不正的货。”宝玉只顾拆开看那字儿,也不理会袭人这些话。袭人见他看那帖儿,皱一回眉,又笑一笑儿,又摇摇头儿,后来光景竟大不耐烦起来。袭人等他看完了,问道:“是什么事情?”宝玉也不答言,把那帖子已经撕作几段,袭人见这般光景,也不便再问,便问宝玉吃了饭还看书不看。宝玉道:“可笑芸儿这孩子竟这样的混账。”袭人见他所答非所问,便微微的笑着问道:“到底是什么事?”宝玉道:“问他作什么,咱们吃饭罢。吃了饭歇着罢,心里闹的怪烦的。”说着叫小丫头子点了一个火儿来,把那撕的帖儿烧了。
\end{parag}


\begin{parag}
    一时小丫头们摆上饭来。宝玉只是怔怔的坐着,袭人连哄带怄催着吃了一口儿饭,便搁下了,仍是闷闷的歪在床上。一时间,忽然掉下泪来。此时袭人麝月都摸不着头脑。麝月道:“好好儿的,这又是为什么?都是什么芸儿雨儿的,不知什么事弄了这么个浪帖子来,惹的这么傻了的似的,哭一会子,笑一会子。要天长日久闹起这闷葫芦来,可叫人怎么受呢。”说着,竟伤起心来。袭人旁边由不得要笑,便劝道:“好妹妹,你也别怄人了。他一个人就够受了,你又这么着。他那帖子上的事难道与你相干?”麝月道:“你混说起来了。知道他帖儿上写的是什么混账话,你混往人身上扯。要那么说,他帖儿上只怕倒与你相干呢。”袭人还未答言,只听宝玉在床上噗哧的一声笑了,爬起来抖了抖衣裳,说:“咱们睡觉罢,别闹了。明日我还起早念书呢。”说着便躺下睡了。一宿无话。
\end{parag}


\begin{parag}
    次日宝玉起来梳洗了,便往家塾里去。走出院门,忽然想起,叫焙茗略等,急忙转身回来叫:“麝月姐姐呢?”麝月答应着出来问道:“怎么又回来了?”宝玉道:“今日芸儿要来了,告诉他别在这里闹,再闹我就回老太太和老爷去了。”麝月答应了,宝玉才转身去了。刚往外走着,只见贾芸慌慌张张往里来,看见宝玉连忙请安,说:“叔叔大喜了。”那宝玉估量着是昨日那件事,便说道:“你也太冒失了,不管人心里有事没事,只管来搅。”贾芸陪笑道:“叔叔不信只管瞧去,人都来了,在咱们大门口呢。”宝玉越发急了,说:“这是那里的话!”正说着,只听外边一片声嚷起来。贾芸道:“叔叔听这不是?”宝玉越发心里狐疑起来,只听一个人嚷道:“你们这些人好没规矩,这是什么地方,你们在这里混嚷。”那人答道:“谁叫老爷升了官呢,怎么不叫我们来吵喜呢。别人家盼着吵还不能呢。”宝玉听了,才知道是贾政升了郎中了,人来报喜的。心中自是甚喜。连忙要走时,贾芸赶着说道:“叔叔乐不乐?叔叔的亲事要再成了,不用说是两层喜了。”宝玉红了脸,啐了一口道:“呸!没趣儿的东西!还不快走呢。”贾芸把脸红了道:“这有什么的,我看你老人家就不——”宝玉沉着脸道:“就不什么?”贾芸未及说完,也不敢言语了。
\end{parag}


\begin{parag}
    宝玉连忙来到家塾中,只见代儒笑着说道:“我才刚听见你老爷升了。你今日还来了么?”宝玉陪笑道:“过来见了太爷,好到老爷那边去。”代儒道:“今日不必来了,放你一天假罢。可不许回园子里顽去。你年纪不小了,虽不能办事,也当跟着你大哥他们学学才是。”宝玉答应着回来。刚走到二门口,只见李贵走来迎着,旁边站住笑道:“二爷来了么,奴才才要到学里请去。”宝玉笑道:“谁说的?”李贵道:“老太太才打发人到院里去找二爷,那边的姑娘们说二爷学里去了。刚才老太太打发人出来叫奴才去给二爷告几天假,听说还要唱戏贺喜呢,二爷就来了。”说着,宝玉自己进去。进了二门,只见满院里丫头老婆都是笑容满面,见他来了,笑道:“二爷这早晚才来,还不快进去给老太太道喜去呢。”
\end{parag}


\begin{parag}
    宝玉笑着进了房门,只见黛玉挨着贾母左边坐着呢,右边是湘云。地下邢王二夫人。探春,惜春,李纨,凤姐,李纹,李绮,邢岫烟一干姐妹,都在屋里,只不见宝钗,宝琴,迎春三人。宝玉此时喜的无话可说,忙给贾母道了喜,又给邢王二夫人道喜,一一见了众姐妹,便向黛玉笑道:“妹妹身体可大好了?”黛玉也微笑道:“大好了。听见说二哥哥身上也欠安,好了么?”宝玉道:“可不是,我那日夜里忽然心里疼起来,这几天刚好些就上学去了,也没能过去看妹妹。”黛玉不等他说完,早扭过头和探春说话去了。凤姐在地下站着笑道:“你两个那里象天天在一处的,倒象是客一般,有这些套话,可是人说的‘相敬如宾’了。”说的大家一笑。林黛玉满脸飞红,又不好说,又不好不说,迟了一回儿,才说道:“你懂得什么?”众人越发笑了。凤姐一时回过味来,才知道自己出言冒失,正要拿话岔时,只见宝玉忽然向黛玉道:“林妹妹,你瞧芸儿这种冒失鬼。”说了一句,方想起来,便不言语了。招的大家又都笑起来,说:“这从那里说起。”黛玉也摸不着头脑,也跟着讪讪的笑。宝玉无可搭讪,因又说道:“可是刚才我听见有人要送戏,说是几儿?”大家都瞅着他笑。凤姐儿道:“你在外头听见,你来告诉我们。你这会子问谁呢?”宝玉得便说道:“我外头再去问问去。”贾母道:“别跑到外头去,头一件看报喜的笑话,第二件你老子今日大喜,回来碰见你,又该生气了。”宝玉答应了个“是”,才出来了。
\end{parag}


\begin{parag}
    这里贾母因问凤姐谁说送戏的话,凤姐道:“说是舅太爷那边说,后儿日子好,送一班新出的小戏儿给老太太,老爷,太太贺喜。”因又笑着说道:“不但日子好,还是好日子呢。”说着这话,却瞅着黛玉笑。黛玉也微笑。王夫人因道:“可是呢,后日还是外甥女儿的好日子呢。”贾母想了一想,也笑道:“可见我如今老了,什么事都糊涂了。亏了有我这凤丫头是我个‘给事中’。既这么着,很好,他舅舅家给他们贺喜,你舅舅家就给你做生日,岂不好呢。”说的大家都笑起来,说道:“老祖宗说句话儿都是上篇上论的,怎么怨得有这么大福气呢。”说着,宝玉进来,听见这些话,越发乐的手舞足蹈了。一时,大家都在贾母这边吃饭,甚热闹,自不必说。饭后,那贾政谢恩回来,给宗祠里磕了头,便来给贾母磕头,站着说了几句话,便出去拜客去了。这里接连着亲戚族中的人来来去去,闹闹穰穰,车马填门,貂蝉满座,真是:
\end{parag}


\begin{poem}
    \begin{pl}
        花到正开蜂蝶闹,月逢十足海天宽。
    \end{pl}
\end{poem}


\begin{parag}
    如此两日,已是庆贺之期。这日一早,王子腾和亲戚家已送过一班戏来,就在贾母正厅前搭起行台。外头爷们都穿着公服陪侍,亲戚来贺的约有十余桌酒。里面为着是新戏,又见贾母高兴,便将琉璃戏屏隔在后厦,里面也摆下酒席。上首薛姨妈一桌,是王夫人宝琴陪着,对面老太太一桌,是邢夫人岫烟陪着,下面尚空两桌,贾母叫他们快来,一回儿,只见凤姐领着众丫头,都簇拥着林黛玉来了。黛玉略换了几件新鲜衣服,打扮得宛如嫦娥下界,含羞带笑的出来见了众人。湘云,李纹,李纨都让他上首座,黛玉只是不肯。贾母笑道:“今日你坐了罢。”薛姨妈站起来问道:“今日林姑娘也有喜事么?”贾母笑道:“是他的生日。”薛姨妈道:“咳,我倒忘了。”走过来说道:“恕我健忘,回来叫宝琴过来拜姐姐的寿。”黛玉笑说“不敢”。大家坐了。那黛玉留神一看,独不见宝钗,便问道:“宝姐姐可好么?为什么不过来?”薛姨妈道:“他原该来的,只因无人看家,所以不来。”黛玉红着脸微笑道:“姨妈那里又添了大嫂子,怎么倒用宝姐姐看起家来?大约是他怕人多热闹,懒待来罢。我倒怪想他的。”薛姨妈笑道:“难得你惦记他。他也常想你们姊妹们,过一天我叫他来,大家叙叙。”
\end{parag}


\begin{parag}
    说着,丫头们下来斟酒上菜,外面已开戏了。出场自然是一两出吉庆戏文,乃至第三出,只见金童玉女,旗幡宝幢,引着一个霓裳羽衣的小旦,头上披着一条黑帕,唱了一回儿进去了。众皆不识,听见外面人说:“这是新打的《蕊珠记》里的《冥升》。小旦扮的是嫦娥,前因堕落人寰,几乎给人为配,幸亏观音点化,他就未嫁而逝,此时升引月宫。不听见曲里头唱的‘人间只道风情好,那知道秋月春花容易抛,几乎不把广寒宫忘却了!’”第四出是《吃糠》,第五出是达摩带着徒弟过江回去,正扮出些海市蜃楼,好不热闹。
\end{parag}


\begin{parag}
    众人正在高兴时,忽见薛家的人满头汗闯进来,向薛蝌说道:“二爷快回去,并里头回明太太也请速回去,家中有要事。”薛蝌道:“什么事?”家人道:“家去说罢。”薛蝌也不及告辞就走了。薛姨妈见里头丫头传进话去,更骇得面如土色,即忙起身,带着宝琴,别了一声,即刻上车回去了。弄得内外愕然。贾母道:“咱们这里打发人跟过去听听,到底是什么事,大家都关切的。”众人答应了个“是”。不说贾府依旧唱戏,单说薛姨妈回去,只见有两个衙役站在二门口,几个当铺里伙计陪着,说:“太太回来自有道理。”正说着,薛姨妈已进来了。那衙役们见跟从着许多男妇簇拥着一位老太太,便知是薛蟠之母。看见这个势派,也不敢怎么,只得垂手侍立,让薛姨妈进去了。
\end{parag}


\begin{parag}
    那薛姨妈走到厅房后面,早听见有人大哭,却是金桂。薛姨妈赶忙走来,只见宝钗迎出来,满面泪痕,见了薛姨妈,便道:“妈妈听了先别着急,办事要紧。”薛姨妈同着宝钗进了屋子,因为头里进门时已经走着听见家人说了,吓的战战兢兢的了,一面哭着,因问:“到底是和谁?”只见家人回道:“太太此时且不必问那些底细,凭他是谁,打死了总是要偿命的,且商量怎么办才好。”薛姨妈哭着出来道:“还有什么商议?”家人道:“依小的们的主见,今夜打点银两同着二爷赶去和大爷见了面,就在那里访一个有斟酌的刀笔先生,许他些银子,先把死罪撕掳开,回来再求贾府去上司衙门说情。还有外面的衙役,太太先拿出几两银子来打发了他们。我们好赶着办事。”薛姨妈道:“你们找着那家子,许他发送银子,再给他些养济银子,原告不追,事情就缓了。”宝钗在帘内说道:“妈妈,使不得。这些事越给钱越闹的凶,倒是刚才小厮说的话是。”薛姨妈又哭道:“我也不要命了,赶到那里见他一面,同他死在一处就完了。”宝钗急的一面劝,一面在帘子里叫人”快同二爷办去罢。”丫头们搀进薛姨妈来。薛蝌才往外走,宝钗道:“有什么信打发人即刻寄了来,你们只管在外头照料。”薛蝌答应着去了。这宝钗方劝薛姨妈,那里金桂趁空儿抓住香菱,又和他嚷道:“平常你们只管夸他们家里打死了人一点事也没有,就进京来了的,如今撺掇的真打死人了。平日里只讲有钱有势有好亲戚,这时侯我看着也是唬的慌手慌脚的了。大爷明儿有个好歹儿不能回来时,你们各自干你们的去了,撂下我一个人受罪!”说着,又大哭起来。这里薛姨妈听见,越发气的发昏。宝钗急的没法。正闹着,只见贾府中王夫人早打发大丫头过来打听来了。宝钗虽心知自己是贾府的人了,一则尚未提明,二则事急之时,只得向那大丫头道:“此时事情头尾尚未明白,就只听见说我哥哥在外头打死了人被县里拿了去了,也不知怎么定罪呢。刚才二爷才去打听去了,一半日得了准信,赶着就给那边太太送信去。你先回去道谢太太惦记着,底下我们还有多少仰仗那边爷们的地方呢。”那丫头答应着去了。薛姨妈和宝钗在家抓摸不着。
\end{parag}


\begin{parag}
    过了两日,只见小厮回来,拿了一封书交给小丫头拿进来。宝钗拆开看时,书内写着:
\end{parag}


\begin{qute2sp}
    大哥人命是误伤,不是故杀。今早用蝌出名补了一张呈纸进去,尚未批出。大哥前头口供甚是不好,待此纸批准后再录一堂,能够翻供得好,便可得生了。快向当铺内再取银五百两来使用。千万莫迟。并请太太放心。余事问小厮。
\end{qute2sp}


\begin{parag}
    宝钗看了,一一念给薛姨妈听了。薛姨妈拭着眼泪说道:“这么看起来,竟是死活不定了。”宝钗道:“妈妈先别伤心,等着叫进小厮来问明了再说。”一面打发小丫头把小厮叫进来。薛姨妈便问小厮道:“你把大爷的事细说与我听听。”小厮道:“我那一天晚上听见大爷和二爷说的,把我唬糊涂了。”未知小厮说出什么话来,下回分解。
\end{parag}