\chap{四十六}{尷尬人難免尷尬事 鴛鴦女誓絕鴛鴦偶}


\begin{parag}
    \begin{note}庚:此回亦有本而筆,非泛泛之筆也。\end{note}
\end{parag}


\begin{parag}
    \begin{note}庚:只看他題綱用“尷尬”二字於邢夫人,可知包藏含蓄文字之中莫能量也。\end{note}
\end{parag}


\begin{parag}
    \begin{note}蒙回前總:裹腳與纏頭,欲覓終身伴。顧影自爲憐,靜住深深院。好事不稱心,惡語將人慢。誓死守香閨,遠卻揚花片。\end{note}
\end{parag}


\begin{parag}
    話說林黛玉直到四更將闌,方漸漸的睡去,暫且無話。如今且說鳳姐兒因見邢夫人叫他,不知何事,忙另穿戴了一番,坐車過來。邢夫人將房內人遣出,悄向鳳姐兒道:“叫你來不爲別事,有一件爲難的事,老爺託我,我不得主意,先和你商議。老爺因看上了老太太的鴛鴦,要他在房裏,叫我和老太太討去。我想這倒平常有的事,只是怕老太太不給,你可有法子?”鳳姐兒聽了,忙道:“依我說,竟別碰這個釘子去。老太太離了鴛鴦,飯也喫不下去的,那裏就捨得了?況且平日說起閒話來,老太太常說,老爺如今上了年紀,作什麼左一個小老婆右一個小老婆放在屋裏,沒的耽誤了人家。放著身子不保養,官兒也不好生作去,成日家和小老婆喝酒。太太聽這話,很喜歡老爺呢?這會子迴避還恐迴避不及,倒拿草棍兒戳老虎的鼻子眼兒去了!太太別惱,我是不敢去的。明放著不中用,而且反招出沒意思來。老爺如今上了年紀,行事不妥,太太該勸纔是。比不得年輕,作這些事無礙。如今兄弟、侄兒、兒子、孫子一大羣,還這麼鬧起來,怎樣見人呢?”邢夫人冷笑道: “大家子三房四妾的也多,偏咱們就使不得?我勸了也未必依。就是老太太心愛的丫頭,這麼鬍子蒼白了又作了官的一個大兒子,要了作房裏人,也未必好駁回的。我叫了你來,不過商議商議,你先派上了一篇不是。也有叫你去的理?自然是我說去。你倒說我不勸,你還不知道那性子的,勸不成,先和我惱了。”
\end{parag}


\begin{parag}
    鳳姐兒知道邢夫人稟性愚犟(注:蒙本此作“拙”),只知承順賈赦以自保,次則婪取財貨爲自得,家下一應大小事務,俱由賈赦擺佈。凡出入銀錢事務,一經他手,便剋嗇異常,以賈赦浪費爲名,“須得我就中儉省,方可償補”,兒女奴僕,一人不靠,一言不聽的。如今又聽邢夫人如此的話,便知他又弄左性,勸了不中用,連忙陪笑說道:“太太這話說的極是。我能活了多大,知道什麼輕重?想來父母跟前,別說一個丫頭,就是那麼大的活寶貝,不給老爺給誰?背地裏的話那裏信得?我竟是個呆子。璉二爺或有日得了不是,老爺太太恨的那樣,恨不得立刻拿來一下子打死;及至見了面,也罷了,依舊拿著老爺太太心愛的東西賞他。如今老太太待老爺,自然也是那樣了。依我說,老太太今兒喜歡,要討今兒就討去。我先過去哄著老太太發笑,等太太過去了,我搭訕著走開,把屋子裏的人我也帶開,太太好和老太太說的。給了更好,不給也沒妨礙,衆人也不知道。”邢夫人見他這般說,便又喜歡起來,又告訴他道:“我的主意先不和老太太要。老太太要說不給,這事便死了。我心裏想著先悄悄的和鴛鴦說。他雖害臊,我細細的告訴了他,他自然不言語,就妥了。那時再和老太太說,老太太雖不依,擱不住他願意,常言‘人去不中留’,自然這就妥了。”鳳兒姐笑道:“到底是太太有智謀,這是千妥萬妥的。別說是鴛鴦,憑他是誰,那一個不想巴高望上,不想出頭的?這半個主子不做,倒願意做個丫頭,將來配個小子就完了。”邢夫人笑道:“正是這個話了。別說鴛鴦,就是那些執事的大丫頭,誰不願意這樣呢。你先過去,別露一點風聲,我吃了晚飯就過來。”
\end{parag}


\begin{parag}
    鳳姐兒暗想:“鴛鴦素習是個可惡的,雖如此說,保不嚴他就願意。我先過去了,太太后過去,若他依了便沒話說;倘或不依,太太是多疑的人,只怕就疑我走了風聲,使他拿腔作勢的。那時太太又見了應了我的話,羞惱變成怒,拿我出起氣來,倒沒意思。不如同著一齊過去了,他依也罷,不依也罷,就疑不到我身上了。”想畢,因笑道:“方纔臨來,舅母那邊送了兩籠子鵪鶉,我吩咐他們炸了,原要趕太太晚飯上送過來的。我才進大門時,見小子們抬車,說太太的車拔了縫,拿去收拾去了。不如這會子坐了我的車一齊過去倒好。”邢夫人聽了,便命人來換衣服。鳳姐忙著伏侍了一回,孃兒兩個坐車過來。鳳姐兒又說道:“太太過老太太那裏去,我若跟了去,老太太若問起我過去作什麼的,倒不好。不如太太先去,我脫了衣裳再來。”
\end{parag}


\begin{parag}
    邢夫人聽了有理,便自往賈母處,和賈母說了一回閒話,便出來假託往王夫人房裏去,從後門出去,打鴛鴦的臥房前過。只見鴛鴦正然坐在那裏做針線,見了邢夫人,忙站起來。邢夫人笑道:“做什麼呢?我瞧瞧,你扎的花兒越發好了。”一面說,一面便接他手內的針線瞧了一瞧,只管贊好。放下針線,又渾身打量。只見他穿著半新的藕合色的綾襖,青緞掐牙背心,下面水綠裙子。蜂腰削背,鴨蛋臉面,烏油頭髮,高高的鼻子,兩邊腮上微微的幾點雀斑。鴛鴦見這般看他,自己倒不好意思起來,心裏便覺詫異,因笑問道:“太太,這會子不早不晚的,過來做什麼?”邢夫人使個眼色兒,跟的人退出。邢夫人便坐下,拉著鴛鴦的手笑道:“我特來給你道喜來了。”鴛鴦聽了,心中已猜著三分,不覺紅了臉,低了頭不發一言。聽邢夫人道:“你知道你老爺跟前竟沒有個可靠的人,\begin{note}庚雙夾:說得得體。我正想開口一句不知如何說,如此則妙極是極,如聞如見。\end{note}心裏再要買一個,又怕那些人牙子家出來的不乾不淨,也不知道毛病兒,買了來家,三日兩日,又要肏鬼吊猴的。因滿府裏要挑一個家生女兒收了,又沒個好的:不是模樣兒不好,就是性子不好,有了這個好處,沒了那個好處。因此冷眼選了半年,這些女孩子裏頭,就只你是個尖兒,模樣兒,行事作人,溫柔可靠,一概是齊全的。意思要和老太太討了你去,收在屋裏。你比不得外頭新買的,你這一進去了,進門就開了臉,就封你姨娘,又體面,又尊貴。你又是個要強的人,俗語說的,‘金子終得金子換’,誰知竟被老爺看重了你。如今這一來,你可遂了素日誌大心高的願了,也堵一堵那些嫌你的人的嘴。跟了我回老太太去!”說著拉了他的手就要走。鴛鴦紅了臉,奪手不行。邢夫人知他害臊,因又說道:“這有什麼臊處?你又不用說話,只跟著我就是了。”鴛鴦只低了頭不動身。邢夫人見他這般,便又說道:“難道你不願意不成?若果然不願意,可真是個傻丫頭了。放著主子奶奶不作,倒願意作丫頭!三年二年,不過配上個小子,還是奴才。你跟了我們去,你知道我的性子又好,又不是那不容人的人。老爺待你們又好。過一年半載,生下個一男半女,你就和我並肩了。家裏的人你要使喚誰,誰還不動?現成主子不做去,錯過這個機會,後悔就遲了。”鴛鴦只管低了頭,仍是不語。邢夫人又道:“你這麼個響快人,怎麼又這樣積粘起來?有什麼不稱心之處,只管說與我,我管你遂心如意就是了。”鴛鴦仍不語。邢夫人又笑道:“想必你有老子娘,你自己不肯說話,怕臊。你等他們問你,這也是理。讓我問他們去,叫他們來問你,有話只管告訴他們。”說畢,便往鳳姐兒房中來。
\end{parag}


\begin{parag}
    鳳姐兒早換了衣服,因房內無人,便將此話告訴了平兒。平兒也搖頭笑道:“據我看,此事未必妥。平常我們背著人說起話來,聽他那主意,未必是肯的。也只說著瞧罷了。”鳳姐兒道:“太太必來這屋裏商議。依了還可,若不依,白討個臊,當著你們,豈不臉上不好看。你說給他們炸鵪鶉,再有什麼配幾樣,預備喫飯。你且別處逛逛去,估量著去了再來。”平兒聽說,照樣傳給婆子們,便逍遙自在的往園子裏來。
\end{parag}


\begin{parag}
    這裏鴛鴦見邢夫人去了,必在鳳姐兒房裏商議去了,必定有人來問他的,不如躲了這裏,\begin{note}庚雙夾:終不免女兒氣,不知躲在哪裏方無人來羅唣,寫得可憐可愛。\end{note}因找了琥珀說道:“老太太要問我,只說我病了,沒喫早飯,往園子裏逛逛就來。”琥珀答應了。鴛鴦也往園子裏來,各處遊玩,不想正遇見平兒。平兒因見無人,便笑道:“新姨娘來了!”鴛鴦聽了,便紅了臉,說道:“怪道你們串通一氣來算計我!等著我和你主子鬧去就是了。”平兒聽了,自悔失言,便拉他到楓樹底下,\begin{note}庚雙夾:隨筆帶出妙景,正愁園中草木黃落,不想看此一句便恍如置身於千霞萬錦絳雪紅霜之中矣。\end{note}坐在一塊石上,越性把方纔鳳姐過去回來所有的形景言詞始末原由告訴與他。鴛鴦紅了臉,向平兒冷笑道:“這是咱們好,比如襲人、琥珀、素雲、紫鵑、彩霞、玉釧兒、麝月、翠墨,跟了史姑娘去的翠縷,死了的可人和金釧,去了的茜雪,\begin{note}庚雙夾:餘按此一算,亦是十二釵,真鏡中花、水中月、雲中豹、林中之鳥、穴中之鼠、無數可考、無人可指、有跡可追、有形可據、九曲八折、遠響近影、迷離煙灼、縱橫隱現、千奇百怪、眩目移神、現千手千眼大遊戲法也。脂硯齋。\end{note}連上你我,這十來個人,從小兒什麼話兒不說?什麼事兒不作?這如今因都大了,各自幹各自的去了,\begin{note}庚雙夾:此語已可傷,猶未各自幹各自去,後日更有各自之處也,知之乎!\end{note}然我心裏仍是照舊,有話有事,並不瞞你們。這話我且放在你心裏,且別和二奶奶說:別說大老爺要我做小老婆,就是太太這會子死了,他三媒六聘的娶我去作大老婆,我也不能去。”
\end{parag}


\begin{parag}
    平兒方欲笑答,只聽山石背後哈哈的笑道:“好個沒臉的丫頭,虧你不怕牙磣。”二人聽了不免吃了一驚,忙起身向山石背後找尋,不是別個,卻是襲人笑著走了出來問:“什麼事情?告訴我。”說著,三人坐在石上。平兒又把方纔的話說與襲人聽道:“真真這話論理不該我們說,這個大老爺太好色了,略平頭正臉的,他就不放手了。”平兒道:“你既不願意,我教你個法子,不用費事就完了。”鴛鴦道:“什麼法子?你說來我聽。”平兒笑道:“你只和老太太說,就說已經給了璉二爺了,大老爺就不好要了。”鴛鴦啐道:“什麼東西!你還說呢!前兒你主子不是這麼混說的?誰知應到今兒了!”襲人笑道:“他們兩個都不願意,我就和老太太說,叫老太太說把你已經許了寶玉了,大老爺也就死了心了。”鴛鴦又是氣,又是臊,又是急,因罵道:“兩個蹄子不得好死的!人家有爲難的事,拿著你們當正經人,告訴你們與我排解排解,你們倒替換著取笑兒。你們自爲都有了結果了,將來都是做姨娘的。據我看,天下的事未必都遂心如意。你們且收著些兒,別忒樂過了頭兒!”二人見他急了,忙陪笑央告道:“好姐姐,別多心,咱們從小兒都是親姊妹一般,不過無人處偶然取個笑兒。你的主意告訴我們知道,也好放心。”鴛鴦道:“什麼主意!我只不去就完了。”平兒搖頭道:“你不去未必得干休。大老爺的性子你是知道的。雖然你是老太太房裏的人,此刻不敢把你怎麼樣,將來難道你跟老太太一輩子不成?也要出去的。那時落了他的手,倒不好了。”鴛鴦冷笑道:“老太太在一日,我一日不離這裏;若是老太太歸西去了,他橫豎還有三年的孝呢,沒個娘才死了他先納小老婆的!等過三年,知道又是怎麼個光景,那時再說。縱到了至急爲難,我剪了頭髮作姑子去;不然,還有一死。一輩子不嫁男人,又怎麼樣?樂得乾淨呢!”平兒襲人笑道:“真這蹄子沒了臉,越發信口兒都說出來了。”鴛鴦道:“事到如此,臊一會怎麼樣!你們不信,慢慢的看著就是了。太太才說了,找我老子娘去。我看他南京找去!”平兒道:“你的父母都在南京看房子,沒上來,終久也尋的著。現在還有你哥哥嫂子在這裏。可惜你是這裏的的家生女兒,不如我們兩個人是單在這裏。”鴛鴦道:“家生女兒怎麼樣?‘牛不喫水強按頭’?我不願意,難道殺我的老子娘不成?”
\end{parag}


\begin{parag}
    正說著,只見他嫂子從那邊走來。襲人道:“當時找不著你的爹孃,一定和你嫂子說了。”鴛鴦道:“這個娼婦專管是個‘九國販駱駝的’,聽了這話,他有個不奉承去的!”說話之間,已來到跟前。他嫂子笑道:“那裏沒找到,姑娘跑了這裏來!你跟了我來,我和你說話。”平兒襲人都忙讓坐。他嫂子說:“姑娘們請坐,我找我們姑娘說句話。” 襲人平兒都裝不知道,笑道:“什麼話這樣忙?我們這裏猜謎兒贏手批子打呢,等猜了這個再去。”鴛鴦道:“什麼話?你說罷。”他嫂子笑道:“你跟我來,到那裏我告訴你,橫豎有好話兒。”鴛鴦道:“可是大太太和你說的那話?”他嫂子笑道:“姑娘既知道,還奈何我!快來,我細細的告訴你可是天大的喜事。”鴛鴦聽說,立起身來,照他嫂子臉上下死勁啐了一口,指著他罵道:“你快夾著屄嘴離了這裏,好多著呢!什麼‘好話’!宋徽宗的鷹,趙子昂的馬,都是好畫兒。什麼 ‘喜事’!狀元痘兒灌的漿兒又滿是喜事。怪道成日家羨慕人家女兒作了小老婆了,一家子都仗著他橫行霸道的,一家子都成了小老婆了!看的眼熱了,也把我送在火坑裏去。我若得臉呢,你們外頭橫行霸道,自己就封自己是舅爺了。我若不得臉敗了時,你們把忘八脖子一縮,生死由我。”一面說,一面哭,平兒襲人攔著勸。他嫂子臉上下不來,因說道:“願意不願意,你也好說,不犯著牽三掛四的。俗語說,‘當著矮人,別說矮話’。姑奶奶罵我,我不敢還言;這二位姑娘並沒惹著你,小老婆長小老婆短,大家臉上怎麼過得去?”襲人平兒忙道:“你倒別這麼說,他也並不是說我們,你倒別牽三掛四的。你聽見那位太太、太爺們封我們做小老婆?況且我們兩個也沒有爹孃哥哥兄弟在這門子裏仗著我們橫行霸道的。他罵的人自有他罵的,我們犯不著多心。”鴛鴦道:“他見我罵了他,他臊了,沒的蓋臉,又拿話挑唆你們兩個,幸虧你們兩個明白。原是我急了,也沒分別出來,他就挑出這個空兒來。”他嫂子自覺沒趣,賭氣去了。
\end{parag}


\begin{parag}
    鴛鴦氣得還罵,平兒襲人勸他一回,方纔罷了。平兒因問襲人道:“你在那裏藏著做甚麼的?我們竟沒看見你。”襲人道:“我因爲往四姑娘房裏瞧我們寶二爺去的,誰知遲了一步,說是來家裏來了。我疑惑怎麼不遇見呢,想要往林姑娘家裏找去,又遇見他的人說也沒去。我這裏正疑惑是出園子去了,可巧你從那裏來了,我一閃,你也沒看見。後來他又來了。我從這樹後頭走到山子石後,我卻見你兩個說話來了,誰知你們四個眼睛沒見我。”
\end{parag}


\begin{parag}
    一語未了,又聽身後笑道:“四個眼睛沒見你?你們六個眼睛竟沒見我!”三人唬了一跳,回身一看,不是別個,正是寶玉走來。\begin{note}庚雙夾:通部情案皆必從石兄掛號,然各有各稿,穿插神妙。\end{note}襲人先笑道:“叫我好找,你那裏來?”寶玉笑道:“我從四妹妹那裏出來,迎頭看見你來了,我就知道是找我去的,我就藏了起來哄你。看你低著頭過去了,進了院子就出來了,逢人就問。我在那裏好笑,只等你到了跟前唬你一跳的,後來見你也藏藏躲躲的,我就知道也是要哄人了。我探頭往前看了一看,卻是他兩個,所以我就繞到你身後。你出去,我就躲在你躲的那裏了。”平兒笑道:“咱們再往後找找去,只怕還找出兩個人來也未可知。”寶玉笑道:“這可再沒了。”鴛鴦已知話俱被寶玉聽了,只伏在石頭上裝睡。寶玉推他笑道:“這石頭上冷,咱們回房裏去睡,豈不好?”說著拉起鴛鴦來,又忙讓平兒來家坐喫茶。平兒和襲人都勸鴛鴦走,鴛鴦方立起身來,四人竟往怡紅院來。寶玉將方纔的話俱已聽見,心中自然不快,只默默的歪在牀上,任他三人在外間說笑。
\end{parag}


\begin{parag}
    那邊邢夫人因問鳳姐兒鴛鴦的父母,鳳姐因回說:“他爹的名字叫金彩,\begin{note}庚雙夾:姓金名彩,由“鴛鴦”二字化出,因文而生文也。\end{note}兩口子都在南京看房子,從不大上京。他哥哥金文翔,\begin{note}庚雙夾:更妙!\end{note}現在是老太太那邊的買辦。他嫂子也是老太太那邊漿洗的頭兒。”\begin{note}庚雙夾:只鴛鴦一家寫得榮府中人各有各職,如目已睹。\end{note}邢夫人便令人叫了他嫂子金文翔媳婦來,細細說與他。金家媳婦自是喜歡,興興頭頭找鴛鴦,只望一說必妥,不想被鴛鴦搶白一頓,又被襲人平兒說了幾句,羞惱回來,便對邢夫人說:“不中用,他倒罵了我一場。”因鳳姐兒在旁,不敢提平兒,只說:“襲人也幫著他搶白我,也說了許多不知好歹的話,回不得主子的。太太和老爺商議再買罷。諒那小蹄子也沒有這麼大福,我們也沒有這麼大造化。”邢夫人聽了,因說道:“又與襲人什麼相干?他們如何知道的?”又問:“還有誰在跟前?”金家的道:“還有平姑娘。”鳳姐兒忙道:“你不該拿嘴巴子打他回來?我一出了門,他就逛去了;回家來連一個影兒也摸不著他!他必定也幫著說什麼呢!”金家的道:“平姑娘沒在跟前,遠遠的看著倒象是他,可也不真切,不過是我白忖度。”鳳姐便命人去:“快打了他來,告訴他我來家了,太太也在這裏,請他來幫個忙兒。”豐兒忙上來回道:“林姑娘打發了人下請字請了三四次,他纔去了。奶奶一進門我就叫他去的。林姑娘說:‘告訴你奶奶,我煩他有事呢。’”鳳姐兒聽了方罷,故意的還說:“天天煩他,有些什麼事!”
\end{parag}


\begin{parag}
    邢夫人無計,吃了飯回家,晚間告訴了賈赦。賈赦想了一想,即刻叫賈璉來說:“南京的房子還有人看著,不止一家,即刻叫上金彩來。”賈璉回道:“上次南京信來,金彩已經得了痰迷心竅,那邊連棺材銀子都賞了,不知如今是死是活,便是活著,人事不知,叫來也無用。他老婆子又是個聾子。”賈赦聽了,喝了一聲,又罵:“下流囚攮的,偏你這麼知道,還不離了我這裏!”唬得賈璉退出,一時又叫傳金文翔。賈璉在外書房伺候著,又不敢家去,又不敢見他父親,只得聽著。一時金文翔來了,小幺兒們直帶入二門裏去,隔了五六頓飯的工夫纔出來去了。賈璉暫且不敢打聽,隔了一會,又打聽賈赦睡了,方纔過來。至晚間鳳姐兒告訴他,方纔明白。
\end{parag}


\begin{parag}
    鴛鴦一夜沒睡,至次日,他哥哥回賈母接他家去逛逛,賈母允了,命他出去。鴛鴦意欲不去,只怕賈母疑心,只得勉強出來。他哥哥只得將賈赦的話說與他,又許他怎麼體面,又怎麼當家作姨娘。鴛鴦只咬定牙不願意。他哥哥無法,少不得去回覆了賈赦。賈赦怒起來,因說道:“我這話告訴你,叫你女人向他說去,就說我的話:‘自古嫦娥愛少年’,他必定嫌我老了,大約他戀著少爺們,多半是看上了寶玉,只怕也有賈璉。果有此心,叫他早早歇了心,我要他不來,此後誰還敢收?此是一件。第二件,想著老太太疼他,將來自然往外聘作正頭夫妻去。叫他細想,憑他嫁到誰家去,也難出我的手心。除非他死了,或是終身不嫁男人,我就伏了他!若不然時,叫他趁早回心轉意,有多少好處。”賈赦說一句,金文翔應一聲“是”。賈赦道:“你別哄我,我明兒還打發你太太過去問鴛鴦,你們說了,他不依,便沒你們的不是。若問他,他再依了,仔細你的腦袋!”
\end{parag}


\begin{parag}
    金文翔忙應了又應,退出回家,也不等得告訴他女人轉說,竟自已對面說了這話。把個鴛鴦氣的無話可回,想了一想,便說道:“便願意去,也須得你們帶了我回聲老太太去。”他哥嫂聽了,只當回想過來,都喜之不勝。他嫂子即刻帶了他上來見賈母。
\end{parag}


\begin{parag}
    可巧王夫人、薛姨媽、李紈、鳳姐兒、寶釵等姊妹並外頭的幾個執事有頭臉的媳婦,都在賈母跟前湊趣兒呢。鴛鴦喜之不盡,拉了他嫂子,到賈母跟前跪下,一行哭,一行說,把邢夫人怎麼來說,園子裏他嫂子又如何說,今兒他哥哥又如何說,“因爲不依,方纔大老爺越性說我戀著寶玉,不然要等著往外聘,我到天上,這一輩子也跳不出他的手心去,終久要報仇。我是橫了心的,當著衆人在這裏,我這一輩子莫說是‘寶玉’,便是‘寶金’‘寶銀’‘寶天王’‘寶皇帝’,橫豎不嫁人就完了!就是老太太逼著我,我一刀子抹死了,也不能從命!若有造化,我死在老太太之先;若沒造化,該討喫的命,伏侍老太太歸了西,我也不跟著我老子娘哥哥去,我或是尋死,或是剪了頭髮當尼姑去!若說我不是真心,暫且拿話來支吾,日後再圖別的,天地鬼神,日頭月亮照著嗓子,從嗓子裏頭長疔爛了出來,爛化成醬在這裏!”原來他一進來時,便袖了一把剪子,一面說著,一面左手打開頭髮,右手便鉸。衆婆娘丫鬟忙來拉住,已剪下半綹來了。衆人看時,幸而他的頭髮極多,鉸的不透,連忙替他挽上。賈母聽了,氣的渾身亂戰,口內只說:“我通共剩了這麼一個可靠的人,他們還要來算計!”因見王夫人在旁,便向王夫人道:“你們原來都是哄我的!外頭孝敬,暗地裏盤算我。有好東西也來要,有好人也要,剩了這麼個毛丫頭,見我待他好了,你們自然氣不過,弄開了他,好擺弄我!”王夫人忙站起來,不敢還一言。\begin{note}庚雙夾:千奇百怪,王婦人亦有罪乎?老人家遷怒之言必應如此。\end{note}薛姨媽見連王夫人怪上,反不好勸的了。李紈一聽見鴛鴦的話,早帶了姊妹們出去。
\end{parag}


\begin{parag}
    探春有心的人,想王夫人雖有委曲,如何敢辯;薛姨媽也是親姊妹,自然也不好辯的;寶釵也不便爲姨母辯;李紈、鳳姐、寶玉一概不敢辯;這正用著女孩兒之時,迎春老實,惜春小,因此窗外聽了一聽,便走進來陪笑向賈母道:“這事與太太什麼相干?老太太想一想,也有大伯子要收屋裏的人,小嬸子如何知道?便知道,也推不知道。”猶未說完,賈母笑道:“可是我老糊塗了!姨太太別笑話我。你這個姐姐他極孝順我,不象我那大太太一味怕老爺,婆婆跟前不過應景兒。可是委屈了他。”薛姨媽只答應“是”,又說:“老太太偏心,多疼小兒子媳婦,也是有的。”賈母道:“不偏心!”因又說道:“寶玉,我錯怪了你娘,你怎麼也不提我,看著你娘受委屈?”寶玉笑道:“我偏著娘說大爺大娘不成?通共一個不是,我娘在這裏不認,卻推誰去?我倒要認是我的不是,老太太又不信。”賈母笑道: “這也有理。你快給你娘跪下,你說太太別委屈了,老太太有年紀了,看著寶玉罷。”寶玉聽了,忙走過去,便跪下要說;王夫人忙笑著拉他起來,說:“快起來,快起來,斷乎使不得。終不成你替老太太給我賠不是不成?”寶玉聽說,忙站起來。\begin{note}庚雙夾:寶玉亦有罪了。\end{note}賈母又笑道:“鳳姐兒也不提我。”\begin{note}庚雙夾:阿鳳也有了罪。奇奇怪怪之文,所謂《石頭記》不是作出來的。\end{note}鳳姐兒笑道:“我倒不派老太太的不是,老太太倒尋上我了?”賈母聽了,與衆人都笑道:“這可奇了!倒要聽聽這不是。”鳳姐兒道:“誰教老太太會調理人,調理的水蔥兒似的,怎麼怨得人要?我幸虧是孫子媳婦,若是孫子,我早要了,還等到這會子呢。”賈母笑道:“這倒是我的不是了?”鳳姐兒笑道:“自然是老太太的不是了。”賈母笑道:“這樣,我也不要了,你帶了去罷!”鳳姐兒道:“等著修了這輩子,來生託生男人,我再要罷。”賈母笑道:“你帶了去,給璉兒放在屋裏,看你那沒臉的公公還要不要了!”鳳姐兒道:“璉兒不配,就只配我和平兒這一對燒糊了的卷子和他混罷。”說的衆人都笑起來了。丫鬟回說:“大太太來了。”王夫人忙迎了出去。要知端的
\end{parag}


\begin{parag}
    \begin{note}蒙回末總:鴛鴦女從熱鬧中別具一副腸胃,不輕許人一事,是官途中藥石仙方。\end{note}
\end{parag}

