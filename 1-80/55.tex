\chap{五十五}{辱親女愚妾爭閒氣 欺幼主刁奴蓄險心}

\begin{parag}
    \begin{note}蒙回前:此回接上文,恰似黃鐘大呂,後轉出羽調商聲,別有清涼滋味。\end{note}
\end{parag}


\begin{parag}
    且說元宵已過,只因當今以孝治天下,目下宮中有一位太妃欠安,故各嬪妃皆爲之減膳謝妝,不獨不能省親,亦且將宴樂俱免。故榮府今歲元宵亦無燈謎之集。
\end{parag}


\begin{parag}
    剛將年事忙過,鳳姐便小月了。在家一月,不能理事,天天兩三個太醫用藥。鳳姐自恃強壯,雖不出門,然籌劃計算,想起什麼事來,便命平兒去回王夫人,任人諫勸,他只不聽。王夫人便覺失了膀臂,一個人能有多少精血,凡有了大事,自己主張;將家中瑣碎之事,一應都暫令李紈協理。李紈是個尚德不尚才的,未免逞縱了下人。王夫人便命探春合同李紈裁處,只說過了一月,鳳姐將息好了,仍交與他。誰知鳳姐稟賦氣血不足,兼年幼不知保養,平生爭強鬥志,心力使虧,故雖系小月,竟着實虧虛下來,一月之後,復添了下紅之症。他雖不肯說出來,衆人看他面目黃瘦,便知失於調養,不令他操心。他自己也怕成了大症,遺笑於人,便想偷空調養,恨不得一時復舊如常。誰知一時難痊,調養到八九月間,才漸漸的起復過來,下紅也漸漸止了。此是後話。
\end{parag}


\begin{parag}
    如今且說目今王夫人見他如此,探春與李紈驟難卸事,園中人多,又恐失於照管,因又特請了寶釵來,託他各處小心,“老婆子們不中用,得空兒就鬥牌喫酒,白日裏睡覺,夜裏鬥牌,我都知道的。鳳丫頭在外頭,他們還有個懼怕,如今他們又該取便了。好孩子,你還是個妥當的人,你兄弟妹妹們又小,我又沒工夫,你替我辛苦兩天,照看照看。凡有想不到的事,你來告訴我,別等老太太問出來,我沒話回。那些人不好了,你只管說。他們不聽,你來告訴我。別弄出大事來纔好。”寶釵聽說只得答應了。
\end{parag}


\begin{parag}
    時屆孟春,黛玉又犯了嗽疾。湘雲亦因時氣所感,亦臥病於蘅蕪苑,一天醫藥不斷。探春同李紈相住間隔,二人近日同事,不比往年,來往回話人等亦不便,故二人議定:每日早晨皆到園門口南邊的三間小花廳上去會齊辦事,喫過早飯於午錯方回房。這三間廳原系預備省親之時衆執事太監起坐之處,故省親之後也用不著了,每日只有婆子們上夜。如今天已和暖,不用十分修飾,只不過略略的鋪陳了,便可他二人起坐。這廳上也有一匾,題著“輔仁諭德”四字,家下俗呼皆只叫“議事廳”兒。如今他二人每日卯正至此,午正方散。凡一應執事媳婦等來往回話者,絡繹不絕。
\end{parag}


\begin{parag}
    衆人先聽見李紈獨辦,各各心中暗喜,以爲李紈素日原是個厚道多恩無罰的,自然比鳳姐兒好搪塞。便添了一個探春,也都想著不過是個未出閨閣的青年小姐,且素日也最平和恬淡,因此都不在意,比鳳姐兒前更懈怠了許多。只三四日後,幾件事過手,漸覺探春精細處不讓鳳姐,只不過是言語安靜,性情和順而已。\begin{note}庚雙夾:這是小姐身份耳,阿鳳未出閣想亦如此。\end{note}可巧連日有王公侯伯世襲官員十幾處,皆系榮寧非親即友或世交之家,或有升遷,或有黜降,或有婚喪紅白等事,王夫人賀吊迎送,應酬不暇,前邊更無人。他二人便一日皆在廳上起坐。寶釵便一日在上房監察,至王夫人回方散。每於夜間針線暇時,臨寢之先,坐了小轎帶領園中上夜人等各處巡察一次。他三人如此一理,更覺比鳳姐兒當差時倒更謹慎了些。因而裏外下人都暗中抱怨說:“剛剛的倒了一個‘巡海夜叉’,又添了三個 ‘鎮山太歲’,越性連夜裏偷著喫酒頑的工夫都沒了。”
\end{parag}


\begin{parag}
    這日王夫人正是往錦鄉侯府去赴席,李紈與探春早已梳洗,伺候出門去後,回至廳上坐了。剛喫茶時,只見吳新登的媳婦進來回說:“趙姨娘的兄弟趙國基昨日死了。昨日回過太太,太太說知道了,叫回姑娘奶奶來。”說畢,便垂手旁侍,再不言語。彼時來回話者不少,都打聽他二人辦事如何:若辦得妥當,大家則安個畏懼之心;若少有嫌隙不當之處,不但不畏伏,出二門還要編出許多笑話來取笑。吳新登的媳婦心中已有主意,若是鳳姐前,他便早已獻勤說出許多主意,又查出許多舊例來任鳳姐兒揀擇施行。\begin{note}庚雙夾:可知雖有才幹,亦必有羽翼方可。\end{note}如今他藐視李紈老實,探春是青年的姑娘,所以只說出這一句話來,試他二人有何主見。探春便問李紈。李紈想了一想,便道:“前兒襲人的媽死了,聽見說賞銀四十兩。這也賞他四十兩罷了。”吳新登家的聽了,忙答應了是,接了對牌就走。探春道:“你且回來。”吳新登家的只得回來。探春道:“你且別支銀子。我且問你:那幾年老太太屋裏的幾位老姨奶奶,也有家裏的也有外頭的這兩個分別。家裏的若死了人是賞多少,外頭的死了人是賞多少,你且說兩個我們聽聽。”一問,吳新登家的便都忘了,忙陪笑回說:“這也不是什麼大事,賞多少誰還敢爭不成?”探春笑道:“這話胡鬧。依我說,賞一百倒好。若不按例,別說你們笑話,明兒也難見你二奶奶。”吳新登家的笑道:“既這麼說,我查舊帳去,此時卻記不得。”探春笑道:“你辦事辦老了的,還記不得,倒來難我們。你素日回你二奶奶也現查去?若有這道理,鳳姐姐還不算利害,也就是算寬厚了!還不快找了來我瞧。再遲一日,不說你們粗心,反象我們沒主意了。”吳新登家的滿面通紅,忙轉身出來。衆媳婦們都伸舌頭,這裏又回別的事。
\end{parag}


\begin{parag}
    一時,吳家的取了舊賬來。探春看時,兩個家裏的賞過皆二十兩,兩個外頭的皆賞過四十兩。外還有兩個外頭的,一個賞過一百兩,一個賞過六十兩。這兩筆底下皆有原故:一個是隔省遷父母之柩,外賞六十兩;一個是現買葬地,外賞二十兩。探春便遞與李紈看了。探春便說:“給他二十兩銀子。把這帳留下,我們細看看。”吳新登家的去了。
\end{parag}


\begin{parag}
    忽見趙姨娘進來,李紈探春忙讓坐。趙姨娘開口便說道:“這屋裏的人都踩下我的頭去還罷了。姑娘你也想一想,該替我出氣纔是。”一面說,一面眼淚鼻涕哭起來。探春忙道:“姨娘這話說誰,我竟不解。誰踩姨娘的頭?說出來我替姨娘出氣。”趙姨娘道:“姑娘現踩我,我告訴誰!”探春聽說,忙站起來,說道:“我並不敢。”李紈也站起來勸。趙姨娘道:“你們請坐下,聽我說。我這屋裏熬油似的熬了這麼大年紀,又有你和你兄弟,這會子連襲人都不如了,我還有什麼臉?連你也沒臉面,別說我了!”探春笑道:“原來爲這個。我說我並不敢犯法違理。”一面便坐了,拿帳翻與趙姨娘看,又念與他聽,又說道:“這是祖宗手裏舊規矩,人人都依著,偏我改了不成?也不但襲人,將來環兒收了外頭的,自然也是同襲人一樣。這原不是什麼爭大爭小的事,講不到有臉沒臉的話上。他是太太的奴才,我是按著舊規矩辦。說辦的好,領祖宗的恩典、太太的恩典;若說辦的不均,那是他糊塗不知福,也只好憑他抱怨去。太太連房子賞了人,我有什麼有臉之處;一文不賞,我也沒什麼沒臉之處。依我說,太太不在家,姨娘安靜些養神罷了,何苦只要操心。太太滿心疼我,因姨娘每每生事,幾次寒心。我但凡是個男人,可以出得去,我必早走了,立一番事業,那時自有我一番道理。偏我是女孩兒家,一句多話也沒有我亂說的。太太滿心裏都知道。如今因看重我,才叫我照管家務,還沒有做一件好事,姨娘倒先來作踐我。倘或太太知道了,怕我爲難不叫我管,那才正經沒臉,連姨娘也真沒臉!”一面說,一面不禁滾下淚來。趙姨娘沒了別話答對,便說道:“太太疼你,你越發拉扯拉扯我們。你只顧討太太的疼,就把我們忘了。”探春道:“我怎麼忘了?叫我怎麼拉扯?這也問你們各人,那一個主子不疼出力得用的人?那一個好人用人拉扯的?”李紈在旁只管勸說:“姨娘別生氣。也怨不得姑娘,他滿心裏要拉扯,口裏怎麼說的出來。”探春忙道:“這大嫂子也糊塗了。我拉扯誰?誰家姑娘們拉扯奴才了?他們的好歹,你們該知道,與我什麼相干。”趙姨娘氣的問道:“誰叫你拉扯別人去了?你不當家我也不來問你。你如今現說一是一,說二是二。如今你舅舅死了,你多給了二三十兩銀子,難道太太就不依你?分明太太是好太太,都是你們尖酸刻薄,可惜太太有恩無處使。姑娘放心,這也使不著你的銀子。明兒等出了閣,我還想你額外照看趙家呢。如今沒有長羽毛,就忘了根本,只揀高枝兒飛去了!”探春沒聽完,已氣的臉白氣噎,抽抽咽咽的一面哭,一面問道:“誰是我舅舅?我舅舅年下才升了九省檢點,那裏又跑出一個舅舅來?我倒素習按理尊敬,越發敬出這些親戚來了。既這麼說,環兒出去爲什麼趙國基又站起來,又跟他上學?爲什麼不拿出舅舅的款來?何苦來,誰不知道我是姨娘養的,必要過兩三個月尋出由頭來,徹底來翻騰一陣,生怕人不知道,故意的表白表白。也不知誰給誰沒臉?幸虧我還明白,但凡糊塗不知理的,早急了。”李紈急的只管勸,趙姨娘只管還嘮叨。
\end{parag}


\begin{parag}
    忽聽有人說:“二奶奶打發平姑娘說話來了。”趙姨娘聽說,方把口止住。只見平兒進來,趙姨娘忙陪笑讓坐,又忙問:“你奶奶好些?我正要瞧去,就只沒得空兒。”李紈見平兒進來,因問他來做什麼。平兒笑道:“奶奶說,趙姨奶奶的兄弟沒了,恐怕奶奶和姑娘不知有舊例,若照常例,只得二十兩。如今請姑娘裁奪著,再添些也使得。”探春早已拭去淚痕,忙說道:“又好好的添什麼,誰又是二十四個月養下來的?不然也是那出兵放馬背著主子逃出命來過的人不成?你主子真個倒巧,叫我開了例,他做好人,拿著太太不心疼的錢,樂的做人情。你告訴他,我不敢添減,混出主意。他添他施恩,等他好了出來,愛怎麼添了去。”平兒一來時已明白了對半,今聽這一番話,越發會意,見探春有怒色,便不敢以往日喜樂之時相待,只一邊垂手默侍。
\end{parag}


\begin{parag}
    時值寶釵也從上房中來,探春等忙起身讓坐。未及開言,又有一個媳婦進來回事。因探春才哭了,便有三四個小丫鬟捧了沐盆、巾帕、靶鏡物來。此時探春因盤膝坐在矮板榻上,那捧盆的丫鬟走至跟前,便雙膝跪下,高捧沐盆;那兩個小丫鬟,也都在旁屈膝捧著巾帕並靶鏡脂粉之飾。平兒見侍書不在這裏,便忙上來與探春挽袖卸鐲,又接過一條大手巾來,將探春面前衣襟掩了。探春方伸手向面盆中盥沐。那媳婦便回道:“回奶奶姑娘,家學裏支環爺和蘭哥兒的一年公費。”平兒先道:“你忙什麼!你睜著眼看見姑娘洗臉,你不出去伺候著,先說話來。二奶奶跟前你也這麼沒眼色來著?姑娘雖然恩寬,我去回了二奶奶,只說你們眼裏都沒姑娘,你們都吃了虧,可別怨我。”唬的那個媳婦忙陪笑道:“我粗心了。”一面說,一面忙退出去。
\end{parag}


\begin{parag}
    探春一面勻臉,一面向平兒冷笑道:“你遲了一步,還有可笑的:連吳姐姐這麼個辦老了事的,也不查清楚了,就來混我們。幸虧我們問他,他竟有臉說忘了。我說他回你主子事也忘了再找去?我料著你那主子未必有耐性兒等他去找。”平兒忙笑道:“他有這一次,管包腿上的筋早折了兩根。姑娘別信他們。那是他們瞅著大奶奶是個菩薩,姑娘又是個靦腆小姐,固然是託懶來混。”說著,又向門外說道:“你們只管撒野,等奶奶大安了,咱們再說。”門外的衆媳婦都笑道:“姑娘,你是個最明白的人,俗語說,‘一人作罪一人當’,我們並不敢欺蔽小姐。如今小姐是嬌客,若認真惹惱了,死無葬身之地。”平兒冷笑道:“你們明白就好了。” 又陪笑向探春道:“姑娘知道二奶奶本來事多,那裏照看的這些,保不住不忽略。俗語說‘旁觀者清’,這幾年姑娘冷眼看著,或有該添該減的去處二奶奶沒行到,姑娘竟一添減,頭一件於太太的事有益,第二件也不枉姑娘待我們奶奶的情義了。”話未說完,寶釵李紈皆笑道:“好丫頭,真怨不得鳳丫頭偏疼他!本來無可添減的事,如今聽你一說,倒要找出兩件來斟酌斟酌,不辜負你這話。”探春笑道:“我一肚子氣,沒人煞性子,正要拿他奶奶出氣去,偏他碰了來,說了這些話,叫我也沒了主意了。”一面說,一面叫進方纔那媳婦來問:“環爺和蘭哥兒家學裏這一年的銀子,是做那一項用的?”那媳婦便回說:“一年學裏喫點心或者買紙筆,每位有八兩銀子的使用。”探春道:“凡爺們的使用,都是各屋領了月錢的。環哥的是姨娘領二兩,寶玉的是老太太屋裏襲人領二兩,蘭哥兒的是大奶奶屋裏領。怎麼學裏每人又多這八兩?原來上學去的是爲這八兩銀子!從今兒起,把這一項蠲了。平兒,回去告訴你奶奶,我的話,把這一條務必免了。”平兒笑道: “早就該免。舊年奶奶原說要免的,因年下忙,就忘了。”那個媳婦只得答應著去了。就有大觀園中媳婦捧了飯盒來。
\end{parag}


\begin{parag}
    侍書素雲早已抬過一張小飯桌來,平兒也忙著上菜。探春笑道:“你說完了話幹你的去罷,在這裏忙什麼。”平兒笑道:“我原沒事的。二奶奶打發了我來,一則說話,二則恐這裏人不方便,原是叫我幫著妹妹們伏侍奶奶姑娘的。”探春因問:“寶姑娘的飯怎麼不端來一處喫?”丫鬟們聽說,忙出至檐外命媳婦去說:“寶姑娘如今在廳上一處喫,叫他們把飯送了這裏來。”探春聽說,便高聲說道:“你別混支使人!那都是辦大事的管家娘子們,你們支使他要飯要茶的,連個高低都不知道!平兒這裏站著,你叫叫去。”
\end{parag}


\begin{parag}
    平兒忙答應了一聲出來。那些媳婦們都忙悄悄的拉住笑道:“那裏用姑娘去叫,我們已有人叫去了。”一面說,一面用手帕撣石磯上說:“姑娘站了半天乏了,這太陽影裏且歇歇。”平兒便坐下。又有茶房裏的兩個婆子拿了個坐褥鋪下,說:“石頭冷,這是極乾淨的,姑娘將就坐一坐兒罷。”平兒忙陪笑道:“多謝。”一個又捧了一碗精緻新茶出來,也悄悄笑說:“這不是我們的常用茶,原是伺候姑娘們的,姑娘且潤一潤罷。”平兒忙欠身接了,因指衆媳悄悄說道:“你們太鬧的不象了。他是個姑娘家,不肯發威動怒,這是他尊重,你們就藐視欺負他。果然招他動了大氣,不過說他個粗糙就完了,你們就現吃不了的虧。他撒個嬌兒,太太也得讓他一二分,二奶奶也不敢怎樣。你們就這麼大膽子小看他,可是雞蛋往石頭上碰。”衆人都忙道:“我們何嘗敢大膽了,都是趙姨奶奶鬧的。”平兒也悄悄的說: “罷了,好奶奶們。‘牆倒衆人推’,那趙姨奶奶原有些倒三不著兩,有了事就都賴他。你們素日那眼裏沒人,心術利害,我這幾年難道還不知道?二奶奶若是略差一點兒的,早被你們這些奶奶治倒了。饒這麼著,得一點空兒,還要難他一難,好幾次沒落了你們的口聲。衆人都道他利害,你們都怕他,惟我知道他心裏也就不算不怕你們呢。前兒我們還議論到這裏,再不能依頭順尾,必有兩場氣生。那三姑娘雖是個姑娘,你們都橫看了他。二奶奶這些大姑子小姑子裏頭,也就只單畏他五分。你們這會子倒不把他放在眼裏了。”
\end{parag}


\begin{parag}
    正說著,只見秋紋走來。衆媳婦忙趕著問好,又說:“姑娘也且歇一歇,裏頭擺飯呢。等撤下飯桌子,再回話去。”秋紋笑道:“我比不得你們,我那裏等得。” 說著便直要上廳去。平兒忙叫:“快回來。”秋紋回頭見了平兒,笑道:“你又在這裏充什麼外圍的防護?”一面回身便坐在平兒褥上。平兒悄問:“回什麼?”秋紋道:“問一問寶玉的月錢我們的月錢多早晚才領。”平兒道:“這什麼大事。你快回去告訴襲人,說我的話,憑有什麼事今兒都別回。若回一件,管駁一件;回一百件,管駁一百件。”秋紋聽了,忙問:“這是爲什麼了?”平兒與衆媳婦等都忙告訴他原故,又說:“正要找幾件利害事與有體面的人開例作法子,鎮壓與衆人作榜樣呢。何苦你們先來碰在這釘子上。你這一去說了,他們若拿你們也作一二件榜樣,又礙著老太太、太太;若不拿著你們作一二件,人家又說偏一個向一個,仗著老太太、太太威勢的就怕,也不敢動,只拿著軟的作鼻子頭。你聽聽罷,二奶奶的事,他還要駁兩件,才壓的衆人口聲呢。”秋紋聽了,伸舌笑道:“幸而平姐姐在這裏,沒的臊一鼻子灰。我趕早知會他們去。”說著,便起身走了。
\end{parag}


\begin{parag}
    接著寶釵的飯至,平兒忙進來伏侍。那時趙姨娘已去,三人在板牀上喫飯。寶釵面南,探春面西,李紈面東。衆媳婦皆在廊下靜候,裏頭只有他們緊跟常侍的丫鬟伺候,別人一概不敢擅入。這些媳婦們都悄悄的議論說:“大家省事罷,別安著沒良心的主意。連吳大娘才都討了沒意思,咱們又是什麼有臉的。”他們一邊悄議,等飯完回事。只覺裏面鴉雀無聲,並不聞碗箸之聲。一時只見一個丫鬟將簾櫳高揭,又有兩個將桌擡出。茶房內早有三個丫頭捧著三沐盆水,見飯桌已出,三人便進去了。一回又捧出沐盆並漱盂來,方有侍書、素雲、鶯兒三個,每人用茶盤捧了三蓋碗茶進去。一時等他三人出來,侍書命小丫頭子:“好生伺候著,我們喫飯來換你們,別又偷坐著去。”衆媳婦們方慢慢的一個一個的安分回事,不敢如先前輕慢疏忽了。
\end{parag}


\begin{parag}
    探春氣方漸平,因向平兒道:“我有一件大事,早要和你奶奶商議,如今可巧想起來。你吃了飯快來。寶姑娘也在這裏,咱們四個人商議了,再細細問你奶奶可行可止。”平兒答應回去。
\end{parag}


\begin{parag}
    鳳姐因問爲何去了這一日,平兒便笑著將方纔的原故細細說與他聽了。鳳姐兒笑道:“好,好,好,好個三姑娘!我說他不錯。只可惜他命薄,沒託生在太太肚裏。”平兒笑道:“奶奶也說糊塗話了。他便不是太太養的,難道誰敢小看他,不與別的一樣看了?”鳳姐兒嘆道:“你那裏知道,雖然庶出一樣,女兒卻比不得男人,將來攀親時,如今有一種輕狂人,先要打聽姑娘是正出是庶出,多有爲庶出不要的。殊不知別說庶出,便是我們的丫頭,比人家的小姐還強呢。將來不知那個沒造化的挑庶正誤了事呢,也不知那個有造化的不挑庶正的得了去。”說著,又向平兒笑道:“你知道,我這幾年生了多少省儉的法子,一家子大約也沒個不背地裏恨我的。我如今也是騎上老虎了。雖然看破些,無奈一時也難寬放;二則家裏出去的多,進來的少。凡百大小事仍是照著老祖宗手裏的規矩,卻一年進的產業又不及先時。多省儉了,外人又笑話,老太太、太太也受委屈,家下人也抱怨刻薄;若不趁早兒料理省儉之計,再幾年就都賠盡了。”平兒道:“可不是這話!將來還有三四位姑娘,還有兩三個小爺,一位老太太,這幾件大事未完呢。”風姐兒笑道:“我也慮到這裏,倒也夠了:寶玉和林妹妹他兩個一娶一嫁,可以使不著官中的錢,老太太自有梯己拿出來。二姑娘是大老爺那邊的,也不算。剩下三四個,滿破著每人花上一萬銀子。環哥娶親有限,花上三千兩銀子,不拘那裏省一抿子也就夠了。老太太事出來,一應都是全了的,不過零星雜項,便費也滿破三五千兩。如今再儉省些,陸續也就夠了。只怕如今平空又生出一兩件事來,可就了不得了。──咱們且別慮後事,你且吃了飯,快聽他商議什麼。這正碰了我的機會,我正愁沒個膀臂。雖有個寶玉,他又不是這裏頭的貨,縱收伏了他也不中用。大奶奶是個佛爺,也不中用。二姑娘更不中用,亦且不是這屋裏的人。四姑娘小呢。蘭小子更小。環兒更是個燎毛的小凍貓子,只等有熱竈火坑讓他鑽去罷。真真一個娘肚子裏跑出這個天懸地隔的兩個人來,我想到這裏就不伏。再者林丫頭和寶姑娘他兩個倒好,偏又都是親戚,又不好管咱家務事。況且一個是美人燈兒,風吹吹就壞了;一個是拿定了主意,‘不干己事不張口,一問搖頭三不知’,也難十分去問他。倒只剩了三姑娘一個,心裏嘴裏都也來的,又是咱家的正人,太太又疼他,雖然面上淡淡的,皆因是趙姨娘那老東西鬧的,心裏卻是和寶玉一樣呢。比不得環兒,實在令人難疼,要依我的性早攆出去了。如今他既有這主意,正該和他協同,大家做個膀臂,\begin{note}庚雙夾:阿鳳有才處全在擇人收納膀臂羽翼,並非一味以才自恃者,可知這方是大才。\end{note}我也不孤不獨了。按正理,天理良心上論,咱們有他這個人幫著,咱們也省些心,於太太的事也有些益。若按私心藏奸上論,我也太行毒了,也該抽頭退步。回頭看了看,再要窮追苦克,人恨極了,暗地裏笑裏藏刀,咱們兩個才四個眼睛,兩個心,一時不防,倒弄壞了。趁著緊溜之中,他出頭一料理,衆人就把往日咱們的恨暫可解了。還有一件,我雖知你極明白,恐怕你心裏挽不過來,如今囑咐你:他雖是姑娘家,心裏卻事事明白,不過是言語謹慎;他又比我知書識字,更厲害一層了。如今俗語‘擒賊必先擒王’,他如今要作法開端,一定是先拿我開端。倘或他要駁我的事,你可別分辯,你只越恭敬,越說駁的是纔好。千萬別想著怕我沒臉,和他一犟,就不好了。”平兒不等說完,便笑道:“你太把人看糊塗了。我才已經行在先,這會子又反囑咐我。”鳳姐兒笑道:“我是恐怕你心裏眼裏只有了我,一概沒有別人之故,不得不囑咐。既已行在先,更比我明白了。你又急了,滿口裏‘你’‘我’起來。”平兒道:“偏說‘你’!你不依,這不是嘴巴子,再打一頓。難道這臉上還沒嘗過的不成!”鳳姐兒笑道:“你這小蹄子,要掂多少過子才罷。看我病的這樣,還來慪我。過來坐下,橫豎沒人來,咱們一處喫飯是正經。”
\end{parag}


\begin{parag}
    說著,豐兒等三四個小丫頭子進來放小炕桌。鳳姐只吃燕窩粥,兩碟子精緻小菜,每日分例菜已暫減去。豐兒便將平兒的四樣分例菜端至桌上,與平兒盛了飯來。平兒屈一膝於炕沿之上,半身猶立於炕下,陪鳳姐兒吃了飯,\begin{note}庚雙夾:鳳姐之才又在能邀買人心。\end{note}伏侍漱盥。漱畢,囑咐了豐兒些話,方往探春處來。只見院中寂靜,人已散出。要知端的
\end{parag}


\begin{parag}
    \begin{note}蒙回末總:噫!事有難易哉?探春以姑娘之尊、賈母之愛、以王夫人之付託、以鳳姐之未謝事,暫代數月。而奸奴蜂起,內外欺侮,珠璣小事,突動風波,不亦難乎?以鳳姐之聰明,以鳳姐之才力,以鳳姐之權術,以鳳姐之貴寵,以鳳姐之日夜焦勞,百般彌縫,猶不免騎虎難下,爲移禍東兵之計,不亦難乎?況聰明才力不及鳳姐,又無賈母之愛、姑娘之尊、太太之付託而欲左支右吾撐前達後,不更難乎?士方有志作一番事業,每讀至此,不禁爲之投書以起,三複流連而欲泣也!\end{note}
\end{parag}
