\chap{三十五}{白玉釧親嘗蓮葉羹 黃金鶯巧結梅花絡}

\begin{parag}
    \begin{note}蒙回前總:情因相愛反相傷,何事人多不揣量。黛玉徘徊還自苦,蓮羨甘受使兒枉。\end{note}
\end{parag}


\begin{parag}
    話說寶釵分明聽見林黛玉刻薄他,因記掛着母親哥哥,幷不回頭,一徑去了。這裏林黛玉還自立於花陰之下,遠遠的卻向怡紅院內望着,只見李宮裁、迎春、探春、惜春幷各項人等都向怡紅院內去過之後,一起一起的散盡了,只不見鳳姐兒來,心裏自己盤算道:“如何他不來瞧寶玉?便是有事纏住了,他必定也是要來打個花胡哨,討老太太和太太的好兒纔是。今兒這早晚不來,必有原故。”一面猜疑,一面抬頭再看時,只見花花簇簇一羣人又向怡紅院內來了。定睛看時,只見賈母搭着鳳姐兒的手,後頭邢夫人、王夫人,跟着周姨娘幷丫嬛媳婦等人都進院去了。黛玉看了不覺點頭,想起有父母的人的好處來,早又淚珠滿面。少頃,只見寶釵薛姨媽等也進入去了。忽見紫鵑從背後走來,說道:“姑娘吃藥去罷,開水又冷了。”黛玉道:“你到底要怎麼樣?只是催,我喫不喫,管你什麼相干!”紫鵑笑道:“咳嗽的纔好了些,又不吃藥了。如今雖然是五月裏,\begin{note}蒙側:閨中相憐之情,令人羨慕之至。\end{note}天氣熱,到底也該還小心些。大清早起,在這個潮地方跕了半日,也該回去歇息歇息了。”一句話提醒了黛玉,方覺得有點腿痠;呆了半日,方慢慢的扶着紫鵑,回瀟湘館來。
\end{parag}


\begin{parag}
    一進院門,只見滿地下竹影參差,苔痕濃淡,不覺又想起《西廂記》中所云“幽僻處可有人行,點蒼苔白露冷冷”二句來,因暗暗的嘆道:“雙文,雙文,誠爲命薄人矣。然你雖命薄,尚有孀母弱弟;今日林黛玉之命薄,一幷連孀母弱弟俱無。古人云‘佳人命薄’,然我又非佳人,何命薄勝於雙文哉!”一面想,一面只管走,不防廊上的鸚哥見林黛玉來了,嘎的一聲撲了下來,倒嚇了一跳,因說道:“作死的!又扇了我一頭灰。”那鸚哥仍飛上架去,便叫:“雪雁,快掀簾子,姑娘來了。”黛玉便止住步,以手扣架道:“添了食水不曾?”。那鸚哥便長嘆一聲,竟大似林黛玉素日吁嗟音韻,接着念道:“儂今葬花人笑癡,他年葬儂知是誰?試看春盡花漸落,便是紅顏老死時。一朝春盡紅顏老,花落人亡兩不知!”\begin{note}蒙側:苦成的字句,到今日聽了,競做一場笑話。\end{note}黛玉、紫鵑聽了都笑起來。紫鵑笑道:“這都是素日姑娘唸的,難爲他怎麼記了。”黛玉便命將架摘下來,另掛在月洞窗外的鉤上,於是進了屋子,在月洞窗內坐了。喫畢藥,只見窗外竹影映入紗來,滿屋內陰陰翠潤,幾簟生涼。黛玉無可釋悶,便隔着紗窗調逗鸚哥作戲,又將素日所喜的詩詞也教與他念。這且不在話下。
\end{parag}


\begin{parag}
    且說薛寶釵來至家中,只見母親正自梳頭呢。一見他來了,便說道:“你大清早起跑來作什麼?”寶釵道:“我瞧瞧媽身上好不好。昨兒我去了,不知他可又過來鬧了沒有?”一面說,一面在他母親身旁坐了,由不得哭將起來。薛姨媽見他一哭,自己撐不住,也就哭了一場,一面又勸他:“我的兒,你別委屈了,你等我處分那孽障;你要有個好歹,我指望那一個來!”薛蟠在外聽見,連忙跑了過來,對着寶釵,左一個揖,右一個揖,只說:“好妹妹,恕我這次罷!原是我昨兒吃了酒,回來的晚了,路上撞客着了,來家未醒,不知胡說了什麼,連自己也不知道,怨不得你生氣。”寶釵原是掩面哭的,聽如此說,由不得又好笑了,遂抬頭向地下啐了一口,說道:“你不用做這些像生兒。我知道你的心裏多嫌我們孃兒兩個,你是變着法兒叫我們離了,你就心淨了!”薛蟠聽說,連忙笑道:“妹妹這話從那裏說起來的,這樣我連立足之地都沒了。妹妹從來不是這樣多心說歪話的人。”薛姨媽忙又接着道:“你只會聽見你妹妹的歪話,難道昨兒晚上你說的那話就該的不成?當真是你發昏了!”薛蟠道:“媽也不必生氣,妹妹也不用煩惱,從今以後我再不同他們一處喫酒閒逛如何?”寶釵笑道:“這不明白過來了!”\begin{note}蒙側:親生兄妹形景,逼真貼切。\end{note}薛姨媽道:“你要有這個橫勁,那龍也下蛋了!”薛蟠道:“我若再和他們一處逛,妹妹聽見了,只管啐我,再叫我‘畜生’、‘不是人’,如何?何苦來爲我一個人,孃兒兩個天天操心!媽爲我生氣還有可恕,若只管叫妹妹爲我操心,我更不是人了!如今父親沒了,我不能多孝順媽,多疼妹妹,反教娘生氣,妹妹煩惱,真連個畜生也不如了!”口裏談,眼睛裏禁不起也滾下淚來。\begin{note}蒙側:又是一樣哭法,不過是情之所致。\end{note}薛姨媽本不哭了,聽他一說又勾起傷心來。寶釵勉強笑道:“你鬧夠了,這會子又招着媽哭起來了!”薛蟠聽說,忙收了淚,笑道:“我何曾招媽哭?來罷,來罷!丟下這個別提了。叫香菱來倒茶妹妹喫。”寶釵道:“我也不喫茶,等媽洗了手,我們就道去了!”薛蟠道:“妹妹的項圈我瞧瞧,只怕該炸一炸去了。”寶釵道:“黃澄澄的又炸他作什麼?”薛蟠又道:“妹妹如今也該添補些衣裳了。要什麼顏色花樣?告訴我。”寶釵道:“連那些衣服我還沒穿遍了,又做什麼?”\begin{note}蒙側:一寫骨肉悔過之情,一寫本等貞靜之女。\end{note}一時,薛姨媽換了衣裳,拉着寶釵進去,薛蟠方出去了。
\end{parag}


\begin{parag}
    這裏薛姨媽和寶釵進園來瞧寶玉,到了怡紅院中,只見抱廈裏外迴廊上許多丫嬛老婆站着,便知賈母等都在這裏。母女兩個進來,大家見過了,只見寶玉躺在榻上。薛姨媽問他可好些。寶玉忙欲欠身,口裏答應着“好些”,又說:“只管驚動姨娘、姐姐,我禁不起!”薛姨娘忙扶他睡下,又問他:“想什麼,只管告訴我。”寶玉笑道:“我想起來,自然和姨娘要去的。”王夫人又問:“你想什麼喫?回來好給你送來的。”寶玉笑道:“也倒不想什麼喫,倒是那一回做的那小荷葉兒小蓮蓬兒的湯還好些。”鳳姐一旁笑道:“聽聽,口味不算高貴,只是太磨牙了。巴巴的,想這個吃了。”賈母便一疊聲的叫人做去。鳳姐兒笑道:“老祖宗別急,等我想一想:這模子誰收着呢?”因回頭吩咐個婆子去問管廚房的要去。那婆子去了半天,來回說:“管廚房的說,四副湯模子都交上來了。”鳳姐兒聽說,想了一想,道:“我記得交給誰了,多半在茶房裏。”一面又遣人去問管茶房的,也不曾收。次後還是管金銀器皿的送了來。
\end{parag}


\begin{parag}
    薛姨媽先接過來瞧時,原來是個小匣子,裏面裝着四副銀模子,都有一尺多長,一寸見方,上面鑿着有豆子大小,也有菊花的,也有梅花的,也有蓮蓬的,也有菱角的,共有三四十樣,打的十分精巧。因笑向賈母王夫人道:“你們府上也都想絕了,喫碗湯還有這些樣子。若不說出來,我見這個也不認得這是作什麼用的。”鳳姐兒也不等人說話,便笑道:“姑媽那裏曉得,這是舊年備膳,他們想的法兒。不知弄些什麼面印出來,借點新荷葉的清香,全仗着好湯,究竟沒意思,誰家常喫他了。那一回呈樣的作了一回,他今日怎麼想起來了。”說着接了過來,遞與個婦人,吩咐廚房裏立刻拿幾隻鶏,另外添了東西,做出十來碗來。王夫人道:“要這些做什麼?”鳳姐兒笑道:“有個原故:這一宗東西家常不大作,今兒寶兄弟提起來了,單作給他喫;老太太、姑媽、太太都不喫,似乎不大好。不如借勢兒弄些大家喫,托賴着連我也上個俊兒。”賈母聽了,笑道:“猴兒,把你乖的!拿着官中的錢你作人。”說的大家笑了。鳳姐也忙笑道:“這不相干,這個小東道我還孝敬的起!”便回頭吩咐婦人:“說給廚房裏只管好生添補着,作了在我的帳上來領銀子。”婦人答應着去了。
\end{parag}


\begin{parag}
    寶釵一旁笑道:“我來了這麼幾年,留神看起來,鳳丫頭憑他怎麼巧,再巧不過老太太去。”賈母聽說便答道:“我如今老了,那裏還巧什麼。當日我象鳳哥兒這麼大年紀,比他還來得呢。他如今雖說不如我們,也就算好了,比你姨娘強遠了。你姨娘可憐見的,不大說話,和木頭似的,在公婆跟前就不大顯好。鳳兒嘴乖,怎麼怨得人疼他。”寶玉笑道:“若這麼說,不大說話的就不疼了?”賈母道:“不大說話的,又有不大說話的可疼之處,嘴乖的,也有一宗可嫌的;倒不如不說的好!”寶玉笑道:“這就是了。我說大嫂子倒不大說話呢,老太太也是和鳳姐姐的一樣看待。若是單是會說話的可疼,這些姊妹裏頭也只是鳳姐姐和林妹妹可疼了。”賈母道:“提起姊妹,不是我當着姨太太的面奉承,千真萬真,從我們家四個女孩兒算起都不如寶丫頭。”薛姨媽聽說,忙笑:“這話老太太是說偏了。”王夫人忙又笑道:“老太太時常背地裏和我說寶丫頭好,這倒不是假話。”寶玉勾着賈母原爲贊林黛玉的,不想反贊起寶釵來,倒也意出望外,便看着寶釵一笑。寶釵早扭過頭去和襲人說話去了。
\end{parag}


\begin{parag}
    忽有人來請喫飯,賈母方立起身來,命寶玉好生養着,又把丫頭們囑咐了一回,方扶着鳳姐兒,讓着薛姨媽,大家出房去了。因問湯好了不曾,又問薛姨媽等:“想什麼喫,只管告訴我,我有本事叫鳳丫頭弄了來咱們喫。”薛姨媽笑道:“老太太也會慪他的。時常他弄了東西孝敬,究竟又吃不了多少。”鳳姐兒笑道:“姑媽倒別這樣說。我們老祖宗只是嫌人肉酸,若不嫌人肉酸,早已把我還吃了呢。”一句話沒說了,引的賈母衆人都哈哈的笑起來。
\end{parag}


\begin{parag}
    寶玉在房裏也撐不住笑了。襲人笑道:“真真的二奶奶的這張嘴怕死人!”寶玉伸手拉着襲人笑道:“你站了這半日,可乏了?”一面說,一面拉他身旁坐了。襲人笑道:“可是又忘了。趁寶姑娘在院子裏,你和他說,煩他鶯兒來打上那幾根絡子。”寶玉笑道:“虧你提起來。”說着,便仰頭向窗外道:“寶姐姐,喫過飯叫鶯兒來,煩他打幾根絛子,可得閒兒?”寶釵聽見,回頭道:“怎麼不得閒兒!一會叫他來就是了。”賈母等尚未聽真,都止步問寶釵。寶釵說明了,大家方明白。賈母又說道:“好孩子,叫他來替你兄弟作幾根。你要人使,我那裏閒着的丫頭多呢,你喜歡誰,只管叫了來使喚。”薛姨媽寶釵等都笑道:“只管叫他來作就是了!有什麼使喚的去處?他天天也是閒着淘氣。”
\end{parag}


\begin{parag}
    大家說着往前步,正走,忽見史湘雲、平兒、香菱等在山石邊掐鳳仙花呢,見了他們走來,都迎上來了。少頃至園外,王夫人恐賈母乏了,便欲讓至上房內坐。賈母也覺腿痠,便點頭依允。王夫人便命丫頭忙先去鋪設坐位。那時趙姨娘推病,只有周姨娘與衆婆娘丫頭們忙着打簾子,立靠背,鋪褥子。賈母扶着鳳姐兒進來,與薛姨媽分賓主坐了。薛寶釵、史湘雲坐在下面。王夫人親捧了茶奉與賈母,李宮裁奉與薛姨媽。賈母向王夫人道:“讓他們小妯娌伏侍,你在那裏坐了,好說話兒。”王夫人方向一張小機子上坐下,便吩咐鳳姐兒道:“老太太的飯在這裏放,添了東西來。”鳳姐兒答應出去,便命人去賈母那邊告訴,那邊的婆娘忙往外傳了,丫頭們忙都趕過來。王夫人便命“請姑娘們去”。請了半天,只有探春惜春兩個來了。迎春身上不奈煩,不喫飯。林黛玉,自不消說。平素十頓飯,只好喫五頓,衆人也不著意了。少頃飯至,衆人調放了桌子。鳳姐兒用手巾裹着一把牙筋跕在地下,笑道:“老祖宗和姑媽不用讓,還聽我說就是了。”賈母笑向薛姨媽道:“我們就是這樣。”薛姨媽笑着應了。於是鳳姐放了四雙:上面兩雙是賈母薛姨媽,兩邊是薛寶釵史湘雲的。王夫人李宮裁等都站在地下看着放菜。鳳姐先忙着要乾淨傢伙來,\begin{note}蒙側:如此。\end{note}替寶玉揀菜。
\end{parag}


\begin{parag}
    少頃,荷葉湯來,賈母看過了。王夫人回頭見玉釧兒在那邊,便命玉釧與寶玉送去。鳳姐道:“他一個人拿不去。”可巧鶯兒和喜兒都來了。寶釵知道他們已吃了飯,便向鶯兒道:“寶兄弟正叫你去打絛子,你們兩個一同去罷。”鶯兒答應,同着玉釧兒出來。鶯兒道:“這麼遠,怪熱的,怎麼端了去?”玉釧笑道:“你放心,我自有道理。”說着,便命一個婆子來,將湯飯等類放在一個捧盒裏,\begin{note}蒙側:大家氣象。\end{note}命他端了跟着,他兩個卻空着手走。一直到了怡紅院門口,玉釧兒方接了過來,同鶯兒進入寶玉房中。襲人、麝月、秋紋三個人正和寶玉頑笑呢,見他兩個來了,都忙起來,笑道:“你兩個來的怎麼碰巧,一齊來了。”一面說,一面接了下來。玉釧便向一張機子上坐了,鶯兒不敢坐下。\begin{note}蒙側:兩人不一樣寫,真是各進其文於後。\end{note}襲人便忙端了個腳踏來,\begin{note}蒙側:寶卿之婢,自應與衆不同。\end{note}鶯兒還不敢坐。寶玉見鶯兒來了,卻倒十分歡喜;忽見了玉釧兒,便想起他姐姐金釧兒來了,又是傷心,又是慚愧,便把鶯兒丟下,且和玉釧兒說話。襲人見把鶯兒不理,恐鶯兒沒好意思的,\begin{note}蒙側:何等幽度。\end{note}又見鶯兒不肯坐,便拉了鶯兒出來,到那邊房裏去喫茶說話兒去了。
\end{parag}


\begin{parag}
    這裏麝月等預備了碗箸來伺候喫飯。寶玉只是不喫,問玉釧兒道:“你母親身子好?”玉釧兒滿臉怒色,正眼也不看寶玉,半日,方說了一個“好”字。寶玉便覺沒趣,半日,只得又陪笑,問道:“誰叫你給我送來的?”玉釧兒道:“不過是奶奶太太們!”寶玉見他還是這樣哭喪,便知他是爲金釧兒的原故;待要虛心下氣模轉他,又見人多,不好下氣的,\begin{note}蒙側:金釧兒如若有知,敢何等感激?\end{note}因而便盡方法,將人都支出去,然後又賠笑問長問短。那玉釧兒先雖不欲,只管見,寶玉一些性氣沒有;憑他怎麼喪謗,還是溫存和氣;自己倒不好意思的了,臉上方有三分喜色。\begin{note}蒙側:也著實不過意。\end{note}寶玉便笑求他:“好姐姐,你把那湯拿了來我嚐嚐。”玉釧兒道:“我從不會喂人東西,等他們來了再喫。”寶玉笑道:“我不是要你餵我。我因爲走不動,你遞給我吃了,你好趕早兒回去交代了,你好喫飯的。我只管耽誤時候,你豈不餓壞了。你要懶待動,我少不了,我忍了疼,下去取來。”說着便要下牀來,拃掙起來,禁不住噯喲之聲。玉釧兒見他這般,忍不住起身說道:“躺下罷!那世裏造了來的業,這會子現世現報。叫我那一個眼睛看的上!”\begin{note}蒙側:偏於此間寫此不情之態,以表白多情之苦。\end{note}一面說,一面哧的一聲又笑了,端過湯來。寶玉笑道:“好姐姐,你要生氣只管在這裏生罷,見了老太太、太太可放和氣些,若還這樣,你就又捱罵了。”玉釧兒道:“喫罷,喫罷!不用和我甜嘴蜜舌的,我可不信這樣話!”說着,催寶玉喝了兩口湯。寶玉故意說:“不好喫,不吃了。”玉釧兒道:“阿彌陀佛!這還不好喫,什麼好喫?”寶玉道:“一點味兒也沒有,你不信,嘗一嘗就知道了。”玉釧果真賭氣嚐了一嘗。寶玉笑道:“這可好吃了!”玉釧兒聽說,方解過意來,原是寶玉哄他喫一口,便說道:“你既說不好喫,這會子說好喫也不給你吃了。”寶玉只管陪笑,央求要喫,\begin{note}蒙側:寫盡多情人無限委屈柔腸。\end{note}玉釧兒又不給他,一面又叫人打發喫飯。
\end{parag}


\begin{parag}
    丫頭方進來時,忽有人來回話:“傅二爺家的兩個嬤嬤來請安,來見二爺。”寶玉聽說,便知是通判傅試家的嬤嬤來了。那傅試原是賈政的門生,年來都賴賈家的名勢得意,賈政也著實看待,故與別個門生不同,他那裏常遣人來走動。寶玉素習最厭勇男蠢婦的,今日卻如何又命這兩個婆子過來?其中原來有個原故:只因那寶玉聞得傅試有個妹子,名喚傅秋芳,也是個瓊閨秀玉。常人傳說:才貌俱全,雖自未親睹,然遐思遙愛之心十分誠敬,不命他們進來,恐薄了傅秋芳,\begin{note}已卯夾:癡想。蒙側:大抵諸色非情不生,非情不合。情之表見於愛,愛象則心無定象,心不定,則諸幻業生,諸魔蜂起,則汲汲乎流與無情。此寶玉之多情而不情之案,凡我同人其留意。\end{note}因此連忙命讓進來。那傅試原是暴發的,因傅秋芳有幾分姿色,聰明過人,那傅試安心仗着妹妹要與豪門貴族結姻,不肯輕意許人,所以耽誤到如今。目今傅秋芳年已二十三歲,尚未許人。爭奈那些豪門貴族又嫌他窮酸,根基淺薄,不肯求配。那傅試與賈家親密,也自有一段心事。今日遣來的兩個婆子偏生是極無知識的,聞得寶玉要見,進來只剛問了好,說了沒兩句話。那玉釧見生人來,也不和寶玉廝鬧了,手裏端着湯只顧聽話。寶玉又只顧和婆子說話,一面喫飯,一面伸手去要湯。兩個人的眼睛都看着人,不想伸猛了手,便將碗撞落,將湯潑了寶玉手上。玉釧兒倒不曾燙着,唬了一跳,忙笑了,“這是怎麼說!”慌的丫頭們忙上來接碗。寶玉自己燙了手倒不覺的,卻只管問玉釧兒:“燙了那裏了?疼不疼?”\begin{note}蒙側:多情人每於苦惱時不自覺,反說彼家苦惱。愛之至,惜之深之故也。\end{note}玉釧兒和衆人都笑了。玉釧兒道:“你自己燙了,只管問我。”寶玉聽說,方覺自己燙了。衆人上來連忙收拾。寶玉也不喫飯了,洗手喫茶,又和那兩個婆子說了兩句話。然後兩個婆子告辭出去,晴雯等送至橋邊方回。
\end{parag}


\begin{parag}
    那兩個婆子見沒人了,一行走,一行談論。這一個笑道:“怪道有人說他們家寶玉是:‘外像好裏頭糊塗,中看不中喫的。’果然有些呆氣。他自己燙了手,倒問人疼不疼,這可不是個呆子?”那一個又笑道:“我前一回來,聽見他們家裏許多人抱怨,千真萬真的有些呆氣。大雨淋的水鶏似的,他反告訴別人:‘下雨了,快避雨去罷!’你說可笑不可笑?時常沒人在跟前,就自哭自笑的;看見燕子,就和燕子說話;河裏看見了魚,就和魚說話;見了星星月亮,不是長吁短嘆,就是咭咭噥噥的。且連一點剛性也沒有,連那些毛丫頭的氣都受的。愛惜東西,連個線頭兒都是好的;糟踏起來,那怕值千值萬的都不管了。”\begin{note}蒙側:如人飲水,冷暖自知。其中深意味,豈能持告君。\end{note}兩個人一面說,一面走出園來,辭別諸人回去,不在話下。\begin{note}庚、已卯、有正、蒙批:寶玉之爲人非此一論亦描寫不盡,寶玉之不肖非此一鄙亦形容不到,試問作者是醜寶玉乎?是贊寶玉乎?試問觀者是喜寶玉乎?是嫌寶玉乎(庚、已卯批:是“惡”寶玉乎)?\end{note}
\end{parag}


\begin{parag}
    如今且說襲人見人去,便攜了鶯兒過來,問寶玉打什麼絡子。寶玉笑向鶯兒道:“才只顧說話,就忘了你。煩你來,不爲別的,也替我打幾根絡子。”鶯兒道:“裝什麼的絡子?”寶玉見問,便笑道:“不管裝什麼的,你都每樣打幾個罷。”\begin{note}蒙側:富家子弟每多有如是語,只不自覺耳。\end{note}鶯兒拍手笑道:“這還了得!要這樣,十年也打不完了!”寶玉笑道:“好姐姐,你閒着也沒事,都替我打了罷。”襲人笑道:“那裏一時都打得完?如今先揀要緊的打兩個罷。”鶯兒道:“什麼要緊?!不過是扇子、香墜兒、汗巾子。”寶玉道:“汗巾子就好。”鶯兒道:“汗巾子是什麼顏色的?”寶玉道:“大紅的。”鶯兒道:“大紅的須是黑絡子纔好看的,或是石青的才壓的住顏色。”寶玉道:“松花色配什麼?”鶯兒道:“松花配桃紅。”寶玉笑道:“這才姣艶!再要雅淡之中帶些姣艶。”鶯兒道:“蔥綠柳黃是我最愛的。”寶玉道:“也罷了,也打一條桃紅,再打一條蔥綠。”鶯兒道:“什麼花樣呢?”寶玉道:“共有幾樣花樣?”鶯兒道:“一炷香,朝天凳象眼塊方,勝連環梅花柳葉。”寶玉道:“前兒你替三姑娘打的那花樣是什麼?”鶯兒道:“那是攢心梅花。”寶玉道:“就是那樣好。”一面說,一面叫襲人剛拿了線來,窗外婆子說“姑娘們的飯都有了。”寶玉道:“你們喫飯去,快吃了來罷。”襲人笑道:“有客在這裏,我們怎好去的!”\begin{note}蒙側:人情物理,一絲不亂。\end{note}鶯兒一面理線,一面笑道:“這話又打那裏說起,正緊快吃了來罷。”襲人等聽說方去了,只留下兩個小丫頭聽呼喚。
\end{parag}


\begin{parag}
    寶玉一面看鶯兒打絡子,一面說閒話,因問他:“十幾歲了?”鶯兒手裏打着,一面答話說:“十六歲了。”寶玉道:“你本姓什麼?”鶯兒道:“姓黃。”寶玉笑道:“這個名、姓倒對了,果然是個黃鶯兒。”鶯兒笑道:“我的名字本來是兩個字,叫作金鶯。姑娘嫌拗口,就單叫鶯兒,如今就叫開了。”寶玉道:“寶姐姐也算疼你了。明兒寶姐姐出閣,少不得是你跟去了!”鶯兒抿嘴一笑。寶玉笑道:“我常常和襲人說,明兒不知那一個有福的消受你們主子奴才兩個呢!”\begin{note}蒙側:是有心?是無心?\end{note}鶯兒笑道:“你還不知道我們姑娘有幾樣世人都沒有的好處呢,模樣兒還在次。”寶玉見鶯兒姣憨婉轉,語笑如癡,早不勝其情了,那更提起寶釵來!便問他道:“好處在那裏?好姐姐,細細告訴我聽。”鶯兒笑道:“我告訴你,你可不許又告訴他去!”\begin{note}蒙側:閨房閒話,著實幽韻。\end{note}寶玉笑道:“這個自然的。”正說着,只聽外頭說道:“怎麼這樣靜悄悄的!”二人回頭看時,不是別人,正是寶釵來了。寶玉忙讓坐。寶釵坐了,因問鶯兒“打什麼呢?”一面問,一面向他手裏去瞧,纔打了半截。寶釵笑道:“這有什麼趣兒,倒不如打個絡子把玉絡上呢。”一句話提醒了寶玉,便拍手笑道:“倒是姐姐說得是,我就忘了。只是配個什麼顏色纔好?”寶釵道:“若用雜色斷然使不得,大紅又犯了色,黃的又不起眼,黑的又過暗。等我想個法兒:把那金線拿來,配着黑珠兒線,一根一根的拈上,打成絡子,這纔好看。”
\end{parag}


\begin{parag}
    寶玉聽說,喜之不盡,一疊聲便叫襲人來取金線。正值襲人端了兩碗菜走進來,告訴寶玉道:“今兒奇怪,纔剛太太打發人替我送了兩碗菜來。”寶玉笑道:“必定是今兒菜多,送來給你們大家喫的。”襲人道:“不是!指名給我送來,還不叫我過去磕頭?這可是奇了!”寶釵笑道:“給你的,你就吃了,這有什麼可猜疑的。”襲人笑道:“從來沒有的事,倒叫我不好意思的。”寶釵抿嘴一笑,說道:“這就不好意思了?\begin{note}蒙側:寶玉之慧性靈心。\end{note}明兒還有比這個更教你不好意思的呢。”襲人聽了話內有因,素知寶釵不是輕嘴薄舌奚落人的,自己方想起上日王夫人的意思來,便不再提,將菜與寶玉看了說:“洗了手來拿線。”說畢,便一直的出去了。喫過飯,洗了手,進來拿金線與鶯兒打絡子。此時寶釵早被薛蟠遣人來請出去了。
\end{parag}


\begin{parag}
    這裏寶玉正看着打絡子,忽見邢夫人那邊遣了兩個丫嬛送了兩樣菓子來與他喫,問他‘可走得了?若走得動,叫哥兒明兒過來散散心,’太太著實記掛着呢!寶玉忙道:“若走得了,必請太太的安去。疼的比先好些,請太太放心罷。”一面叫他兩個坐下,一面又叫秋紋來,把才那果子拿一半送與林姑娘去。秋紋答應了,剛欲去時,只聽黛玉在院內說話,寶玉忙叫:“快請。”要知端的,且聽下回分解。
\end{parag}


\begin{parag}
    \begin{note}蒙回末總:此回是以情說法,警醒世人。黛玉回情凝思默度,忘其有身,忘其有病;而寶玉千屈萬折因情,忘其尊卑,忘其痛苦,幷忘其性情。愛河之深無底,何可氾濫?一溺其中,非死不止。且凡愛者不專,新舊疊增,豈能盡了其多情之心?不能不流於無情之地。究其立意,倏忽千里而自不覺。誠可悲乎!\end{note}
\end{parag}

