\chap{二十一}{贤袭人娇嗔箴宝玉 俏平儿软语救贾琏}

\begin{parag}
    \begin{note}庚辰:有客题《红楼梦》一律,失其姓氏,惟见其诗意骇警,故录于斯:“自执金矛又执戈,自相戕戮自张罗。茜纱公子情无限,脂砚先生恨几多。是幻是真空历遍,闲风闲月枉吟哦。情机转得情天破,情不情兮奈我何?”凡是书题者不少,此为绝调。诗句警拔,且深知拟书底里,惜乎失名矣!\end{note}
\end{parag}


\begin{parag}
    \begin{note}蒙回前批:按此回之文固妙,然未见后三十回犹不见此之妙。此回“娇嗔箴宝玉”、“软语救贾琏”,后文“薛宝钗借词含讽谏,王熙凤知命强英雄”。今只从二婢说起,后则直指其主。然今日之袭人、之宝玉,亦他日之袭人、他日之宝玉也。今日之平儿、之贾琏,亦他日之平儿、他日之贾琏也。何今日之玉犹可箴,他日之玉已不可箴耶?今日之琏犹可救,他日之琏已不能救耶?箴与谏无异也,而袭人安在哉?宁不悲乎!救与强无别也,甚矣!但此日阿凤英气何如是也,他日之身微运蹇,亦何如是也?人世之变迁,倏忽如此!\end{note}
\end{parag}


\begin{parag}
    \begin{note}蒙回前批:今日写袭人,后文写宝钗;今日写平儿,后文写阿凤。文是一样情理,景况光阴,事却天壤矣!多少恨泪洒出此两回书。\end{note}
\end{parag}


\begin{parag}
    \begin{note}蒙回前批:此回袭人三大功,直与宝玉一生三大病映射。\end{note}
\end{parag}


\begin{parag}
    话说史湘云跑了出来,怕林黛玉赶上,宝玉在后忙说:“仔细绊跌了!那里就赶上了?”林黛玉赶到门前,被宝玉叉手在门框上拦住,笑劝道:“饶他这一遭罢。”林黛玉搬著手说道:“我若饶过云儿,再不活著!”湘云见宝玉拦住门,料黛玉不能出来,\begin{note}庚辰双行夹批:写得湘云与宝玉又亲厚之极,却不见疏远黛玉,是何情思耶?\end{note}便立住脚笑道:“好姐姐,饶我这一遭罢。”恰值宝钗来在湘云身后,也笑道:“我劝你两个看宝兄弟分上,都丢开手罢。”\begin{note}庚辰双行夹批:好极,妙极!玉、颦、云三人已难解难分,插入宝钗云“我劝你两个看宝玉兄弟分上”,话只一句,便将四人一齐笼住,不知孰远孰近,孰亲孰疏,真好文字!\end{note}黛玉道:“我不依。你们是一气的,都戏弄我不成!”\begin{note}庚辰双行夹批:话是颦儿口吻,虽属尖利,真实堪爱堪怜。\end{note}宝玉劝道:“谁敢打趣你!你不打趣他,他焉敢说你?”\begin{note}庚辰双行夹批:好!二“你”字连二“他”字,华灼之至!\end{note}四人正难分解,\begin{note}庚辰双行夹批:好!前三人,今忽四人,俱是书中正眼,不可少矣。\end{note}有人来请吃饭,方往前边来。\begin{note}庚辰双行夹批:好文章!正是闺中女儿口角之事。若只管谆谆不已,则成何文矣!\end{note}
\end{parag}


\begin{parag}
    那天早又掌灯时分,王夫人、李纨、凤姐、迎、探、惜等都往贾母这边来,大家闲话了一回,各自归寝。湘云仍往黛玉房中安歇。\begin{note}庚辰双行夹批:前文黛玉未来时,湘云、宝玉则随贾母。今湘云已去,黛玉既来,年岁渐成,宝玉各自有房,黛玉亦各有房,故湘云自应同黛玉一处也。\end{note}
\end{parag}


\begin{parag}
    宝玉送他二人到房,那天已二更多时,袭人来催了几次,方回自己房中来睡。次日天明时,便披衣靸鞋往黛玉房中来,不见紫鹃、翠缕二人,只见他姊妹两个尚卧在衾内。那林黛玉\begin{note}庚辰双行夹批:写黛玉身分。\end{note}严严密密裹著一幅杏子红绫被,安稳合目而睡。\begin{note}庚辰双行夹批:一个睡态。\end{note}那史湘云却一把青丝拖于枕畔,被只齐胸,一弯雪白的膀子撂于被外,又带著两个金镯子。\begin{note}庚辰双行夹批:又一个睡态。写黛玉之睡态,俨然就是娇弱女子,可怜。湘云之态,则俨然是个娇态女儿,可爱。真是人人俱尽,个个活跳,吾不知作者胸中埋伏多少裙钗。\end{note}宝玉见了,叹道:\begin{note}庚辰双行夹批:“叹”字奇!除玉卿外,世人见之自曰喜也。\end{note}“睡觉还是不老实!回来风吹了,又嚷肩窝疼了。”一面说,一面轻轻的替他盖上。林黛玉早已醒了,\begin{note}庚辰侧批:不醒不是黛玉了。\end{note}觉得有人,就猜著定是宝玉,因翻身一看,果中其料。因说道:“这早晚就跑过来作什么?”宝玉笑道:“这天还早呢!你起来瞧瞧。”黛玉道:“你先出去,让我们起来。”\begin{note}庚辰侧批:一丝不乱。\end{note}宝玉听了,转身出至外边。
\end{parag}


\begin{parag}
    黛玉起来叫醒湘云,二人都穿了衣服。宝玉复又进来,坐在镜台旁边,只见紫鹃、雪雁进来伏侍梳洗。湘云洗了面,翠缕便拿残水要泼,宝玉道:“站著,我趁势洗了就完了,省得又过去费事。”说著便走过来,弯腰洗了两把。\begin{note}庚辰侧批:妙在两把。\end{note}紫鹃递过香皂去,宝玉道:“这盆里的就不少,不用搓了。”再洗了两把,便要手巾。\begin{note}庚辰侧批:在怡红何其费事多多。\end{note}翠缕道:“还是这个毛病儿,多早晚才改。”\begin{note}庚辰侧批:冷眼人旁点,一丝不漏。\end{note}宝玉也不理,忙忙的要过青盐擦了牙,嗽了口,完毕,见湘云已梳完了头,便走过来笑道:“好妹妹,替我梳上头罢。”湘云道:“这可不能了。”宝玉笑道:“好妹妹,你先时怎么替我梳了呢?”湘云道:“如今我忘了,\begin{note}庚辰眉批:“忘了”二字在娇憨。\end{note}怎么梳呢?”宝玉道:“横竖我不出门,又不带冠子勒子,不过打几根散辫子就完了。”说著,又千妹妹万妹妹的央告。\begin{note}庚辰眉批:口中自是应声而出,捉笔人却从何处设想而来,成此天然对答。壬午九月。\end{note}湘云只得扶过他的头来,一一梳篦。在家不戴冠,并不总角,只将四围短发编成小辫,往顶心发上归了总,编一根大辫,红绦结住。自发顶至辫梢,一路四颗珍珠,下面有金坠脚。湘云一面编著,一面说道:“这珠子只三颗了,这一颗不是的。\begin{note}庚辰侧批:梳头亦有文字,前已叙过,今将珠子一穿插,却天生有是事。\end{note}我记得是一样的,怎么少了一颗?”宝玉道:“丢了一颗。”湘云道:“必定是外头去掉下来,不防被人拣了去,倒便宜他。”\begin{note}庚辰双行夹批:妙谈!道“到便宜他”四字,是大家千金口吻。近日多用 “可惜了的”四字。今失一珠,不闻此四字。妙极!是极!\end{note}\begin{note}庚辰眉批:“到便宜他”四字与“忘了”二字是一气而来,将一侯府千金白描矣。畸笏。\end{note}黛玉一旁盥手,冷笑道:\begin{note}庚辰侧批:纯用画家烘染法。\end{note}“也不知是真丢了,也不知是给了人镶什么戴去了!”宝玉不答,\begin{note}庚辰双行夹批:有神理,有文章。\end{note}因镜台两边俱是妆奁等物,顺手拿起来赏玩,\begin{note}庚辰双行夹批:何赏玩也?写来奇特。\end{note}不觉又顺手拈了胭脂,意欲要往口边送,\begin{note}庚辰双行夹批:是袭人劝后余文。\end{note}因又怕史湘云说。\begin{note}庚辰双行夹批:好极!的是宝玉也。\end{note}正犹豫间,湘云果在身后看见,一手掠著辫子,便伸手来“拍”的一下,从手中将胭脂打落,说道:“这不长进的毛病儿,多早晚才改过!”\begin{note}庚辰侧批:前翠缕之言并非白写。\end{note}
\end{parag}


\begin{parag}
    一语未了,只见袭人进来,看见这般光景,知是梳洗过了,只得回来自己梳洗。忽见宝钗走来,因问道:“宝兄弟那去了?”袭人含笑道:“宝兄弟那里还有在家的工夫!”宝钗听说,心中明白。又听袭人叹道:“姊妹们和气,也有个分寸礼节,也没个黑家白日闹的!凭人怎么劝,都是耳旁风。”宝钗听了,心中暗忖道: “倒别看错了这个丫头,听他说话,倒有些识见。”\begin{note}庚辰双行夹批:此是宝卿初试,已下渐成知已,盖宝卿从此心察得袭人果贤女子也。\end{note}宝钗便在炕上坐了,\begin{note}庚辰双行夹批:好!逐回细看,宝卿待人接物,不疏不亲,不远不近。可厌之人,亦未见冷淡之态,形诸声色;可喜之人,亦未见醴密之情,形诸声色。今日“便在炕上坐了”,盖深取袭卿矣。二人文字,此回为始。详批于此,诸公请记之。\end{note}慢慢的闲言中套问他年纪家乡等语,留神窥察,其言语志量深可敬爱。\begin{note}庚辰双行夹批:四字包罗许多文章笔墨,不似近之开口便云“非诸女子之可比者”,此句大坏。然袭人故佳矣,不书此句是大手眼。\end{note}
\end{parag}


\begin{parag}
    一时宝玉来了,宝钗方出去。\begin{note}庚辰双行夹批:奇文!写得钗、玉二人形景较诸人皆近,何也?宝玉之心,凡女子前不论贵贱,皆亲密之至,岂于宝钗前反生远心哉?盖宝钗之行止端肃恭严,不可轻犯,宝玉欲近之,而恐一时有渎,故不敢狎犯也。宝钗待下愚尚且和平亲密,何反于兄弟前有远心哉?盖宝玉之形景已泥于闺阁,近之则恐不逊,反成远离之端也。故二人之远,实相近之至也。至颦儿于宝玉实近之至矣,却远之至也。不然,后文如何反较胜角口诸事皆出于颦哉?以及宝玉砸玉,颦儿之泪枯,种种孽障,种种忧忿,皆情之所陷,更何辩哉?此一回将宝玉、袭人、钗、颦、云等行止大概一描,已启后大观园中文字也。今详批于此,后久不忽矣。钗与玉远中近,颦与玉近中远,是要紧两大股,不可粗心看过。\end{note}宝玉便问袭人道:“怎么宝姐姐和你说的这么热闹,见我进来就跑了?”\begin{note}庚辰侧批:此问必有。\end{note}问一声不答,再问时,袭人方道:“你问我么?我那里知道你们的原故。”宝玉听了这话,见他脸上气色非往日可比,便笑道:“怎么动了真气?”\begin{note}庚辰双行夹批:宝玉如此。\end{note}袭人冷笑道:“我那里敢动气!只是从今以后别再进这屋子了。横竖有人伏侍你,再别来支使我。我仍旧还伏侍老太太去。”一面说,一面便在炕上合眼倒下。\begin{note}蒙侧批:是醋?是谏?不敢拟定,似在可否之间!\end{note}\begin{note}蒙双行夹批:醋妒妍憨假态,至矣尽矣!观者但莫认真此态为幸。\end{note}宝玉见了这般景况,深为骇异,\begin{note}蒙双行夹批:好!可知未尝见袭人之如此技艺也!\end{note}禁不住赶来劝慰。那袭人只管合了眼不理。\begin{note}庚辰双行夹批:与颦儿前番娇态如何?愈觉可爱犹甚。\end{note}宝玉无了主意,因见麝月进来,\begin{note}庚辰双行夹批:偏麝月来,好文章!\end{note}便问道:“你姐姐怎么了?”\begin{note}庚辰双行夹批:如见如闻。\end{note}麝月道:“我知道么?问你自己便明白了。”\begin{note}庚辰双行夹批:又好麝月!\end{note}宝玉听说,呆了一回,自觉无趣,便起身叹道:“不理我罢,我也睡去。”说著,便起身下炕,到自己床上歪下。袭人听他半日无动静,微微的打鼾,\begin{note}庚辰侧批:真乎?诈乎?\end{note}料他睡著,便起身拿一领斗蓬来,替他刚压上,只听“忽”的一声,\begin{note}庚辰侧批:文是好文,唐突我袭卿,吾不忍也。\end{note}宝玉便掀过去,也仍合目装睡。\begin{note}庚辰双行夹批:写得烂熳。\end{note}袭人明知其意,便点头冷笑道:“你也不用生气,从此后我只当哑子,再不说你一声儿,如何?”宝玉禁不住起身问道:“我又怎么了?你又劝我。你劝我也罢了,才刚又没见你劝我,一进来你就不理我,赌气睡了。我还摸不著是为什么,这会子你又说我恼了。\begin{note}庚辰侧批:这是委屈了石兄。\end{note}我何尝听见你劝我什么话了。”袭人道:“你心里还不明白,还等我说呢!”\begin{note}庚辰侧批:亦是囫囵语,却从有生以来肺腑中出,千斤重。\end{note}\begin{note}庚辰眉批:《石头记》每用囫囵语处,无不精绝奇绝,且总不觉相犯。壬午九月。畸笏。\end{note}
\end{parag}


\begin{parag}
    正闹著,贾母遣人来叫他吃饭,方往前边来,胡乱吃了半碗,仍回自己房中。只见袭人睡在外头炕上,麝月在旁边抹骨牌。宝玉素知麝月与袭人亲厚,一并连麝月也不理,揭起软帘自往里间来。麝月只得跟进来。宝玉便推他出去,说:“不敢惊动你们。”麝月只得笑著出来,唤了两个小丫头进来。宝玉拿一本书,歪著看了半天,因要茶,抬头只见两个小丫头在地下站著。一个大些儿的生得十分水秀,\begin{note}庚辰双行夹批:二字奇绝!多少娇态包括一尽。今古野史中无有此文也。\end{note}宝玉便问:“你叫什么名字?”那丫头便说:“叫蕙香。”\begin{note}庚辰双行夹批:也好。\end{note}宝玉便问:“是谁起的?”蕙香道:“我原叫芸香的,\begin{note}庚辰双行夹批:原俗。\end{note}是花大姐姐改了蕙香。”宝玉道:“正经该叫‘晦气’罢了,什么蕙香呢!”\begin{note}庚辰双行夹批:好极!趣极!\end{note}又问:“你姊妹几个?”蕙香道:“四个。”宝玉道: “你第几?”蕙香道:“第四。”宝玉道:“明儿就叫‘四儿’,不必什么‘蕙香’‘兰气’的。那一个配比这些花,没的玷辱了好名好姓。”\begin{note}庚辰双行夹批: “花袭人”三字在内,说的有趣。\end{note}一面说,一面命他倒了茶来吃。袭人和麝月在外间听了抿嘴而笑。\begin{note}庚辰双行夹批:一丝不漏,好精神!\end{note}
\end{parag}


\begin{parag}
    这一日,宝玉也不大出房,\begin{note}庚辰双行夹批:此是袭卿第一功劳也。\end{note}也不和姊妹丫头等厮闹,\begin{note}庚辰双行夹批:此是袭卿第二功劳也。\end{note}自己闷闷的,只不过拿著书解闷,或弄笔墨,\begin{note}庚辰双行夹批:此虽未必成功,较往日终有微补小益,所谓袭卿有三大功劳也。\end{note}也不使唤众人,只叫四儿答应。谁知四儿是个聪敏乖巧不过的丫头,\begin{note}庚辰双行夹批:又是一个有害无益者。作者一生为此所误,批者一生亦为此所误,于开卷凡见如此人,世人故为喜,余反抱恨,盖四字误人甚矣。被误者深感此批。\end{note}见宝玉用他,他变尽方法笼络宝玉。\begin{note}庚辰双行夹批:他好,但不知袭卿之心思何如?\end{note}至晚饭后,宝玉因吃了两杯酒,眼饧耳热之际,若往日则有袭人等大家喜笑有兴,今日却冷清清的一人对灯,好没兴趣。待要赶了他们去,又怕他们得了意,以后越发来劝,\begin{note}庚辰双行夹批:宝玉恶劝,此是第一大病也。\end{note}若拿出做上的规矩来镇唬,似乎无情太甚。\begin{note}庚辰双行夹批:宝玉重情不重礼,此是第二大病也。\end{note}说不得横心只当他们死了,横竖自然也要过的。便权当他们死了,毫无牵挂,反能怡然自悦。\begin{note}庚辰双行夹批:此意却好,但袭卿辈不应如此弃也。宝玉之情,今古无人可比,固矣。然宝玉有情极之毒,亦世人莫忍为者,看至后半部则洞明矣。此是宝玉三大病也。宝玉有此世人莫忍为之毒,故后文方有“悬崖撒手”一回。若他人得宝钗之妻、麝月之婢,岂能弃而为僧哉?此宝玉一生偏僻处。\end{note}因命四儿剪灯烹茶,自己看一回《南华经》。正看至《外篇•胠箧》一则,其文曰:
\end{parag}
\begin{qute2sp}

    故绝圣弃知,大盗乃止,擿玉毁珠,小盗不起,焚符破玺,而民朴鄙,掊斗折衡,而民不争,殚残天下之圣法,而民始可与论议。擢乱六律,铄绝竽瑟,塞瞽旷之耳,而天下始人含其聪矣;灭文章,散五采,胶离朱之目,而天下始人含其明矣,毁钩绳而弃规矩,攦工倕之指,而天下始人有其巧矣。\begin{note}庚辰双行夹批:此上语本《庄子》。\end{note}
\end{qute2sp}


\begin{parag}
    看至此,意趣洋洋,趁著酒兴,不禁提笔续曰:\begin{note}蒙侧批:敢续!\end{note}\begin{note}庚辰眉批:趁著酒兴不禁而续,是作者自站地步处,谓余何人耶,敢续《庄子》?然奇极怪极之笔,从何设想,怎不令人叫绝?己卯冬夜。\end{note}\begin{note}庚辰眉批:这亦暗露玉兄闲窗净几、不寂不离之工业。壬午孟夏。\end{note}
\end{parag}
\begin{qute2sp}

    焚花散麝,而闺阁始人含其劝矣,戕宝钗之仙姿,灰黛玉之灵窍,丧减情意,而闺阁之美恶始相类矣。彼含其劝,则无参商之虞矣,戕其仙姿,无恋爱之心矣,灰其灵窍,无才思之情矣。彼钗、玉、花、麝者,皆张其罗而穴其隧,所以迷眩缠陷天下者也。\begin{note}庚辰双行夹批:直似庄老,奇甚怪甚!庚辰眉批:赵香梗先生《秋树根偶谭》内兖州少陵台有子美祠为郡守毁为已祠。先生叹子美生遭丧乱,奔走无家,孰料千百年后数椽片瓦犹遭贪吏之毒手。甚矣,才人之厄也!因改公《茅屋为秋风所破歌》数句,为少陵解嘲:“少陵遗像太守欺无力,忍能对面为盗贼,公然折克非已祠,旁人有口呼不得,梦归来兮闻叹息,白日无光天地黑。安得旷宅千万间,太守取之不尽生欢颜,公祠免毁安如山。”读之令人感慨悲愤,心常耿耿。壬午九月。因索书甚迫,姑志于此,非批《石头记》也。为续《庄子因》数句,真是打破胭脂阵,坐透红粉关,另开生面之文,无可评处。\end{note}
\end{qute2sp}


\begin{parag}
    续毕,掷笔就寝。头刚著枕便忽睡去,一夜竟不知所之,直至天明方醒。\begin{note}庚辰双行夹批:此犹是袭人余功也。想每日每夜,宝玉自是心忙身忙口忙之极,今则怡然自适。虽此一刻,于身心无所补益,能有一时之闲闲自若,亦岂非袭卿之所使然耶?\end{note}翻身看时,只见袭人和衣睡在衾上。\begin{note}庚辰双行夹批:神极之笔!试思袭人不来同卧亦不成文字,来同卧更不成文字。却云“和衣衾上”,正是来同卧不来同卧之间。何神奇文妙绝矣!好袭人!真好石头记得真,真好述者述得不错,真好批者批得出。\end{note}宝玉将昨日的事已付与度外,\begin{note}蒙双行夹批:更好!可见玉卿的是天真烂漫之人也!近之所谓□公子又曰“老好人”、“无心道人”是也!殊不知尚古淳风。\end{note}便推他说道:“起来好生睡,看冻著了。”
\end{parag}


\begin{parag}
    原来袭人见他无晓夜和姊妹们厮闹,若直劝他,料不能改,故用柔情以警之,料他不过半日片刻仍复好了。不想宝玉一日夜竟不回转,自己反不得主意,直一夜没好生睡得。今忽见宝玉如此,料他心意回转,便越性不睬他。宝玉见他不应,便伸手替他解衣,刚解开了钮子,被袭人将手推开,\begin{note}庚辰侧批:好看煞!\end{note}又自扣了。宝玉无法,只得拉他的手笑道:“你到底怎么了?”连问几声,袭人睁眼说道:“我也不怎么。你睡醒了,你自过那边房里去梳洗,再迟了就赶不上。”\begin{note}庚辰双行夹批:说得好痛快。\end{note}宝玉道:“我过那里去?”\begin{note}庚辰双行夹批:问得更好。\end{note}袭人冷笑道:“你问我,\begin{note}庚辰侧批:三字如闻。\end{note}我知道?你爱往那里去,就往那里去。从今咱们两个丢开手,省得鸡声鹅斗,叫别人笑。横竖那边腻了过来,这边又有个什么‘四儿’‘五儿’伏侍。我们这起东西,可是‘白玷辱了好名好姓’的。”宝玉笑道:“你今儿还记著呢!”\begin{note}庚辰双行夹批:非浑一纯粹,那能至此!\end{note}袭人道:“一百年还记著呢!比不得你,拿著我的话当耳旁风,夜里说了,早起就忘了。”\begin{note}庚辰双行夹批:这方是正文,直勾起“花解语”一回文字。\end{note}宝玉见他娇嗔满面,情不可禁,\begin{note}庚辰侧批:又用幻笔瞒过看官。\end{note}便向枕边拿起一根玉簪来,一跌两段,说道:“我再不听你说,就同这个一样。”\begin{note}蒙侧批:迎头一棒!\end{note}袭人忙的拾了簪子,说道:“大清早起,这是何苦来!听不听什么要紧,\begin{note}庚辰侧批:已留后文地步。\end{note}也值得这种样子。”宝玉道:“你那里知道我心里急!”袭人笑道:\begin{note}庚辰双行夹批:自此方笑。\end{note}“你也知道著急么!可知我心里怎么著?快起来洗脸去罢。”\begin{note}庚辰侧批:结得一星渣滓全无,且合怡红常事。\end{note}说著,二人方起来梳洗。
\end{parag}


\begin{parag}
    宝玉往上房去后,谁知黛玉走来,见宝玉不在房中,因翻弄案上书看,可巧翻出昨儿的《庄子》来。看至所续之处,不觉又气又笑,不禁也提笔续书一绝云:
\end{parag}


\begin{poem}
    \begin{pl}无端弄笔是何人?作践南华《庄子因》。\end{pl}

    \begin{pl}不悔自己无见识,却将丑语怪他人。\end{pl}
    \begin{note}庚辰侧批:不用宝玉见此诗,若长若短亦是大手法。庚辰双行夹批:骂得痛快,非颦儿不可。真好颦儿,真好颦儿!好诗!若云知音者颦儿也。至此方完“箴玉”半回。庚辰眉批:又借阿颦诗自相鄙驳,可见余前批不谬。己卯冬夜。庚辰眉批:宝玉不见诗,是后文余步也,《石头记》得力所在。丁亥夏。 笏叟。\end{note}
\end{poem}


\begin{parag}
    写毕,也往上房来见贾母,后往王夫人处来。
\end{parag}


\begin{parag}
    谁知凤姐之女大姐病了,正乱著请大夫来诊脉。大夫便说:“替夫人奶奶们道喜,姐儿发热是见喜了,并非别病。”王夫人凤姐听了,忙遣人问:“可好不好?”医生回道:“病虽险,却顺,\begin{note}庚辰侧批:在“子嗣艰难”化出。\end{note}倒还不妨。预备桑虫猪尾要紧。”凤姐听了,登时忙将起来:一面打扫房屋供奉痘疹娘娘,一面传与家人忌煎炒等物,一面命平儿打点铺盖衣服与贾琏隔房,一面又拿大红尺头与奶子丫头亲近人等裁衣。\begin{note}庚辰双行夹批:几个“一面”,写得如见其景。\end{note}外面又打扫净室,款留两个医生,轮流斟酌诊脉下药,十二日不放家去。贾琏只得搬出外书房来斋戒,\begin{note}庚辰侧批:此二字内生出许多事来。\end{note}凤姐与平儿都随著王夫人日日供奉娘娘。
\end{parag}


\begin{parag}
    那个贾琏,只离了凤姐便要寻事,独寝了两夜,便十分难熬,便暂将小厮们内有清俊的选来出火。不想荣国府内有一个极不成器破烂酒头厨子,名叫多官,\begin{note}庚辰双行夹批:今是多多也,妙名!\end{note}人见他懦弱无能,都唤他作“多浑虫”。\begin{note}庚辰双行夹批:更好!今之浑虫更多也。\end{note}因他自小父母替他在外娶了一个媳妇,今年方二十来往年纪,生得有几分人才,见者无不羡爱。他生性轻浮,最喜拈花惹草,多浑虫又不理论,只是有酒有肉有钱,便诸事不管了,所以荣宁二府之人都得入手。因这个媳妇美貌异常,轻浮无比,众人都呼他作“多姑娘儿”。\begin{note}庚辰双行夹批:更妙!\end{note}如今贾琏在外熬煎,往日也曾见过这媳妇,失过魂魄,只是内惧娇妻,外惧娈宠,不曾下得手。那多姑娘儿也曾有意于贾琏,只恨没空。今闻贾琏挪在外书房来,他便没事也要走两趟去招惹。惹的贾琏似饥鼠一般,少不得和心腹的小厮们计议,合同遮掩谋求,多以金帛相许。小厮们焉有不允之理,况都和这媳妇是好友,一说便成。是夜二鼓人定,多浑虫醉昏在炕,贾琏便溜了来相会。进门一见其态,早已魄飞魂散,也不用情谈款叙,便宽衣动作起来。谁知这媳妇有天生的奇趣,一经男子挨身,便觉遍身筋骨瘫软,\begin{note}庚辰双行夹批:淫极!亏想的出!\end{note}使男子如卧绵上,\begin{note}庚辰双行夹批:如此境界,自胜西方、蓬莱等处。\end{note}更兼淫态\begin{note}庚辰双行夹批:总为后文宝玉一篇作引。\end{note}浪言,压倒娼妓,诸男子至此岂有惜命者哉。\begin{note}庚辰侧批:凉水灌顶之句。\end{note}那贾琏恨不得连身子化在他身上。\begin{note}庚辰双行夹批:亲极之语,趣极之语。\end{note}那媳妇故作浪语,在下说道: “你家女儿出花儿,供著娘娘,你也该忌两日,倒为我脏了身子。快离了我这里罢。”\begin{note}庚辰侧批:淫妇勾人,惯加反语,看官著眼。\end{note}贾琏一面大动,一面喘吁吁答道:“你就是娘娘!我那里管什么娘娘!”\begin{note}庚辰侧批:乱语不伦,的是有之。\end{note}那媳妇越浪,贾琏越丑态毕露。\begin{note}蒙双行夹批:可以喷饭!\end{note}一时事毕,两个又海誓山盟,难分难舍,\begin{note}庚辰双行夹批:著眼,再从前看如何光景。\end{note}此后遂成相契。\begin{note}庚辰双行夹批:趣闻!“相契”作如此用,“相契”扫地矣。庚辰眉批:一部书中,只有此一段丑极太露之文,写于贾琏身上,恰极当极!己卯冬夜。\end{note}\begin{note}庚辰眉批:看官熟思:写珍、琏辈当以何等文方妥方恰也?壬午孟夏。\end{note}\begin{note}庚辰眉批:此段系书中情之瘕疵,写为阿凤生日泼醋回及“夭风流”宝玉悄看晴雯回作引,伏线千里外之笔也。丁亥夏。畸笏。\end{note}
\end{parag}


\begin{parag}
    一日大姐毒尽癍回,\begin{note}庚辰侧批:好快日子吓!\end{note}十二日后送了娘娘,合家祭天祀祖,还愿焚香,庆贺放赏已毕,贾琏仍复搬进卧室。见了凤姐,正是俗语云“新婚不如远别”,更有无限恩爱,自不必烦絮。\begin{note}庚辰侧批:隐得好。\end{note}
\end{parag}


\begin{parag}
    次日早起,凤姐往上屋去后,平儿收拾贾琏在外的衣服铺盖,不承望枕套中抖出一绺青丝来。平儿会意,忙拽在袖内,\begin{note}庚辰双行夹批:好极!不料平儿大有袭卿之身分,可谓何地无材,盖遭际有别耳。\end{note}便走至这边房内来,拿出头发来,向贾琏笑道:“这是什么?”\begin{note}庚辰双行夹批:好看之极!\end{note}贾琏看见著了忙,\begin{note}庚辰批:也有今日。\end{note}抢上来要夺。平儿便跑,被贾琏一把揪住,按在炕上,掰手要夺,口内笑道:“小蹄子,你不趁早拿出来,我把你膀子橛折了。”\begin{note}庚辰侧批:无情太甚!\end{note}平儿笑道:“你就是没良心的。我好意瞒著他来问,你倒赌狠!你只赌狠,等他回来我告诉他,\begin{note}庚辰侧批:有是语,恐卿口不应。\end{note}看你怎么著。”贾琏听说,忙陪笑央求道:“好人,赏我罢,我再不赌狠了。”\begin{note}庚辰双行夹批:好听好看之极,迥不犯袭卿。\end{note}
\end{parag}


\begin{parag}
    一语未了,只听凤姐声音进来。\begin{note}庚辰侧批:《石头记》大法小法累累如是,并不为厌。惊天骇地之文!如何?不知下文怎样了结,使贾琏及观者一齐丧胆。\end{note}贾琏听见松了手,平儿刚起身,凤姐已走进来,命平儿快开匣子,替太太找样子。平儿忙答应了找时,凤姐见了贾琏,忽然想起来,便问平儿:“拿出去的东西都收进来了么?”平儿道:“收进来了。”凤姐道:“可少什么没有?”平儿道:“我也怕丢下一两件,细细的查了查,也不少。”凤姐道:“不少就好,只是别多出来罢?”\begin{note}庚辰侧批:看至此,宁不拍案叫绝?庚辰双行夹批:奇!\end{note}平儿笑道:“不丢万幸,谁还添出来呢?”\begin{note}庚辰侧批:可儿可儿,卿亦明知故说耳。\end{note}凤姐冷笑道:“这半个月难保干净,或者有相厚的丢下的东西:戒指、汗巾、香袋儿,再至于头发、指甲,都是东西。”\begin{note}庚辰双行夹批:好阿凤,令人胆寒。\end{note}一席话,说的贾琏脸都黄了。贾琏在凤姐身后,只望著平儿杀鸡抹脖使眼色儿。\begin{note}蒙侧批:作丈夫者,要当自重!\end{note}平儿只装著看不见,\begin{note}庚辰侧批:余自有三分主意。\end{note}因笑道:“怎么我的心就和奶奶的心一样!我就怕有这些个,留神搜了一搜,竟一点破绽也没有。奶奶不信时,那些东西我还没收呢,奶奶亲自翻寻一遍去。”\begin{note}庚辰双行夹批:好平儿!遍天下惧内者来感谢。\end{note}凤姐笑道:“傻丫头,\begin{note}庚辰双行夹批:可叹可笑,竟不知谁傻。\end{note}他便有这些东西,那里就叫咱们翻著了!”\begin{note}庚辰双行夹批:好阿凤,好文字,虽系闺中女儿口角小事,读之不无聪明得失痴心真假之感。\end{note}说著,寻了样子又上去了。
\end{parag}


\begin{parag}
    平儿指著鼻子,\begin{note}庚辰侧批:好看煞。\end{note}晃著头笑道:\begin{note}庚辰侧批:可儿,可儿。\end{note}“这件事怎么回谢我呢?”\begin{note}庚辰双行夹批:姣俏如见,迥不犯袭卿麝月一笔。\end{note}喜的个贾琏身痒难挠,\begin{note}庚辰侧批:不但贾兄痒痒,即批书人此刻几乎落笔。试部看官此际若何光景?\end{note}跑上来搂著,“心肝肠肉”乱叫乱谢。平儿仍拿了头发笑道:“这是我一生的把柄了。好就好,不好就抖露出这事来。”贾琏笑道:“你只好生收著罢,千万别叫他知道。”口里说著,瞅他不防,便抢了过来,\begin{note}庚辰侧批:毕肖。琏兄不分玉石,但负我平姐。奈何,奈何!\end{note}笑道:“你拿著终是祸患,不如我烧了他完事了。”\begin{note}庚辰双行夹批:妙!设使平儿再不致泄露,故仍用贾琏抢回,后文遗失,过脉也。\end{note}一面说著,一面便塞于靴掖内。平儿咬牙道:“没良心的东西,过了河就拆桥,明儿还想我替你撒谎!”贾琏见他娇俏动情,便搂著求欢,被平儿夺手跑了,急的贾琏弯著腰恨道:“死促狭小淫妇!一定浪上人的火来,他又跑了。”\begin{note}庚辰双行夹批:丑态如见,淫声如闻,今古淫书未有之章法。\end{note}平儿在窗外笑道:“我浪我的,谁叫你动火了?\begin{note}庚辰双行夹批:妙极之谈。直是理学工夫,所谓不可正照风月鉴也。\end{note}难道图你\begin{note}庚辰侧批:阿平,“你” 字作牵强,余不画押。一笑。\end{note}受用一回,叫他知道了,又不待见我。”\begin{note}庚辰双行夹批:凤姐醋妒,于平儿前犹如是,况他人乎!余谓凤姐必是甚于诸人。观者不信,今平儿说出,然乎?否乎?\end{note}贾琏道:“你不用怕他,等我性子上来,把这醋罐打个稀烂,他才认得我呢!他防我象防贼的,只许他同男人说话,不许我和女人说话,我和女人略近些,他就疑惑,他不论小叔子侄儿,大的小的,说说笑笑,就不怕我吃醋了。\begin{note}蒙侧批:作者又何必如此想?亦犯此病也!\end{note}以后我也不许他见人!”\begin{note}庚辰双行夹批:无理之甚,却是妙极趣谈,天下惧内者背后之谈皆如此。\end{note}平儿道:“他醋你使得,你醋他使不得。他原行的正走的正,你行动便有个坏心,连我也不放心,别说他了。”贾琏道:“你两个一口贼气。都是你们行的是,我凡行动都存坏心。\begin{note}蒙侧批:一片俗气!\end{note}多早晚都死在我手里!”
\end{parag}


\begin{parag}
    一句未了,凤姐走进院来,因见平儿在窗外,就问道:“要说话两个人不在屋里说,怎么跑出一个来,隔著窗子,是什么意思?”贾琏在窗内接道:“你可问他,倒象屋里有老虎吃他呢。”\begin{note}庚辰双行夹批:好!庚辰眉批:此等章法是在戏场上得来,一笑。畸笏。\end{note}平儿道:“屋里一个人没有,我在他跟前作什么?”凤姐儿笑道:“正是没人才好呢。”平儿听说,便说道:“这话是说我呢?”凤姐笑道:\begin{note}蒙双行夹批:“笑”字妙!平儿反正色,凤姐反陪笑,奇极意外之文。\end{note}“不说你说谁?”平儿道:“别叫我说出好话来了。”说著,也不打帘子让凤姐,自己先摔帘子进来,\begin{note}庚辰侧批:若在屋里,何敢如此形景,不要加上许多小心?平儿平儿,有你说嘴的。\end{note}往那边去了。凤姐自掀帘子进来,说道:“平儿疯魔了。这蹄子认真要降伏我,仔细你的皮要紧!”贾琏听了,已绝倒在炕上,\begin{note}庚辰侧批:惧内形景写尽了。\end{note}拍手笑道:“我竟不知平儿这么利害,从此倒伏他了。”凤姐道:“都是你惯的他,我只和你说!”贾琏听说忙道:“你两个不卯,又拿我来作人。我躲开你们。”凤姐道:“我看你躲到那里去。”\begin{note}蒙侧批:世俗之态熏人。\end{note}贾琏道:“我就来。”凤姐道:“我有话和你商量。”不知商量何事,且听下回分解。\begin{note}庚辰侧批:收得淡雅之至!\end{note}正是:
\end{parag}


\begin{poem}
    \begin{pl}淑女从来多抱怨,娇妻自古便含酸。\end{pl}
    \begin{note}庚辰双行夹批:二语包尽古今万世裙衩。\end{note}
\end{poem}


\begin{parag}
    \begin{note}蒙回末总评:不惜恩爱为良人,方是温存一脉真。俗子妒妇浑可笑,语言便自笑风尘。\end{note}
\end{parag}
