\chap{一百零九}{候芳魂五兒承錯愛 還孽債迎女返真元}



\begin{parag}
    話說寶釵叫襲人問出原故,恐寶玉悲傷成疾,便將黛玉臨死的話與襲人假作閒談,說是:“人生在世,有意有情,到了死後各自幹各自的去了,並不是生前那樣個人死後還是這樣。活人雖有癡心,死的竟不知道。況且林姑娘既說仙去,他看凡人是個不堪的濁物,那裏還肯混在世上。只是人自己疑心,所以招些邪魔外祟來纏擾了。”寶釵雖是與襲人說話,原說給寶玉聽的。襲人會意,也說是“沒有的事。若說林姑娘的魂靈兒還在園裏,我們也算好的,怎麼不曾夢見了一次。”寶玉在外聞聽得,細細的想道:“果然也奇。我知道林妹妹死了,那一日不想幾遍,怎麼從沒夢過。想是他到天上去了,瞧我這凡夫俗子不能交通神明,所以夢都沒有一個兒。我就在外間睡着,或者我從園裏回來,他知道我的實心,肯與我夢裏一見。我必要問他實在那裏去了,我也時常祭奠。若是果然不理我這濁物,竟無一夢,我便不想他了。”主意已定,便說:“我今夜就在外間睡了,你們也不用管我。”寶釵也不強他,只說:“你不要胡思亂想。你不瞧瞧,太太因你園裏去了急得話都說不出來。若是知道還不保養身子,倘或老太太知道了,又說我們不用心。”寶玉道:“白這麼說罷咧,我坐一會子就進來。你也乏了,先睡罷。”寶釵知他必進來的,假意說道:“我睡了,叫襲姑娘伺候你罷。”寶玉聽了,正合機宜。候寶釵睡了,他便叫襲人麝月另鋪設下一副被褥,常叫人進來瞧二奶奶睡着了沒有。寶釵故意裝睡,也是一夜不寧。那寶玉知是寶釵睡着,便與襲人道:“你們各自睡罷,我又不傷感。你若不信,你就伏侍我睡了再進去,只要不驚動我就是了。”襲人果然伏侍他睡下,便預備下了茶水,關好了門,進裏間去照應一回,各自假寐,寶玉若有動靜,再爲出來。寶玉見襲人等進來,便將坐更的兩個婆子支到外頭,他輕輕的坐起來,暗暗的祝了幾句,便睡下了,欲與神交。起初再睡不着,以後把心一靜,便睡去了。豈知一夜安眠,直到天亮。寶玉醒來,拭眼坐起來想了一回,並無有夢,便嘆口氣道:“正是‘悠悠生死別經年,魂魄不曾來入夢’。”寶釵卻一夜反沒有睡着,聽寶玉在外邊念這兩句,便接口道:“這句又說莽撞了,如若林妹妹在時,又該生氣了。”寶玉聽了,反不好意思,只得起來搭訕着往裏間走來,說:“我原要進來的,不覺得一個盹兒就打着了。”寶釵道:“你進來不進來與我什麼相干。”襲人等本沒有睡,眼見他們兩個說話,即忙倒上茶來。已見老太太那邊打發小丫頭來,問:“寶二爺昨睡得安頓麼?若安頓時,早早的同二奶奶梳洗了就過去。”襲人便說:“你去回老太太,說寶玉昨夜很安頓,回來就過來。”小丫頭去了。
\end{parag}


\begin{parag}
    寶釵起來梳洗了,鶯兒襲人等跟着先到賈母那裏行了禮,便到王夫人那邊起至鳳姐都讓過了,仍到賈母處,見他母親也過來了。大家問起:“寶玉晚上好麼?”寶釵便說:“回去就睡了,沒有什麼。”衆人放心,又說些閒話。只見小丫頭進來說:“二姑奶奶要回去了。聽見說孫姑爺那邊人來到大太太那裏說了些話,大太太叫人到四姑娘那邊說不必留了,讓他去罷。如今二姑奶奶在大太太那邊哭呢,大約就過來辭老太太。”賈母衆人聽了,心中好不自在,都說:“二姑娘這樣一個人,爲什麼命裏遭着這樣的人,一輩子不能出頭。這便怎麼好!”說着,迎春進來,淚痕滿面,因爲是寶釵的好日子,只得含着淚,辭了衆人要回去。賈母知道他的苦處,也不便強留,只說道:“你回去也罷了。但是不要悲傷,碰着了這樣人,也是沒法兒的。過幾天我再打發人接你去。”迎春道:“老太太始終疼我,如今也疼不來了。可憐我只是沒有再來的時候了。”說着,眼淚直流。衆人都勸道:“這有什麼不能回來的?比不得你三妹妹,隔得遠,要見面就難了。”賈母等想起探春,不覺也大家落淚,只爲是寶釵的生日,即轉悲爲喜說:“這也不難,只要海疆平靜,那邊親家調進京來,就見的着了。”大家說:“可不是這麼着呢。”說着,迎春只得含悲而別。衆人送了出來,仍回賈母那裏。從早至暮,又鬧了一天。
\end{parag}


\begin{parag}
    衆人見賈母勞乏,各自散了。獨有薛姨媽辭了賈母,到寶釵那裏,說道:“你哥哥是今年過了,直要等到皇恩大赦的時候減了等纔好贖罪。這幾年叫我孤苦伶仃怎麼處!我想要與你二哥哥完婚,你想想好不好?”寶釵道:“媽媽是爲着大哥哥娶了親唬怕的了,所以把二哥哥的事猶豫起來。據我說很該就辦。邢姑娘是媽媽知道的,如今在這裏也很苦,娶了去雖說我家窮,究竟比他傍人門戶好多着呢。”薛姨媽道:“你得便的時候就去告訴老太太,說我家沒人,就要揀日子了。”寶釵道:“媽媽只管同二哥哥商量,挑個好日子,過來和老太太,大太太說了,娶過去就完了一宗事。這裏大太太也巴不得娶了去纔好。”薛姨媽道:“今日聽見史姑娘也就回去了,老太太心裏要留你妹妹在這裏住幾天,所以他住下了。我想他也是不定多早晚就走的人了,你們姊妹們也多敘幾天話兒。”寶釵道:“正是呢。”於是薛姨媽又坐了一坐,出來辭了衆人回去了。
\end{parag}


\begin{parag}
    卻說寶玉晚間歸房,因想昨夜黛玉竟不入夢,“或者他已經成仙,所以不肯來見我這種濁人也是有的,不然就是我的性兒太急了,也未可知。”便想了個主意,向寶釵說道:“我昨夜偶然在外間睡着,似乎比在屋裏睡的安穩些,今日起來心裏也覺清靜些。我的意思還要在外間睡兩夜,只怕你們又來攔我。”寶釵聽了,明知早晨他嘴裏唸詩是爲着黛玉的事了。想來他那個呆性是不能勸的,倒好叫他睡兩夜,索性自己死了心也罷了,況兼昨夜聽他睡的倒也安靜,便道:“好沒來由,你只管睡去,我們攔你作什麼!但只不要胡思亂想,招出些邪魔外祟來。”寶玉笑道:“誰想什麼!”襲人道:“依我勸二爺竟還是屋裏睡罷,外邊一時照應不到,着了風倒不好。”寶玉未及答言,寶釵卻向襲人使了個眼色。襲人會意,便道:“也罷,叫個人跟着你罷,夜裏好倒茶倒水的。”寶玉便笑道:“這麼說,你就跟了我來。”襲人聽了倒沒意思起來,登時飛紅了臉,一聲也不言語。寶釵素知襲人穩重,便說道:“他是跟慣了我的,還叫他跟着我罷。叫麝月五兒照料着也罷了。況且今日他跟着我鬧了一天也乏了,該叫他歇歇了。”寶玉只得笑着出來。寶釵因命麝月五兒給寶玉仍在外間鋪設了,又囑咐兩個人醒睡些,要茶要水都留點神兒。
\end{parag}


\begin{parag}
    兩個答應着出來,看見寶玉端然坐在牀上,閉目合掌,居然象個和尚一般,兩個也不敢言語,只管瞅着他笑。寶釵又命襲人出來照應。襲人看見這般卻也好笑,便輕輕的叫道:“該睡了,怎麼又打起坐來了!”寶玉睜開眼看見襲人,便道:“你們只管睡罷,我坐一坐就睡。”襲人道:“因爲你昨日那個光景,鬧的二奶奶一夜沒睡。你再這麼着,成何事體。”寶玉料着自己不睡都不肯睡,便收拾睡下。襲人又囑咐了麝月等幾句,才進去關門睡了。這裏麝月五兒兩個人也收拾了被褥,伺候寶玉睡着,各自歇下。
\end{parag}


\begin{parag}
    那知寶玉要睡越睡不着,見他兩個人在那裏打鋪,忽然想起那年襲人不在家時晴雯麝月兩個人伏侍,夜間麝月出去,晴雯要唬他,因爲沒穿衣服着了涼,後來還是從這個病上死的。想到這裏,一心移在晴雯身上去了。忽又想起鳳姐說五兒給晴雯脫了個影兒,因又將想晴雯的心腸移在五兒身上。自己假裝睡着,偷偷的看那五兒,越瞧越象晴雯,不覺呆性復發。聽了聽,裏間已無聲息,知是睡了。卻見麝月也睡着了,便故意叫了麝月兩聲,卻不答應。五兒聽見寶玉喚人,便問道:“二爺要什麼?”寶玉道:“我要漱漱口。”五兒見麝月已睡,只得起來重新剪了蠟花,倒了一鍾茶來,一手託着漱盂。卻因趕忙起來的,身上只穿着一件桃紅綾子小襖兒,鬆鬆的挽着一個纂兒。寶玉看時,居然晴雯復生。忽又想起晴雯說的“早知擔個虛名,也就打個正經主意了”,不覺呆呆的呆看,也不接茶。
\end{parag}


\begin{parag}
    那五兒自從芳官去後,也無心進來了。後來聽見鳳姐叫他進來伏侍寶玉,竟比寶玉盼他進來的心還急。不想進來以後,見寶釵襲人一般尊貴穩重,看着心裏實在敬慕,又見寶玉瘋瘋傻傻,不似先前風致,又聽見王夫人爲女孩子們和寶玉頑笑都攆了:所以把這件事擱在心上,倒無一毫的兒女私情了。怎奈這位呆爺今晚把他當作晴雯,只管愛惜起來。那五兒早已羞得兩頰紅潮,又不敢大聲說話,只得輕輕的說道:“二爺漱口啊。”寶玉笑着接了茶在手中,也不知道漱了沒有,便笑嘻嘻的問道:“你和晴雯姐姐好不是啊?”五兒聽了摸不着頭腦,便道:“都是姐妹,也沒有什麼不好的。”寶玉又悄悄的問道:“晴雯病重了我看他去,不是你也去了麼?”五兒微微笑着點頭兒。寶玉道:“你聽見他說什麼了沒有?”五兒搖着頭兒道:“沒有。”寶玉已經忘神,便把五兒的手一拉。五兒急得紅了臉,心裏亂跳,便悄悄說道:“二爺有什麼話只管說,別拉拉扯扯的。”寶玉才放了手,說道:“他和我說來着,‘早知擔了個虛名,也就打正經主意了。’你怎麼沒聽見麼?”五兒聽了這話明明是輕薄自己的意思,又不敢怎麼樣,便說道:“那是他自己沒臉,這也是我們女孩兒家說得的嗎。”寶玉着急道:“你怎麼也是這麼個道學先生!我看你長的和他一模一樣,我才肯和你說這個話,你怎麼倒拿這些話來糟踏他!”此時五兒心中也不知寶玉是怎麼個意思,便說道:“夜深了,二爺也睡罷,別緊着坐着,看涼着。剛纔奶奶和襲人姐姐怎麼囑咐了?”寶玉道:“我不涼。”說到這裏,忽然想起五兒沒穿着大衣服,就怕他也象晴雯着了涼,便說道:“你爲什麼不穿上衣服就過來!”五兒道:“爺叫的緊,那裏有盡着穿衣裳的空兒。要知道說這半天話兒時,我也穿上了。”寶玉聽了,連忙把自己蓋的一件月白綾子綿襖兒揭起來遞給五兒,叫他披上。五兒只不肯接,說:“二爺蓋着罷,我不涼。我涼我有我的衣裳。”說着,回到自己鋪邊,拉了一件長襖披上。又聽了聽,麝月睡的正濃,才慢慢過來說:“二爺今晚不是要養神呢嗎?”寶玉笑道:“實告訴你罷,什麼是養神,我倒是要遇仙的意思。”五兒聽了,越發動了疑心,便問道:“遇什麼仙?”寶玉道:“你要知道,這話長着呢。你挨着我來坐下,我告訴你。”五兒紅了臉笑道:“你在那裏躺着,我怎麼坐呢。”寶玉道:“這個何妨。那一年冷天,也是你麝月姐姐和你晴雯姐姐頑,我怕凍着他,還把他攬在被裏渥着呢。這有什麼的!大凡一個人總不要酸文假醋纔好。”五兒聽了,句句都是寶玉調戲之意。那知這位呆爺卻是實心實意的話兒。五兒此時走開不好,站着不好,坐下不好,倒沒了主意了,因微微的笑着道:“你別混說了,看人家聽見這是什麼意思。怨不得人家說你專在女孩兒身上用工夫,你自己放着二奶奶和襲人姐姐都是仙人兒似的,只愛和別人胡纏。明兒再說這些話,我回了二奶奶,看你什麼臉見人。”正說着,只聽外面咕咚一聲,把兩個人嚇了一跳。裏間寶釵咳嗽了一聲。寶玉聽見,連忙呶嘴兒。五兒也就忙忙的息了燈悄悄的躺下了。原來寶釵襲人因昨夜不曾睡,又兼日間勞乏了一天,所以睡去,都不曾聽見他們說話。此時院中一響,早已驚醒,聽了聽,也無動靜。寶玉此時躺在牀上,心裏疑惑:“莫非林妹妹來了,聽見我和五兒說話故意嚇我們的?”翻來覆去,胡思亂想,五更以後,才朦朧睡去。
\end{parag}


\begin{parag}
    卻說五兒被寶玉鬼混了半夜,又兼寶釵咳嗽,自己懷着鬼胎,生怕寶釵聽見了,也是思前想後,一夜無眠。次日一早起來,見寶玉尚自昏昏睡着,便輕輕的收拾了屋子。那時麝月已醒,便道:“你怎麼這麼早起來了,你難道一夜沒睡嗎?”五兒聽這話又似麝月知道了的光景,便只是訕笑,也不答言。不一時,寶釵襲人也都起來,開了門見寶玉尚睡,卻也納悶:“怎麼外邊兩夜睡得倒這般安穩?”及寶玉醒來,見衆人都起來了,自己連忙爬起,揉着眼睛,細想昨夜又不曾夢見,可是仙凡路隔了。慢慢的下了牀,又想昨夜五兒說的寶釵襲人都是天仙一般,這話卻也不錯,便怔怔的瞅着寶釵。寶釵見他發怔,雖知他爲黛玉之事,卻也定不得夢不夢,只是瞅的自己倒不好意思,便道:“二爺昨夜可真遇見仙了麼?”寶玉聽了,只道昨晚的話寶釵聽見了,笑着勉強說道:“這是那裏的話!”那五兒聽了這一句,越發心虛起來,又不好說的,只得且看寶釵的光景。只見寶釵又笑着問五兒道:“你聽見二爺睡夢中和人說話來着麼?”寶玉聽了,自己坐不住,搭訕着走開了。五兒把臉飛紅,只得含糊道:“前半夜倒說了幾句,我也沒聽真。什麼‘擔了虛名’,又什麼‘沒打正經主意’,我也不懂,勸着二爺睡了,後來我也睡了,不知二爺還說來着沒有。”寶釵低頭一想:“這話明是爲黛玉了。但盡着叫他在外頭,恐怕心邪了招出些花妖月姊來。況兼他的舊病原在姊妹上情重,只好設法將他的心意挪移過來,然後能免無事。”想到這裏,不免面紅耳熱起來,也就訕訕的進房梳洗去了。
\end{parag}


\begin{parag}
    且說賈母兩日高興,略喫多了些,這晚有些不受用,第二天便覺着胸口飽悶。鴛鴦等要回賈政。賈母不叫言語,說:“我這兩日嘴饞些喫多了點子,我餓一頓就好了。你們快別吵嚷。”於是鴛鴦等並沒有告訴人。
\end{parag}


\begin{parag}
    這日晚間,寶玉回到自己屋裏,見寶釵自賈母王夫人處才請了晚安回來。寶玉想着早起之事,未免赧顏抱慚。寶釵看他這樣,也曉得是個沒意思的光景,因想着:“他是個癡情人,要治他的這病,少不得仍以癡情治之。”想了一回,便問寶玉道:“你今夜還在外間睡去罷咧?”寶玉自覺沒趣,便道:“裏間外間都是一樣的。”寶釵意欲再說,反覺不好意思。襲人道:“罷呀,這倒是什麼道理呢。我不信睡得那麼安穩!”五兒聽見這話,連忙接口道:“二爺在外間睡,別的倒沒什麼,只是愛說夢話,叫人摸不着頭腦兒,又不敢駁他的回。”襲人便道:“我今日挪到牀上睡睡,看說夢話不說?你們只管把二爺的鋪蓋鋪在裏間就完了。”寶釵聽了,也不作聲。寶玉自己慚愧不來,那裏還有強嘴的分兒,便依着搬進裏間來。一則寶玉負愧,欲安慰寶釵之心,二則寶釵恐寶玉思鬱成疾,不如假以詞色,使得稍覺親近,以爲移花接木之計。於是當晚襲人果然挪出去。寶玉因心中愧悔,寶釵欲攏絡寶玉之心,自過門至今日,方纔如魚得水,恩愛纏綿,所謂二五之精妙合而凝的了。此是後話。
\end{parag}


\begin{parag}
    且說次日寶玉寶釵同起,寶玉梳洗了先過賈母這邊來。這裏賈母因疼寶玉,又想寶釵孝順,忽然想起一件東西,便叫鴛鴦開了箱子,取出祖上所遺一個漢玉玦,雖不及寶玉他那塊玉石,掛在身上卻也稀罕。鴛鴦找出來遞與賈母,便說道:“這件東西我好象從沒見的,老太太這些年還記得這樣清楚,說是那一箱什麼匣子裏裝着,我按着老太太的話一拿就拿出來了。老太太怎麼想着拿出來做什麼?”賈母道:“你那裏知道,這塊玉還是祖爺爺給我們老太爺,老太爺疼我,臨出嫁的時候叫了我去親手遞給我的。還說:‘這玉是漢時所佩的東西,很貴重,你拿着就象見了我的一樣。’我那時還小,拿了來也不當什麼,便撩在箱子裏。到了這裏,我見咱們家的東西也多,這算得什麼,從沒帶過,一撩便撩了六十多年。今兒見寶玉這樣孝順,他又丟了一塊玉,故此想着拿出來給他,也象是祖上給我的意思。”一時寶玉請了安,賈母便喜歡道:“你過來,我給你一件東西瞧瞧。”寶玉走到牀前,賈母便把那塊漢玉遞給寶玉。寶玉接來一瞧,那玉有三寸方圓,形似甜瓜,色有紅暈,甚是精緻。寶玉口口稱讚。賈母道:“你愛麼?這是我祖爺爺給我的,我傳了你罷。”寶玉笑着請了個安謝了,又拿了要送給他母親瞧。賈母道:“你太太瞧了告訴你老子,又說疼兒子不如疼孫子了。他們從沒見過。”寶玉笑着去了。寶釵等又說了幾句話,也辭了出來。自此賈母兩日不進飲食,胸口仍是結悶,覺得頭暈目眩,咳嗽。邢王二夫人鳳姐等請安,見賈母精神尚好,不過叫人告訴賈政,立刻來請了安。賈政出來,即請大夫看脈。不多一時,大夫來診了脈,說是有年紀的人停了些飲食,感冒些風寒,略消導發散些就好了。開了方子,賈政看了,知是尋常藥品,命人煎好進服。以後賈政早晚進來請安,一連三日,不見稍減。賈政又命賈璉:“打聽好大夫,快去請來瞧老太太的病。咱們家常請的幾個大夫,我瞧着不怎麼好,所以叫你去。”賈璉想了一想,說道:“記得那年寶兄弟病的時候,倒是請了一個不行醫的來瞧好了的,如今不如找他。”賈政道:“醫道卻是極難的,愈是不興時的大夫倒有本領。你就打發人去找來罷。”賈璉即忙答應去了,回來說道:“這劉大夫新近出城教書去了,過十來天進城一次。這時等不得,又請了一位,也就來了。”賈政聽了,只得等着。不題。
\end{parag}


\begin{parag}
    且說賈母病時,合宅女眷無日不來請安。一日,衆人都在那裏,只見看園內腰門的老婆子進來,回說:“園裏的櫳翠庵的妙師父知道老太太病了,特來請安。”衆人道:“他不常過來,今兒特地來,你們快請進來。”鳳姐走到牀前回賈母。岫煙是妙玉的舊相識,先走出去接他。只見妙玉頭帶妙常髻,身上穿一件月白素綢襖兒,外罩一件水田青緞鑲邊長背心,拴着秋香色的絲絛,腰下系一條淡墨畫的白綾裙,手執麈尾念珠,跟着一個侍兒,飄飄拽拽的走來。岫煙見了問好,說是“在園內住的日子,可以常常來瞧瞧你。近來因爲園內人少,一個人輕易難出來。況且咱們這裏的腰門常關着,所以這些日子不得見你。今兒幸會。”妙玉道:“頭裏你們是熱鬧場中,你們雖在外園裏住,我也不便常來親近。如今知道這裏的事情也不大好,又聽說是老太太病着,又掂記你,並要瞧瞧寶姑娘。我那管你們的關不關,我要來就來,我不來你們要我來也不能啊。”岫煙笑道:“你還是那種脾氣。”一面說着,已到賈母房中。衆人見了都問了好。妙玉走到賈母牀前問候,說了幾句套話。賈母便道:“你是個女菩薩,你瞧瞧我的病可好得了好不了?”妙玉道:“老太太這樣慈善的人,壽數正有呢。一時感冒,喫幾貼藥想來也就好了。有年紀人只要寬心些。”賈母道:“我倒不爲這些,我是極愛尋快樂的。如今這病也不覺怎樣,只是胸隔悶飽,剛纔大夫說是氣惱所致。你是知道的,誰敢給我氣受,這不是那大夫脈理平常麼。我和璉兒說了,還是頭一個大夫說感冒傷食的是,明兒仍請他來。”說着,叫鴛鴦吩咐廚房裏辦一桌淨素菜來,請他在這裏便飯。妙玉道:“我已喫過午飯了,我是不喫東西的。”王夫人道:“不喫也罷,咱們多坐一會說些閒話兒罷。”妙玉道:“我久已不見你們,今兒來瞧瞧。”又說了一回話便要走,回頭見惜春站着,便問道:“四姑娘爲什麼這樣瘦?不要只管愛畫勞了心。”惜春道:“我久不畫了。如今住的房屋不比園裏的顯亮,所以沒興畫。”妙玉道:“你如今住在那一所了?”惜春道:“就是你才進來的那個門東邊的屋子。你要來很近。”妙玉道:“我高興的時候來瞧你。”惜春等說着送了出去,回身過來,聽見丫頭們回說大夫在賈母那邊呢。衆人暫且散去。
\end{parag}


\begin{parag}
    那知賈母這病日重一日,延醫調治不效,以後又添腹瀉。賈政着急,知病難醫,即命人到衙門告假,日夜同王夫人親視湯藥。一日,見賈母略進些飲食,心裏稍寬。只見老婆子在門外探頭,王夫人叫彩雲看去,問問是誰。彩雲看了是陪迎春到孫家去的人,便道:“你來做什麼?”婆子道:“我來了半日,這裏找不着一個姐姐們,我又不敢冒撞,我心裏又急。”彩雲道:“你急什麼?又是姑爺作踐姑娘不成麼?”婆子道:“姑娘不好了。前兒鬧了一場,姑娘哭了一夜,昨日痰堵住了。他們又不請大夫,今日更利害了。”彩雲道:“老太太病着呢,別大驚小怪的。”王夫人在內已聽見了,恐老太太聽見不受用,忙叫彩雲帶他外頭說去。豈知賈母病中心靜,偏偏聽見,便道:“迎丫頭要死了麼?”王夫人便道:“沒有。婆子們不知輕重,說是這兩日有些病,恐不能就好,到這裏問大夫。”賈母道:“瞧我的大夫就好,快請了去。”王夫人便叫彩雲叫這婆子去回大太太去,那婆子去了。這裏賈母便悲傷起來,說是:“我三個孫女兒,一個享盡了福死了,三丫頭遠嫁不得見面,迎丫頭雖苦,或者熬出來,不打量他年輕輕兒的就要死了。留着我這麼大年紀的人活着做什麼!”王夫人鴛鴦等解勸了好半天。那時寶釵李氏等不在房中,鳳姐近來有病,王夫人恐賈母生悲添病,便叫人叫了他們來陪着,自己回到房中,叫彩雲來埋怨這婆子不懂事,”以後我在老太太那裏,你們有事不用來回。”丫頭們依命不言。豈知那婆子剛到邢夫人那裏,外頭的人已傳進來說:“二姑奶奶死了。”邢夫人聽了,也便哭了一場。現今他父親不在家中,只得叫賈璉快去瞧看。知賈母病重,衆人都不敢回。可憐一位如花似月之女,結褵年餘,不料被孫家揉搓以致身亡。又值賈母病篤,衆人不便離開,竟容孫家草草完結。
\end{parag}


\begin{parag}
    賈母病勢日增,只想這些好女兒。一時想起湘雲,便打發人去瞧他。回來的人悄悄的找鴛鴦,因鴛鴦在老太太身旁,王夫人等都在那裏,不便上去,到了後頭找了琥珀,告訴他道:“老太太想史姑娘,叫我們去打聽。那裏知道史姑娘哭得了不得,說是姑爺得了暴病,大夫都瞧了,說這病只怕不能好,若變了個癆病,還可捱過四五年。所以史姑娘心裏着急。又知道老太太病,只是不能過來請安,還叫我不要在老太太面前提起。倘或老太太問起來,務必託你們變個法兒回老太太纔好。”琥珀聽了,咳了一聲,就也不言語了,半日說道:“你去罷。”琥珀也不便回,心裏打算告訴鴛鴦,叫他撒謊去,所以來到賈母牀前,只見賈母神色大變,地下站着一屋子的人,嘁嘁的說“瞧着是不好了”,也不敢言語了。這裏賈政悄悄的叫賈璉到身旁,向耳邊說了幾句話。賈璉輕輕的答應出去了,便傳齊了現在家的一干家人說:“老太太的事待好出來了,你們快快分頭派人辦去。頭一件先請出板來瞧瞧,好掛裏子。快到各處將各人的衣服量了尺寸,都開明瞭,便叫裁縫去做孝衣。那棚槓執事都去講定。廚房裏還該多派幾個人。”賴大等回道:“二爺,這些事不用爺費心,我們早打算好了。只是這項銀子在那裏打算?”賈璉道:“這種銀子不用打算了,老太太自己早留下了。剛纔老爺的主意只要辦的好,我想外面也要好看。”賴大等答應,派人分頭辦去。
\end{parag}


\begin{parag}
    賈璉復回到自己房中,便問平兒:“你奶奶今兒怎麼樣?”平兒把嘴往裏一努說:“你瞧去。”賈璉進內,見鳳姐正要穿衣,一時動不得,暫且靠在炕桌兒上。賈璉道:“你只怕養不住了。老太太的事今兒明兒就要出來了,你還脫得過麼。快叫人將屋裏收拾收拾就該扎掙上去了。若有了事,你我還能回來麼。”鳳姐道:“咱們這裏還有什麼收拾的,不過就是這點子東西,還怕什麼!你先去罷,看老爺叫你。我換件衣裳就來。”
\end{parag}


\begin{parag}
    賈璉先回到賈母房裏,向賈政悄悄的回道:“諸事已交派明白了。”賈政點頭。外面又報太醫進來了,賈璉接入,又診了一回,出來悄悄的告訴賈璉:“老太太的脈氣不好,防着些。”賈璉會意,與王夫人等說知。王夫人即忙使眼色叫鴛鴦過來,叫他把老太太的裝裹衣服預備出來。鴛鴦自去料理。賈母睜眼要茶喝,邢夫人便進了一杯蔘湯。賈母剛用嘴接着喝,便道:“不要這個,倒一鍾茶來我喝。”衆人不敢違拗,即忙送上來,一口喝了,還要,又喝一口,便說:“我要坐起來。”賈政等道:“老太太要什麼只管說,可以不必坐起來纔好。”賈母道:“我喝了口水,心裏好些,略靠着和你們說說話。”珍珠等用手輕輕的扶起,看見賈母這回精神好些。未知生死,下回分解。
\end{parag}