\chap{五十一}{薛小妹新编怀古诗 胡庸医乱用虎狼药}
\begin{parag}
    \begin{note}蒙回前总批:文有一语写出大景者,如“园中不见一女子”句,俨然大家规模。“疑是姑娘”一语,又俨然庸医口角,新医行径。笔大如椽。\end{note}
\end{parag}


\begin{parag}
    众人闻得宝琴将素习所经过各省内的古迹为题,作了十首怀古绝句,内隐十物,皆说这自然新巧。都争著看时,只见写道是:
\end{parag}


\begin{poem}
    \begin{pl}赤壁怀古 其一
    \end{pl}
    \begin{pl}赤壁沉埋水不流,徒留名姓载空舟。
    \end{pl}
    \begin{pl}喧阗一炬悲风冷,无限英魂在内游。
    \end{pl}
    \emptypl

    \begin{pl}交趾怀古 其二
    \end{pl}
    \begin{pl}铜铸金镛振纪纲,声传海外播戎羌。
    \end{pl}
    \begin{pl}马援自是功劳大,铁笛无烦说子房。
    \end{pl}
    \emptypl

    \begin{pl}钟山怀古 其三
    \end{pl}
    \begin{pl}名利何曾伴汝身,无端被诏出凡尘。
    \end{pl}
    \begin{pl}牵连大抵难休绝,莫怨他人嘲笑频。
    \end{pl}
    \emptypl
    \begin{pl}淮阴怀古 其四
    \end{pl}
    \begin{pl}壮士须防恶犬欺,三齐位定盖棺时。
    \end{pl}
    \begin{pl}寄言世俗休轻鄙,一饭之恩死也知。
    \end{pl}
    \emptypl

    \begin{pl}广陵怀古 其五
    \end{pl}
    \begin{pl}蝉噪鸦栖转眼过,隋堤风景近如何。
    \end{pl}
    \begin{pl}只缘占得风流号,惹得纷纷口舌多。
    \end{pl}
    \emptypl

    \begin{pl}桃叶渡怀古 其六
    \end{pl}
    \begin{pl}衰草闲花映浅池,桃枝桃叶总分离。
    \end{pl}
    \begin{pl}六朝梁栋多如许,小照空悬壁上题。
    \end{pl}
    \emptypl
    \begin{pl}青冢怀古 其七
    \end{pl}
    \begin{pl}黑水茫茫咽不流,冰弦拨尽曲中愁。
    \end{pl}
    \begin{pl}汉家制度诚堪叹,樗栎应惭万古羞。
    \end{pl}
    \emptypl

    \begin{pl}马嵬怀古 其八
    \end{pl}
    \begin{pl}寂寞脂痕渍汗光,温柔一旦付东洋。
    \end{pl}
    \begin{pl}只因遗得风流迹,此日衣衾尚有香。
    \end{pl}
    \emptypl
    \begin{pl}蒲东寺怀古 其九
    \end{pl}
    \begin{pl}小红骨贱最身轻,私掖偷携强撮成。
    \end{pl}
    \begin{pl}虽被夫人时吊起,已经勾引彼同行。
    \end{pl}
    \emptypl
    \begin{pl}梅花观怀古 其十
    \end{pl}
    \begin{pl}不在梅边在柳边,个中谁拾画婵娟。
    \end{pl}
    \begin{pl}团圆莫忆春香到,一别西风又一年。
    \end{pl}
\end{poem}


\begin{parag}
    众人看了,都称奇道妙。宝钗先说道:“前八首都是史鉴上有据的;后二首却无考,我们也不大懂得,不如另作两首为是。”\begin{note}庚辰双行夹批:如何必得宝钗此驳方是好文,后文若真另作亦必无趣,若不另作,又有何法省之,看他下文如何。\end{note}黛玉忙拦道:\begin{note}庚辰双行夹批:好极!非黛玉不可。脂砚。\end{note}“这宝姐姐也忒 ‘胶柱鼓瑟’,矫揉造作了。这两首虽于史鉴上无考,咱们虽不曾看这些外传,不知底里,难道咱们连两本戏也没有见过不成?那三岁孩子也知道,何况咱们?”探春便道:“这话正是了。”\begin{note}庚辰双行夹批:余谓颦儿必有尖语来讽,不望竟有此饰词代为解释,此则真心以待宝钗也。\end{note}李纨又道:“况且他原是到过这个地方的。这两件事虽无考,古往今来,以讹传讹,好事者竟故意的弄出这古迹来以愚人。比如那年上京的时节,单是关夫子的坟,倒见了三四处。关夫子一生事业,皆是有据的,如何又有许多的坟?自然是后来人敬爱他生前为人,只怕从这敬爱上穿凿出来,也是有的。及至看《广舆记》上,不止关夫子的坟多,自古来有些名望的人,坟就不少,无考的古迹更多。如今这两首虽无考,凡说书唱戏,甚至于求的签上皆有注批,老小男女,俗语口头,人人皆知皆说的。况且又并不是看了《西厢》《牡丹》的词曲,怕看了邪书。这竟无妨,只管留著。”宝钗听说,方罢了。\begin{note}庚辰双行夹批:此为三染无痕也,妙极!天花无缝之文。\end{note}大家猜了一回,皆不是。
\end{parag}


\begin{parag}
    冬日天短,不觉又是前头吃晚饭之时,一齐前来吃饭。因有人回王夫人说:“袭人的哥哥花自芳进来说,他母亲病重了,想他女儿。他来求恩典,接袭人家去走走。”王夫人听了,便道:“人家母女一场,岂有不许他去的。”一面就叫了凤姐儿来,告诉了凤姐儿,命酌量去办理。
\end{parag}


\begin{parag}
    凤姐儿答应了,回至房中,便命周瑞家的去告诉袭人原故。又吩咐周瑞家的:“再将跟著出门的媳妇传一个,你两个人,再带两个小丫头子,跟了袭人去。外头派四个有年纪跟车的。要一辆大车,你们带著坐;要一辆小车,给丫头们坐。”周瑞家的答应了,才要去,凤姐儿又道:“那袭人是个省事的,你告诉他说我的话:叫他穿几件颜色好衣裳,大大的包一包袱衣裳拿著,包袱也要好好的,手炉也要拿好的。临走时,叫他先来我瞧瞧。”周瑞家的答应去了。
\end{parag}


\begin{parag}
    半日,果见袭人穿戴来了,两个丫头与周瑞家的拿著手炉与衣包。凤姐儿看袭人头上戴著几枝金钗珠钏,倒华丽;又看身上穿著桃红百子刻丝银鼠袄子,葱绿盘金彩绣绵裙,外面穿著青缎灰鼠褂。凤姐儿笑道:“这三件衣裳都是太太的,赏了你倒是好的;但只这褂子太素了些,如今穿著也冷,你该穿一件大毛的。”袭人笑道:“太太就只给了这灰鼠的,还有一件银鼠的。说赶年下再给大毛的,还没有得呢。”凤姐儿笑道:“我倒有一件大毛的,我嫌风毛儿出不好了,正要改去。也罢,先给你穿去罢。等年下太太给作的时节我再作罢,只当你还我一样。”众人都笑道:“奶奶惯会说这话。成年家大手大脚的,替太太不知背地里赔垫了多少东西,真真的赔的是说不出来,那里又和太太算去?偏这会子又说这小气话取笑儿。”凤姐儿笑道:“太太那里想的到这些?究竟这又不是正经事,再不照管,也是大家的体面。说不得我自己吃些亏,把众人打扮体统了,宁可我得个好名也罢了。一个一个象‘烧糊了的卷子’似的,人先笑话我当家倒把人弄出个花子来。”众人听了,都叹说:“谁似奶奶这样圣明!在上体贴太太,在下又疼顾下人。”一面说,一面只见凤姐儿命平儿将昨日那件石青刻丝八团天马皮褂子拿出来,与了袭人。又看包袱,只得一个弹墨花绫水红绸里的夹包袱,里面只包著两件半旧棉袄与皮褂。凤姐儿又命平儿把一个玉色绸里的哆罗呢的包袱拿出来,又命包上一件雪褂子。
\end{parag}


\begin{parag}
    平儿走去拿了出来,一件是半旧大红猩猩毡的,一件是大红羽纱的。袭道:“一件就当不起了。”平儿笑道:“你拿这猩猩毡的。把这件顺手拿将出来,叫人给邢大姑娘送去。昨儿那么大雪,人人都是有的,不是猩猩毡就是羽缎羽纱的,十来件大红衣裳,映著大雪好不齐整。就只他穿著那件旧毡斗蓬,越发显的拱肩缩背,好不可怜见的。如今把这件给他罢。”凤姐儿笑道:“我的东西,他私自就要给人。我一个还花不够,再添上你提著,更好了!”众人笑道:“这都是奶奶素日孝敬太太,疼爱下人。若是奶奶素日是小气的,只以东西为事,不顾下人的,姑娘那里还敢这样了。”凤姐儿笑道:“所以知道我的心的,也就是他还知三分罢了。”说著,又嘱咐袭人道:“你妈若好了就罢;若不中用了,只管住下,打发人来回我,我再另打发人给你送铺盖去。可别使人家的铺盖和梳头的家伙。”又吩咐周瑞家的道:“你们自然也知道这里的规矩的,也不用我嘱咐了。”周瑞家的答应:“都知道。我们这去到那里,总叫他们的人回避。若住下,必是另要一两间内房的。”说著,跟了袭人出去,又吩咐预备灯笼,遂坐车往花自芳家来,不在话下。
\end{parag}


\begin{parag}
    这里凤姐又将怡红院的嬷嬷唤了两个来,吩咐道:“袭人只怕不来家,你们素日知道那大丫头们,那两个知好歹,派出来在宝玉屋里上夜。你们也好生照管著,别由著宝玉胡闹。”两个嬷嬷去了,一时来回说:“派了晴雯和麝月在屋里,我们四个人原是轮流著带管上夜的。”凤姐儿听了,点头道:“晚上催他早睡,早上催他早起。”老嬷嬷们答应了,自回园去。一时果有周瑞家的带了信回凤姐儿说:“袭人之母业已停床,不能回来。”凤姐儿回明了王夫人,一面著人往大观园去取他的铺盖妆奁。
\end{parag}


\begin{parag}
    宝玉看著晴雯麝月二人打点妥当,送去之后,晴雯麝月皆卸罢残妆,脱换过裙袄。晴雯只在熏笼上围坐。麝月笑道:“你今儿别装小姐了,我劝你也动一动儿。”晴雯道:“等你们都去尽了,我再动不迟。有你们一日,我且受用一日。”麝月笑道:“好姐姐,我铺床,你把那穿衣镜的套子放下来,上头的划子划上,你的身量比我高些。”说著,便去与宝玉铺床。晴雯嗐了一声,笑道:“人家才坐暖和了,你就来闹。”此时宝玉正坐著纳闷,想袭人之母不知是死是活,忽听见晴雯如此说,便自己起身出去,放下镜套,划上消息,进来笑道:“你们暖和罢,都完了。”晴雯笑道:“终久暖和不成的,我又想起来汤婆子还没拿来呢。”麝月道: “这难为你想著!他素日又不要汤婆子,咱们那熏笼上暖和,比不得那屋里炕冷,今儿可以不用。”宝玉笑道:“这个话,你们两个都在那上头睡了,我这外边没个人,我怪怕的,一夜也睡不著。”晴雯道:“我是在这里。麝月往他外边睡去。”说话之间,天已二更,麝月早已放下帘幔,移灯炷香,伏侍宝玉卧下,二人方睡。
\end{parag}


\begin{parag}
    晴雯自在熏笼上,麝月便在暖阁外边。至三更以后,宝玉睡梦之中,便叫袭人。叫了两声,无人答应,自己醒了,方想起袭人不在家,自己也好笑起来。晴雯已醒,因笑唤麝月道:“连我都醒了,他守在旁边还不知道,真是个挺死尸的。”麝月翻身打个哈气笑道:“他叫袭人,与我什么相干!”因问作什么。宝玉要吃茶,麝月忙起来,单穿红绸小棉袄儿。宝玉道:“披上我的袄儿再去,仔细冷著。”麝月听说,回手便把宝玉披著起夜的一件貂颏满襟暖袄披上,下去向盆内洗手,先倒了一钟温水,拿了大漱盂,宝玉漱了一口;然后才向茶格上取了茶碗,先用温水涮了一涮,向暖壶中倒了半碗茶,递与宝玉吃了;自己也漱了一漱,吃了半碗。晴雯笑道:“好妹子,也赏我一口儿。”麝月笑道:“越发上脸儿了!”晴雯道:“好妹妹,明儿晚上你别动,我伏侍你一夜,如何?”麝月听说,只得也伏侍他漱了口,倒了半碗茶与他吃过。麝月笑道:“你们两个别睡,说著话儿,我出去走走回来。”晴雯笑道:“外头有个鬼等著你呢。” 宝玉道:“外头自然有大月亮的,我们说话,你只管去。”一面说,一面便嗽了两声。
\end{parag}


\begin{parag}
    麝月便开了后门,揭起毡帘一看,果然好月色。晴雯等他出去,便欲唬他玩耍。仗著素日比别人气壮,不畏寒冷,也不披衣,只穿著小袄,便蹑手蹑脚的下了薰笼,随后出来。宝玉笑劝道:“看冻著,不是顽的。”晴雯只摆手,随后出了房门。只见月光如水,忽然一阵微风,只觉侵肌透骨,不禁毛骨森然。心下自思道: “怪道人说热身子不可被风吹,这一冷果然利害。”一面正要唬麝月,只听宝玉高声在内道:“晴雯出去了!”晴雯忙回身进来,笑道:“那里就唬死了他?偏你惯会这蝎蝎螫螫老婆汉像的!”宝玉笑道:“倒不为唬坏了他,头一则你冻著也不好;二则他不防,不免一喊,倘或唬醒了别人,不说咱们是顽意,倒反说袭人才去了一夜,你们就见神见鬼的。你来把我的这边被掖一掖。”晴雯听说,便上来掖了掖,伸手进去渥一渥时,宝玉笑道:“好冷手!我说看冻著。”一面又见晴雯两腮如胭脂一般,用手摸了一摸,也觉冰冷。宝玉道:“快进被来来渥渥罢。”一语未了,只听咯噔的一声门响,麝月慌慌张张的笑了进来,说道:“吓了我一跳好的。黑影子里,山子石后头,只见一个人蹲著。我才要叫喊,原来是那个大锦鸡,见了人一飞,飞到亮处来,我才看真了。若冒冒失失一嚷,倒闹起人来。”一面说,一面洗手,又笑道:“晴雯出去我怎么不见?一定是要唬我去了。”宝玉笑道:“这不是他,在这里渥呢!我若不叫的快,可是倒唬一跳。”晴雯笑道:“也不用我唬去,这小蹄子已经自怪自惊的了。”一面说,一面仍回自己被中去了。麝月道:“你就这么‘跑解马’似的打扮得伶伶俐俐的出去了不成?”宝玉笑道:“可不就这么去了。”麝月道:“你死不拣好日子!你出去站一站,把皮不冻破了你的。”说著,又将火盆上的铜罩揭起,拿灰锹重将熟炭埋了一埋,拈了两块素香放上,仍旧罩了,至屏后重剔了灯,方才睡下。
\end{parag}


\begin{parag}
    晴雯因方才一冷,如今又一暖,不觉打了两个喷嚏。宝玉叹道:“如何?到底伤了风了。”麝月笑道:“他早起就嚷不受用,一日也没吃饭。他这会还不保养些,还要捉弄人。明儿病了,叫他自作自受。”宝玉问:“头上可热?”晴雯嗽了两声,说道:“不相干,那里这么娇嫩起来了。”说著,只听外间房中十锦格上的自鸣钟当当两声,外间值宿的老嬷嬷嗽了两声,因说道:“姑娘们睡罢,明儿再说罢。”宝玉方悄悄的笑道:“咱们别说话了,又惹他们说话。”说著,方大家睡了。
\end{parag}


\begin{parag}
    至次日起来,晴雯果觉有些鼻塞声重,懒怠动弹。宝玉道:“快不要声张!太太知道,又叫你搬了家去养息。家去虽好,到底冷些,不如在这里。你就在里间屋里躺著,我叫人请了大夫,悄悄的从后门来瞧瞧就是了。”晴雯道:“虽如此说,你到底要告诉大奶奶一声儿,不然一时大夫来了,人问起来,怎么说呢?”宝玉听了有理,便唤一个老嬷嬷吩咐道:“你回大奶奶去,就说晴雯白冷著了些,不是什么大病。袭人又不在家,他若家去养病,这里更没有人了。传一个大夫,悄悄的从后门进来瞧瞧,别回太太罢了。”老嬷嬷去了半日,来回说:“大奶奶知道了,说两剂药吃好了便罢,若不好时,还是出去为是。如今时气不好,恐沾带了别人事小,姑娘们的身子要紧的。”晴雯睡在暖阁里,只管咳嗽,听了这话,气的喊道:“我那里就害瘟病了,只怕过了人!我离了这里,看你们这一辈子都别头疼脑热的。”说著,便真要起来。宝玉忙按他,笑道:“别生气,这原是他的责任,唯恐太太知道了说他不是,白说一句。你素习好生气,如今肝火自然盛了。”
\end{parag}


\begin{parag}
    正说时,人回大夫来了。宝玉便走过来,避在书架之后。只见两三个后门口的老嬷嬷带了一个大夫进来。这里的丫鬟都回避了,有三四个老嬷嬷放下暖阁上的大红绣幔,晴雯从幔中单伸出手去。那大夫见这只手上有两根指甲,足有三寸长,尚有金凤花染的通红的痕迹,便忙回过头来。有一个老嬷嬷忙拿了一块手帕掩了。那大夫方诊了一回脉,起身到外间,向嬷嬷们说道:“小姐的症是外感内滞,近日时气不好,竟算是个小伤寒。幸亏是小姐素日饮食有限,风寒也不大,不过是血气原弱,偶然沾带了些,吃两剂药疏散疏散就好了。”说著,便又随婆子们出去。
\end{parag}


\begin{parag}
    彼时,李纨已遣人知会过后门上的人及各处丫鬟回避,那大夫只见了园中的景致,并不曾见一女子。一时出了园门,就在守园门的小厮们的班房内坐了,开了药方。老嬷嬷道:“你老爷且别去,我们小爷罗唆,恐怕还有话说。”大夫忙道:“方才不是小姐,是位爷不成?那屋子竟是绣房一样,又是放下幔子来的,如何是位爷呢?”老嬷嬷悄悄笑道:“我的老爷,怪道小厮们才说今儿请了一位新大夫来了,真不知我们家的事。那屋子是我们小哥儿的,那人是他屋里的丫头,倒是个大姐,那里的小姐?若是小姐的绣房,小姐病了,你那么容易就进去了?”说著,拿了药方进去。
\end{parag}


\begin{parag}
    宝玉看时,上面有紫苏、桔梗、防风、荆芥等药,后面又有枳实、麻黄。宝玉道:“该死,该死,他拿著女孩儿们也象我们一样的治,如何使得!凭他有什么内滞,这枳实、麻黄如何禁得。谁请了来的?快打发他去罢!再请一个熟的来。”老婆子道:“用药好不好,我们不知道这理。如今再叫小厮去请王太医去倒容易,只是这大夫又不是告诉总管房请来的,这轿马钱是要给他的。”宝玉道:“给他多少?”婆子道:“少了不好看,也得一两银子,才是我们这门户的礼。”宝玉道: “王太医来了给他多少?”婆子笑道:“王太医和张太医每常来了,也并没个给钱的,不过每年四节大趸送礼,那是一定的年例。这人新来了一次,须得给他一两银子去。”宝玉听说,便命麝月去取银子。麝月道:“花大奶奶还不知搁在那里呢?”宝玉道:“我常见他在螺甸小柜子里取钱,我和你找去。”说著,二人来至宝玉堆东西的房子,开了螺甸柜子,上一格子都是些笔墨、扇子、香饼、各色荷包、汗巾等物;下一格却是几串钱。于是开了抽屉,才看见一个小簸箩内放著几块银子,倒也有一把戥子。麝月便拿了一块银子,提起戥子来问宝玉:“那是一两的星儿?”宝玉笑道:“你问我?有趣,你倒成了才来的了。”麝月也笑了,又要去问人。宝玉道:“拣那大的给他一块就是了。又不作买卖,算这些做什么!”麝月听了,便放下戥子,拣了一块掂了一掂,笑道:“这一块只怕是一两了。宁可多些好,别少了,叫那穷小子笑话,不说咱们不识戥子,倒说咱们有心小器似的。”那婆子站在外头台矶上,笑道:“那是五两的锭子夹了半边,这一块至少还有二两呢!这会子又没夹剪,姑娘收了这块,再拣一块小些的罢。”麝月早掩了柜子出来,笑道:“谁又找去!多了些你拿了去罢。”宝玉道:“你只快叫茗烟再请王大夫去就是了。”婆子接了银子,自去料理。
\end{parag}


\begin{parag}
    一时茗烟果请了王太医来,诊了脉后,说的病症与前相仿,只是方上果没有枳实、麻黄等药,倒有当归、陈皮、白芍等,药之分量较先也减了些。宝玉喜道: “这才是女孩儿们的药,虽然疏散,也不可太过。旧年我病了,却是伤寒内里饮食停滞,他瞧了,还说我禁不起麻黄、石膏、枳实等狼虎药。我和你们一比,我就如那野坟圈子里长的几十年的一棵老杨树,你们就如秋天芸儿进我的那才开的白海棠,连我禁不起的药,你们如何禁得起。”麝月等笑道:“野坟里只有杨树不成?难道就没有松柏?我最嫌的是杨树,那么大笨树,叶子只一点子,没一丝风,他也是乱响。你偏比他,也太下流了。” 宝玉笑道:“松柏不敢比。连孔子都说:‘岁寒然后知松柏之后凋也。’可知这两件东西高雅,不怕羞臊的才拿他混比呢。”
\end{parag}


\begin{parag}
    说著,只见老婆子取了药来。宝玉命把煎药的银吊子找了出来,\begin{note}庚辰双行夹批:“找”字神理,乃不常用之物也。\end{note}就命在火盆上煎。晴雯因说:“正经给他们茶房里煎去,弄得这屋里药气,如何使得。”宝玉道:“药气比一切的花香果子香都雅。神仙采药烧药,再者高人逸士采药治药,最妙的一件东西。这屋里我正想各色都齐了,就只少药香,如今恰好全了。”一面说,一面早命人煨上。又嘱咐麝月打点东西,遣老嬷嬷去看袭人,劝他少哭。一一妥当,方过前边来贾母王夫人处问安吃饭。
\end{parag}


\begin{parag}
    正值凤姐儿和贾母王夫人商议说:“天又短又冷,不如以后大嫂子带著姑娘们在园子里吃饭一样。等天长暖和了,再来回的跑也不妨。”王夫人笑道:“这也是好主意。刮风下雪倒便宜。吃些东西受了冷气也不好;空心走来,一肚子冷风,压上些东西也不好。不如后园门里头的五间大房子,横竖有女人们上夜的,挑两个厨子女人在那里,单给他姊妹们弄饭。新鲜菜蔬是有分例的,在总管房里支去,或要钱,或要东西;那些野鸡、獐、狍各样野味,分些给他们就是了。”贾母道:“我也正想著呢,就怕又添一个厨房多事些。”凤姐道:“并不多事。一样的分例,这里添了,那里减了。就便多费些事,小姑娘们冷风朔气的,\begin{note}庚辰双行夹批: “朔”字又妙!“朔”作韶上音也,用此音奇想奇想。\end{note}别人还可,第一林妹妹如何禁得住?就连宝兄弟也禁不住,何况众位姑娘。”贾母道:“正是这话了。上次我要说这话,我见你们的大事太多了,如今又添出这些事来,……”要知端的──
\end{parag}


\begin{parag}
    \begin{note}蒙回末总批:此回再从猜谜著色,便与前回末犯重,且又是一幅即景联诗图矣,成何趣味?就灯谜中生一番谶评,别有情思,迥非凡艳。\end{note}
\end{parag}


\begin{parag}
    \begin{note}蒙回末总批:阁起灯谜,接入袭人了却不就袭人一面写照,作者大有苦心。盖袭人不盛饰则非大家威仪,如盛饰又岂有其母临危?而盛饰者乎,在春(凤)姐一面,衣服车马仆从房屋铺盖等物一一点检色色亲嘱,即得掌家人体统,而袭人之俊俏风神毕现。\end{note}
\end{parag}


\begin{parag}
    \begin{note}蒙回末总批:文有数千言写一琐事者,如一吃茶,偏能于未吃之前既吃以后,细细描写;如一拿银,偏能于开柜时生无数波折,平银时又生无数波折。心细如发。\end{note}
\end{parag}
