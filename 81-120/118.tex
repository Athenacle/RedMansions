\chap{一百一十八}{記微嫌舅兄欺弱女 驚謎語妻妾諫癡人}



\begin{parag}
    說話邢王二夫人聽尤氏一段話,明知也難挽回。王夫人只得說道:“姑娘要行善,這也是前生的夙根,我們也實在攔不住。只是咱們這樣人家的姑娘出了家,不成了事體。如今你嫂子說了準你修行,也是好處。卻有一句話要說,那頭髮可以不剃的,只要自己的心真,那在頭髮上頭呢。你想妙玉也是帶髮修行的,不知他怎樣凡心一動,才鬧到那個分兒。姑娘執意如此,我們就把姑娘住的房子便算了姑娘的靜室。所有服侍姑娘的人也得叫他們來問:他若願意跟的,就講不得說親配人,若不願意跟的,另打主意。”惜春聽了,收了淚,拜謝了邢王二夫人,李紈,尤氏等。王夫人說了,便問彩屏等誰願跟姑娘修行。彩屏等回道:“太太們派誰就是誰。”王夫人知道不願意,正在想人。襲人立在寶玉身後,想來寶玉必要大哭,防着他的舊病。豈知寶玉嘆道:“真真難得。”襲人心裏更自傷悲。寶釵雖不言語,遇事試探,見是執迷不醒,只得暗中落淚。王夫人才要叫了衆丫頭來問。忽見紫鵑走上前去,在王夫人面前跪下,回道:“剛纔太太問跟四姑娘的姐姐,太太看着怎麼樣?”王夫人道:“這個如何強派得人的,誰願意他自然就說出來了。”紫鵑道:“姑娘修行自然姑娘願意,並不是別的姐姐們的意思。我有句話回太太,我也並不是拆開姐姐們,各人有各人的心。我服侍林姑娘一場,林姑娘待我也是太太們知道的,實在恩重如山,無以可報。他死了,我恨不得跟了他去。但是他不是這裏的人,我又受主子家的恩典,難以從死。如今四姑娘既要修行,我就求太太們將我派了跟着姑娘,服侍姑娘一輩子。不知太太們準不準。若準了,就是我的造化了。”邢王二夫人尚未答言,只見寶玉聽到那裏,想起黛玉一陣心酸,眼淚早下來了。衆人才要問他時,他又哈哈的大笑,走上來道:“我不該說的。這紫鵑蒙太太派給我屋裏,我纔敢說。求太太準了他罷,全了他的好心。”王夫人道:“你頭裏姊妹出了嫁,還哭得死去活來,如今看見四妹妹要出家,不但不勸,倒說好事,你如今到底是怎麼個意思,我索性不明白了。”寶玉道:“四妹妹修行是已經準的了,四妹妹也是一定主意了。若是真的,我有一句話告訴太太,若是不定的,我就不敢混說了。”惜春道:“二哥哥說話也好笑,一個人主意不定便扭得過太太們來了?我也是象紫鵑的話,容我呢,是我的造化,不容我呢。還有一個死呢。那怕什麼!二哥哥既有話,只管說。”寶玉道:“我這也不算什麼泄露了,這也是一定的。我念一首詩給你們聽聽罷!”衆人道:“人家苦得很的時侯,你倒來做詩。慪人!”寶玉道:“不是做詩,我到一個地方兒看了來的。你們聽聽罷。”衆人道:“使得。你就唸念,別順着嘴兒胡謅。”寶玉也不分辯,便說道:
\end{parag}


\begin{poem}

    \begin{pl}
        勘破三春景不長,緇衣頓改昔年妝。
    \end{pl}


    \begin{pl}
        可憐繡戶侯門女,獨臥青燈古佛旁!
    \end{pl}

\end{poem}


\begin{parag}
    李紈寶釵聽了,詫異道:“不好了,這人入了迷了。”王夫人聽了這話,點頭嘆息,便問寶玉:“你到底是那裏看來的?”寶玉不便說出來,回道:“太太也不必問,我自有見的地方。”王夫人回過味來,細細一想,便更哭起來道:“你說前兒是頑話,怎麼忽然有這首詩?罷了,我知道了,你們叫我怎麼樣呢!我也沒有法兒了,也只得由着你們罷!但是要等我合上了眼,各自幹各自的就完了!”寶釵一面勸着,這個心比刀絞更甚,也掌不住便放聲大哭起來。襲人已經哭的死去活來,幸虧秋紋扶着。寶玉也不啼哭,也不相勸,只不言語。賈蘭賈環聽到那裏,各自走開。李紈竭力的解說:“總是寶兄弟見四妹妹修行,他想來是痛極了,不顧前後的瘋話,這也作不得準的。獨有紫鵑的事情準不準,好叫他起來。”王夫人道:“什麼依不依,橫豎一個人的主意定了,那也扭不過來的。可是寶玉說的也是一定的了。”紫鵑聽了磕頭。惜春又謝了王夫人。紫鵑又給寶玉寶釵磕了頭。寶玉念聲“阿彌陀佛!難得,難得。不料你倒先好了!”寶釵雖然有把持,也難掌住。只有襲人,也顧不得王夫人在上,便痛哭不止,說:“我也願意跟了四姑娘去修行。”寶玉笑道:“你也是好心,但是你不能享這個清福的。”襲人哭道:“這麼說,我是要死的了!”寶玉聽到那裏,倒覺傷心,只是說不出來。因時已五更,寶玉請王夫人安歇,李紈等各自散去。彩屏等暫且伏侍惜春回去,後來指配了人家。紫鵑終身伏侍,毫不改初。此是後話。
\end{parag}


\begin{parag}
    且言賈政扶了賈母靈柩一路南行,因遇着班師的兵將船隻過境,河道擁擠,不能速行,在道實在心焦。幸喜遇見了海疆的官員,聞得鎮海統制欽召回京,想來探春一定回家,略略解些煩心。只打聽不出起程的日期,心裏又煩燥。想到盤費算來不敷,不得已寫書一封,差人到賴尚榮任上借銀五百,叫人沿途迎上來應需用。那人去了幾日,賈政的船纔行得十數里。那家人回來,迎上船隻,將賴尚榮的稟啓呈上。書內告了多少苦處,備上白銀五十兩。賈政看了生氣,即命家人立刻送還,將原書發回,叫他不必費心。那家人無奈,只得回到賴尚榮任所。
\end{parag}


\begin{parag}
    賴尚榮接到原書銀兩,心中煩悶,知事辦得不周到,又添了一百,央求來人帶回,幫着說些好話。豈知那人不肯帶回,撂下就走了。賴尚榮心下不安,立刻修書到家,回明他父親,叫他設法告假贖出身來。於是賴家託了賈薔賈芸等在王夫人面前乞恩放出。賈薔明知不能,過了一日,假說王夫人不依的話回覆了。賴家一面告假,一面差人到賴尚榮任上,叫他告病辭官。王夫人並不知道。
\end{parag}


\begin{parag}
    那賈芸聽見賈薔的假話,心裏便沒想頭,連日在外又輸了好些銀錢,無所抵償,便和賈環相商。賈環本是一個錢沒有的,雖是趙姨娘積蓄些微,早被他弄光了,那能照應人家。便想起鳳姐待他刻薄,要趁賈璉不在家要擺佈巧姐出氣,遂把這個當叫賈芸來上,故意的埋怨賈芸道:“你們年紀又大,放着弄銀錢的事又不敢辦,倒和我沒有錢的人相商。”賈芸道:“三叔,你這話說的倒好笑,咱們一塊兒頑,一塊兒鬧,那裏有銀錢的事。”賈環道:“不是前兒有人說是外藩要買個偏房,你們何不和王大舅商量把巧姐說給他呢?”賈芸道:“叔叔,我說句招你生氣的話,外藩花了錢買人,還想能和咱們走動麼。”賈環在賈芸耳邊說了些話,賈芸雖然點頭,只道賈環是小孩子的話,也不當事。恰好王仁走來說道:“你們兩個人商量些什麼,瞞着我麼?”賈芸便將賈環的話附耳低言的說了。王仁拍手道:“這倒是一種好事,又有銀子。只怕你們不能,若是你們敢辦,我是親舅舅,做得主的。只要環老三在大太太跟前那麼一說,我找邢大舅再一說,太太們問起來你們齊打夥說好就是了。”賈環等商議定了,王仁便去找邢大舅,賈芸便去回邢王二夫人,說得錦上添花。
\end{parag}


\begin{parag}
    王夫人聽了雖然入耳,只是不信。邢夫人聽得邢大舅知道,心裏願意,便打發人找了邢大舅來問他。那邢大舅已經聽了王仁的話,又可分肥,便在邢夫人跟前說道:“若說這位郡王,是極有體面的。若應了這門親事,雖說是不是正配,保管一過了門,姊夫的官早復了,這裏的聲勢又好了。”邢夫人本是沒主意人,被傻大舅一番假話,哄得心動,請了王仁來一問,更說得熱鬧。於是邢夫人倒叫人出去追着賈芸去說。王仁即刻找了人去到外藩公館說了。那外藩不知底細,便要打發人來相看。賈芸又鑽了相看的人,說明“原是瞞着合宅的,只是王府相親。等到成了,他祖母作主,親舅舅的保山,是不怕的。”那相看的人應了。賈芸便送信與邢夫人,並回了王夫人。那李紈寶釵等不知原故,只道是件好事,也都歡喜。
\end{parag}


\begin{parag}
    那日果然來了幾個女人,都是豔妝麗服。邢夫人接了進去,敘了些閒話。那來人本知是個誥命,也不敢待慢。邢夫人因事未定,也沒有和巧姐說明,只說有親戚來瞧,叫他去見。那巧姐到底是個小孩子,那管這些,便跟了奶媽過來。平兒不放心,也跟着來。只見有兩個宮人打扮的,見了巧姐便渾身上下一看,更又起身來拉着巧姐的手又瞧了一遍,略坐了一坐就走了。倒把巧姐看得羞臊,回到房中納悶,想來沒有這門親戚,便問平兒。平兒先看見來頭,卻也猜着八九必是相親的。“但是二爺不在家,大太太作主,到底不知是那府裏的。若說是對頭親,不該這樣相看。瞧那幾個人的來頭,不象是本支王府,好象是外頭路數如今且不必和姑娘說明,且打聽明白再說。”
\end{parag}


\begin{parag}
    平兒心下留神打聽。那些丫頭婆子都是平兒使過的,平兒一問,所有聽見外頭的風聲都告訴了。平兒便嚇的沒了主意,雖不和巧姐說,便趕着去告訴了李紈寶釵,求他二人告訴王夫人。王夫人知道這事不好,便和邢夫人說知。怎奈邢夫人信了兄弟並王仁的話,反疑心王夫人不是好意,便說:“孫女兒也大了,現在璉兒不在家,這件事我還做得主。況且是他親舅爺爺和他親舅舅打聽的,難道倒比別人不真麼!我橫豎是願意的。倘有什麼不好,我和璉兒也抱怨不着別人!”
\end{parag}


\begin{parag}
    王夫人聽了這些話,心下暗暗生氣,勉強說些閒話,便走了出來,告訴了寶釵,自己落淚。寶玉勸道:“太太別煩惱,這件事我看來是不成的。這又是巧姐兒命裏所招,只求太太不管就是了。”王夫人道:“你一開口就是瘋話。人家說定了就要接過去。若依平兒的話,你璉二哥可不抱怨我麼。別說自己的侄孫女兒,就是親戚家的,也是要好纔好。邢姑娘是我們作媒的,配了你二大舅子,如今和和順順的過日子不好麼。那琴姑娘梅家娶了去,聽見說是豐衣足食的很好。就是史姑娘是他叔叔的主意,頭裏原好,如今姑爺癆病死了,你史妹妹立志守寡,也就苦了。若是巧姐兒錯給了人家兒,可不是我的心壞?”正說着,平兒過來瞧寶釵,並探聽邢夫人的口氣。王夫人將邢夫人的話說了一遍。平兒呆了半天,跪下求道:“巧姐兒終身全仗着太太。若信了人家的話,不但姑娘一輩子受了苦,便是璉二爺回來怎麼說呢!”王夫人道:“你是個明白人,起來,聽我說。巧姐兒到底是大太太孫女兒,他要作主,我能夠攔他麼?”寶玉勸道:“無妨礙的,只要明白就是了。”平兒生怕寶玉瘋顛嚷出來,也並不言語,回了王夫人竟自去了。
\end{parag}


\begin{parag}
    這裏王夫人想到煩悶,一陣心痛,叫丫頭扶着勉強回到自己房中躺下,不叫寶玉寶釵過來,說睡睡就好的。自己卻也煩悶,聽見說李嬸孃來了也不及接待。只見賈蘭進來請了安,回道:“今早爺爺那裏打發人帶了一封書子來,外頭小子們傳進來的。我母親接了正要過來,因我老孃來了,叫我先呈給太太瞧,回來我母親就過來來回太太。還說我老孃要過來呢。”說着,一面把書子呈上。王夫人一面接書,一面問道:“你老孃來作什麼?”賈蘭道:“我也不知道。我只見我老孃說,我三姨兒的婆婆家有什麼信兒來了。”王夫人聽了,想起來還是前次給甄寶玉說了李綺,後來放定下茶,想來此時甄家要娶過門,所以李嬸孃來商量這件事情,便點點頭兒。一面拆開書信,見上面寫着道:
\end{parag}


\begin{parag}
    近因沿途俱系海疆凱旋船隻,不能迅速前行。聞探姐隨翁婿來都,不知曾有信否?前接到璉侄手稟,知大老爺身體欠安,亦不知已有確信否?寶玉蘭哥場期已近,務須實心用功,不可怠惰。老太太靈柩抵家,尚需日時。我身體平善,不必掛念。此諭寶玉等知道。月日手書。蓉兒另稟。王夫人看了,仍舊遞給賈蘭,說:“你拿去給你二叔瞧瞧,還交給你母親罷。”正說着,李紈同李嬸過來。請安問好畢,王夫人讓了坐。李嬸孃便將甄家要娶李綺的話說了一遍。大家商議了一會子。李紈因問王夫人道:“老爺的書子太太看過了麼?”王夫人道:“看過了。”賈蘭便拿着給他母親瞧。李紈看了道:“三姑娘出門了好幾年,總沒有來,如今要回京了。太太也放了好些心。”王夫人道:“我本是心痛,看見探丫頭要回來了,心裏略好些。只是不知幾時纔到。”李嬸孃便問了賈政在路好。李紈因向賈蘭道:“哥兒瞧見了?場期近了,你爺爺掂記的什麼似的。你快拿了去給二叔叔瞧去罷。”李嬸孃道:“他們爺兒兩個又沒進過學,怎麼能下場呢?”王夫人道:“他爺爺做糧道的起身時,給他們爺兒兩個援了例監了。”李嬸孃點頭。賈蘭一面拿著書子出來,來找寶玉。
\end{parag}


\begin{parag}
    卻說寶玉送了王夫人去後,正拿着《秋水》一篇在那裏細玩。寶釵從裏間走出,見他看的得意忘言,便走過來一看,見是這個,心裏着實煩悶。細想他只顧把這些出世離羣的話當作一件正經事,終久不妥。看他這種光景,料勸不過來,便坐在寶玉旁邊怔怔的坐着。寶玉見他這般,便道:“你這又是爲什麼?”寶釵道:“我想你我既爲夫婦,你便是我終身的倚靠,卻不在情慾之私。論起榮華富貴,原不過是過眼煙雲,但自古聖賢,以人品根柢爲重。”寶玉也沒聽完,把那書本擱在旁邊,微微的笑道:“據你說人品根柢,又是什麼古聖賢,你可知古聖賢說過‘不失其赤子之心’。那赤子有什麼好處,不過是無知無識無貪無忌。我們生來已陷溺在貪嗔癡愛中,猶如污泥一般,怎麼能跳出這般塵網。如今才曉得‘聚散浮生’四字,古人說了,不曾提醒一個。既要講到人品根柢,誰是到那太初一步地位的!”寶釵道:“你既說‘赤子之心’,古聖賢原以忠孝爲赤子之心,並不是遁世離羣無關無係爲赤子之心。堯舜禹湯周孔時刻以救民濟世爲心,所謂赤子之心,原不過是‘不忍’二字。若你方纔所說的,忍於拋棄天倫,還成什麼道理?”寶玉點頭笑道:“堯舜不強巢許,武周不強夷齊。”寶釵不等他說完,便道:“你這個話益發不是了。古來若都是巢許夷齊,爲什麼如今人又把堯舜周孔稱爲聖賢呢!況且你自比夷齊,更不成話,伯夷叔齊原是生在商末世,有許多難處之事,所以纔有託而逃。當此聖世,咱們世受國恩,祖父錦衣玉食,況你自有生以來,自去世的老太太以及老爺太太視如珍寶。你方纔所說,自己想一想是與不是。”寶玉聽了也不答言,只有仰頭微笑。寶釵因又勸道:“你既理屈詞窮,我勸你從此把心收一收,好好的用用功。但能搏得一第,便是從此而止,也不枉天恩祖德了。”寶玉點了點頭,嘆了口氣說道:“一第呢,其實也不是什麼難事,倒是你這個‘從此而止,不枉天恩祖德’卻還不離其宗。”寶釵未及答言,襲人過來說道:“剛纔二奶奶說的古聖先賢,我們也不懂。我只想着我們這些人從小兒辛辛苦苦跟着二爺,不知陪了多少小心,論起理來原該當的,但只二爺也該體諒體諒。況二奶奶替二爺在老爺太太跟前行了多少孝道,就是二爺不以夫妻爲事,也不可太辜負了人心。至於神仙那一層更是謊話,誰見過有走到凡間來的神仙呢!那裏來的這麼個和尚,說了些混話,二爺就信了真。二爺是讀書的人,難道他的話比老爺太太還重麼!”寶玉聽了,低頭不語。
\end{parag}


\begin{parag}
    襲人還要說時,只聽外面腳步走響,隔着窗戶問道:“二叔在屋裏呢麼?”寶玉聽了,是賈蘭的聲音,便站起來笑道:“你進來罷。”寶釵也站起來。賈蘭進來笑容可掬的給寶玉寶釵請了安,問了襲人的好,——襲人也問了好——便把書子呈給寶玉瞧。寶玉接在手中看了,便道:“你三姑姑回來了。”賈蘭道:“爺爺既如此寫,自然是回來的了。”寶玉點頭不語,默默如有所思。賈蘭便問:“叔叔看見爺爺後頭寫的叫咱們好生唸書了?叔叔這一程子只怕總沒作文章罷?”寶玉笑道:“我也要作幾篇熟一熟手,好去誆這個功名。”賈蘭道:“叔叔既這樣,就擬幾個題目,我跟着叔叔作作,也好進去混場,別到那時交了白卷子惹人笑話。不但笑話我,人家連叔叔都要笑話了。”寶玉道:“你也不至如此。”說着,寶釵命賈蘭坐下。寶玉仍坐在原處,賈蘭側身坐了。兩個談了一回文,不覺喜動顏色。寶釵見他爺兒兩個談得高興,便仍進屋裏去了。心中細想寶玉此時光景,或者醒悟過來了,只是剛纔說話,他把那“從此而止”四字單單的許可,這又不知是什麼意思了。寶釵尚自猶豫,惟有襲人看他愛講文章,提到下場,更又欣然。心裏想道:“阿彌陀佛!好容易講四書似的纔講過來了!”這裏寶玉和賈蘭講文,鶯兒沏過茶來,賈蘭站起來接了。又說了一會子下場的規矩並請甄寶玉在一處的話,寶玉也甚似願意。一時賈蘭回去,便將書子留給寶玉了。
\end{parag}


\begin{parag}
    那寶玉拿著書子,笑嘻嘻走進來遞給麝月收了,便出來將那本《莊子》收了,把幾部向來最得意的,如《參同契》《元命苞》《五燈會元》之類,叫出麝月秋紋鶯兒等都搬了擱在一邊。寶釵見他這番舉動,甚爲罕異,因欲試探他,便笑問道:“不看他倒是正經,但又何必搬開呢。”寶玉道:“如今才明白過來了。這些書都算不得什麼,我還要一火焚之,方爲乾淨。”寶釵聽了更欣喜異常。只聽寶玉口中微吟道:“內典語中無佛性,金丹法外有仙丹。”寶釵也沒很聽真,只聽得”無佛性”“有仙丹”幾個字,心中轉又狐疑,且看他作何光景。寶玉便命麝月秋紋等收拾一間靜室,把那些語錄名稿及應制詩之類都找出來擱在靜室中,自己卻當真靜靜的用起功來。寶釵這才放了心。
\end{parag}


\begin{parag}
    那襲人此時真是聞所未聞,見所未見,便悄悄的笑着向寶釵道:“到底奶奶說話透徹,只一路講究,就把二爺勸明白了。就只可惜遲了一點兒,臨場太近了。”寶釵點頭微笑道:“功名自有定數,中與不中倒也不在用功的遲早。但願他從此一心巴結正路,把從前那些邪魔永不沾染就是好了。”說到這裏,見房裏無人,便悄說道:“這一番悔悟回來固然很好,但只一件,怕又犯了前頭的舊病,和女孩兒們打起交道來,也是不好。”襲人道:“奶奶說的也是。二爺自從信了和尚,才把這些姐妹冷淡了,如今不信和尚,真怕又要犯了前頭的舊病呢。我想奶奶和我二爺原不大理會,紫鵑去了,如今只他們四個,這裏頭就是五兒有些個狐媚子,聽見說他媽求了大奶奶和奶奶,說要討出去給人家兒呢。但是這兩天到底在這裏呢。麝月秋紋雖沒別的,只是二爺那幾年也都有些頑頑皮皮的。如今算來只有鶯兒二爺倒不大理會,況且鶯兒也穩重。我想倒茶弄水只叫鶯兒帶着小丫頭們伏侍就夠了,不知奶奶心裏怎麼樣。”寶釵道:“我也慮的是這些,你說的倒也罷了。”從此便派鶯兒帶着小丫頭伏侍。
\end{parag}


\begin{parag}
    那寶玉卻也不出房門,天天只差人去給王夫人請安。王夫人聽見他這番光景,那一種欣慰之情,更不待言了。到了八月初三,這一日正是賈母的冥壽。寶玉早晨過來磕了頭,便回去,仍到靜室中去了。飯後,寶釵襲人等都和姊妹們跟着邢王二夫人在前面屋裏說閒話兒。寶玉自在靜室冥心危坐,忽見鶯兒端了一盤瓜果進來說:“太太叫人送來給二爺喫的。這是老太太的克什。”寶玉站起來答應了,復又坐下,便道:“擱在那裏罷。”鶯兒一面放下瓜果一面悄悄向寶玉道:“太太那裏誇二爺呢。”寶玉微笑。鶯兒又道:“太太說了,二爺這一用功,明兒進場中了出來,明年再中了進士,作了官,老爺太太可就不枉了盼二爺了。”寶玉也只點頭微笑。鶯兒忽然想起那年給寶玉打絡子的時候寶玉說的話來,便道:“真要二爺中了,那可是我們姑奶奶的造化了。二爺還記得那一年在園子裏,不是二爺叫我打梅花絡子時說的,我們姑奶奶後來帶着我不知到那一個有造化的人家兒去呢。如今二爺可是有造化的罷咧。”寶玉聽到這裏,又覺塵心一動,連忙斂神定息,微微的笑道:“據你說來,我是有造化的,你們姑娘也是有造化的,你呢?”鶯兒把臉飛紅了,勉強道:“我們不過當丫頭一輩子罷咧,有什麼造化呢!”寶玉笑道:“果然能夠一輩子是丫頭,你這個造化比我們還大呢!”鶯兒聽見這話似乎又是瘋話了,恐怕自己招出寶玉的病根來,打算着要走。只見寶玉笑着說道:“傻丫頭,我告訴你罷。”未知寶玉又說出什麼話來,且聽下回分解。
\end{parag}