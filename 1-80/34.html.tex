\chap{三十四}{情中情因情感妹妹 错里错以错劝哥哥}


\begin{parag}
    \begin{note}蒙回前总:两条素怡,一片真心,三首新诗,万行珠泪。\end{note}
\end{parag}


\begin{parag}
    话说袭人见贾母王夫人等去后,便走来宝玉身边坐下,含泪问他:“怎么就打到这步田地?”宝玉叹气说道:“不过为那些事,问他做什么!只是下半截疼的很,你瞧瞧打坏了那里。”袭人听说,便轻轻的伸手进去,将中衣褪下。宝玉略动一动,便咬著牙叫“嗳哟”,袭人连忙停住手,如此三四次才褪了下来。袭人看时,只见腿上半段青紫,都有四指宽的僵痕高了起来。袭人咬著牙说道:“我的娘,怎么下这般的狠手!你但凡听我一句话,也不得到这步地位。幸而没动筋骨,倘或打出个残疾来,可叫人怎么样呢!”
\end{parag}


\begin{parag}
    正说著,只听丫鬟们说:“宝姑娘来了。”袭人听见,知道穿不及中衣,便拿了一床袷纱被替宝玉盖了。只见宝钗手里托著一丸药走进来,\begin{note}蒙侧:请问是关心不是关心?\end{note}向袭人说道:“晚上把这药用酒研开,替他敷上,把那淤血的热毒散开,可以就好了。”说毕,递与袭人,又问道:“这会子可好些?”宝玉一面道谢说:“好了。”又让坐。宝钗见他睁开眼说话,不象先时,心中也宽慰了好些,便点头叹道:“早听人一句话,\begin{note}蒙侧:同袭人语。\end{note}也不至 今日。别说老太太、太太心疼,就是我们看著,心里也疼。”刚说了半句又忙咽住,自悔说的话急了,不觉的就红了脸,\begin{note}蒙侧:行云流水,微露半含时。\end{note}低下头来。宝玉听得这话如此亲切稠密,大有深意,忽见他又咽住不往下说,红了脸,低下头只管弄衣带,那一种娇羞怯怯,非可形容得出者,不觉心中大畅,将疼痛早丢在九霄云外,心中自思:“我不过挨了几下打,他们一个个就有这些怜惜悲感之态露出,令人可玩可观,可怜可敬。假若我一时竟遭殃横死,他们还不知是何等悲 感呢!\begin{note}蒙侧:得遇知己者,多生此等疑思疑喜。\end{note}既是他们这样,我便一时死了,得他们如此,一生事业纵然尽付东流,亦无足叹惜,冥冥之中若不怡然自得, 亦可谓糊涂鬼祟矣。”想著,只听宝钗问袭人道:“怎么好好的动了气,就打起来了?”袭人便把焙茗的话说了出来。宝玉原来还不知道贾环的话,见袭人说出方才知道。因又拉上薛蟠,惟恐宝钗沉心,忙又止住袭人道:“薛大哥哥从来不这样的,你们不可混猜度。”宝钗听说,便知道是怕他多心,用话相拦袭人,因心中暗暗 想道:“打的这个形像,疼还顾不过来,还是这样细心,怕得罪了人,可见在我们身上也算是用心了。\begin{note}蒙侧:天下古今英雄同一感慨。\end{note}你既这样用心,何不在外头大事上做工夫,老爷也欢喜了,也不能吃这样亏。但你固然怕我沉心,所以拦袭人的话,难道我就不知我的哥哥素日恣心纵欲,毫无防范的那种心性。当日为一个秦钟,还闹的天翻地覆,自然如今比先又更利害了。”想毕,因笑道:“你们也不必怨这个,怨那个。据我想,到底宝兄弟素日不正,肯和那些人来往,老爷才生 气。就是我哥哥说话不防头,一时说出宝兄弟来,也不是有心调唆:一则也是本来的实话,二则他原不理论这些防嫌小事。袭姑娘从小儿只见宝兄弟这么样细心的人,\begin{note}蒙侧:心头口头不觉透漏。\end{note}你何尝见过天不怕地不怕、心里有什么口里就说什么的人。”袭人因说出薛蟠来,见宝玉拦他的话,早已明白自己说造次了, 恐宝钗没意思,听宝钗如此说,更觉羞愧无言。宝玉又听宝钗这番话,一半是堂皇正大,一半是去己疑心,更觉比先畅快了。方欲说话时,只见宝钗起身说道:“明儿再来看你,你好生养著罢。方才我拿了药来交给袭人,晚上敷上管就好了。\begin{note}蒙侧:何等关心。\end{note}”说著便走出门去。袭人赶著送出院外,说:“姑娘倒费心了。改日宝二爷好了,亲自来谢。”宝钗回头笑道:“有什么谢处。你只劝他好生静养,别胡思乱想的就好了。\begin{note}蒙侧:的确真心。\end{note}不必惊动老太太、太太众人,倘或吹到老爷耳朵里,虽然彼时不怎么样,将来对景,终是要吃亏的。\begin{note}蒙侧:要紧。\end{note}”说著,一面去了。
\end{parag}


\begin{parag}
    袭人抽身回来,心内著实感激宝钗。进来见宝玉沉思默默似睡非睡的模样,因而退出房外,自去栉沐。宝玉默默的躺在床上,无奈臀上作痛,如针挑刀挖一般, 更又热如火炙,略展转时,禁不住“嗳哟”之声。那时天色将晚,因见袭人去了,却有两三个丫鬟伺候,此时并无呼唤之事,因说道:“你们且去梳洗,等我叫时再来。”众人听了,也都退出。
\end{parag}


\begin{parag}
    这里宝玉昏昏默默,只见蒋玉菡走了进来,诉说忠顺府拿他之事;又见金钏儿进来哭说为他投井之情。宝玉半梦半醒,都不在意。忽又觉有人推他,恍恍忽忽听 得有人悲戚之声。宝玉从梦中惊醒,睁眼一看,不是别人,却是林黛玉。宝玉犹恐是梦,忙又将身子欠起来,向脸上细细一认,只见两个眼睛肿的桃儿一般,满面泪光,不是黛玉,却是那个?宝玉还欲看时,怎奈下半截疼痛难忍,支持不住,便“嗳哟”一声,仍就倒下,叹了一声,说道:“你又做什么跑来!虽说太阳落下去, 那地上的余热未散,走两趟又要受了暑。我虽然挨了打,并不觉疼痛。我这个样儿,只装出来哄他们,好在外头布散与老爷听,其实是假的。你不可认真。\begin{note}蒙侧:有这样一段语,方不没灭颦颦儿之痛哭眼肿。英雄失足,每每至死不改,皆犹此而。\end{note}”此时林黛玉虽不是嚎啕大哭,然越是这等无声之泣,气噎喉堵,更觉得利害。听了宝玉这番话,心中虽然有万句言词,只是不能说得,半日,方抽抽噎噎的说道:“你从此可都改了罢!\begin{note}蒙侧:心血淋漓酿成此数字。\end{note}”宝玉听说, 便长叹一声,道:“你放心,别说这样话。就便为这些人死了,\begin{note}蒙侧:文气斩动。\end{note}也是情愿的!(校者注:蒙本此处无“也是情愿的”,换作“况已是活过来 了”)”一句话未了,只见院外人说:“二奶奶来了。”林黛玉便知是凤姐来了,连忙立起身说道:“我从后院子去罢,回来再来。”宝玉一把拉住道:“这可奇了,好好的怎么怕起他来。”林黛玉急的跺脚,悄悄的说道:“你瞧瞧我的眼睛,又该他取笑开心呢。\begin{note}蒙侧:不避嫌疑,不惜声名,破格牵连,诚为可叹,著实 可怜。\end{note}”宝玉听说赶忙的放手。黛玉三步两步转过床后,出后院而去。凤姐从前头已进来了,问宝玉:“可好些了?想什么吃,叫人往我那里取去。”接著,薛姨妈又来了。一时贾母又打发了人来。
\end{parag}


\begin{parag}
    至掌灯时分,宝玉只喝了两口汤,便昏昏沉沉的睡去。接著,周瑞媳妇、吴新登媳妇、郑好时媳妇这几个有年纪常往来的,听见宝玉挨了打,也都进来。袭人忙迎出来,悄悄的笑道:“婶婶们来迟了一步,\begin{note}蒙侧:袭卿善词令会周旋。\end{note}二爷才睡著了。”说著,一面带他们到那边房里坐了,倒茶与他们吃。那几个媳妇子都悄悄的坐了一回,向袭人说:“等二爷醒了,你替我们说罢。”
\end{parag}


\begin{parag}
    袭人答应了,送他们出去。刚要回来,只见王夫人使个婆子来,口称“太太叫一个跟二爷的人呢”。袭人见说,想了一想,便回身悄悄告诉晴雯、麝月、檀云、 秋纹等说:“太太叫人,你们好生在房里,我去了就来。”\begin{note}蒙侧:身任其责,不惮劳烦。\end{note}说毕,同那婆子一径出了园子,来至上房。王夫人正坐在凉榻上摇著芭蕉扇子,见他来了,说:“不管叫个谁来也罢了。你又丢下他来了,谁伏侍他呢?”袭人见说,连忙陪笑回道:“二爷才睡安稳了,那四五个丫头如今也好了,会伏侍二爷了,太太请放心。恐怕太太有什么话吩咐,打发他们来,一时听不明白,倒耽误了。\begin{note}蒙侧:能事解事能了事。\end{note}”王夫人道:“也没甚话,白问问他这 会子疼的怎么样。”袭人道:“宝姑娘送去的药,我给二爷敷上了,\begin{note}蒙侧:补足。\end{note}比先好些了。先疼的躺不稳,这会子都睡沉了,可见好些了。”王夫人又 问:“吃了什么没有?”袭人道:“老太太给的一碗汤,喝了两口,只嚷干喝,要吃酸梅汤。我想著酸梅是个收敛的东西,才刚挨了打,又不许叫喊,自然急的那热 毒热血未免不存在心里,倘或吃下这个去激在心里,再弄出大病来,可怎么样呢。因此我劝了半天才没吃,\begin{note}蒙侧:能事态。\end{note}只拿那糖腌的玫瑰卤子和了吃,吃了半碗,又嫌吃絮了,不香甜。”王夫人道:“嗳哟,你不该早来和我说。前儿有人送了两瓶子香露来,原要给他点子的,我怕他胡糟踏了,就没给。既是他嫌那些 玫瑰膏子絮烦,把这个拿两瓶子去。一碗水里只用挑一茶匙儿,就香的了不得呢。”说著就唤彩云来,“把前儿的那几瓶香露拿了来。”袭人道:“只拿两瓶来罢, 多了也白糟踏。等不够再要,再来取也是一样。”彩云听说,去了半日,果然拿了两瓶来,付与袭人。袭人看时,只见两个玻璃小瓶,却有三寸大小,上面螺丝银盖,鹅黄笺上写著“木樨清露”,那一个写著“玫瑰清露”。袭人笑道:“好金贵东西!这么个小瓶儿,能有多少?”王夫人道:“那是进上的,你没看见鹅黄笺子?你好生替他收著,别糟踏了。”
\end{parag}


\begin{parag}
    袭人答应著,方要走时,王夫人又叫:“站著,我想起一句话来问你。”袭人忙又回来。王夫人见房内无人,便问道:“我恍惚听见宝玉今儿挨打,是环儿在老 爷跟前说了什么话。你可听见这个了?你要听见,告诉我听听,我也不吵出来教人知道是你说的。”袭人道:“我倒没听见这话,为二爷霸占著戏子,人家来和老爷要,为这个打的。”王夫人摇头说道:“也为这个,还有别的原故。”袭人道:“别的原故实在不知道了。我今儿在太太跟前大胆说句不知好歹的话。论理……”说 了半截忙又咽住。王夫人道:“你只管说。”袭人笑道:“太太别生气,我就说了。”王夫人道:“我有什么生气的,你只管说来。”袭人道:“论理,我们二爷也须得老爷教训两顿。若老爷再不管,将来不知做出什么事来呢。”王夫人一闻此言,便合掌念声“阿弥陀佛”,\begin{note}蒙侧:袭卿之心,所谓良人所仰望而终身也。今若此,能不痛哭流泣以成此语?\end{note}由不得赶著袭人叫了一声“我的儿,亏了你也明白,这话和我的心一样。我何曾不知道管儿子,先时你珠大爷在,我是怎么样管 他,难道我如今倒不知管儿子了?只是有个原故:如今我想,我已经快五十岁的人,通共剩了他一个,他又长的单弱,况且老太太宝贝似的,若管紧了他,倘或再有个好歹,或是老太太气坏了,那时上下不安,岂不倒坏了,所以就纵坏了他。我常常掰著口儿劝一阵,说一阵,气的骂一阵,哭一阵,彼时他好,过后儿还是不相干,端的吃了亏才罢了。若打坏了,将来我靠谁呢!\begin{note}蒙侧:变转之句,勉强之言,真体贴,尽溺爱之心。\end{note}”说著,由不得滚下泪来。
\end{parag}


\begin{parag}
    袭人见王夫人这般悲感,自己也不觉伤了心,陪著落泪。又道:“二爷是太太养的,岂不心疼。便是我们做下人的伏侍一场,大家落个平安,也算是造化了。要这样起来,连平安都不能了。那一日那一时我不劝二爷,只是再劝不醒。偏生那些人又肯亲近他,也怨不得他这样,总是我们劝的倒不好了。今儿太太提起这话来, 我还记挂著一件事,每要来回太太,讨太太个主意。只是我怕太太疑心,不但我的话白说了,且连葬身之地都没了。\begin{note}蒙侧:打进一层。非有前项,如许讲究这一 层,即为唐突了。\end{note}” 王夫人听了这话内有因,忙问道:“我的儿,你有话只管说。近来我因听见众人背前背后都夸你,我只说你不过是在宝玉身上留心,或是诸人跟前和气,这些小意思好,所以将你和老姨娘一体行事。谁知你方才和我说的话全是大道理,正和我的想头一样。你有什么只管说什么,只别教别人知道就是了。”袭人道:“我也没什么 别的说。我只想著讨太太一个示下,怎么变个法儿,以后竟还教二爷搬出园外来就好了。”王夫人听了,吃一大惊,忙拉了袭人的手问道:“宝玉难道和谁作怪了不成?”袭人忙回道:“太太别多心,并没有这话。这不过是我的小见识。如今二爷也大了,里头姑娘们也大了,况且林姑娘宝姑娘又是两姨姑表姊妹,虽说是姊妹 们,到底是男女之分,日夜一处起坐不方便,由不得叫人悬心,\begin{note}蒙侧:远忧近虑,言言字字真是可人。\end{note}便是外人看著也不象。一家子的事,俗语说的‘没事常思有事’,世上多少无头脑的事,多半因为无心中做出,有心人看见,当做有心事,反说坏了。只是预先不防著,断然不好。二爷素日性格,太太是知道的。他又偏好在我们队里闹,倘或不防,前后错了一点半点,不论真假,人多口杂,那起小人的嘴有什么避讳,心顺了,说的比菩萨还好,心不顺,就贬的连畜牲不如。二爷将 来倘或有人说好,不过大家直过没事;若叫人说出一个不好字来,我们不用说,粉身碎骨,罪有万重,都是平常小事,便后来二爷一生的声名品行岂不完了,\begin{note}蒙侧:袭卿爱人以德,竟至如此。字字逼来,不觉令人静听。看官自省,且可阔略戒之。\end{note}二则太太也难见老爷。俗语又说‘君子防不然’,不如这会子防避的为是。 太太事情多,一时固然想不到。我们想不到则可,既想到了,若不回明太太,罪越重了。近来我为这事日夜悬心,又不好说与人,惟有灯知道罢了。”王夫人听了这话,如雷轰电掣一般,正触了金钏儿之事,心内越发感爱袭人不尽,忙笑道:“我的儿,你竟有这个心胸,想的这样周全!我何曾又不想到这里,只是这几次有事就忘了。你今儿这一番话提醒了我。难为你成全我娘儿两个声名体面,真真我竟不知道你这样好。罢了,你且去罢,我自有道理。\begin{note}蒙侧:溺爱者偏会如此说。\end{note}只是还有一句话:你如今既说了这样的话,我就把他交给你了,好歹留心,保全了他,就是保全了我。我自然不辜负你。”
\end{parag}


\begin{parag}
    袭人连连答应著去了。回来正值宝玉睡醒,袭人回明香露之事。宝玉喜不自禁,即令调来尝试,果然香妙非常。因心下记挂著黛玉,满心里要打发人去,只是怕袭人,便设一法,先使袭人往宝钗那里去借书。
\end{parag}


\begin{parag}
    袭人去了,宝玉便命晴雯来\begin{note}蒙双夹:前文晴雯放肆原有把柄所恃也。\end{note}吩咐道:“你到林姑娘那里看看他做什么呢。他要问我,只说我好了。”晴雯道: “白眉赤眼,做什么去呢?到底说句话儿,也象一件事。”宝玉道:“没有什么可说的。”晴雯道:“若不然,或是送件东西,或是取件东西,不然我去了怎么搭讪 呢?”宝玉想了一想,便伸手拿了两条手帕子撂与晴雯,笑道:“也罢,就说我叫你送这个给他去了。”晴雯道:“这又奇了。他要这半新不旧的两条手帕子?他又要恼了,说你打趣他。”宝玉笑道:“你放心,他自然知道。”
\end{parag}


\begin{parag}
    晴雯听了,只得拿了帕子往潇湘馆来。只见春纤正在栏杆上晾手帕子,\begin{note}蒙侧:送的是手帕,晾的是手帕,妙文。\end{note}见他进来,忙摆手儿,说:“睡下了。” 晴雯走进来,满屋黑魆。并未点灯。黛玉已睡在床上。问是谁。晴雯忙答道:“晴雯。”黛玉道:“做什么?”晴雯道:“二爷送手帕子来给姑娘。”黛玉听了,心中发闷:“做什么送手帕子来给我?”因问:“这帕子是谁送他的?必是上好的,叫他留著送别人罢,我这会子不用这个。” 晴雯笑道:“不是新的,就是家常旧的。”林黛玉听见,越发闷住,著实细心搜求,思忖一时,方大悟过来,连忙说:“放下,去罢。”晴雯听了,只得放下,抽身 回去,一路盘算,不解何意。
\end{parag}


\begin{parag}
    这里林黛玉体贴出手帕子的意思来,不觉神魂驰荡:宝玉这番苦心,能领会我这番苦意,又令我可喜;我这番苦意,不知将来如何,又令我可悲;忽然好好的送 两块旧帕子来,若不是领我深意,单看了这帕子,又令我可笑;再想令人私相传递与我,又可惧;我自己每每好哭,想来也无味,又令我可愧。如此左思右想,一时五内沸然炙起。黛玉由不得余意绵缠,令掌灯,也想不起嫌疑避讳等事,便向案上研墨蘸笔,便向那两块旧帕上走笔写道:
\end{parag}


\begin{poem}
    \begin{pl}其一\end{pl}

    \begin{pl}眼空蓄泪泪空垂,暗洒闲抛却为谁?\end{pl}

    \begin{pl}尺幅鲛鮹劳解赠,叫人焉得不伤悲!\end{pl}
    \emptypl

    \begin{pl}其二\end{pl}

    \begin{pl}抛珠滚玉只偷潸,镇日无心镇日闲;\end{pl}

    \begin{pl}枕上袖边难拂拭,任他点点与斑斑。\end{pl}
    \emptypl

    \begin{pl}其三\end{pl}

    \begin{pl}彩线难收面上珠,湘江旧迹已模糊;\end{pl}

    \begin{pl}窗前亦有千竿竹,不识香痕渍也无?\end{pl}
\end{poem}


\begin{parag}
    林黛玉还要往下写时,觉得浑身火热,面上作烧,走至镜台揭起锦袱一照,只见腮上通红,自羡压倒桃花,却不知病由此萌。一时方上床睡去,犹拿著那帕子思索,不在话下。
\end{parag}


\begin{parag}
    却说袭人来见宝钗,谁知宝钗不在园内,往他母亲那里去了,袭人便空手回来。等至二更,宝钗方回来。原来宝钗素知薛蟠情性,心中已有一半疑是薛蟠调唆了人来告宝玉的,谁知又听袭人说出来,越发信了。究竟袭人是听焙茗说的,那焙茗也是私心窥度,并未据实,竟认准是他说的。那薛蟠都因素日有这个名声,其实这一次却不是他干的,被人生生的一口咬死是他,有口难分。这日正从外头吃了酒回来,见过母亲,只见宝钗在这里,说了几句闲话,因问:“听见宝兄弟吃了亏,是 为什么?”薛姨妈正为这个不自在,见他问时,便咬著牙道:“不知好歹的东西,都是你闹的,你还有脸来问!”薛蟠见说,便怔了,忙问道:“我何尝闹什么?” 薛姨妈道:“你还装憨呢!人人都知道是你说的,还赖呢。”薛蟠道:“人人说我杀了人,也就信了罢?”薛姨妈道:“连你妹妹都知道是你说的,难道他也赖你不 成?”宝钗忙劝道:“妈和哥哥且别叫喊,消消停停的,就有个青红皂白了。”因向薛蟠道:“是你说的也罢,不是你说的也罢,事情也过去了,不必较证,倒把小 事儿弄大了。我只劝你从此以后在外头少去胡闹,少管别人的事。天天一处大家胡逛,你是个不防头的人,过后儿没事就罢了,倘或有事,不是你干的,人人都也疑惑是你干的,不用说别人,我就先疑惑。”薛蟠本是个心直口快的人,一生见不得这样藏头露尾的事,又见宝钗劝他不要逛去,他母亲又说他犯舌,宝玉之打是他治的,早已急的乱跳,赌身发誓的分辩。又骂众人:“谁这样赃派我?我把那囚攮的牙敲了才罢!分明是为打了宝玉,没的献勤儿,拿我来作幌子。难道宝玉是天王? 他父亲打他一顿,一家子定要闹几天。那一回为他不好,姨爹打了他两下子,过后老太太不知怎么知道了,说是珍大哥哥治的,好好的叫了去骂了一顿。今儿越发拉上我了!既拉上,我也不怕,越性进去把宝玉打死了,我替他偿了命,大家干净。”一面嚷,一面抓起一根门闩来就跑。慌的薛姨妈一把抓住,骂道:“作死的孽障,你打谁去?你先打我来!”薛蟠急的眼似铜铃一般,嚷道:“何苦来!又不叫我去,又好好的赖我。将来宝玉活一日,我担一日的口舌,不如大家死了清净。” 宝钗忙也上前劝道:“你忍耐些儿罢。妈急的这个样儿,你不说来劝妈,你还反闹的这样。别说是妈,便是旁人来劝你,也为你好,倒把你的性子劝上来了。”薛蟠道:“这会子又说这话。都是你说的!”宝钗道:“你只怨我说,再不怨你顾前不顾后的形景。”薛蟠道:“你只会怨我顾前不顾后,你怎么不怨宝玉外头招风惹草的那个样子!别说多的,只拿前儿琪官的事比给你们听:那琪官,我们见过十来次的,我并未和他说一句亲热话;怎么前儿他见了,连姓名还不知道,就把汗巾子给他了?难道这也是我说的不成?”薛姨妈和宝钗急的说道:“还提这个!可不是为这个打他呢。可见是你说的了。”薛蟠道:“真真的气死了人了!赖我说的我不 恼,我只为一个宝玉闹的这么天翻地覆的。”宝钗道:“谁闹了?你先持刀动杖的闹起来,倒说别人闹。”薛蟠见宝钗说的句句有理,难以驳正,比母亲的话反难回答,因此便要设法拿话堵回他去,就无人敢拦自己的话了;也因正在气头儿上,未曾想话之轻重,便说道:“好妹妹,你不用和我闹,我早知道你的心了。从先妈和 我说,你这金要拣有玉的才可正配,你留了心,见宝玉有那劳什骨子,你自然如今行动护著他。”话未说了,把个宝钗气怔了,拉著薛姨妈哭道:“妈妈你听,哥哥说的是什么话!”\begin{note}蒙侧:描写薛蟠,不过要补足宝钗告袭人前项之言。\end{note}薛蟠见妹妹哭了,便知自己冒撞了,便堵气走到自己房里安歇不提。
\end{parag}


\begin{parag}
    这里薛姨妈气的乱战,一面又劝宝钗道:“你素日知那孽障说话没道理,明儿我叫他给你陪不是。”宝钗满心委屈气忿,待要怎样,又怕他母亲不安,少不得含泪别了母亲,各自回来,到房里整哭了一夜。次日早起来,也无心梳洗,胡乱整理整理,便出来瞧母亲。可巧遇见林黛玉独立在花阴之下,问他那里去。薛宝钗因说 “家去”,口里说著,便只管走。黛玉见他无精打采的去了,又见眼上有哭泣之状,大非往日可比,便在后面笑道:“姐姐也自保重些儿。就是哭出两缸眼泪来,也 医不好棒疮!”\begin{note}蒙侧:自己眼肿为谁?偏是以此笑人。笑人世间人多犯此症。\end{note}不知宝钗如何答对,且听下回分解。
\end{parag}


\begin{parag}
    \begin{note}蒙回末总:人有百折不挠之真心,方能成旷世稀有之事业。宝玉意中诸多辐辏,所谓“求仁得仁,又和怨?”凡人作臣作子,出入家庭庙朝,能推此心此志,忠孝之不、事业之不立耶?\end{note}
\end{parag}

