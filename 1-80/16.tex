\chap{一十七}{賈元春才選鳳藻宮 秦鯨卿夭逝黃泉路}

\begin{parag}
    \begin{note}甲:幼兒小女之死,得情之正氣,又爲癡貪輩一針灸。鳳姐惡跡多端,莫大於此件者:受贓婚以致人命。賈府連日鬧熱非常,寶玉無見無聞,卻是寶玉正文。夾寫秦、智數句,下半回方不突然。\end{note}
\end{parag}


\begin{parag}
    \begin{note}甲:黛玉回,方解寶玉爲秦鍾之憂悶,是天然之章法。平兒借香菱答話,是補菱姐近來著落。趙嫗討情閒文,卻引出通部脈絡。所謂由小及大,譬如登高必自卑之意。細思大觀園一事,若從如何奉旨起造,又如何分派衆人,從頭細細直寫將來,幾千樣細事,如何能順筆一氣寫清?又將落於死板拮据之鄉,故只用璉鳳夫妻二人一問一答,上用趙嫗討情作引,下文蓉薔來說事作收,餘者隨筆筆略一點染,則耀然洞徹矣。此是避難法。\end{note}
\end{parag}


\begin{parag}
    \begin{note}甲:大觀園用省親事出題,是大關鍵處,方見大手筆行文之立意。\end{note}
\end{parag}


\begin{parag}
    \begin{note}甲:借省親事寫南巡,出脫心中多少憶昔感今。\end{note}
\end{parag}


\begin{parag}
    \begin{note}甲:極熱鬧極忙中,寫秦鍾夭逝,可知除“情”字,俱非寶玉正文。\end{note}
\end{parag}


\begin{parag}
    \begin{note}甲:大鬼小鬼論勢利興衰,罵盡攢炎附勢之輩。\end{note}
\end{parag}


\begin{parag}
    \begin{note}蒙:請看財勢與情根,萬物難逃造化門。曠典傳來空好聽。那如知己解溫存?\end{note}
\end{parag}


\begin{parag}
    話說寶玉見收拾了外書房,約定與秦鍾讀夜書。偏那秦鐘的秉賦最弱,因在郊外受了些風霜,又與智能兒偷期綣繾,未免失於調養,\begin{note}庚側:勿笑。這樣無能,卻是寫與人看。\end{note}回來時便咳嗽傷風,懶進飲食,大有不勝之態,遂不敢出門,只在家中養息。\begin{note}甲側:爲下文伏線。\end{note}寶玉便掃了興,只得付於無可奈何,且自靜候大愈時再約。\begin{note}甲側:所謂“好事多魔”也。[庚本多署名“脂硯”。]\end{note}
\end{parag}


\begin{parag}
    那鳳姐已是得了雲光的回信,俱已妥協。老尼達知張家,果然那守備忍氣吞聲的受了前聘之物。誰知那張家父母如此愛勢貪財,卻養了個知義多情的女兒,\begin{note}庚側:所謂“老鴉窩裏出鳳凰”,此女是在十二釵之外副者。\end{note}聞得父母退了前夫,他便將一條麻繩悄悄的自縊了。那守備之子聞得金哥自縊,他也是個極多情的,遂也投河而死,不負妻義。\begin{note}庚側:一雙美滿夫妻。\end{note}張李兩家沒趣,真是人財兩空。這裏鳳姐卻坐享了三千兩,\begin{note}庚側:如何消繳?造孽者不知,自有知者。\end{note}王夫人等連一點消息也不知道。自此鳳姐膽識愈壯,以後有了這樣的事,便恣意的作爲起來,也不消多記。\begin{note}甲雙夾:一段收拾過阿鳳心機膽量,真與雨村是一對亂世之奸雄。後文不必細寫其事,則知其乎生之作爲。回首時,無怪乎其慘痛之態,使天下癡心人同來一警,或可期共入於恬然自得之鄉矣。脂硯。\end{note}
\end{parag}


\begin{parag}
    一日正是賈政的生辰,寧榮二處人丁都齊集慶賀,熱鬧非常。忽有門吏忙忙進來,至席前報說:“有六宮都太監夏老爺來降旨。”唬得賈赦賈政等一干人不知是何消息,忙止了戲文,撤去酒席,擺了香案,啓中門跪接。早見六宮都太監夏守忠乘馬而至,前後左右又有許多內監跟從。那夏守忠也不曾負詔捧敕,至檐前下馬,滿面笑容,走至廳上,面南而立,口內說:“特旨:立刻宣賈政入朝,在臨敬殿陛見。”說畢,也不及喫茶,便乘馬去了。賈政等不知是何兆頭。只得急忙更衣入朝。\begin{note}庚眉:潑天喜事卻如此開宗。出人意料外之文也。壬午季春。\end{note}
\end{parag}


\begin{parag}
    賈母等閤家人等心中皆惶惶不定,不住的使人飛馬來往探信。有兩個時辰工夫,忽見賴大等三四個管家喘吁吁跑進儀門報喜,又說“奉老爺命,速請老太太帶領太太等進朝謝恩”等語。那時賈母正心神不定,在大堂廊下佇立,\begin{note}庚側:慈母愛子寫盡。迴廊下佇立與“日暮倚廬仍悵望”對景,餘掩卷而泣。\end{note}\begin{note}庚眉:“日暮倚廬仍悵望”,南漢先生句也。\end{note}那邢夫人、王夫人、尤氏、李紈、鳳姐、迎春姊妹以及薛姨媽等皆在一處,聽如此信至,賈母便喚進賴大來細問端的。賴大稟道:“小的們只在臨敬門外伺候,裏頭的信息一概不能得知。後來還是夏太監出來道喜,說咱們家大小姐晉封爲鳳藻宮尚書,加封賢德妃。後來老爺出來亦如此吩咐小的。如今老爺又往東宮去了,速請老太太領著太太們去謝恩。”賈母等聽了方心神安定,不免又都洋洋喜氣盈腮。\begin{note}庚側:字眼,留神。亦人之常情。\end{note}於是都按品級大妝起來。賈母帶領邢夫人、王夫人、尤氏,一共四乘大轎入朝。賈赦、賈珍亦換了朝服,帶領賈蓉、賈薔奉侍賈母大轎前往。於是寧榮兩處上下里外,莫不欣然踊躍,\begin{note}[秦氏生魂先告鳳姐矣。]\end{note}個個面上皆有得意之狀,言笑鼎沸不絕。
\end{parag}


\begin{parag}
    誰知近日水月庵的智能私逃進城,\begin{note}甲側:好筆仗,好機軸。\end{note}\begin{note}甲眉:忽然接水月庵,似大脫卸。及讀至後,方知爲緊收。此大段有如歌疾調迫之際,忽聞戛然檀板截斷,真見其大力量處,卻便於寫寶玉之文。\end{note}找至秦鍾家下看視秦鍾,不意被秦業知覺,將智能逐出,將秦鍾打了一頓,自己氣的老病發作,三五日光景鳴呼死了。秦鍾本自怯弱,又帶病未愈,受了笞杖,今見老父氣死,此時悔痛無及,更又添了許多症候。因此寶玉心中悵然如有所失。\begin{note}庚眉:凡用寶玉收拾,俱是大關鍵。\end{note}雖聞得元春晉封之事,亦未解得愁悶。\begin{note}甲雙夾:眼前多少熱鬧文字不寫,卻從萬人意外撰出一段悲傷,是別人不屑寫者,亦別人之不能處。\end{note}賈母等如何謝恩,如何回家,親朋如何來慶賀,寧榮兩處近日如何熱鬧,衆人如何得意,獨他一個皆視有如無,毫不曾介意。\begin{note}庚側:的的真真寶玉。\end{note}因此衆人嘲他越發呆了。\begin{note}甲雙夾:大奇至妙之文,卻用寶玉一人連用五“如何”,隱過多少繁華勢利等文。試思若不如此,必至種種寫到,其死板拮据、瑣碎雜亂,何可勝哉?故只借寶玉一人如此一寫,省卻多少閒文,卻有無限煙波。庚側:越發呆了。\end{note}
\end{parag}


\begin{parag}
    且喜賈璉與黛玉回來,先遣人來報信,明日就可到家,寶玉聽了,方略有些喜意。\begin{note}甲雙夾:不如此,後文秦鍾死去,將何以慰寶玉?\end{note}細問原由,方知賈雨村也進京陛見,皆由王子騰累上保本,此來後補京缺,與賈璉是同宗弟兄,又與黛玉有師從之誼,故同路作伴而來。林如海已葬入祖墳了,諸事停妥,賈璉方進京的。本該出月到家,因聞元春喜信,遂晝夜兼程而進,一路俱各平安。寶玉只聞得黛玉“平安”二字,餘者也就不在意了。\begin{note}甲雙夾:又從天外寫出一段離合來,總爲掩過寧、榮兩處許多瑣細閒筆。處處交代清楚,方好起大觀園也。\end{note}
\end{parag}


\begin{parag}
    好容易\begin{note}庚側:三字是寶玉心中。\end{note}盼至明日午錯,果報:“璉二爺和林姑娘進府了。”見面時彼此悲喜交接,未免又大哭一陣,後又致喜慶之詞。\begin{note}甲雙夾:世界上亦如此,不獨書中瞬息,觀此便可省悟。\end{note}寶玉心中品度黛玉,越發出落的超逸了。黛玉又帶了許多書籍來,忙著打掃臥室,安插器具,又將些紙筆等物分送寶釵、迎春、寶玉等人。寶玉又將北靜王所贈鶺鴒香串珍重取出來,轉贈黛玉。黛玉說:“什麼臭男人拿過的!我不要他。”遂擲而不取。寶玉只得收回,暫且無話。\begin{note}甲雙夾:略一點黛玉情性,趕忙收住,正留爲後文地步。\end{note}
\end{parag}


\begin{parag}
    且說賈璉自回家參見過衆人,回至房中。正值鳳姐近日多事之時,無片刻閒暇之工,\begin{note}甲雙夾:補阿鳳二句最不可少。\end{note}見賈璉遠路歸來,少不得撥冗接待,\begin{note}庚側:寫得尖利刻薄。\end{note}房內無外人,便笑道:“國舅老爺大喜!國舅老爺一路風塵辛苦。\begin{note}甲側:嬌音如聞,俏態如見,少年夫妻常事,的確有之。\end{note}小的聽見昨日的頭起報馬來報,說今日大駕歸府,略預備了一杯水酒撣塵,\begin{note}庚側:卻是爲下文作引。\end{note}不知賜光謬領否?”賈璉笑道:“豈敢豈敢,多承多承!”\begin{note}庚側:一言答不上,蠢才蠢才!\end{note}一面平兒與衆丫鬟參拜畢,獻茶。賈璉遂問別後家中的諸事,又謝鳳姐的操持勞碌。鳳姐道:“我那裏管得這些事!見識又淺,口角又笨,心腸又直率,人家給個棒槌,我就認作針。臉又軟,擱不住人給兩句好話,心裏就慈悲了。況且又沒經歷過大事,膽子又小,太太略有些不自在,就嚇的我連覺也睡不著了。我苦辭了幾回,太太又不容辭,倒反說我圖受用,不肯習學了。殊不知我是捻著一把汗兒呢。一句也不敢多說,一步也不敢多走。\begin{note}甲眉:此等文字,作者盡力寫來,是欲諸公認得阿鳳,好看以後之書,勿作等閒看過。\end{note}你是知道的,咱們家所有的這些管家奶奶們,那一位是好纏的?\begin{note}甲側:獨這一句不假。脂硯。\end{note}錯一點兒他們就笑話打趣,偏一點兒他們就指桑罵槐的報怨。‘坐山觀虎鬥’、‘借劍殺人’、‘引風吹火’、‘站乾岸兒’、‘推倒油瓶兒不扶’,都是全掛子的武藝。況且我年紀輕,頭等不壓衆,怨不得不放我在眼裏。更可笑\begin{note}庚側:三字是得意口氣。\end{note}那府裏忽然蓉兒媳婦死 了,珍大哥又再三再四的在太太跟前跪著討情,只要請我幫他幾日;我是再四推辭,太太斷不依,只得從命。依舊被我鬧了個馬仰人翻,\begin{note}庚側:得意之至口氣。\end{note}更不成個體統,至今珍大哥哥還報怨後悔呢。你這一來了,明兒你見了他,好歹描補描補,就說我年紀小,原沒見過世面,誰叫大爺錯委他的。”\begin{note}甲眉:阿鳳之弄璉兄如弄小兒,可思之至。\end{note}\begin{note}庚側:阿鳳之弄璉兄如弄小兒,可怕可畏!若生於小戶,落在貧家,璉兄死矣!\end{note}
\end{parag}


\begin{parag}
    正說著,\begin{note}甲雙夾:又用斷法方妙。蓋此等文斷不可無,亦不可太多。\end{note}只聽外間有人說話,鳳姐便問:“是誰?”平兒進來回道:“姨太太打發了香菱妹子來問我一句話,我已經說了,打發他回去了。”賈璉笑道:“正是呢,方纔我見姨媽去,不防和一個年輕的小媳婦子撞了個對面,生的好齊整模樣。\begin{note}庚側:酒色之徒。\end{note}我疑惑咱家並無此人,說話時因問姨媽,誰知就是上京來買的那小丫頭,名喚香菱的,竟與薛大傻子作了房裏人,開了臉,越發出挑的標緻了。那薛大傻子真玷辱了他。”\begin{note}甲雙夾:垂涎如見,試問兄寧有不玷平兒乎?脂硯。\end{note}鳳姐道:“噯!\begin{note}庚側:如聞。\end{note}往蘇杭走了一趟回來,也該見些世面了,\begin{note}甲側:這“世面”二字,單指女色也。\end{note}還是這樣眼饞肚飽的。你要愛他,不值什麼,我去拿平兒換了他來如何?\begin{note}甲側:奇談,是阿鳳口中方有此等語句。\end{note}\begin{note}甲眉:用平兒口頭謊言,寫補菱卿一項實事,並無一絲痕跡,而有作者有多少機括。\end{note}那薛老大\begin{note}甲側:又一樣稱呼,各得神理。\end{note}也是‘喫著碗裏看著鍋裏’的,這一年來的光景,他爲要香菱不能到手,\begin{note}甲側:補前文之未到,且並將香菱身分寫出。脂硯。\end{note}和姨媽打了多少饑荒。也因姨媽看著香菱模樣兒好還是末則,其爲人行事,卻又比別的女孩子不同,溫柔安靜,差不多的主子姑娘也跟他不上呢,\begin{note}甲雙夾:何曾不是主子姑娘?蓋卿不知來歷也,作者必用阿鳳一讚,方知蓮卿尊重不虛。\end{note}故此擺酒請客的費事,明堂正道的與他作了妾。過了沒半月,也看的馬棚風一般了,我倒心裏可惜了的。”\begin{note}甲雙夾:一段納寵之文,偏於阿風口中補出,亦奸猾幻妙之至!\end{note}一語未了,二門上的小廝傳報:“老爺在大書房等二爺呢。”賈璉聽了,忙忙整衣出去。
\end{parag}


\begin{parag}
    這裏鳳姐乃問平兒:“方纔姨媽有什麼事,巴巴打發了香菱來?”\begin{note}甲側:必有此一問。\end{note}平兒笑道:“那裏來的香菱,是我借他暫撒個謊。\begin{note}甲側:卿何嘗謊言?的是補菱姐正文。\end{note}奶奶說說,旺兒嫂子越發連個承算也沒了。”\begin{note}庚側:此處系平兒搗鬼。\end{note}說著,又走到鳳姐身邊,悄悄的說道:\begin{note}庚側:如聞如見。\end{note}“奶奶的那利錢銀子,遲不送來,早不送來,這會子二爺在家,他且送這個來了。\begin{note}甲側:總是補遺。\end{note}幸虧我在堂屋裏撞見,不然時走了來回奶奶,二爺倘或問奶奶是什麼利錢,奶奶自然不肯瞞二爺的,\begin{note}甲側:平姐欺看書人了。\end{note}\begin{note}庚側:可兒可兒,鳳姐竟被他哄了。\end{note}少不得照實告訴二爺。我們二爺那脾氣,油鍋裏的錢還要找出來花呢,聽見奶奶有了這個梯已,他還不放心的花了呢。所以我趕著接了過來,叫我說了他兩句,誰知奶奶偏聽見了問,我就撒謊說香菱來了。”\begin{note}甲側:雙夾:一段平兒見識作用,不枉阿鳳平日刮目,又伏下多少後文,補盡前文未到。\end{note}鳳姐聽了笑道:“我說呢,姨媽知道你二爺來了,忽刺巴的反打發個房裏人來了?原來是你這蹄子肏鬼。”\begin{note}庚側:疼極反罵。\end{note}
\end{parag}


\begin{parag}
    說話時賈璉已進來,鳳姐便命擺上酒饌來,夫妻對坐。鳳姐雖善飲,卻不敢任興,\begin{note}甲雙夾:百忙中又點出大家規範,所謂無不周詳,無不貼切。\end{note}只陪侍著賈璉。一時賈璉的乳母趙嬤嬤走來,賈璉鳳姐忙讓喫酒,令其上炕去。趙嬤嬤執意不肯。平兒等早於炕下設下一杌,又有一小腳踏,趙嬤嬤在腳踏上坐了。賈璉向桌上揀兩盤餚饌與他放在杌上自喫。鳳姐又道:“媽媽很嚼不動那個,倒沒的硌了他的牙。”\begin{note}庚側:何處著想?卻是自然有的。\end{note}因向平兒道:“早起我說那一碗火腿燉肘子很爛,正好給媽媽喫,你怎麼不拿了去趕著叫他們熱來?”又道:“媽媽,你嘗一嘗你兒子帶來的惠泉酒。”\begin{note}庚側:補點不到之文,像極!\end{note}嬤嬤道:“我喝呢,奶奶也喝一鍾,怕什麼?只不要過多了就是了。\begin{note}甲雙夾:寶玉之李嬤,此處偏又寫一趙嬤,特犯不犯。先有梨香院一回,今又寫此一回,兩兩遙對,卻無一等相重,一事合掌。\end{note}我這會子跑了來,倒也不爲飲酒,倒有一件正經事,奶奶好歹記在心裏,疼顧我些罷。我們這爺,只是嘴裏說的好,到了跟前就忘了我們。幸虧我從小兒奶了你這麼大。我也老了,有的是那兩個兒子,你就另眼照看他們些,別人也不敢呲牙兒的。\begin{note}庚側:爲薔、蓉作引。\end{note}我還再四的求了幾遍,你答應的倒好,到如今還是燥屎。\begin{note}庚側:有是乎?\end{note}這如今又從天上跑出這一件大喜事來,那裏用不著人?所以倒是來和奶奶說是正經。靠著我們爺,只怕我還餓死了呢。”
\end{parag}


\begin{parag}
    鳳姐笑道:“媽媽你放心,兩個奶哥哥都交給我。你從小兒奶的兒子,你還有什麼不知他那脾氣的?拿著皮肉倒往那不相干的外人身上貼。可是現放著奶哥哥,那一個不比人強?你疼顧照看他們,誰敢說個‘不’字兒?\begin{note}庚側:會送情。\end{note}沒的白便宜了外人。——我這話也說錯了,我們看著是‘外人’,你卻是看著‘內人’一樣呢。”\begin{note}庚側:可兒可兒!\end{note}說的滿屋裏人都笑了。嬤嬤也笑個不住,又唸佛道:“可是屋子裏跑出青天來了。若說‘內人’‘外人’這些混帳原故,我們爺是沒有,\begin{note}甲側:千真萬真,是沒有。一笑。\end{note}\begin{note}庚側:有是語,像極,畢肖。乳母護子。\end{note}不過是臉軟心慈,擱不住人求兩句罷了。”鳳姐笑道:“可不是呢,有‘內人’的他才慈軟呢,他在咱們娘兒們跟前纔是剛硬呢!”嬤嬤笑道:“奶奶說的太盡情了,我也樂了,再喫一杯好酒。從此我們奶奶作了主,我就沒的愁了。”
\end{parag}


\begin{parag}
    賈璉此時沒好意思,只是訕笑喫酒,說“胡說”二字,“快盛飯來,喫碗子還要往珍大爺那邊去商議事呢。”鳳姐道:“可是別誤了正事。纔剛老爺叫你作什麼?”\begin{note}庚雙夾:一段趙嫗討情閒文,卻引出通部脈絡。所謂由小及大,譬如登高必自卑之意。細思大觀園一事,若從如何奉旨起造,又如何分派衆人,從頭細細直寫將來,幾千樣細事,如何能順筆一氣寫清?又將落於死板拮据之鄉,放只用璉鳳夫妻二人一問一答,上用趙嫗討情作引,下文蓉薔來說事作收,餘者隨筆順筆略一點染,則耀然洞徹矣。此是避難法。\end{note}賈璉道:“就爲省親。”\begin{note}甲雙夾:二字醒眼之極,卻只如此寫來。\end{note}\begin{note}甲眉:大觀園用省親事出題,是大關鍵事,方見大手筆行文之立意。畸笏。\end{note}鳳姐忙問道:\begin{note}甲雙夾:“忙”字最要緊,特於鳳姐口中出此字,可知事關巨要,非同淺細,是此書中正眼矣。\end{note}“省親的事竟準了不成?”\begin{note}甲雙夾:問得珍重,可知是外方人意外之事。脂硯。\end{note}賈璉笑道:“雖不十分準,也有八分準了。”\begin{note}甲雙夾:如此故頓一筆,更妙!見得事關重大,非一語可了者,亦是大篇文章,抑揚頓挫之至。\end{note}鳳姐笑道:“可見當今的隆恩。歷來聽書看戲,古時從未有的。”\begin{note}甲雙夾:於閨閣中作此語,直與擊壤同聲。脂硯。\end{note}趙嬤嬤又接口道:“可是呢,我也老糊塗了。我聽見上上下下吵嚷了這些日子,什麼省親不省親,我也不理論他去;如今又說省親,到底是怎麼個原故?”\begin{note}甲側:補近日之事,啓下回之文。\end{note}\begin{note}甲眉:趙嬤一問是文章家進一步門庭法則。\end{note}\begin{note}庚眉:自政老生日用降旨截住,賈母等進朝如此熱鬧,用秦業死岔開,只寫幾個“如何”,將潑天喜事交代完了,緊接黛玉回,璉、鳳閒話,以老嫗勾出省親事來。其千頭萬緒,合榫貫連,無一毫痕跡,如此等,是書多多,不能枚舉。想兄在青埂蜂上,經鍛鍊後,參透重關至恆河沙數,如否?餘曰:萬不能有此機括,有此筆力,恨不得面問果否。嘆嘆!丁亥春。笏叟。\end{note}賈璉道:\begin{note}甲側:大觀園一篇大文,千頭萬緒,從何處寫起,今故用賈璉夫妻問答之間,閒閒敘出,觀者已省大半。後再用蓉、薔二人重一渲染。便省卻多少贅瘤筆墨。此是避難法。\end{note}“如今當今貼體萬人之心,世上至大莫如‘孝’字,想來父母兒女之性,皆是一理,不是貴賤上分別的。當今自爲日夜侍奉太上皇、皇太后,尚不能略盡孝意,因見宮裏嬪妃才人等皆是入宮多年,拋離父母音容,豈有不思想之理?在兒女思想父母,是分所應當。想父母在家,若只管思念兒女,竟不能見,倘因此成疾致病,甚至死亡,皆由朕躬禁錮,不能使其遂天倫之願,亦大傷天和之事。故啓奏太上皇、皇太后,每月逢二六日期,準其椒房眷屬入宮請候看視。於是太上皇、皇太后大喜,深贊當今至孝純仁,體天格物。因此二位老聖人又下旨意,說椒房眷屬入宮,未免有國體儀制,母女尚不能愜懷。竟大開方便之恩,特降諭諸椒房貴戚,除二六日入宮之恩外,凡有重宇別院之家,可以駐蹕關防之處,不妨啓請內廷鑾輿入其私第,庶可略盡骨肉私情、天倫中之至性。此旨一下,誰不踊躍感戴?現今周貴人父親已在家裏動了工了,修蓋省親別院呢。又有吳貴妃的父親吳天佑家,也往城外踏看地方去了。\begin{note}甲側:又一樣佈置。\end{note}這豈非有八九分了?”
\end{parag}


\begin{parag}
    趙嬤嬤道:“阿彌陀佛!原來如此。這樣說,咱們家也要預備接咱們大小姐了?”\begin{note}庚側:文忠公之嬤。\end{note}賈璉道:“這何用說呢!不然,這會子忙的是什麼?”\begin{note}甲側:一段閒談中補明多少文章。真是費長房壺中天地也。\end{note}鳳姐笑道:“若果如此,我可也見個大世面了。可恨我小几歲年紀,若早生二三十年,如今這些老人家也不薄我沒見世面了。\begin{note}甲側:忽接入此句,不知何意,似屬無謂。\end{note}說起當年太祖皇帝仿舜巡的故事,比一部書還熱鬧,\begin{note}庚側:既知舜巡而又說熱鬧,此婦人女子口頭也。\end{note}我偏沒造化趕上。”\begin{note}庚側:不用忙,往後看。\end{note}趙嬤嬤道:“噯喲喲,那可是千載希逢的!那時候我才記事兒,咱們賈府正在姑蘇揚州一帶監造海舫,修理海塘,只預備接駕一次,\begin{note}庚側:又要瞞人。\end{note}把銀子都花的像倘海水似的!說起來……”鳳姐忙接道:\begin{note}甲側:又截得好。“忙”字妙!上文“說起來”必未完,粗心看去則說疑團,殊不知正傳神處。\end{note}“我們王府也預備過一次。那時候我爺爺單管各國進貢朝賀的事,凡有的外國人來,都是我們家養活。\begin{note}甲側:點出阿鳳所有外國奇玩等物。\end{note}粵、閩、滇、浙所有的洋船貨物都是我們家的。”
\end{parag}


\begin{parag}
    趙嬤嬤道:“那是誰不知道的?如今還有個口號兒呢,說‘東海少了白玉牀,龍王來請江南王’,\begin{note}庚側:應前“葫蘆案”。\end{note}這說的就是奶奶府上了。還有如今現在江南的甄家,\begin{note}甲側:甄家正是大關鍵、大節目,勿作泛泛口頭語看。\end{note}噯喲喲,\begin{note}庚側:口氣如聞。\end{note}好勢派!獨他家接駕四次,\begin{note}庚側:點正題正文。\end{note}若不是我們親眼看見,告訴誰誰也不信的。別講銀子成了土泥,\begin{note}庚側:極力一寫,非誇也,可想而知。\end{note}憑是世上所有的,沒有不是堆山塞海的,‘罪過可惜’四個字竟顧不得了。”\begin{note}庚側:真有是事,經過見過。\end{note}鳳姐道:“常聽見我們太爺們也這樣說,豈有不信的。\begin{note}庚側:對證。\end{note}只納罕他家怎麼就這麼富貴呢?”趙嬤嬤道:“告訴奶奶一句話,也不過拿著皇帝家的銀子往皇帝身上使罷了!\begin{note}甲側:是不忘本之言。\end{note}誰家有那些錢買這個虛熱鬧去?”\begin{note}甲側:最要緊語。人苦不自知。能作是語者吾未嘗見。\end{note}
\end{parag}


\begin{parag}
    正說的熱鬧,王夫人又打發了來瞧鳳姐吃了飯不曾。鳳姐便知有事等他,忙忙的吃了半碗飯,漱口要走,\begin{note}庚側:好頓挫。\end{note}又有二門上小廝們回:“東府裏蓉、薔二位哥兒來了。”賈璉才漱了口,平兒捧著盆盥手,見他二人來了,便問:“什麼話?快說。”鳳姐且止步稍候,聽他二人回些什麼。賈蓉先回說:“我父親打發我來回叔叔:老爺們已經議定了,\begin{note}庚側:簡淨之至!\end{note}從東邊一帶,藉著東府裏花園起,轉至北邊,一共丈量準了,三里半大,可以蓋造省親別院了。\begin{note}庚側:園基乃一部之主,必當如此寫清。\end{note}已經傳人畫圖樣去了,\begin{note}庚側:後一圖伏線。大觀園系玉兄與十二釵之太虛幻境,豈可草率?\end{note}明日就得。叔叔纔回家,未免勞乏,不用過我們那邊去,\begin{note}庚側:應前賈璉口中。\end{note}有話明日一早再請過去面議。”賈璉笑著忙說:“多謝大爺費心體諒,我就不過去了。正經是這個主意才省事,蓋造也容易;若採置別處地方去,那更費事,且倒不成體統。你回去說這樣很好,若老爺們再要改時,全仗大爺諫阻,萬不可另尋地方。明日一早我給大爺去請安去,再議細主。”賈蓉忙應幾個“是”。\begin{note}庚側:園已定矣。\end{note}
\end{parag}


\begin{parag}
    賈薔又近前回說:“下姑蘇聘請教習,採買女孩子,置辦樂器行頭等事,大爺派了侄兒,\begin{note}庚側:“畫薔”一回伏線。\end{note}帶領著來管家兩個兒子,還有單聘仁、卜固修兩個清客相公,一同前去,所以命我來見叔叔。”\begin{note}庚側:凡各物事工價重大,兼伏隱著情字者,莫如此件。故園定後便先寫此一件,餘便不必細寫矣。\end{note}賈璉聽了,將賈薔打諒了打諒,\begin{note}庚側:有神。\end{note}笑道:“你能在這一行麼?\begin{note}庚側:勾下文。\end{note}這個事雖不算甚大,裏頭大有藏掖的。”\begin{note}甲側:射利人微露心跡。庚側:射利語,可嘆!是親侄。\end{note}賈薔笑道:“只好學習著辦罷了。”
\end{parag}


\begin{parag}
    賈蓉在身旁燈影下悄拉鳳姐的衣襟,鳳姐會意,因笑道:“你也太操心了,難道你父親比你還不會用人?”“偏你又怕他不在行了。誰都是在行的?孩子們已長的這麼大了,‘沒喫過豬肉,也看見過豬跑’。大爺派他去,原不過是個坐纛旗兒,難道認真的叫他講價錢會經紀去呢!依我說就很好。”賈璉道:“自然是這樣。並不是我駁回,少不得替他算計算計。”因問:“這一項銀子動那一處的?賈薔道:”才也議到這裏。賴爺爺\begin{note}甲側:此等稱呼,令人酸鼻。\end{note}\begin{note}庚側:好稱呼。\end{note}說,不用從京裏帶下去,江南甄家還收著我們五萬銀子。明日寫一封書信會票我們帶去,先支三萬,下剩二萬存著,等置辦花燭簾櫳帳縵的使費。”賈璉點頭道:”這個主意好。“\begin{note}庚眉:《石頭記》中多作心傳神會之文,不必道明。一道明白,便入庸俗之套。\end{note}
\end{parag}


\begin{parag}
    鳳姐忙向賈薔道:\begin{note}甲側:再不略讓一步,正是阿鳳一生短處。脂硯。\end{note}“既這樣,我有兩個在行妥當人,你就帶他們去辦,這個便宜了你呢。”賈薔忙陪笑說:“正要和嬸嬸討兩個人呢,\begin{note}甲側:寫賈薔乖處。脂硯。\end{note}這可巧了。”因問名字。鳳姐便問趙嬤嬤。彼時趙嬤嬤已聽呆了話,平兒忙笑推他,\begin{note}蒙側:真是強將手下無弱兵。至精至細。\end{note}他才醒悟過來,忙說:“一個叫趙天梁,一個叫趙天棟。”鳳姐道:“可別忘了,我可幹我的去了。”說著便出去了。賈蓉忙送出來,又悄悄的向鳳姐道:“嬸子要什麼東西,吩咐我開個帳給薔兄弟帶了去,叫他按帳置辦了來。”鳳姐笑\begin{note}庚側:有神。\end{note}道:“別放你孃的屁!\begin{note}庚側:像極,的是阿鳳。\end{note}我的東西還沒處撂呢,稀罕你們鬼鬼崇崇的?”說著一逕去了。\begin{note}甲側:阿鳳欺人處如此。忽又寫到利弊,真令人一嘆。脂硯。\end{note}\begin{note}甲眉:從頭至尾細看阿鳳之待蓉、薔,可爲一體一黨,然尚作如此語欺蓉,其待他人可知矣。\end{note}
\end{parag}


\begin{parag}
    這裏賈薔也悄問賈璉:“要什麼東西?順便織來孝敬。”賈璉笑道:“你別興頭。才學著辦事,倒先學會了這把戲。我短了什麼,少不得寫信來告訴你,\begin{note}庚側:又作此語,不犯阿鳳。\end{note}且不要論到這裏。”說畢,打發他二人去了。接著回事的人來,不止三四次,賈璉害乏,便傳與二門上,一應不許傳報,俱等明日料理。鳳姐至三更時分方下來安歇,\begin{note}庚側:好文章,一句內隱兩處若許事情。\end{note}一宿無話。
\end{parag}


\begin{parag}
    次早賈璉起來,見過賈赦賈政,便往寧府中來,合同老管事的人等,並幾位世交門下清客相公,審察兩府地方,繕畫省親殿宇,一面察度辦理人丁。自此後,各行匠役齊集,金銀銅錫以及土木磚瓦之物,搬運移送不歇。\begin{note}蒙側:一總。\end{note}先令匠人拆寧府會芳園牆垣樓閣,直接入榮府東大院中。榮府東邊所有下人一帶羣房盡已拆去。當日寧榮二宅,雖有一小巷界斷不通,\begin{note}甲側:補明,使觀者如身臨足到。\end{note}然這小巷亦系私地,並非官道,故可以連屬。會芳園本是從北拐角牆下引來一股活水,今亦無煩再引。\begin{note}甲側:園中諸景,最要緊是水,亦必寫明方妙。餘最鄙近之修造園亭者,徒以頑石土堆爲佳,不知引泉一道。甚至丹青,唯知亂作山石樹木,不知畫泉之法,亦是恨事。脂硯齋。\end{note}其山石樹木雖不敷用,賈赦住的乃是榮府舊園,其中竹樹山石以及亭榭欄杆等物,皆可挪就前來。如此兩處又甚近,湊來一處,省得許多財力,縱亦不敷,所添亦有限。全虧一個老明公號山子野\begin{note}甲側:妙號,隨事生名。\end{note}者,一一籌劃起造。
\end{parag}


\begin{parag}
    賈政不慣於俗務,\begin{note}庚側:這也少不得的一節文字,省下筆來好作別樣。\end{note}只憑賈赦、賈珍、賈璉、賴大、來升、林之孝、吳新登、詹光、程日興等幾人安插擺佈。凡堆山鑿池,起樓豎閣,種竹裁花,一應點景等事,又有山子野制度。下朝閒暇,不過各處看望看望,最要緊處和賈赦等商議商議便罷了。賈赦只在家高臥,有芥豆之事,賈珍等或自去回明,或寫略節;或有話說,便傳呼賈璉、賴大等來領命。賈蓉單管打造金銀器皿。\begin{note}蒙側:好差。\end{note}賈薔已起身往姑蘇去了。賈珍、賴大等又點人丁,開冊籍,監工等事,一筆不能寫到,不過一時喧闐熱鬧非常而已。暫且無話。
\end{parag}


\begin{parag}
    且說寶玉近因家中有這等大事,賈政不來問他的書,\begin{note}庚側:一筆不漏。\end{note}心中是件暢事;無奈秦鍾之病日重一日,也著實懸心,不能樂業。\begin{note}甲側:“天下本無事,庸人自擾之”,世上人個個如此,又非此情鐘意切。\end{note}\begin{note}甲眉:偏於極熱鬧處寫出大不得意之文,卻無絲毫牽強,且有許多令人笑不了、哭不了、嘆不了、悔不了,唯以大白酬我作者。壬午季春。畸笏。\end{note}這日一早起來才梳洗畢,意欲回了賈母去望候秦鍾,忽見茗煙在二門照壁前探頭縮腦,寶玉忙出來問他:“作什麼?”茗煙道:“秦相公不中用了!”\begin{note}甲側:從茗煙口中寫出,省卻多少閒文。\end{note}寶玉聽說,嚇了一跳,忙問道:“我昨兒才瞧了他來,\begin{note}庚側:點常去。\end{note}還明明白白,怎麼就不中用了?”茗煙道:“我也不知道,纔剛是他家的老頭子來特告訴我的。”寶玉聽了,忙轉身回明賈母。賈母吩咐:“好生派妥當人跟去,到那裏盡一盡同窗之情就回來,不許多耽擱了。”寶玉聽了,忙忙的更衣出來,車猶未備,\begin{note}甲側:頓一筆方不板。\end{note}急的滿廳亂轉。一時催促的車到,忙上了車,李貴、茗煙等跟隨。來至秦鍾門首,悄無一人,\begin{note}甲側:目睹蕭條景況。\end{note}遂蜂擁至內室,唬的秦鐘的兩個遠房嬸母並幾個弟兄都藏之不迭。\begin{note}甲側:妙!這嬸母兄弟是特來等分絕戶傢俬的,不表可知。\end{note}
\end{parag}


\begin{parag}
    此時秦鍾已發過兩三次昏了,移牀易簀多時矣。\begin{note}甲側:餘亦欲哭。\end{note}寶玉一見,便不禁失聲。李貴忙勸道:“不可不可,秦相公是弱症,未免炕上挺扛的骨頭不受用,\begin{note}庚側:李貴亦能道此等語。\end{note}所以暫且挪下來鬆散些。哥兒如此,豈不反添了他的病。”寶玉聽了,方忍住近前,見秦鐘面如白蠟,合目呼吸於枕上。寶玉忙叫道:“鯨兄!寶玉來了。”連叫兩三聲,秦鐘不睬。寶玉又道:“寶玉來了。”那秦鍾早已魂魄離身,只剩得一口悠悠餘氣在胸,正見許多鬼判持牌提索來捉他。\begin{note}甲側:看至此一句令人失望,再看至後面數語,方知作者故意借世俗愚談愚論設譬,喝醒天下迷人,翻成千古未見之奇文奇筆。\end{note}\begin{note}庚眉:《石頭記》一部中皆是近情近理必有之事,必有之言。又如此等荒唐不經之談,間亦有之,是作者故意遊戲之筆,聊以破色取笑,非如別書認真說鬼話也。\end{note}那秦鍾魂魄那裏肯就去,又記念著家中無人掌管家務,\begin{note}甲側:扯淡之極,令人發一大笑。餘請諸公莫笑,且請再思。\end{note}又記掛著父親還有留積下的三四千兩銀子,\begin{note}甲雙夾:更屬可笑,更可痛哭。\end{note}又記掛著智能尚無下落,\begin{note}甲雙夾:忽從死人心中補出活人原由,更奇更奇。\end{note}因此百般求告鬼判。無奈這些鬼判都不肯徇私,反叱吒秦鍾道:“虧你還是讀過書人,豈不知俗語說的:‘閻王叫你三更死,誰敢留人到五更。’\begin{note}庚眉:可想鬼不讀書,信矣哉!\end{note}我們陰間上下都是鐵面無私的,不比你們陽間瞻情顧意,\begin{note}庚側:寫殺了。\end{note}有許多的關礙處。”正鬧著,那秦鍾魂魄忽聽見“寶玉來了”四字,便忙又央求道:“列位神差,略發慈悲,讓我回去,和這一個好朋友說一句話就來的。”衆鬼道:“又是什麼好朋友?”秦鍾道:“不瞞列位,就是榮國公的孫子,小名寶玉。”都判官聽了,先就唬慌起來,忙喝罵鬼使道:“我說你們放了他回去走走罷,你們斷不依我的話,如今只等他請出個運旺時盛的人來才罷。”\begin{note}甲雙夾:如聞其聲,試問誰曾見都判來,觀此則又見一都判跳出來。調侃世情固深,然遊戲筆墨一至於此,真可壓倒古今小說。這纔算是小說。\end{note}衆鬼見都判如此,也都忙了手腳,一面又報怨道:“你老人家先是那等雷霆電雹,原來見不得‘寶玉’二字。\begin{note}甲側:調侃“寶玉”二字,極妙!脂硯。\end{note}\begin{note}甲眉:世人見“寶玉”而不動心者爲誰?\end{note}依我們愚見,他是陽,我們是陰,怕他們也無益於我們。”\begin{note}甲側:神鬼也講有益無益。\end{note}\begin{note}列:此章無非笑趨勢之人。\end{note}都判道:“放屁!俗語說的好,‘天下官管天下事’,自古人鬼之道卻是一般,陰陽並無二理。\begin{note}庚雙夾:更妙!愈不通愈妙,愈錯會意愈奇。脂硯。\end{note}別管他陰也罷,陽也罷,還是把他放回沒有錯了的。”\begin{note}庚側:名曰搗鬼。\end{note}衆鬼聽說,只得將秦魂放回,哼了一聲,微開雙目,見寶玉在側,乃勉強嘆道:“怎麼不肯早來?\begin{note}庚側:千言萬語只此一句。\end{note}再遲一步也不能見了。”寶玉忙攜手垂淚道:“有什麼話留下兩句。”\begin{note}庚雙夾:只此句便足矣。\end{note}秦鍾道:“並無別話。以前你我見識自爲高過世人,我今日才知自誤了。\begin{note}庚雙夾:誰不悔遲!\end{note}以後還該立志功名,以榮耀顯達爲是。”\begin{note}庚側:此刻無此二語,亦非玉兄之知己。\end{note}\begin{note}庚眉:觀者至此,必料秦鍾另有異樣奇語,然卻只以此二語爲囑。試思若不如此爲囑,不但不近人情,亦且太露穿鑿。讀此則知全是悔遲之恨。\end{note}說畢,便長嘆一聲,蕭然長逝了。\begin{note}庚雙夾:若是細述一番,則不成《石頭記》之文矣。\end{note}
\end{parag}


\begin{parag}
    \begin{note}蒙回末總:大凡有勢者未嘗有意欺人。然羣小蜂起,浸潤左右,伏首下氣,奴顏悲膝,或激或順,不計事之可否,以要一時之利。有勢者自任豪爽,鬥露才華,未審利害,高下其手,偶有成就,一試再試,習以爲常,則物理人情皆所不論。又財貨豐餘,衣食無憂,則所樂者必曠世所無。要其必獲,一笑百萬,是所不惜。其不知排場已立,收斂實難,從此勉強,至成蹇窘,時衰運敗,百計顛翻。昔年豪爽,今朝指背。此千古英雄同一慨嘆者。大抵作者發大慈大悲願,欲諸公開巨眼,得見毫微,塞本窮源,以成無礙極樂之至意也。\end{note}
\end{parag}
