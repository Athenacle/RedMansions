\chap{一百一十四}{王熙鳳歷幻返金陵 甄應嘉蒙恩還玉闕}



\begin{parag}
    卻說寶玉寶釵聽說鳳姐病的危急,趕忙起來。丫頭秉燭伺候。正要出院,只見王夫人那邊打發人來說:“璉二奶奶不好了,還沒有嚥氣,二爺二奶奶且慢些過去罷。璉二奶奶的病有些古怪,從三更天起到四更時候,璉二奶奶沒有住嘴說些胡話,要船要轎的,說到金陵歸入冊子去。衆人不懂,他只是哭哭喊喊的。璉二爺沒有法兒,只得去糊了船轎,還沒拿來,璉二奶奶喘着氣等呢。叫我們過來說,等璉二奶奶去了再過去罷。”寶玉道:“這也奇,他到金陵做什麼?”襲人輕輕的和寶玉說道:“你不是那年做夢,我還記得說有多少冊子,不是璉二奶奶也到那裏去麼?”寶玉聽了點頭道:“是呀,可惜我都不記得那上頭的話了。這麼說起來,人都有個定數的了。但不知林妹妹又到那裏去了?我如今被你一說,我有些懂得了。若再做這個夢時,我得細細的瞧一瞧,便有未卜先知的分兒了。”襲人道:“你這樣的人可是不可和你說話的,偶然提了一句,你便認起真來了嗎?就算你能先知了,你有什麼法兒!”寶玉道:“只怕不能先知,若是能了,我也犯不着爲你們瞎操心了。”
\end{parag}


\begin{parag}
    兩個正說着,寶釵走來問道:“你們說什麼?”寶玉恐他盤詰,只說:“我們談論鳳姐姐。”寶釵道:“人要死了,你們還只管議論人。舊年你還說我咒人,那個籤不是應了麼?”寶玉又想了一想,拍手道:“是的,是的。這麼說起來,你倒能先知了。我索性問問你,你知道我將來怎麼樣?”寶釵笑道:“這是又胡鬧起來了。我是就他求的簽上的話混解的,你就認了真了。你就和邢妹妹一樣的了,你失了玉,他去求妙玉扶乩,批出來的衆人不解,他還背地裏和我說妙玉怎麼前知,怎麼參禪悟道。如今他遭此大難,他如何自己都不知道,這可是算得前知嗎?就是我偶然說着了二奶奶的事情,其實知道他是怎麼樣了,只怕我連我自己也不知道呢。這樣下落可不是虛誕的事,是信得的麼!”寶玉道:“別提他了。你只說邢妹妹罷,自從我們這裏連連的有事,把他這件事竟忘記了。你們家這麼一件大事怎麼就草草的完了,也沒請親喚友的。”寶釵道:“你這話又是迂了。我們家的親戚只有咱們這裏和王家最近。王家沒了什麼正經人了。咱們家遭了老太太的大事,所以也沒請,就是璉二哥張羅了張羅。別的親戚雖也有一兩門子,你沒過去,如何知道。算起來我們這二嫂子的命和我差不多,好好的許了我二哥哥,我媽媽原想體體面面的給二哥哥娶這房親事的。一則爲我哥哥在監裏,二哥哥也不肯大辦,二則爲咱家的事,三則爲我二嫂子在大太太那邊忒苦,又加着抄了家,大太太是苛刻一點的,他也實在難受:所以我和媽媽說了,便將將就就的娶了過去。我看二嫂子如今倒是安心樂意的孝敬我媽媽,比親媳婦還強十倍呢。待二哥哥也是極盡婦道的,和香菱又甚好,二哥哥不在家,他兩個和和氣氣的過日子。雖說是窮些,我媽媽近來倒安逸好些。就是想起我哥哥來不免悲傷。況且常打發人家裏來要使用,多虧二哥哥在外頭帳頭兒上討來應付他的。我聽見說城裏有幾處房子已經典去,還剩了一所在那裏,打算着搬去住。”寶玉道:“爲什麼要搬?住在這裏你來去也便宜些,若搬遠了,你去就要一天了。”寶釵道:“雖說是親戚,倒底各自的穩便些。那裏有個一輩子住在親戚家的呢。”
\end{parag}


\begin{parag}
    寶玉還要講出不搬去的理,王夫人打發人來說:“璉二奶奶嚥了氣了。所有的人多過去了,請二爺二奶奶就過去。”寶玉聽了,也掌不住跺腳要哭。寶釵雖也悲慼,恐寶玉傷心,便說:“有在這裏哭的,不如到那邊哭去。”於是兩人一直到鳳姐那裏。只見好些人圍着哭呢。寶釵走到跟前,見鳳姐已經停牀,便大放悲聲。寶玉也拉着賈璉的手大哭起來。賈璉也重新哭泣。平兒等因見無人勸解,只得含悲上來勸止了。衆人都悲哀不止。賈璉此時手足無措,叫人傳了賴大來,叫他辦理喪事。自己回明瞭賈政去,然後行事。但是手頭不濟,諸事拮据,又想起鳳姐素日來的好處,更加悲哭不已,又見巧姐哭的死去活來,越發傷心。哭到天明,即刻打發人去請他大舅子王仁過來。那王仁自從王子騰死後,王子勝又是無能的人,任他胡爲,已鬧的六親不和。今知妹子死了,只得趕着過來哭了一場。見這裏諸事將就,心下便不舒服,說:“我妹妹在你家辛辛苦苦當了好幾年家,也沒有什麼錯處,你們家該認真的發送發送纔是。怎麼這時候諸事還沒有齊備!”賈璉本與王仁不睦,見他說些混賬話,知他不懂的什麼,也不大理他。王仁便叫了他外甥女兒巧姐過來說:“你娘在時,本來辦事不周到,只知道一味的奉承老太太,把我們的人都不大看在眼裏。外甥女兒,你也大了,看見我曾經沾染過你們沒有!如今你娘死了,諸事要聽着舅舅的話。你母親孃家的親戚就是我和你二舅舅了。你父親的爲人我也早知道的了,只有重別人,那年什麼尤姨娘死了,我雖不在京,聽見人說花了好些銀子。如今你娘死了,你父親倒是這樣的將就辦去嗎!你也不快些勸勸你父親。”巧姐道:“我父親巴不得要好看,只是如今比不得從前了。現在手裏沒錢,所以諸事省些是有的。”王仁道:“你的東西還少麼!”巧姐兒道:“舊年抄去,何嘗還了呢。”王仁道:“你也這樣說。我聽見老太太又給了好些東西,你該拿出來。”巧姐又不好說父親用去,只推不知道。王仁便道:“哦,我知道了,不過是你要留着做嫁妝罷咧。”巧姐聽了,不敢回言,只氣得哽噎難鳴的哭起來了。平兒生氣說道:“舅老爺有話,等我們二爺進來再說,姑娘這麼點年紀,他懂的什麼。”王仁道:“你們是巴不得二奶奶死了,你們就好爲王了。我並不要什麼,好看些也是你們的臉面。”說着,賭氣坐着。巧姐滿懷的不舒服,心想:“我父親並不是沒情,我媽媽在時舅舅不知拿了多少東西去,如今說得這樣乾淨。”於是便不大瞧得起他舅舅了。豈知王仁心裏想來,他妹妹不知攢積了多少,雖說抄了家,那屋裏的銀子還怕少嗎。”必是怕我來纏他們,所以也幫着這麼說,這小東西兒也是不中用的。”從此王仁也嫌了巧姐兒了。
\end{parag}


\begin{parag}
    賈璉並不知道,只忙着弄銀錢使用。外頭的大事叫賴大辦了,裏頭也要用好些錢,一時實在不能張羅。平兒知他着急,便叫賈璉道:“二爺也別過於傷了自己的身子。”賈璉道:“什麼身子,現在日用的錢都沒有,這件事怎麼辦!偏有個糊塗行子又在這裏蠻纏,你想有什麼法兒!”平兒道:“二爺也不用着急,若說沒錢使喚,我還有些東西舊年幸虧沒有抄去,在裏頭。二爺要就拿去當着使喚罷。”賈璉聽了,心想難得這樣,便笑道:“這樣更好,省得我各處張羅。等我銀子弄到手了還你。”平兒道:“我的也是奶奶給的,什麼還不還,只要這件事辦的好看些就是了。”賈璉心裏倒着實感激他,便將平兒的東西拿了去當錢使用,諸凡事情便與平兒商量。秋桐看着心裏就有些不甘,每每口角里頭便說:“平兒沒有了奶奶,他要上去了。我是老爺的人,他怎麼就越過我去了呢。”平兒也看出來了,只不理他。倒是賈璉一時明白,越發把秋桐嫌了,一時有些煩惱便拿着秋桐出氣。邢夫人知道,反說賈璉不好。賈璉忍氣。不題。
\end{parag}


\begin{parag}
    再說鳳姐停了十餘天,送了殯。賈政守着老太太的孝,總在外書房。那時清客相公漸漸的都辭去了,只有個程日興還在那裏,時常陪着說說話兒。提起“家運不好,一連人口死了好些,大老爺和珍大爺又在外頭,家計一天難似一天。外頭東莊地畝也不知道怎麼樣,總不得了呀!”程日興道:“我在這裏好些年,也知道府上的人那一個不是肥己的。一年一年都往他家裏拿,那自然府上是一年不夠一年了。又添了大老爺珍大爺那邊兩處的費用,外頭又有些債務,前兒又破了好些財,要想衙門裏緝賊追贓是難事。老世翁若要安頓家事,除非傳那些管事的來,派一個心腹的人各處去清查清查,該去的去,該留的留,有了虧空着在經手的身上賠補,這就有了數兒了。那一座大的園子人家是不敢買的。這裏頭的出息也不少,又不派人管了。那年老世翁不在家,這些人就弄神弄鬼兒的,鬧的一個人不敢到園裏。這都是家人的弊。此時把下人查一查,好的使着,不好的便攆了,這纔是道理。”賈政點頭道:“先生你所不知,不必說下人,便是自己的侄兒也靠不住。若要我查起來,那能一一親見親知。況我又在服中,不能照管這些了。我素來又兼不大理家,有的沒的,我還摸不着呢。”程日興道:“老世翁最是仁德的人,若在別家的,這樣的家計,就窮起來,十年五載還不怕,便向這些管家的要也就夠了。我聽見世翁的家人還有做知縣的呢。”賈政道:“一個人若要使起家人們的錢來,便了不得了,只好自己儉省些。但是冊子上的產業,若是實有還好,生怕有名無實了。”程日興道:“老世翁所見極是。晚生爲什麼說要查查呢!”賈政道:“先生必有所聞。”程日興道:“我雖知道些那些管事的神通,晚生也不敢言語的。”賈政聽了,便知話裏有因,便嘆道:“我自祖父以來都是仁厚的,從沒有刻薄過下人。我看如今這些人一日不似一日了。在我手裏行出主子樣兒來,又叫人笑話。”
\end{parag}


\begin{parag}
    兩人正說着,門上的進來回道:“江南甄老爺到來了。”賈政便問道:“甄老爺進京爲什麼?”那人道:“奴才也打聽了,說是蒙聖恩起復了。”賈政道:“不用說了,快請罷。”那人出去請了進來。那甄老爺即是甄寶玉之父,名叫甄應嘉,表字友忠,也是金陵人氏,功勳之後。原與賈府有親,素來走動的。因前年掛誤革了職,動了家產。今遇主上眷念功臣,賜還世職,行取來京陛見。知道賈母新喪,特備祭禮擇日到寄靈的地方拜奠,所以先來拜望。賈政有服不能遠接,在外書房門口等着。那位甄老爺一見,便悲喜交集,因在制中不便行禮,便拉着了手敘了些闊別思念的話,然後分賓主坐下,獻了茶,彼此又將別後事情的話說了。賈政問道:“老親翁幾時陛見的?”甄應嘉道:“前日。”賈政道:“主上隆恩,必有溫諭。”甄應嘉道:“主上的恩典真是比天還高,下了好些旨意。”賈政道:“什麼好旨意?”甄應嘉道:“近來越寇猖獗,海疆一帶小民不安,派了安國公征剿賊寇。主上因我熟悉土疆,命我前往安撫,但是即日就要起身。昨日知老太太仙逝,謹備瓣香至靈前拜奠,稍盡微忱。”賈政即忙叩首拜謝,便說:“老親翁即此一行,必是上慰聖心,下安黎庶,誠哉莫大之功,正在此行。但弟不克親睹奇才,只好遙聆捷報。現在鎮海統制是弟舍親,會時務望青照。”甄應嘉道:“老親翁與統制是什麼親戚?”賈政道:“弟那年在江西糧道任時,將小女許配與統制少君,結褵已經三載。因海口案內未清,繼以海寇聚奸,所以音信不通。弟深念小女,俟老親翁安撫事竣後,拜懇便中請爲一視。弟即修數行煩尊紀帶去,便感激不盡了。”甄應嘉道:“兒女之情,人所不免,我正在有奉託老親翁的事。日蒙聖恩召取來京,因小兒年幼,家下乏人,將賤眷全帶來京。我因欽限迅速,晝夜先行,賤眷在後緩行,到京尚需時日。弟奉旨出京,不敢久留。將來賤眷到京,少不得要到尊府,定叫小犬叩見。如可進教,遇有姻事可圖之處,望乞留意爲感。”賈政一一答應。那甄應嘉又說了幾句話,就要起身,說:“明日在城外再見。”賈政見他事忙,諒難再坐,只得送出書房。
\end{parag}


\begin{parag}
    賈璉寶玉早已伺候在那裏代送,因賈政未叫,不敢擅入。甄應嘉出來,兩人上去請安。應嘉一見寶玉,呆了一呆,心想:“這個怎麼甚象我家寶玉?只是渾身縞素。”因問:“至親久闊,爺們都不認得了。”賈政忙指賈璉道:“這是家兄名赦之子璉二侄兒。”又指着寶玉道:“這是第二小犬,名叫寶玉。”應嘉拍手道奇:“我在家聽見說老親翁有個銜玉生的愛子,名叫寶玉。因與小兒同名,心中甚爲罕異。後來想着這個也是常有的事,不在意了。豈知今日一見,不但面貌相同,且舉止一般,這更奇了。”問起年紀,比這裏的哥兒略小一歲。賈政便因提起承屬包勇,問及令郎哥兒與小兒同名的話述了一遍。應嘉因屬意寶玉,也不暇問及那包勇的得妥,只連連的稱道:“真真罕異!”因又拉了寶玉的手,極致殷勤。又恐安國公起身甚速,急須預備長行,勉強分手徐行。賈璉寶玉送出,一路又問了寶玉好些的話。及至登車去後,賈璉寶玉回來見了賈政,便將應嘉問的話回了一遍。
\end{parag}


\begin{parag}
    賈政命他二人散去。賈璉又去張羅算明鳳姐喪事的賬目。寶玉回到自己房中,告訴了寶釵,說是:“常提的甄寶玉,我想一見不能,今日倒先見了他父親了。我還聽得說寶玉也不日要到京了,要來拜望我老爺呢。又人人說和我一模一樣的,我只不信。若是他後兒到了咱們這裏來,你們都去瞧去,看他果然和我象不象。”寶釵聽了道:“噯,你說話怎麼越發不留神了,什麼男人同你一樣都說出來了,還叫我們瞧去嗎!”寶玉聽了,知是失言,臉上一紅,連忙的還要解說。不知何話,下回分解。
\end{parag}