\chap{二}{贾夫人仙逝扬州城 冷子兴演说荣国府}

\begin{parag}
    \begin{note}甲戌、庚辰、己卯:此回亦非正文,本旨只在冷子兴一人,即俗谓“冷中出热,无中生有”也。其演说荣府一篇者,盖因族大人多,若从作者笔下一一叙出,尽一二回不能得明\begin{subnote}按:己卯作“明白”\end{subnote},则成何文字?故借用冷字一人,略出其大半\begin{subnote}按:庚辰“略出其文半”;己卯作“略出其文”\end{subnote},使阅者心中,已有一荣府隐隐在心,然后用黛玉、宝钗等两三次皴染,则耀然于心中、眼中矣。此即画家三染法也。\end{note}
\end{parag}


\begin{parag}
    \begin{note}甲戌、庚辰、己卯:未写荣府正人,先写外戚,是由远及近\begin{subnote}按:己卯作“由近及远”\end{subnote},由小至大也。若使先叙出荣府,然后一一叙及外戚,又一一至朋友、至奴仆,其死板、拮据之笔,岂作十二钗人,手中之物也?今先写外戚者,正是写荣国一府也。故又怕闲文赘瘰,开笔即写贾夫人已死,是特使黛玉入荣\begin{subnote}按:庚辰、己卯皆作“荣府”\end{subnote}之速也。\end{note}
\end{parag}


\begin{parag}
    \begin{note}甲戌、庚辰、己卯:通灵宝玉于士隐梦中一出,今又于子兴口中一出,阅者已洞然矣。然后于黛玉、宝钗二人目中极精、极细一描,则是文章锁合处。盖不肯一笔直下,有若放闸之水、然信之爆\begin{subnote}按:己卯作“燃信之爆竹”\end{subnote},使其精华一泄而无余也。究竟此玉原应出自钗黛目中,方有照应。今预从子兴口中说出,实虽写,而却未写。观其后文,可知此一回则是虚敲傍击之文,笔则是,反逆隐回\begin{subnote}按:庚辰、己卯皆作“反逆隐曲”\end{subnote}之笔。\end{note}
\end{parag}


\begin{parag}
    \begin{note}蒙:以百回之大文,先以此回作两大笔以帽之,诚是大观。世态人情,尽盘旋于其间,而一丝不乱,非聚龙象力者,其孰能哉?\end{note}
\end{parag}


\begin{parag}
    诗云:\begin{note}甲戌行夹批:只此一诗便妙极!此等才情,自是雪芹平生所长,余自谓评书非关评诗也。\end{note}
\end{parag}


\begin{poem}
    \begin{pl}
        一局输嬴料不真,香销茶尽尚逡巡。\end{pl}

    \begin{pl}
        欲知目下兴衰兆,须问傍观冷眼人。\end{pl}\begin{note}甲戌眉批:故用冷子兴演说。\end{note}
\end{poem}


\begin{parag}
    却说封肃因听见公差传唤,忙出来陪笑启问。那些人只嚷:“快请出甄爷来!”\begin{note}甲戌侧批:一丝不乱。\end{note}封肃忙陪笑道:“小人姓封,幷不姓甄。只有当日小婿姓甄,今已出家一二年了,不知可是问他?”那些公人道:“我们也不知什么真假,\begin{note}甲戌侧批:点睛妙笔。\end{note}因奉太爷之命来问。他既是你女婿,便带了你去亲见太爷面禀,省得乱跑。”说著,不容封肃多言,大家推拥他去了。封家人个个都惊慌,不知何兆。
\end{parag}


\begin{parag}
    那天约二更时,只见封肃方回来,欢天喜地。\begin{note}甲戌侧批:出自封肃口内,便省却多少闲文。\end{note}众人忙问端的。他乃说道:“原来本府新升的太爷姓贾名化,本贯胡州人氏,曾与女婿旧日相交。方才在咱门前过去,因见娇杏\begin{note}甲戌侧批:侥幸也。托言当日丫头回顾,故有今日,亦不过偶然侥幸耳,非真实得尘中英杰也。非近日小说中满纸红拂紫烟之可比。甲戌眉批:余批重出。余阅此书,偶有所得,即笔录之。非从首至尾阅过复从首加批者,故偶有复处。且诸公之批,自是诸公眼界;脂斋之批,亦有脂斋取乐处。后每一阅,亦必有一语半言,重加批评于侧,故又有于前后照应之说等批。\end{note}那丫头买线,所以他只当女婿移住于此。我一一将原故回明,那太爷倒伤感叹息了一回,又问外孙女儿,\begin{note}甲戌侧批:细。\end{note}我说看灯丢了。太爷说:‘不妨,我自使番役,务必探访回来。’\begin{note}甲戌侧批:为葫芦案伏线。\end{note}说了一回话,临走倒送了我二两银子。”甄家娘子听了,不免心中伤感。\begin{note}甲戌侧批:所谓“旧事凄凉不可闻”也。\end{note}一宿无话。
\end{parag}


\begin{parag}
    至次日,早有雨村遣人送了两封银子、四匹锦缎,答谢甄家娘子,\begin{note}甲戌侧批:雨村已是下流人物,看此,今之如雨村者亦未有矣。\end{note}又寄一封密书与封肃,转托问甄家娘子要那娇杏作二房。\begin{note}甲戌侧批:谢礼却为此。险哉,人之心也!\end{note}封肃喜的屁滚尿流,巴不得去奉承,便在女儿前一力撺掇成了,\begin{note}甲戌侧批:一语道尽。\end{note}乘夜只用一乘小轿,便把娇杏送进去了。雨村欢喜,自不必说,乃封百金赠封肃,外谢甄家娘子许多物事,令其好生养赡,以待寻访女儿下落。\begin{note}甲戌侧批:找前伏后。士隐家一段小枯荣至此结住,所谓真不去假焉来也!\end{note}封肃回家无话。
\end{parag}


\begin{parag}
    却说娇杏这丫鬟,便是那年回顾雨村者。因偶然一顾,便弄出这段事来,亦是自己意料不到之奇缘。\begin{note}甲戌侧批:注明一笔,更妥当。\end{note}谁想他命运两济,\begin{note}甲戌眉批:好极!与英莲“有命无运”四字,遥遥相映射。莲,主也;杏,仆也。今莲反无运,而杏则两全,可知世人原在运数,不在眼下之高低也。此则大有深意存焉。\end{note}不承望自到雨村身边,只一年便生了一子,又半载,雨村嫡妻忽染疾下世,雨村便将他扶册作正室夫人了。正是:
\end{parag}


\begin{poem}
    \begin{pl}
        偶因一著错,\end{pl}\begin{note}甲戌侧批:妙极!盖女儿原不应私顾外人之谓。\end{note}

    \begin{pl}
        便为人上人。\end{pl}\begin{note}甲戌侧批:更妙!可知守礼俟命,终为俄莩。其调侃寓意不小。甲眉:从来只见集古集唐等句,未见集俗语者。此又更奇之至!\end{note}

\end{poem}


\begin{parag}
    原来,雨村因那年士隐赠银之后,他于十六日便起身入都。至大比之期,不料他十分得意,已会了进士,选入外班,今已升了本府知府。虽才干优长,未免有些贪酷之弊,且又恃才侮上,那些官员皆侧目而视。\begin{note}甲戌侧批:此亦奸雄必有之理。\end{note}不上一年,便被上司寻了个空隙,作成一本,参他“生情狡猾,擅纂礼仪,且沽清正之名,而暗结虎狼之属,致使地方多事,民命不堪”\begin{note}甲戌侧批:此亦奸雄必有之事。\end{note}等语。龙颜大怒,即批革职。该部文书一到,本府官员无不喜悦。那雨村心中虽十分惭恨,却面上全无一点怨色,仍是嘻笑自若。\begin{note}甲戌侧批:此亦奸雄必有之态。\end{note}交代过公事,将历年做官积的些资本幷家小人属送至原籍,安排妥协,\begin{note}甲戌侧批:先云“根基已尽”,故今用此四字,细甚!\end{note}却是自已担风袖月,游览天下胜迹。\begin{note}甲戌侧批:已伏下至金陵一节矣。\end{note}
\end{parag}


\begin{parag}
    那日,偶又游至维扬地面,因闻得今岁盐政点的是林如海。这林如海姓林名海,表字如海。\begin{note}甲戌侧批:盖云“学海文林”也。总是暗写黛玉。\end{note}乃是前科的探花,今已升至兰台寺大夫,\begin{note}甲戌眉批:官制半遵古名亦好。余最喜此等半有半无,半古半今,事之所无,理之必有,极玄极幻,荒唐不经之处。\end{note}本贯姑苏\begin{note}甲戌侧批:十二钗正出之地,故用真。\end{note}人氏,今钦点出为巡盐御史,到任方一月有余。
\end{parag}


\begin{parag}
    原来这林如海之祖,曾袭过列侯,今到如海,业经五世。起初时,只封袭三世,因当今隆恩盛德,远迈前代,\begin{note}甲戌眉批:可笑近时小说中,无故极力称扬浪子淫女,临收结时,还必致感动朝廷,使君父同入其情欲之界,明遂其意,何无人心之至!不知彼作者有何好处,有何谢!报到朝廷高庙之上,直将半生淫朽秽资睿德,又苦拉君父作一干证护身符,强媒硬保,得遂其淫欲哉!\end{note}额外加恩,至如海之父,又袭了一代;至如海,便从科第出身。虽系钟鼎之家,却亦是书香\begin{note}甲戌侧批:要紧二字,盖钟鼎亦必有书香方至美。\end{note}之族。只可惜这林家支庶不盛,子孙有限,虽有几门,却与如海俱是堂族而已,没甚亲支嫡派的。\begin{note}甲戌侧批:总为黛玉极力一写。\end{note}今如海年已四十,只有一个三岁之子,偏又于去岁死了。虽有几房姬妾,\begin{note}甲戌侧批:带写贤妻。\end{note}奈他命中无子,亦无可如何之事。今只有嫡妻贾氏,生得一女,乳名黛玉,年方五岁。夫妻无子,故爱如珍宝,且又见他聪明清秀,\begin{note}甲戌侧批:看他写黛玉,只用此四字。可笑近来小说中,满纸“天下无二”“古今无双”等字。\end{note}便也欲使他读书识得几个字,不过假充养子之意,聊解膝下荒凉之叹。\begin{note}甲戌眉批:如此叙法,方是至情至理之妙文。最可笑者,近小说中满纸班昭蔡琰、文君道韫。\end{note}
\end{parag}


\begin{parag}
    雨村正值偶感风寒,病在旅店,将一月光景方渐愈。一因身体劳倦,二因盘费不继,也正欲寻个合式之处,暂且歇下。幸有两个旧友,亦在此境居住,\begin{note}甲戌侧批:写雨村自得意后之交识也。又为冷子兴作引。\end{note}因闻得盐政欲聘一西宾,雨村便相托友力,谋了进去,且作安身之计。妙在只一个女学生,幷两个伴读丫鬟,这女学生年又小,身体又极怯弱,工课不限多寡,故十分省力。
\end{parag}


\begin{parag}
    堪堪又是一载的光阴,谁知女学生之母贾氏夫人一疾而终。女学生侍汤奉药,守丧尽哀,遂又将辞馆别图。林如海意欲令女学生守制读书,故又将他留下。近因女学生哀痛过伤,本自怯弱多病,\begin{note}甲戌侧批:又一染。\end{note}触犯旧症,遂连日不曾上学。\begin{note}甲戌眉批:上半回已终,写“仙逝”正为黛玉也。故一句带过,恐闲文有妨正笔。\end{note}雨村闲居无聊,每当风日晴和,饭后便出来闲步。
\end{parag}


\begin{parag}
    这日,偶至郭外,意欲赏鉴那村野风光。\begin{note}甲戌眉批:大都世人意料此,终不能此;不及彼者,而反及彼。故特书意在村野风光,却忽遇见子兴一篇荣国繁华气象。\end{note}忽信步至一山环水旋、茂林深竹之处,隐隐的有座庙宇,门巷倾颓,墙垣朽败,门前有额,题著“智通寺”三字,\begin{note}甲戌侧批:谁为智者?又谁能通?一叹。\end{note}门旁又有一副旧破的对联,曰:
\end{parag}


\begin{poem}
    \begin{pl}
        身后有余忘缩手,眼前无路想回头。\end{pl}\begin{note}甲夹批:先为宁、荣诸人当头一喝,却是为余一喝。\end{note}
\end{poem}


\begin{parag}
    雨村看了,因想到:这两句话,文虽浅近,其意则深。\begin{note}甲戌侧批:一部书之总批。\end{note}我也曾游过些名山大刹,倒不曾见过这话头,其中想必有个翻过筋斗来的亦未可知,\begin{note}甲戌侧批:随笔带出禅机,又为后文多少语录不落空。\end{note}何不进去试试?想著走入,只有一个龙钟老僧在那里煮粥。\begin{note}甲戌侧批:是雨村火气。\end{note}雨村见了,便不在意。\begin{note}甲戌侧批:火气。\end{note}及至问他两句话,那老僧既聋且昏,\begin{note}甲戌侧批:是翻过来的。\end{note}齿落舌钝,\begin{note}甲戌侧批:是翻过来的。\end{note}所答非所问。
\end{parag}


\begin{parag}
    雨村不耐烦,便仍出来,\begin{note}甲戌眉批:毕竟雨村还是俗眼,只能识得阿凤、宝玉、黛玉等未觉之先,却不识得既证之后。甲戌眉批:未出宁、荣繁华盛处,却先写一荒凉小景;未写通部入世迷人,却先写一出世醒人。回风舞雪,倒峡逆波,别小说中所无之法。\end{note}意欲到那村肆中沽饮三杯,以助野趣,于是款步行来,将入肆门,只见座上吃酒之客有一人起身大笑,接了出来,口内说:“奇遇,奇遇!”雨村忙看时,此人是都中在古董行中贸易的号冷子兴者,\begin{note}甲戌侧批:此人不过借为引绳,不必细写。\end{note}旧日在都相识。雨村最赞这冷子兴是个有作为大本领的人,这子兴又借雨村斯文之名,故二人说话投机,最相契合。雨村忙笑问道:“老兄何日到此?弟竟不知。今日偶遇,真奇缘也。”子兴道:“去年岁底到家,今因还要入都,从此顺路找个敝友说一句话,承他之情,留我多住两日。我也无紧事,且盘桓两日,待月半时也就起身了。今日敝友有事,我因闲步至此,且歇歇脚。不期这样巧遇!”一面说,一面让雨村同席坐了,另整上酒肴来。二人闲谈漫饮,叙些别后之事。\begin{note}甲戌侧批:好!若多谈则累赘。\end{note}
\end{parag}


\begin{parag}
    雨村因问:“近日都中可有新闻没有?”\begin{note}甲戌侧批:不突然,亦常问常答之言。\end{note}子兴道:“倒没有什么新闻,倒是老先生你贵同宗家,\begin{note}甲戌侧批:雨村已无族中矣,何及此耶?看他下文。\end{note}出了一件小小的异事。”雨村笑道:“弟族中无人在都,何谈及此?”子兴笑道:“你们同姓,岂非同宗一族?”雨村问是谁家。
\end{parag}


\begin{parag}
    子兴道:“荣国府贾府中,可也不玷辱了先生的门楣了?”\begin{note}甲戌侧批:刳小人之心肺,闻小人之口角。\end{note}雨村笑道:“原来是他家。若论起来,寒族人丁却不少,自东汉贾复以来,支派繁盛,各省皆有,\begin{note}甲戌侧批:此话纵真,亦必谓是雨村欺人语。\end{note}谁逐细考查得来?若论荣国一支,却是同谱。但他那等荣耀,我们不便去攀扯,至今故越发生疏难认了。”子兴叹\begin{note}甲戌侧批:叹得怪。\end{note}道:“老先生休如此说。如今的这宁、荣两门,也都萧疏了,不比先时的光景。”\begin{note}甲戌侧批:记清此句。可知书中之荣府已是末世了。\end{note}雨村道:“当日宁荣两宅的人口也极多,如何就萧疏了?”\begin{note}甲戌侧批:作者之意原只写末世,此已是贾府之末世了。\end{note}冷子兴道:“正是,说来也话长。”雨村道:“去岁我到金陵地界,因欲游览六朝遗迹,那日进了石头城,\begin{note}甲戌侧批:点睛神妙。\end{note}从他老宅门前经过。街东是宁国府,街西是荣国府,二宅相连,竟将大半条街占了。大门前虽冷落无人,\begin{note}甲戌侧批:好!写出空宅。\end{note}隔著围墙一望,里面厅殿楼阁,也还都峥嵘轩峻,就是后\begin{note}甲戌侧批:“后”字何不直用“西”字?甲戌侧批:恐先生堕泪,故不敢用“西”字。\end{note}一带花园子里面树木山石,也还都有蓊蔚洇润之气,那里像个衰败之家?”
\end{parag}


\begin{parag}
    冷子兴笑道:“亏你是进士出身,原来不通!古人有云:‘百足之虫,死而不僵。’如今虽说不及先年那样兴盛,较之平常仕宦之家,到底气象不同。如今生齿日繁,事务日盛,主仆上下,安富尊荣者尽多,运筹谋画者无一,\begin{note}甲戌侧批:二语乃今古富贵世家之大病。\end{note}其日用排场费用,又不能将就省俭,如今外面的架子虽未甚倒,\begin{note}甲戌侧批:“甚”字好!盖已半倒矣。\end{note}内囊却也尽上来了。这还是小事,更有一件大事。谁知这样钟鸣鼎食之家,翰墨诗书之族,\begin{note}甲戌侧批:两句写出荣府。\end{note}如今的儿孙,竟一代不如一代了!”\begin{note}甲戌眉批:文是极好之文,理是必有之理,话则极痛极悲之话。\end{note}雨村听说,也纳罕道:“这样诗礼之家,岂有不善教育之理?别门不知,只说这宁、荣二宅,是最教子有方的。”\begin{note}甲戌侧批:一转有力。\end{note}
\end{parag}


\begin{parag}
    子兴叹道:“正说的是这两门呢。待我告诉你。当日宁国公\begin{note}甲戌侧批:演。\end{note}与荣国公\begin{note}甲戌侧批:源。\end{note}是一母同胞弟兄两个。宁公居长,生了四个儿子。\begin{note}甲戌侧批:贾蔷、贾菌之祖,不言可知矣。\end{note}宁公死后,贾代化袭了官,\begin{note}甲戌侧批:第二代。\end{note}也养了两个儿子。长名贾敷,至八九岁上便死了,只剩了次子贾敬袭了官,\begin{note}甲戌侧批:第三代。\end{note}如今一味好道,只爱烧丹炼汞,\begin{note}甲戌侧批:亦是大族末世常有之事。叹叹!\end{note}余者一概不在心上。幸而早年留下一子,名唤贾珍,\begin{note}甲戌侧批:第四代。\end{note}因他父亲一心想作神仙,把官倒让他袭了。他父亲又不肯回原籍来,只在都中城外和道士们胡羼。这位珍爷倒生了一个儿子,今年才十六岁,名叫贾蓉。\begin{note}甲戌侧批:至蓉五代。\end{note}如今敬老爹一概不管。这珍爷那里肯读书,只一味高乐不了,把宁国府竟翻了过来,也没有人敢来管他。\begin{note}甲戌侧批:伏后文。\end{note}再说荣府你听,方才所说异事,就出在这里。自荣公死后,长子贾代善袭了官,\begin{note}甲戌侧批:第二代。\end{note}娶的也是金陵世勋史侯家的小姐\begin{note}甲戌侧批:因湘云,故及之。\end{note}为妻,生了两个儿子:长子贾赦,次子贾政。\begin{note}甲戌侧批:第三代。\end{note}如今代善早已去世,太夫人\begin{note}甲戌侧批:记真,湘云祖姑史氏太君也。\end{note}尚在。长子贾赦袭著官。\begin{note}[伏下贾琏凤姐当家之文。]\end{note}次子贾政,自幼酷喜读书,祖父最疼。原欲以科甲出身的,不料代善临终时遗本一上,皇上因恤先臣,即时令长子袭官外,问还有几子,立刻引见,遂额外赐了这政老爹一个主事之衔,\begin{note}甲戌侧批:嫡真实事,非妄拟也。\end{note}令其入部习学,如今现已升了员外郎了。\begin{note}甲戌侧批:总是称功颂德。\end{note}这政老爹的夫人王氏,\begin{note}甲戌侧批:记清。\end{note}头胎生的公子,名唤贾珠,十四岁进学,不到二十岁就娶了妻生了子,\begin{note}甲戌侧批:此即贾兰也。至兰第五代。\end{note}一病死了。\begin{note}甲戌侧批:略可望者即死,叹叹!\end{note}第二胎生了一位小姐,生在大年初一,这就奇了,不想后来又生一位公子,\begin{note}甲戌眉批:一部书中第一人却如此淡淡带出,故不见后来玉兄文字繁难。\end{note}说来更奇,一落胎胞,嘴里便衔下一块五彩晶莹的玉来,上面还有许多字迹,\begin{note}甲戌侧批:青埂顽石已得下落。\end{note}就取名叫作宝玉。你道是新奇异事不是?”\begin{note}正是宁、荣二处支谱。\end{note}
\end{parag}


\begin{parag}
    雨村笑道:“果然奇异。只怕这人来历不小。”
\end{parag}


\begin{parag}
    子兴冷笑道:“万人皆如此说,因而乃祖母便先爱如珍宝。那年周岁时,政老爹便要试他将来的志向,便将那世上所有之物摆了无数,与他抓取。谁知他一概不取,伸手只把些脂粉钗环抓来。政老爹便大怒了,说:‘将来酒色之徒耳!’因此便大不喜悦。独那史老太君还是命根一样。说来又奇,如今长了七八岁,虽然淘气异常,但其聪明乖觉处,百个不及他一个。说起孩子话来也奇怪,他说:‘女儿是水作的骨肉,男人是泥作的骨肉。\begin{note}甲戌侧批:真千古奇文奇情。\end{note}我见了女儿,我便清爽;见了男子,便觉浊臭逼人。’你道好笑不好笑?将来色鬼无移了!”\begin{note}甲戌侧批:没有这一句,雨村如何罕然厉色,幷后奇奇怪怪之论?\end{note}雨村罕然厉色忙止道:“非也!可惜你们不知道这人来历。大约政老前辈也错以淫魔色鬼看待了。若非多读书识事,加以致知格物之功,悟道参玄之力,不能知也。”
\end{parag}


\begin{parag}
    子兴见他说得这样重大,忙请教其端。雨村道:“天地生人,除大仁大恶两种,余者皆无大异。若大仁者,则应运而生,大恶者,则应劫而生。运生世治,劫生世危。尧,舜,禹,汤,文,武,周,召,孔,孟,董,韩,周,程,张,朱,皆应运而生者。蚩尤,共工,桀,纣,始皇,王莽,曹操,桓温,安禄山,秦桧等,皆应劫而生者。\begin{note}甲戌侧批:此亦略举大概几人而言。\end{note}大仁者,修治天下;大恶者,挠乱天下。清明灵秀,天地之正气,仁者之所秉也;残忍乖僻,天地之邪气,恶者之所秉也。今当运隆祚永之朝,太平无为之世,清明灵秀之气所秉者,上至朝廷,下及草野,比比皆是。所余之秀气,漫无所归,遂为甘露,为和风,洽然溉及四海。彼残忍乖僻之邪气,不能荡溢于光天化日之中,遂凝结充塞于深沟大壑之内,偶因风荡,或被云催,略有摇动感发之意,一丝半缕误而泄出者,偶值灵秀之气适过,正不容邪,邪复妒正,\begin{note}甲戌侧批:譬得好。\end{note}两不相下,亦如风水雷电,地中既遇,既不能消,又不能让,必至搏击掀发后始尽。故其气亦必赋人,发泄一尽始散。使男女偶秉此气而生者,在上则不能成仁人君子,下亦不能为大凶大恶。\begin{note}甲戌侧批:恰极,是确论。\end{note}置之于万万人中,其聪俊灵秀之气,则在万万人之上,其乖僻邪谬不近人情之态,又在万万人之下。若生于公侯富贵之家,则为情痴情种,若生于诗书清贫之族,则为逸士高人,纵再偶生于薄祚寒门,断不能为走卒健仆,甘遭庸人驱制驾驭,必为奇优名倡。如前代之许由、陶潜、阮籍、嵇康、刘伶、王谢二族、顾虎头、陈后主、唐明皇、宋徽宗、刘庭芝、温飞卿、米南宫、石曼卿、柳耆卿、秦少游,近日之倪云林、唐伯虎、祝枝山,再如李龟年、黄幡绰、敬新磨、卓文君、红拂、薛涛、崔莺、朝云之流。此皆易地则同之人也。”
\end{parag}


\begin{parag}
    子兴道:“依你说,成则王侯败则贼了?\begin{note}甲戌侧批:《女仙外史》中论魔道已奇,此又非《外史》之立意,故觉愈奇。\end{note}”雨村道:“正是这意。你还不知,我自革职以来,这两年遍游各省,也曾遇见两个异样孩子。\begin{note}甲戌侧批:先虚陪一个。\end{note}所以,方才你一说这宝玉,我就猜著了八九亦是这一派人物。不用远说,只金陵城内,钦差金陵省体仁院总裁\begin{note}甲戌侧批:此衔无考,亦因寓怀而设,置而勿论。\end{note}甄家,\begin{note}甲戌眉批:又一真正之家,特与假家遥对,故写假则知真。\end{note}你可知么?”子兴道:“谁人不知!这甄府和贾府就是老亲,又系世交。两家来往,极其亲热的。便在下也和他家来往非止一日了。”\begin{note}甲戌侧批:说大话之走狗,毕真。\end{note}雨村笑道:“去岁我在金陵,也曾有人荐我到甄府处馆。我进去看其光景,谁知他家那等显贵,却是个富而好礼之家,\begin{note}甲戌侧批:如闻其声。甲戌眉批:只一句便是一篇世家传,与子兴口中是两样。\end{note}倒是个难得之馆。但这一个学生,虽是启蒙,却比一个举业的还劳神。说起来更可笑,他说:‘必得两个女儿伴著我读书,我方能认得字,心里也明白,不然我自己心里糊涂。’\begin{note}甲戌侧批:甄家之宝玉乃上半部不写者,故此处极力表明,以遥照贾家之宝玉,凡写贾家之宝玉,则正为真宝玉传影。蒙侧批:灵玉却只一块,而宝玉有两个,情性如一,亦如六耳、悟空之意耶?\end{note}又常对跟他的小厮们说:‘这女儿两个字,极尊贵,极清净的,比那阿弥陀佛,元始天尊的这两个宝号还更尊荣无对的呢!\begin{note}甲戌眉批:如何只以释、老二号为譬,略不敢及我先师儒圣等人?余则不敢以顽劣目之。\end{note}你们这浊口臭舌,万不可唐突了这两个字,要紧。但凡要说时,必须先用清水香茶\begin{note}甲戌侧批:恭敬。\end{note}漱了口才可,设若失错,\begin{note}甲戌侧批:罪过。\end{note}便要凿牙穿腮等事。’其暴虐浮躁,顽劣憨痴,种种异常。只一放了学,进去见了那些女儿们,其温厚和平,聪敏文雅,\begin{note}甲戌侧批:与前八个字嫡对。\end{note}竟又变了一个。因此,他令尊也曾下死笞楚过几次,无奈竟不能改。每打的吃疼不过时,他便‘姐姐’‘妹妹’乱叫起来。\begin{note}甲戌眉批:以自古未闻之奇语,故写成自古未有之奇文。此是一部书中大调侃寓意处。盖作者实因鹡鸰之悲、棠棣之威,故撰此闺阁庭帏之传。\end{note}后来听得里面女儿们拿他取笑:‘因何打急了只管叫姐妹做甚?莫不是求姐妹去说情讨饶?你岂不愧些!’他回答的最妙。他说:‘急疼之时,只叫“姐姐”“妹妹”字样,或可解疼也未可知,因叫了一声,便果觉不疼了,遂得了秘法。每疼痛之极,便连叫姐妹起来了。’你说可笑不可笑?也因祖母溺爱不明,每因孙辱师责子,因此我就辞了馆出来。如今在这巡盐御史林家做馆了。你看,这等子弟,必不能守祖父之根基,从师长之规谏的。只可惜他家几个姊妹都是少有的。”\begin{note}甲戌侧批:实点一笔,余谓作者必有。\end{note}
\end{parag}


\begin{parag}
    子兴道:“便是贾府中,现有的三个也不错。政老爹的长女,名元\begin{note}甲戌侧批:原也。\end{note}春,现因贤孝才德,选入宫作女史\begin{note}甲戌侧批:因汉以前例,妙!\end{note}去了。二小姐乃赦老爹之妾所出,名迎\begin{note}甲戌侧批:应也。\end{note}春,三小姐乃政老爹之庶出,名探\begin{note}甲戌侧批:叹也。\end{note}春,四小姐乃宁府珍爷之胞妹,名唤惜\begin{note}甲戌侧批:息也。\end{note}春。因史老夫人极爱孙女,都跟在祖母这边一处读书,听得个个不错。”\begin{note}复接前文未及,正词源三叠。\end{note}雨村道:“更妙在甄家的风俗,女儿之名,亦皆从男子之名命字,不似别家另外用这些春红香玉等艶字的,何得贾府亦乐此俗套?”
\end{parag}


\begin{parag}
    子兴道:“不然,只因现今大小姐是正月初一日所生,故名元春,余者方从了春字。上一辈的,却也是从兄弟而来的。现有对证:目今你贵东家林公之夫人,即荣府中赦、政二公之胞妹,在家时名唤贾敏。不信时,你回去细访可知。”雨村拍案笑道:“怪道这女学生读至凡书中有‘敏’字,皆念作‘密’字,每每如是;写字遇著‘敏’字,又减一二笔,我心中就有些疑惑。今听你说的,是为此无疑矣。怪道我这女学生言语举止另是一样,不与近日女子相同,度其母必不凡,方得其女,今知为荣府之孙,又不足罕矣。可伤上月竟亡故了。”子兴叹道:“老姊妹四个,这一个是极小的,又没了。长一辈的姊妹,一个也没了。只看这小一辈的,将来之东床如何呢。”
\end{parag}


\begin{parag}
    雨村道:“正是,方才说这政公,已有衔玉之儿,又有长子所遗一个弱孙。这赦老竟无一个不成?”子兴道:“政公既有玉儿之后,其妾又生了一个,\begin{note}甲戌侧批:带出贾环。\end{note}倒不知其好歹。只眼前现有二子一孙,却不知将来如何。若问那赦公,也有二子。长名贾琏,今已二十来往了。亲上作亲,娶的就是政老爹夫人王氏之内侄女,\begin{note}甲戌侧批:另出熙凤一人。\end{note}今已娶了二年。这位琏爷身上现捐的是个同知,也是不肯读书,于世路上好机变,言谈去的,所以如今只在乃叔政老爷家住著,帮著料理些家务。谁知自娶了他令夫人之后,倒上下无一人不称颂他夫人的,琏爷倒退了一射之地。说模样又极标致,言谈又爽利,心机又极深细,竟是个男人万不及一的。”\begin{note}甲戌侧批:未见其人,先已有照。甲戌眉批:非警幻案下而来为谁?\end{note}
\end{parag}


\begin{parag}
    雨村听了,笑道:“可知我前言不谬。\begin{note}甲戌侧批:略一总住。\end{note}你我方才所说的这几个人,都只怕是那正邪两赋而来一路之人,未可知也。”子兴道:“邪也罢,正也罢,只顾算别人家的帐,你也吃一杯酒才好。”雨村道:“正是,只顾说话,竟多吃了几杯。”子兴笑道:“说著别人家的闲话,正好下酒,\begin{note}甲戌侧批:盖云此一段话亦为世人茶酒之笑谈耳。\end{note}即多吃几杯何妨。”雨村向窗外看\begin{note}甲戌侧批:画。\end{note}道:“天也晚了,仔细关了城。我们慢慢的进城再谈,未为不可。”于是,二人起身,算还酒帐。\begin{note}甲戌侧批:不得谓此处收得索然,盖原非正文也。\end{note}
\end{parag}


\begin{parag}
    方欲走时,又听得后面有人叫道:“雨村兄,恭喜了!特来报个喜信的。”
\end{parag}


\begin{parag}
    \begin{note}甲戌侧批:此等套头,亦不得不用。\end{note}雨村忙回头看时——\begin{note}己卯夹批:语言太烦,令人不耐。古人云“惜墨如金”,看此视墨如土矣,虽演至千万回亦可也。\end{note}
\end{parag}


\begin{parag}
    \begin{note}蒙、戚:先自写幸遇之情于前,而叙借口谈幻境之情于后。世上不平事,道路口如碑。虽作者之苦心,亦人情之必有。雨村之遇娇杏,是此文之总帽,故在前。冷子兴之谈,是事迹之总帽,故叙写于后。冷暖世情,比比如画。\end{note}
\end{parag}


\begin{parag}
    \begin{note}蒙、戚:有情原比无情苦,生死相关总在心。也是前缘天作合,何妨黛玉泪淋淋。\end{note}
\end{parag}

