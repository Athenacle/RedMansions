\chap{九十二}{評女傳巧姐慕賢良 玩母珠賈政參聚散}



\begin{parag}
    話說寶玉從瀟湘館出來,連忙問秋紋道:“老爺叫我作什麼?”秋紋笑道:“沒有叫,襲人姐姐叫我請二爺,我怕你不來,才哄你的。”寶玉聽了才把心放下,因說:“你們請我也罷了,何苦來唬我。”說着,回到怡紅院內。襲人便問道:“你這好半天到那裏去了?”寶玉道:“在林姑娘那邊,說起薛姨媽寶姐姐的事來,便坐住了。”襲人又問道:“說些什麼?”寶玉將打禪語的話述了一遍。襲人道:“你們再沒個計較,正經說些家常閒話兒,或講究些詩句,也是好的,怎麼又說到禪語上了。又不是和尚。”寶玉道:“你不知道,我們有我們的禪機,別人是插不下嘴去的。”襲人笑道:“你們參禪參翻了,又叫我們跟着打悶葫蘆了。”寶玉道:“頭裏我也年紀小,他也孩子氣,所以我說了不留神的話,他就惱了。如今我也留神,他也沒有惱的了。只是他近來不常過來,我又唸書,偶然到一處,好象生疏了似的。”襲人道:“原該這麼着纔是。都長了幾歲年紀了,怎麼好意思還象小孩子時候的樣子。”寶玉點頭道:“我也知道。如今且不用說那個。我問你,老太太那裏打發人來說什麼來着沒有?”襲人道:“沒有說什麼。”寶玉道:“必是老太太忘了。明兒不是十一月初一日麼,年年老太太那裏必是個老規矩,要辦消寒會,齊打夥兒坐下喝酒說笑。我今日已經在學房裏告了假了,這會子沒有信兒,明兒可是去不去呢?若去了呢,白白的告了假,若不去,老爺知道了又說我偷懶。”襲人道:“據我說,你竟是去的是。才唸的好些兒了,又想歇着。依我說也該上緊些纔好。昨兒聽見太太說,蘭哥兒唸書真好,他打學房裏回來,還各自唸書作文章,天天晚上弄到四更多天才睡。你比他大多了,又是叔叔,倘或趕不上他,又叫老太太生氣。倒不如明兒早起去罷。”麝月道:“這樣冷天,已經告了假又去,倒叫學房裏說:既這麼着就不該告假呀,顯見的是告謊假脫滑兒。依我說落得歇一天。就是老太太忘記了,咱們這裏就不消寒了麼,咱們也鬧個會兒不好麼。”襲人道:“都是你起頭兒,二爺更不肯去了。”麝月道:“我也是樂一天是一天,比不得你要好名兒,使喚一個月再多得二兩銀子!”襲人啐道:“小蹄子,人家說正經話,你又來胡拉混扯的了。”麝月道:“我倒不是混拉扯,我是爲你。”襲人道:“爲我什麼?”麝月道:“二爺上學去了,你又該咕嘟着嘴想着,巴不得二爺早一刻兒回來,就有說有笑的了。這會兒又假撇清,何苦呢!我都看見了。”
\end{parag}


\begin{parag}
    襲人正要罵他,只見老太太那裏打發人來說道:“老太太說了,叫二爺明兒不用上學去呢。明兒請了姨太太來給他解悶,只怕姑娘們都來,家裏的史姑娘,邢姑娘,李姑娘們都請了,明兒來赴什麼消寒會呢。”寶玉沒有聽完便喜歡道:“可不是,老太太最高興的,明日不上學是過了明路的了。”襲人也便不言語了。那丫頭回去。寶玉認真唸了幾天書,巴不得頑這一天。又聽見薛姨媽過來,想着”寶姐姐自然也來”。心裏喜歡,便說:“快睡罷,明日早些起來。”於是一夜無話。到了次日,果然一早到老太太那裏請了安,又到賈政王夫人那裏請了安,回明瞭老太太今兒不叫上學,賈政也沒言語,便慢慢退出來,走了幾步便一溜煙跑到賈母房中。見衆人都沒來,只有鳳姐那邊的奶媽子帶了巧姐兒,跟着幾個小丫頭過來,給老太太請了安,說:“我媽媽先叫我來請安,陪着老太太說說話兒。媽媽回來就來。”賈母笑道:“好孩子,我一早就起來了,等他們總不來,只有你二叔叔來了。”那奶媽子便說:“姑娘給你二叔叔請安。”寶玉也問了一聲“妞妞好?”巧姐兒道:“我昨夜聽見我媽媽說,要請二叔叔去說話。”寶玉道:“說什麼呢?”巧姐兒道:“我媽媽說,跟着李媽認了幾年字,不知道我認得不認得。我說都認得,我認給媽媽瞧。媽媽說我瞎認,不信,說我一天儘子頑,那裏認得。我瞧着那些字也不要緊,就是那《女孝經》也是容易唸的。媽媽說我哄他,要請二叔叔得空兒的時候給我理理。”賈母聽了,笑道:“好孩子,你媽媽是不認得字的,所以說你哄他。明兒叫你二叔叔理給他瞧瞧,他就信了。”寶玉道:“你認了多少字了?”巧姐兒道:“認了三千多字,唸了一本《女孝經》,半個月頭裏又上了《列女傳》。”寶玉道:“你念了懂得嗎?你要不懂,我倒是講講這個你聽罷。”賈母道:“做叔叔的也該講究給侄女聽聽。”寶玉道:“那文王后妃是不必說了,想來是知道的。那姜後脫簪待罪,齊國的無鹽雖醜,能安邦定國,是后妃裏頭的賢能的。若說有才的,是曹大姑、班婕妤、蔡文姬、謝道韞諸人。孟光的荊釵布裙,鮑宣妻的提甕出汲,陶侃母的截髮留賓,還有畫荻教子的,這是不厭貧的。那苦的裏頭,有樂昌公主破鏡重圓,蘇蕙的迴文感主。那孝的是更多了,木蘭代父從軍,曹娥投水尋父的屍首等類也多,我也說不得許多。那個曹氏的引刀割鼻,是魏國的故事。那守節的更多了,只好慢慢的講。若是那些豔的,王嬙、西子、樊素、小蠻、絳仙等。妒的是禿妾發,怨洛神等類,也少。文君、紅拂是女中的……”賈母聽到這裏,說:“夠了,不用說了。你講的太多,他那裏還記得呢。”巧姐兒道:“二叔叔才說的,也有念過的,也有沒念過的。念過的二叔叔一講,我更知道了好些。”寶玉道:“那字是自然認得的了,不用再理。明兒我還上學去呢。”巧姐兒道:“我還聽見我媽媽昨兒說,我們家的小紅頭裏是二叔叔那裏的,我媽媽要了來,還沒有補上人呢。我媽媽想着要把什麼柳家的五兒補上,不知二叔叔要不要。”寶玉聽了更喜歡,笑着道:“你聽你媽媽的話!要補誰就補誰罷咧,又問什麼要不要呢。”因又向賈母笑道:“我瞧大妞妞這個小模樣兒,又有這個聰明兒,只怕將來比鳳姐姐還強呢,又比他認的字。”賈母道:“女孩兒家認得字呢也好,只是女工針黹倒是要緊的。”巧姐兒道:“我也跟着劉媽媽學着做呢,什麼扎花兒咧,拉鎖子,我雖弄不好,卻也學着會做幾針兒。”賈母道:“咱們這樣人家固然不仗着自己做,但只到底知道些,日後纔不受人家的拿捏。”巧姐兒答應着“是”,還要寶玉解說《列女傳》,見寶玉呆呆的,也不敢再說。
\end{parag}


\begin{parag}
    你道寶玉呆的是什麼?只因柳五兒要進怡紅院,頭一次是他病了不能進來,第二次王夫人攆了晴雯,大凡有些姿色的,都不敢挑。後來又在吳貴家看晴雯去,五兒跟着他媽給晴雯送東西去,見了一面,更覺嬌娜嫵媚。今日虧得鳳姐想着,叫他補入小紅的窩兒,竟是喜出望外了。所以呆呆的想他。
\end{parag}


\begin{parag}
    賈母等着那些人,見這時候還不來,又叫丫頭去請。回來李紈同着他妹子,探春、惜春、史湘雲、黛玉都來了,大家請了賈母的安。衆人廝見。獨有薛姨媽未到,賈母又叫請去。果然姨媽帶着寶琴過來。寶玉請了安,問了好。只不見寶釵、邢岫煙二人。黛玉便問起“寶姐姐爲何不來?”薛姨媽假說身上不好。邢岫煙知道薛姨媽在坐,所以不來。寶玉雖見寶釵不來,心中納悶,因黛玉來了,便把想寶釵的心暫且擱開。不多時,邢王二夫人也來了。鳳姐聽見婆婆們先到了,自己不好落後,只得打發平兒先來告假,說是正要過來,因身上發熱,過一回兒就來。賈母道:“既是身上不好,不來也罷。咱們這時候很該喫飯了。”丫頭們把火盆往後挪了一挪兒,就在賈母榻前一溜擺下兩桌,大家序次坐下。吃了飯,依舊圍爐閒談,不須多贅。
\end{parag}


\begin{parag}
    且說鳳姐因何不來?頭裏爲着倒比邢王二夫人遲了,不好意思,後來旺兒家的來回說:“迎姑娘那裏打發人來請奶奶安,還說並沒有到上頭,只到奶奶這裏來。”鳳姐聽了納悶,不知又是什麼事,便叫那人進來,問:“姑娘在家好?”那人道:“有什麼好的,奴才並不是姑娘打發來的,實在是司棋的母親央我來求奶奶的。”鳳姐道:“司棋已經出去了,爲什麼來求我?”那人道:“自從司棋出去,終日啼哭。忽然那一日他表兄來了,他母親見了,恨得什麼似的,說他害了司棋,一把拉住要打。那小子不敢言語。誰知司棋聽見了,急忙出來老着臉和他母親道:‘我是爲他出來的,我也恨他沒良心。如今他來了,媽要打他,不如勒死了我。’他母親罵他:‘不害臊的東西,你心裏要怎麼樣?’司棋說道:‘一個女人配一個男人。我一時失腳上了他的當,我就是他的人了,決不肯再失身給別人的。我恨他爲什麼這樣膽小,一身作事一身當,爲什麼要逃。就是他一輩子不來了,我也一輩子不嫁人的。媽要給我配人,我原拼着一死的。今兒他來了,媽問他怎麼樣。若是他不改心,我在媽跟前磕了頭,只當是我死了,他到那裏,我跟到那裏,就是討飯喫也是願意的。’他媽氣得了不得,便哭着罵着說:‘你是我的女兒,我偏不給他,你敢怎麼着。’那知道那司棋這東西糊塗,便一頭撞在牆上,把腦袋撞破,鮮血直流,竟死了。他媽哭着救不過來,便要叫那小子償命。他表兄說道:‘你們不用着急。我在外頭原發了財,因想着他纔回來的,心也算是真了。你們若不信,只管瞧。’說着,打懷裏掏出一匣子金珠首飾來。他媽媽看見了便心軟了,說:‘你既有心,爲什麼總不言語?’他外甥道:‘大凡女人都是水性楊花,我若說有錢,他便是貪圖銀錢了。如今他只爲人,就是難得的。我把金珠給你們,我去買棺盛殮他。’那司棋的母親接了東西,也不顧女孩兒了,便由着外甥去。那裏知道他外甥叫人抬了兩口棺材來。司棋的母親看見詫異,說:‘怎麼棺材要兩口?’他外甥笑道:‘一口裝不下,得兩口才好。’司棋的母親見他外甥又不哭,只當是他心疼的傻了。豈知他忙着把司棋收拾了,也不啼哭,眼錯不見,把帶的小刀子往脖子裏一抹,也就抹死了。司棋的母親懊悔起來,倒哭得了不得。如今坊上知道了,要報官。他急了,央我來求奶奶說個人情,他再過來給奶奶磕頭。”鳳姐聽了,詫異道:“那有這樣傻丫頭,偏偏的就碰見這個傻小子!怪不得那一天翻出那些東西來,她心裏沒事人似的,敢只是這麼個烈性孩子。論起來,我也沒這麼大工夫管他這些閒事,但只你纔說的叫人聽着怪可憐見兒的。也罷了,你回去告訴他,我和你二爺說,打發旺兒給他撕擄就是了。”鳳姐打發那人去了,才過賈母這邊來。不提。
\end{parag}


\begin{parag}
    且說賈政這日正與詹光下大棋,通局的輸贏也差不多,單爲着一隻角兒死活未分,在那裏打劫。門上的小廝進來回道:“外面馮大爺要見老爺。”賈政道:“請進來。”小廝出去請了,馮紫英走進門來。賈政即忙迎着。馮紫英進來,在書房中坐下,見是下棋,便道:“只管下棋,我來觀局。”詹光笑道:“晚生的棋是不堪瞧的。”馮紫英道:“好說,請下罷。”賈政道:“有什麼事麼?”馮紫英道:“沒有什麼話。老伯只管下棋,我也學幾着兒。”賈政向詹光道:“馮大爺是我們相好的,既沒事,我們索性下完了這一局再說話兒。馮大爺在旁邊瞧着。”馮紫英道:“下采不下采?”詹光道:“下采的。”馮紫英道:“下采的是不好多嘴的。”賈政道:“多嘴也不妨,橫豎他輸了十來兩銀子,終久是不拿出來的。往後只好罰他做東便了。”詹光笑道:“這倒使得。”馮紫英道:“老伯和詹公對下麼?”賈政笑道:“從前對下,他輸了,如今讓他兩個子兒,他又輸了。時常還要悔幾着,不叫他悔他就急了。”詹光也笑道:“沒有的事。”賈政道:“你試試瞧。”大家一面說笑,一面下完了。做起棋來,詹光還了棋頭,輸了七個子兒。馮紫英道:“這盤終喫虧在打劫裏頭。老伯劫少,就便宜了。”
\end{parag}


\begin{parag}
    賈政對馮紫英道:“有罪,有罪。咱們說話兒罷。”馮紫英道:“小侄與老伯久不見面,一來會會,二來因廣西的同知進來引見,帶了四種洋貨,可以做得貢的。一件是圍屏,有二十四扇槅子,都是紫檀雕刻的。中間雖說不是玉,卻是絕好的硝子石,石上鏤出山水人物樓臺花鳥等物。一扇上有五六十個人,都是宮妝的女子,名爲《漢宮春曉》。人的眉目口鼻以及出手衣褶,刻得又清楚又細膩。點綴佈置都是好的。我想尊府大觀園中正廳上卻可用得着。還有一個鐘錶,有三尺多高,也是一個小童兒拿着時辰牌,到了什麼時候他就報什麼時辰。裏頭也有些人在那裏打十番的。這是兩件重笨的,卻還沒有拿來。現在我帶在這裏兩件卻有些意思兒。”就在身邊拿出一個錦匣子,見幾重白錦裹着,揭開了錦子,第一層是一個玻璃盒子,裏頭金托子大紅縐綢託底,上放着一顆桂圓大的珠子,光華耀目。馮紫英道:“據說這就叫做母珠。”因叫拿一個盤兒來。詹光即忙端過一個黑漆茶盤,道:“使得麼?”馮紫英道:“使得。”便又向懷裏掏出一個白絹包兒,將包兒裏的珠子都倒在盤子裏散着,把那顆母珠擱在中間,將盤置於桌上。看見那些小珠子兒滴溜滴溜滾到大珠身邊來,一回兒把這顆大珠子抬高了,別處的小珠子一顆也不剩,都粘在大珠上。詹光道:“這也奇怪。”賈政道:“這是有的,所以叫做母珠,原是珠之母。”那馮紫英又回頭看着他跟來的小廝道:“那個匣子呢?”那小廝趕忙捧過一個花梨木匣子來。大家打開看時,原來匣內襯着虎紋錦,錦上迭着一束藍紗。詹光道:“這是什麼東西?”馮紫英道:“這叫做鮫綃帳。”在匣子裏拿出來時,迭得長不滿五寸,厚不上半寸,馮紫英一層一層的打開,打到十來層,已經桌上鋪不下了。馮紫英道:“你看裏頭還有兩折,必得高屋裏去才張得下。這就是鮫絲所織,暑熱天氣張在堂屋裏頭,蒼蠅蚊子一個不能進來,又輕又亮。”賈政道:“不用全打開,怕迭起來倒費事。”詹光便與馮紫英一層一層摺好收拾。馮紫英道:“這四件東西價兒也不很貴,兩萬銀他就賣。母珠一萬,鮫綃帳五千,《漢宮春曉》與自鳴鐘五千。”賈政道:“那裏買得起。”馮紫英道:“你們是個國戚,難道宮裏頭用不着麼?”賈政道:“用得着的很多,只是那裏有這些銀子。等我叫人拿進去給老太太瞧瞧。”馮紫英道:“很是。”
\end{parag}


\begin{parag}
    賈政便着人叫賈璉把這兩件東西送到老太太那邊去,並叫人請了邢王二夫人鳳姐兒都來瞧着,又把兩件東西一一試過。賈璉道:“他還有兩件:一件是圍屏。一件是樂鍾。共總要賣二萬銀子呢。”鳳姐兒接着道:“東西自然是好的,但是那裏有這些閒錢。咱們又不比外任督撫要辦貢。我已經想了好些年了,象咱們這種人家,必得置些不動搖的根基纔好,或是祭地,或是義莊,再置些墳屋。往後子孫遇見不得意的事,還是點兒底子,不到一敗塗地。我的意思是這樣,不知老太太、老爺、太太們怎麼樣。若是外頭老爺們要買,只管買。”賈母與衆人都說:“這話說的倒也是。”賈璉道:“還了他罷。原是老爺叫我送給老太太瞧,爲的是宮裏好進。誰說買來擱在家裏?老太太還沒開口,你便說了一大些喪氣話!”
\end{parag}


\begin{parag}
    說着,便把兩件東西拿了出去,告訴了賈政,說老太太不要。便與馮紫英道:“這兩件東西好可好,就只沒銀子。我替你留心,有要買的人,我便送信給你去。”馮紫英只得收拾好,坐下說些閒話,沒有興頭,就要起身。賈政道:“你在我這裏吃了晚飯去罷。”馮紫英道:“罷了,來了就叨擾老伯嗎!”賈政道:“說那裏的話。”正說着,人回:“大老爺來了。”賈赦早已進來。彼此相見,敘些寒溫。不一時擺上酒來,餚饌羅列,大家喝着酒。至四五巡後,說起洋貨的話,馮紫英道:“這種貨本是難消的,除非要象尊府這種人家,還可消得,其餘就難了。”賈政道:“這也不見得。”賈赦道:“我們家裏也比不得從前了,這回兒也不過是個空門面。”馮紫英又問:“東府珍大爺可好麼?我前兒見他,說起家常話兒來,提到他令郎續娶的媳婦,遠不及頭裏那位秦氏奶奶了。如今後娶的到底是那一家的,我也沒有問起。”賈政道:“我們這個侄孫媳婦兒,也是這裏大家,從前做過京畿道的胡老爺的女孩兒。”紫英道:“胡道長我是知道的。但是他家教上也不怎麼樣。也罷了,只要姑娘好就好。”
\end{parag}


\begin{parag}
    賈璉道:“聽得內閣里人說起,賈雨村又要升了。”賈政道:“這也好,不知準不準。”賈璉道:“大約有意思的了。”馮紫英道:“我今兒從吏部裏來,也聽見這樣說。雨村老先生是貴本家不是?”賈政道:“是。”馮紫英道:“是有服的還是無服的?”賈政道:“說也話長。他原籍是浙江湖州府人,流寓到蘇州,甚不得意。有個甄士隱和他相好,時常賙濟他。以後中了進士,得了榜下知縣,便娶了甄家的丫頭。如今的太太不是正配。豈知甄士隱弄到零落不堪,沒有找處。雨村革了職以後,那時還與我家並未相識,只因舍妹丈林如海林公在揚州巡鹽的時候,請他在家做西席,外甥女兒是他的學生。因他有起復的信要進京來,恰好外甥女兒要上來探親,林姑老爺便託他照應上來的,還有一封薦書,託我吹噓吹噓。那時看他不錯,大家常會。豈知雨村也奇,我家世襲起,從代字輩下來,寧榮兩宅人口房舍以及起居事宜,一概都明白,因此遂覺得親熱了。”因又笑說道:“幾年門子也會鑽了。由知府推升轉了御史,不過幾年,升了吏部侍郎,署兵部尚書。爲着一件事降了三級,如今又要升了。”馮紫英道:“人世的榮枯,仕途的得失,終屬難定。”賈政道:“象雨村算便宜的了。還有我們差不多的人家就是甄家,從前一樣功勳,一樣的世襲,一樣的起居,我們也是時常往來。不多幾年,他們進京來差人到我這裏請安,還很熱鬧。一回兒抄了原籍的家財,至今杳無音信,不知他近況若何,心下也着實惦記。看了這樣,你想做官的怕不怕?”賈赦道:“咱們家是最沒有事的。”馮紫英道:“果然,尊府是不怕的。一則裏頭有貴妃照應,二則故舊好親戚多,三則你家自老太太起至於少爺們,沒有一個刁鑽刻薄的。”賈政道:“雖無刁鑽刻薄,卻沒有德行才情。白白的衣租食稅,那裏當得起。”賈赦道:“咱們不用說這些話,大家喫酒罷。”大家又喝了幾杯,擺上飯來。喫畢,喝茶。馮家的小廝走來輕輕的向紫英說了一句,馮紫英便要告辭了。賈赦賈政道:“你說什麼?”小廝道:“外面下雪,早已下了梆子了。”賈政叫人看時,已是雪深一寸多了。賈政道:“那兩件東西你收拾好了麼?”馮紫英道:“收好了。若尊府要用,價錢還自然讓些。”賈政道:“我留神就是了。”紫英道:“我再聽信罷。天氣冷,請罷,別送了。”賈赦賈政便命賈璉送了出去。未知後事如何,下回分解。
\end{parag}