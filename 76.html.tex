\chap{七十六}{ 凸碧堂品笛感凄清 凹晶馆联诗悲寂寞}
\begin{parag}
    \begin{note}蒙回前总批:此回著笔最难,不叙中元夜宴则漏,叙夜宴则与上元相犯,不叙诸人酬和则俗,叙酬和又与起社相犯,诸人在贾政面前吟诗,诸人各自为一席,又非礼,既叙夜宴再叙酬和,不漏不俗更不相犯。云行月移,水流花放,别有机括,深宜玩恭。\end{note}
\end{parag}


\begin{parag}
    话说贾赦贾政带领贾珍等散去不提。且说贾母这里命将围屏撤去,两席并而为一。众媳妇另行擦桌整果,更杯洗箸,陈设一番。贾母等都添了衣,盥漱吃茶,方又入坐,团团围绕。贾母看时,宝钗姊妹二人不在坐内,知他们家去圆月去了,且李纨凤姐二人又病著,少了四个人,便觉冷清了好些。\begin{note}庚辰双行夹批:不想这次中秋反写得十分凄楚。\end{note}贾母因笑道:“往年你老爷们不在家,咱们越性请过姨太太来,大家赏月,却十分闹热。忽一时想起你老爷来,又不免想到母子夫妻儿女不能一处,也都没兴。及至今年你老爷来了,正该大家团圆取乐,又不便请他们娘儿们来说说笑笑。况且他们今年又添了两口人,也难丢了他们跑到这里来。偏又把凤丫头病了,有他一人来说说笑笑,还抵得十个人的空儿。可见天下事总难十全。”说毕,不觉长叹一声,遂命拿大杯来斟热酒。王夫人笑道:“今日得母子团圆,自比往年有趣。往年娘儿们虽多,终不似今年自己骨肉齐全的好。”贾母笑道:“正是为此,所以才高兴拿大杯来吃酒。你们也换大杯才是。”邢夫人等只得换上大杯来。因夜深体乏,且不能胜酒,未免都有些倦意,无奈贾母兴犹未阑,只得陪饮。
\end{parag}


\begin{parag}
    贾母又命将罽毡铺于阶上,命将月饼西瓜果品等类都叫搬下去,令丫头媳妇们也都团团围坐赏月。贾母因见月至中天,比先越发精彩可爱,因说:“如此好月,不可不闻笛。”因命人将十番上女孩子传来。贾母道:“音乐多了,反失雅致,只用吹笛的远远的吹起来就够了。”说毕,刚才去吹时,只见跟邢夫人的媳妇走来向邢夫人前说了两句话。贾母便问:“说什么事?”那媳妇便回说:“方才大老爷出去,被石头绊了一下,崴了腿。”贾母听说,忙命两个婆子快看去,又命邢夫人快去。邢夫人遂告辞起身。贾母便又说:“珍哥媳妇也趁著便就家去罢,我也就睡了。”尤氏笑道:“我今日不回去了,定要和老祖宗吃一夜。” 贾母笑道:“使不得,使不得。你们小夫妻家,今夜不要团圆团圆,如何为我耽搁了。”尤氏红了脸,笑道:“老祖宗说的我们太不堪了。我们虽然年轻,已经是十来年的夫妻,也奔四十岁的人了。况且孝服未满,陪著老太太顽一夜还罢了,岂有自去团圆的理。”贾母听说,笑道:“这话很是,我倒也忘了孝未满。可怜你公公已是二年多了,\begin{note}庚辰双行夹批:不是算贾敬,却是算赦死期也。\end{note}可是我倒忘了,该罚我一大杯。既这样,你就越性别送,陪著我罢了。你叫蓉儿媳妇送去,就顺便回去罢。”尤氏说了。蓉妻答应著,送出邢夫人,一同至大门,各自上车回去。不在话下。
\end{parag}


\begin{parag}
    这里贾母仍带众人赏了一回桂花,又入席换暖酒来。正说著闲话,猛不防只听那壁厢桂花树下,呜呜咽咽,悠悠扬扬,吹出笛声来。趁著这明月清风,天空地净,真令人烦心顿解,万虑齐除,都肃然危坐,默默相赏。听约两盏茶时,方才止住,大家称赞不已。于是遂又斟上暖酒来。贾母笑道:“果然可听么?”众人笑道:“实在可听。我们也想不到这样,须得老太太带领著,我们也得开些心胸。”贾母道:“这还不大好,须得拣那曲谱越慢的吹来越好。”说著,便将自己吃的一个内造瓜仁油松穰月饼,又命斟一大杯热酒,送给谱笛之人,慢慢的吃了再细细的吹一套来。媳妇们答应了,方送去,只见方才瞧贾赦的两个婆子回来了,说:“右脚面上白肿了些,如今调服了药,疼的好些了,也不甚大关系。”贾母点头叹道:“我也太操心。打紧说我偏心,我反这样。”因就将方才贾赦的笑话说与王夫人尤氏等听。王夫人等因笑劝道:“这原是酒后大家说笑,不留心也是有的,岂有敢说老太太之理。老太太自当解释才是。”只见鸳鸯拿了软巾兜与大斗篷来,说:“夜深了,恐露水下来,风吹了头,须要添了这个。坐坐也该歇了。”贾母道:“偏今儿高兴,你又来催。难道我醉了不成,偏到天亮!”因命再斟酒来。一面戴上兜巾,披了斗篷,大家陪著又饮,说些笑话。只听桂花阴里,呜呜咽咽,袅袅悠悠,又发出一缕笛音来,果真比先越发凄凉。大家都寂然而坐。夜静月明,且笛声悲怨,贾母年老带酒之人,听此声音,不免有触于心,禁不住堕下泪来。众人彼此都不禁有凄凉寂寞之意,半日,方知贾母伤感,才忙转身陪笑,发语解释。\begin{note}庚辰双行夹批:“转身”妙!画出对月听笛如痴如呆、不觉尊长在上之形景来。\end{note}又命暖酒,且住了笛。尤氏笑道:“我也就学一个笑话,说与老太太解解闷。”贾母勉强笑道:“这样更好,快说来我听。”尤氏说道:“一家子养了四个儿子:大儿子只一个眼睛,二儿子只一个耳朵,三儿子只一个鼻子眼,四儿子倒都齐全,偏又是个哑叭。”正说到这里,只见贾母已朦胧双眼,似有睡去之态。\begin{note}庚辰双行夹批:总写出凄凉无兴景况来。\end{note}尤氏方住了,忙和王夫人轻轻的请醒。贾母睁眼笑道: “我不困,白闭闭眼养神。你们只管说,我听著呢。”王夫人等笑道:“夜已四更了,风露也大,请老太太安歇罢。明日再赏十六,也不辜负这月色。”贾母道: “那里就四更了?”王夫人笑道:“实已四更,他们姊妹们熬不过,都去睡了。”贾母听说,细看了一看,果然都散了,只有探春在此。贾母笑道:“也罢。你们也熬不惯,况且弱的弱,病的病,去了倒省心。只是三丫头可怜见的,尚还等著。你也去罢,我们散了。”说著,便起身,吃了一口清茶,便有预备下的竹椅小轿,便围著斗篷坐上,两个婆子搭起,众人围随出园去了。不在话下。
\end{parag}


\begin{parag}
    这里众媳妇收拾杯盘碗盏时,却少了个细茶杯,各处寻觅不见,又问众人:“必是谁失手打了。撂在那里,告诉我拿了磁瓦去交收是证见,不然又说偷起来。” 众人都说:“没有打了,只怕跟姑娘的人打了,也未可知。你细想想,或问问他们去。”一语提醒了这管家伙的媳妇,因笑道:“是了,那一会儿记得是翠缕拿著的。我去问他。”说著便去找时,刚下了甬道,就遇见了紫鹃和翠缕来了。\begin{note}庚辰双行夹批:妙!又出一个。\end{note}翠缕便问道:“老太太散了,可知我们姑娘那去了?”\begin{note}庚辰双行夹批:更妙!\end{note}这媳妇道:“我来问那一个茶钟往那里去了,你们倒问我要姑娘。”翠缕笑道:“我因倒茶给姑娘吃的,展眼回头,就连姑娘也没了。”那媳妇道:“太太才说都睡觉去了。你不知那里顽去了,还不知道呢。”翠缕向紫鹃道:“断乎没有悄悄的睡去之理,只怕在那里走了一走。如今见老太太散了,赶过前边送去,也未可知。我们且往前边找找去。有了姑娘,自然你的茶钟也有了。你明日一早再找,有什么忙的。”媳妇笑道:“有了下落就不必忙了,明儿就和你要罢。”说毕回去,仍查收家伙。这里紫鹃和翠缕便往贾母处来。不在话下。
\end{parag}


\begin{parag}
    原来黛玉和湘云二人并未去睡觉。只因黛玉见贾府中许多人赏月,贾母犹叹人少,不似当年热闹,又提宝钗姊妹家去母女弟兄自去赏月等语,不觉对景感怀,自去俯栏垂泪。宝玉近因晴雯病势甚重,诸务无心,\begin{note}庚辰双行夹批:带一笔,妙!更觉谨密不漏。\end{note}王夫人再四遣他去睡,他也便去了。探春又因近日家事著恼,无暇游玩。虽有迎春惜春二人,偏又素日不大甚合。所以只剩了湘云一人宽慰他,因说:“你是个明白人,何必作此形像自苦。我也和你一样,我就不似你这样心窄。何况你又多病,还不自己保养。可恨宝姐姐,姊妹天天说亲道热,早已说今年中秋要大家一处赏月,必要起社,大家联句,到今日便弃了咱们,自己赏月去了。社也散了,诗也不作了。倒是他们父子叔侄纵横起来。你可知宋太祖说的好:‘卧榻之侧,岂许他人酣睡。’他们不作,咱们两个竟联起句来,明日羞他们一羞。”黛玉见他这般劝慰,不肯负他的豪兴,因笑道:“你看这里这等人声嘈杂,有何诗兴。”湘云笑道:“这山上赏月虽好,终不及近水赏月更妙。你知道这山坡底下就是池沿,山坳里近水一个所在就是凹晶馆。可知当日盖这园子时就有学问。这山之高处,就叫凸碧;山之低洼近水处,就叫作凹晶。这‘凸’‘凹’二字,历来用的人最少。如今直用作轩馆之名,更觉新鲜,不落窠臼。可知这两处一上一下,一明一暗,一高一矮,一山一水,竟是特因玩月而设此处。有爱那山高月小的,便往这里来;有爱那皓月清波的,便往那里去。只是这两个字俗念作‘洼’‘拱’二音,便说俗了,不大见用,只陆放翁用了一个‘凹’字,说‘古砚微凹聚墨多’,还有人批他俗,岂不可笑。”林黛玉道:“也不只放翁才用,古人中用者太多。如江淹《青苔赋》,东方朔《神异经》,以至《画记》上云张僧繇画一乘寺的故事,不可胜举。只是今人不知,误作俗字用了。实和你说罢,这两个字还是我拟的呢。因那年试宝玉,因他拟了几处,也有存的,也有删改的,也有尚未拟的。这是后来我们大家把这没有名色的也都拟出来了,注了出处,写了这房屋的坐落,一并带进去与大姐姐瞧了。他又带出来,命给舅舅瞧过。谁知舅舅倒喜欢起来,又说:‘早知这样,那日该就叫他姊妹一并拟了,岂不有趣。’所以凡我拟的,一字不改都用了。如今就往凹晶馆去看看。”
\end{parag}


\begin{parag}
    说著,二人便同下了山坡。只一转弯,就是池沿,沿上一带竹栏相接,直通著那边藕香榭的路径。\begin{note}庚辰双行夹批:点明妙!不然此园竟有多大地亩了。\end{note}因这几间就在此山怀抱之中,乃凸碧山庄之退居,因洼而近水,故颜其额曰“凹晶溪馆”。因此处房宇不多,且又矮小,故只有两个老婆子上夜。今日打听得凸碧山庄的人应差,与他们无干,这两个老婆子关了月饼果品并犒赏的酒食来,二人吃得既醉且饱,早已息灯睡了。\begin{note}庚辰双行夹批:妙极!此处又进一步写法。如王夫人云“他姊妹可怜,那里像当日林姑妈那样”,又如贾母云“如今人少,当日人多”等数语,此为进一步法也。也有退一步法,如宝钗之对邢岫烟云“此一时也,彼一时也,如今比不得先的话了,只好随事适分”,又如凤姐之对平儿云“如今我也明白了,我如今也要作好好先生罢”等类,此为退一步法也。今有方收拾,故贾母高乐却又写出二婆子高乐,此进一步之实事也。如前文海棠诗四首已足,忽又用湘云独成二律反压卷,此又进一步之实事也。所谓“法法皆全,丝丝不爽”也。\end{note}
\end{parag}


\begin{parag}
    黛玉湘云见息了灯,湘云笑道:“倒是他们睡了好。咱们就在这卷棚底下近水赏月如何?”二人遂在两个湘妃竹墩上坐下。只见天上一轮皓月,池中一轮水月,上下争辉,如置身于晶宫鲛室之内。微风一过,粼粼然池面皱碧铺纹,真令人神清气净。湘云笑道:“怎得这会子坐上船吃酒倒好。这要是我家里这样,我就立刻坐船了。”黛玉笑道:“正是古人常说的好,‘事若求全何所乐’。据我说,这也罢了,偏要坐船起来。”湘云笑道:“得陇望蜀,人之常情。可知那些老人家说的不错。说贫穷之家自为富贵之家事事趁心,告诉他说竟不能遂心,他们不肯信的;必得亲历其境,他方知觉了。就如咱们两个,虽父母不在,然却也忝在富贵之乡,只你我竟有许多不遂心的事。”黛玉笑道:“不但你我不能趁心,就连老太太,太太以至宝玉探丫头等人,无论事大事小,有理无理,其不能各遂其心者,同一理也,何况你我旅居客寄之人哉!”\begin{note}庚辰双行夹批:以立未不怡然得享自然之乐者矣。书中若干女子从生及婢未必有,各有所觉、各有所恃、各有所长者皆未如宝宝无可关切筹划,可叹。\end{note}湘云听说,恐怕黛玉又伤感起来,忙道:“休说这些闲话,咱们且联诗。”
\end{parag}


\begin{parag}
    正说间,只听笛韵悠扬起来。黛玉笑道:“今日老太太、太太高兴了,这笛子吹的有趣,到是助咱们的兴趣了。\begin{note}庚辰双行夹批:妙!正是吹笛之时分,认作一处之笛也。\end{note}咱两个都爱五言,就还是五言排律罢。”湘云道:“限何韵?”黛玉笑道:“咱们数这个栏杆的直棍,这头到那头为止。他是第几根就用第几韵。若十六根,便是‘一先’起。这可新鲜?”湘云笑道:“这倒别致。”于是二人起身,便从头数至尽头,止得十三根。湘云道:“偏又是‘十三元’了。这韵少,作排律只怕牵强不能押韵呢。少不得你先起一句罢了。”黛玉笑道:“倒要试试咱们谁强谁弱,只是没有纸笔记。”湘云道:“不妨,明儿再写。只怕这一点聪明还有。” 黛玉道:“我先起一句现成的俗语罢。”因念道:
\end{parag}
\begin{poem}
    \begin{pl}三五中秋夕,\end{pl}
\end{poem}


\begin{parag}
    湘云想了一想,道:
\end{parag}
\begin{poem}
    \begin{pl} 清游拟上元。撒天箕斗灿,\end{pl}
\end{poem}


\begin{parag}
    林黛玉笑道:
\end{parag}
\begin{poem}
    \begin{pl}
        匝地管弦繁。几处狂飞盏,
    \end{pl}
\end{poem}


\begin{parag}
    湘云笑道:“这一句‘几处狂飞盏’有些意思。这倒要对的好呢。”想了一想,笑道:
\end{parag}


\begin{poem}
    \begin{pl}
        谁家不启轩。轻寒风剪剪,
    \end{pl}
\end{poem}


\begin{parag}
    黛玉道:“对的比我的却好。只是底下这句又说熟话了,就该加劲说了去才是。”湘云道:“诗多韵险,也要铺陈些才是。纵有好的,且留在后头。”黛玉笑道:“到后头没有好的,我看你羞不羞。”因联道:
\end{parag}


\begin{poem}
    \begin{pl}
        良夜景暄暄。争饼嘲黄发,
    \end{pl}
\end{poem}


\begin{parag}
    湘云笑道:“这句不好,是你杜撰,用俗事来难我了。”黛玉笑道:“我说你不曾见过书呢。吃饼是旧典,唐书唐志你看了来再说。”湘云笑道:“这也难不倒我,我也有了。”因联道:
\end{parag}


\begin{poem}
    \begin{pl}
        分瓜笑绿嫒。香新荣玉桂,
    \end{pl}
\end{poem}


\begin{parag}
    黛玉笑道:“分瓜可是实实的你杜撰了。”湘云笑道:“明日咱们对查了出来大家看看,这会子别耽误工夫。”黛玉笑道:“虽如此,下句也不好,不犯著又用‘玉桂’‘金兰’等字样来塞责。”因联道:
\end{parag}


\begin{poem}
    \begin{pl}
        色健茂金萱。蜡烛辉琼宴,
    \end{pl}
\end{poem}


\begin{parag}
    湘云笑道:“‘金萱’二字便宜了你,省了多少力。这样现成的韵被你得了,只是不犯著替他们颂圣去。况且下句你也是塞责了。”黛玉笑道:“你不说‘玉桂’,我难道强对个‘金萱’么?再也要铺陈些富丽,方才是即景之实事。”湘云只得又联道:
\end{parag}


\begin{poem}
    \begin{pl}
        觥筹乱绮园。分曹尊一令,
    \end{pl}
\end{poem}


\begin{parag}
    黛玉笑道:“下句好,只是难对些。”因想了一想,联道:
\end{parag}
\begin{poem}
    \begin{pl}
        射复听三宣。骰彩红成点,

    \end{pl}
\end{poem}


\begin{parag}
    湘云笑道:“‘三宣’有趣,竟化俗成雅了。只是下句又说上骰子。”少不得联道:
\end{parag}
\begin{poem}
    \begin{pl}

        传花鼓滥喧。晴光摇院宇,

    \end{pl}
\end{poem}


\begin{parag}
    黛玉笑道:“对的却好。下句又溜了,只管拿些风月来塞责。”湘云道:“究竟没说到月上,也要点缀点缀,方不落题。”黛玉道:“且姑存之,明日再斟酌。”因联道:
\end{parag}
\begin{poem}
    \begin{pl}
        素彩接乾坤。赏罚无宾主,
    \end{pl}
\end{poem}


\begin{parag}
    湘云道:“又说他们作什么,不如说咱们。”只得联道:
\end{parag}
\begin{poem}
    \begin{pl}
        吟诗序仲昆。构思时倚槛,
    \end{pl}
\end{poem}


\begin{parag}
    黛玉道:“这可以入上你我了。”因联道:
\end{parag}


\begin{poem}
    \begin{pl}
        拟景或依门。酒尽情犹在,
    \end{pl}
\end{poem}


\begin{parag}
    湘云说道:“是时侯了。”乃联道:
\end{parag}


\begin{poem}
    \begin{pl}
        更残乐已谖。渐闻语笑寂,
    \end{pl}
\end{poem}


\begin{parag}
    黛玉说道:“这时侯可知一步难似一步了。”因联道:
\end{parag}
\begin{poem}
    \begin{pl}
        空剩雪霜痕。阶露团朝菌,
    \end{pl}
\end{poem}


\begin{parag}
    湘云笑道:“这一句怎么押韵,让我想想。”因起身负手,想了一想,笑道:“够了,幸而想出一个字来,几乎败了。”因联道:
\end{parag}


\begin{poem}
    \begin{pl}

        庭烟敛夕棔。秋湍泻石髓,
    \end{pl}
\end{poem}


\begin{parag}
    黛玉听了,不禁也起身叫妙,说:“这促狭鬼,果然留下好的。这会子才说‘棔’字,亏你想得出。”湘云道:“幸而昨日看历朝文选见了这个字,我不知是何树,因要查一查。宝姐姐说不用查,这就是如今俗叫作明开夜合的。我信不及,到底查了一查,果然不错。看来宝姐姐知道的竟多。”黛玉笑道:“‘棔’字用在此时更恰,也还罢了。只是‘秋湍’一句亏你好想。只这一句,别的都要抹倒。我少不得打起精神来对一句,只是再不能似这一句了。”因想了一想,道:
\end{parag}


\begin{poem}
    \begin{pl}
        风叶聚云根。宝婺情孤洁,
    \end{pl}
\end{poem}


\begin{parag}
    湘云道:“这对的也还好。只是下一句你也溜了,幸而是景中情,不单用‘宝婺’来塞责。”因联道:
\end{parag}
\begin{poem}
    \begin{pl}
        银蟾气吐吞。药经灵兔捣,
    \end{pl}
\end{poem}


\begin{parag}
    黛玉不语点头,半日随念道:
\end{parag}
\begin{poem}
    \begin{pl}
        人向广寒奔。犯斗邀牛女,
    \end{pl}
\end{poem}


\begin{parag}
    湘云也望月点首,联道:
\end{parag}
\begin{poem}
    \begin{pl}
        乘槎待帝孙。虚盈轮莫定,
    \end{pl}
\end{poem}


\begin{parag}
    黛玉笑道:“又用比兴了。”因联道:
\end{parag}
\begin{poem}
    \begin{pl}
        晦朔魄空存。壶漏声将涸,
    \end{pl}
\end{poem}


\begin{parag}
    湘云方欲联时,黛玉指池中黑影与湘云看道:“你看那河里怎么象个人在黑影里去了,敢是个鬼罢?”湘云笑道:“可是又见鬼了。我是不怕鬼的,等我打他一下。”因弯腰拾了一块小石片向那池中打去,只听打得水响,一个大圆圈将月影荡散复聚者几次。\begin{note}庚辰双行夹批:写得出。试思若非亲历其境者如何摹写得如此。\end{note}只听那黑影里嘎然一声,却飞起一个大白鹤来,\begin{note}庚辰双行夹批:写得出。\end{note}直往藕香榭去了。黛玉笑道:“原来是他,猛然想不到,反吓了一跳。”湘云笑道:“这个鹤有趣,倒助了我了。”因联道:
\end{parag}
\begin{poem}
    \begin{pl}

        窗灯焰已昏。寒塘渡鹤影,

    \end{pl}
\end{poem}


\begin{parag}
    林黛玉听了,又叫好,又跺足,说:“了不得,这鹤真是助他的了!这一句更比‘秋湍’不同,叫我对什么才好?‘影’字只有一个‘魂’字可对,况且‘寒塘渡鹤’何等自然,何等现成,何等有景且又新鲜,我竟要搁笔了。”湘云笑道:“大家细想就有了,不然就放著明日再联也可。”黛玉只看天,不理他,半日,猛然笑道:“你不必说嘴,我也有了,你听听。”因对道:
\end{parag}


\begin{poem}
    \begin{pl}
        冷月葬花魂。
    \end{pl}
\end{poem}


\begin{parag}
    湘云拍手赞道:“果然好极!非此不能对。好个‘葬花魂’!”因又叹道:“诗固新奇,只是太颓丧了些。你现病著,不该作此过于清奇诡谲之语。”黛玉笑道:“不如此如何压倒你。下句竟还未得,只为用工在这一句了。”
\end{parag}


\begin{parag}
    一语未了,只见栏外山石后转出一个人来,笑道:“好诗,好诗,果然太悲凉了。不必再往下联,若底下只这样去,反不显这两句了,倒觉得堆砌牵强。”二人不防,倒唬了一跳。细看时,不是别人,却是妙玉。二人皆诧异,\begin{note}庚辰双行夹批:原可诧异,余亦诧异。\end{note}因问:“你如何到了这里?”妙玉笑道:“我听见你们大家赏月,又吹的好笛,我也出来玩赏这清池皓月。顺脚走到这里,忽听见你两个联诗,更觉清雅异常,故此听住了。只是方才我听见这一首中,有几句虽好,只是过于颓败凄楚。此亦关人之气数而有,所以我出来止住。如今老太太都已早散了,满园的人想俱已睡熟了,你两个的丫头还不知在那里找你们呢。你们也不怕冷了?快同我来,到我那里去吃杯茶,只怕就天亮了。”黛玉笑道:“谁知道就这个时侯了。”
\end{parag}


\begin{parag}
    三人遂一同来至栊翠庵中。只见龛焰犹青,炉香未烬。几个老嬷嬷也都睡了,只有小丫鬟在蒲团上垂头打盹。妙玉唤他起来,现去烹茶。忽听叩门之声,小丫鬟忙去开门看时,却是紫鹃翠缕与几个老嬷嬷来找他姊妹两个。进来见他们正吃茶,因都笑道:“要我们好找,一个园里走遍了,连姨太太那里都找到了。才到了那山坡底下小亭里找时,可巧那里上夜的正睡醒了。我们问他们,他们说,方才亭外头棚下两个人说话,后来又添了一个,听见说大家往庵里去。我们就知是这里了。” 妙玉忙命小丫鬟引他们到那边去坐著歇息吃茶。自取了笔砚纸墨出来,将方才的诗命他二人念著,遂从头写出来。黛玉见他今日十分高兴,便笑道:“从来没见你这样高兴。我也不敢唐突请教,这还可以见教否?若不堪时,便就烧了;若或可政,即请改正改正。”妙玉笑道:“也不敢妄加评赞。只是这才有了二十二韵。我意思想著你二位警句已出,再若续时,恐后力不加。我竟要续貂,又恐有玷。”黛玉从没见妙玉作过诗,今见他高兴如此,忙说:“果然如此,我们的虽不好,亦可以带好了。”妙玉道:“如今收结,到底还该归到本来面目上去。若只管丢了真情真事且去搜奇捡怪,一则失了咱们的闺阁面目,二则也与题目无涉了。”二人皆道极是。妙玉遂提笔一挥而就,递与他二人道:“休要见笑。依我必须如此,方翻转过来,虽前头有凄楚之句,亦无甚碍了。”二人接了看时,只见他续道:
\end{parag}


\begin{poem}
    \begin{pl}
        香篆销金鼎,脂冰腻玉盆。
    \end{pl}
    \begin{pl}
        箫增嫠妇泣,衾倩侍儿温。
    \end{pl}
    \begin{pl}
        空帐悬文凤,闲屏掩彩鸳。
    \end{pl}
    \begin{pl}
        露浓苔更滑,霜重竹难扪。
    \end{pl}
    \begin{pl}
        犹步萦纡沼,还登寂历原。
    \end{pl}
    \begin{pl}
        石奇神鬼搏,木怪虎狼蹲。
    \end{pl}
    \begin{pl}
        赑屃朝光透,罘罳晓露屯。
    \end{pl}
    \begin{pl}
        振林千树鸟,啼谷一声猿。
    \end{pl}
    \begin{pl}
        歧熟焉忘径,泉知不问源。
    \end{pl}
    \begin{pl}
        钟鸣栊翠寺,鸡唱稻香村。
    \end{pl}
    \begin{pl}
        有兴悲何继,无愁意岂烦。
    \end{pl}
    \begin{pl}
        芳情只自遣,雅趣向谁言。
    \end{pl}
    \begin{pl}
        彻旦休云倦,烹茶更细论。
    \end{pl}

\end{poem}


\begin{parag}
    后书:《右中秋夜大观园即景联句三十五韵》。
\end{parag}


\begin{parag}
    黛玉湘云二人皆赞赏不已,说:“可见我们天天是舍近而求远。现有这样诗仙在此,却天天去纸上谈兵。”妙玉笑道:“明日再润色。此时想也快天亮了,到底要歇息歇息才是。”林史二人听说,便起身告辞,带领丫鬟出来。妙玉送至门外,看他们去远,方掩门进来。不在话下。
\end{parag}


\begin{parag}
    这里翠缕向湘云道:“大奶奶那里还有人等著咱们睡去呢。如今还是那里去好?”湘云笑道:“你顺路告诉他们,叫他们睡罢。我这一去未免惊动病人,不如闹林姑娘半夜去罢。”说著,大家走至潇湘馆中,有一半人已睡去。二人进去,方才卸妆宽衣,盥漱已毕,方上床安歇。紫鹃放下绡帐,移灯掩门出去。谁知湘云有择席之病,虽在枕上,只是睡不著。黛玉又是个心血不足常常失眠的,今日又错过困头,自然也是睡不著。二人在枕上翻来覆去。黛玉因问道:“怎么你还没睡著?” 湘云微笑道:“我有择席的病,况且走了困,只好躺躺罢。你怎么也睡不著?”黛玉叹道:\begin{note}庚辰双行夹批:一“笑”一“叹”,只二字便写出平日之形景。\end{note}“我这睡不著也并非今日,大约一年之中,通共也只好睡十夜满足的。”湘云道:“都是你病的原故,所以……”不知下文什么──
\end{parag}


\begin{parag}
    \begin{note}蒙回末总评:诗词清远闲旷,自是慧业才人,何须赘评?须看他众人联句填词时,个人性情,个人意见,叙来恰肖其人;二人联诗时,一番讥评,一番赏叹,叙来更得其神。再看漏永吟残,忽开一洞天福地,字字出人意表。\end{note}
\end{parag}


\begin{parag}
    \begin{note}蒙回末总评:只一品笛,疑有疑无,若近若远,有无限逸致。\end{note}
\end{parag}