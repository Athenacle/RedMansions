\chap{七十一}{嫌隙人有心生嫌隙 鴛鴦女無意遇鴛鴦}


\begin{parag}
    \begin{note}蒙回前總:敘賈母開壽誕,與寧府祭宗祠是一樣手筆,俱爲五鳳裁詔體。\end{note}
\end{parag}


\begin{parag}
    話說賈政回京之後,諸事完畢,賜假一月在家歇息。因年景漸老,事重身衰,又近因在外幾年,骨肉離異,今得晏然復聚於庭室,自覺喜幸不盡。一應大小事務一概益發付於度外,只是看書,悶了便與清客們下棋喫酒,或日間在裏面母子夫妻共敘天倫庭闈之樂。
\end{parag}


\begin{parag}
    因今歲八月初三日乃賈母八旬之慶,又因親友全來,恐筵宴排設不開,便早同賈赦及賈珍賈璉等商議,議定於七月二十八日起至八月初五日止榮寧兩處齊開筵宴,寧國府中單請官客,榮國府中單請堂客,大觀園中收拾出綴錦閣並嘉蔭堂等幾處大地方來作退居。二十八日請皇親駙馬王公諸公主郡主王妃國君太君夫人等,二十九日便是閣下都府督鎮及誥命等,三十日便是諸官長及誥命並遠近親友及堂客。初一日是賈赦的家宴,初二日是賈政,初三日是賈珍賈璉,初四日是賈府中合族長幼大小共湊的家宴。初五日是賴大林之孝等家下管事人等共湊一日。自七月上旬,送壽禮者便絡繹不絕。禮部奉旨:欽賜金玉如意一柄,綵緞四端,金玉環四個,帑銀五百兩。元春又命太監送出金壽星一尊,沉香拐一隻,伽南珠一串,福壽香一盒,金錠一對,銀錠四對,綵緞十二匹,玉杯四隻。餘者自親王駙馬以及大小文武官員之家凡所來往者,莫不有禮,不能勝記。堂屋內設下大桌案,鋪了紅氈,將凡所有精細之物都擺上,請賈母過目。賈母先一二日還高興過來瞧瞧,後來煩了,也不過目,只說:“叫鳳丫頭收了,改日悶了再瞧。”
\end{parag}


\begin{parag}
    至二十八日,兩府中俱懸燈結彩,屏開鸞鳳,褥設芙蓉,笙簫鼓樂之音,通衢越巷。寧府中本日只有北靜王、南安郡王、永昌駙馬、樂善郡王並幾個世交公侯應襲,榮府中南安王太妃、北靜王妃並幾位世交公侯誥命。賈母等皆是按品大妝迎接。大家廝見,先請入大觀園內嘉蔭堂,茶畢更衣,方出至榮慶堂上拜壽入席。大家謙遜半日,方纔入席。上面兩席是南北王妃,下面依敘,便是衆公侯誥命。左邊下手一席,陪客是錦鄉侯誥命與臨昌伯誥命,右邊下手一席,方是賈母主位。邢夫人王夫人帶領尤氏、鳳姐並族中幾個媳婦,兩溜雁翅站在賈母身後侍立。林之孝賴大家的帶領衆媳婦都在竹簾外面侍候上菜上酒,周瑞家的帶領幾個丫鬟在圍屏後侍候呼喚。凡跟來的人,早又有人別處管待去了。一時臺上參了場,臺下一色十二個未留髮的小廝侍候。須臾,一小廝捧了戲單至階下,先遞與回事的媳婦。這媳婦接了,才遞與林之孝家的,用一小茶盤託上,挨身入簾來遞與尤氏的侍妾佩鳳。佩鳳接了才奉與尤氏。尤氏託著走至上席,南安太妃謙讓了一回,點了一出吉慶戲文,然後又謙讓了一回,北靜王妃也點了一出。衆人又讓了一回,命隨便揀好的唱罷了。少時,菜已四獻,湯始一道,跟來各家的放了賞。大家便更衣復入園來,另獻好茶。
\end{parag}


\begin{parag}
    南安太妃因問寶玉,賈母笑道:“今日幾處廟裏念‘保安延壽經’,他跪經去了。”又問衆小姐們,賈母笑道:“他們姊妹們病的病,弱的弱,見人靦腆,所以叫他們給我看屋子去了。有的是小戲子,傳了一班在那邊廳上陪著他姨娘家姊妹們也看戲呢。”南安太妃笑道:“既這樣,叫人請來。”賈母回頭命鳳姐兒去把史、薛、林帶來,“再只叫你三妹妹陪著來罷。”鳳姐答應了,來至賈母這邊,只見他姊妹們正喫果子看戲,寶玉也才從廟裏跪經回來。鳳姐兒說了話。寶釵姊妹與黛玉探春湘雲五人來至園中,大家見了,不過請安問好讓坐等事。衆人中也有見過的,還有一兩家不曾見過的,都齊聲誇讚不絕。其中湘雲最熟,南安太妃因笑道:“你在這裏,聽見我來了還不出來,還只等請去。我明兒和你叔叔算帳。”因一手拉著探春,一手拉著寶釵,問幾歲了,又連聲誇讚。因又鬆了他兩個,又拉著黛玉寶琴,也著實細看,極誇一回。又笑道:“都是好的,你不知叫我誇那一個的是。”早有人將備用禮物打點出五分來:金玉戒指各五個,腕香珠五串。南安太妃笑道: “你姊妹們別笑話,留著賞丫頭們罷。”五人忙拜謝過。北靜王妃也有五樣禮物,餘者不必細說。
\end{parag}


\begin{parag}
    吃了茶,園中略逛了一逛,賈母等因又讓入席。南安太妃便告辭,說身上不快,“今日若不來,實在使不得,因此恕我竟先要告別了。”賈母等聽說,也不便強留,大家又讓了一回,送至園門,坐轎而去。接著北靜王妃略坐一坐也就告辭了。餘者也有終席的,也有不終席的。
\end{parag}


\begin{parag}
    賈母勞乏了一日,次日便不會人,一應都是邢夫人王夫人管待。有那些世家子弟拜壽的,只到廳上行禮,賈赦、賈政、賈珍等還禮管待,至寧府坐席。不在話下。
\end{parag}


\begin{parag}
    這幾日,尤氏晚間也不回那府裏去,白日間待客,晚間在園內李氏房中歇宿。這日晚間伏侍過賈母晚飯後,賈母因說:“你們也乏了,我也乏了,早些尋一點子喫的歇歇去。明兒還要起早鬧呢。”尤氏答應著退了出來,到鳳姐兒房裏來喫飯。鳳姐兒在樓上看著人收送禮的新圍屏,只有平兒在房裏與鳳姐兒疊衣服。尤氏因問:“你們奶奶吃了飯了沒有?”平兒笑道:“喫飯豈不請奶奶去的。”尤氏笑道:“既這樣,我別處找喫的去。餓的我受不得了。”說著就走。平兒忙笑道:“奶奶請回來。這裏有點心,且點補一點兒,回來再喫飯。”尤氏笑道:“你們忙的這樣,我園裏和他姊妹們鬧去。”一面說,一面就走。平兒留不住,只得罷了。
\end{parag}


\begin{parag}
    且說尤氏一徑來至園中,只見園中正門與各處角門\begin{note}庚夾:伏下文。\end{note}仍未關,猶吊著各色彩燈,因回頭命小丫頭叫該班的女人。那丫鬟走入班房中,竟沒一個人影,回來回了尤氏。尤氏便命傳管家的女人。這丫頭應了便出去,到二門外鹿頂內,乃是管事的女人議事取齊之所。到了這裏,只有兩個婆子分菜果呢。因問:“那一位奶奶在這裏?東府奶奶立等一位奶奶,有話吩咐。”這兩個婆子只顧分菜果,又聽見是東府裏的奶奶,不大在心上,因就回說:“管家奶奶們才散了。”小丫頭道:“散了,你們家裏傳他去。”婆子道:“我們只管看屋子,不管傳人。姑娘要傳人再派傳人的去。”小丫頭聽了道:“噯呀,噯呀,這可反了!怎麼你們不傳去?你哄那新來了的,怎麼哄起我來了!素日你們不傳誰傳去!這會子打聽了梯己信兒,或是賞了那位管家奶奶的東西,你們爭著狗顛兒似的傳去的,不知誰是誰呢。璉二奶奶要傳,你們可也這麼回?”這兩個婆子一則吃了酒,二則被這丫頭揭挑著弊病,便羞激怒了,因回口道:“扯你的臊!我們的事,傳不傳不與你相干!你不用揭挑我們,你想想,你那老子娘在那邊管家爺們跟前比我們還更會溜呢。什麼‘清水下雜麪,你喫我也見’的事,各家門,另家戶,你有本事,排場你們那邊人去。我們這邊,你們還早些呢!”丫頭聽了,氣白了臉,因說道:“好,好,這話說的好!”一面轉身進來回話。
\end{parag}


\begin{parag}
    尤氏已早入園來,因遇見了襲人、寶琴、湘雲三人同著地藏庵的兩個姑子正說故事頑笑,尤氏因說餓了,先到怡紅院,襲人裝了幾樣葷素點心出來與尤氏喫。兩個姑子、寶琴、湘雲等都喫茶,仍說故事。那小丫頭子一徑找了來,氣狠狠的把方纔的話都說了出來。尤氏聽了,冷笑道:“這是兩個什麼人?”兩個姑子並寶琴湘雲等聽了,生怕尤氏生氣,忙勸說:“沒有的事,必是這一個聽錯了。”兩個姑子笑推這丫頭道:“你這孩子好性氣,那糊塗老嬤嬤們的話,你也不該來回纔是。咱們奶奶萬金之軀,勞乏了幾日,黃湯辣水沒喫,咱們哄他歡喜一會還不得一半兒,說這些話做什麼。”襲人也忙笑拉出他去,說:“好妹子,你且出去歇歇,我打發人叫他們去。”尤氏道:“你不要叫人,你去就叫這兩個婆子來,到那邊把他們家的鳳兒叫來。”襲人笑道:“我請去。”尤氏道:“偏不要你去。”兩個姑子忙立起身來,笑道:“奶奶素日寬洪大量,今日老祖宗千秋,奶奶生氣,豈不惹人談論。”寶琴湘雲二人也都笑勸。尤氏道:“不爲老太太的千秋,我斷不依。且放著就是了。”
\end{parag}


\begin{parag}
    說話之間,襲人早又遣了一個丫頭去到園門外找人,可巧遇見周瑞家的,這小丫頭子就把這話告訴周瑞家的。周瑞家的雖不管事,因他素日仗著是王夫人的陪房,原有些體面,心性乖滑,專管各處獻勤討好,所以各處房裏的主人都喜歡他。他今日聽了這話,忙的便跑入怡紅院來,一面飛走,一面口內說:“氣壞了奶奶了,可了不得!我們家裏,如今慣的太不堪了。偏生我不在跟前,若在跟前,且打給他們幾個耳刮子,再等過了這幾日算帳。”尤氏見了他,也便笑道:“周姐姐你來,有個理你說說。這早晚門還大開著,明燈蠟燭,出入的人又雜,倘有不防的事,如何使得?因此叫該班的人吹燈關門。誰知一個人芽兒也沒有。”周瑞家的道: “這還了得!前兒二奶奶還吩咐了他們,說這幾日事多人雜,一晚就關門吹燈,不是園裏人不許放進去。今兒就沒了人。這事過了這幾日,必要打幾個纔好。”尤氏又說小丫頭子的話。周瑞家的道:“奶奶不要生氣,等過了事,我告訴管事的打他個臭死。只問他們,誰叫他們說這‘各家門各家戶’的話!我已經叫他們吹了燈,關上正門和角門子。”正亂著,只見鳳姐兒打發人來請喫飯。尤氏道:“我也不餓了,才吃了幾個餑餑,請你奶奶自喫罷。”
\end{parag}


\begin{parag}
    一時周瑞家的得便出去,便把方纔的事回了鳳姐,又說:“這兩個婆婆就是管家奶奶,時常我們和他說話,都似狠蟲一般。奶奶若不戒飭,大奶奶臉上過不去。”鳳姐道:“既這麼著,記上兩個人的名字,等過了這幾日,捆了送到那府裏憑大嫂子開發,或是打幾下子,或是開恩饒了他們,隨他去就是了,什麼大事。” 周瑞家的聽了,巴不得一聲兒,素日因與這幾個人不睦,出來了便命一個小廝到林之孝家傳鳳姐的話,立刻叫林之孝家的進來見大奶奶,一面又傳人立刻捆起這兩個婆子來,交到馬圈裏派人看守。
\end{parag}


\begin{parag}
    林之孝家的不知有什麼事,此時已經點燈,忙坐車進來,先見鳳姐。至二門上傳進話去,丫頭們出來說:“奶奶才歇了。大奶奶在園裏,叫大娘見了大奶奶就是了。”林之孝家的只得進園來到稻香村,丫鬟們回進去,尤氏聽了反過意不去,忙喚進他來,因笑向他道:“我不過爲找人找不著因問你,你既去了,也不是什麼大事,誰又把你叫進來,倒要你白跑一遭。不大的事,已經撒開手了。”林之孝家的也笑道:“二奶奶打發人傳我,說奶奶有話吩咐。”尤氏笑道:“這是那裏的話,只當你沒去,白問你。這是誰又多事告訴了鳳丫頭,大約周姐姐說的。家去歇著罷,沒有什麼大事。”李紈又要說原故,尤氏反攔住了。林之孝家的見如此,只得便回身出園去。可巧遇見趙姨娘,姨娘因笑道:“噯喲喲,我的嫂子!這會子還不家去歇歇,還跑些什麼?”林之孝家的便笑說何曾不家去的,如此這般進來了。又是個齊頭故事。趙姨娘原是好察聽這些事的,且素日又與管事的女人們扳厚,互相連絡,好作首尾。方纔之事,已竟聞得八九,聽林之孝家的如此說,便恁般如此告訴了林之孝家的一遍,林之孝家的聽了,笑道:“原來是這事,也值一個屁!開恩呢,就不理論,心窄些兒,也不過打幾下子就完了。”趙姨娘道:“我的嫂子,事雖不大,可見他們太張狂了些。巴巴的傳進你來,明明戲弄你,頑算你。快歇歇去,明兒還有事呢,也不留你喫茶去。”
\end{parag}


\begin{parag}
    說畢,林之孝家的出來,到了側門前,就有方纔兩個婆子的女兒上來哭著求情。林之孝家的笑道:“你這孩子好糊塗,誰叫你娘喫酒混說了,惹出事來,連我也不知道。二奶奶打發人捆他,連我還有不是呢。我替誰討請去。”這兩個小丫頭子才七八歲,原不識事,只管哭啼求告。纏的林之孝家的沒法,因說道:“糊塗東西!你放著門路不去,卻纏我來。你姐姐現給了那邊太太作陪房費大娘的兒子,你走過去告訴你姐姐,叫親家娘和太太一說,什麼完不了的事!”一語提醒了一個,那一個還求。林之孝家的啐道:“糊塗攮的!他過去一說,自然都完了。沒有個單放了他媽,又只打你媽的理。”說畢,上車去了。
\end{parag}


\begin{parag}
    這一個小丫頭果然過來告訴了他姐姐,和費婆子說了。這費婆子原是邢夫人的陪房,起先也曾興過時,只因賈母近來不大作興邢夫人,所以連這邊的人也減了威勢。凡賈政這邊有些體面的人,那邊各各皆虎視耽耽。這費婆子常倚老賣老,仗著邢夫人,常喫些酒,嘴裏胡罵亂怨的出氣。如今賈母慶壽這樣大事,幹看著人家逞才賣技辦事,呼幺喝六弄手腳,心中早已不自在,指雞罵狗,閒言閒語的亂鬧。這邊的人也不和他較量。如今聽了周瑞家的捆了他親家,越發火上澆油,仗著酒興,指著隔斷的牆\begin{note}庚雙夾:細緻之甚。\end{note}大罵了一陣,便走上來求邢夫人,說他親家並沒什麼不是,“不過和那府裏的大奶奶的小丫頭白鬥了兩句話,周瑞家的便調唆了咱家二奶奶捆到馬圈裏,等過了這兩日還要打。求太太──我那親家娘也是七八十歲的老婆子──和二奶奶說聲,饒他這一次罷。”邢夫人自爲要鴛鴦之後討了沒意思,後來見賈母越發冷淡了他,鳳姐的體面反勝自己,且前日南安太妃來了,要見他姊妹,賈母又只令探春出來,迎春竟似有如無,自己心內早已怨忿不樂,只是使不出來。又值這一干小人在側,他們心內嫉妒挾怨之事不敢施展,便背地裏造言生事,調撥主人。先不過是告那邊的奴才,後來漸次告到鳳姐“只哄著老太太喜歡了他好就中作威作福,轄治著璉二爺,調唆二太太,把這邊的正經太太倒不放在心上。”後來又告到王夫人,說:“老太太不喜歡太太,都是二太太和璉二奶奶調唆的。”邢夫人縱是鐵心銅膽的人,婦女家終不免生些嫌隙之心,近日因此著實惡絕鳳姐。今聽了如此一篇話,也不說長短。
\end{parag}


\begin{parag}
    至次日一早,見過賈母,衆族人中到齊,坐席開戲。賈母高興,又見今日無遠親,都是自己族中子侄輩,只便衣常妝出來,堂上受禮。當中獨設一榻,引枕靠背腳踏俱全,自己歪在榻上。榻之前後左右,皆是一色的小矮凳,寶釵、寶琴、黛玉、湘雲、迎春、探春、惜春姊妹等圍繞。因賈㻞之母也帶了女兒喜鸞,賈瓊之母也帶了女兒四姐兒,還有幾房的孫女兒,大小共有二十來個。賈母獨見喜鸞和四姐兒生得又好,說話行事與衆不同,心中喜歡,便命他兩個也過來榻前同坐。寶玉卻在榻上腳下與賈母捶腿。首席便是薛姨媽,下邊兩溜皆順著房頭輩數下去。簾外兩廊都是族中男客,也依次而坐。
\end{parag}


\begin{parag}
    先是那女客一起一起行禮,後方是男客行禮。賈母歪在榻上,只命人說“免了罷”,早已都行完了。然後賴大等帶領衆人,從儀門直跪至大廳上,磕頭禮畢,又是衆家下媳婦,然後各房的丫鬟,足鬧了兩三頓飯時。然後又抬了許多雀籠來,在當院中放了生。賈赦等焚過了天地壽星紙,方開戲飲酒。直到歇了中臺,賈母方進來歇息,命他們取便,因命鳳姐兒留下喜鸞四姐兒頑兩日再去。鳳姐兒出來便和他母親說,他兩個母親素日都承鳳姐的照顧,也巴不得一聲兒。他兩個也願意在園內頑耍,至晚便不回家了。
\end{parag}


\begin{parag}
    邢夫人直至晚間散時,當著許多人陪笑和鳳姐求情說:“我聽見昨兒晚上二奶奶生氣,打發周管家的娘子捆了兩個老婆子,可也不知犯了什麼罪。論理我不該討情,我想老太太好日子,發狠的還舍錢舍米,周貧濟老,咱們家先倒折磨起人家來了。不看我的臉,權且看老太太,竟放了他們罷。”說畢,上車去了。鳳姐聽了這話,又當著許多人,又羞又氣,一時抓尋不著頭腦,憋得臉紫漲,回頭向賴大家的等笑道:\begin{note}庚雙夾:又寫笑,妙!凡鳳真怒處必曰“笑”,絲絲不錯。\end{note}“這是那裏的話。昨兒因爲這裏的人得罪了那府裏的大嫂子,我怕大嫂子多心,所以儘讓他發放,並不爲得罪了我。這又是誰的耳報神這麼快。”王夫人因問爲什麼事,鳳姐兒笑將昨日的事說了。尤氏也笑道:“連我並不知道。你原也太多事了。”鳳姐兒道:“我爲你臉上過不去,所以等你開發,不過是個禮。就如我在你那裏有人得罪了我,你自然送了來盡我。憑他是什麼好奴才,到底錯不過這個禮去。這又不知誰過去沒的獻勤兒,這也當一件事情去說。”王夫人道:“你太太說的是。就是珍哥兒媳婦也不是外人,也不用這些虛禮。老太太的千秋要緊,放了他們爲是。”說著,回頭便命人去放了那兩個婆子。鳳姐由不得越想越氣越愧,不覺的灰心轉悲,滾下淚來。因賭氣回房哭泣,又不使人知覺。偏是賈母打發了琥珀來叫立等說話。琥珀見了,詫異道:“好好的,這是什麼原故?那裏立等你呢。”鳳姐聽了,忙擦乾了淚,洗面另施了脂粉,方同琥珀過來。
\end{parag}


\begin{parag}
    賈母因問道:“前兒這些人家送禮來的共有幾家有圍屏?”鳳姐兒道:“共有十六家有圍屏,十二架大的,四架小的炕屏。內中只有江南甄家\begin{note}庚雙夾:好,一提甄家。蓋真事將顯,假事將盡。\end{note}一架大屏十二扇,大紅緞子緙絲‘滿牀笏’,一面是泥金‘百壽圖’的,是頭等的。還有粵海將軍鄔家一架玻璃的還罷了。”賈母道:“既這樣,這兩架別動,好生擱著,我要送人的。”鳳姐兒答應了。鴛鴦忽過來向鳳姐兒面上只管瞧,引的賈母問說:“你不認得他?只管瞧什麼。”鴛鴦笑道:“怎麼他的眼腫腫的,所以我詫異,只管看。”賈母聽說,便叫進前來,也覷著眼看。鳳姐笑道:“才覺的一陣癢癢,揉腫了些。”鴛鴦笑道: “別又是受了誰的氣了不成?”鳳姐道:“誰敢給我氣受,便受了氣,老太太好日子,我也不敢哭的。”賈母道:“正是呢。我正要喫晚飯,你在這裏打發我喫,剩下的你就和珍兒媳婦吃了。你兩個在這裏幫著兩個師傅替我揀佛豆兒,你們也積積壽,前兒你姊妹們和寶玉都揀了,如今也叫你們揀揀,別說我偏心。”說話時,先擺上一桌素的來。兩個姑子吃了,然後才擺上葷的,賈母喫畢,擡出外間。尤氏鳳姐兒二人正喫,賈母又叫把喜鸞四姐兒二人也叫來,跟他二人喫畢,洗了手,點上香,捧過一升豆子來。兩個姑子先念了佛偈,然後一個一個的揀在一個簸籮內,每揀一個,念一聲佛。明日煮熟了,令人在十字街結壽緣。賈母歪著聽兩個姑子又說些佛家的因果善事。
\end{parag}


\begin{parag}
    鴛鴦早已聽見琥珀說鳳姐哭之事,又和平兒前打聽得原故。晚間人散時,便回說:“二奶奶還是哭的,那邊大太太當著人給二奶奶沒臉。”賈母因問爲什麼原故,鴛鴦便將原故說了。賈母道:“這纔是鳳丫頭知禮處,難道爲我的生日由著奴才們把一族中的主子都得罪了也不管罷。這是太太素日沒好氣,不敢發作,所以今兒拿著這個作法子,明是當著衆人給鳳兒沒臉罷了。”正說著,只見寶琴等進來,也就不說了。賈母因問:“你在那裏來?”寶琴道:“在園裏林姐姐屋裏大家說話的。”賈母忽想起一事來,忙喚一個老婆子來,吩咐他:“到園裏各處女人們跟前囑咐囑咐,留下的喜姐兒和四姐兒雖然窮,也和家裏的姑娘們是一樣,大家照看經心些。我知道咱們家的男男女女都是‘一個富貴心,兩隻體面眼’,未必把他兩個放在眼裏。有人小看了他們,我聽見可不依。”婆子應了方要走時,鴛鴦道:“我說去罷。他們那裏聽他的話。”說著,便一徑往園子來。
\end{parag}


\begin{parag}
    先到稻香村中,李紈與尤氏都不在這裏。問丫鬟們,說“都在三姑娘那裏呢。”鴛鴦回身又來至曉翠堂,果見那園中人都在那裏說笑。見他來了,都笑說:“你這會子又跑來做什麼?”又讓他坐。鴛鴦笑道:“不許我也逛逛麼?”於是把方纔的話說了一遍。李紈忙起身聽了,就叫人把各處的頭兒喚了一個來。令他們傳與諸人知道。不在話下。這裏尤氏笑道:“老太太也太想的到,實在我們年輕力壯的人捆上十個也趕不上。”李紈道:“鳳丫頭仗著鬼聰明兒,還離腳蹤兒不遠。咱們是不能的了。”鴛鴦道:“罷喲,還提鳳丫頭虎丫頭呢,他也可憐見兒的。雖然這幾年沒有在老太太、太太跟前有個錯縫兒,暗裏也不知得罪了多少人。總而言之,爲人是難作的:若太老實了沒有個機變,公婆又嫌太老實了,家裏人也不怕;若有些機變,未免又治一經損一經。如今咱們家裏更好,新出來的這些底下奴字號的奶奶們,一個個心滿意足,都不知要怎麼樣纔好,少有不得意,不是背地裏咬舌根,就是挑三窩四的。我怕老太太生氣,一點兒也不肯說。不然我告訴出來,大家別過太平日子。這不是我當著三姑娘說,老太太偏疼寶玉,有人背地裏怨言還罷了,算是偏心。如今老太太偏疼你,我聽著也是不好。這可笑不可笑?”探春笑道:“糊塗人多,那裏較量得許多。我說倒不如小人家人少,雖然寒素些,倒是歡天喜地,大家快樂。我們這樣人家人多,外頭看著我們不知千金萬金小姐,何等快樂,殊不知我們這裏說不出來的煩難,更利害。”寶玉道:“誰都象三妹妹好多心。事事我常勸你,總別聽那些俗語,想那俗事,只管安富尊榮纔是。比不得我們沒這清福,該應濁鬧的。”尤氏道:“誰都像你,真是一心無掛礙,只知道和姊妹們頑笑,餓了喫,困了睡,再過幾年,不過還是這樣,一點後事也不慮。”寶玉笑道:“我能夠和姊妹們過一日是一日,死了就完了。什麼後事不後事。”李紈等都笑道:“這可又是胡說。就算你是個沒出息的,終老在這裏,難道他姊妹們都不出門的?”尤氏笑道:“怨不得人都說他是假長了一個胎子,究竟是個又傻又呆的。”寶玉笑道:“人事莫定,知道誰死誰活。倘或我在今日明日,今年明年死了,也算是遂心一輩子了。”衆人不等說完,便說:“可是又瘋了,別和他說話纔好。若和他說話,不是呆話就是瘋話。”喜鸞因笑道:“二哥哥,你別這樣說,等這裏姐姐們果然都出了閣,橫豎老太太、太太也寂寞,我來和你作伴兒。”李紈尤氏等都笑道:“姑娘也別說呆話,難道你是不出門的?這話哄誰。”說的喜鸞低了頭。當下已是起更時分,大家各自歸房安歇,衆人都且不提。
\end{parag}


\begin{parag}
    且說鴛鴦一徑回來,剛至園門前,只見角門虛掩,猶未上閂。此時園內無人來往,只有該班的房內燈光掩映,微月半天。\begin{note}庚雙夾:是月起更處旬時也。\end{note}鴛鴦又不曾有個作伴的,也不曾提燈籠,獨自一個,腳步又輕,所以該班的人皆不理會。偏生又要小解,因下了甬路,尋微草處,行至一湖山石後大桂樹陰下來。\begin{note}庚雙夾:是八月,隨筆點景。\end{note}剛轉過石後,只聽一陣衣衫響,嚇了一驚不小。定睛一看,只見是兩個人在那裏,見他來了,便想往石後樹叢藏躲。鴛鴦眼尖,趁月色見準一個穿紅裙子梳鬅頭高大豐壯身材的,\begin{note}庚雙夾:是月下所見之像,故不寫至容貌也。\end{note}是迎春房裏的司棋。鴛鴦只當他和別的女孩子也在此方便,見自己來了,故意藏躲恐嚇著耍,\begin{note}庚雙夾:此見得是女兒們常事,觀書者自亦爲如此事。\end{note}因便笑叫道:“司棋你不快出來,嚇著我,我就喊起來當賊拿了。這麼大丫頭了,沒個黑家白日的只是頑不夠。”這本是鴛鴦的戲語,叫他出來。誰知他賊人膽虛,\begin{note}庚雙夾:更奇,不知所爲何事。\end{note}只當鴛鴦已看見他的首尾了,生恐叫喊起來使衆人知覺更不好,且素日鴛鴦又和自己親厚不比別人,便從樹後跑出來,一把拉住鴛鴦,便雙膝跪下,只說:“好姐姐,千萬別嚷!”鴛鴦反不知因何,忙拉他起來,笑問道:“這是怎麼說?”司棋滿臉紅脹,又流下淚來。鴛鴦再一回想,那一個人影恍惚象個小廝,心下便猜疑了八九,\begin{note}庚雙夾:是聰敏女兒,妙!\end{note}自己反羞的面紅耳赤,\begin{note}庚雙夾:是嬌貴女兒,筆筆皆到。\end{note}又怕起來。因定了一會,忙悄問:“那個是誰?”司棋復跪下道:“是我姑舅兄弟。”鴛鴦啐了一口,道:“要死,要死。”\begin{note}庚雙夾:如見其面,如聞其聲。\end{note}司棋又回頭悄道:“你不用藏著,姐姐已看見了,快出來磕頭。”那小廝聽了,只得也從樹後爬出來,磕頭如搗蒜。鴛鴦忙要回身,司棋拉住苦求,哭道:“我們的性命,都在姐姐身上,只求姐姐超生要緊!”鴛鴦道: “你放心,我橫豎不告訴一個人就是了。”一語未了,只聽角門上有人說道:“金姑娘已出去了,角門上鎖罷。”鴛鴦正被司棋拉住,不得脫身,聽見如此說,便接聲道:“我在這裏有事,且略住手,我出來了。”司棋聽了,只得鬆手讓他去了──
\end{parag}


\begin{parag}
    \begin{note}蒙回末總:敘一番燈火未息,門戶未關。敘一番趙姨失體,費婆憋氣。敘一番林家託大,周家獻勤。敘一番鳳姐灰心,鴛鴦傳信。非爲本文渲染,全爲下文引逗,良工苦心,可謂慘淡經營。\end{note}
\end{parag}


\begin{parag}
    \begin{note}蒙回末總:司棋事後從鴛鴦誤嚇得來,是善周全處。方與鴛鴦前後行景不致矛盾。何等精細如此。\end{note}
\end{parag}
