\chap{九十二}{评女传巧姐慕贤良 玩母珠贾政参聚散}



\begin{parag}
    话说宝玉从潇湘馆出来,连忙问秋纹道:“老爷叫我作什么?”秋纹笑道:“没有叫,袭人姐姐叫我请二爷,我怕你不来,才哄你的。”宝玉听了才把心放下,因说:“你们请我也罢了,何苦来唬我。”说着,回到怡红院内。袭人便问道:“你这好半天到那里去了?”宝玉道:“在林姑娘那边,说起薛姨妈宝姐姐的事来,便坐住了。”袭人又问道:“说些什么?”宝玉将打禅语的话述了一遍。袭人道:“你们再没个计较,正经说些家常闲话儿,或讲究些诗句,也是好的,怎么又说到禅语上了。又不是和尚。”宝玉道:“你不知道,我们有我们的禅机,别人是插不下嘴去的。”袭人笑道:“你们参禅参翻了,又叫我们跟着打闷葫芦了。”宝玉道:“头里我也年纪小,他也孩子气,所以我说了不留神的话,他就恼了。如今我也留神,他也没有恼的了。只是他近来不常过来,我又念书,偶然到一处,好象生疏了似的。”袭人道:“原该这么着才是。都长了几岁年纪了,怎么好意思还象小孩子时候的样子。”宝玉点头道:“我也知道。如今且不用说那个。我问你,老太太那里打发人来说什么来着没有?”袭人道:“没有说什么。”宝玉道:“必是老太太忘了。明儿不是十一月初一日么,年年老太太那里必是个老规矩,要办消寒会,齐打伙儿坐下喝酒说笑。我今日已经在学房里告了假了,这会子没有信儿,明儿可是去不去呢?若去了呢,白白的告了假,若不去,老爷知道了又说我偷懒。”袭人道:“据我说,你竟是去的是。才念的好些儿了,又想歇着。依我说也该上紧些才好。昨儿听见太太说,兰哥儿念书真好,他打学房里回来,还各自念书作文章,天天晚上弄到四更多天才睡。你比他大多了,又是叔叔,倘或赶不上他,又叫老太太生气。倒不如明儿早起去罢。”麝月道:“这样冷天,已经告了假又去,倒叫学房里说:既这么着就不该告假呀,显见的是告谎假脱滑儿。依我说落得歇一天。就是老太太忘记了,咱们这里就不消寒了么,咱们也闹个会儿不好么。”袭人道:“都是你起头儿,二爷更不肯去了。”麝月道:“我也是乐一天是一天,比不得你要好名儿,使唤一个月再多得二两银子!”袭人啐道:“小蹄子,人家说正经话,你又来胡拉混扯的了。”麝月道:“我倒不是混拉扯,我是为你。”袭人道:“为我什么?”麝月道:“二爷上学去了,你又该咕嘟着嘴想着,巴不得二爷早一刻儿回来,就有说有笑的了。这会儿又假撇清,何苦呢!我都看见了。”
\end{parag}


\begin{parag}
    袭人正要骂他,只见老太太那里打发人来说道:“老太太说了,叫二爷明儿不用上学去呢。明儿请了姨太太来给他解闷,只怕姑娘们都来,家里的史姑娘,邢姑娘,李姑娘们都请了,明儿来赴什么消寒会呢。”宝玉没有听完便喜欢道:“可不是,老太太最高兴的,明日不上学是过了明路的了。”袭人也便不言语了。那丫头回去。宝玉认真念了几天书,巴不得顽这一天。又听见薛姨妈过来,想着”宝姐姐自然也来”。心里喜欢,便说:“快睡罢,明日早些起来。”于是一夜无话。到了次日,果然一早到老太太那里请了安,又到贾政王夫人那里请了安,回明了老太太今儿不叫上学,贾政也没言语,便慢慢退出来,走了几步便一溜烟跑到贾母房中。见众人都没来,只有凤姐那边的奶妈子带了巧姐儿,跟着几个小丫头过来,给老太太请了安,说:“我妈妈先叫我来请安,陪着老太太说说话儿。妈妈回来就来。”贾母笑道:“好孩子,我一早就起来了,等他们总不来,只有你二叔叔来了。”那奶妈子便说:“姑娘给你二叔叔请安。”宝玉也问了一声“妞妞好?”巧姐儿道:“我昨夜听见我妈妈说,要请二叔叔去说话。”宝玉道:“说什么呢?”巧姐儿道:“我妈妈说,跟着李妈认了几年字,不知道我认得不认得。我说都认得,我认给妈妈瞧。妈妈说我瞎认,不信,说我一天尽子顽,那里认得。我瞧着那些字也不要紧,就是那《女孝经》也是容易念的。妈妈说我哄他,要请二叔叔得空儿的时候给我理理。”贾母听了,笑道:“好孩子,你妈妈是不认得字的,所以说你哄他。明儿叫你二叔叔理给他瞧瞧,他就信了。”宝玉道:“你认了多少字了?”巧姐儿道:“认了三千多字,念了一本《女孝经》,半个月头里又上了《列女传》。”宝玉道:“你念了懂得吗?你要不懂,我倒是讲讲这个你听罢。”贾母道:“做叔叔的也该讲究给侄女听听。”宝玉道:“那文王后妃是不必说了,想来是知道的。那姜后脱簪待罪,齐国的无盐虽丑,能安邦定国,是后妃里头的贤能的。若说有才的,是曹大姑、班婕妤、蔡文姬、谢道韫诸人。孟光的荆钗布裙,鲍宣妻的提瓮出汲,陶侃母的截发留宾,还有画荻教子的,这是不厌贫的。那苦的里头,有乐昌公主破镜重圆,苏蕙的回文感主。那孝的是更多了,木兰代父从军,曹娥投水寻父的尸首等类也多,我也说不得许多。那个曹氏的引刀割鼻,是魏国的故事。那守节的更多了,只好慢慢的讲。若是那些艳的,王嫱、西子、樊素、小蛮、绛仙等。妒的是秃妾发,怨洛神等类,也少。文君、红拂是女中的……”贾母听到这里,说:“够了,不用说了。你讲的太多,他那里还记得呢。”巧姐儿道:“二叔叔才说的,也有念过的,也有没念过的。念过的二叔叔一讲,我更知道了好些。”宝玉道:“那字是自然认得的了,不用再理。明儿我还上学去呢。”巧姐儿道:“我还听见我妈妈昨儿说,我们家的小红头里是二叔叔那里的,我妈妈要了来,还没有补上人呢。我妈妈想着要把什么柳家的五儿补上,不知二叔叔要不要。”宝玉听了更喜欢,笑着道:“你听你妈妈的话!要补谁就补谁罢咧,又问什么要不要呢。”因又向贾母笑道:“我瞧大妞妞这个小模样儿,又有这个聪明儿,只怕将来比凤姐姐还强呢,又比他认的字。”贾母道:“女孩儿家认得字呢也好,只是女工针黹倒是要紧的。”巧姐儿道:“我也跟着刘妈妈学着做呢,什么扎花儿咧,拉锁子,我虽弄不好,却也学着会做几针儿。”贾母道:“咱们这样人家固然不仗着自己做,但只到底知道些,日后才不受人家的拿捏。”巧姐儿答应着“是”,还要宝玉解说《列女传》,见宝玉呆呆的,也不敢再说。
\end{parag}


\begin{parag}
    你道宝玉呆的是什么?只因柳五儿要进怡红院,头一次是他病了不能进来,第二次王夫人撵了晴雯,大凡有些姿色的,都不敢挑。后来又在吴贵家看晴雯去,五儿跟着他妈给晴雯送东西去,见了一面,更觉娇娜妩媚。今日亏得凤姐想着,叫他补入小红的窝儿,竟是喜出望外了。所以呆呆的想他。
\end{parag}


\begin{parag}
    贾母等着那些人,见这时候还不来,又叫丫头去请。回来李纨同着他妹子,探春、惜春、史湘云、黛玉都来了,大家请了贾母的安。众人厮见。独有薛姨妈未到,贾母又叫请去。果然姨妈带着宝琴过来。宝玉请了安,问了好。只不见宝钗、邢岫烟二人。黛玉便问起“宝姐姐为何不来?”薛姨妈假说身上不好。邢岫烟知道薛姨妈在坐,所以不来。宝玉虽见宝钗不来,心中纳闷,因黛玉来了,便把想宝钗的心暂且搁开。不多时,邢王二夫人也来了。凤姐听见婆婆们先到了,自己不好落后,只得打发平儿先来告假,说是正要过来,因身上发热,过一回儿就来。贾母道:“既是身上不好,不来也罢。咱们这时候很该吃饭了。”丫头们把火盆往后挪了一挪儿,就在贾母榻前一溜摆下两桌,大家序次坐下。吃了饭,依旧围炉闲谈,不须多赘。
\end{parag}


\begin{parag}
    且说凤姐因何不来?头里为着倒比邢王二夫人迟了,不好意思,后来旺儿家的来回说:“迎姑娘那里打发人来请奶奶安,还说并没有到上头,只到奶奶这里来。”凤姐听了纳闷,不知又是什么事,便叫那人进来,问:“姑娘在家好?”那人道:“有什么好的,奴才并不是姑娘打发来的,实在是司棋的母亲央我来求奶奶的。”凤姐道:“司棋已经出去了,为什么来求我?”那人道:“自从司棋出去,终日啼哭。忽然那一日他表兄来了,他母亲见了,恨得什么似的,说他害了司棋,一把拉住要打。那小子不敢言语。谁知司棋听见了,急忙出来老着脸和他母亲道:‘我是为他出来的,我也恨他没良心。如今他来了,妈要打他,不如勒死了我。’他母亲骂他:‘不害臊的东西,你心里要怎么样?’司棋说道:‘一个女人配一个男人。我一时失脚上了他的当,我就是他的人了,决不肯再失身给别人的。我恨他为什么这样胆小,一身作事一身当,为什么要逃。就是他一辈子不来了,我也一辈子不嫁人的。妈要给我配人,我原拼着一死的。今儿他来了,妈问他怎么样。若是他不改心,我在妈跟前磕了头,只当是我死了,他到那里,我跟到那里,就是讨饭吃也是愿意的。’他妈气得了不得,便哭着骂着说:‘你是我的女儿,我偏不给他,你敢怎么着。’那知道那司棋这东西糊涂,便一头撞在墙上,把脑袋撞破,鲜血直流,竟死了。他妈哭着救不过来,便要叫那小子偿命。他表兄说道:‘你们不用着急。我在外头原发了财,因想着他才回来的,心也算是真了。你们若不信,只管瞧。’说着,打怀里掏出一匣子金珠首饰来。他妈妈看见了便心软了,说:‘你既有心,为什么总不言语?’他外甥道:‘大凡女人都是水性杨花,我若说有钱,他便是贪图银钱了。如今他只为人,就是难得的。我把金珠给你们,我去买棺盛殓他。’那司棋的母亲接了东西,也不顾女孩儿了,便由着外甥去。那里知道他外甥叫人抬了两口棺材来。司棋的母亲看见诧异,说:‘怎么棺材要两口?’他外甥笑道:‘一口装不下,得两口才好。’司棋的母亲见他外甥又不哭,只当是他心疼的傻了。岂知他忙着把司棋收拾了,也不啼哭,眼错不见,把带的小刀子往脖子里一抹,也就抹死了。司棋的母亲懊悔起来,倒哭得了不得。如今坊上知道了,要报官。他急了,央我来求奶奶说个人情,他再过来给奶奶磕头。”凤姐听了,诧异道:“那有这样傻丫头,偏偏的就碰见这个傻小子!怪不得那一天翻出那些东西来,她心里没事人似的,敢只是这么个烈性孩子。论起来,我也没这么大工夫管他这些闲事,但只你纔说的叫人听着怪可怜见儿的。也罢了,你回去告诉他,我和你二爷说,打发旺儿给他撕掳就是了。”凤姐打发那人去了,才过贾母这边来。不提。
\end{parag}


\begin{parag}
    且说贾政这日正与詹光下大棋,通局的输赢也差不多,单为着一只角儿死活未分,在那里打劫。门上的小厮进来回道:“外面冯大爷要见老爷。”贾政道:“请进来。”小厮出去请了,冯紫英走进门来。贾政即忙迎着。冯紫英进来,在书房中坐下,见是下棋,便道:“只管下棋,我来观局。”詹光笑道:“晚生的棋是不堪瞧的。”冯紫英道:“好说,请下罢。”贾政道:“有什么事么?”冯紫英道:“没有什么话。老伯只管下棋,我也学几着儿。”贾政向詹光道:“冯大爷是我们相好的,既没事,我们索性下完了这一局再说话儿。冯大爷在旁边瞧着。”冯紫英道:“下采不下采?”詹光道:“下采的。”冯紫英道:“下采的是不好多嘴的。”贾政道:“多嘴也不妨,横竖他输了十来两银子,终久是不拿出来的。往后只好罚他做东便了。”詹光笑道:“这倒使得。”冯紫英道:“老伯和詹公对下么?”贾政笑道:“从前对下,他输了,如今让他两个子儿,他又输了。时常还要悔几着,不叫他悔他就急了。”詹光也笑道:“没有的事。”贾政道:“你试试瞧。”大家一面说笑,一面下完了。做起棋来,詹光还了棋头,输了七个子儿。冯紫英道:“这盘终吃亏在打劫里头。老伯劫少,就便宜了。”
\end{parag}


\begin{parag}
    贾政对冯紫英道:“有罪,有罪。咱们说话儿罢。”冯紫英道:“小侄与老伯久不见面,一来会会,二来因广西的同知进来引见,带了四种洋货,可以做得贡的。一件是围屏,有二十四扇槅子,都是紫檀雕刻的。中间虽说不是玉,却是绝好的硝子石,石上镂出山水人物楼台花鸟等物。一扇上有五六十个人,都是宫妆的女子,名为《汉宫春晓》。人的眉目口鼻以及出手衣褶,刻得又清楚又细腻。点缀布置都是好的。我想尊府大观园中正厅上却可用得着。还有一个钟表,有三尺多高,也是一个小童儿拿着时辰牌,到了什么时候他就报什么时辰。里头也有些人在那里打十番的。这是两件重笨的,却还没有拿来。现在我带在这里两件却有些意思儿。”就在身边拿出一个锦匣子,见几重白锦裹着,揭开了锦子,第一层是一个玻璃盒子,里头金托子大红绉绸托底,上放着一颗桂圆大的珠子,光华耀目。冯紫英道:“据说这就叫做母珠。”因叫拿一个盘儿来。詹光即忙端过一个黑漆茶盘,道:“使得么?”冯紫英道:“使得。”便又向怀里掏出一个白绢包儿,将包儿里的珠子都倒在盘子里散着,把那颗母珠搁在中间,将盘置于桌上。看见那些小珠子儿滴溜滴溜滚到大珠身边来,一回儿把这颗大珠子抬高了,别处的小珠子一颗也不剩,都粘在大珠上。詹光道:“这也奇怪。”贾政道:“这是有的,所以叫做母珠,原是珠之母。”那冯紫英又回头看着他跟来的小厮道:“那个匣子呢?”那小厮赶忙捧过一个花梨木匣子来。大家打开看时,原来匣内衬着虎纹锦,锦上迭着一束蓝纱。詹光道:“这是什么东西?”冯紫英道:“这叫做鲛绡帐。”在匣子里拿出来时,迭得长不满五寸,厚不上半寸,冯紫英一层一层的打开,打到十来层,已经桌上铺不下了。冯紫英道:“你看里头还有两折,必得高屋里去才张得下。这就是鲛丝所织,暑热天气张在堂屋里头,苍蝇蚊子一个不能进来,又轻又亮。”贾政道:“不用全打开,怕迭起来倒费事。”詹光便与冯紫英一层一层折好收拾。冯紫英道:“这四件东西价儿也不很贵,两万银他就卖。母珠一万,鲛绡帐五千,《汉宫春晓》与自鸣钟五千。”贾政道:“那里买得起。”冯紫英道:“你们是个国戚,难道宫里头用不着么?”贾政道:“用得着的很多,只是那里有这些银子。等我叫人拿进去给老太太瞧瞧。”冯紫英道:“很是。”
\end{parag}


\begin{parag}
    贾政便着人叫贾琏把这两件东西送到老太太那边去,并叫人请了邢王二夫人凤姐儿都来瞧着,又把两件东西一一试过。贾琏道:“他还有两件:一件是围屏。一件是乐钟。共总要卖二万银子呢。”凤姐儿接着道:“东西自然是好的,但是那里有这些闲钱。咱们又不比外任督抚要办贡。我已经想了好些年了,象咱们这种人家,必得置些不动摇的根基才好,或是祭地,或是义庄,再置些坟屋。往后子孙遇见不得意的事,还是点儿底子,不到一败涂地。我的意思是这样,不知老太太、老爷、太太们怎么样。若是外头老爷们要买,只管买。”贾母与众人都说:“这话说的倒也是。”贾琏道:“还了他罢。原是老爷叫我送给老太太瞧,为的是宫里好进。谁说买来搁在家里?老太太还没开口,你便说了一大些丧气话!”
\end{parag}


\begin{parag}
    说着,便把两件东西拿了出去,告诉了贾政,说老太太不要。便与冯紫英道:“这两件东西好可好,就只没银子。我替你留心,有要买的人,我便送信给你去。”冯紫英只得收拾好,坐下说些闲话,没有兴头,就要起身。贾政道:“你在我这里吃了晚饭去罢。”冯紫英道:“罢了,来了就叨扰老伯吗!”贾政道:“说那里的话。”正说着,人回:“大老爷来了。”贾赦早已进来。彼此相见,叙些寒温。不一时摆上酒来,肴馔罗列,大家喝着酒。至四五巡后,说起洋货的话,冯紫英道:“这种货本是难消的,除非要象尊府这种人家,还可消得,其余就难了。”贾政道:“这也不见得。”贾赦道:“我们家里也比不得从前了,这回儿也不过是个空门面。”冯紫英又问:“东府珍大爷可好么?我前儿见他,说起家常话儿来,提到他令郎续娶的媳妇,远不及头里那位秦氏奶奶了。如今后娶的到底是那一家的,我也没有问起。”贾政道:“我们这个侄孙媳妇儿,也是这里大家,从前做过京畿道的胡老爷的女孩儿。”紫英道:“胡道长我是知道的。但是他家教上也不怎么样。也罢了,只要姑娘好就好。”
\end{parag}


\begin{parag}
    贾琏道:“听得内阁里人说起,贾雨村又要升了。”贾政道:“这也好,不知准不准。”贾琏道:“大约有意思的了。”冯紫英道:“我今儿从吏部里来,也听见这样说。雨村老先生是贵本家不是?”贾政道:“是。”冯紫英道:“是有服的还是无服的?”贾政道:“说也话长。他原籍是浙江湖州府人,流寓到苏州,甚不得意。有个甄士隐和他相好,时常周济他。以后中了进士,得了榜下知县,便娶了甄家的丫头。如今的太太不是正配。岂知甄士隐弄到零落不堪,没有找处。雨村革了职以后,那时还与我家并未相识,只因舍妹丈林如海林公在扬州巡盐的时候,请他在家做西席,外甥女儿是他的学生。因他有起复的信要进京来,恰好外甥女儿要上来探亲,林姑老爷便托他照应上来的,还有一封荐书,托我吹嘘吹嘘。那时看他不错,大家常会。岂知雨村也奇,我家世袭起,从代字辈下来,宁荣两宅人口房舍以及起居事宜,一概都明白,因此遂觉得亲热了。”因又笑说道:“几年门子也会钻了。由知府推升转了御史,不过几年,升了吏部侍郎,署兵部尚书。为着一件事降了三级,如今又要升了。”冯紫英道:“人世的荣枯,仕途的得失,终属难定。”贾政道:“象雨村算便宜的了。还有我们差不多的人家就是甄家,从前一样功勋,一样的世袭,一样的起居,我们也是时常往来。不多几年,他们进京来差人到我这里请安,还很热闹。一回儿抄了原籍的家财,至今杳无音信,不知他近况若何,心下也着实惦记。看了这样,你想做官的怕不怕?”贾赦道:“咱们家是最没有事的。”冯紫英道:“果然,尊府是不怕的。一则里头有贵妃照应,二则故旧好亲戚多,三则你家自老太太起至于少爷们,没有一个刁钻刻薄的。”贾政道:“虽无刁钻刻薄,却没有德行才情。白白的衣租食税,那里当得起。”贾赦道:“咱们不用说这些话,大家吃酒罢。”大家又喝了几杯,摆上饭来。吃毕,喝茶。冯家的小厮走来轻轻的向紫英说了一句,冯紫英便要告辞了。贾赦贾政道:“你说什么?”小厮道:“外面下雪,早已下了梆子了。”贾政叫人看时,已是雪深一寸多了。贾政道:“那两件东西你收拾好了么?”冯紫英道:“收好了。若尊府要用,价钱还自然让些。”贾政道:“我留神就是了。”紫英道:“我再听信罢。天气冷,请罢,别送了。”贾赦贾政便命贾琏送了出去。未知后事如何,下回分解。
\end{parag}